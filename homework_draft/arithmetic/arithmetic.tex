\documentclass[12pt, a4paper]{article}
\usepackage{cmap} % Улучшенный поиск русских слов в полученном pdf-файле
\usepackage[T2A]{fontenc} % Поддержка русских букв
\usepackage[utf8]{inputenc} % Кодировка utf8
\usepackage[english, russian]{babel} % Языки: русский, английский
\usepackage{enumitem}
\usepackage{pscyr} % Нормальные шрифты
\usepackage{soulutf8}
\usepackage{amsmath}
\usepackage{amsthm}
\usepackage{amssymb}
\usepackage{scrextend}
\usepackage{titling}
\usepackage{indentfirst}
\usepackage{cancel}
\usepackage{soulutf8}
\usepackage{wrapfig}
\usepackage{gensymb}
\usepackage[dvipsnames,table,xcdraw]{xcolor}
\usepackage{tikz}

%Русские символы в списке
\makeatletter
\AddEnumerateCounter{\asbuk}{\russian@alph}{щ}
\makeatother

%Дублирование знаков при переносе
\newcommand*{\hm}[1]{#1\nobreak\discretionary{}%
	{\hbox{$\mathsurround=0pt #1$}}{}}

\usepackage{graphicx}
\graphicspath{{pic/}}
\DeclareGraphicsExtensions{.pdf,.png,.jpg}

%Изменеие параметров листа
\usepackage[left=15mm,right=15mm,
top=2cm,bottom=2cm,bindingoffset=0cm]{geometry}


\usepackage{fancyhdr}
\pagestyle{fancy}
\usepackage{multicol}

\setlength\parindent{1,5em}
\usepackage{indentfirst}

\begin{document}
		
\lhead{Упрощение выражений}
\rhead{Школа <<Симметрия>>}

\section{Дроби}
\subsection{Произведение дробей}
	\begin{multicols}{3}
		\begin{enumerate}
			\item $\dfrac{7b^4}{5c^5y}\cdot\dfrac{18c^4y^3}{35b^4c}$ Ответ: \fbox{$\dfrac{2y^2}{5c^2}$}
			\item $\left(\dfrac{xy}{ab}\right)^2\cdot\dfrac{xab}{y^2}$ Ответ: \fbox{$\dfrac{x^3}{ab}$}
		\end{enumerate}
	\end{multicols}	
\section{Арифметические корни}
\begin{enumerate}
	\item \textit{(Никольский 8кл. с.52 Пример 4, 155, 167)} Вычислить:
	\begin{enumerate}[label=\asbuk*)]
		\item $(2\sqrt{8}+3\sqrt{5}-7\sqrt{2})(\sqrt{72}+\sqrt{20}-4\sqrt{2})$
		\item  $\sqrt{245 \cdot 27 \cdot 60}$
		\item $\sqrt{6 \cdot 30 \cdot 245}$
		\item $(2\sqrt{6}+5\sqrt{3}-7\sqrt{2})(\sqrt{6}-2\sqrt{3}+4\sqrt{2})$

	\end{enumerate}
\end{enumerate}
\section{Тригонометрия}
\begin{enumerate}
	\item \textit{(Никольский 10кл. 7.46, 7.47, 7.61)} Вычислить:
	\begin{enumerate}[label=\asbuk*)]
		\item $3\cos0+2\sin\dfrac{\pi}{2}-4\cos\dfrac{\pi}{2}-7\sin(-\pi)$
		\item $\cos\dfrac{\pi}{2}-3\sin\left(-\dfrac{3\pi}{4}\right)+4\cos(-2\pi)-2\sin(-3\pi)$
		\item $\sin\dfrac{\pi}{4}+\cos\left(-\dfrac{3\pi}{4}\right)+4\cos(-2\pi)-2\sin(-3\pi)$
		\item $3\cos\dfrac{7\pi}{4}+2\sin\dfrac{3\pi}{4}-\sin\left(-\dfrac{9\pi}{4}\right)+7\cos\dfrac{13\pi}{2}$
		\item $3\sin\left(-\dfrac{3\pi}{2}\right)-4\cos\left(-\dfrac{11\pi}{2}\right)+5\sin7\pi+\cos(-11\pi)$
		\item $3\cos\dfrac{\pi}{3}-2\sin\dfrac{2\pi}{3}+7\cos\left(-\dfrac{2\pi}{3}\right)-\sin\left(-\dfrac{5\pi}{4}\right)$
		\item $2\sin\left(-\dfrac{5\pi}{6}\right)+11\cos\left(-\dfrac{7\pi}{3}\right)+\sin\dfrac{7\pi}{6}-8\cos\dfrac{2\pi}{3}$
		\item $-6\cos\left(-\dfrac{\pi}{6}\right)-2\sin\left(-\dfrac{\pi}{2}\right)-5\sin\left(-\dfrac{5\pi}{6}\right)+\cos\dfrac{7\pi}{6}$
	\end{enumerate}
\end{enumerate}
\end{document}