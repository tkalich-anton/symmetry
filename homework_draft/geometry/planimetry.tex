\documentclass[10pt, a4paper]{article}
\usepackage{cmap}
\usepackage[T2A]{fontenc}
\usepackage[utf8]{inputenc}
\usepackage[english, russian]{babel}
\usepackage[dvipsnames,table,xcdraw]{xcolor}
\usepackage{
	amsmath,
	amssymb,
	scrextend,
	enumitem,
	pscyr,
	multicol,
	cmap,
	titling,
	indentfirst,
	cancel,
	wrapfig,
	gensymb,
	tikz,
	graphicx,
	fancyhdr,
	mathrsfs,
	graphbox,
	indentfirst
}
%Параметры страницы
\usepackage[left=15mm,right=15mm,
top=2cm,bottom=2cm]{geometry}
\pagestyle{fancy}
%Путь к картинкам
\graphicspath{{pic/}}
\DeclareGraphicsExtensions{.pdf,.png,.jpg}
%Числа в списке второго уровня по умолчанию
\renewcommand{\labelenumii}{\arabic{enumii})}
%Новые команды
\definecolor{silver}{rgb}{0.7, 0.7, 0.7}
\definecolor{dark}{rgb}{0.3, 0.3, 0.3}
\definecolor{harvestgold}{rgb}{0.85, 0.57, 0.0}
\newcommand{\answer}[1]{\textcolor{silver}{\fbox{#1}}}
\newcommand{\source}[1]{\textcolor{silver}{\textit{(#1)}}}
\newcommand{\ranswer}[1]{\textcolor{silver}{\begin{flushright}\vspace{-1em}\fbox{#1}\end{flushright}}}
\newcommand{\leveli}{\textcolor{dark}{$\blacksquare\square\square$}\hspace{0.5em}}
\newcommand{\levelii}{\textcolor{dark}{$\blacksquare\blacksquare\square$}\hspace{0.5em}}
\newcommand{\leveliii}{\textcolor{dark}{$\blacksquare\blacksquare\blacksquare$}\hspace{0.5em}}

%Русские символы в списке
\AddEnumerateCounter{\asbuk}{\russian@alph}{щ}

%Сеттеры
\setlength{\parindent}{5ex}
\setlength{\parskip}{1em}

\begin{document}
		
\lhead{Упрощение выражений}
\rhead{Школа <<Симметрия>>}

\section{Треугольники}
\subsection{Признаки равенства треугольников}
	\begin{enumerate}
		\item \source{Гордин Р.К. Планиметрия, №1.40} Медиана $AM$ треугольника $ABC$ перпендикулярна его биссектрисе $BK$. Найдите $AB$, если $BC = 12$.
		\item \source{Гордин Р.К. Планиметрия, №1.41} Прямая,  проведенная  через  вершину  $A$  треугольника $ABC$ перпендикулярно его медиане $BD$, 
		делит эту медиану пополам. Найдите отношение сторон $AB$ и $AC$.
		\item \source{Гордин Р.К. Планиметрия, №1.42} Стороны  равностороннего  треугольника  делятся  точками $K$, $L$, $M$ в одном и том же отношении (считая по часовой стрелке).  Докажите,  что  треугольник $KLM$  также  равносторонний.
		\item \source{Гордин Р.К. Планиметрия, №1.34} Докажите, что в равных треугольниках соответствующие медианы равны.
		\item \source{Гордин Р.К. Планиметрия, №1.35} Докажите, что в равных треугольниках соответствующие биссектрисы равны.
		\item \source{Гордин Р.К. Планиметрия, №1.37} Докажите, что биссектриса равнобедренного треугольника, проведенная из вершины, является также медианой
		и высотой.
		\item \source{Гордин Р.К. Планиметрия, №1.38} Медиана треугольника является также его высотой.
		Докажите, что такой треугольник равнобедренный.
		\item \source{Гордин Р.К. Планиметрия, №1.46} В треугольнике $ABC$ медиана $AM$ продолжена за точку $M$ на расстояние, равное $AM$. Найдите расстояние от полученной точки до вершин $B$ и $C$, если $AB = 7$, $AC = 11$.
		\item \source{Гордин Р.К. Планиметрия, №1.47} Биссектриса треугольника является его медианой. Докажите, что треугольник равнобедренный.
		\item \source{Гордин Р.К. Планиметрия, №1.50} Докажите признаки равенства прямоугольных тре-
		угольников:
		\begin{enumerate}[label=\asbuk*)]
			\item по двум катетам;
			\item по катету и гипотенузе;
			\item по катету и прилежащему острому углу;
			\item по гипотенузе и острому углу.
		\end{enumerate}
		\item \source{Гордин Р.К. Планиметрия, №1.51} Докажите, что серединный перпендикуляр к отрезку есть геометрическое место точек, равноудаленных от концов этого отрезка.
		\item \source{Гордин Р.К. Планиметрия, №1.52} Две различные окружности пересекаются в точках $A$ и $B$. Докажите, что прямая, проходящая через центры окружностей, делит отрезок $AB$ пополам и перпендикулярна ему.
		\item \source{Ткалич А.А.} Две различные окружности с центрами в точках $O_1$ и $O_2$ пересекаются в точках $A$ и $B$. Прямая, проходящая через центры окружностей, пересекает отрезок $AB$ в точке $K$. Докажите, что треугольники $O_1KA$ и $O_1KB$ равны.
		\item \source{Гордин Р.К. Планиметрия, №1.54} Докажите признак равенства прямоугольных треугольников по катету и противолежащему углу.
		\item \source{Гордин Р.К. Планиметрия, №1.57} Докажите, что в равных треугольниках соответствующие высоты равны между собой.
		\item \source{Гордин Р.К. Планиметрия, №1.58} Докажите, что серединный перпендикуляр к отрезку является его осью симметрии.
		\item \source{Гордин Р.К. Планиметрия, №1.52} Докажите, что диагонали четырехугольника с равными сторонами взаимно перпендикулярны.
		\item \source{Гордин Р.К. Планиметрия, №1.60} Точки $M$ и $N$ --- середины равных сторон $AD$ и $BC$ четырехугольника $ABCD$. Серединные перпендикуляры к сторонам $AB$ и $CD$ пересекаются в точке $P$. Докажите, что серединный перпендикуляр к отрезку $MN$ проходит через точку $P$.
		\item \source{Гордин Р.К. Планиметрия, №1.61} Две высоты треугольника равны между собой. Докажите, что треугольник равнобедренный.
		\item \source{Гордин Р.К. Планиметрия, №1.62} Высоты треугольника $ABC$, проведенные из вершин $B$ и $C$, пересекаются в точке $M$. Известно, что $BM = CM$. Докажите, что треугольник ABC равнобедренный.
		\item \source{Гордин Р.К. Планиметрия, №1.63} Найдите геометрическое место внутренних точек угла, равноудаленных от его сторон.
		\item \source{Гордин Р.К. Планиметрия, №1.64} Докажите, что биссектриса угла является его осью симметрии.
		\item \source{Гордин Р.К. Планиметрия, №1.65} Через вершины $A$ и $C$ треугольника $ABC$ проведены прямые, перпендикулярные биссектрисе угла $ABC$, пересекающие прямые $CB$ и $BA$ в точках $K$ и $M$ соответственно. Найдите $AB$, если $BM = 8$, $KC = 1$.
		\item \source{Гордин Р.К. Планиметрия, №1.66} Через данную точку проведите прямую, пересекающую две данные прямые под равными углами.
	\end{enumerate}
\subsection{Параллельность}
\subsection{Окружность}
\end{document}