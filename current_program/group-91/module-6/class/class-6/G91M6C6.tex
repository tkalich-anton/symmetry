\documentclass[12pt, a4paper]{article}
\usepackage{cmap} % Улучшенный поиск русских слов в полученном pdf-файле
\usepackage[T2A]{fontenc} % Поддержка русских букв
\usepackage[utf8]{inputenc} % Кодировка utf8
\usepackage[english, russian]{babel} % Языки: русский, английский
\usepackage{enumitem}
\usepackage{pscyr} % Нормальные шрифты
\usepackage{amsmath}
\usepackage{amsthm}
\usepackage{amssymb}
\usepackage{scrextend}
\usepackage{titling}
\usepackage{indentfirst}
\usepackage{cancel}
\usepackage{soulutf8}
\usepackage{wrapfig}
\usepackage{gensymb}
\usepackage[dvipsnames,table,xcdraw]{xcolor}
\usepackage{tikz}

%Русские символы в списке
\makeatletter
\AddEnumerateCounter{\asbuk}{\russian@alph}{щ}
\makeatother

%Дублирование знаков при переносе
\newcommand*{\hm}[1]{#1\nobreak\discretionary{}%
	{\hbox{$\mathsurround=0pt #1$}}{}}

\usepackage{graphicx}
\graphicspath{{pic/}}
\DeclareGraphicsExtensions{.pdf,.png,.jpg}

%Изменеие параметров листа
\usepackage[left=15mm,right=15mm,
top=2cm,bottom=2cm,bindingoffset=0cm]{geometry}

\usepackage{fancyhdr}
\pagestyle{fancy}
\usepackage{multicol}
\setlength\parindent{1,5em}
\usepackage{indentfirst}
\begin{document}
	
	\lhead{Группа 91}
	\chead{Модуль 6 Урок №6}
	\rhead{Школа <<Симметрия>>}
	\begin{enumerate}
		\item В среднем на $150$ карманных фонариков приходится три неисправных. Найдите вероятность купить работающий фонарик.
		\item На экзамене $40$ билетов, Яша не выучил $4$ из них. Найдите вероятность того, что ему попадется выученный билет.
		\item Максим выбирает трехзначное число. Найдите вероятность того, что оно делится на $98$.
		\item В каждой пятой банке кофе согласно условиям акции есть приз. Призы распределены по банкам случайно. Галя покупает банку кофе в надежде выиграть приз. Найдите вероятность того, что Галя не найдёт приз в своей банке.
		\item У бабушки $20$ чашек: $9$ с красными цветами, остальные с синими. Бабушка наливает чай в случайно выбранную чашку. Найдите вероятность того, что это будет чашка с синими цветами.
		\item Для экзамена подготовили билеты с номерами от $1$ до $25$. Какова вероятность того, что наугад взятый учеником билет имеет номер, являющийся двузначным числом?
		\item В чемпионате по футболу участвуют $16$ команд, которые жеребьевкой распределяются на $4$ группы: $A, B, C$ и $D$. Какова вероятность того, что команда России не попадает в группу $A$?
		\item Лене надо подписать $972$ открытки. Ежедневно она подписывает на одно и то же количество открыток больше по сравнению с предыдущим днём. Известно, что за первый день Лена подписала 20 открыток. Определите, сколько открыток было подписано за седьмой день, если вся работа была выполнена за $18$ дней
		\item Хозяин договорился с рабочим, что они выкопают ему колодец на следующих условиях: за первый метро он им заплатит $3700$ рублей, а за каждый следующий метр – на $1700$ рублей больше, чем за предыдущий. Сколько рублей хозяин должен будет заплатить рабочим, если они выкопают колодец глубиной $8$ метров?
		\item Тренер посоветовал Андрею в первый день занятий провести на беговой дорожке $22$ минуты, а на каждом следующем занятии увеличивать время, проведённое на беговой дорожке, на $4$ минуты, пока оно не достигнет $60$ минут, а дальше продолжать тренироваться по $60$ минут каждый день. За сколько занятий, начиная с первого, Андрей проведёт на беговой дорожке в сумме $4$ часа $48$ минут?
	\end{enumerate}
\end{document}