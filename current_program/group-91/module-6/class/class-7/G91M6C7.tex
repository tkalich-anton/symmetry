\documentclass[12pt, a4paper]{article}
\usepackage{cmap} % Улучшенный поиск русских слов в полученном pdf-файле
\usepackage[T2A]{fontenc} % Поддержка русских букв
\usepackage[utf8]{inputenc} % Кодировка utf8
\usepackage[english, russian]{babel} % Языки: русский, английский
\usepackage{enumitem}
\usepackage{pscyr} % Нормальные шрифты
\usepackage{amsmath}
\usepackage{amsthm}
\usepackage{amssymb}
\usepackage{scrextend}
\usepackage{titling}
\usepackage{indentfirst}
\usepackage{cancel}
\usepackage{soulutf8}
\usepackage{wrapfig}
\usepackage{gensymb}
\usepackage[dvipsnames,table,xcdraw]{xcolor}
\usepackage{tikz}

%Русские символы в списке
\makeatletter
\AddEnumerateCounter{\asbuk}{\russian@alph}{щ}
\makeatother

%Дублирование знаков при переносе
\newcommand*{\hm}[1]{#1\nobreak\discretionary{}%
	{\hbox{$\mathsurround=0pt #1$}}{}}

\usepackage{graphicx}
\graphicspath{{pic/}}
\DeclareGraphicsExtensions{.pdf,.png,.jpg}

%Изменеие параметров листа
\usepackage[left=15mm,right=15mm,
top=2cm,bottom=2cm,bindingoffset=0cm]{geometry}

\usepackage{fancyhdr}
\pagestyle{fancy}
\usepackage{multicol}
\setlength\parindent{1,5em}
\usepackage{indentfirst}
\begin{document}
	
	\lhead{Группа 91}
	\chead{Модуль 6 Урок №7}
	\rhead{Школа <<Симметрия>>}
	\begin{enumerate}
		\item Бизнесмен Бубликов получил в $2000$ году прибыль в размере $5000$ рублей. Каждый следующий год его прибыль увеличивалась на $300\%$ по сравнению с предыдущим годом. Сколько рублей заработал Бубликов за $2003$ год?
		\item Клиент взял в банке кредит в размере $50 000$ р. на $5$ лет под $20$$\%$ годовых. Какую сумму он должен вернуть в банк в конце срока, если проценты начисляются ежегодно на текущую сумму долга и весь кредит с процентами возвращается в банк после срока?
		 \item Компания «Альфа» начала инвестировать средства в перспективную отрасль в $2001$ году, имея капитал в размере $5000$ долларов. Каждый год, начиная с $2002$ года, она получала прибыль, которая составляла $200\%$ от капитала предыдущего года. А компания «Бета» начала инвестировать средства в другую отрасль в $2003$ году, имея капитал в размере $10000$ долларов, и, начиная с $2004$ года, ежегодно получала прибыль, составляющую $400\%$ от капитала предыдущего года. На сколько долларов капитал одной из компаний был больше капитала другой к концу $2006$ года, если прибыль из оборота не изымалась?
		 \item Бактерия, попав в живой организм, к концу $20-$й минуты делится на две бактерии, каждая из них к концу следующих $20$ минут делится опять на две и т.д. Сколько бактерий окажется в организме через $4$ часа, если по истеяении четвертого часа в организм из окружающей среды попала еще одна бактерия?
		\item Однажды богач заключил выгодную, как ему казалось, сделку с человеком, который в течение $15$ дней ежедневно должен приносить по $1000$ р., а взамен в первый день богач должен был отдать $10$ р., во второй – $20$ р., в третий – $40$ р., в четвертый – $80$ р. и т.д. в течение $15$ дней. Сколько денег получил богач и сколько он отдал? кто выиграл от этой сделки? В ответ запишите, сколько рублей потерял богач за $15$ дней.
	\end{enumerate}
\end{document}