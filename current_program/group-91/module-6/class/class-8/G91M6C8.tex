\documentclass[12pt, a4paper]{article}
\usepackage{cmap} % Улучшенный поиск русских слов в полученном pdf-файле
\usepackage[T2A]{fontenc} % Поддержка русских букв
\usepackage[utf8]{inputenc} % Кодировка utf8
\usepackage[english, russian]{babel} % Языки: русский, английский
\usepackage{enumitem}
\usepackage{pscyr} % Нормальные шрифты
\usepackage{amsmath}
\usepackage{amsthm}
\usepackage{amssymb}
\usepackage{scrextend}
\usepackage{titling}
\usepackage{indentfirst}
\usepackage{cancel}
\usepackage{soulutf8}
\usepackage{wrapfig}
\usepackage{gensymb}
\usepackage[dvipsnames,table,xcdraw]{xcolor}
\usepackage{tikz}

%Русские символы в списке
\makeatletter
\AddEnumerateCounter{\asbuk}{\russian@alph}{щ}
\makeatother

%Дублирование знаков при переносе
\newcommand*{\hm}[1]{#1\nobreak\discretionary{}%
	{\hbox{$\mathsurround=0pt #1$}}{}}

\usepackage{graphicx}
\graphicspath{{pic/}}
\DeclareGraphicsExtensions{.pdf,.png,.jpg}

%Изменеие параметров листа
\usepackage[left=15mm,right=15mm,
top=2cm,bottom=2cm,bindingoffset=0cm]{geometry}

\usepackage{fancyhdr}
\pagestyle{fancy}
\usepackage{multicol}
\setlength\parindent{1,5em}
\usepackage{indentfirst}
\begin{document}
	
	\lhead{Группа 91}
	\chead{Модуль 6 Урок №8}
	\rhead{Школа <<Симметрия>>}
	\begin{enumerate}
		\item Вычислите:\\\\
		$\left( 3,5 \cdot 24 - 5\dfrac{2}{3}:\dfrac{1}{18}\right)\cdot 5 $
		\item Упростите выражения:
		\begin{multicols}{2}
			\begin{enumerate}[label=\asbuk*)]
				\item $2(x-1)+3(2-x)$
				\item $x(x^2-y^2)+y(xy-y)^2$
				\item $\left( \dfrac{1}{2}a-2b\right)\left( \dfrac{1}{4}a^2+ab+4b^2\right)-\left( \dfrac{1}{8}a^3-8b^3\right)$
				\item $(x+1)(x+2)-(x+3)(x+4)$
			\end{enumerate}
		\end{multicols}
	\item На биржевых торгах в понедельник вечером цена акции банка «Городской» повысилась на некоторое количество процентов, а во вторник произошло снижение стоимости акции на то же число процентов. В результате во вторник вечером цена акции составила $99\%$ от ее первоначальной цены в понедельник утром. На сколько процентов менялась котировка акции в понедельник и во вторник?
	\item Алик, Миша и Вася покупали блокноты и трехкопеечные карандаши. Алик купил $2$ блокнота и $4$ карандаша, Миша – блокнот и $6$ карандашей, Вася – блокнот и $3$ карандаша. Оказалось, что суммы, которые уплатили Алик, Миша и Вася, образуют геометрическую прогрессию. Сколько стоит блокнот?
	\item Три конькобежца, скорости которых в некотором порядке образуют геометрическую прогрессию, одновременно стартуют (из одного места) по кругу. Через некоторое время второй конькобежец обгоняет первого, пробежав на $40$0 метров больше его. Третий конькобежец пробегает то расстояние, который пробежал первый к моменту обгона его вторым, за время на $\dfrac{2}{3}$ мин больше, чем первый. Найдите скорость первого конькобежца в м/мин.
	\end{enumerate}
\end{document}