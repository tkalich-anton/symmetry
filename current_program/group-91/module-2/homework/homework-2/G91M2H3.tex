\documentclass[12pt, a4paper]{article}
\usepackage{cmap} % Улучшенный поиск русских слов в полученном pdf-файле
\usepackage[T2A]{fontenc} % Поддержка русских букв
\usepackage[utf8]{inputenc} % Кодировка utf8
\usepackage[english, russian]{babel} % Языки: русский, английский
\usepackage{enumitem}
\usepackage{pscyr} % Нормальные шрифты
\usepackage{amsmath}
\usepackage{amsthm}
\usepackage{amssymb}
\usepackage{scrextend}
\usepackage{titling}
\usepackage{indentfirst}
\usepackage{cancel}
\usepackage{soulutf8}
\usepackage{wrapfig}
\usepackage{gensymb}
\usepackage[dvipsnames,table,xcdraw]{xcolor}
\usepackage{tikz}

%Русские символы в списке
\makeatletter
\AddEnumerateCounter{\asbuk}{\russian@alph}{щ}
\makeatother

%Дублирование знаков при переносе
\newcommand*{\hm}[1]{#1\nobreak\discretionary{}%
	{\hbox{$\mathsurround=0pt #1$}}{}}

\usepackage{graphicx}
\graphicspath{{pic/}}
\DeclareGraphicsExtensions{.pdf,.png,.jpg}

%Изменеие параметров листа
\usepackage[left=15mm,right=15mm,
top=2cm,bottom=2cm,bindingoffset=0cm]{geometry}

\usepackage{fancyhdr}
\pagestyle{fancy}
\usepackage{multicol}

\setlength\parindent{1,5em}
\usepackage{indentfirst}

\begin{document}
		
\lhead{Группа 91}
\chead{Модуль 2 Урок 7}
\rhead{Школа <<Симметрия>>}

\begin{enumerate}
	\item Биссектрисы двух углов треугольника пересекаются под углом 110\degree. Найдите третий угол треугольника.
	\item Один из углов треугольника равен 40\degree. Найдите угол между высотами, проведенными из вершин двух других углов.
	\item Острый угол прямоугольного треугольника равен 30\degree. Докажите, что высота и медиана, проведенные из вершины прямого угла, делят его на три равные части.
	\item На продолжениях гипотенузы $AB$ прямоугольного треугольника $ABC$ за точки $A$ и $B$ соответственно взяты точки $K$ и $M$, причем $AK = AC$ и $BM = BC$. Найдите угол $MCK$.
	\item Через точку $A$ окружности с центром $O$ проведены диаметр $AB$ и хорда $AC$. Докажите, что угол $BAC$ вдвое меньше угла $BOC$.
	\item Через точку $A$, лежащую на окружности, проведены диаметр $AB$ и хорда $AC$, причем $AC = 8$ и $\angle BAC = 30\degree$. Найдите хорду CM, перпендикулярную AB.
	\item Через концы диаметра окружности проведены две хорды, пересекающиеся на окружности и равные 12 и 16. Найдите расстояния от центра окружности до этих хорд.
	\item На катете $AC$ прямоугольного треугольника $ABC$ как на диаметре построена окружность, пересекающая гипотенузу $AB$ в точке $K$. Найдите $CK$, если $AC = 2$ и $\angle A = 30\degree$
	\item Докажите, что окружность, построенная на стороне равностороннего треугольника как на диаметре, проходит через середины двух других сторон треугольника.
	\item Докажите, что окружность, построенная на боковой стороне равнобедренного треугольника как на диаметре, проходит через середину основания.
	\item Окружность, построенная на стороне треугольника как на диаметре, проходит через середину другой стороны. Докажите, что треугольник равнобедренный.
\end{enumerate}
\end{document}