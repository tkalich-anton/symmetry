\documentclass[12pt, a4paper]{article}
\usepackage{cmap} % Улучшенный поиск русских слов в полученном pdf-файле
\usepackage[T2A]{fontenc} % Поддержка русских букв
\usepackage[utf8]{inputenc} % Кодировка utf8
\usepackage[english, russian]{babel} % Языки: русский, английский
\usepackage{enumitem}
\usepackage{pscyr} % Нормальные шрифты
\usepackage{soulutf8}
\usepackage{amsmath}
\usepackage{amsthm}
\usepackage{amssymb}
\usepackage{scrextend}
\usepackage{titling}
\usepackage{indentfirst}
\usepackage{cancel}
\usepackage{soulutf8}
\usepackage{wrapfig}
\usepackage{gensymb}
\usepackage[dvipsnames,table,xcdraw]{xcolor}
\usepackage{tikz}
\usepackage{multicol}

%Русские символы в списке
\makeatletter
\AddEnumerateCounter{\asbuk}{\russian@alph}{щ}
\makeatother

%Дублирование знаков при переносе
\newcommand*{\hm}[1]{#1\nobreak\discretionary{}%
	{\hbox{$\mathsurround=0pt #1$}}{}}

\usepackage{graphicx}
\graphicspath{{pic/}}
\DeclareGraphicsExtensions{.pdf,.png,.jpg}

%Изменеие параметров листа
\usepackage[left=15mm,right=15mm,
top=1cm,bottom=2cm,bindingoffset=0cm]{geometry}

%\usepackage{fancyhdr}
%\pagestyle{fancy}

\setlength\parindent{1,5em}
\usepackage{indentfirst}

\begin{document}
\section*{Билет 1}
\begin{enumerate}
	\item \textit{(1 балл)} Что такое угол? Назовите виды углов.
	\item \textit{(1 балл)} Что такое перпендикуляр к прямой? Что является основанием перпендикуляра? Сколько различных перпендикуляров можно опустить из точки на прямую?
	\item \textit{(2 балла)} Что такое прямоугольный треугольник? Как называются стороны прямоугольного треугольника? Сформулировать признаки равенства прямоугольных треугольников.
	\item \textit{(2 балла)} Что значит, что окружность вписана в треугольник? Где лежит центр окружности, вписанной в треугольник?
	\item \textit{(2 балла)} Что такое хорда? Какая хорда будет самая длинная в окружности? Объясните почему.
	\item \textit{(4 балла)} На стороне $AB$ квадрата $ABCD$ построен равносторонний треугольник $ABM$. Найдите угол $DMC$.
	\item \textit{(4 балла)} Хорда большей из двух концентрических окружностей касается меньшей. Докажите, что точка касания делит эту хорду пополам.
	\item \textit{(4 балла)} Окружность, построенная на стороне треугольника как на диаметре, проходит через середину другой стороны. Докажите, что треугольник равнобедренный.
	\item \textit{(Дополнительная задача)} Кошка сидит на середине лестницы, прислоненной к стене. Концы лестницы начинают скользить по стене и полу. Какова траектория движения кошки?
\end{enumerate}
\section*{Билет 2}
\begin{enumerate}
	\item \textit{(1 балл)} Что такое смежные углы? Каким свойством они обладают?
	\item \textit{(1 балл)} Что такое наклонная, проведенная из точки к прямой? В чем отличие наклонной от перпендикуляра?
	\item \textit{(2 балла)} Перечислите свойства равнобедренного треугольника. Перечислите свойства равностороннего треугольника.
	\item \textit{(2 балла)} Что значит, что окружность вписана в треугольник? Где лежит центр окружности, вписанной в треугольник?
	\item \textit{(2 балла)} Сформулируйте теорему о радиусе, проведенном через середину хорды. Верна ли обратная теорема?
	\item \textit{(4 балла)} Два угла треугольника равны $10\degree$ и $70\degree$. Найдите угол между высотой и биссектрисой, проведенными из вершины третьего угла треугольника.
	\item \textit{(4 балла)} Один из углов треугольника равен $60\degree$. Найдите угол между высотами, проведенными из вершин двух других углов.
	\item \textit{(4 балла)} Докажите, что окружность, построенная на стороне равностороннего треугольника как на диаметре, проходит через середины двух других сторон треугольника.
	\item \textit{(Дополнительная задача)} Кошка сидит на середине лестницы, прислоненной к стене. Концы лестницы начинают скользить по стене и полу. Какова траектория движения кошки?
\end{enumerate}
\section*{Билет 3}
\begin{enumerate}
\item \textit{(1 балл)} Что такое вертикальные углы? Каким свойством они обладают?
\item \textit{(1 балл)} Что такое перпендикуляр к прямой? Что можно сказать о двух различных перпендикулярах, проведенных к одной прямой?
\item \textit{(2 балла)} Как называется множество точек, равноудаленных от сторон некоторого угла? Объясните почему.
\item \textit{(2 балла)} Сформулируйте теорему об угле в 30\degree в прямоугольном треугольнике.
\item \textit{(2 балла)} Что такое секущая к окружности? Как называется часть секущей, заключенная в окружности? Что такое касательная к окружности? Сформулируйте теорему о касательных, проведенных из одной точки к окружности.
\item \textit{(4 балла)} На стороне $AB$ квадрата $ABCD$ построен равносторонний треугольник $ABM$. Найдите угол $DMC$.
\item \textit{(4 балла)} Через точку $A$, лежащую на окружности, проведены диаметр $AB$ и хорда $AC$, причем $AC = 8$ и $\angle BAC = 30\degree$. Найдите хорду $CM$, перпендикулярную $AB$.
\item \textit{(4 балла)} Окружность, построенная на стороне треугольника как на диаметре, проходит через середину другой стороны. Докажите, что треугольник равнобедренный.
\item \textit{(Дополнительная задача)} Кошка сидит на середине лестницы, прислоненной к стене. Концы лестницы начинают скользить по стене и полу. Какова траектория движения кошки?
\end{enumerate}
\section*{Билет 4}
\begin{enumerate}
\item \textit{(1 балл)} Назовите виды треугольников. Перечислите свойства равностороннего треугольника.
\item \textit{(1 балл)} Какие существуют пары углов, образованных пересечением двух прямых и секущей? Назовите их свойства, если прямые параллельны.
\item \textit{(2 балла)} Что означает, что прямые взаимно перпендикулярны? Могут ли три прямые быть попарно взаимно перпендикулярны? А четыре прямые?
\item \textit{(2 балла)} Что такое медиана треугольника? Сколько можно провести медиан в одном треугольнике?
\item \textit{(2 балла)} Что такое касательная к окружности? Сколько можно провести касательных к окружности через точку, которая находится вне окружности?
\item \textit{(4 балла)} Один из углов треугольника равен $50\degree$. Найдите угол между высотами, проведенными из вершин двух других углов.
\item \textit{(4 балла)} Известно, что $AB$ — диаметр окружности, а хорды $AC$ и $BD$ параллельны. Докажите, что $AC = BD$, а $CD$ также диаметр.
\item \textit{(4 балла)} Докажите, что окружность, построенная на стороне равностороннего треугольника как на диаметре, проходит через середины двух других сторон треугольника.
\item \textit{(Дополнительная задача)} Кошка сидит на середине лестницы, прислоненной к стене. Концы лестницы начинают скользить по стене и полу. Какова траектория движения кошки?
\end{enumerate}
\section*{Билет 5}
\begin{enumerate}
\item \textit{(1 балл)} Что является расстоянием от точки до точки?
\item \textit{(1 балл)} Чему равен угол между биссектрисами двух смежных углов? Объясните почему.
\item \textit{(2 балла)} Что такое высота треугольника? Сколько можно провести высот в одном треугольнике?
\item \textit{(2 балла)} Как называется множество точек, равноудаленных от концов некоторого отрезка? Объясните почему.
\item \textit{(2 балла)} Что такое окружность? Что такое секущая к окружности? Как называется часть секущей, заключенная в окружности? Сколько точек пересечения имеют окружность и секущая?
\item \textit{(4 балла)} На стороне $AB$ квадрата $ABCD$ построен равносторонний треугольник $ABM$. Найдите угол $DMC$.
\item \textit{(4 балла)} Биссектрисы двух углов треугольника пересекаются под углом $110\degree$. Найдите третий угол треугольника.
\item \textit{(4 балла)} Окружность, построенная на стороне треугольника как на диаметре, проходит через середину другой стороны. Докажите, что треугольник равнобедренный.
\item \textit{(Дополнительная задача)} Кошка сидит на середине лестницы, прислоненной к стене. Концы лестницы начинают скользить по стене и полу. Какова траектория движения кошки?
\end{enumerate}
\section*{Билет 6}
\begin{enumerate}
\item \textit{(1 балл)} Что такое угол? Назовите виды углов.
\item \textit{(1 балл)} Сформулируйте теорему о сумме углов треугольника.
\item \textit{(2 балла)} Что является расстоянием от точки до прямой?
\item \textit{(2 балла)} Что значит, что окружность описанна вокруг треугольника? Где лежит центр этой окружности?
\item \textit{(2 балла)} Докажите, что если через концы диаметра провести касательные к окружности, то эти касательные будут параллельны.
\item \textit{(4 балла)} Через вершину $B$ треугольника $ABC$ проведена прямая, параллельная прямой $AC$. Образовавшиеся при этом три угла с вершиной $B$ относятся как $3 : 10 : 5$. Найдите углы треугольника $ABC$.
\item \textit{(4 балла)} Угол при основании $BC$ равнобедренного треугольника $ABC$ вдвое больше угла при вершине $A$, $BD$ — биссектриса треугольника. Докажите, что $AD = BC$.
\item \textit{(4 балла)} Докажите, что окружность, построенная на стороне равностороннего треугольника как на диаметре, проходит через середины двух других сторон треугольника.
\item \textit{(Дополнительная задача)} Кошка сидит на середине лестницы, прислоненной к стене. Концы лестницы начинают скользить по стене и полу. Какова траектория движения кошки?
\end{enumerate}
\section*{Билет 7}
\begin{enumerate}
\item \textit{(1 балл)} Назовите виды треугольников. Перечислите свойства равнобедренного треугольника.
\item \textit{(1 балл)} Что такое биссектриса треугольника? Сколько можно провести биссектрис в одном треугольнике?
\item \textit{(2 балла)} Чему равен угол между касательной к окружности и радиусом, проведенным к точке касания? Объясните почему.
\item \textit{(2 балла)} Что можно сказать о треугольнике, если известно, что один из его углов равен сумме двух других его углов?
\item \textit{(2 балла)} Как называется множество точек, равноудаленных от концов некоторого отрезка? Объясните почему.
\item \textit{(4 балла)} Два угла треугольника равны $10\degree$ и $70\degree$. Найдите угол между высотой и биссектрисой, проведенными из вершины третьего угла треугольника.
\item \textit{(4 балла)} Биссектрисы двух углов треугольника пересекаются под углом $120\degree$. Найдите третий угол треугольника.
\item \textit{(4 балла)} Докажите, что окружность, построенная на стороне равностороннего треугольника как на диаметре, проходит через середины двух других сторон треугольника.
\item \textit{(Дополнительная задача)} Кошка сидит на середине лестницы, прислоненной к стене. Концы лестницы начинают скользить по стене и полу. Какова траектория движения кошки?
\end{enumerate}
\section*{Билет 8}
\begin{enumerate}
\item \textit{(1 балл)} Что такое угол? Назовите виды углов.
\item \textit{(1 балл)} Что можно сказать о двух прямых, если известно, что возможно провести третью прямую, которая будет перпендикулярна каждой из первых двух прямых?
\item \textit{(2 балла)} Где расположен центр окружности, описанной вокруг прямоугольного треугольника?
\item \textit{(2 балла)} Сформулируйте теорему о медиане, проведенной из вершины прямого угла в прямоугольном треугольнике.
\item \textit{(2 балла)} Что такое касательная к окружности? Сформулируйте теорему о касательных, проведенных из одной точки к окружности.
\item \textit{(4 балла)} Один из углов треугольника равен $70\degree$. Найдите угол между высотами, проведенными из вершин двух других углов.
\item \textit{(4 балла)} На катете $AC$ прямоугольного треугольника $ABC$ как на диаметре построена окружность, пересекающая гипотенузу $AB$ в точке $K$. Найдите $CK$, если $AC = 2$ и $\angle A = 30\degree$
\item \textit{(4 балла)} Докажите, что окружность, построенная на боковой стороне равнобедренного треугольника как на диаметре, проходит через середину основания.
\item \textit{(Дополнительная задача)} Кошка сидит на середине лестницы, прислоненной к стене. Концы лестницы начинают скользить по стене и полу. Какова траектория движения кошки?
\end{enumerate}
\section*{Билет 9}
\begin{enumerate}
\item \textit{(1 балл)} Что такое перпендикуляр к прямой? Что является основанием перпендикуляра? Сколько различных перпендикуляров можно опустить из точки на прямую?
\item \textit{(1 балл)} Что такое окружность? Какие окружности называются концентрическими?
\item \textit{(2 балла)} Чему равен угол между биссектрисами двух внутренних односторонних углов?
\item \textit{(2 балла)} Чему равен угол между касательной к окружности и радиусом, проведенным к точке касания? Объясните почему.
\item \textit{(2 балла)} Перечислите четыре замечательные точки треугольника.
\item \textit{(4 балла)} Через вершину $B$ треугольника $ABC$ проведена прямая, параллельная прямой $AC$. Образовавшиеся при этом три угла с вершиной $B$ относятся как $3 : 10 : 5$. Найдите углы треугольника $ABC$.
\item \textit{(4 балла)} Хорда большей из двух концентрических окружностей касается меньшей. Докажите, что точка касания делит эту хорду пополам.
\item \textit{(4 балла)} Докажите, что окружность, построенная на боковой стороне равнобедренного треугольника как на диаметре, проходит через середину основания.
\item \textit{(Дополнительная задача)} Кошка сидит на середине лестницы, прислоненной к стене. Концы лестницы начинают скользить по стене и полу. Какова траектория движения кошки?
\end{enumerate}
\end{document}