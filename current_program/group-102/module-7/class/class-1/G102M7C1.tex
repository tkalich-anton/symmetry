\documentclass[12pt, a4paper]{article}
\usepackage{../../../../../style}
\DeclareUnicodeCharacter{202F}{\,}
\begin{document}
	
\lhead{Группа 102} \chead{Модуль 7} \rhead{Школа <<Симметрия>>} \cfoot{}
\begin{center}
	\Large
	\textbf{Занятие №2}
\end{center}
\begin{enumerate}[label=\textbf{\arabic*.}]
	\item По вкладу «А» банк в течение трёх лет в конце каждого года увеличивает на 10\% сумму, имеющуюся на вкладе в начале года, а по вкладу «Б» — увеличивает на 11\% в течение каждого из первых двух лет. Найдите наименьшее целое число процентов за третий год по вкладу «Б», при котором за все три года этот вклад всё ещё останется выгоднее вклада «А».
	\item Планируется открыть вклад на 4 года, положив на счет целое число миллионов рублей. В конце каждого года сумма, лежащая на вкладе, увеличивается на 10\%, а в начале третьего и четвертого года вклад пополняется на 3 миллиона рублей. Найдите наименьший первоначальный вклад, при котором начисленные проценты за весь срок будут более 5 миллионов рублей.
	\item Решить уравнения:
	\begin{multicols}{2}
		\begin{enumerate}[label=\asbuk*)]
			\item \( \sin x - \sqrt{3}\cos x = 0 \)
			\item \( \sin x + 5\cos x = 0 \)
			\item \( \sin^2 x - 3\sin x\cos x + 2\cos^2 x = 0 \)
			\item \( \dfrac{1}{\cos^2 x}+\dfrac{3}{\sin\left( \dfrac{\pi}{2}+x \right)}+2=0 \)
			\item \( \sin^3 x - 7\sin x \cos^2 x + 6\cos^3 x = 0 \)
		\end{enumerate}
	\end{multicols}
\end{enumerate}
\end{document}