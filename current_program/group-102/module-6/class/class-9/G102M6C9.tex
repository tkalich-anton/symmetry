\documentclass[12pt, a4paper]{article}
\usepackage{../../../../../style}
\DeclareUnicodeCharacter{202F}{\,}
\begin{document}
	
\lhead{Группа 102} \chead{Модуль 7} \rhead{Школа <<Симметрия>>} \cfoot{}
\begin{center}
	\Large
	\textbf{Занятие №1}
\end{center}
\begin{enumerate}[label=\textbf{\arabic*.}]
	\item Население города за \( 2 \) года увеличилось с \( 20 000 \) до \( 22 050 \) человек. Найдите средний ежегодный процент роста населения этого города.
	\item Саша и Паша положили в банк по 50 000 рублей на три года под 10\% годовых Однако через год и Саша, и Паша сняли со своих счетов соответственно 10\% и 20\% имеющихся денег. Еще через год каждый из них снял со своего счета соответственно 20 000 рублей и 15 000 рублей. У кого из братьев к концу третьего года на счету окажется большая сумма денег? На сколько рублей?
	\item В банк был положен вклад под 10\% годовых. Через год, после начисления процентов, вкладчик снял со счета 2000 рублей, а еще через год (опять после начисления процентов) снова внес 2000 рублей. Вследствие этих действий через три года со времени открытия вклада вкладчик получил сумму меньше запланированной (если бы не было промежуточных операций со вкладом). На сколько рублей меньше запланированной суммы он получил?
	\item По вкладу «А» банк в течение трёх лет в конце каждого года увеличивает на 10\% сумму, имеющуюся на вкладе в начале года, а по вкладу «Б» — увеличивает на 11\% в течение каждого из первых двух лет. Найдите наименьшее целое число процентов за третий год по вкладу «Б», при котором за все три года этот вклад всё ещё останется выгоднее вклада «А».
	\item Планируется открыть вклад на 4 года, положив на счет целое число миллионов рублей. В конце каждого года сумма, лежащая на вкладе, увеличивается на 10\%, а в начале третьего и четвертого года вклад пополняется на 3 миллиона рублей. Найдите наименьший первоначальный вклад, при котором начисленные проценты за весь срок будут более 5 миллионов рублей.
\end{enumerate}
\end{document}