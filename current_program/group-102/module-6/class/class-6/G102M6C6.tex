\documentclass[12pt, a4paper]{article}
\usepackage{../../../../../style}
\DeclareUnicodeCharacter{202F}{\,}
\begin{document}
	
\lhead{Группа 102} \chead{Модуль 6} \rhead{Школа <<Симметрия>>} \cfoot{}
\begin{center}
	\Large
	\textbf{Занятие №6}
\end{center}
\begin{enumerate}[label=\textbf{\arabic*.}]
	\item Длина ребра правильного тетраэдра $ABCD$ равна $1$. $M$ --- середина ребра $BC$, $L$ --- середина ребра $AB$.
	\begin{enumerate}[label=\asbuk*)]
		\item Докажите, что плоскость параллельная прямой $CL$ и содержащая прямую $DM$, делит ребро $AB$ в отношении $3 : 1$, считаю от вершины $A$.
		\item Найдите угол между прямыми $DM$ и $CL$.
	\end{enumerate}
	\item Дан прямоугольный параллелепипед $ABCDA_1B_1C_1D_1$.
	\begin{enumerate}[label=\asbuk*)]
		\item Докажите, что все грани тетраэдра $ACB_1D_1$ — равные треугольники (тетраэдр, обладающий таким свойством, называют \textit{равногранным}).
		\item Найдите угол между плоскостью $A_1BC$ и прямой $BC_1$, если $AA_1 = 8$, $AB = 6$, $BC = 15$.
	\end{enumerate}
	\item Дана правильная треугольная призма $ABCA_1B_1C_1$ с основаниями $ABC$ и $A_1B_1C_1$. Скрещивающиеся диагонали $BA_1$ и $CB_1$ боковых граней $AA_1B_1B$ и $BB_1C_1C$ перпендикулярны.
	\begin{enumerate}[label=\asbuk*)]
		\item Докажите, что $AB : AA_1 =\sqrt{2} : 1$.
		\item Найдите угол между прямой $BA_1$ и плоскостью $BCC_1$.
	\end{enumerate}
\end{enumerate}
\end{document}