\documentclass[12pt, a4paper]{article}
\usepackage{cmap} % Улучшенный поиск русских слов в полученном pdf-файле
\usepackage[T2A]{fontenc} % Поддержка русских букв
\usepackage[utf8]{inputenc} % Кодировка utf8
\usepackage[english, russian]{babel} % Языки: русский, английский
\usepackage{enumitem}
\usepackage{pscyr} % Нормальные шрифты
\usepackage{soulutf8}
\usepackage{amsmath}
\usepackage{amsthm}
\usepackage{amssymb}
\usepackage{scrextend}
\usepackage{titling}
\usepackage{indentfirst}
\usepackage{cancel}
\usepackage{soulutf8}
\usepackage{wrapfig}
\usepackage{gensymb}
\usepackage[dvipsnames,table,xcdraw]{xcolor}
\usepackage{tikz}
\usepackage{multicol}

%Русские символы в списке
\makeatletter
\AddEnumerateCounter{\asbuk}{\russian@alph}{щ}
\makeatother

%Дублирование знаков при переносе
\newcommand*{\hm}[1]{#1\nobreak\discretionary{}%
	{\hbox{$\mathsurround=0pt #1$}}{}}

\usepackage{graphicx}
\graphicspath{{pic/}}
\DeclareGraphicsExtensions{.pdf,.png,.jpg}

%Изменеие параметров листа
\usepackage[left=15mm,right=15mm,
top=1cm,bottom=2cm,bindingoffset=0cm]{geometry}

%\usepackage{fancyhdr}
%\pagestyle{fancy}

\setlength\parindent{1,5em}
\usepackage{indentfirst}

\begin{document}
	
	\section*{Билет 1}
	\begin{enumerate}
		\item \textit{(1 балл)} В каких четвертях знаки синуса и косинуса совпадают?
		\item \textit{(1 балл)} Какой четверти может принадлежать угол $x$, если $\tg x$ отрицательный?
		\item \textit{(1 балл)} Переведите 30 градусов в радианы.
		\item \textit{(1 балл)} Назовите хотя бы один угол в радианной мере, косинус которого равен 1.
		\item \textit{(1 балл)} Сформулируйте основное тригонометрическое тождество.
		\item \textit{(1 балл)} Вычислите $\cos (-30^{\circ})$.
		\item \textit{(1 балл)} Вычислите $\sin 405^{\circ}$.
		\item \textit{(2 балла)} Выведите формулу $\sin x \cdot \cos y$.
		\item \textit{(2 балла)} Выведите формулу $\cos x - \cos y$.
		\item \textit{(3 балла)} Вычислите $\sin (2x-\pi)  \cos (x-3\pi)  + \sin (2x-\dfrac{9\pi}{2}) \cos (x+\dfrac{\pi}{2})$
		\item \textit{(3 балла)} Упростите выражение $sin(-1,3\pi)cos(-1,7\pi)tg(-0,7\pi)+\sin 0,8\pi \cos 1,8\pi \tg 1,2\pi$
<<<<<<< HEAD
		\item \textit{(3 балла)} Упростите выражение $\sin^2 x(1+\sin^{-1}x + \ctg x)(1 - \sin^{-1}x + \ctg x)$
=======
		\item \textit{(3 балла)} Докажите тождество $\sin^6x + \cos^6x + 3\sin^2x\cos^2x = 1$
>>>>>>> e1dd97d14c11fb9b3168e986de1c5be79f526490
	\end{enumerate}
	\section*{Билет 2}
	\begin{enumerate}
		\item \textit{(1 балл)}	В каких четвертях отрицательный синус, тангенс? 
		\item \textit{(1 балл)}	Какой четверти принадлежит $x$ , если $\cos x$ отрицательный?
		\item \textit{(1 балл)}	Переведите 150 градусов в радианы.
<<<<<<< HEAD
		\item \textit{(1 балл)}	Назовите хотя бы один угол в радианной мере, при котором $\sin$  равен $1$
=======
		\item \textit{(1 балл)}	Назовите угол, при котором $\sin$  равен $1$
>>>>>>> e1dd97d14c11fb9b3168e986de1c5be79f526490
		\item \textit{(1 балл)} Выразите тангенс через синус и косинус
		\item \textit{(1 балл)} Вычислите $\tg (-45^{\circ})$
		\item \textit{(1 балл)}	Вычислите $\cos (390^{\circ})$.
		\item \textit{(2 балла)} Выведите формулу $\sin x \cdot sin y$ 
		\item \textit{(2 балла)} Выведите формулу $cos x + \cos y$.
		\item \textit{(3 балла)} Вычислите $\dfrac{\tg(\dfrac{3\pi}{2}-x)-\cos (\pi-x) \sin (3\pi + x)}{(\cos (3,5\pi - x) + \sin (1,5\pi + x))^2-1}$
		\item \textit{(3 балла)} Упростите выражение $\sin(x+2\pi) \cos (2x-\dfrac{7\pi}{2}) + \sin (\dfrac{3\pi}{2}-x) \sin (2x-\dfrac{5\pi}{2})$
<<<<<<< HEAD
		\item \textit{(3 балла)} Упростите выражение $\sin^2 x(1+\sin^{-1}x + \ctg x)(1 - \sin^{-1}x + \ctg x)$
	\end{enumerate}
	
	\newpage 
	
	\section*{Билет 3}
	\begin{enumerate}
		\item \textit{(1 балл)}	Есть ли четверти, в которых тангенс и синус положительны одновременно? 
		\item \textit{(1 балл)}	Какой четверти принадлежит $x$ , если $\cos x$ положительный?
		\item \textit{(1 балл)}	Переведите 120 градусов в радианы
		\item \textit{(1 балл)}	Назовите хотя бы один угол в радианной мере, при котором $\cos$  равен $0,5$.
		\item \textit{(1 балл)} Чему равно произведение тангенса и котангенс.
		\item \textit{(1 балл)} Вычислите $\sin (-60^{\circ})$.
		\item \textit{(1 балл)}	Вычислите $\cos 420^{\circ}$.
		\item \textit{(2 балла)} Выведите формулу $\sin x \cdot \cos y$.
		\item \textit{(2 балла)} Выведите формулу $\sin x - \sin y$.
		\item \textit{(3 балла)} Вычислите $(\dfrac{\cos (2,5\pi + x)}{\ctg (3\pi + x)} - \sin (-x) \tg (\dfrac{5\pi}{2}+x))^2 + \dfrac{\tg x}{tg (\dfrac{3\pi}{2}+x)}$
		\item \textit{(3 балла)} Упростите выражение $\dfrac{\ctg(270^{\circ}-x)}{1-\tg^2(x-180^{\circ})} \cdot \dfrac{\ctg^2(360^{\circ}-x)-1}{\ctg(180^{\circ}+x)}$
		\item \textit{(3 балла)} Упростите выражение $\sin^2 x(1+\sin^{-1}x + \ctg x)(1 - \sin^{-1}x + \ctg x)$
	\end{enumerate}
	\section*{Билет 4}
	\begin{enumerate}
		\item \textit{(1 балл)}	В каких четвертях знаки синуса и косинуса совпадают? 
		\item \textit{(1 балл)}	Какой четверти принадлежит x , если tg x положительный?
		\item \textit{(1 балл)}	Переведите 30 градусов в радианы
		\item \textit{(1 балл)}	Назовите хотя бы один угол в радианной мере, при котором $\tg$  равен $1$
		\item \textit{(1 балл)} Выразите котангенс через синус и косинус
		\item \textit{(1 балл)} Вычислите $\tg (-45^{\circ})$
		\item \textit{(1 балл)}	Вычислите $\tg (420^{\circ})$.
		\item \textit{(2 балла)} Выведите формулу $\sin x \cdot cos y$ 
		\item \textit{(2 балла)} Выведите формулу $cos x + \cos y$.
		\item \textit{(3 балла)} Вычислите $\sin(-1,3\pi) \cos (-1,7 \pi) \tg (-0,7 \pi)+ \sin 0,8 \pi \cos 1,8 \pi \tg 1,2 \pi$
		\item \textit{(3 балла)} Упростите выражение $\dfrac{\cos^2 (x-270^{\circ})}{\sin^{-2}(x+90^{\circ})-1}+\dfrac{\sin^2 (x+270^{\circ})}{\cos^{-2}(x-90^{\circ})-1}$
		\item \textit{(3 балла)} Упростите выражение $\sin^2 x(1+\sin^{-1}x + \ctg x)(1 - \sin^{-1}x + \ctg x)$
	\end{enumerate}
	
	
	\newpage 
	
	\section*{Билет 5}
	\begin{enumerate}
		\item \textit{(1 балл)}	Назовите четверти, в которых положителен синус, косинус 
		\item \textit{(1 балл)}	Какой четверти принадлежит $x$ , если $\sin x$ положительный?
		\item \textit{(1 балл)}	Переведите 90 градусов в радианы
		\item \textit{(1 балл)}	Назовите хотя бы один угол в радианной мере, при котором $\sin$  равен $0,5$.
		\item \textit{(1 балл)}	Выразите котангенс через синус и косинус.
		\item \textit{(1 балл)} Вычислите $\sin (-45^{\circ})$.
		\item \textit{(1 балл)}	Вычислите $\sin 405^{\circ}$.
		\item \textit{(2 балла)} Выведите формулу $\cos x \cdot \cos y$.
		\item \textit{(2 балла)} Выведите формулу $\sin x + \sin y$.
		\item \textit{(3 балла)} Вычислите $(\dfrac{\cos (2,5\pi + x)}{\ctg (3\pi + x)} - \sin (-x) \tg (\dfrac{5\pi}{2}+x))^2 + \dfrac{\tg x}{tg (\dfrac{3\pi}{2}+x)}$
		\item \textit{(3 балла)} Упростите выражение $\dfrac{\cos^2 (x-270^{\circ})}{\sin^{-2}(x+90^{\circ})-1}+\dfrac{\sin^2 (x+270^{\circ})}{\cos^{-2}(x-90^{\circ})-1}$
		\item \textit{(3 балла)} Упростите выражение $\sin^2 x(1+\sin^{-1}x + \ctg x)(1 - \sin^{-1}x + \ctg x)$
	\end{enumerate}
	\section*{Билет 6}
	\begin{enumerate}
		\item \textit{(1 балл)}	Назовите четверти, в которых положителен синус, косинус 
		\item \textit{(1 балл)}	Какой четверти принадлежит $x$ , если $\sin x$ отрицательный?
		\item \textit{(1 балл)}	Переведите 90 градусов в радианы
		\item \textit{(1 балл)}	Назовите хотя бы один угол в радианной мере, при котором $\cos$  равен $0,5$
		\item \textit{(1 балл)} Напишите основное тригонометрическое тождество
		\item \textit{(1 балл)} Вычислите $\cos (-30^{\circ})$
		\item \textit{(1 балл)}	Вычислите $\sin (390^{\circ})$.
		\item \textit{(2 балла)} Выведите формулу $\sin x \cdot cos y$ 
		\item \textit{(2 балла)} Выведите формулу $cos x - \cos y$.
		\item \textit{(3 балла)} Вычислите $\sin(-1,3\pi) \cos (-1,7 \pi) \tg (-0,7 \pi)+ \sin 0,8 \pi \cos 1,8 \pi \tg 1,2 \pi$
		\item \textit{(3 балла)} Упростите выражение $\sin(2x-\pi) \cos (x-3\pi) + \sin (2x- \dfrac{9 \pi}{2}) \cos (x+\dfrac{\pi}{2})$
		\item \textit{(3 балла)} Упростите выражение $\sin^2 x(1+\sin^{-1}x + \ctg x)(1 - \sin^{-1}x + \ctg x)$
	\end{enumerate}
	
	\newpage 
	
	\section*{Билет 7}
	\begin{enumerate}
		\item \textit{(1 балл)}	В каких четвертях знаки синуса и косинуса совпадают? 
		\item \textit{(1 балл)}	Какой четверти принадлежит $x$ , если $\cos x$ положительный?
		\item \textit{(1 балл)}	Переведите 150 градусов в радианы
		\item \textit{(1 балл)}	Назовите хотя бы один угол в радианной мере, при котором $\tg$  равен $1$.
		\item \textit{(1 балл)}	Напишите основное тригонометрическое тождество.
		\item \textit{(1 балл)} Вычислите $\tg (-45^{\circ})$.
		\item \textit{(1 балл)}	Вычислите $\sin 390^{\circ}$.
		\item \textit{(2 балла)} Выведите формулу $\sin x \cdot \cos y$.
		\item \textit{(2 балла)} Выведите формулу $\sin x - \sin y$.
		\item \textit{(3 балла)} Вычислите $(\dfrac{\cos (2,5\pi + x)}{\ctg (3\pi + x)} - \sin (-x) \tg (\dfrac{5\pi}{2}+x))^2 + \dfrac{\tg x}{tg (\dfrac{3\pi}{2}+x)}$
		\item \textit{(3 балла)} Упростите выражение $\sin (x+2\pi) \cos (2x-\dfrac{7\pi}{2})+ \sin (\dfrac{3\pi}{2}-x) \sin (2x - \dfrac{5\pi}{2})$
		\item \textit{(3 балла)} Упростите выражение $\sin^2 x(1+\sin^{-1}x + \ctg x)(1 - \sin^{-1}x + \ctg x)$
	\end{enumerate}
	\section*{Билет 8}
	\begin{enumerate}
		\item \textit{(1 балл)}	В каких четвертях отрицательный синус, тангенс?
		\item \textit{(1 балл)}	Какой четверти принадлежит $x$ , если $\tg x$ положительный?
		\item \textit{(1 балл)}	Переведите 90 градусов в радианы
		\item \textit{(1 балл)}	Назовите хотя бы один угол в радианной мере, при котором $\cos$  равен $0,5$
		\item \textit{(1 балл)} Выразите тангенс через синус и косинус
		\item \textit{(1 балл)} Вычислите $\sin (-60^{\circ})$
		\item \textit{(1 балл)}	Вычислите $\cos (420^{\circ})$.
		\item \textit{(2 балла)} Выведите формулу $\sin x \cdot \sin y$ 
		\item \textit{(2 балла)} Выведите формулу $\cos x + \cos y$.
		\item \textit{(3 балла)} Вычислите $\dfrac{\tg (\dfrac{3\pi}{2}-x)-\cos(\pi - x) \sin(3\pi + x)}{(\cos(3,5\pi - x)+ \sin (1,5 \pi + x))^2-1}$
		\item \textit{(3 балла)} Упростите выражение $\dfrac{\ctg(270^{\circ}-x)}{1-\tg^2(x-180^{\circ})} \cdot \dfrac{\ctg^2(360^{\circ}-x)-1}{\ctg(180^{\circ}+x)}$
		\item \textit{(3 балла)} Упростите выражение $\sin^2 x(1+\sin^{-1}x + \ctg x)(1 - \sin^{-1}x + \ctg x)$
	\end{enumerate}
	
	\newpage 
	
	\section*{Билет 9}
	\begin{enumerate}
		\item \textit{(1 балл)}	Есть ли четверти, в которых тангенс и синус положительны одновременно? 
		\item \textit{(1 балл)}	Какой четверти принадлежит $x$ , если $\sin x$ отрицательный?
		\item \textit{(1 балл)}	Переведите 30 градусов в радианы
		\item \textit{(1 балл)}	Назовите хотя бы один угол в радианной мере, при котором $\sin$  равен $0,5$.
		\item \textit{(1 балл)}	Выразите тангенс через синус и косинус.
		\item \textit{(1 балл)} Вычислите $\cos (-30^{\circ})$.
		\item \textit{(1 балл)}	Вычислите $\cos 420^{\circ}$.
		\item \textit{(2 балла)} Выведите формулу $\sin x \cdot \sin y$.
		\item \textit{(2 балла)} Выведите формулу $\sin x + \sin y$.
		\item \textit{(3 балла)} Вычислите $\sin (-1,3\pi) \cos (-1,7 \pi) \tg (-0,7\pi)+ \sin 0,8\pi \cos 1,8 \pi \tg 1,2 \pi$
		\item \textit{(3 балла)} Упростите выражение $\sin (x+2\pi) \cos (2x-\dfrac{7\pi}{2})+ \sin (\dfrac{3\pi}{2}-x) \sin (2x - \dfrac{5\pi}{2})$
		\item \textit{(3 балла)} Упростите выражение $\sin^2 x(1+\sin^{-1}x + \ctg x)(1 - \sin^{-1}x + \ctg x)$
	\end{enumerate}
=======
		\item \textit{(3 балла)} Докажите тождество $\sin^6x + cos^6x + 3\sin^2x \ cos^2x = 1$
	\end{enumerate}

\newpage 

\section*{Билет 3}
\begin{enumerate}
	\item \textit{(1 балл)}	Есть ли четверти, в которых тангенс и синус положительны одновременно? 
	\item \textit{(1 балл)}	Какой четверти принадлежит $x$ , если $\cos x$ положительный?
	\item \textit{(1 балл)}	Переведите 120 градусов в радианы
	\item \textit{(1 балл)}	Назовите угол, при котором $\cos$  равен $0,5$.
	\item \textit{(1 балл)} Чему равно произведение тангенса и котангенс.
	\item \textit{(1 балл)} Вычислите $\sin (-60^{\circ})$.
	\item \textit{(1 балл)}	Вычислите $\cos 420^{\circ}$.
	\item \textit{(2 балла)} Выведите формулу $\sin x \cdot \cos y$.
	\item \textit{(2 балла)} Выведите формулу $\sin x - \sin y$.
	\item \textit{(3 балла)} Вычислите $(\dfrac{\cos (2,5\pi + x)}{\ctg (3\pi + x)} - \sin (-x) \tg (\dfrac{5\pi}{2}+x))^2 + \dfrac{\tg x}{tg (\dfrac{3\pi}{2}+x)}$
	\item \textit{(3 балла)} Упростите выражение $\dfrac{\ctg(270^{\circ}-x)}{1-\tg^2(x-180^{\circ})} \cdot \dfrac{\ctg^2(360^{\circ}-x)-1}{\ctg(180^{\circ}+x)}$
	\item \textit{(3 балла)} Докажите тождество $\sin^6x + \cos^6x + 3\sin^2x\cos^2x = 1$
\end{enumerate}
\section*{Билет 4}
\begin{enumerate}
	\item \textit{(1 балл)}	В каких четвертях знаки синуса и косинуса совпадают? 
	\item \textit{(1 балл)}	Какой четверти принадлежит x , если tg x положительный?
	\item \textit{(1 балл)}	Переведите 30 градусов в радианы
	\item \textit{(1 балл)}	Назовите угол, при котором $\tg$  равен $1$
	\item \textit{(1 балл)} Выразите котангенс через синус и косинус
	\item \textit{(1 балл)} Вычислите $\tg (-45^{\circ})$
	\item \textit{(1 балл)}	Вычислите $\tg (420^{\circ})$.
	\item \textit{(2 балла)} Выведите формулу $\sin x \cdot cos y$ 
	\item \textit{(2 балла)} Выведите формулу $cos x + \cos y$.
	\item \textit{(3 балла)} Вычислите $\sin(-1,3\pi) \cos (-1,7 \pi) \tg (-0,7 \pi)+ \sin 0,8 \pi \cos 1,8 \pi \tg 1,2 \pi$
	\item \textit{(3 балла)} Упростите выражение $\dfrac{\cos^2 (x-270^{\circ})}{\sin^{-2}(x+90^{\circ})-1}+\dfrac{\sin^2 (x+270^{\circ})}{\cos^{-2}(x-90^{\circ})-1}$
	\item \textit{(3 балла)} Докажите тождество $\sin^6x + cos^6x + 3\sin^2x \ cos^2x = 1$
\end{enumerate}


\newpage 

\section*{Билет 5}
\begin{enumerate}
	\item \textit{(1 балл)}	Назовите четверти, в которых положителен синус, косинус 
	\item \textit{(1 балл)}	Какой четверти принадлежит $x$ , если $\sin x$ положительный?
	\item \textit{(1 балл)}	Переведите 90 градусов в радианы
	\item \textit{(1 балл)}	Назовите угол, при котором $\sin$  равен $0,5$.
	\item \textit{(1 балл)}	Выразите котангенс через синус и косинус.
	\item \textit{(1 балл)} Вычислите $\sin (-45^{\circ})$.
	\item \textit{(1 балл)}	Вычислите $\sin 405^{\circ}$.
	\item \textit{(2 балла)} Выведите формулу $\cos x \cdot \cos y$.
	\item \textit{(2 балла)} Выведите формулу $\sin x + \sin y$.
	\item \textit{(3 балла)} Вычислите $(\dfrac{\cos (2,5\pi + x)}{\ctg (3\pi + x)} - \sin (-x) \tg (\dfrac{5\pi}{2}+x))^2 + \dfrac{\tg x}{tg (\dfrac{3\pi}{2}+x)}$
	\item \textit{(3 балла)} Упростите выражение $\dfrac{\cos^2 (x-270^{\circ})}{\sin^{-2}(x+90^{\circ})-1}+\dfrac{\sin^2 (x+270^{\circ})}{\cos^{-2}(x-90^{\circ})-1}$
	\item \textit{(3 балла)} Докажите тождество $\sin^6x + \cos^6x + 3\sin^2x\cos^2x = 1$
\end{enumerate}
\section*{Билет 6}
\begin{enumerate}
	\item \textit{(1 балл)}	Назовите четверти, в которых положителен синус, косинус 
	\item \textit{(1 балл)}	Какой четверти принадлежит $x$ , если $\sin x$ отрицательный?
	\item \textit{(1 балл)}	Переведите 90 градусов в радианы
	\item \textit{(1 балл)}	Назовите угол, при котором $\cos$  равен $0,5$
	\item \textit{(1 балл)} Напишите основное тригонометрическое тождество
	\item \textit{(1 балл)} Вычислите $\cos (-30^{\circ})$
	\item \textit{(1 балл)}	Вычислите $\sin (390^{\circ})$.
	\item \textit{(2 балла)} Выведите формулу $\sin x \cdot cos y$ 
	\item \textit{(2 балла)} Выведите формулу $cos x - \cos y$.
	\item \textit{(3 балла)} Вычислите $\sin(-1,3\pi) \cos (-1,7 \pi) \tg (-0,7 \pi)+ \sin 0,8 \pi \cos 1,8 \pi \tg 1,2 \pi$
	\item \textit{(3 балла)} Упростите выражение $\sin(2x-\pi) \cos (x-3\pi) + \sin (2x- \dfrac{9 \pi}{2}) \cos (x+\dfrac{\pi}{2})$
	\item \textit{(3 балла)} Докажите тождество $\sin^6x + cos^6x + 3\sin^2x \ cos^2x = 1$
\end{enumerate}

\newpage 

\section*{Билет 7}
\begin{enumerate}
\item \textit{(1 балл)}	В каких четвертях знаки синуса и косинуса совпадают? 
\item \textit{(1 балл)}	Какой четверти принадлежит $x$ , если $\cos x$ положительный?
\item \textit{(1 балл)}	Переведите 150 градусов в радианы
\item \textit{(1 балл)}	Назовите угол, при котором $\tg$  равен $1$.
\item \textit{(1 балл)}	Напишите основное тригонометрическое тождество.
\item \textit{(1 балл)} Вычислите $\tg (-45^{\circ})$.
\item \textit{(1 балл)}	Вычислите $\sin 390^{\circ}$.
\item \textit{(2 балла)} Выведите формулу $\sin x \cdot \cos y$.
\item \textit{(2 балла)} Выведите формулу $\sin x - \sin y$.
\item \textit{(3 балла)} Вычислите $(\dfrac{\cos (2,5\pi + x)}{\ctg (3\pi + x)} - \sin (-x) \tg (\dfrac{5\pi}{2}+x))^2 + \dfrac{\tg x}{tg (\dfrac{3\pi}{2}+x)}$
\item \textit{(3 балла)} Упростите выражение $\sin (x+2\pi) \cos (2x-\dfrac{7\pi}{2})+ \sin (\dfrac{3\pi}{2}-x) \sin (2x - \dfrac{5\pi}{2})$
\item \textit{(3 балла)} Докажите тождество $\sin^6x + \cos^6x + 3\sin^2x\cos^2x = 1$
\end{enumerate}
\section*{Билет 8}
\begin{enumerate}
\item \textit{(1 балл)}	В каких четвертях отрицательный синус, тангенс?
\item \textit{(1 балл)}	Какой четверти принадлежит $x$ , если $\tg x$ положительный?
\item \textit{(1 балл)}	Переведите 90 градусов в радианы
\item \textit{(1 балл)}	Назовите угол, при котором $\cos$  равен $0,5$
\item \textit{(1 балл)} Выразите тангенс через синус и косинус
\item \textit{(1 балл)} Вычислите $\sin (-60^{\circ})$
\item \textit{(1 балл)}	Вычислите $\cos (420^{\circ})$.
\item \textit{(2 балла)} Выведите формулу $\sin x \cdot \sin y$ 
\item \textit{(2 балла)} Выведите формулу $\cos x + \cos y$.
\item \textit{(3 балла)} Вычислите $\dfrac{\tg (\dfrac{3\pi}{2}-x)-\cos(\pi - x) \sin(3\pi + x)}{(\cos(3,5\pi - x)+ \sin (1,5 \pi + x))^2-1}$
\item \textit{(3 балла)} Упростите выражение $\dfrac{\ctg(270^{\circ}-x)}{1-\tg^2(x-180^{\circ})} \cdot \dfrac{\ctg^2(360^{\circ}-x)-1}{\ctg(180^{\circ}+x)}$
\item \textit{(3 балла)} Докажите тождество $\sin^6x + cos^6x + 3\sin^2x \ cos^2x = 1$
\end{enumerate}

\newpage 

\section*{Билет 9}
\begin{enumerate}
	\item \textit{(1 балл)}	Есть ли четверти, в которых тангенс и синус положительны одновременно? 
	\item \textit{(1 балл)}	Какой четверти принадлежит $x$ , если $\sin x$ отрицательный?
	\item \textit{(1 балл)}	Переведите 30 градусов в радианы
	\item \textit{(1 балл)}	Назовите угол, при котором $\sin$  равен $0,5$.
	\item \textit{(1 балл)}	Выразите тангенс через синус и косинус.
	\item \textit{(1 балл)} Вычислите $\cos (-30^{\circ})$.
	\item \textit{(1 балл)}	Вычислите $\cos 420^{\circ}$.
	\item \textit{(2 балла)} Выведите формулу $\sin x \cdot \sin y$.
	\item \textit{(2 балла)} Выведите формулу $\sin x + \sin y$.
	\item \textit{(3 балла)} Вычислите $\sin (-1,3\pi) \cos (-1,7 \pi) \tg (-0,7\pi)+ \sin 0,8\pi \cos 1,8 \pi \tg 1,2 \pi$
	\item \textit{(3 балла)} Упростите выражение $\sin (x+2\pi) \cos (2x-\dfrac{7\pi}{2})+ \sin (\dfrac{3\pi}{2}-x) \sin (2x - \dfrac{5\pi}{2})$
	\item \textit{(3 балла)} Докажите тождество $\sin^6x + \cos^6x + 3\sin^2x\cos^2x = 1$
\end{enumerate}
>>>>>>> e1dd97d14c11fb9b3168e986de1c5be79f526490
\end{document}
