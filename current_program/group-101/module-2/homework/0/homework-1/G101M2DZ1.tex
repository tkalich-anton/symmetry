\documentclass[12pt, a4paper]{article}
\usepackage{cmap} % Улучшенный поиск русских слов в полученном pdf-файле
\usepackage[T2A]{fontenc} % Поддержка русских букв
\usepackage[utf8]{inputenc} % Кодировка utf8
\usepackage[english, russian]{babel} % Языки: русский, английский
\usepackage{enumitem}
\usepackage{pscyr} % Нормальные шрифты
\usepackage{soulutf8}
\usepackage{amsmath}
\usepackage{amsthm}
\usepackage{amssymb}
\usepackage{scrextend}
\usepackage{titling}
\usepackage{indentfirst}
\usepackage{cancel}
\usepackage{soulutf8}
\usepackage{wrapfig}
\usepackage{gensymb}
\usepackage[dvipsnames,table,xcdraw]{xcolor}
\usepackage{tikz}

%Русские символы в списке
\makeatletter
\AddEnumerateCounter{\asbuk}{\russian@alph}{щ}
\makeatother

%Дублирование знаков при переносе
\newcommand*{\hm}[1]{#1\nobreak\discretionary{}%
	{\hbox{$\mathsurround=0pt #1$}}{}}

\usepackage{graphicx}
\graphicspath{{pic/}}
\DeclareGraphicsExtensions{.pdf,.png,.jpg}

%Изменеие параметров листа
\usepackage[left=15mm,right=15mm,
top=2cm,bottom=2cm,bindingoffset=0cm]{geometry}


\usepackage{fancyhdr}
\pagestyle{fancy}
\usepackage{multicol}

\setlength\parindent{1,5em}
\usepackage{indentfirst}

\begin{document}
		
\lhead{Группа 101}
\chead{Модуль 2 ДЗ№1}
\rhead{Школа <<Симметрия>>}

\section*{Тригонометрия. Решение иррациональных уравнений.}
\begin{enumerate}
	\item \textit{(2 балла)} Вычислите:
	\begin{multicols}{2}
		\begin{enumerate}[label=\asbuk*)]
			\item $4\cos{45\degree}\cdot\ctg{60\degree}\tg{60\degree}-3\sin{45\degree}$
			\item $\dfrac{1-2\sin^2{60\degree}}{2\cos^2{60\degree}-1}$
			\item $2\cos{30\degree}-\ctg{45}+\sin^2{60\degree}+\ctg^2{60\degree}$
			\item $\dfrac{\sin^{20}{60\degree}\cdot\tg^{10}{30\degree}}{\sin^{25}{30\degree}\cdot\ctg^5{30\degree}}$
			\item $\dfrac{\sin^3{60\degree}\tg{30\degree}}{8\cos60\degree-2\cos30\degree\cdot\ctg30\degree}$
		\end{enumerate}
	\end{multicols}	
	\item \textit{(1 балл)} Известно, что $\cos{\beta}=0,5$. Верно ли, что $\beta=300\degree$? Объясние почему.
	\item \textit{(2 балла)} Вычислите:
	\begin{multicols}{2}
		\begin{enumerate}[label=\asbuk*)]
			\item $\sqrt{6}\sin120\degree\cos315\degree$
			\item $\sin225\degree\cos120\degree\tg330\degree\ctg240\degree$
			\item $\dfrac{3\sin120\degree+2\cos150\degree}{\tg210\degree+\ctg210\degree}$
			\item $(\sin300\degree)^{-2}+4\tg300\degree\cdot\sin300\degree$
		\end{enumerate}
	\end{multicols}	
	\item \textit{(1 балл)} Укажите наибольшее и наименьшее значение выражения:
	\begin{multicols}{2}
		\begin{enumerate}[label=\asbuk*)]
			\item $1+\sin\alpha$
			\item $2\cos^2{\alpha}-1$
			\item $-|\sin\alpha|$
			\item $|3+4\sin\alpha|$
		\end{enumerate}
	\end{multicols}	
	\item \textit{(1 балл)} Вычислите значение выражения:
	$$\sin(-300\degree)\cdot\cos(-135\degree)\cdot\tg(-210\degree)\cdot\ctg(-120\degree)$$
	\item \textit{(2 балла)} Решить уравнение:
	\begin{enumerate}[label=\asbuk*)]
		\item $x-3\sqrt{x-1}+1=0$
		\item $(x^2-2x+8)\sqrt{x^2-x-6}=0$
		\item $\sqrt{-2x-1}=\sqrt{x^2-36}$
		\item $\sqrt{x-3}+\sqrt{6-x}=\sqrt{3}$
	\end{enumerate}
	\item \textit{(1 балл)} Решить систему уравнений:
	$$
	\left\{
	\begin{aligned}
		x+\dfrac{3}{y}=-1,\\
		\dfrac{x}{y}=-\dfrac{2}{3}
	\end{aligned}
	\right.
	$$
	
	
\end{enumerate}
\end{document}