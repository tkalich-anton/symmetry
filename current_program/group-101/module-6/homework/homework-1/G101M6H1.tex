\documentclass[12pt, a4paper]{article}
\usepackage{../../../../../style}
\begin{document}
	\lhead{Группа 101} \chead{Модуль 6 Урок №1-2} \rhead{Школа <<Симметрия>>}
	\begin{enumerate}
		\item Принадлежит ли точка с координатами $(7,5;2,5)$ уравнению прямой $y=\frac{1}{3}x$?
		\item Найдите координаты точки пересечения пересечения прямых $y=-0,5x-2$ и $y=0,5x+8$
		\item Решить систему уравнений:
		$$\left\{
		\begin{array}{l}
			x-y-7=0,\\
			3x-y+7=6
		\end{array}
		\right.$$ \answer{$(-4;-11)$}
		\item \funcexer
		{Прямые $f(x)=x-5,5$ и $g(x)$ пересекаются в точке с координатами $(a;b)$. Найдите $a+b$. \ranswer{$-10$}}
		{../graphs/graph_6/graph_6}
		
	\end{enumerate}
\end{document}