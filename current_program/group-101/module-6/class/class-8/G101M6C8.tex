\documentclass[12pt, a4paper]{article}
\usepackage{../../../../../style}
\DeclareUnicodeCharacter{202F}{\,}
\begin{document}
	
\lhead{Группа 102} \chead{Модуль 6} \rhead{Школа <<Симметрия>>} \cfoot{}
\begin{center}
	\Large
	\textbf{Занятие №8}
\end{center}
\begin{multicols}{2}
	\begin{enumerate}[label=\textbf{\arabic*.}]
		\item \( 7^x+2\cdot7^{1-x}-9=0 \)
		\item \( \dfrac{3^{x+1}+5}{3^x-1}-\dfrac{3^{x+1}-5}{3^x+1}=6 \)
		\item \( \sin \dfrac{x}{2} = -\dfrac{\sqrt{2}}{2} \)
		\item \( \cos\left(x+\dfrac{\pi}{6}\right)=-\dfrac{\sqrt{3}}{2} \)
		\item \( \sin^2 x = \dfrac{1}{5} \)
		\item \( \sin^2 x + 2\cos x - 2 = 0 \)
		\item \( \cos 2x + \cos x = 0 \)
		\item \( 2\sin^2 x = 3\cos x \)
		\item \( \sin3x\cos x + \sin x\cos 3x = 0 \)
		\item \( \sin x\cos\dfrac{\pi}{3}+\sin\dfrac{\pi}{3}\cos x = 0 \)
		\item \( 2\cos 2x - 5 = 8\sin x \)
		\item \( \dfrac{\sqrt{3}}{2}\sin x - \dfrac{1}{2}\cos x = \dfrac{\sqrt{3}}{2} \)
	\end{enumerate}
\end{multicols}
\end{document}