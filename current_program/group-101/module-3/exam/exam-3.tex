\documentclass[12pt, a4paper]{article}
\usepackage{cmap} % Улучшенный поиск русских слов в полученном pdf-файле
\usepackage[T2A]{fontenc} % Поддержка русских букв
\usepackage[utf8]{inputenc} % Кодировка utf8
\usepackage[english, russian]{babel} % Языки: русский, английский
\usepackage{enumitem}
\usepackage{pscyr} % Нормальные шрифты
\usepackage{soulutf8}
\usepackage{amsmath}
\usepackage{amsthm}
\usepackage{amssymb}
\usepackage{scrextend}
\usepackage{titling}
\usepackage{indentfirst}
\usepackage{cancel}
\usepackage{soulutf8}
\usepackage{wrapfig}
\usepackage{gensymb}
\usepackage[dvipsnames,table,xcdraw]{xcolor}
\usepackage{tikz}
\usepackage{multicol}

%Русские символы в списке
\makeatletter
\AddEnumerateCounter{\asbuk}{\russian@alph}{щ}
\makeatother

%Дублирование знаков при переносе
\newcommand*{\hm}[1]{#1\nobreak\discretionary{}%
	{\hbox{$\mathsurround=0pt #1$}}{}}

\usepackage{graphicx}
\graphicspath{{pic/}}
\DeclareGraphicsExtensions{.pdf,.png,.jpg}

%Изменеие параметров листа
\usepackage[left=15mm,right=15mm,
top=1cm,bottom=2cm,bindingoffset=0cm]{geometry}

%\usepackage{fancyhdr}
%\pagestyle{fancy}

\setlength\parindent{1,5em}
\usepackage{indentfirst}

\begin{document}
\section*{Контрольная работа №3}
\begin{enumerate}
	\item \textit{(2 балл)} Постройте сечение четырехугольной пирамиды $SABCD$ плоскостью, проходящей через вершины $A$ и $S$ и точку $K$, принадлежащую ребру $BC$.
	\item \textit{(2 балла)} Точки $M$ и $N$ — середины рёбер соответственно $AA_1$ и $AB$ треугольной призмы $ABCA_1B_1C_1$. Постройте сечение призмы плоскостью, проходящей через точки $M$, $N$ и $C_1$.
	\item \textit{(3 балла)} Точка $P$ — середина ребра $AD$ параллелепипеда $ABCDA_1B_1C_1D_1$. Постройте сечение параллелепипеда плоскостью, проходящей через точку $P$ параллельно прямым $BD$ и $CB_1$.
	\item \textit{(4 балла)} Основание пирамиды $SABCD$ — параллелограмм $ABCD$ с центром $O$. Точка $M$ лежит на отрезке $SO$, причём $OM :MS = 1 : 3$. Постройте сечение пирамиды плоскостью, проходящей через
	прямую $AM$ параллельно прямой $BD$.
	\item \textit{(4 балла)} Дана правильная шестиугольная призма $ABCDEFA_1B_1C_1D_1E_1F_1$. Боковое ребро $AA_1$ равно стороне основания $ABCDEF$. Найдите углы между прямыми $EA_1$ и $AB$.
	\item \textit{(5 баллов)} Основание шестиугольной пирамиды $SABCDEF$ —правильный шестиугольник $ABCDEF$. Точки $M$ и $N$ —середины рёбер $BC$ и $EF$. Постройте сечение пирамиды плоскостью, проходящей через
	прямую $MN$ параллельно ребру $SA$.
\end{enumerate}
\end{document}