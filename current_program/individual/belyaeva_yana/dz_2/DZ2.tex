\documentclass[12pt, a5paper]{article}
\usepackage{cmap} % Улучшенный поиск русских слов в полученном pdf-файле
\usepackage[T2A]{fontenc} % Поддержка русских букв
\usepackage[utf8]{inputenc} % Кодировка utf8
\usepackage[english, russian]{babel} % Языки: русский, английский
\usepackage{enumitem}
\usepackage{pscyr} % Нормальные шрифты

%Русские символы в списке
\makeatletter
\AddEnumerateCounter{\asbuk}{\russian@alph}{щ}
\makeatother
%Дублирование знаков при переносе
\newcommand*{\hm}[1]{#1\nobreak\discretionary{}%
	{\hbox{$\mathsurround=0pt #1$}}{}}

%Изменеие параметров листа
\usepackage[left=1cm,right=1cm,
top=2cm,bottom=2cm,bindingoffset=0cm]{geometry}

\usepackage{amsmath,amsthm,amssymb,scrextend}
\usepackage{fancyhdr}
\pagestyle{fancy}
\usepackage{multicol}

\setlength\parindent{1,5em}
\usepackage{indentfirst}

\begin{document}
	
	\lhead{Ноябрь}
	\rhead{Школа <<Симметрия>>}
	
	\section*{Домашняя работа}
	\begin{enumerate}
		\item Решите уравнение $$\dfrac{x^2+x-5}{x}+\dfrac{3x}{x^2+x-5}+4=0$$
		\item Вычислите $$2\sqrt{5\sqrt{48}}+3\sqrt{40\sqrt{12}}-2\sqrt{15\sqrt{27}}$$
		\item Вычислите $$\left(\dfrac{15}{\sqrt{6}-1}+\dfrac{4}{2-\sqrt{6}}\right)\cdot(\sqrt{6}+1)$$
		\item Вершины $M$ и $N$ квадрата $KLMN$ лежат на гипотенузе $AB$ прямоугольного треугольника $ABC$ ($N$ между $B$ и $M$), а вершины $K$ и $L$ —
		на катетах $BC$ и $AC$ соответственно. Известно, что $AM =3$ и $BN =7$. Найдите площадь квадрата.
	\end{enumerate}
\end{document}