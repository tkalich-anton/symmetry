\documentclass[12pt, a4paper]{article}
\usepackage{../style}
\begin{document}
	
\lhead{Учебная программа 2021-2022 г.} \rhead{Беляева Яна}
\section*{Февраль}
	\textsc{\textbf{Тема:} Планиметрия. Окружности.}
	\begin{enumerate}[label=\textbf{\arabic*})]
		\item \textbf{15.02.22:} Касательные к окружности.
		\item \textbf{22.02.22:} Касательные к окружности.
	\end{enumerate}
\section*{Март}
	\textsc{\textbf{Тема:} Решение финансовых задач.}
	\begin{enumerate}[label=\textbf{\arabic*})]
		\item \textbf{1.03.22:} Задачи на вклады. Понятие сложного процента.
		\item \textbf{8.03.22:} Задачи на кредиты. Две формы платежей: аннуитетный и дифференцированный.
		\item \textbf{15.03.22:} Задачи на кредиты с дифференцированным платежом.
		\item \textbf{22.03.22:} Задачи на кредиты с аннуитетным платежом.
		\item \textbf{29.03.22:} Задачи на кредиты с смешанным условием.
	\end{enumerate}
\section*{Апрель}
	\textsc{\textbf{Тема:} Стереометрия. Задачи на поиск углов.}
	\begin{enumerate}[label=\textbf{\arabic*})]
		\item \textbf{5.04.22:} Поиск угла между скрещивающимися прямыми; между прямой и плоскостью
		\item \textbf{12.04.22:} Поиск угла между плоскостями.
		\item \textbf{19.04.22:} Расстояние от точки до прямой. Расстояние от точки до плоскости.
		\item \textbf{26.04.22:} Проработка всех тем.
	\end{enumerate}
\section*{Май}
	\textsc{\textbf{Тема:} Уравнения. Неравенства. Задачи с параметрами.}
	\begin{enumerate}[label=\textbf{\arabic*})]
		\item \textbf{3.05.22:} Показательные и логарифмические уравнения.
		\item \textbf{10.05.22:} Показательные и логарифмические уравнения.
		\item \textbf{17.05.22:} Дробно-рациональные неравенства.
		\item \textbf{24.05.22:} Иррациоанльные неравенства.
		\item \textbf{31.05.22:} Показательные неравенства.
	\end{enumerate}
\end{document}