\documentclass[12pt, a5paper]{article}
\usepackage{../../../../style_a5}

\begin{document}
	
\lhead{Февраль}
\rhead{Максутов Кирилл}
\cfoot{}
\section*{Домашняя работа}
\begin{enumerate}
	\item Решить уравнение: \( \dfrac{12}{x^2-9}+\dfrac{x}{x-1}=\dfrac{2}{x+3} \)
	\item Решить уравнение: \( (x-1)\sqrt{x^2-x-6}=6x-6 \)
	\item Решить неравенство: \( |x^2-7x+5|\leqslant 0 \)
	\item Решить неравенство: \( \dfrac{x^3-4x^2-12x}{x^2-4x-12}\leqslant 0 \)
	\item Решить неравенство: \( \sqrt{3x^2-10x+7}>2 \)
	\item Решить систему неравенств:
	$$\left\{
	\begin{array}{l}
		x+3|y| = 2,\vspace{0,2cm}\\
		3x-y=1
	\end{array}
	\right.$$
	\item Высота равнобедренного треугольника, опущенная на боковую сторону, разбивает её на отрезки, равные \( 2 \) и \( 1 \), считая от вершины треугольника. Найдите эту высоту.
	\item Диагонали параллелограмма \( ABCD \) пересекаются в точке \( O \). Периметр параллелограмма равен \( 12 \), а разность периметров треугольников \( BOC \) и \( COD \) равна \( 2 \). Найдите стороны
	параллелограмма.
\end{enumerate}
\end{document}