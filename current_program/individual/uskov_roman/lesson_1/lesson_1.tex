\documentclass[12pt, a4paper]{article}
\usepackage{../../../../style}
\lhead{Усков Роман} \chead{Занятие 1} \cfoot{} \showanswers
\begin{document}
\begin{enumcols}[label=\textbf{\arabic*.}]
	\item Вычислить:
	\begin{enumcols}[itemcolumns=2]
		\item \( 2120054+321988 \)
		\item \( 2010710-254623 \)
	\end{enumcols}
	\item Вычислить:
	\begin{enumcols}[itemcolumns=4]
		\item \( 132\cdot5 \)
		\item \( 9\cdot5007 \)
		\item \( 92\cdot16 \)
		\item \( 663\cdot26 \)
		\item \( 450\cdot212 \)
		\item \( 390\cdot420 \)
		\item \( 5412\cdot212 \)
	\end{enumcols}
	\item Вычислить:
	\begin{enumcols}[itemcolumns=2]
		\item \( 672:42+21\cdot39 \)
		\item \( 65254:79-75563:97 \)
		\item \( 989:43-912:48 \)
		\item \( 37115:65+72675:85 \)
		\item \( 239\cdot324 - 156\cdot315 + 156\cdot315 \)
		\item \( 808:8-909:9+242:2-636:3+5 \)
		\item \( 31905:45+571\cdot33-33\cdot571 \)
	\end{enumcols}
	\item В первый день туристы прошли пешком 18 км, а во второй день они проехали на автобусе в 5 раз больше. В третий день они проплыли на лодке половину от расстояния, которое они проехали на автобусе. Какое расстояние туристы преодолели за три дня?
	\item На овощную базу сначала привезли помидоры на 6 машинах по 120 ящиков в каждой, потом еще на 8 машинах по 140 ящиков в каждой. Сколько всего ящиков помидоров привезли на базу?
	\item Задумали число. К задуманному числу прибавили 15, потом поделили его на 3. После этого результат уменьшили на 1 и затем увеличили в 4 раза. В результате проделанных действий получили 40. Найдите задуманное число.
\end{enumcols}
\end{document}