\documentclass[12pt, a4paper]{article}
\usepackage{../../../../style}
\lhead{Усков Роман} \chead{} \cfoot{} \showanswers
\begin{document}
\title{Домашняя работа №1}
\begin{enumcols}[label=\textbf{\arabic*.}]
	\item Вычислить:
	\begin{enumcols}[itemcolumns=3]
		\item \( 212\cdot57 \)
		\item \( 690\cdot2100 \)
		\item \( 12345\cdot54321 \)
	\end{enumcols}
	\item Вычислить:
	\begin{enumcols}[itemcolumns=1]
		\item \( ((16000:32-1640:82):15\cdot7000-192000):40 \)
		\item \( 960:(2000:(10002-(6085+2926)-966)) \)
		\item \( (31440+1040:(150-2400:(67+53))\cdot20):395+1001 \)
	\end{enumcols}
	\item В первый день туристы прошли пешком 18 км, а во второй день они проехали на автобусе в 5 раз больше. В третий день они проплыли на лодке половину от расстояния, которое они проехали на автобусе. Какое расстояние туристы преодолели за три дня?
	\item На овощную базу сначала привезли помидоры на 6 машинах по 120 ящиков в каждой, потом еще на 8 машинах по 140 ящиков в каждой. Сколько всего ящиков помидоров привезли на базу?
	\item Задумали число. К задуманному числу прибавили 15, потом поделили его на 3. После этого результат уменьшили на 1 и затем увеличили в 4 раза. В результате проделанных действий получили 40. Найдите задуманное число.
\end{enumcols}
\end{document}