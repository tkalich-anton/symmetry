\documentclass[12pt, a4paper]{article}
\usepackage{../../../../style}
\lhead{Курятников Андрей} \chead{} \cfoot{} \showanswers
\begin{document}
	\title{Занятие №2}
	\begin{enumcols}[label=\textbf{\arabic*.}]
		\item Вычислить:
		\begin{enumcols}[itemcolumns=2]
			\item \( \left( 1\dfrac{11}{24}+\dfrac{13}{36} \right)\cdot1,44-\dfrac{8}{15}\cdot0,5625 \)
			\item \( 1,456:\dfrac{7}{25}+\dfrac{5}{16}:0,125+4\dfrac{1}{2}\cdot0,8 \)
			\item \( \dfrac{\left( \dfrac{1}{6}+0,1+\dfrac{1}{15} \right):\left( \dfrac{1}{6}+0,1-\dfrac{1}{15} \right)}{\left( 0,5-\dfrac{1}{3}+0,25-\dfrac{1}{5} \right):\left( 0,25-\dfrac{1}{6} \right)} \)
		\end{enumcols}
		\item Вычислить:
		\begin{enumcols}[itemcolumns=4]
			\item \( \sqrt{25\cdot49} \)
			\item \( \sqrt{49\cdot64\cdot100} \)
			\item \( \sqrt{8\cdot50} \)
			\item \( \sqrt{32\cdot72} \)
			\item \( \sqrt{40\cdot55\cdot22} \)
			\item \( \sqrt{245\cdot27\cdot60} \)
			\item \( \sqrt{242\cdot98} \)
			\item \( \sqrt{25000}\cdot\sqrt{1000} \)
		\end{enumcols}
		\item Вынести множитель из-под знака корня:
		\begin{enumcols}[itemcolumns=5]
			\item \( \sqrt{12} \)
			\item \( \sqrt{20} \)
			\item \( \sqrt{50} \)
			\item \( \sqrt{147} \)
			\item \( \sqrt{972} \)
		\end{enumcols}
		\item Сравнить числа: \( 3\sqrt{2} \) и \( 2\sqrt{3} \)
		\item Расположить в порядке возрастания числа:
		\[ \sqrt{32};\hspace{0.5cm}\sqrt{30};\hspace{0.5cm}3\sqrt{3};\hspace{0.5cm}5\sqrt{2};\hspace{0.5cm}\dfrac{1}{2}\sqrt{72} \]
		\item Упростить выражение: \( (3\sqrt{8}+\sqrt{18}+\sqrt{50}-2\sqrt{72})\cdot\sqrt{2} \)
	\end{enumcols}
	\title{Домашняя работа}
	\begin{enumcols}[label=\textbf{\arabic*.}]
		\item Вычислить:
		\begin{enumcols}[itemcolumns=2]
			\item \( 2\dfrac{3}{4}:\left( 1\dfrac{1}{2}-\dfrac{2}{5} \right)+\left( \dfrac{3}{4}+\dfrac{5}{6} \right):3\dfrac{1}{6} \)
			\item \( \left( \dfrac{2}{15}+1\dfrac{7}{12} \right)\cdot\dfrac{30}{103}-\left( 2:2\dfrac{1}{4} \right)\cdot\dfrac{9}{32} \)
			\item \( \dfrac{\left( \dfrac{7}{2000}+0,0065 \right):0,001}{\left( \dfrac{3}{3125}+0,00004 \right)\cdot\dfrac{1}{0,0001}} \)
		\end{enumcols}
		\item Вычислить:
		\begin{enumcols}[itemcolumns=4]
			\item \( \sqrt{36\cdot100} \)
			\item \( \sqrt{16\cdot25\cdot121} \)
			\item \( \sqrt{18\cdot50} \)
			\item \( \sqrt{20}\cdot\sqrt{45} \)
			\item \( \sqrt{21}\cdot\sqrt{35}\cdot\sqrt{15} \)
			\item \( \sqrt{640}\cdot\sqrt{1000} \)
		\end{enumcols}
		\item Вынести множитель из-под знака корня:
		\begin{enumcols}[itemcolumns=5]
			\item \( \sqrt{18} \)
			\item \( \sqrt{27} \)
			\item \( \sqrt{45} \)
			\item \( \sqrt{396} \)
			\item \( \sqrt{676} \)
		\end{enumcols}
		\item Сравнить числа: \( 3\sqrt{10} \) и \( 4\sqrt{6} \)
		\item Расположить в порядке возрастания числа:
		\[ \sqrt{12};\hspace{0.5cm}3\sqrt{2};\hspace{0.5cm}4;\hspace{0.5cm}0,5\sqrt{40};\hspace{0.5cm}\sqrt{3^2-2^2} \]
		\item Упростить выражение: \( (7\sqrt{2}-5\sqrt{6}-3\sqrt{8}+4\sqrt{20})\cdot3\sqrt{2} \)
	\end{enumcols}
\end{document}