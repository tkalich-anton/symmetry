\documentclass[12pt, a5paper]{article}
\usepackage{cmap} % Улучшенный поиск русских слов в полученном pdf-файле
\usepackage[T2A]{fontenc} % Поддержка русских букв
\usepackage[utf8]{inputenc} % Кодировка utf8
\usepackage[english, russian]{babel} % Языки: русский, английский
\usepackage{enumitem}
\usepackage{pscyr} % Нормальные шрифты

%Русские символы в списке
\makeatletter
\AddEnumerateCounter{\asbuk}{\russian@alph}{щ}
\makeatother
%Дублирование знаков при переносе
\newcommand*{\hm}[1]{#1\nobreak\discretionary{}%
	{\hbox{$\mathsurround=0pt #1$}}{}}

%Изменеие параметров листа
\usepackage[left=1cm,right=1cm,
top=2cm,bottom=2cm,bindingoffset=0cm]{geometry}

\usepackage{amsmath,amsthm,amssymb,scrextend}
\usepackage{fancyhdr}
\pagestyle{fancy}
\usepackage{multicol}

\setlength\parindent{1,5em}
\usepackage{indentfirst}

\begin{document}
	
	\lhead{Сентябрь}
	\rhead{Школа <<Симметрия>>}
	
	\section*{Домашняя работа №1}
	\begin{enumerate}
		\item Постройте график функции:
		\begin{enumerate}[label=\asbuk*)]
			\item $y=\dfrac{|x|}{x-1}$
			\item $y=|x-5|+|x-2|$
			\item $y=\dfrac{|x-2|}{x-2}+\dfrac{|x-3|}{x-3}$
			\item $y=1-\sqrt{x^4+2x^2+1}$\\
			(подсказка $\sqrt{x^2}=|x|$)
			\item $y=|-x^2+6x-8|$
		\end{enumerate}
		\item Решите уравнения:
		\begin{enumerate}[label=\asbuk*)]
			\item $(-x)^{-1}=\dfrac{2}{3}$
			\item $x^2-4|x|-21=0$
			\item $(x-2)^2-8|x-2|+15=0$
			\item $x^3-x^2=2x$
		\end{enumerate}
		\item Упростите выражение:
		$$\dfrac{a^{-n}+b^{-n}}{a^{-2n}-b^{-2n}}:\left(\dfrac{1}{b^{-n}}-\dfrac{1}{a^{-n}}\right)^{-1}$$
		\item Окружность, вписанная в трапецию, делит ее боковую сторону на отрезки 3 и 4. Найдите радиус окружности.
		\item Через середину диагонали $KM$ прямоугольника $KLMN$ перпендикулярно этой диагонали проведена прямая, пересекающая сторону $KL$ и $MN$ в точках $A$ и $B$ соответственно. Известно, что $AB=BM=6$. Найдите большую сторону прямоугольника.
	\end{enumerate}
\end{document}