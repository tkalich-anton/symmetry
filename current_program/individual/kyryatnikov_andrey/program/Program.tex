\documentclass[12pt, a4paper]{article}
\usepackage{../../../../style}
\begin{document}
	
\lhead{Учебный план на Июнь} \chead{Курятников Андрей}
\lfoot{В учебный план могут быть внесены изменения на протяжении всего периода обучения.} \cfoot{}
\renewcommand{\footrulewidth}{1pt}
\section*{Неделя 1}
	\textsc{\textbf{Тема:} Базовая алгебра + арифметика}
	\begin{enumerate}[label=\textbf{\arabic*})]
		\item \textbf{9.06.22:} Арифметические действия с обыкновенными и десятичными дробями. Свойства степеней. Основные методы преобразования выражений, применение ФСУ.
		\item \textbf{10.06.22:} Упрощение дробных выражений. Практика.
	\end{enumerate}
\section*{Неделя 2}
	\textsc{\textbf{Тема:} Преобразование алгебраических выражений}
\begin{enumerate}[label=\textbf{\arabic*})]
	\item \textbf{14.06.22:} Преобразование и упрощение дробных выражений. Свойства корней. Иррациональные выражения.
	\item \textbf{17.06.22:} Практическое занятие.
\end{enumerate}
\section*{Неделя 3}
	\textsc{\textbf{Тема:} Решение целых и дробных уравнений}
\begin{enumerate}[label=\textbf{\arabic*})]
	\item \textbf{21.06.22:} Решение целых уравнений. Введение новой переменной. Случаи, когда левая и правая части -- множители. Уравнения с модулем. Уравнения с числами в знаменателях.
	\item \textbf{24.06.22:} Решение дробных уравнений.
\end{enumerate}
\section*{Неделя 4}
	\textsc{\textbf{Тема:} Решение неравенств. Метод интервалов.}
\begin{enumerate}[label=\textbf{\arabic*})]
	\item \textbf{28.06.22:} Квадратные неравенства. Частные случаи. Неравенства с модулем.
	\item \textbf{1.07.22:} Дробные неравенства. Практика.
\end{enumerate}
\end{document}