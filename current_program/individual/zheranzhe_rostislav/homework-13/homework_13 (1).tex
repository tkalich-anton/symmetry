\documentclass[12pt, a4paper]{article}
\usepackage{../../../../style}
\begin{document}
	
\lhead{Жеранже Ростислав} \cfoot{}
\rhead{Домашняя работа}
\begin{enumerate}
	\item Представь выражение в виде разности квадратов
	\begin{multicols}{2}
		\begin{enumerate}[label=\textbf{\arabic*)}]
			\item $x^4-1$
			\item $16y^2-49x^2$
			\item $4a^2-4$
			\item $9p^4-16q^6$
			\item $m^6-25$
			\item $36m^2-16n^2$
		\end{enumerate}
	\end{multicols}
	\item Запиши выражение в виде многочлена:
	\begin{multicols}{2}
		\begin{enumerate}[label=\textbf{\arabic*)}]
			\item $(q+p)(p^2-pq+q^2)$
			\item $(ab-3)(a^2b^2+3ab+9)$
			\item $(2+x)(4-2x+x^2)$
			\item $(25-5m+m^2)(5+m)$
			\item $\left( 4y^2-xy+\dfrac{1}{4}x^2\right)\left(\dfrac{1}{2}x+2y \right)  $
			\item $(1,21q^2+0,22pq+0,04p^2)(0,2p-1,1q)$
			\item $(2+n^2)(n^4-2n^2+4)$
			\item $(p^3+q^2)(q^4-p^3q^2+p^6)$
			\item $(a^4+1)(a^8-a^4+1)$
		\end{enumerate}
	\end{multicols}
	\item Представь выражение в виде суммы или разности кубов:
	\begin{multicols}{2}
		\begin{enumerate}[label=\textbf{\arabic*)}]
			\item $1+m^6$
			\item $a^9+27b^3$
			\item $\dfrac{1}{8}+x^6y^9$
			\item $27a^3+b^3$
			\item $125x^3-8y^3$
			\item $m^{12}-64$
		\end{enumerate}
	\end{multicols}
\item Повтори все формулы и свойства степеней
\end{enumerate}
\end{document}

