\documentclass[12pt, a4paper]{article}
\usepackage{../../../../style}
\begin{document}
	
\lhead{Жеранже Ростислав} \cfoot{}
\rhead{Домашняя работа}
\begin{enumerate}
	\item Преобразуй выражение в многочлен стандартного вида
	\begin{multicols}{2}
		\begin{enumerate}[label=\textbf{\arabic*)}]
			\item $(a-b)^2$
			\item $(5+p)^2$
			\item $(4-3y)^2$
			\item $(5p+2q)^2$
			\item $(x^3-y)^2$
			\item $(a^3+ab)^2$
			\item $(3p^2-2q)^2$
			\item $(4a^2b+3ab^2)^2$
		\end{enumerate}
	\end{multicols}
	\item Преобразуй выражение в многочлен:
	\begin{enumerate}[label=\textbf{\arabic*)}]
		\item $\left( \dfrac{1}{2}+a\right) ^2$
		\item $\left(\dfrac{3}{4}x-\dfrac{1}{5}y \right)^2 $
		\item $\left(0,2m+2,1n \right)^2 $
		\item $\left(0,4p-0,3q \right)^2 $
	\end{enumerate}
	\item Представь многочлен в виде квадрата суммы или разности:
	\begin{multicols}{2}
		\begin{enumerate}[label=\textbf{\arabic*)}]
			\item $4x^2+12xy^2+9y^4$
			\item $9x^2-6x+1$
			\item $0,16a^4+0,8a^2b^2+b^4$
			\item $\dfrac{25}{36}p^2+\dfrac{5}{3}p+1$
		\end{enumerate}
	\end{multicols}
\end{enumerate}
\end{document}