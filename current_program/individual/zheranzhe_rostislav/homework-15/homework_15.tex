% !TeX spellcheck = ru_RU-Russian
\documentclass[12pt, a4paper]{article}
\usepackage{../../../../style}
\begin{document}
	
\lhead{Жеранже Ростислав} \cfoot{}
\rhead{Домашняя работа}
\begin{enumerate}
	\item Упрости выражение:
	\begin{multicols}{2}
		\begin{enumerate}[label=\textbf{\arabic*)}]
			\item $(p+6)^2-4(3-p)(3+p)$
			\item $-(2+m)^2+2(1+m)^2-2(1-m)(m+1)$
			\item $(x+y)^2-(x-y)^2$
			\item $(4x+3y)^2-(3x-4y)^2$
			\item $(2x-4y)^2-(5x+y)^2$
			\item $(x^2-2y)^2-y^4$
			\item $(4x^2+3y)^2-9y^4$
			\item $(3+m)(m^2-3m+9)-m(m-2)$
			\item $(p^6-q^3)(p^6+q^3)-(p^8-p^4p^2+q^4)(p^4+q^2)$
		\end{enumerate}
	\end{multicols}
	\item Запиши выражение в виде многочлена:
	\begin{enumcols}[columns=2, label=\textbf{\arabic*)}]
			\item $(x+2)(b^4-x^2)(x-2)$
			\item $(a-b)(a-b)(a+b)(a+b)$
			\item $(p+q)^2(p-q)^2$
			\item $(5+m)(m-5)(m-5)(m+5)$
			\item $3(2-3m)^2-3(2-3m)(3m+2)$
			\item $2(1-5x)^2-2(5x+1)(1-5x)$
	\end{enumcols}
\end{enumerate}
\end{document}

