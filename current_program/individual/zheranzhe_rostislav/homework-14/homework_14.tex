% !TeX spellcheck = ru_RU-Russian
\documentclass[12pt, a4paper]{article}
\usepackage{../../../../style}
\begin{document}
	
\lhead{Жеранже Ростислав} \cfoot{}
\rhead{Домашняя работа}
\begin{enumerate}
	\item Приведи к общему знаменателю
	\begin{multicols}{2}
		\begin{enumerate}[label=\textbf{\arabic*)}]
			\item $\dfrac{x}{5}$ и $\dfrac{-3}{7}$
			\item $\dfrac{5}{3x}$ и $\dfrac{7}{6}$
			\item $\dfrac{1}{5x}$ и $\dfrac{13}{-10x}$
			\item $\dfrac{4x}{x-1}$ и $\dfrac{2-7x}{1-x}$
			\item $\dfrac{x}{5+x}$ и $\dfrac{3}{x+5}$
			\item $\dfrac{2x}{3x+6}$ и $\dfrac{5}{x+2}$
			\item $\dfrac{1}{2+x}$ и $\dfrac{x-1}{x^2-4}$
			\item $\dfrac{5x}{3-x}$ и $\dfrac{2}{x^2-9}$
		\end{enumerate}
	\end{multicols}
	\item Запиши выражение в виде многочлена:
	\begin{multicols}{2}
		\begin{enumerate}[label=\textbf{\arabic*)}]
			\item $(a+b)^3$
			\item $(x+3z)^3$
			\item $(x-1)^3$
			\item $(a-2b)^3$
		\end{enumerate}
	\end{multicols}
	\item Вычисли: \\\\
	$\dfrac{\left( 6\dfrac{7}{12}-3\dfrac{17}{36}\right)\cdot2,5-4\dfrac{1}{3}:0,65 }{4:\dfrac{1}{4}-0,5}$
\end{enumerate}
\end{document}

