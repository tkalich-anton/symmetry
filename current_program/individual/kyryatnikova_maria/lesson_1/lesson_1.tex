\documentclass[12pt, a4paper]{article}
\usepackage{cmap} % Улучшенный поиск русских слов в полученном pdf-файле
\usepackage[T2A]{fontenc} % Поддержка русских букв
\usepackage[utf8]{inputenc} % Кодировка utf8
\usepackage[english, russian]{babel} % Языки: русский, английский
\usepackage{enumitem}
\usepackage{pscyr} % Нормальные шрифты

%Русские символы в списке
\makeatletter
\AddEnumerateCounter{\asbuk}{\russian@alph}{щ}
\makeatother
%Дублирование знаков при переносе
\newcommand*{\hm}[1]{#1\nobreak\discretionary{}%
	{\hbox{$\mathsurround=0pt #1$}}{}}

%Изменеие параметров листа
\usepackage[left=1cm,right=1cm,
top=2cm,bottom=2cm,bindingoffset=0cm]{geometry}

\usepackage{amsmath,amsthm,amssymb,scrextend}
\usepackage{fancyhdr}
\pagestyle{fancy}
\usepackage{multicol}

\setlength\parindent{1,5em}
\usepackage{indentfirst}

\begin{document}
	
	\lhead{Курятникова Мария}
	\rhead{Школа <<Симметрия>>}
	\begin{enumerate}
		\item Вычислить:
		\[ \dfrac{(0,73^3-0,73\cdot0,27^2):0,023+2,4}{(18,544:3,05-1,83)\cdot0,16} \]
		\item Упростить выражение:
		\begin{enumerate}
			\item \( (x-y)(x^2+xy+y^2) \)
			\item \( (2x-3)^2-(x-1)^2-(3x^2-10x+-12) \)
			\item \( (x-1)(x^2+x+1)-(x+1)(1-x+x^2) \)
		\end{enumerate}
	\end{enumerate}
\end{document}

