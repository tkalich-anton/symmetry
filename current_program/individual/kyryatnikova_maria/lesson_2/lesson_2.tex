\documentclass[12pt, a4paper]{article}
\usepackage{../../../../style}
\lhead{Курятникова Мария} \chead{} \cfoot{} \showanswers
\begin{document}
	\title{Занятие №2}
	\begin{enumcols}[label=\textbf{\arabic*.}]
		\item Вычислить:
		\begin{enumcols}[itemcolumns=2]
			\item \( \left( 1\dfrac{11}{24}+\dfrac{13}{36} \right)\cdot1,44-\dfrac{8}{15}\cdot0,5625 \)
			\item \( 1,456:\dfrac{7}{25}+\dfrac{5}{16}:0,125+4\dfrac{1}{2}\cdot0,8 \)
			\item \( \dfrac{\left( \dfrac{1}{6}+0,1+\dfrac{1}{15} \right):\left( \dfrac{1}{6}+0,1-\dfrac{1}{15} \right)}{\left( 0,5-\dfrac{1}{3}+0,25-\dfrac{1}{5} \right):\left( 0,25-\dfrac{1}{6} \right)} \)
		\end{enumcols}
		\item Выполнить умножение:
		\begin{enumcols}[itemcolumns=3]
			\item \( (a+1)(a^2+1) \)
			\item \( (5m+7n)(2n+4m) \)
			\item \( (x+1)(x^2-x+1) \)
			\item \( (xy^2+3a^2)(3xy+x^3) \)
			\item \( -(x-2y)(2x+y) \)
			\item \( \left( \dfrac{1}{3}-x \right)\left( \dfrac{1}{2}x-3 \right) \)
		\end{enumcols}
		\item Выполнить умножение и, при необходимости, привести подобные слагаемые:
		\begin{enumcols}[itemcolumns=2]
			\item \( (x^2+3x-2)(2x^2-x+4) \)
			\item \( (x-3)(2x-1)(7+2x) \)
		\end{enumcols}
		\item Упростить целое выражение:
		\begin{enumcols}[itemcolumns=2]
			\item \( (5ab^2+4b^3)(3ab^3-4a^2)-18a^2b^3 \)
			\item \( 2-(-4x+1)(x-1)+2(6x-4)(x+3) \)
		\end{enumcols}
	\end{enumcols}
	\title{Домашняя работа}
	\begin{enumcols}[label=\textbf{\arabic*.}]
		\item Вычислить:
		\begin{enumcols}[itemcolumns=2]
			\item \( \left( \dfrac{1}{2}+0,8-\dfrac{3}{5}\right)\cdot\left( 3+5\dfrac{8}{25}-0,12 \right) \)
			\item \( \left( 2,314-\dfrac{1}{4} \right):\dfrac{1}{50}+\left( 1\dfrac{11}{16}+0,7125 \right):3 \)
			\item \( \dfrac{\dfrac{3}{4}\cdot\left( 4,4-3,75+8\dfrac{7}{15}+8\dfrac{7}{60} \right)}{\left( 3\dfrac{1}{2}-2,75 \right):0,2} \)
		\end{enumcols}
		\item Выполнить умножение:
		\begin{enumcols}[itemcolumns=3]
			\item \( (x-2)(x^2+3) \)
			\item \( (3x+2y)(2x-3y) \)
			\item \( (2x^2-y^2)(y^2+2x^3) \)
			\item \( -(x-y)(x+y) \)
			\item \( \left( 1\dfrac{1}{2}x-y \right)\left( 2\dfrac{1}{3}y-\dfrac{1}{3}x \right) \)
		\end{enumcols}
		\item Выполнить умножение и, при необходимости, привести подобные слагаемые:
		\begin{enumcols}[itemcolumns=2]
			\item \( (x^2+5x-1)(3x^2-2x+1) \)
			\item \( (2x-1)(3x+2)(5+3x) \)
			\item \( \left( \dfrac{1}{2}x+12 \right)\left( 3\dfrac{1}{2}-\dfrac{1}{4}x \right) \)
		\end{enumcols}
		\item Упростить целое выражение:
		\begin{enumcols}[itemcolumns=2]
			\item \( (7x^3y^2-xy)(-2x^2y^2+5xy^3)+12x^5y^4 \)
			\item \( a^2(a^2-b^2)-(a^3-a^2b+ab^2-b^3)(a+b) \)
		\end{enumcols}
	\end{enumcols}
\end{document}