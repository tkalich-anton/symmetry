	\documentclass[12pt, a4paper]{article}
\usepackage{../../style}
\lhead{Для кураторов} \cfoot{}
\begin{document}
	\section*{Список тем по классам}
		\subsection*{5 класс}
			\begin{enumerate}[label=\textbf{\arabic*}.]
				\item Работа с натуральными числами. Решение столбиком. Сравнение натуральных чисел.
				\item Степень с натуральным показателем.
				\item Задачи на нахождение двух чисел по их сумме и разности.
				\item Представление натуральных чисел на координатном луче.
				\item Прямая. Луч. Отрезок. Измерение отрезков.
				\item Единицы длины и площади.
				\item Площадь прямоугольника.
				\item Углы и измерение углов.
				\item Треугольник. Окружность и круг. Сфера и шар. Прямоугольный параллелепипед.
				\item Единицы объема. Объем прямоугольного параллелепипеда.
				\item Единицы массы. Единицы времени.
				\item Задачи на движение.
				\item Обыкновенные дроби. Понятие дробей. Равенство дробей. Сравнение дробей. Законы сложения и вычитания дробей с одинаковыми знаменателями.
				\item Умножение и деление дробей. Нахождение части целого и целого по его части.
				\item Понятие смешанного числа. Сложение, вычитание, умножение и деление смешанных чисел.
				\item Представление дробей на координатном луче.
			\end{enumerate}
		\subsection*{6 класс}
			\begin{enumerate}[label=\textbf{\arabic*}.]
				\item Отношение чисел. Деление числа в данном отношении. Пропорции. Понятие прямой и обратной пропорциональности.
				\item Понятие о проценте. Нахождение процента от числа и числа по процентному значению. Задачи на проценты.
				\item Понятие вероятности события.
				\item Отрицательные числа. Противоположные числа. Модуль числа.
				\item Сравнение целых чисел. Законы сложения и вычитания целых чисел.
				\item Произведение и деление целых чисел.
				\item Раскрытие скобок и заключение в скобки.
				\item Представление целых чисел на координатной оси.
				\item Рациональные числа. Сравнение рациональных чисел.
				\item Уравнения. Решение задач с помощью уравнений.
				\item Буквенные выражения. Значение буквенного выражения.
			\end{enumerate}
		\subsection*{7 класс}
		\subsubsection*{Алгебра}
			\begin{enumerate}[label=\textbf{\arabic*}.]
				\item Понятие буквенного (алгебраического) и числового (арифметического) выражения.
				\item Числовое значение целого выражения.
				\item Одночлены. Понятие одночленов. Стандартный вид. Подобные одночлены.
				\item Многочлены. Стандартный вид. Сумма и разность многочленов. Произведение многочлена и одночлена. Произведение многочленов.
				\item Понятие тождества. Тождественное равенство целых выражений.
				\item Формулы сокращенного умножения. Квадрат суммы и квадрат разности. Выделение полного квадрата. Разность квадратов. Сумма и разность кубов. Куб суммы и куб разности. Разложение многочлена на множители.
				\item Понятие степени с целым показателем. Свойства степени с целым показателем.
				\item Линейные уравнения с одним неизвестным. Решение задач с помощью линейных уравнений.
				\item Системы линейных уравнений. Способ подстановки. Способ сложения. Решение задач при помощи систем линейных уравнений.
			\end{enumerate}
		\subsubsection*{Геометрия}
			\begin{enumerate}[label=\textbf{\arabic*}.]
				\item Виды углов. Острый, тупой, прямой, развернутый. Смежные углы. Вертикальные углы.
				\item Виды треугольников. Остроугольный, тупоугольный, прямоугольный, равнобедренный, равносторонний.
				\item Признаки равенства треугольников.
				\item Признаки равенства прямоугольных треугольников.
				\item Углы, образованные при пересечении параллельных прямых третьей прямой.
				\item Биссектриса, медиана, высота, серединный перпендикуляр.
				\item Понятие окружности. Хорда, касательная, радиус, диаметр, секущая.
				\item Теорема об угле в \( 30\degree \) в прямоугольном треугольнике.
				\item Теорема о сумме углов в треугольнике.
				\item Понятие геометрического места точек. Биссектриса, окружность, серединный перпендикуляр как ГМТ.
			\end{enumerate}
		\subsection*{8 класс}
			\subsubsection*{Алгебра}
			\begin{enumerate}[label=\textbf{\arabic*}.]
				\item Понятие квадратного корня. Арифметический квадратный корень. Алгебраический корень.
				\item Квадратные уравнение. Понятие полного, неполного и приведенного квадратного уравнения. Применение формул дискриминанта и теоремы Виета.
				\item Решение задач с помощью квадратных уравнений.
				\item Рациональные уравнения. Уравнения высших степеней. Биквадратные уравнения. Распадающиеся уравнения.
				\item Уравнения, одна часть которого алгебраическая дробь, а другая --- нуль.
				\item Решение задач с помощью рациональных уравнений.
				\item Понятие равносильного уравнения и уравнения-следствия.
				\item Функции. Линейная функция. Квадратичная функция. Дробно-линейная функция. Функции, содержащие модуль.
				\item Системы рациональных уравнений.
				\item Понятие числового промежутка.
				\item Решение линейных неравенств.
			\end{enumerate}
			\subsubsection*{Геометрия}
			\begin{enumerate}[label=\textbf{\arabic*}.]
				\item Четырехугольники. Свойства параллелограмма.
				\item Прямоугольник, ромб и квадрат. Их свойства.
				\item Трапеция. Теорема Фалеса.
				\item Понятие вписанной и описанной окружности. Вневписанная окружность. Положение центров этих окружностей.
				\item Касательные к окружности. Теорема о касательных.
				\item Подобие фигур. Признаки подобия треугольников.
				\item Понятие \( \sin x \), \( \cos x \), \( \tg x \), \( \ctg x \), как отношение сторон в прямоугольном треугольнике.
				\item Площадь треугольника. Площадь четырехугольника.
			\end{enumerate}
		\subsection*{9 класс}
			\subsubsection*{Алгебра}
			\begin{enumerate}[label=\textbf{\arabic*}.]
				\item Разложение квадратного трехчлена на множители.
				\item Решение квадратных неравенств.
				\item Рациональные неравенства. Метод интервалов.
				\item Корень степени \( n \). Корни четной и нечетной степени. Свойства корней степени \( n \).
				\item Иррациональные уравнения. Корень равно число. Корень равно корень.
				\item Последовательности. Арифметическая прогрессия. Геометрическая прогрессия.
			\end{enumerate}
			\subsubsection*{Геометрия}
			\begin{enumerate}[label=\textbf{\arabic*}.]
				\item Четырехугольники. Свойства параллелограмма.
				\item Прямоугольник, ромб и квадрат. Их свойства.
				\item Трапеция. Теорема Фалеса.
				\item Понятие вписанной и описанной окружности. Вневписанная окружность. Положение центров этих окружностей.
				\item Касательные к окружности. Теорема о касательных.
				\item Подобие фигур. Признаки подобия треугольников.
				\item Понятие \( \sin x \), \( \cos x \), \( \tg x \), \( \ctg x \), как отношение сторон в прямоугольном треугольнике.
				\item Площадь треугольника. Площадь четырехугольника.
			\end{enumerate}
		\subsection*{10 класс}
		\subsection*{11 класс}
\end{document}