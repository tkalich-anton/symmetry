\documentclass[12pt, a4paper]{article}
\usepackage{../../../../../style}
\begin{document}
	
	\lhead{Группа 71}
	\chead{Модуль 6 Урок №5-6}
	\rhead{Школа <<Симметрия>>}
	\begin{enumerate}
		\item Через точку $M$ проведены две касательные $MA$ и $MB$ к окружности ($A$ и $B$ — точки касания). Докажите,
		что $MA = MB$.
		\item Точки $A$ и $B$ лежат на окружности. Касательные к окружности, проведенные через эти точки, пересекаются в точке $C$. Найдите углы треугольника $ABC$, если $AB = AC$.
		\item Расстояние от точки $M$ до центра $O$ окружности равно диаметру. Через точку $M$ проведены две прямые, касающиеся окружности в точках $A$ и $B$. Найдите углы треугольника $AOB$.
		\item Хорда большей из двух концентрических окружностей касается меньшей. Докажите, что точка касания делит эту хорду пополам.
		\item Докажите, что центр окружности, вписанной в угол, расположен на его биссектрисе.
		\item Две прямые касаются окружности с центром $O$ в точках $A$ и $B$ и пересекаются в точке $C$. Найдите угол между этими
		прямыми, если $\angle ABO = 40 \degree$.
		\item Две прямые, пересекающиеся в точке $C$, касаются окружности в точках $A$ и $B$. Известно, что $\angle ACB = 120 \degree$. Докажите, что сумма отрезков $AC$ и $BC$ равна отрезку $OC$.
		\item Окружность касается двух параллельных прямых и
		их секущей. Докажите, что отрезок секущей, заключенный между параллельными прямыми, виден из центра окружности под
		прямым углом.
		\item Точка $D$ лежит на стороне $BC$ треугольника $ABC$. В треугольник $ABD$ и $ACD$ вписаны окружности с центрами $O_1$ и $O_2$. Докажите, что отрезок $O_1O_2$ виден из точки $D$ под прямым углом.
		\item Центр окружности, описанной около треугольника,	совпадает с центром вписанной окружности. Найдите углы треугольника.
	\end{enumerate}
\end{document}