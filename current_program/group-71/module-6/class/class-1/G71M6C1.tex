\documentclass[12pt, a4paper]{article}
\usepackage{../../../../../style}
\begin{document}
	
\lhead{Группа 71} \chead{Модуль 6 Урок №1-2} \rhead{Школа <<Симметрия>>}
\section{Повторение}
\begin{enumerate}
	\item Угол треугольника равен сумме двух других его углов. Докажите, что треугольник прямоугольный.
	\item Медиана треугольника равна половине стороны, к которой она проведена. Докажите, что треугольник прямоугольный.
	\item Докажите, что внешний угол треугольника равен сумме двух внутренних углов, не смежных с ним.
	\item Перечислите свойства равнобедренного треугольника.
\end{enumerate}
\section{Задачи по теме урока}
\begin{enumerate}
	\item Докажите следующие свойства окружности:
		\begin{enumerate}[label=\asbuk*)]
		\item  диаметр, перпендикулярный хорде, делит ее пополам;
		\item  диаметр, проходящий через середину хорды, не являющейся диаметром, перпендикулярен этой хорде;
		\item окружность симметрична относительно каждого своего
		диаметра;
		\item  дуги окружности, заключенные между параллельными
		хордами, равны;
		\item  расстояния от центра окружности до равных хорд равны.
		\end{enumerate}
	\item Через точку на окружности проведены диаметр и хорда, равная радиусу. Найдите угол между ними.
	\item Через точку $A$ окружности с центром $O$ проведены диаметр $AB$ и хорда $AC$. Докажите, что угол $BAC$ вдвое меньше угла $BOC$.
	\item Найдите угол между радиусами $OA$ и $OB$, если расстояние от центра $O$ окружности до хорды $AB$ вдвое меньше $AB$.
	\item Даны две концентрические окружности и пересекающая их прямая. Докажите, что отрезки этой прямой, заключенные между окружностями, равны.
	\item Две хорды окружности взаимно перпендикулярны. Докажите, что расстояние от точки их пересечения до центра окружности равно расстоянию между их серединами.
	\item Биссектриса внутреннего угла при вершине $A$ и биссектриса внешнего угла при вершине $C$ треугольника $ABC$ пересекаются в точке $M$. Найдите $\angle BMC$, если $\angle BAC = 40\degree$.
\end{enumerate}
\end{document}