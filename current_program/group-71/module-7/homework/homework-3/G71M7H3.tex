\documentclass[12pt, a4paper]{article}
\usepackage{cmap} % Улучшенный поиск русских слов в полученном pdf-файле
\usepackage[T2A]{fontenc} % Поддержка русских букв
\usepackage[utf8]{inputenc} % Кодировка utf8
\usepackage[english, russian]{babel} % Языки: русский, английский
\usepackage{enumitem}
\usepackage{pscyr} % Нормальные шрифты
\usepackage{amsmath}
\usepackage{amsthm}
\usepackage{amssymb}
\usepackage{scrextend}
\usepackage{titling}
\usepackage{indentfirst}
\usepackage{cancel}
\usepackage{soulutf8}
\usepackage{wrapfig}
\usepackage{gensymb}
\usepackage[dvipsnames,table,xcdraw]{xcolor}
\usepackage{tikz}

%Русские символы в списке
\makeatletter
\AddEnumerateCounter{\asbuk}{\russian@alph}{щ}
\makeatother

%Дублирование знаков при переносе
\newcommand*{\hm}[1]{#1\nobreak\discretionary{}%
	{\hbox{$\mathsurround=0pt #1$}}{}}

\usepackage{graphicx}
\graphicspath{{pic/}}
\DeclareGraphicsExtensions{.pdf,.png,.jpg}

%Изменеие параметров листа
\usepackage[left=15mm,right=15mm,
top=2cm,bottom=2cm,bindingoffset=0cm]{geometry}

\usepackage{fancyhdr}
\pagestyle{fancy}
\usepackage{multicol}

\setlength\parindent{1,5em}
\usepackage{indentfirst}
\begin{document}
	
	\lhead{Группа 71}
	\chead{Модуль 7 Домашняя работа №3}
	\rhead{Школа <<Симметрия>>}
	\section*{Домашняя работа №3}
	\begin{enumerate}
		\item \textit{(4 балла)} Решите системы уравнений:
			\begin{enumerate}[label=\asbuk*)]
		\item 
		$\left\{
		\begin{array}{l}
			x-y-2=-1,\\
			x+y-5=0
		\end{array}
		\right.$
		\item 
		$\left\{
		\begin{array}{l}
			y-3x=0,\\
			x-2y=-10
		\end{array}
		\right.$
		\item 
		$\left\{
		\begin{array}{l}
			2x+y-1=0,\\
			3x+2y+5=0
		\end{array}
		\right.$
		\item 
		$\left\{
		\begin{array}{l}
			x+2y-3=0,\\
			x+y=-1
		\end{array}
		\right.$ 
	\end{enumerate}
		\item \textit{(2 балла)} Сократите дроби:
		\begin{multicols}{2}
			\begin{enumerate}[label=\asbuk*)]
		\item $\dfrac{24a^5b^7c}{44a^7b^4c}$ 
		\item $\dfrac{ab(a+3)}{a^2b(a+3)}$ 
		\item $\dfrac{15a(a-b)}{20b(a-b)}$ 
		\item $\dfrac{2(x+y)}{4ax}$ 
		\item $\dfrac{a+b}{a+b}$ 
		\item $\dfrac{2(x-1)}{5(x-1)}$ 
		\item $\dfrac{3a(a-b)}{6a(a+b)}$ 
		\item $\dfrac{4x(x-y^3)}{16x^2y(x-y)}$ 
		\end{enumerate}
		\end{multicols}
		\item \textit{(2 балла)} Найдите решение уравнения $-12x+17y=-87$, состоящее из двух противоположных чисел.
		\item \textit{(2 балла)} Из пунктов $A$ и $B$, расстояние между которыми равно $24$ км, одновременно навстречу друг другу вышли два туриста. Через $2$ ч после начала движения они еще не встретились, а расстояние между ними составляло $6$ км. Еще через $2$ ч одному из них оставалось пройти до пункта $B$ на $4$ км меньше, чем другому до пункта $A$. Найдите скорость каждого туриста.
	
	\end{enumerate}
\end{document}