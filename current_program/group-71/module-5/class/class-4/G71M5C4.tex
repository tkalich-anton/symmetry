\documentclass[12pt, a4paper]{article}
\usepackage{cmap} % Улучшенный поиск русских слов в полученном pdf-файле
\usepackage[T2A]{fontenc} % Поддержка русских букв
\usepackage[utf8]{inputenc} % Кодировка utf8
\usepackage[english, russian]{babel} % Языки: русский, английский
\usepackage{enumitem}
\usepackage{pscyr} % Нормальные шрифты
\usepackage{soulutf8}
\usepackage{amsmath}
\usepackage{amsthm}
\usepackage{amssymb}
\usepackage{scrextend}
\usepackage{titling}
\usepackage{indentfirst}
\usepackage{cancel}
\usepackage{soulutf8}
\usepackage{wrapfig}
\usepackage{gensymb}
\usepackage[dvipsnames,table,xcdraw]{xcolor}
\usepackage{tikz}

%Русские символы в списке
\makeatletter
\AddEnumerateCounter{\asbuk}{\russian@alph}{щ}
\makeatother

%Дублирование знаков при переносе
\newcommand*{\hm}[1]{#1\nobreak\discretionary{}%
	{\hbox{$\mathsurround=0pt #1$}}{}}

\usepackage{graphicx}
\graphicspath{{pic/}}
\DeclareGraphicsExtensions{.pdf,.png,.jpg}

%Изменеие параметров листа
\usepackage[left=15mm,right=15mm,
top=2cm,bottom=2cm,bindingoffset=0cm]{geometry}


\usepackage{fancyhdr}
\pagestyle{fancy}
\usepackage{multicol}

\setlength\parindent{1,5em}
\usepackage{indentfirst}

\begin{document}
	
\lhead{Группа 101}
\chead{Модуль 5 Урок 2}
\rhead{Школа <<Симметрия>>}

\begin{enumerate}
	\item Упростить выражение:
	\begin{multicols}{2}
		\begin{enumerate}[label=\asbuk*)]
			\item $a(a-b)+b(a+b)+(a-b)(a+b)$
			\item $(m-n)(m+n)-(m-n)^2+2n^2$
			\item $(x-y)^2-(x-y)(y-x)+2xy$
			\item $(2x+5y)(5y-2x)-3(x+2y)(x-2y)$
			\item $(a+6)^2-4(3-a)(3+a)$
			\item $-(2+x)^2+2(1+x)^2-2(1-x)(x+1)$
			\item $(2x+y)^2-(2x-y)^2$
		\end{enumerate}
	\end{multicols}
	\item Разложить на множители выражение:
	\begin{multicols}{2}
		\begin{enumerate}[label=\asbuk*)]
			\item $(3a+2)^2-a^2$
			\item $(4x+1)^2-(x+3)^2$
			\item $(2x^2-b)^2-x^4$
			\item $(x^2-2y)^2-y^4$
			\item $(4a+3b)^2-(3a-4b)^2$
		\end{enumerate}
	\end{multicols}
	\item Разложить двучлен на множители:
	\begin{multicols}{4}
		\begin{enumerate}[label=\asbuk*)]
			\item $x^3+y^3$
			\item $a^3+8$
			\item $x^6+63b^3$
			\item $a^9-27b^3$
			\item $x^6y^9-\dfrac{1}{8}$
			\item $m^6+n^{15}$
			\item $64a^3+1000b^3$
			\item $125x^6-8y^9$
		\end{enumerate}
	\end{multicols}
	\item Запишите в виде многочлена:
	\begin{multicols}{2}
		\begin{enumerate}[label=\asbuk*)]
			\item $(5-a)(a^2+5a+25)$
			\item $(2m+5n)(4m^2-10mn+25n^2)$
			\item $\left(\dfrac{1}{2}x-\dfrac{1}{3}y\right)\left(\dfrac{1}{4}x^2+\dfrac{1}{6}xy+\dfrac{1}{9}y^2\right)$
			\item $(0,1x+0,2y)(0,04x^2+0,02xy+0,01y^2)$
		\end{enumerate}
	\end{multicols}
	\item Упростить выражение:
	\begin{multicols}{2}
		\begin{enumerate}[label=\asbuk*)]
			\item $(x+1)(x^2-x+1)$
			\item $(a^3-b^3)(a^3+b^3)+(a^2+b^2)(a^4-a^2b^2+b^4)$
			\item $(x-1)(x^2+x+1)-(1+x)(1-x+x^2)$
			\item $(3+m)(m^2-3m+9)-m(m-2)^2$
			\item $n^5(2+n^2)(n^2-2)-(m-n^3)(m^2+mn^3+n^6)$
		\end{enumerate}
	\end{multicols}
\end{enumerate}
	
\end{document}