\documentclass[12pt, a4paper]{article}
\usepackage{cmap} % Улучшенный поиск русских слов в полученном pdf-файле
\usepackage[T2A]{fontenc} % Поддержка русских букв
\usepackage[utf8]{inputenc} % Кодировка utf8
\usepackage[english, russian]{babel} % Языки: русский, английский
\usepackage{enumitem}
\usepackage{pscyr} % Нормальные шрифты
\usepackage{amsmath}
\usepackage{amsthm}
\usepackage{amssymb}
\usepackage{scrextend}
\usepackage{titling}
\usepackage{indentfirst}
\usepackage{cancel}
\usepackage{soulutf8}
\usepackage{wrapfig}
\usepackage{gensymb}
\usepackage[dvipsnames,table,xcdraw]{xcolor}
\usepackage{tikz}

%Русские символы в списке
\makeatletter
\AddEnumerateCounter{\asbuk}{\russian@alph}{щ}
\makeatother

%Дублирование знаков при переносе
\newcommand*{\hm}[1]{#1\nobreak\discretionary{}%
	{\hbox{$\mathsurround=0pt #1$}}{}}

\usepackage{graphicx}
\graphicspath{{pic/}}
\DeclareGraphicsExtensions{.pdf,.png,.jpg}

%Изменеие параметров листа
\usepackage[left=15mm,right=15mm,
top=2cm,bottom=2cm,bindingoffset=0cm]{geometry}

\usepackage{fancyhdr}
\pagestyle{fancy}
\usepackage{multicol}

\setlength\parindent{1,5em}
\usepackage{indentfirst}
\begin{document}
	
	\lhead{Группа 71}
	\chead{Модуль 5 Домашняя работа №2}
	\rhead{Школа <<Симметрия>>}
	\section*{Домашняя работа №2}
	\begin{enumerate}
		\item \textit{(2 балла)} Преобразуйте выражение в многочлен:
		\begin{multicols}{2}
			\begin{enumerate}[label=\asbuk*)]
			\item $(2x+3)^2$
			\item $(10c+0,1y)^2$
			\item $(\dfrac{1}{3}m-3y)^2$
			\item $(1,2a-3,1c)^2$
		\end{enumerate}
	\end{multicols}
		\item \textit{(3 балла)} Разложите на множители многочлены:
		\begin{multicols}{2}
		\begin{enumerate}[label=\asbuk*)]
			\item $27+u^9$
			\item $\dfrac{1}{27}x^3+\dfrac{1}{125}y^3$
			\item $1+27y^3$
			\item $8-v^3$
			\item $1-\dfrac{1}{8}x^3$
			\item $x^9-y^9$
		\end{enumerate}
	\end{multicols}
	\item \textit{(2 балла)} Впишите вместо * одночлен так, чтобы получилось тождество:
			\begin{enumerate}[label=\asbuk*)]
			\item $(2a+*)(2a-*)=4a^2-b^2$
			\item $(*-3x)(3x+*)=16y^2-9x^2$
			\end{enumerate}
		\item \textit{(2 балла)} Запишите выражение в виде многочлена:
		\begin{multicols}{2}
			\begin{enumerate}[label=\asbuk*)]
				\item $(x+y)^3$
				\item $(2z+4)^3$
				\item $(a-\dfrac{1}{2})^3$
				\item $(x-0,2b)^3$
			\end{enumerate}
		\end{multicols}
	\item \textit{(1 балл)} Упростите выражение двумя способами :
	$(x-1)^3-(x+1)^3$
	
	\textit{Подсказка:} \begin{enumerate}[label=\asbuk*)]
	 \item  Применить формулу разности кубов;
	\item Раскрыть отдельно куб разности и куб суммы, упростить
\end{enumerate}
	\end{enumerate}
\end{document}