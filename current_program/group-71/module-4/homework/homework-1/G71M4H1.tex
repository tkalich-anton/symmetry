\documentclass[12pt, a4paper]{article}
\usepackage{cmap} % Улучшенный поиск русских слов в полученном pdf-файле
\usepackage[T2A]{fontenc} % Поддержка русских букв
\usepackage[utf8]{inputenc} % Кодировка utf8
\usepackage[english, russian]{babel} % Языки: русский, английский
\usepackage{enumitem}
\usepackage{pscyr} % Нормальные шрифты
\usepackage{soulutf8}
\usepackage{amsmath}
\usepackage{amsthm}
\usepackage{amssymb}
\usepackage{scrextend}
\usepackage{titling}
\usepackage{indentfirst}
\usepackage{cancel}
\usepackage{soulutf8}
\usepackage{wrapfig}
\usepackage{gensymb}
\usepackage[dvipsnames,table,xcdraw]{xcolor}
\usepackage{tikz}

%Русские символы в списке
\makeatletter
\AddEnumerateCounter{\asbuk}{\russian@alph}{щ}
\makeatother

%Дублирование знаков при переносе
\newcommand*{\hm}[1]{#1\nobreak\discretionary{}%
	{\hbox{$\mathsurround=0pt #1$}}{}}

\usepackage{graphicx}
\graphicspath{{pic/}}
\DeclareGraphicsExtensions{.pdf,.png,.jpg}

%Изменеие параметров листа
\usepackage[left=15mm,right=15mm,
top=2cm,bottom=2cm,bindingoffset=0cm]{geometry}


\usepackage{fancyhdr}
\pagestyle{fancy}
\usepackage{multicol}

\setlength\parindent{1,5em}
\usepackage{indentfirst}

\begin{document}
		
\lhead{Группа 71}
\chead{Модуль 4 ДЗ№1}
\rhead{Школа <<Симметрия>>}

\begin{enumerate}
	\item \textit{(2 балла)} Вычислить: $$4\dfrac{2}{7}:1\dfrac{5}{21}+\left(4\dfrac{3}{13}\cdot\dfrac{14}{15}-3\dfrac{1}{3}\right)$$
	\item \textit{(2 балла)} Внешние углы треугольника $ABC$ при вершинах $A$ и $C$ равны $115\degree$ и $140\degree$. Прямая, параллельная прямой $AC$, пересекает стороны $AB$ и $BC$ в точках $M$ и $N$. Найдите углы треугольника $BMN$.
	\item \textit{(2 балла)} Через середину $M$ отрезка с концами на двух параллельных прямых проведена прямая, пересекающая эти прямые в точках $A$ и $B$. Докажите, что $M$ также середина $AB$.
	\item \textit{(2 балла)} Угол при основании BC равнобедренного треугольника $ABC$ вдвое больше угла при вершине $A$, BD — биссектриса треугольника. Докажите, что $AD = BC$.
	\item \textit{(2 балла)} Прямая пересекает параллельные прямые $a$ и $b$ в точках $A$ и $B$ соответственно. Биссектриса одного из образовавшихся углов с вершиной $B$ пересекает прямую $a$ в точке $C$. Найдите $AB$, если $AC = 7$.
\end{enumerate}

\end{document}