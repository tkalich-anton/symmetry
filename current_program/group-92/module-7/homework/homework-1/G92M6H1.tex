\documentclass[12pt, a4paper]{article}
\usepackage{../../../../../style}
\begin{document}
	\lhead{Группа 92} \chead{Модуль 6 Домашняя работа 1} \rhead{Школа <<Симметрия>>} \cfoot{}
	\begin{enumerate}[label=\textbf{\arabic*.}]
		\item \textit{(2б.)} Решите уравнения:
		\begin{enumerate}[label=\asbuk*)]
			\begin{multicols}{2}
				\item $(2x-3)^2-9=0$
				\item $\dfrac{5x^2-48}{8}-\dfrac{33-2x^2}{6}=3\dfrac{5}{6}$
				\item $\left(x+\dfrac{1}{2}\right)\left(x-\dfrac{1}{2}\right)=\dfrac{5}{16}$
				\item $\dfrac{1}{x+3}-\dfrac{6}{9-x^2}=\dfrac{3}{x^2-6x+9}$
			\end{multicols}
		\end{enumerate}
		\item \textit{(2б.)} Решите неравенства: 
		\begin{enumerate}[label=\asbuk*)]
			\begin{multicols}{2}
				\item $\dfrac{x^2}{5}-\dfrac{3x}{7}<0$
				\item $-4-3x\geqslant1-2x^2$
				\item $\dfrac{12-x^2}{4}-\dfrac{x}{3}\leqslant\dfrac{(x-6)^2}{12}$
				\item $(x^2-16)(x^2-x-2)(x+2)>0$
			\end{multicols}
		\end{enumerate}
		\item \textit{(1б.)} Токарь до обеденного перерыва обточил $24$ детали, что составляет $60\%$ сменной нормы. Сколько деталей должен обточить токарь за смену?
		\item \textit{(1б.)} В $2019$ году в городском квартале проживало $72000$ человек. В $2020$ году, в результате строительства новых домов, число жителей выросло на $2\%$, а в $2021$ году – на $3\%$ по сравнению с $2020$ годом. На сколько человек увеличилась численность города с $2019$ по $2021$ год?
		\item \textit{(2б.)} Цена холодильника в магазине ежегодно уменьшается на одно и то же число процентов от предыдущей цены. Определите, на сколько процентов каждый год уменьшалась цена холодильника, если, выставленный на продажу за $19 800$ рублей, через два года был продан за $16 038$ рублей.
		\item \textit{(2б.)} Имеется кусок сплава меди с оловом общей массой 12 кг, содержащий 45\% меди. Сколько чистого олова надо прибавить в этому куску сплава, чтобы полученный новый сплав содержал 40\% меди?
	\end{enumerate}
\end{document}