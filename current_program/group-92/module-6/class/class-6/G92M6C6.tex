\documentclass[12pt, a4paper]{article}
\usepackage{cmap} % Улучшенный поиск русских слов в полученном pdf-файле
\usepackage[T2A]{fontenc} % Поддержка русских букв
\usepackage[utf8]{inputenc} % Кодировка utf8
\usepackage[english, russian]{babel} % Языки: русский, английский
\usepackage{enumitem}
\usepackage{pscyr} % Нормальные шрифты
\usepackage{amsmath}
\usepackage{amsthm}
\usepackage{amssymb}
\usepackage{scrextend}
\usepackage{titling}
\usepackage{indentfirst}
\usepackage{cancel}
\usepackage{soulutf8}
\usepackage{wrapfig}
\usepackage{gensymb}
\usepackage[dvipsnames,table,xcdraw]{xcolor}
\usepackage{tikz}

%Русские символы в списке
\makeatletter
\AddEnumerateCounter{\asbuk}{\russian@alph}{щ}
\makeatother

%Дублирование знаков при переносе
\newcommand*{\hm}[1]{#1\nobreak\discretionary{}%
	{\hbox{$\mathsurround=0pt #1$}}{}}

\usepackage{graphicx}
\graphicspath{{pic/}}
\DeclareGraphicsExtensions{.pdf,.png,.jpg}

%Изменеие параметров листа
\usepackage[left=15mm,right=15mm,
top=2cm,bottom=2cm,bindingoffset=0cm]{geometry}

\usepackage{fancyhdr}
\pagestyle{fancy}
\usepackage{multicol}
\setlength\parindent{1,5em}
\usepackage{indentfirst}
\begin{document}
	
	\lhead{Группа 92}
	\chead{Модуль 6 Урок 6}
	\rhead{Школа <<Симметрия>>}
	\begin{enumerate}
		\item Решите уравнения:
			\begin{enumerate}[label=\asbuk*)]
				\begin{multicols}{2}
				\item $-6-5x^2=-4x^4$
				\item $-3x^2-4+x^4=0$
				\item $3-2x^4=x^2$
				\item $3x^4+1=4x^2$
				\item $\dfrac{2}{x}+\dfrac{10}{x^2-2x}=\dfrac{1+2x}{x-2}$
				\item $\dfrac{12}{3x-x^2}+\dfrac{3x+5}{x-3}+1=0$
				\item $\dfrac{3x}{x+1}+\dfrac{2}{x}=\dfrac{2x+5}{x^2+x}$
				\item $\dfrac{33}{a^2-11a}+\dfrac{a-4}{11-a}=-\dfrac{3}{a}$
				\end{multicols}
			\end{enumerate}
		\item Решите неравенства: 
		\begin{enumerate}[label=\asbuk*)]
			\begin{multicols}{2}
				\item $(x-1)(25-x^2)(x^2-4x+4)>0$
				\item $10x^2-30+20x\leqslant0$
				\item $\dfrac{x-1}{x-2}>0$
				\item $\dfrac{x+3}{x-5}>0$
				\item $\dfrac{(x+1)(x+2)}{x-3}>0$
				\item $\dfrac{15x-5x^2}{12x-3x^2}>0$
				\item $\dfrac{x^2+6x+5}{x+2}<0$
				\item $\dfrac{(x-1)^2(x-2)}{(x-3)^2}\geqslant 0$
			\end{multicols}
		\end{enumerate}
	\item На экзамене по геометрии школьнику достаётся одна задача из сборника. Вероятность того, что эта задача по теме «Углы», равна $0,1$. Вероятность того, что это окажется задача по теме «Параллелограмм», равна $0,6$. В сборнике нет задач, которые одновременно относятся к этим двум темам. Найдите вероятность того, что на экзамене школьнику достанется задача по одной из этих двух тем.
	\item В фирме такси в данный момент свободно $20$ машин: $9$ черных, $4$ желтых и $7$ зеленых. По вызову выехала одна из машин, случайно оказавшаяся ближе всего к заказчику. Найдите вероятность того, что к нему приедет желтое такси.
	\end{enumerate}
\end{document}