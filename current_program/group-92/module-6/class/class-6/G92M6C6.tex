\documentclass[12pt, a4paper]{article}
\usepackage{../../../../../style}
\begin{document}
	
	\lhead{Группа 92} \chead{Модуль 6} \rhead{Школа <<Симметрия>>} \cfoot{}
	\begin{center}
		\Large
		\textbf{Занятие №6}
	\end{center}
	\begin{enumerate}[label=\textbf{\arabic*.}]
		\item Решите уравнения:
		\begin{enumerate}[label=\asbuk*)]
			\begin{multicols}{2}
			\item $3-2x^4=x^2$
			\item $x^3+4x^2+4x+1=0$
			\end{multicols}
		\end{enumerate}
		\item Решите уравнение: $$\dfrac{2}{x}+\dfrac{10}{x^2-2x}=\dfrac{1+2x}{x-2}$$
		\item Решите уравнение: $$\left(x+\dfrac{1}{x}\right)^2-4\dfrac{1}{2}\left(x+\dfrac{1}{x}\right)+5=0$$
		\item Решить систему уравнений:
		$$\left\{
		\begin{array}{l}
			\dfrac{3x-2y}{5}+\dfrac{5x-3y}{3}=x+1,\vspace{0,2cm}\\
			\dfrac{2x-3y}{3}+\dfrac{4x-3y}{2}=y+1
		\end{array}
		\right.$$
		\item Решите неравенства: 
		\begin{enumerate}[label=\asbuk*)]
			\begin{multicols}{2}
				\item $10x^2-30+20x\leqslant0$
				\item $(x-1)(25-x^2)(x^2-4x+4)>0$
				\item $\dfrac{(x+1)(x+2)}{x-3}>0$
				\item $\dfrac{x^2+6x+5}{x+2}<0$
			\end{multicols}
		\end{enumerate}
		\item Решить систему неравенств:
		$$\left\{
		\begin{array}{l}
			(x-1)(x-2)>0,\vspace{0,2cm}\\
			(x-1)(x-3)>0
		\end{array}
		\right.$$
		\item Решить систему неравенств:
		$$\left\{
		\begin{array}{l}
			x^2>4,\vspace{0,2cm}\\
			\dfrac{x^2-9}{x^2-8x+16}>0
		\end{array}
		\right.$$
		\item Участок прямоугольной формы обнесен изгородью. Если от него отрезать по прямой некоторую часть так, что оставшаяся часть окажется квадратом, то при этом его площадь уменьшится на $400$ м$^2$, а изгородь уменьшится на $20$ м. Определить первоначальные размеры участка (длину изгороди и площадь участка).
		\item Пешеход, идущий из дома на железнодорожную станцию, пройдя за первый час $3$ км, рассчитал, что он опоздает к отходу поезда на $40$ мин, если будет идти с той же скоростью. Поэтому остальной путь он прошел со скоростью на $4$ км/ч и прибыл на станцию за $15$ мин до отхода поезда. Чему равно расстояние от дома до станции и с какой постоянной на всем пути скоростью пешеход пришел бы на станцию точно к отходу поезда?
	\end{enumerate}
\end{document}