\documentclass[12pt, a4paper]{article}
\usepackage{../../../../../style}
\begin{document}
\lhead{Группа 92} \chead{Модуль 6 Занятие №5} \rhead{Школа <<Симметрия>>} \cfoot{}
\begin{enumerate}[label=\textbf{\arabic*.}]
	\item Решите уравнения:
	\begin{enumerate}[label=\asbuk*)]
		\begin{multicols}{2}
		\item $(x-1)^2-1=0$
		\item $\dfrac{4x^2-1}{3}-\dfrac{3x^2+8}{5}=1$
		\item $\dfrac{3}{x^2-2x+1}-\dfrac{1}{x^2-1}=\dfrac{1}{x+1}$
		\item $\dfrac{2x-3x^2}{5}-\dfrac{7x^2-x}{4}=\dfrac{x^2}{2}$
		\item $x^4-7x^2+12=0$
		\end{multicols}
	\end{enumerate}
	\item Решите неравенства: 
	\begin{enumerate}[label=\asbuk*)]
		\begin{multicols}{2}
			\item $(x-1)(x-3)(x-5)\geqslant0$
			\item $4x^2+7x>0$
			\item $x^2+3,5x-2>0$
			\item $x^2-\dfrac{0,16-2x}{4}\leqslant0,2$
			\item $\dfrac{12-x^2}{4}-\dfrac{x}{3}<\dfrac{(x-6)^2}{12}$
			\item $(4+x)(9-x^2)(x^2-2x+1)>0$
		\end{multicols}
	\end{enumerate}
	\item Расстояние между городами $A$ и $B$ равно 750 км. Из города $A$ в город $B$ со скоростью $50$ км/ч выехал первый автомобиль а через три часа после этого навстречу ему из города $B$ выехал со скоростью $70$ км/ч второй автомобиль. На каком расстоянии от города $A$ автомобили встретятся?
	\item Рыболов в $5$ часов утра на моторной лодке отправился от пристани против течения реки, через некоторое время бросил якорь, $2$ часа ловил рыбу и вернулся обратно в $10$ часов утра того же дня. На какое расстояние от пристани он отдалился, если скорость реки равна $2$ км/ч, а собственная скорость лодки $6$	 км/ч?
	\item Поезд, двигаясь равномерно со скоростью $63$ км/ч, проезжает мимо идущего в том же направлении параллельно путям со скоростью $3$ км/ч пешехода за $57$ секунд. Найдите длину поезда в метрах.
	\item Пешеход, идущий из дома на железнодорожную станцию, пройдя за первый час $3$ км, рассчитал, что он опоздает к отходу поезда на $40$ мин, если будет идти с той же скоростью. Поэтому остальной путь он прошел со скоростью на $4$ км/ч и прибыл на станцию за $15$ мин до отхода поезда. Чему равно расстояние от дома до станции и с какой постоянной на всем пути скоростью пешеход пришел бы на станцию точно к отходу поезда?
\end{enumerate}
\end{document}