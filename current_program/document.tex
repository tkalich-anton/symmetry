	\documentclass[10pt,]{article}
\usepackage{../style}
\begin{document}
\begin{listofex}
	\item Вычислить: \( \dfrac{1,23\cdot45,7}{12,3\cdot0,457}\)\\[0.5em]
	\textit{Решение:} Для начала представим десятичные дроби в виде обыкновенных. Для этого запишем все число без запятой в числитель, а в знаменателе запишем единицу и после нее такое количество нулей, сколько знаков было справа от запятой. Тогда:
	\[ 1,23=\dfrac{123}{100};\;45,7=\dfrac{457}{10};\;12,3=\dfrac{123}{10};\;0,457=\dfrac{457}{1000} \]
	Все вместе будет выглядеть так:
	\[
	\dfrac{1,23\cdot45,7}{12,3\cdot0,457}=
	\cfrac{\cfrac{123}{100}\cdot\cfrac{457}{10}}{\cfrac{123}{10}\cdot\cfrac{457}{1000}}
	=\cfrac{\cfrac{123\cdot457}{100\cdot10}}{\cfrac{123\cdot457}{10\cdot1000}}
	\]
	Теперь представим основную черту дроби в виде знака деления и применим правило деления дробей:
	\[
	\left( \dfrac{123\cdot457}{1000} \right):\left( \dfrac{123\cdot457}{10000} \right)=
	\dfrac{123\cdot457}{1000}\cdot\dfrac{10000}{123\cdot457}=
	\dfrac{123\cdot457\cdot10000}{123\cdot457\cdot1000}=
	\dfrac{10000}{1000}=10
	\]
	\item Вычислить: \( \dfrac{(5a^2)^3\cdot(6b)^2}{(30a^3b)^2} \)\\[0.5em]
	\textit{Решение:} Вспомним следующее свойства степени:
	\[ (a\cdot b)^c=a^c\cdot b^c;\;(a^b)^c=a^{b\cdot c} \]
	Тогда в нашем случае получим:
	\[ \dfrac{(5a^2)^3\cdot(6b)^2}{(30a^3b)^2}=\dfrac{5^3\cdot(a^2)^3\cdot6^2\cdot b^2}{30^2\cdot(a^3)^2\cdot b^2}=\dfrac{125\cdot36\cdot a^6\cdot b^2}{900\cdot a^6\cdot b^2} \]
	Сократим буквенные множители и получим:
	\[ \dfrac{125\cdot36\cdot a^6\cdot b^2}{900\cdot a^6\cdot b^2}=\dfrac{4500}{900}=\dfrac{45}{9}=5 \]
	\item Найдите \( p(x)+p(6-x) \), если \( p(x)=\dfrac{x(6-x)}{x-3} \) при \( x\neq3 \)\\
	\( \dfrac{x(6-x)}{x-3}+\dfrac{(6-x)(6-(6-x))}{(6-x)-3}=\dfrac{6x-x^2}{x-3}+\dfrac{(6-x)(6-6+x)}{6-x-3}=\dfrac{6x-x^2}{x-3}+\dfrac{(6-x)x}{3-x}=\dfrac{6x-x^2}{x-3}-\dfrac{6x-x^2}{x-3}=\dfrac{6x-x^2-(6x-x^2)}{x-3}=\dfrac{0}{x-3}=0 \)
	\item Найдите \( 61a-11b+50 \), если \( \dfrac{2a-7b+5}{7a-2b+5}=9 \)\\
	\( \dfrac{2a-7b+5}{7a-2b+5}=9 \); \( 2a-7b+5=9(7a-2b+5) \); \( 2a-7b+5=63a-18b+45 \); \( 2a-7b+5-63a+18b-45=0 \); \( -61a+11b-10=0 \); \( 61a-11b+10=0 \)\\
	\( 61a-11b+50-(61a-11b+10)=40 \)
	\item \( \dfrac{\sqrt[9]{7}\cdot\sqrt[18]{7}}{\sqrt[6]{7}}=\dfrac{7^{\tfrac{1}{9}}\cdot7^{\tfrac{1}{18}}}{7^{\tfrac{1}{6}}}=\dfrac{7^{\tfrac{1}{9}+\tfrac{1}{18}}}{7^{\tfrac{1}{5}}}=\dfrac{7^{\tfrac{2}{18}+\tfrac{1}{18}}}{7^{\tfrac{1}{6}}}=\dfrac{7^{\tfrac{3}{18}}}{7^{\tfrac{1}{6}}}=\dfrac{7^{\tfrac{1}{6}}}{7^{\tfrac{1}{6}}}=1 \)
	\item \( \dfrac{(\sqrt{3}+\sqrt{11})^2}{7+\sqrt{33}}=\dfrac{3+2\sqrt{11}\sqrt{3}+11}{7+\sqrt{33}}=\dfrac{14+2\sqrt{33}}{7+\sqrt{33}}=\dfrac{2(7+\sqrt{33})}{7+\sqrt{33}}=2 \)
	\item \( \dfrac{\sqrt[9]{\sqrt{m}}}{\sqrt{16\sqrt[9]{m}}}=\dfrac{(\sqrt{m})^{\tfrac{1}{9}}}{(16\sqrt[9]{m})^{\tfrac{1}{2}}}=\dfrac{(m^{\tfrac{1}{2}})^{\tfrac{1}{9}}}{(16)^{\tfrac{1}{2}}\cdot(m^{\tfrac{1}{9}})^{\tfrac{1}{2}}}=\dfrac{m^{\tfrac{1}{18}}}{4\cdot m^{\tfrac{1}{18}}}=\dfrac{1}{4}=0,25 \)
	\newpage
	\item \( (x-6)^2=-24x \)\\
	\( x^2-2\cdot6\cdot x+36+24x=0 \)\\
	\( x^2-12x+36+24x=0 \)\\
	\( x^2+12x+36=0 \)\\
	\( D=b^2-4ac=(12)^2-4\cdot1\cdot36=144-144=0 \)\\
	т.к. \( D=0 \), то корень будет один:
	\( x=\dfrac{-b}{2a}=\dfrac{-12}{2\cdot1}=-6 \)
	\item \( \dfrac{x-6}{7x+3}=\dfrac{x-6}{5x-1} \)\\
	\( \dfrac{x-6}{7x+3}-\dfrac{x-6}{5x-1}=0 \)
	\( (x-6)\left( \dfrac{1}{7x+3}-\dfrac{1}{5x-1} \right) =0\)\\
	\( x-6=0 \) \quad \quad \quad \( \dfrac{1}{7x+3}-\dfrac{1}{5x-1}=0 \)\\
	\( x=6 \) \quad \quad \quad \( \dfrac{5x-1-(7x+3)}{(7x+3)(5x-1)}=0 \)\\
	\( x=6 \) \quad \quad \quad \( \dfrac{5x-1-7x-3}{(7x+3)(5x-1)}=0 \)\\
	\( x=6 \) \quad \quad \quad \( \dfrac{-2x-4}{(7x+3)(5x-1)}=0 \)\\
	\( x=6 \) \quad \quad \quad \( -2x=4 \)\\
	\( x=6 \) \quad \quad \quad \( x=-2 \)\\
	\item \( \sqrt{\dfrac{6}{4x-54}}=\dfrac{1}{7} \)\\
	\( \left( \sqrt{\dfrac{6}{4x-54}} \right)^2=\left( \dfrac{1}{7} \right)^2 \)\\
	\( \dfrac{6}{4x-54}=\dfrac{1}{49} \)\\
	\( 6\cdot49=(4x-54)\cdot1 \)\\
	\( 294=4x-54 \)\\
	\( 348=4x \)\\
	\( x=87 \)
\end{listofex}
\end{document}