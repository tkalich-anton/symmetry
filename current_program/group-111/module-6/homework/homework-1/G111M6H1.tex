\documentclass[12pt, a4paper]{article}
\usepackage{../../../../../style}
\begin{document}
	\lhead{Группа 111} \chead{Модуль 6 Домашняя работа 1} \rhead{Школа <<Симметрия>>} \cfoot{}
	\begin{enumerate}[label=\textbf{\arabic*.}]
		\item Высота равнобедренного треугольника, опущенная на боковую сторону, разбивает её на отрезки, равные \( 2 \) и \( 1 \), считая от вершины треугольника. Найдите эту высоту.
		
		\item Катеты прямоугольного треугольника равны \( 5 \) и \( 7 \). Найдите биссектрису треугольника, проведённую из вершины прямого угла.
		
		\item Найдите площадь равнобедренного треугольника, если высота, опущенная на основание, равна \( 10 \), а высота, опущенная на боковую сторону, равна \( 12 \).
		\ranswer{Гордин ЕГЭ: \( 5.12 \); Ответ: \( 75 \)}
		
		\item Окружность касается сторон \( AB \) и \( BC \) треугольника \( ABC \)
		в точках \( D \) и \( E \) соответственно. Найдите высоту треугольника \( ABC \),
		опущенную из вершины \( A \), если \( AB = 5 \), \( AC = 2 \), а точки \( A \), \( D \), \( E \), \( C \) лежат на одной окружности.
		\ranswer{Гордин ЕГЭ: \( 5.21 \); Ответ: \( \frac{4\sqrt{6}}{5} \)}
		
		\item Биссектриса \( CD \) угла \( ACB \) при основании \( BC \) равнобедренного треугольника \( ABC \) делит сторону \( AB \) так, что \( AD = BC \). Найдите биссектрису \( CD \) и площадь треугольника \( ABC \), если \( BC = 2 \).
		\ranswer{Гордин ЕГЭ: \( 5.24 \); Ответ: \( 5 \) и \( \sqrt{5+2\sqrt{5}} \)}
		
		\item Три равных окружности проходят через одну точку и попарно пересекаются в трех других точках \( A \), \( B \) и \( C \). Докажите, что треугольник \( ABC \) равен треугольнику с вершинами в центрах окружностей.
		\ranswer{Гордин Планиметрия 7-9: \( 2.56 \)}
		
		\item Один из углов прямоугольной трапеции равен \( 120\degree \), большее основание равно \( 12 \). Найдите отрезок, соединяющий середины диагоналей, если известно, что меньшая диагональ трапеции равна ее большему основанию.
		\ranswer{Гордин Планиметрия 7-9: \( 2.140 \); Ответ: \( 3 \)}
		
	\end{enumerate}
\end{document}