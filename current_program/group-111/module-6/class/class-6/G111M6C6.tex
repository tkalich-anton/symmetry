\documentclass[12pt, a4paper]{article}

\DeclareUnicodeCharacter{202F}{\,}
\usepackage{../../../../../style}
\begin{document}
	
	\lhead{Группа 111}
	\chead{Модуль 6}
	\rhead{Школа <<Симметрия>>}
	\begin{center}
		\Large
		\textbf{Занятие №6}
	\end{center}
	\begin{enumerate}
		\item На сторонах \( AB \), \( BC \) и \( AC \) треугольника \( ABC \) отмечены точки \( C_1 \), \( A_1 \) и \( B_1 \) соответственно, причём \( AC_1 : C_1B = 8 : 3 \), \( BA_1 : A_1C = 1 : 2 \), \( CB_1 : B_1A = 3 : 1 \). Отрезки \( BB_1 \) и \( CC_1 \) пересекаются в точке \( D \).
		\begin{enumerate}[label=\asbuk*)]
			\item Докажите, что \( ADA1B1 \) — параллелограмм.
			\item Найдите \( CD \), если отрезки \( AD \) и \( BC \) перпендикулярны, \( AC = 28 \), \( BC = 18 \).
		\end{enumerate}
	\end{enumerate}
\end{document}