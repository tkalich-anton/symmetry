\documentclass[12pt, a4paper]{article}
\usepackage{cmap} % Улучшенный поиск русских слов в полученном pdf-файле
\usepackage[T2A]{fontenc} % Поддержка русских букв
\usepackage[utf8]{inputenc} % Кодировка utf8
\usepackage[english, russian]{babel} % Языки: русский, английский
\usepackage{enumitem}
\usepackage{pscyr} % Нормальные шрифты
\usepackage{soulutf8}
\usepackage{amsmath}
\usepackage{amsthm}
\usepackage{amssymb}
\usepackage{scrextend}
\usepackage{titling}
\usepackage{indentfirst}
\usepackage{cancel}
\usepackage{soulutf8}
\usepackage{wrapfig}
\usepackage{gensymb}
\usepackage[dvipsnames,table,xcdraw]{xcolor}
\usepackage{tikz}

%Русские символы в списке
\makeatletter
\AddEnumerateCounter{\asbuk}{\russian@alph}{щ}
\makeatother

%Дублирование знаков при переносе
\newcommand*{\hm}[1]{#1\nobreak\discretionary{}%
	{\hbox{$\mathsurround=0pt #1$}}{}}

\usepackage{graphicx}
\graphicspath{{pic/}}
\DeclareGraphicsExtensions{.pdf,.png,.jpg}

%Изменеие параметров листа
\usepackage[left=15mm,right=15mm,
top=2cm,bottom=2cm,bindingoffset=0cm]{geometry}


\usepackage{fancyhdr}
\pagestyle{fancy}
\usepackage{multicol}

\setlength\parindent{1,5em}
\usepackage{indentfirst}

\begin{document}
	
	\lhead{Группа 101}
	\chead{Модуль 5 Урок 2}
	\rhead{Школа <<Симметрия>>}
	
	\begin{enumerate}
		\item Стены здания снаружи решено облицевать плиткой. Здание имеет форму прямоугольного параллелепипеда. Его длина, ширина и высота равны 25 м, 15 м и 10 м соответственно. Суммарная площадь окон и входных дверей составляет 10\% от площади стен. Одного ящика плитки хватает на облицовку 3 кв. м, ящики с плиткой продаются только целиком. Плитку купили с запасом в 10\% от площади облицовки. Сколько ящиков плитки было куплено?
		\item Численность волков в двух заповедниках в 2009 году составляла 220 особей. Через год обнаружили, что в первом заповеднике численность волков возросла на 10\%, а во втором — на 20\%. В результате общая численность волков в двух заповедниках составила 250 особей. Сколько волков было в первом заповеднике в 2009 году?
		\item Какой вклад выгоднее: первый—на 1 год под 13\% годовых, или второй—на 3 месяца (с автоматической пролонгацией
		каждые три месяца в течение года) под 12\% годовых? При расчётах считайте, что один месяц равен $\dfrac{1}{12}$ части года.
		\item По вкладу «А» банк в конце каждого года планирует увеличивать на 20\% сумму, имеющуюся на вкладе в начале года, а по вкладу «Б» — увеличивать эту сумму на 10\% в первый год и на одинаковое целое число $n$ процентов и за второй, и за третий годы. Найдите наименьшее значение $n$, при котором за три года хранения вклад «Б» окажется выгоднее вклада «А» при одинаковых суммах первоначальных взносов.
		\item В июле 2019 года планируется взять кредит в банке в размере $S$ тыс. рублей (где $S$—натуральное число) сроком на 3 года. Условия его возврата таковы:
		\begin{itemize}
			\item каждый январь долг увеличивается на 15\% по сравнению с концом предыдущего года;
			\item с февраля по июнь каждого года необходимо выплатить одним платежом часть долга;
			\item в июле каждого года долг должен составлять часть кредита в соответствии со следующей таблицей.
		\end{itemize}
		\begin{tabular}{ | l | c | c |  c | c |}
			\hline
			Месяц и год & Июль 2019 & Июль 2020 & Июль 2021 & Июль 2022 \\ \hline
			Долг (в тыс. рублей) & S & 0,7S & 0,4S & 0 \\
			\hline
		\end{tabular}
		\\Найдите наименьшее значение S, при котором каждая из выплат
		будет составлять целое число тысяч рублей.
		\item Банк предоставляет кредит сроком на 10 лет под 19\% годовых на следующих условиях: ежегодно заёмщик возвращает банку 19\% от непогашенной части кредита и $\dfrac{1}{10}$ суммы кредита. Так, в первый год заёмщик выплачивает $\dfrac{1}{10}$ суммы кредита и 19\% от всей суммы кредита, во второй год заёмщик выплачивает $\dfrac{1}{10}$ суммы кредита и 19\% от $\dfrac{9}{10}$ суммы кредита и т. д. Во сколько раз сумма, которую выплатит банку заёмщик, будет больше суммы кредита, если заёмщик не воспользуется досрочным погашением кредита?
		\item Алексей взял кредит в банке на срок 17 месяцев. По договору Алексей должен вернуть кредит ежемесячными платежами. В конце каждого месяца к оставшейся сумме долга добавляется $r\%$ этой суммы, и своим ежемесячным платежом Алексей погашает эти добавленные проценты и уменьшает сумму долга. Ежемесячные платежи подбираются так, чтобы долг уменьшался на одну и ту же величину каждый месяц. Известно, что общая сумма, выплаченная Алексеем банку за весь срок кредитования, оказалась на 27\% больше, чем сумма, взятая им в кредит. Найдите $r$.
	\end{enumerate}
	
\end{document}