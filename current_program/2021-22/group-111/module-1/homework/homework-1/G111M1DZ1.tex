\documentclass[12pt, a4paper]{article}
\usepackage[T2A]{fontenc}
\usepackage[utf8]{inputenc}
\usepackage[russian]{babel}
\usepackage{enumitem}
\usepackage{pscyr} % Нормальные шрифты

%Русские символы в списке
\makeatletter
\AddEnumerateCounter{\asbuk}{\russian@alph}{щ}
\makeatother
%Дублирование знаков при переносе
\newcommand*{\hm}[1]{#1\nobreak\discretionary{}%
	{\hbox{$\mathsurround=0pt #1$}}{}}

\usepackage{graphicx}
\graphicspath{{pic/}}
\DeclareGraphicsExtensions{.pdf,.png,.jpg}

%Изменеие параметров листа
\usepackage[left=2cm,right=2cm,
top=2cm,bottom=2cm,bindingoffset=0cm]{geometry}

\usepackage{amsmath,amsthm,amssymb,scrextend}
\usepackage{fancyhdr}
\pagestyle{fancy}
\usepackage{multicol}

\begin{document}
		
	\lhead{Группа 111}
	\chead{Модуль 1 Домашняя работа №1}
	\rhead{Школа <<Симметрия>>}
	
	\section*{Текстовые задачи}
		\begin{enumerate}
			\item Имеются два куска кабеля разных сортов. Масса первого куска равна 65 кг; другой, длина которого на 3 м больше массы каждого метра первого куска, имеет массу 120 кг. Найти длины этих кусков.
			\item В сосуд, содержащий 9 литров 14–процентного водного раствора некоторого вещества, добавили 5 литров воды. Сколько процентов составляет концентрация получившегося раствора?
			\item Имеется два сплава. Первый содержит 15\% никеля, второй — 35\% никеля. Из этих двух сплавов получили третий сплав массой 140 кг, содержащий 30\% никеля. На сколько килограммов масса первого сплава была меньше массы второго?
			\item Старший брат на мотоцикле, а младший на велосипеде совершили двухчасовую безостановочную поездку в лес и боратно. При этом мотоциклист проезжал каждый километр на 4 мин. быстрее, чем велосипедист. Сколько киллометров проехал каждый из братьев за 2 ч, если известно, что путь, проделанный старшим братом за это время, на 40 км больше?
			\item Товарный поезд был задержан в пути на 12 мин, а затем на расстоянии 60 км наверстал потерянное время, увеличив скорость на 15км/ч. Найти вероначальную скорость.
			\item Денежная премия была распределена между тремя изобретателями: первый получил половину всей премии без $\dfrac{3}{22}$ того, что получили двое других вместе. Второй получил $\dfrac{1}{4}$ всей премии и $\dfrac{1}{56}$ денег, полученных вместе двумя остальными. Третий получил 300 000 руб. Как велика была премия и сколько денег получил каждый изобретатель?
			\item Определить целое положительное число по следующим данным: если его записать цифрами и присоединить справа 4, то получится число, делящееся без остатка на число, большее искомого на 4, причем полученное частное представляет собой число, меньшее делителя на 27.
			\item По двум окружностям равномерно вращаются две точки. Одна из них совершает полный оборот на 5 секунд быстрее, чем другая, и поэтому успевает сделать за 1 мин на два оборота больше. Сколько оборотов в минуту совершает каждая точка?
			\item Числители трех дробей пропорциональны числам 1, 2, 5, а знаменатели пропорциональны соответственно 1, 3, 7. Среднее арифметическое этих дробей равно $\dfrac{200}{441}$. Найти эти дроби.
			\item В одном бассейне имеется 200 м$^3$ воды, а в другом — 112 м$^3$. Открывают краны, через которые наполняются бассейны. Через сколько часов количество воды в бассейнах будет одинаковым, если во второй бассейн вливается на 22 м$^3$ в час больше, чем в первый?
			\item Рабочий день уменьшился с 8 часов до 7. На сколько процентов нужно повысить производительность труда, чтобы при тех же расценках заработная плата возросла на 5\%?
			\item Сумма всех четных двухзначных чисел разделилась на одно из них бех остатка. Полученное частное отличается от делителя только порядком цифр, а сумма его цифр равна 9. Какое двухзначное число являлось делителем?
			\subsection*{Дополнительные задачи}
			\item Несколько рабочих выполняют задание за 14 дней. Если бы их было на 4 человека больше и каждый бы работал на 1 ч дольше, это же задание было бы выполнено за 10 дней. Если же их было \underline{еще} на 6 человек больше и каждый бы работал \underline{еще} на 1 ч в день дольше, задание было бы выполнено за 7 дней. Сколько было рабочих и сколько часов в день они работали?
			\item Поезд был задержан на станции отправления на 1 ч 42 мин. Получив сигнал отправления, машинист повел состав по такому графику: на участке, составляющем 0,9 всего пути от станции отправления до станции назначения, он поддерживал скорость на 20\% выше обычной и 0,1 пути вел состав со скоростью на 25\% выше обычной. В результате поезд прибыл на станцию назначения вовремя. Какова продолжительность движения этого поезда между станциями при обычной скорости?
		\end{enumerate}
\end{document}