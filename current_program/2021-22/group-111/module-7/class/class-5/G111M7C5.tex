\documentclass[12pt, a4paper]{article}
\usepackage{../../../../../style}
\begin{document}
	
\lhead{Группа 111} \chead{Модуль 7} \rhead{Школа <<Симметрия>>} \cfoot{}
\begin{center}
	\Large
	\textbf{Занятие №5}
\end{center}
\begin{enumerate}[label=\textbf{\arabic*.}]
	\item При каких значениях \( a \) сумма корней уравнения \( x^2-2a(x-1)-1=0 \) равна сумме квадратов его корней?
	\item При каких значениях параметра \( a \) уравнение \[ (a+4x-x^2-1)(a+1-|x-2|)=0 \] имеет три корня?
	\item При каких значениях параметра \( a \) уравнение \( x^4-(3a-1)x^2+2a^2-a=0 \) имеет два решения?
	\item При каких значениях параметра \( a \) уравнение \( (x^2-2x)^2-(a+2)(x^2-2x)+3a-3=0 \) имеет четыре решения?
	\item При каких значениях параметра \( a \) уравнение \[ \dfrac{4^{-x^2}-a\cdot2^{1-x^2}+a}{2^{1-x^2}-1}=3 \] имеет хотя бы одно решение?
\end{enumerate}
\end{document}