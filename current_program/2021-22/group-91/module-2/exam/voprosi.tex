\documentclass[12pt, a4paper]{article}
\usepackage{cmap} % Улучшенный поиск русских слов в полученном pdf-файле
\usepackage[T2A]{fontenc} % Поддержка русских букв
\usepackage[utf8]{inputenc} % Кодировка utf8
\usepackage[english, russian]{babel} % Языки: русский, английский
\usepackage{enumitem}
\usepackage{pscyr} % Нормальные шрифты
\usepackage{soulutf8}
\usepackage{amsmath}
\usepackage{amsthm}
\usepackage{amssymb}
\usepackage{scrextend}
\usepackage{titling}
\usepackage{indentfirst}
\usepackage{cancel}
\usepackage{soulutf8}
\usepackage{wrapfig}
\usepackage{gensymb}
\usepackage[dvipsnames,table,xcdraw]{xcolor}
\usepackage{tikz}

%Русские символы в списке
\makeatletter
\AddEnumerateCounter{\asbuk}{\russian@alph}{щ}
\makeatother

%Дублирование знаков при переносе
\newcommand*{\hm}[1]{#1\nobreak\discretionary{}%
	{\hbox{$\mathsurround=0pt #1$}}{}}

\usepackage{graphicx}
\graphicspath{{pic/}}
\DeclareGraphicsExtensions{.pdf,.png,.jpg}

%Изменеие параметров листа
\usepackage[left=15mm,right=15mm,
top=2cm,bottom=2cm,bindingoffset=0cm]{geometry}

\usepackage{fancyhdr}
\pagestyle{fancy}
\usepackage{multicol}

\setlength\parindent{1,5em}
\usepackage{indentfirst}

\begin{document}
		
\lhead{Экзамен Модуль №2}
\rhead{Школа <<Симметрия>>}

\section{Список вопросов}
\begin{enumerate}
	\subsection{Углы}
	\item Что такое угол? Назовите виды углов.
	\item Что такое смежные углы? Каким свойством они обладают?
	\item Что такое вертикальные углы? Каким свойством они обладают?
	\item Какие существуют пары углов, образованных пересечением двух прямых и секущей? Назовите их свойства, если прямые параллельны.
	\item Чему равен угол между биссектрисами двух смежных углов?
	\item Являются ли биссектрисы двух вертикальных углов дополнениями друг друга до прямой?
	\item Чему равен угол между биссектрисами двух внутренних односторонних углов?
	\subsection{Множества точек. Перпендикуляры и наклонные.}
	\item Что является расстоянием от точки до точки?
	\item Что такое перпендикуляр, опущенный из точки на прямую?
	\item Что является расстоянием от точки до прямой?
	\item Что является основанием перпендикуляра?
	\item Сколько различных перпендикуляров можно опустить из точки на прямую?
	\item Что можно сказать о двух различных перпендикулярах, проведенных к одной прямой?
	\item Будут ли параллельны два перпендикуляра, проведенные к двум параллельным прямым? Объясните почему.
	\item Что можно сказать о двух прямых, если известно, что возможно провести третью прямую, которая будет перпендикулярна каждой из первых двух прямых?
	\item Что такое наклонная, проведенная из точки к прямой?
	\item Что означает, что прямые взаимно перпендикулярны? Могут ли три прямые быть попарно взаимно перпендикулярны? А четыре прямые?
	\item Чем отличается опущенный перпендикуляр от восстановленного?
	\item Как называется множество точек, равноудаленных от сторон некоторого угла? Объясните почему.
	\item Как называется множество точек, равноудаленных от некоторой точки на заданное расстояние?
	\item Как называется множество точек, равноудаленных от концов некоторого отрезка? Объясните почему.
	\item Как называется множество точек, равноудаленных от двух параллельных прямых?
	\subsection{Треугольники}
	\item Назовите виды треугольников.
	\item Какой треугольник называется правильным?
	\item Перечислите свойства равнобедренного треугольника.
	\item Перечислите свойства равностороннего треугольника.
	\item Что такое внутренний и внешний угол треугольника? Сформулируйте теорему о внешнем угле треугольника.
	\item Сформулируйте теорему о сумме углов треугольника.
	\item Сформулируйте признаки равенства треугольников.
	\item Сформулируйте признаки равенства прямоугольных треугольников.
	\item Пересекутся ли биссектриса одного внутреннего угла и биссектрисы двух внешних углов, не смежных с первым, в одной точке?
	\item Что можно сказать о треугольнике, если известно, что один из его углов равен сумме двух других его углов?
	\item Что такое медиана треугольника? Сколько можно провести медиан в одном треугольнике?
	\item Что такое высота треугольника? Сколько можно провести высот в одном треугольнике?
	\item Что такое биссектриса треугольника? Сколько можно провести биссектрис в одном треугольнике?
	\item Что можно сказать о треугольнике, если известно, что одна из его медиан равна половине стороны, к которой она проведена?
	\item Сформулируйте теорему о медиане, проведенной из вершины прямого угла в прямоугольном треугольнике.
	\item Сформулируйте теорему об угле в 30\degree в прямоугольном треугольнике.
	\item Где расположен центр окружности, описанной вокруг прямоугольного треугольника?
	\item Что значит, что окружность описанна вокруг треугольника?
	\item Что значит, что окружность вписана в треугольник?
	\item Где лежит центр окружности, вписанной в треугольник?
	\item Где лежит центр окружности, описанной вокруг треугольника?
	\item Перечислите четыре замечательные точки треугольника.
	\subsection{Окружности}
	\item Что такое окружность?
	\item Что такое касательная к окружности?
	\item Сколько можно провести касательных к окружности через точку, которая находится внутри окружности? Объясните почему.
	\item Сколько можно провести касательных к окружности через точку, которая находится на окружности? Объясните почему.
	\item Сколько можно провести касательных к окружности через точку, которая находится вне окружности? 
	\item Сколько точек пересечения имеют окружность и секущая?
	\item Чему равен угол между касательной к окружности и радиусом, проведенным к точке касания? Объясните почему.
	\item Что такое хорда?
	\item Что такое диаметр?
	\item Какая хорда будет самая длинная в окружности?
	\item Докажите, что если через концы диаметра провести касательные к окружности, то эти касательные будут параллельны.
	\item Сформулируйте теорему о радиусе, проведенном через середину хорды.
	\item Сформулируйте теорему о радиусе, перпендикулярном хорде.
	\item Сформулируйте теорему о касательных, проведенных из одной точки к окружности.
	\item Где расположен центр вписанной в угол окружности?
	\item Что такое концентрические окружности?
	\item Что такое вневписанная окружность? Сколько таких окружностей имеет треугольник?
\end{enumerate}
\section{Список задач}
\begin{enumerate}
	\item Два угла треугольника равны $10\degree$ и $70\degree$. Найдите угол между высотой и биссектрисой, проведенными из вершины третьего угла треугольника.
	\item На стороне $AB$ квадрата $ABCD$ построен равносторонний треугольник $ABM$. Найдите угол $DMC$.
	\item Хорда большей из двух концентрических окружностей касается меньшей. Докажите, что точка касания делит эту хорду пополам.
	\item Один из углов треугольника равен $50\degree$. Найдите угол между высотами, проведенными из вершин двух других углов.
	\item Биссектрисы двух углов треугольника пересекаются под углом $110\degree$. Найдите третий угол треугольника.
	\item Известно, что $AB$ — диаметр окружности, а хорды $AC$ и $BD$ параллельны. Докажите, что $AC = BD$, а $CD$ также диаметр.
	\item Угол при основании $BC$ равнобедренного треугольника $ABC$ вдвое больше угла при вершине $A$, $BD$ — биссектриса треугольника. Докажите, что $AD = BC$.
	\item На катете $AC$ прямоугольного треугольника $ABC$ как на диаметре построена окружность, пересекающая гипотенузу $AB$ в точке $K$. Найдите $CK$, если $AC = 2$ и $\angle A = 30\degree$
	\item Докажите, что окружность, построенная на боковой стороне равнобедренного треугольника как на диаметре, проходит через середину основания.
	\item Докажите, что окружность, построенная на стороне равностороннего треугольника как на диаметре, проходит через середины двух других сторон треугольника.
	\item Окружность, построенная на стороне треугольника как на диаметре, проходит через середину другой стороны. Докажите, что треугольник равнобедренный.
	
\end{enumerate}

\end{document}