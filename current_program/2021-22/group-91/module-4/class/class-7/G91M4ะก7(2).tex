\documentclass[12pt, a4paper]{article}
\usepackage{cmap} % Улучшенный поиск русских слов в полученном pdf-файле
\usepackage[T2A]{fontenc} % Поддержка русских букв
\usepackage[utf8]{inputenc} % Кодировка utf8
\usepackage[english, russian]{babel} % Языки: русский, английский
\usepackage{enumitem}
\usepackage{pscyr} % Нормальные шрифты
\usepackage{soulutf8}
\usepackage{amsmath}
\usepackage{amsthm}
\usepackage{amssymb}
\usepackage{scrextend}
\usepackage{titling}
\usepackage{indentfirst}
\usepackage{cancel}
\usepackage{soulutf8}
\usepackage{wrapfig}
\usepackage{gensymb}
\usepackage[dvipsnames,table,xcdraw]{xcolor}
\usepackage{tikz}

%Русские символы в списке
\makeatletter
\AddEnumerateCounter{\asbuk}{\russian@alph}{щ}
\makeatother

%Дублирование знаков при переносе
\newcommand*{\hm}[1]{#1\nobreak\discretionary{}%
	{\hbox{$\mathsurround=0pt #1$}}{}}

\usepackage{graphicx}
\graphicspath{{pic/}}
\DeclareGraphicsExtensions{.pdf,.png,.jpg}

%Изменеие параметров листа
\usepackage[left=15mm,right=15mm,
top=2cm,bottom=2cm,bindingoffset=0cm]{geometry}


\usepackage{fancyhdr}
\pagestyle{fancy}
\usepackage{multicol}

\setlength\parindent{1,5em}
\usepackage{indentfirst}

\begin{document}
	
	\lhead{Группа 91}
	\chead{Модуль 4 Урок №7}
	\rhead{Школа <<Симметрия>>}
	\begin{enumerate}
		\item Вычислить: $$\dfrac{\left(8\dfrac{1}{4}-3,51\right):2,37}{\dfrac{1}{5}\cdot3,17-2,205:3\dfrac{1}{2}}$$
		\item Сократить дробь
		\begin{enumerate}[label=\asbuk*)]
			\begin{multicols}{2}
				\item $\dfrac{a^3-2a^2}{4-a^2}$
				\item $\dfrac{a^2-b^2}{a^2+2ab+b^2}$
				\item $\dfrac{2mn-6m^2}{12m^2n-4mn^2}$
			\end{multicols}
		\end{enumerate}
		\item Вычислить:
		\begin{enumerate}[label=\asbuk*)]
			\begin{multicols}{2}
				\item $\sqrt{8\cdot50}$
				\item $\sqrt{6\cdot30\cdot245}$
				\item $\sqrt{21\cdot35\cdot15}$
			\end{multicols}
		\end{enumerate}
		\item Решить уравнение $2x^2=5+3x$
		\item Решить уравнение $(x+8)(x-9)=-52$
		\item Решить систему неравенств $\left\{
		\begin{aligned}
			2x-1>3x+1,\\
			5x-1>13
		\end{aligned}
		\right.$
		\item Найдите диагональ прямоугольника со сторонами $5$ и $12$.
		\item Дан треугольник с периметром, равным $24$. Найдите периметр треугольника с вершинами в серединах сторон данного.
		\item Стороны треугольника равны $6$ и $9$. Через середину третьей стороны проведены прямые, параллельные двум другим сторонам. Найдите периметр полученного четырехугольника.
		\item Найдите периметр четырехугольника с вершинами в серединах сторон прямоугольника с диагональю, равной $8$.
	\end{enumerate}
\end{document}