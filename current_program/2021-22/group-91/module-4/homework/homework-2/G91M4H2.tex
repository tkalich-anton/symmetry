\documentclass[12pt, a4paper]{article}
\usepackage{cmap} % Улучшенный поиск русских слов в полученном pdf-файле
\usepackage[T2A]{fontenc} % Поддержка русских букв
\usepackage[utf8]{inputenc} % Кодировка utf8
\usepackage[english, russian]{babel} % Языки: русский, английский
\usepackage{enumitem}
\usepackage{pscyr} % Нормальные шрифты
\usepackage{soulutf8}
\usepackage{amsmath}
\usepackage{amsthm}
\usepackage{amssymb}
\usepackage{scrextend}
\usepackage{titling}
\usepackage{indentfirst}
\usepackage{cancel}
\usepackage{soulutf8}
\usepackage{wrapfig}
\usepackage{gensymb}
\usepackage[dvipsnames,table,xcdraw]{xcolor}
\usepackage{tikz}

%Русские символы в списке
\makeatletter
\AddEnumerateCounter{\asbuk}{\russian@alph}{щ}
\makeatother

%Дублирование знаков при переносе
\newcommand*{\hm}[1]{#1\nobreak\discretionary{}%
	{\hbox{$\mathsurround=0pt #1$}}{}}

\usepackage{graphicx}
\graphicspath{{pic/}}
\DeclareGraphicsExtensions{.pdf,.png,.jpg}

%Изменеие параметров листа
\usepackage[left=15mm,right=15mm,
top=2cm,bottom=2cm,bindingoffset=0cm]{geometry}


\usepackage{fancyhdr}
\pagestyle{fancy}
\usepackage{multicol}

\setlength\parindent{1,5em}
\usepackage{indentfirst}

\begin{document}
		
\lhead{Группа 91}
\chead{Модуль 4 ДЗ№2}
\rhead{Школа <<Симметрия>>}

\begin{enumerate}
	\item Вычислить: $$\left(\dfrac{1}{2}+\dfrac{11}{12}+\dfrac{3}{4}+\dfrac{5}{6}\right)\cdot(-5)+(-756):(-36)$$
	\item Вычислить: $$(2\sqrt{3}+7)(3\sqrt{3}-6)-9\sqrt{3}$$
	\item Решить уравнение:$$x^2-5x-24=0$$
	\item Решить уравнение:$$(x+7)(5x-3)=0$$
	\item Решить уравнение:$$(x^2-3x+1)(x^2-4x+4)=0$$
	\item Решить уравнение:$$\dfrac{x^2-x-20}{x-5}=0$$
	\item В прямоугольном треугольнике $ABC$ ($\angle C = 90\degree$) известно, что $BC = 6$, $AC = 8$. Найдите $AB$.
	\item Прямоугольный треугольник $ABC$ вписан в окружность. Гипотенуза $AC=8$, а $\angle BAC = 30\degree$. Найдите $BC$.
	
\end{enumerate}

\end{document}