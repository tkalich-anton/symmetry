\documentclass[12pt, a4paper]{article}
\usepackage{cmap} % Улучшенный поиск русских слов в полученном pdf-файле
\usepackage[T2A]{fontenc} % Поддержка русских букв
\usepackage[utf8]{inputenc} % Кодировка utf8
\usepackage[english, russian]{babel} % Языки: русский, английский
\usepackage{enumitem}
\usepackage{pscyr} % Нормальные шрифты
\usepackage{amsmath}
\usepackage{amsthm}
\usepackage{amssymb}
\usepackage{scrextend}
\usepackage{titling}
\usepackage{indentfirst}
\usepackage{cancel}
\usepackage{soulutf8}
\usepackage{wrapfig}
\usepackage{gensymb}
\usepackage[dvipsnames,table,xcdraw]{xcolor}
\usepackage{tikz}

%Русские символы в списке
\makeatletter
\AddEnumerateCounter{\asbuk}{\russian@alph}{щ}
\makeatother

%Дублирование знаков при переносе
\newcommand*{\hm}[1]{#1\nobreak\discretionary{}%
	{\hbox{$\mathsurround=0pt #1$}}{}}

\usepackage{graphicx}
\graphicspath{{pic/}}
\DeclareGraphicsExtensions{.pdf,.png,.jpg}

%Изменеие параметров листа
\usepackage[left=15mm,right=15mm,
top=2cm,bottom=2cm,bindingoffset=0cm]{geometry}

\usepackage{fancyhdr}
\pagestyle{fancy}
\usepackage{multicol}

\setlength\parindent{1,5em}
\usepackage{indentfirst}
\begin{document}
	
	\lhead{Группа 82}
	\chead{Модуль 6 Урок №1-2}
	\rhead{Школа <<Симметрия>>}
	\begin{enumerate}
		\item Докажите, что середины сторон любого четырехугольника являются вершинами параллелограмма.
		\item Острый угол $A$ ромба $ABCD$ равен $45\degree$, проекция стороны $AB$ на сторону $AD$ равна $12$. Найдите расстояние от центра ромба до стороны $CD$.
		\item Расстояние от середины хорды $BC$ до диаметра $AB$ равно $1$. Найдите хорду $AC$, если $\angle BAC = 30 \degree$.

		\item Две прямые, проходящие через точку $C$, касаются окружности в точках $A$ и $B$. Может ли прямая, проходящая через середины отрезков $AC$ и $BC$, касаться этой окружности?
		\item Сторона треугольника равна $a$. Найдите отрезок, соединяющий середины медиан, проведенных к двум другим сторонам.
		\item Докажите, что середины двух противоположных сторон любого четырехугольника без параллельных сторон и середины его диагоналей являются вершинами параллелограмма.
		\item В выпуклом четырехугольнике $ABCD$ отрезок, соединяющий середины сторон $AB$ и $CD$, равен $1$. Прямые $BC$ и $AD$ перпендикулярны. Найдите отрезок, соединяющий середины диагоналей $AC$ и $BD$.
	\end{enumerate}
\end{document}