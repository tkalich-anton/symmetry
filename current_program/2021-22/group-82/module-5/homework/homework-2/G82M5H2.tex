\documentclass[12pt, a4paper]{article}
\usepackage{cmap} % Улучшенный поиск русских слов в полученном pdf-файле
\usepackage[T2A]{fontenc} % Поддержка русских букв
\usepackage[utf8]{inputenc} % Кодировка utf8
\usepackage[english, russian]{babel} % Языки: русский, английский
\usepackage{enumitem}
\usepackage{pscyr} % Нормальные шрифты
\usepackage{amsmath}
\usepackage{amsthm}
\usepackage{amssymb}
\usepackage{scrextend}
\usepackage{titling}
\usepackage{indentfirst}
\usepackage{cancel}
\usepackage{soulutf8}
\usepackage{wrapfig}
\usepackage{gensymb}
\usepackage[dvipsnames,table,xcdraw]{xcolor}
\usepackage{tikz}

%Русские символы в списке
\makeatletter
\AddEnumerateCounter{\asbuk}{\russian@alph}{щ}
\makeatother

%Дублирование знаков при переносе
\newcommand*{\hm}[1]{#1\nobreak\discretionary{}%
	{\hbox{$\mathsurround=0pt #1$}}{}}

\usepackage{graphicx}
\graphicspath{{pic/}}
\DeclareGraphicsExtensions{.pdf,.png,.jpg}

%Изменеие параметров листа
\usepackage[left=15mm,right=15mm,
top=2cm,bottom=2cm,bindingoffset=0cm]{geometry}

\usepackage{fancyhdr}
\pagestyle{fancy}
\usepackage{multicol}

\setlength\parindent{1,5em}
\usepackage{indentfirst}
\begin{document}
	
	\lhead{Группа 82}
	\chead{Модуль 5 Домашняя работа №2}
	\rhead{Школа <<Симметрия>>}
	\section*{Домашняя работа №2}
	\begin{enumerate}
		\item \textit{(4 балла)} Вычислите:
			\begin{enumerate}[label=\asbuk*)]
				\item $\left( \dfrac{2}{(a-2)^2}-\dfrac{a}{4-a^2}\right):\dfrac{4+a^2}{4-a^2}+\dfrac{2}{a-2}$
				\item $\dfrac{x^2}{x^2+4x+4}\cdot\dfrac{8x^2-32}{x^3-2x^2}+\dfrac{x^5-8x^2}{x}:(x^2-4)$
				\item $\dfrac{2}{5+2\sqrt{6}}+\dfrac{2}{5-2\sqrt{6}}$
				\item $\dfrac{4}{\sqrt{5}-3}+3+\sqrt{5}$
		\end{enumerate}
		\item \textit{(1 балл)} Найдите уравнение прямой, которая проходит через точки с координатами $(-2;-2)$ и $(0;4)$.
		\item \textit{(1 балл)} Найдите координаты точки пересечения пересечения прямых $y=-0,5x-2$ и $y=0,5x+8$.
		\item \textit{(2 балла)} Выяснить, лежат ли точки $A(-2;-2)$, $B(10;4)$ и $C(17;10)$ на одной прямой.
		\item \textit{(2 балла)} Выяснить, можно ли через точки $A(-6;6)$, $B(2;-8)$, $C(-8;-2)$ и $D(14;-6)$ провести две параллельные прямые.
	\end{enumerate}
\end{document}