\documentclass[12pt, a4paper]{article}
\usepackage{../../../../../style}
\DeclareUnicodeCharacter{202F}{\,}
\begin{document}
	
\lhead{Группа 102} \chead{Модуль 6} \rhead{Школа <<Симметрия>>} \cfoot{}
\begin{center}
	\Large
	\textbf{Занятие №8}
\end{center}
\begin{enumerate}[label=\textbf{\arabic*.}]
	\item Дана четырёхугольная пирамида \( SABCD \), основание которой --- прямоугольник \( ABCD \), а высота проходит через центр \( O \)
	основания. Через середину \( A_1 \) бокового ребра \( SA \) проведена плоскость \( \alpha \), параллельная плоскости основания, а через середину \( C_1 \) бокового ребра \( SC \) и ребро \( AB \) --- плоскость \( \beta \). Найдите угол между плоскостями \( \alpha \) и \( \beta \), если \( AB : BC : SA=8 : 6 : 13 \).
	\item Дан прямоугольный параллелепипед \( ABCD_1B_1C_1D_1 \), где сторона \( AB = 2 \), \( AD = 3 \), \( AA_1 = 7 \), точка \( E \) разделяет сторону \( AA_1 \) в отношении \( 4:3 \). Найти угол между плоскостями \( ABC \) и \( BED_1 \).
	\item Дана правильная треугольная призма \( ABCA_1B_1C_1 \). Боковое ребро \( AA_1 \) равно стороне основания \( ABC \). Точка \( M \) --- середина ребра \( BC \). Найдите угол между плоскостями \( CA_1B_1 \) и \( ABC \).
\end{enumerate}
\end{document}