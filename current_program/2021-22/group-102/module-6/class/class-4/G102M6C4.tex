\documentclass[12pt, a4paper]{article}
\usepackage{../../../../../style}
\begin{document}
	
\lhead{Группа 102} \chead{Модуль 6} \rhead{Школа <<Симметрия>>} \cfoot{}
\begin{center}
	\Large
	\textbf{Занятие №4}
\end{center}
\begin{enumerate}[label=\textbf{\arabic*.}]
	\item Основание треугольной пирамиды $DABC$ --- прямоугольный треугольник $ABC$ ($\angle C = 90\degree$). Высота пирамиды проходит через точку $C$.
	\begin{enumerate}[label=\asbuk*)]
		\item Докажите, что противоположные рёбра пирамиды попарно перпендикулярны.
		\item Найдите углы, которые образуют боковые рёбра $DA$ и $DB$ с плоскостью основания, если $AC = 15$, $BC = 20$, а угол между плоскостями $ABC$ и $ABD$ равен $45\degree$.
	\end{enumerate}
	\item Дан прямоугольный параллелепипед $ABCDA_1B_1C_1D_1$, в котором $AD=2$, $AA_1 =4$, $AB=2\sqrt{15}$. Точка $M$ --- середина ребра $C_1D_1$, точка $N$ лежит на ребре $AA_1$, причём $AN = 3$.
	\begin{enumerate}[label=\asbuk*)]
		\item Докажите, что $MN \perp CB_1$
		\item Найдите угол между прямой $MN$ и плоскостью грани $BB_1C_1C$.
	\end{enumerate}
	\item Дана четырёхугольная пирамида $SABCD$, основание которой --- параллелограмм $ABCD$. Точка $K$ --- середина медианы $SM$ грани $CSD$, $N$ --- середина ребра $AB$.
	\begin{enumerate}[label=\asbuk*)]
		\item Постройте точку пересечения прямой $KN$ с плоскостью $ASC$.
		\item Найдите угол между прямой $KN$ и плоскостью $ASC$, если пирамида правильная, а её боковые грани образуют с плоскостью основания углы, равные $60\degree$.
	\end{enumerate}
\end{enumerate}
\end{document}