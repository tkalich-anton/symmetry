\documentclass[12pt, a4paper]{article}
\usepackage{../../../../../style}
\begin{document}
	\lhead{Группа 102} \chead{Модуль 7} \rhead{Школа <<Симметрия>>} \cfoot{}
	\begin{center}
		\Large
		\textbf{Домашняя работа}
	\end{center}
	\begin{enumerate}[label=\textbf{\arabic*.}]
		\item Решить уравнения:
		\begin{multicols}{2}
			\begin{enumerate}[label=\asbuk*)]
				\item \( \tg\left( x+\dfrac{\pi}{4} \right) = 0 \)
				\item \( \tg\left( 2x-\dfrac{\pi}{3} \right)=\dfrac{\sqrt{3}}{3} \)
				\item \( \ctg^2 x = 2 \)
				\item \( \tg^2 x - \sqrt{3}\tg x = 0 \)
			\end{enumerate}
		\end{multicols}
		\item Решить уравнения:
		\begin{multicols}{2}
			\begin{enumerate}[label=\asbuk*)]
				\item \( \sqrt{3}\sin x + \cos x = 0 \)
				\item \( 5\sin x + \cos x = 0 \)
				\item \( \sin^2 x + 3\sin x\cos x - 4\cos^2 x = 0\)
				\item \( \sqrt{2}\cos^2x=\sin\left( \dfrac{\pi}{2}+x \right) \)
			\end{enumerate}
		\end{multicols}
		\item В банк помещена сумма 2 800 000 рублей под 50\% годовых. В конце каждого из первых двух лет хранения после начисления процентов вкладчик дополнительно вносил на счет одну и ту же фиксированную сумму. К концу третьего года после начисления процентов оказалось, что размер вклада увеличился по сравнению с первоначальным на 275\%. Какую сумму вкладчик ежегодно добавлял к вкладу?
		\item По вкладу «А» банк в конце каждого года планирует увеличивать на \( 10\% \) сумму, имеющуюся на вкладе в начале года, а по вкладу «Б» — увеличивать эту сумму на \( 5\% \) в первый год и на одинаковое целое число \( n \) процентов и за второй, и за третий годы. Найдите наименьшее значение \( n \), при котором за три года хранения вклад «Б» окажется выгоднее вклада «А» при одинаковых суммах первоначальных взносов.
		\item Жанна взяла в банке в кредит 1,2 млн рублей на срок 18 месяцев. По договору Жанна должна вносить в банк часть денег в конце каждого месяца. Каждый месяц общая сумма долга возрастает на 2\%, а затем уменьшается на сумму, уплаченную Жанной банку в конце месяца. Суммы, выплачиваемые Жанной, подбираются так, чтобы сумма долга уменьшалась равномерно, то есть на одну и ту же величину каждый месяц. Какую сумму Жанна выплатит банку в течение первого года кредитования?
	\end{enumerate}
\end{document}