\documentclass[12pt, a4paper]{article}
\usepackage{cmap} % Улучшенный поиск русских слов в полученном pdf-файле
\usepackage[T2A]{fontenc} % Поддержка русских букв
\usepackage[utf8]{inputenc} % Кодировка utf8
\usepackage[english, russian]{babel} % Языки: русский, английский
\usepackage{enumitem}
\usepackage{pscyr} % Нормальные шрифты
\usepackage{soulutf8}
\usepackage{amsmath}
\usepackage{amsthm}
\usepackage{amssymb}
\usepackage{scrextend}
\usepackage{titling}
\usepackage{indentfirst}
\usepackage{cancel}
\usepackage{soulutf8}
\usepackage{wrapfig}
\usepackage{gensymb}
\usepackage[dvipsnames,table,xcdraw]{xcolor}
\usepackage{tikz}

%Русские символы в списке
\makeatletter
\AddEnumerateCounter{\asbuk}{\russian@alph}{щ}
\makeatother

%Дублирование знаков при переносе
\newcommand*{\hm}[1]{#1\nobreak\discretionary{}%
	{\hbox{$\mathsurround=0pt #1$}}{}}

\usepackage{graphicx}
\graphicspath{{pic/}}
\DeclareGraphicsExtensions{.pdf,.png,.jpg}

%Изменеие параметров листа
\usepackage[left=15mm,right=15mm,
top=2cm,bottom=2cm,bindingoffset=0cm]{geometry}


\usepackage{fancyhdr}
\pagestyle{fancy}
\usepackage{multicol}

\setlength\parindent{1,5em}
\usepackage{indentfirst}

\begin{document}
	
	\lhead{Группа 102}
	\chead{Модуль 3 ДЗ№2}
	\rhead{Школа <<Симметрия>>}
	\begin{enumerate}
		\item \textit{(1 балл)} Постройте сечение треугольной пирамиды $SABC$ плоскостью, проходящей через вершину $B$ и середины ребер $SA$ и $SC$.
		\item \textit{(1 балл)} Постройте сечение треугольной пирамиды $SABC$ плоскостью, проходящей через вершину $A$, середину ребра $BS$ параллельно ребру $BC$.
		\item \textit{(1 балл)} Постройте сечение треугольной пирамиды $SABC$ плоскостью, проходящей через точку $M$, лежащую на ребре $AS$, $N$, лежащую на ребре $CS$, $K$ лежащую на ребре $BC$. Если $AM:MS=1:2, SN:NC=1:2, BK:KC=1:2$
		\item \textit{(1 балл)} Постройте сечение параллелепипеда $ABCDA_1B_1C_1D_1$ плоскостью, проходящей середины рёбер $A_1B_1$, $CC_1$ и вершину $A$.
		\item \textit{(2 балла)} Основание пирамиды $SABCD$ — параллелограмм $ABCD$. Постройте сечение пирамиды плоскостью, проходящей середину ребра $SA$ и точки $M$ и $N$ рёбер $SB$ и $SC$, если $BM:MS=SN:NC =1:3$.
		\item \textit{(2 балла)} Основания шестиугольной призмы $ABCDEF$ и $A_1B_1C_1E_1F_1$—правильные шестиугольники. Точка $M$ —середина ребра $CC_1$, $O$ —
		центр грани $ABCDEF$. Постройте сечение призмы плоскостью, проходящей через точки $M$, $O$ и $E_1$.
		\item \textit{(2 балла)} Дана правильная четырёхугольная пирамида $SABCD$ с вершиной $S$. Точки $M$ и $N$ —середины рёбер $AB$ и $SC$.\\
		а) Постройте сечение пирамиды плоскостью, проходящей через прямую $MN$ параллельно $SA$.\\
		б) Найдите угол между прямыми $SA$ и $MN$, если боковое ребро пирамиды равно стороне основания.
	\end{enumerate}
\end{document}