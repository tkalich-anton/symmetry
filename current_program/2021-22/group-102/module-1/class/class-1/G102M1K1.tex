\documentclass[12pt, a4paper]{article}
\usepackage{cmap} % Улучшенный поиск русских слов в полученном pdf-файле
\usepackage[T2A]{fontenc} % Поддержка русских букв
\usepackage[utf8]{inputenc} % Кодировка utf8
\usepackage[english, russian]{babel} % Языки: русский, английский
\usepackage{enumitem}
\usepackage{pscyr} % Нормальные шрифты

%Русские символы в списке
\makeatletter
\AddEnumerateCounter{\asbuk}{\russian@alph}{щ}
\makeatother
%Дублирование знаков при переносе
\newcommand*{\hm}[1]{#1\nobreak\discretionary{}%
	{\hbox{$\mathsurround=0pt #1$}}{}}

%Изменеие параметров листа
\usepackage[left=15mm,right=15mm,
top=2cm,bottom=2cm,bindingoffset=0cm]{geometry}

\usepackage{amsmath,amsthm,amssymb,scrextend}
\usepackage{fancyhdr}
\pagestyle{fancy}
\usepackage{multicol}

\usepackage{indentfirst}

\begin{document}
	
\lhead{Группа 102}
\chead{Модуль 1 Урок 1}
\rhead{Школа <<Симметрия>>}

\section*{Уравнения}

\subsection*{Виды алгебраических выражений}

\begin{enumerate}
	\item Рациональные выражения
	\begin{itemize}
		\item \textbf{Целые выражения} — это числа, переменные, а также всевозможные выражения, составленные из них при помощи действий сложения, вычитания, умножения и возведения в степень, которые также могут содержать скобки и деление на отличное от нуля число.
		
		\textit{Примеры:}
		\begin{center}
			$\dfrac{3}{7}$ ; $40x^2y$ ; $\dfrac{7y^2x^3}{5}$
		\end{center}
		\item \textbf{Дробные выражения} — отношение двух, как правило, целых алгебраических выражений.
	
		\textit{Примеры:}
		\begin{center}
			$\dfrac{45}{x+4}+4$ ; $\dfrac{12x^2-4}{(x+y)^3}$ ; $\dfrac{5}{x}$
		\end{center}
	\end{itemize}
	\item \textbf{Иррациональные выражения} — выражения, которые содержат алгебраические корни с переменной в подкоренном выражении.
	
	\textit{Примеры:}
	\begin{center}
		$3+\sqrt{x+1}$ ; $\dfrac{1}{\sqrt{9-x^2}}$ ; $\sqrt{3+x}-\sqrt{3-x}$
	\end{center}
	
	\item \textbf{Тригонометрические выражения} — выражения, которые содержат переменную в аргументе тригонометрической функции.
	
	\textit{Примеры:}
	\begin{center}
		$\sin{x}$ ; $\sin^2{3x}+3\cos^2{3x}$ ; $\dfrac{1}{\sin{2x}}+\dfrac{1}{\cos{2x}}$
	\end{center}

	\item \textbf{Показательные выражения} — выражения, которые содержат переменную в показатели степени.
	
	\textit{Примеры:}
	\begin{center}
		$13-2^x$ ; $5^{x+1}$ ; $2^{2x}-2^x+12$
	\end{center}

	\item \textbf{Логарифмические выражения} — выражения, которые содержат переменную в аргументе или в основании логарифмической функции.
	
	\textit{Примеры:}
	\begin{center}
		$\log_{3x} {5}-\log_{3x} {7}$ ; $1-4\log^2_2 {4x+1}$
	\end{center}

	\item \textbf{Смешанные выражения} — выражения, которые являются "смесью"\ нескольких видов алгебраических выражений.
	
	\textit{Примеры:}
	\begin{center}
		$\log_{3x} {5}(2-\sin{x})$ ; $\dfrac{3+x}{\sqrt{3+x}}+3^{3+x}$
	\end{center}
\end{enumerate}

Все выражения, в которых присутствуют корни, тригонометрические, показательные и логарифмические функции, но в аргументе этих функций отсутствует переменная, будут являться рациональными.

\textit{Примеры рациональных выражений:}
\begin{center}
	$3x+\sqrt{100}$ ; $3x^2+5x-\sin{\dfrac{\pi}{4}}$ ; $\dfrac{12}{\log_2 8}$ ; $5x + 4^{\sqrt{16}}$
\end{center}

Уравнения и неравенства будут называться рациональными (целыми или дробными), иррациональными, тригонометрическими, показательными, логарифмическими или смешанными в зависимости от того, какие выражения они содержат.

\subsection*{Решение уравнений}

Главным этапом решения любого уравнения является свед\'eние его к одному или нескольким простейшим уравнениям. Под простейшими уравнениями, в зависимости от принадлежности к той тому или иному виду, подразумевают:
\begin{itemize}
	\item для целых рациональных уравнений — линейные и квадратные уравнения;
	\item для дробно-рациональных уравнений — уравнения вида $\frac{f(x)}{g(x)}=0$, где $f(x)$ и $g(x)$ — многочлены первой или второй степени;
	\item для иррациональных уравнений—уравнения вида $\sqrt{f(x)}=g(x)$, где $f(x)$ — многочлен первой или второй степени, $g(x)$ — многочлен
	степени не выше первой;
	\item для тригонометрических уравнений—уравнения вида $\sin x =a$,
	$\cos x=a$, $\tg x =a$, где $a$ — действительное число;
	\item для показательных уравнений — уравнения вида $a^{f(x)} = b$, где
	$a$ — положительное действительное число, отличное от единицы, $b$ —
	действительное число, $f(x)$ — многочлен первой или второй степени;
	\item  для логарифмических уравнений — уравнения вида $log_a {f(x)}=b$,
	где $b$ — действительное число, $a$ — положительное действительное
	число, отличное от 1, $f(x)$ — многочлен первой или второй степени.
\end{itemize}

Существует два основных способа сведения уравнения к одному или нескольким простейшим: \underline{алгебраические преобразования} и \underline{замена переменной}. Иногда уравнения будут решаться с помощью применения таких свойств функций, как монотонность и ограниченность.

\subsection*{Целые рациональные уравнения}
Рассмотрим самые основные способы решения целых рациональных уравнений. Конечно, простейшие виды целых уравнений, такие как линейные и квадратные, мы рассматривать не будем, а перейдем сразу к более сложным случаям.
\begin{enumerate}
	\item Условие равенства произведения нулю
	$$f(x)\cdot g(x) = 0
	\Leftrightarrow 
	\left[
	\begin{aligned}
		f(x)&=0,\\
		g(x)&=0.
	\end{aligned}
	\right.$$
	
	В данном методе наша ключевая задача заключается в том, чтобы разложить выражение на множители, предварительно не забыв перенести все слагаемые из правой части в левую и, после разложения на множители, приравнять каждый из множителей к нулю.
	
	\textit{Пример:}
	
	Решить уравнение $5x^3-21x^2=21x-5$
	
	\textit{Решение:}
	
	Перенесем все слагаемые в левую сторону уравнения:
	$$5x^3-21x^2-21x+5=0$$
	Сгруппируем первое слагаемое с последним, а второе с третьим:
	$$5(x^3+1)-21x(x+1)=0$$
	Разложим $x^3+1$ по формуле суммы кубов:
	$$5(x+1)(x^2-x+1)-21x(x+1)=0$$
	Вынесем общий множитель $x+1$ за скобку:
	$$(x+1)(5(x^2-x++1)-21x)=0$$
	Применим условие равенства произведения нулю:
	$$(x+1)(5x^2-26x+5)=0\Leftrightarrow\\$$
	$$\Leftrightarrow 
	\left[
	\begin{array}{l}
		x+1=0,\\
		5x^2-26x+5=0.
	\end{array}
	\right.
	\Leftrightarrow
	\left[
	\begin{array}{l}
		x=-1,\\
		x=\dfrac{1}{5},\\
		x=5
	\end{array}
	\right.$$
		
	ОТВЕТ: $-1$ ; $\dfrac{1}{5}$ ; $5$
	\item Условие равенства степеней
	$$(f(x))^{2n+1}=(g(x))^{2n+1}\Leftrightarrow f(x)=g(x)$$
	$$(f(x))^{2n}=(g(x))^{2n}
	\Leftrightarrow 
	\left[
	\begin{aligned}
		f(x)&=g(x),\\
		f(x)&=-g(x).
	\end{aligned}
	\right.$$
	
	\textit{Пример:}
	
	Решить уравнение $(x-3)^6=(-x^2-2x+1)^3$
	
	\textit{Решение:}
	
	Воспользуемся свойством степени $(a^b)^c=a^{b\cdot c}$:
	$$((x-3)^2)^3=(-x^2-2x+1)^3$$
	Применим условие равенства степеней:
	$$((x-3)^2)^3=(-x^2-2x-1)^3\Leftrightarrow (x-3)^2=-x^2-2x+1$$
	Раскроем $(x-3)^2$ по формуле квадрата разности:
	$$x^2-6x+9=-x^2-2x+1$$
	Решим обыкновенное квадратное уравнение.
	
	ОТВЕТ: $2$
	\item Замена переменной
	
	\textit{Пример:}
	
	Решить уравнение $(4x-9)^3=(4x-9)^5$
	
	\textit{Решение:}
	
	Пусть $4x-9 = t$. Тогда:
	$$t^3=t^5$$
	$$t^3-t^5=0$$
	Вынесем общий множитель за скобку:
	$$t^3(1-t^2)=0
	\Leftrightarrow
	\left[
	\begin{array}{l}
		t^3=0,\\
		1-t^2=0.\\
	\end{array}
	\right.
	\Leftrightarrow
	\left[
	\begin{array}{l}
		t=0,\\
		t^2=1.\\
	\end{array}
	\right.
	\Leftrightarrow
	\left[
	\begin{array}{l}
		t=0,\\
		t=1,\\
		t=-1
	\end{array}
	\right.$$
	Давайте сделаем обратную замену и получим:
	$$\left[
	\begin{array}{l}
		4x-9=0,\\
		4x-9=1,\\
		4x-9=-1
	\end{array}
	\right.
	\Leftrightarrow
	\left[
	\begin{array}{l}
		x=\dfrac{9}{4}=2,25,\\\\
		x=\dfrac{10}{4}=2,5,\\\\
		x=\dfrac{8}{4}=2
	\end{array}
	\right.$$
	
	ОТВЕТ: $2$ ; $2,25$ ; $2,5$
\end{enumerate}
\end{document}


