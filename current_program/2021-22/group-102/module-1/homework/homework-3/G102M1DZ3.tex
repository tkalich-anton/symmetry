\documentclass[12pt, a5paper]{article}
\usepackage{cmap} % Улучшенный поиск русских слов в полученном pdf-файле
\usepackage[T2A]{fontenc} % Поддержка русских букв
\usepackage[utf8]{inputenc} % Кодировка utf8
\usepackage[english, russian]{babel} % Языки: русский, английский
\usepackage{enumitem}
\usepackage{pscyr} % Нормальные шрифты


\usepackage{amsmath,amsthm,amssymb,scrextend, cancel}
\usepackage[dvipsnames,table,xcdraw]{xcolor}
\usepackage{fancyhdr}
\usepackage{multicol}
\usepackage{indentfirst}
\usepackage{graphicx}

%Изменеие параметров листа
\usepackage[left=10mm,right=10mm,
top=2cm,bottom=2cm,bindingoffset=0cm]{geometry}

%Русские символы в списке
\makeatletter
\AddEnumerateCounter{\asbuk}{\russian@alph}{щ}
\makeatother
%Дублирование знаков при переносе
\newcommand*{\hm}[1]{#1\nobreak\discretionary{}%
	{\hbox{$\mathsurround=0pt #1$}}{}}

\setlength\parindent{1,5em}
\setlength{\parskip}{0cm}
\pagestyle{fancy}

\begin{document}
		
\lhead{Группа 101}
\chead{Модуль 1 Урок 8}
\rhead{<<Симметрия>>}

\section*{Решение рациональных и иррациональных уравнений}
\textit{(2 балла)} Решить уравнения:
	\begin{enumerate}
		\item $\sqrt{7x+2}=5x-2$
		\item $10\sqrt{x^2-x-1}=13-\dfrac{3}{\sqrt{x^2-x-1}}$
		\item $x^2+\sqrt{x^2-x+9}=x+3$
		\item $\sqrt{4x+8}-\sqrt{3x-2}=2$
		\item $\left\{
		\begin{aligned}
			\dfrac{(x+5)(x-6)}{y-5}&=0,\\
			\dfrac{(y-5)(x+6)}{x-6}&=0
		\end{aligned}
		\right.$
		
	\end{enumerate}

\end{document}