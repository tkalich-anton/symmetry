\documentclass[12pt, a4paper]{article}
\usepackage{../../../../../style}
\begin{document}
	
	\lhead{Группа 92} \chead{Модуль 6} \rhead{Школа <<Симметрия>>} \cfoot{}
	\begin{center}
		\Large
		\textbf{Занятие №8}
	\end{center}
	\begin{enumerate}[label=\textbf{\arabic*.}]
		\item Решите уравнение: $$\dfrac{7}{x+1}-\dfrac{x+4}{2-2x}=\dfrac{3x^2-38}{x^2-1}$$
		\item Решите уравнение: $$x^2-|x-5|=5$$
		\item Решите уравнение: $$\left( \dfrac{4x-5}{3x+2} \right)^2+\left( \dfrac{3x+2}{5-4x} \right)^2=4,25$$
		\item Население города за \( 2 \) года увеличилось с \( 20 000 \) до \( 22 050 \) человек. Найдите средний ежегодный процент роста населения этого города.
		\item Два тракториста могут вспахать поле на 18 ч быстрее, чем один первый тракторист, и на 32 часа быстрее, чем один второй. За сколько часов может вспахать поле каждый тракторист, работая отдельно?
		\item Имелось два слитка меди. Процент содержания меди в первом слитке был на 40 меньше, чем процент содержания меди во втором слитке. После того, как оба слитка сплавили, получили слиток, содержащий 36\% меди. Найдите процентное содержание меди в первом и во втором слитках, если в первом слитке было 6 кг меди, а во втором --- 12 кг.
		\item По окружности движутся два тела: первое тело проходит круг на 2 с быстрее второго. Если тела движутся в одном направлении, то они встречаются через каждые 60 с. Какую часть окружности проходит каждое тело за 1 с?
	\end{enumerate}
\end{document}