\documentclass[12pt, a4paper]{article}
\usepackage{cmap} % Улучшенный поиск русских слов в полученном pdf-файле
\usepackage[T2A]{fontenc} % Поддержка русских букв
\usepackage[utf8]{inputenc} % Кодировка utf8
\usepackage[english, russian]{babel} % Языки: русский, английский
\usepackage{enumitem}
\usepackage{pscyr} % Нормальные шрифты
\usepackage{amsmath}
\usepackage{amsthm}
\usepackage{amssymb}
\usepackage{scrextend}
\usepackage{titling}
\usepackage{indentfirst}
\usepackage{cancel}
\usepackage{soulutf8}
\usepackage{wrapfig}
\usepackage{gensymb}
\usepackage[dvipsnames,table,xcdraw]{xcolor}
\usepackage{tikz}

%Русские символы в списке
\makeatletter
\AddEnumerateCounter{\asbuk}{\russian@alph}{щ}
\makeatother

%Дублирование знаков при переносе
\newcommand*{\hm}[1]{#1\nobreak\discretionary{}%
	{\hbox{$\mathsurround=0pt #1$}}{}}

\usepackage{graphicx}
\graphicspath{{pic/}}
\DeclareGraphicsExtensions{.pdf,.png,.jpg}

%Изменеие параметров листа
\usepackage[left=15mm,right=15mm,
top=2cm,bottom=2cm,bindingoffset=0cm]{geometry}

\usepackage{fancyhdr}
\pagestyle{fancy}
\usepackage{multicol}
\setlength\parindent{1,5em}
\usepackage{indentfirst}
\begin{document}
	
	\lhead{Группа 92}
	\chead{Модуль 6 Урок 4}
	\rhead{Школа <<Симметрия>>}
	\begin{enumerate}
		\item В сосуд, содержащий $5$ литров $12$–процентного водного раствора некоторого вещества, добавили $7$ литров воды. Сколько процентов составляет концентрация получившегося раствора?
		\item Смешали некоторое количество $15\%$-го раствора некоторого вещества с таким же количеством $19\%$-го раствора этого вещества. Сколько процентов составляет концентрация получившегося раствора?
		\item Смешали $4$ литра $15$–процентного водного раствора некоторого вещества с $6$ литрами $25$–процентного водного раствора этого же вещества. Сколько процентов составляет концентрация получившегося раствора?
		\item Изюм получается в процессе сушки винограда. Сколько килограммов винограда потребуется для получения $20$ килограммов изюма, если виноград содержит $90\%$ воды, а изюм содержит $5\%$ воды?
		\item Имеется два сплава. Первый содержит $10\%$ никеля, второй – $30\%$ никеля. Из этих двух сплавов получили третий сплав массой $200$ кг, содержащий $25\%$ никеля. На сколько килограммов масса первого сплава была меньше массы второго?
		\item Имеется два сплава. Первый сплав содержит $10\%$ меди, второй – $40\%$. Масса второго сплава больше массы первого на $3$ кг. Из этих двух сплавов получили третий сплав, содержащий $30\%$ меди. Найдите массу третьего сплава. Ответ дайте в килограммах.
		\item Смешав $30$-процентный и $60$-процентный растворы кислоты и добавив $10$ кг чистой воды, получили 36-процентный раствор кислоты. Если бы вместо $10$ кг воды добавили $10$ кг $50$-процентного раствора той же кислоты, то получили бы $41$-процентный раствор кислоты. Сколько килограммов $30$-процентного раствора использовали для получения смеси?
		\item Имеются два сосуда. Первый содержит $30$ кг, а второй – $20$ кг раствора кислоты различной концентрации. Если эти растворы смешать, то получится раствор, содержащий $68\%$ кислоты. Если же смешать равные массы этих растворов, то получится раствор, содержащий $70\%$ кислоты. Сколько килограммов кислоты содержится в первом сосуде?
		\item Имеется два сплава. Первый содержит $15\%$ никеля, второй – $35\%$ никеля. Из этих двух сплавов получили третий сплав массой $140$ кг, содержащий $30\%$ никеля. На сколько килограммов масса первого сплава была меньше массы второго?
	\end{enumerate}
\end{document}