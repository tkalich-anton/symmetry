\documentclass[12pt, a4paper]{article}
\usepackage{../../../../../style}
\begin{document}
	
	\lhead{Группа 92}
	\chead{Модуль 6 Урок 3}
	\rhead{Школа <<Симметрия>>}
	\begin{enumerate}
		\item \subimport{../../../../../exercises/arithmetic/coordinate_line/roots}{ex_2_2}
		\item \subimport{../../../../../exercises/algebra/expressions/fractions}{ex_1}
		\item Имеются два сосуда, содержащие 10 кг и 16 кг раствора кислоты различной концентрации. Если их слить вместе, то получится раствор, содержащий 55\% кислоты. Если же слить равные массы этих растворов, то полученный раствор будет содержать 61\% кислоты. Сколько килограммов кислоты содержится в первом растворе?
		\item Смешав $60\%$-ый и $30\%$-ый растворы кислоты и добавив $5$ кг чистой воды, получили $20\%$-ый раствор кислоты. Если бы вместо $5$ кг воды добавили $5$ кг $90\%$-го раствора той же кислоты, то получили бы $70\%$-ый раствор кислоты. Сколько килограммов $60\%$-го раствора использовали для получения смеси?
		\item Свежие фрукты содержат $80\%$ воды а высушенные --- $28\%$. Сколько сухих фруктов получится из $288$ кг свежих фруктов?
		\item Из пунктов $А$ и $В$, расстояние между которыми $19$ км, вышли одновременно навстречу друг другу два пешехода и встретились в $9$ км от $А$. Найдите скорость пешехода, шедшего из $А$, если известно, что он шёл со скоростью, на $1$ км/ч большей, чем пешеход, шедший из $В$, и сделал в пути получасовую остановку.
		\item Расстояние между городами $A$ и $B$ равно $750$ км. Из города $A$ в город $B$ со скоростью $50$ км/ч выехал первый автомобиль, а через три часа после этого навстречу ему из города $B$ выехал со скорость 70 км/ч второй автомобиль. На каком расстоянии от города от $A$ автомобили встретятся?
	\end{enumerate}
\end{document}