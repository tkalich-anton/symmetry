\documentclass[12pt, a4paper]{article}
\usepackage{../../../../../style}
\begin{document}
	\lhead{Группа 92} \chead{Модуль 7 Домашняя работа 1} \rhead{Школа <<Симметрия>>} \cfoot{}
	\begin{enumerate}[label=\textbf{\arabic*.}]
		\item \textit{(2б.)} Решите уравнения:
		\begin{enumerate}[label=\asbuk*)]
			\begin{multicols}{2}
				\item $\dfrac{1}{x-1}+\dfrac{2}{1-x^2}=\dfrac{5}{x^2+2x+1}$
				\item $\dfrac{21}{x}-\dfrac{10}{x-2}-\dfrac{4}{x-3}=0$
			\end{multicols}
		\end{enumerate}
		\item \textit{(2б.)} Упростите выражение: \\
		$\dfrac{x\sqrt{x}-1}{x-4\sqrt{x}+3}-\dfrac{\sqrt{x}+10}{\sqrt{x}-3}$
		\item \textit{(3б.)} Постройте график функции $y=x^2-3|x|-x$  и определите, при каких значениях  $c$  прямая  $y=c$  имеет с графиком три общие точки.
		\item \textit{(3б.)} Постройте график функции $y=\dfrac{x^4-13x^2+36}{(x-3)(x+2)}и$  и определите, при каких значениях параметра $c$ прямая $y=c$ имеет с графиком ровно одну общую точку.
	\end{enumerate}
\end{document}