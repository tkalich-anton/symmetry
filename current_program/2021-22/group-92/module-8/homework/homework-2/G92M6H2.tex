\documentclass[12pt, a4paper]{article}
\usepackage{cmap} % Улучшенный поиск русских слов в полученном pdf-файле
\usepackage[T2A]{fontenc} % Поддержка русских букв
\usepackage[utf8]{inputenc} % Кодировка utf8
\usepackage[english, russian]{babel} % Языки: русский, английский
\usepackage{enumitem}
\usepackage{pscyr} % Нормальные шрифты
\usepackage{soulutf8}
\usepackage{amsmath}
\usepackage{amsthm}
\usepackage{amssymb}
\usepackage{scrextend}
\usepackage{titling}
\usepackage{indentfirst}
\usepackage{cancel}
\usepackage{soulutf8}
\usepackage{wrapfig}
\usepackage{gensymb}
\usepackage[dvipsnames,table,xcdraw]{xcolor}
\usepackage{tikz}

%Русские символы в списке
\makeatletter
\AddEnumerateCounter{\asbuk}{\russian@alph}{щ}
\makeatother

%Дублирование знаков при переносе
\newcommand*{\hm}[1]{#1\nobreak\discretionary{}%
	{\hbox{$\mathsurround=0pt #1$}}{}}

\usepackage{graphicx}
\graphicspath{{pic/}}
\DeclareGraphicsExtensions{.pdf,.png,.jpg}

%Изменеие параметров листа
\usepackage[left=15mm,right=15mm,
top=2cm,bottom=2cm,bindingoffset=0cm]{geometry}


\usepackage{fancyhdr}
\pagestyle{fancy}
\usepackage{multicol}

\setlength\parindent{1,5em}
\usepackage{indentfirst}

\begin{document}
	
	\lhead{Группа 92}
	\chead{Модуль 6 Домашняя работа 2}
	\rhead{Школа <<Симметрия>>}
	
	\begin{enumerate}
		\item \textit{(2 балла)} В сосуд, содержащий $8$ литров $24$-процентного водного раствора некоторого вещества, добавили $4$ литра воды. Сколько процентов составляет концентрация получившегося раствора?
		\item \textit{(2 балла)} Смешали некоторое количество $16$-процентного раствора некоторого вещества с таким же количеством $12$-процентного раствора этого вещества. Сколько процентов составляет концентрация получившегося раствора?
		\item \textit{(2 балла)} Имеется два сплава. Первый сплав содержит $5\%$ меди, второй – $12\%$ меди. Масса второго сплава больше массы первого на $9$ кг. Из этих двух сплавов получили третий сплав, содержащий $10\%$ меди. Найдите массу третьего сплава. Ответ дайте в килограммах.
		\item \textit{(2 балла)} Изюм получается в процессе сушки винограда. Сколько килограммов винограда потребуется для получения $12$ килограммов изюма, если виноград содержит $90\%$ воды, а изюм содержит $5\%$ воды?
		\item \textit{(2 балла)} Имеются два сосуда. Первый содержит $30$ кг, а второй – $15$ кг раствора кислоты различной концентрации. Если эти растворы смешать, то получится раствор, содержащий $34\%$ кислоты. Если же смешать равные массы этих растворов, то получится раствор, содержащий $46\%$ кислоты. Сколько килограммов кислоты содержится в первом сосуде?
	\end{enumerate}
\end{document}