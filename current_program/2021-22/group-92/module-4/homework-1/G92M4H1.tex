\documentclass[12pt, a4paper]{article}
\usepackage{cmap} % Улучшенный поиск русских слов в полученном pdf-файле
\usepackage[T2A]{fontenc} % Поддержка русских букв
\usepackage[utf8]{inputenc} % Кодировка utf8
\usepackage[english, russian]{babel} % Языки: русский, английский
\usepackage{enumitem}
\usepackage{pscyr} % Нормальные шрифты
\usepackage{soulutf8}
\usepackage{amsmath}
\usepackage{amsthm}
\usepackage{amssymb}
\usepackage{scrextend}
\usepackage{titling}
\usepackage{indentfirst}
\usepackage{cancel}
\usepackage{soulutf8}
\usepackage{wrapfig}
\usepackage{gensymb}
\usepackage[dvipsnames,table,xcdraw]{xcolor}
\usepackage{tikz}

%Русские символы в списке
\makeatletter
\AddEnumerateCounter{\asbuk}{\russian@alph}{щ}
\makeatother

%Дублирование знаков при переносе
\newcommand*{\hm}[1]{#1\nobreak\discretionary{}%
	{\hbox{$\mathsurround=0pt #1$}}{}}

\usepackage{graphicx}
\graphicspath{{pic/}}
\DeclareGraphicsExtensions{.pdf,.png,.jpg}

%Изменеие параметров листа
\usepackage[left=15mm,right=15mm,
top=2cm,bottom=2cm,bindingoffset=0cm]{geometry}


\usepackage{fancyhdr}
\pagestyle{fancy}
\usepackage{multicol}

\setlength\parindent{1,5em}
\usepackage{indentfirst}

\begin{document}
	
	\lhead{Группа 92}
	\chead{Модуль 4 ДЗ№1}
	\rhead{Школа <<Симметрия>>}
	\begin{enumerate}
		\item \textit{(2 балла)} Упростить выражение $$\left(\dfrac{y}{x-y}-\dfrac{x}{x+y}\right)\cdot\left(\dfrac{x^2}{y^2}+\dfrac{y^2}{x^2}-2\right)$$
		\item \textit{(1 балл)} Решить уравнение $$(x^2+3x+1)(x^2+3x+3)+1=0$$
		\item \textit{(1 балл)} Решить систему неравенств 
		$$
		\left\{
		\begin{array}{l}
			\dfrac{2x+5}{5}>\dfrac{5x+2}{2}\\\\
			\dfrac{x+2}{5}<\dfrac{x+5}{2}\\
		\end{array}
		\right.
		$$
		\item \textit{(2 балла)} Решить систему неравенств 
		$$
		\left\{
		\begin{array}{l}
			x^2-14x+45 < 0\\
			x^2-11x+30 > 0\\
			\dfrac{2x-3}{x^2-x+2}>0
		\end{array}
		\right.
		$$
		\item \textit{(1 балл)} Биссектриса угла параллелограмма делит сторону параллелограмма на отрезки, равные $3$ и $12$. Найдите стороны параллелограмма.
		\item \textit{(1 балл)} Диагонали прямоугольника равны 8 и пересекаются под углом в $60\degree$. Найдите меньшую сторону прямоугольника.
		\item \textit{(2 балла)} Два угла треугольника равны $10\degree$ и $60\degree$. Найдите угол между высотой и биссектрисой, проведенными из вершины третьего угла треугольника.
	\end{enumerate}
\end{document}