\documentclass[12pt, a5paper]{article}
\usepackage{cmap} % Улучшенный поиск русских слов в полученном pdf-файле
\usepackage[T2A]{fontenc} % Поддержка русских букв
\usepackage[utf8]{inputenc} % Кодировка utf8
\usepackage[english, russian]{babel} % Языки: русский, английский
\usepackage{enumitem}
\usepackage{pscyr} % Нормальные шрифты


\usepackage{amsmath,amsthm,amssymb,scrextend, cancel}
\usepackage[dvipsnames,table,xcdraw]{xcolor}
\usepackage{fancyhdr}
\usepackage{multicol}
\usepackage{indentfirst}
\usepackage{graphicx}

%Изменеие параметров листа
\usepackage[left=10mm,right=10mm,
top=2cm,bottom=2cm,bindingoffset=0cm]{geometry}

%Русские символы в списке
\makeatletter
\AddEnumerateCounter{\asbuk}{\russian@alph}{щ}
\makeatother
%Дублирование знаков при переносе
\newcommand*{\hm}[1]{#1\nobreak\discretionary{}%
	{\hbox{$\mathsurround=0pt #1$}}{}}

\setlength\parindent{1,5em}
\setlength{\parskip}{0cm}
\pagestyle{fancy}

\begin{document}
		
\lhead{Группа 71}
\chead{Модуль 1 Д/З №3}
\rhead{<<Симметрия>>}

\section*{Решение линейных уравнений}
\textbf{Задание №1} \textit{(2 балла)} Решить уравнения:
\begin{enumerate}[label=\asbuk*)]
	\item $3,8x-(1,6-1,2x)=9,6+(3,7-5x)$
	\item $0,15(x-4)=9,9-0,3(x-1)$
	\item $\dfrac{1-y}{7}+y=\dfrac{y}{2}+3$
\end{enumerate}


\textbf{Задание №2} \textit{(2 балла)} Можно ли 59 банок варенья разложить в три ящика так, чтобы в третьем было на 9 банок больше, чем в первом, а во втором — на 4 банки меньше, чем в третьем?

\textbf{Задание №3} \textit{(2 балла)} В корзине было в 2 раза меньше винограда, чем в ящике. После того как в корзину добавили 2 кг, в ней стало винограда на 0,5 кг больше, чем в ящике. Сколько винограда было в корзине?

\end{document}