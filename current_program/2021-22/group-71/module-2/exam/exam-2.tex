\documentclass[12pt, a4paper]{article}
\usepackage{cmap} % Улучшенный поиск русских слов в полученном pdf-файле
\usepackage[T2A]{fontenc} % Поддержка русских букв
\usepackage[utf8]{inputenc} % Кодировка utf8
\usepackage[english, russian]{babel} % Языки: русский, английский
\usepackage{enumitem}
\usepackage{pscyr} % Нормальные шрифты
\usepackage{soulutf8}
\usepackage{amsmath}
\usepackage{amsthm}
\usepackage{amssymb}
\usepackage{scrextend}
\usepackage{titling}
\usepackage{indentfirst}
\usepackage{cancel}
\usepackage{soulutf8}
\usepackage{wrapfig}
\usepackage{gensymb}
\usepackage[dvipsnames,table,xcdraw]{xcolor}
\usepackage{tikz}
\usepackage{multicol}

%Русские символы в списке
\makeatletter
\AddEnumerateCounter{\asbuk}{\russian@alph}{щ}
\makeatother

%Дублирование знаков при переносе
\newcommand*{\hm}[1]{#1\nobreak\discretionary{}%
	{\hbox{$\mathsurround=0pt #1$}}{}}

\usepackage{graphicx}
\graphicspath{{pic/}}
\DeclareGraphicsExtensions{.pdf,.png,.jpg}

%Изменеие параметров листа
\usepackage[left=15mm,right=15mm,
top=1cm,bottom=2cm,bindingoffset=0cm]{geometry}

%\usepackage{fancyhdr}
%\pagestyle{fancy}

\setlength\parindent{1,5em}
\usepackage{indentfirst}

\begin{document}
\section*{Билет 1}
\begin{enumerate}
	\item \textit{(1 балл)} Что такое угол? Назовите виды углов.
	\item \textit{(1 балл)} Что такое перпендикуляр к прямой? Что является основанием перпендикуляра? Сколько различных перпендикуляров можно опустить из точки на прямую?
	\item \textit{(2 балла)} Что такое прямоугольный треугольник? Как называются стороны прямоугольного треугольника? Сформулировать признаки равенства прямоугольных треугольников.
	\item \textit{(2 балла)} Что значит, что окружность вписана в треугольник? Где лежит центр окружности, вписанной в треугольник?
	\item \textit{(2 балла)} Что такое секущая к окружности? Как называется часть секущей, заключенная в окружности? Что такое касательная к окружности? Сколько можно провести касательных к окружности через точку, которая находится внутри окружности?
	\item \textit{(4 балла)} Медиана $AM$ треугольника $ABC$ перпендикулярна его биссектрисе $BK$. Найдите $AB$, если $BC = 10$.
	\item \textit{(4 балла)} Две высоты треугольника равны между собой. Докажите, что треугольник равнобедренный.
	\item \textit{(4 балла)} Биссектрисы $BB_1$ и $CC_1$ треугольника $ABC$ пересекаются в точке $M$, биссектрисы $B_1B_2$ и $C_1C_2$ треугольника $AB_1C_1$ пересекаются в точке $N$. Докажите, что точки $A$,$M$ и $N$ лежат на одной прямой.
	\item \textit{(Дополнительная задача)} Постройте биссектрису угла, вершина которого недоступна.
\end{enumerate}
\section*{Билет 2}
\begin{enumerate}
	\item \textit{(1 балл)} Что такое смежные углы? Каким свойством они обладают?
	\item \textit{(1 балл)} Что такое наклонная, проведенная из точки к прямой? В чем отличие наклонной от перпендикуляра?
	\item \textit{(2 балла)} Перечислите свойства равнобедренного треугольника. Перечислите свойства равностороннего треугольника.
	\item \textit{(2 балла)} Что значит, что окружность вписана в треугольник? Где лежит центр окружности, вписанной в треугольник?
	\item \textit{(2 балла)} Сформулируйте теорему о радиусе, проведенном через середину хорды. Верна ли обратная теорема?
	\item \textit{(4 балла)} Медиана треугольника делит его на два треугольника, периметры которых равны. Докажите, что треугольник равнобедренный.
	\item \textit{(4 балла)} Две высоты треугольника равны между собой. Докажите, что треугольник равнобедренный.
	\item \textit{(4 балла)} Биссектрисы $BB_1$ и $CC_1$ треугольника $ABC$ пересекаются в точке $M$, биссектрисы $B_1B_2$ и $C_1C_2$ треугольника $AB_1C_1$
	пересекаются в точке $N$. Докажите, что точки $A$,$ M$ и $N$ лежат на одной прямой.
	\item \textit{(Дополнительная задача)} Постройте биссектрису угла, вершина которого недоступна.
\end{enumerate}
\section*{Билет 3}
\begin{enumerate}
\item \textit{(1 балл)} Что такое вертикальные углы? Каким свойством они обладают?
\item \textit{(1 балл)} Что такое перпендикуляр к прямой? Что можно сказать о двух различных перпендикулярах, проведенных к одной прямой?
\item \textit{(2 балла)} Как называется множество точек, равноудаленных от сторон некоторого угла? Объясните почему.
\item \textit{(2 балла)} Где лежит центр окружности, описанной вокруг треугольника? Объясните почему.
\item \textit{(2 балла)} Что такое секущая к окружности? Как называется часть секущей, заключенная в окружности? Что такое касательная к окружности?
\item \textit{(4 балла)} Медиана $AM$ треугольника $ABC$ перпендикулярна его биссектрисе $BK$. Найдите $AB$, если $BC = 6$.
\item \textit{(4 балла)} Две высоты треугольника равны между собой. Докажите, что треугольник равнобедренный.
\item \textit{(4 балла)} Биссектрисы $BB_1$ и $CC_1$ треугольника $ABC$ пересекаются в точке $M$, биссектрисы $B_1B_2$ и $C_1C_2$ треугольника $AB_1C_1$ пересекаются в точке $N$. Докажите, что точки $A$,$M$ и $N$ лежат на одной прямой.
\item \textit{(Дополнительная задача)} Постройте биссектрису угла, вершина которого недоступна.
\end{enumerate}
\section*{Билет 4}
\begin{enumerate}
\item \textit{(1 балл)} Назовите виды треугольников. Перечислите свойства равностороннего треугольника.
\item \textit{(1 балл)} Что такое смежные углы? Каким свойством они обладают?
\item \textit{(2 балла)} Что такое прямоугольный треугольник? Как называются стороны прямоугольного треугольника? Сформулируйте признаки равенства прямоугольных треугольников.
\item \textit{(2 балла)} Как называется множество точек, равноудаленных от концов некоторого отрезка? Объясните почему.
\item \textit{(2 балла)} Что такое касательная к окружности? Сколько можно провести касательных к окружности через точку, которая находится вне окружности?
\item \textit{(4 балла)} Медиана треугольника делит его на два треугольника, периметры которых равны. Докажите, что треугольник равнобедренный.
\item \textit{(4 балла)} Две высоты треугольника равны между собой. Докажите, что треугольник равнобедренный.
\item \textit{(4 балла)} Биссектрисы $BB_1$ и $CC_1$ треугольника $ABC$ пересекаются в точке $M$, биссектрисы $B_1B_2$ и $C_1C_2$ треугольника $AB_1C_1$ пересекаются в точке $N$. Докажите, что точки $A$,$M$ и $N$ лежат на одной прямой.
\item \textit{(Дополнительная задача)} Постройте биссектрису угла, вершина которого недоступна.
\end{enumerate}
\section*{Билет 5}
\begin{enumerate}
\item \textit{(1 балл)} Что является расстоянием от точки до точки? Что является расстоянием от точки до прямой?
\item \textit{(1 балл)} Чему равен угол между биссектрисами двух смежных углов? Объясните почему.
\item \textit{(2 балла)} Что такое прямоугольный треугольник? Как называются стороны прямоугольного треугольника? Сформулируйте признаки равенства прямоугольных треугольников.
\item \textit{(2 балла)} Как называется множество точек, равноудаленных от концов некоторого отрезка? Объясните почему.
\item \textit{(2 балла)} Что такое окружность? Что такое секущая к окружности? Как называется часть секущей, заключенная в окружности? Сколько точек пересечения имеют окружность и секущая?
\item \textit{(4 балла)} Медиана $AM$ треугольника $ABC$ перпендикулярна его биссектрисе $BK$. Найдите $AB$, если $BC = 10$.
\item \textit{(4 балла)} Две высоты треугольника равны между собой. Докажите, что треугольник равнобедренный.
\item \textit{(4 балла)} Биссектрисы $BB_1$ и $CC_1$ треугольника $ABC$ пересекаются в точке $M$, биссектрисы $B_1B_2$ и $C_1C_2$ треугольника $AB_1C_1$ пересекаются в точке $N$. Докажите, что точки $A$,$M$ и $N$ лежат на одной прямой.
\item \textit{(Дополнительная задача)} Постройте биссектрису угла, вершина которого недоступна.
\end{enumerate}
\section*{Билет 6}
\begin{enumerate}
\item \textit{(1 балл)} Что такое угол? Назовите виды углов.
\item \textit{(1 балл)} Что является расстоянием от точки до прямой? Что является расстоянием от точки до прямой?
\item \textit{(2 балла)} Что такое прямоугольный треугольник? Как называются стороны прямоугольного треугольника? Сформулируйте признаки равенства прямоугольных треугольников.
\item \textit{(2 балла)} Как называется множество точек, равноудаленных от сторон некоторого угла? Объясните почему.
\item \textit{(2 балла)} Что значит, что окружность описанна вокруг треугольника? Где лежит центр этой окружности?
\item \textit{(4 балла)} Медиана треугольника делит его на два треугольника, периметры которых равны. Докажите, что треугольник равнобедренный.
\item \textit{(4 балла)} Две высоты треугольника равны между собой. Докажите, что треугольник равнобедренный.
\item \textit{(4 балла)} Биссектрисы $BB_1$ и $CC_1$ треугольника $ABC$ пересекаются в точке $M$, биссектрисы $B_1B_2$ и $C_1C_2$ треугольника $AB_1C_1$
пересекаются в точке $N$. Докажите, что точки $A$,$M$ и $N$ лежат на одной прямой.
\item \textit{(Дополнительная задача)} Постройте биссектрису угла, вершина которого недоступна.
\end{enumerate}
\end{document}