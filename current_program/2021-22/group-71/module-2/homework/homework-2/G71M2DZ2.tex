\documentclass[12pt, a4paper]{article}
\usepackage{cmap} % Улучшенный поиск русских слов в полученном pdf-файле
\usepackage[T2A]{fontenc} % Поддержка русских букв
\usepackage[utf8]{inputenc} % Кодировка utf8
\usepackage[english, russian]{babel} % Языки: русский, английский
\usepackage{enumitem}
\usepackage{pscyr} % Нормальные шрифты
\usepackage{soulutf8}
\usepackage{amsmath}
\usepackage{amsthm}
\usepackage{amssymb}
\usepackage{scrextend}
\usepackage{titling}
\usepackage{indentfirst}
\usepackage{cancel}
\usepackage{soulutf8}
\usepackage{wrapfig}
\usepackage{gensymb}
\usepackage[dvipsnames,table,xcdraw]{xcolor}
\usepackage{tikz}

%Русские символы в списке
\makeatletter
\AddEnumerateCounter{\asbuk}{\russian@alph}{щ}
\makeatother

%Дублирование знаков при переносе
\newcommand*{\hm}[1]{#1\nobreak\discretionary{}%
	{\hbox{$\mathsurround=0pt #1$}}{}}

\usepackage{graphicx}
\graphicspath{{pic/}}
\DeclareGraphicsExtensions{.pdf,.png,.jpg}

%Изменеие параметров листа
\usepackage[left=15mm,right=15mm,
top=2cm,bottom=2cm,bindingoffset=0cm]{geometry}


\usepackage{fancyhdr}
\pagestyle{fancy}
\usepackage{multicol}

\setlength\parindent{1,5em}
\usepackage{indentfirst}

\begin{document}
		
\lhead{Группа 71}
\chead{Модуль 2 ДЗ№2}
\rhead{Школа <<Симметрия>>}

\section*{Признаки равенства треугольников.}
\begin{enumerate}
	\item \textit{(2 балла)} Биссектриса треугольника является его высотой. Докажите, что треугольник равнобедренный.
	\item \textit{(2 балла)} Прямая, проведенная через вершину $A$ треугольника $ABC$ перпендикулярно его медиане $BD$, делит эту медиану
	пополам. Найдите отношение сторон AB и AC.
	\item \textit{(2 балла)} Диагонали $AC$ и $BD$ четырехугольника $ABCD$ пересекаются в точке $O$. Периметр треугольника $ABC$ равен периметру треугольника $ABD$, а периметр треугольника $ACD$ периметру треугольника $BCD$. Докажите, что $AO = BO$.
	\item \textit{(2 балла)} Высоты треугольника $ABC$, проведенные из вершин $B$
	и $C$, пересекаются в точке $M$. Известно, что $BM = CM$. Докажите, что треугольник $ABC$ равнобедренный.
	\item \textit{(2 балла)} Решите уравнение:
	\begin{multicols}{2}
		\begin{enumerate}[label=\asbuk*)]
			\item $\dfrac{x-2}{3} - \dfrac{3x}{2} = 5$
			\item $\dfrac{7+9x}{4}+\dfrac{2-x}{9}=7x+1$
		\end{enumerate}
	\end{multicols}
\end{enumerate}
\end{document}