\documentclass[12pt, a4paper]{article}
\usepackage{../../../../../style}
\begin{document}
	
	\lhead{Группа 71}
	\chead{Модуль 6 Урок №7-8}
	\rhead{Школа <<Симметрия>>}
	\begin{enumerate}
		\item В прямой угол вписана окружность радиуса $R$, касающаяся сторон угла в точках $A$ и $B$. Через некоторую точку
		на меньшей дуге $AB$ окружности проведена касательная, отсекающая от данного угла треугольник. Найдите его периметр
		\item Точка $A$ лежит вне данной окружности с центром $O$. Окружность с диаметром $OA$ пересекается с данной в точках $B$ и $C$. Докажите, что прямые $AB$ и $AC$ — касательные к данной окружности.
		\item  Две прямые, проходящие через точку $M$, лежащую вне окружности с центром $O$, касаются окружности в точках $A$ и $B$. Отрезок $OM$ делится окружностью пополам. В каком отношении отрезок $OM$ делится прямой $AB$?
		\item Докажите, что если окружность касается всех сторон четырехугольника, то суммы противоположных сторон четырехугольника равны между собой.
		\item Прямая касается окружности с центром $O$ в точке $A$. Точка $C$ на этой прямой и точка $D$ на окружности расположены по разные стороны оn прямой $OA$. Найдите угол $CAD$, если угол $AOD$ равен $110\degree$.
	\end{enumerate}
\end{document}