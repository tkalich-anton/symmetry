\documentclass[12pt, a4paper]{article}
\usepackage{../../../../../style}
\begin{document}
	
\lhead{Группа 101} \chead{Модуль 6 Занятие №4} \rhead{Школа <<Симметрия>>} \cfoot{}
\begin{enumerate}
	\item Решить уравнения:
	\begin{multicols}{3}
		\begin{enumerate}[label=\textbf{\arabic*)}]
			\item $2^x=2^5$
			\item $3^x=9$
			\item $7^x=\dfrac{1}{49}$
			\item $\left(\dfrac{1}{2}\right)^x=8$
			\item $27^x=3$
			\item $\left(\dfrac{1}{9}\right)^x=3$
			\item $(0,04)^x=0,2$
			\item $49^x=\dfrac{1}{7}$
			\item $\left(\dfrac{1}{2}\right)^x=-8$
			\item $\left(\dfrac{2}{3}\right)^x=1,5$
		\end{enumerate}
	\end{multicols}
	\item Решить уравнения:
	\begin{multicols}{3}
		\begin{enumerate}[label=\textbf{\arabic*)}]
			\item $5^x-5^{x-1}=100$
			\item $3^{2x+1}-9^x=18$
			\item $4^{x+1}-2^{2x-2}=60$
		\end{enumerate}
	\end{multicols}
	\item Решить уравнения:
	\begin{multicols}{2}
		\begin{enumerate}[label=\textbf{\arabic*)}]
			\item $3^x=4$
			\item $5^x=13$
		\end{enumerate}
	\end{multicols}
	\item Решить уравнения:
	\begin{multicols}{2}
		\begin{enumerate}[label=\textbf{\arabic*)}]
			\item $\left(\dfrac{1}{2}\right)^{x^2-3x}=4$
			\item $5^{x^2-2x}=0,2$
		\end{enumerate}
	\end{multicols}
	\item Решить уравнения:
	\begin{multicols}{2}
		\begin{enumerate}[label=\textbf{\arabic*)}]
			\item $9\cdot5^x-25\cdot3^x=0$
			\item $27\cdot4^x-8\cdot9^x=0$
		\end{enumerate}
	\end{multicols}
	\item Решить уравнения:
	\begin{multicols}{2}
		\begin{enumerate}[label=\textbf{\arabic*)}]
			\item $\log_2 x = 5$
			\item $\log_{0,3} x = -1$
			\item $\log_2 (\log_2 x) = 1$
			\item $\log_5 (\log_3 x) = 0$
		\end{enumerate}
	\end{multicols}
	\item Решить уравнения:
	\begin{multicols}{2}
		\begin{enumerate}[label=\textbf{\arabic*)}]
			\item $\log_{16} x + \log_4 x + \log_2 x = 7$
			\item $\log_2 x + 2\log_4 x + 3\log_8 x + 4\log_16 x = 4$
			\item $\log_2 x + 2\log_4 x + 3\log_8 x + 4\log_16 x = 4$
		\end{enumerate}
	\end{multicols}
	\item Решить уравнения:
	\begin{multicols}{2}
		\begin{enumerate}[label=\textbf{\arabic*)}]
			\item $\log_{\frac{1}{2}} (5x-2) = -3$
			\item $\log_3 (3x^2-5x+1) = 1$
		\end{enumerate}
	\end{multicols}
	\item Решить уравнения с помощью введения новой переменной:
	\begin{multicols}{2}
		\begin{enumerate}[label=\textbf{\arabic*)}]
			\item $9^x-5\cdot3^x+6=0$
			\item $\lg^2 x - 3\lg x + 2 = 0$
			\item $5^x+2\cdot5^{-x}-3=0$
			\item $3^{x+1}-\dfrac{2}{3^{x+1}-2}=1$
		\end{enumerate}
	\end{multicols}
\end{enumerate}
\end{document}