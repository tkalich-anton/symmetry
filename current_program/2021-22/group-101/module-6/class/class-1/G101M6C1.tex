\documentclass[12pt, a4paper]{article}
\usepackage{../../../../../style}
\begin{document}
	\lhead{Группа 101} \chead{Модуль 6 Занятие №1} \rhead{Школа <<Симметрия>>}
	\begin{enumerate}
		\item Решить систему уравнений:
		$$\left\{
		\begin{array}{l}
			x-y-7=0,\\
			3x-y+7=6
		\end{array}
		\right.$$
		\item Решить систему уравнений:
		$$\left\{
		\begin{array}{l}
			x-y=1,\\
			2xy-x^2+9y^2=11-4x
		\end{array}
		\right.$$
		\item \funcexer
		{Прямые $f(x)=x-5,5$ и $g(x)$ пересекаются в точке с координатами $(a;b)$. Найдите $a+b$.}
		{../graphs/graph_6/graph_6}
		\item \funcexer
		{Найдите координаты точки пересечения прямых $f(x)$ и $g(x)$. В ответ запишите сумму абсциссы и ординаты.}
		{../graphs/graph_5/graph_5}
		\item \funcexer
		{На рисунке изображен график функции вида $f(x)=ax^2+c$. Вычислите $f(6)$.}
		{../graphs/graph_7/graph_7}
		\item \funcexer
		{На рисунке изображен график функции вида $f(x)=ax^2+c$. Вычислите $f(3)$.}
		{../graphs/graph_9/graph_9}
		\item \funcexer
		{На рисунке изображен график функции вида $f(x)=ax^2+c$. При каком положительном значении аргумента, значение функции будет равно $-44$?.}
		{../graphs/graph_8/graph_8}
		\item \funcexer
		{На рисунке изображен график функции вида $f(x)=ax^2+c$. Найдите $f(c)$.}
		{../graphs/graph_10/graph_10}
		\item \funcexer
		{На рисунке изображен график функции вида $f(x)=ax^2+c$. Найдите $f(a-c)$.}
		{../graphs/graph_11/graph_11}
		\item \funcexer
		{На рисунке изображен график функции вида $f(x)=ax^2+bx+c$, где числа $a$, $b$ и $c$ — целые. Вычислите $f(5)$.}
		{../graphs/graph_12/graph_12}
		\item \funcexer
		{На рисунке изображен график функции вида $f(x)=ax^2+bx+c$, где числа $a$, $b$ и $c$ — целые. Вычислите $f(a)$.}
		{../graphs/graph_13/graph_13}
	\end{enumerate}
\end{document}