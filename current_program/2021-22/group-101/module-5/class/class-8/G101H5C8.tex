\documentclass[10pt, a4paper]{article}
\usepackage{cmap}
\usepackage[T2A]{fontenc}
\usepackage[utf8]{inputenc}
\usepackage[english, russian]{babel}
\usepackage[dvipsnames,table,xcdraw]{xcolor}
\usepackage{
	amsmath,
	amssymb,
	scrextend,
	enumitem,
	pscyr,
	multicol,
	cmap,
	titling,
	indentfirst,
	cancel,
	wrapfig,
	gensymb,
	tikz,
	graphicx,
	fancyhdr,
	mathrsfs,
	graphbox,
	indentfirst
}
%Параметры страницы
\usepackage[left=15mm,right=15mm,
top=2cm,bottom=2cm]{geometry}
\pagestyle{fancy}
%Путь к картинкам
\graphicspath{{pic/}}
\DeclareGraphicsExtensions{.pdf,.png,.jpg}
%Числа в списке второго уровня по умолчанию
\renewcommand{\labelenumii}{\arabic{enumii})}
%Новые команды
\definecolor{silver}{rgb}{0.7, 0.7, 0.7}
\definecolor{dark}{rgb}{0.3, 0.3, 0.3}
\definecolor{harvestgold}{rgb}{0.85, 0.57, 0.0}
\newcommand{\answer}[1]{\textcolor{silver}{\fbox{#1}}}
\newcommand{\ranswer}[1]{\textcolor{silver}{\begin{flushright}\vspace{-1em}\fbox{#1}\end{flushright}}}
\newcommand{\leveli}{\textcolor{dark}{$\blacksquare\square\square$}\hspace{0.5em}}
\newcommand{\levelii}{\textcolor{dark}{$\blacksquare\blacksquare\square$}\hspace{0.5em}}
\newcommand{\leveliii}{\textcolor{dark}{$\blacksquare\blacksquare\blacksquare$}\hspace{0.5em}}

%Русские символы в списке
\AddEnumerateCounter{\asbuk}{\russian@alph}{щ}

%Сеттеры
\setlength{\parindent}{5ex}
\setlength{\parskip}{1em}
\begin{document}
	
	\lhead{Группа 92}
	\chead{Модуль 5 Урок №7}
	\rhead{Школа <<Симметрия>>}
	\begin{enumerate}
		\item \textit{} Решите систему уравнений:
		\begin{multicols}{2}
			\begin{enumerate}[label=\asbuk*)]
				\item $
				\left\{
				\begin{aligned}
					5(x+1)-9x-3>-6(x+2),\\
					3(3+2x)<7x-2(x-8)
				\end{aligned}
				\right.
				$
				\item $
				\left\{
				\begin{aligned}
					x^2-3x+2<0,\\
				\end{aligned}
				\right.
				$
				\item $(x-9)(x-2)>0$
				\item $(2x-1)(3x+5)<0$
				\item $3x^2+x>0$
				\item $x^2-100<0$
				\item $\dfrac{1}{9}x^2\leqslant 1$
				\item $x^2+4x+3<0$
				\item $8x^2-3-2x>0$
			\end{enumerate}
		\end{multicols}
		\item Ире надо подписать $880$ открыток. Ежедневно она подписывает на одно и то же количество открыток больше по сравнению с предыдущим днем. Известно, что за первый день Ира подписала $10$ открыток. Определите, сколько открыток было подписано за восьмой день, если вся работа была выполнена за $16$ дней.
		\item Вероятность того, что новая шариковая ручка пишет плохо (или не пишет), равна 0,19. Покупатель в магазине выбирает одну такую ручку. Найдите вероятность того, что эта ручка пишет хорошо.
		\item Грузовик перевозит партию щебня массой $360$ тонн, ежедневно увеличивая норму перевозки на одно и то же число тонн. Известно, что за первый день было перевезено $3$ тонны щебня. Определите, сколько тонн щебня было перевезено за десятый день, если вся работа была выполнена за $18$ дней.
		\item Максим с папой решили покататься на колесе обозрения. Всего на колесе двадцать кабинок, из них $4$ – синие, $10$ – зеленые, остальные – красные. Кабинки по очереди подходят к платформе для посадки. Найдите вероятность того, что Максим прокатится в красной кабинке.
	\end{enumerate}
\end{document}