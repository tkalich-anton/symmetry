\documentclass[12pt, a4paper]{article}
\usepackage{../../../../../style}
\begin{document}
	\lhead{Группа 101} \chead{Модуль 7} \rhead{Школа <<Симметрия>>} \cfoot{}
	\begin{center}
		\Large
		\textbf{Домашняя работа №1}
	\end{center}
	\begin{enumerate}[label=\textbf{\arabic*.}]
		\item Решить уравнения:
		\begin{multicols}{2}
			\begin{enumerate}[label=\asbuk*)]
				\item \( \tg\left( x+\dfrac{\pi}{4} \right) = 0 \)
				\item \( \tg\left( 2x-\dfrac{\pi}{3} \right)=\dfrac{\sqrt{3}}{3} \)
				\item \( \ctg^2 x = 2 \)
				\item \( \tg^2 x - \sqrt{3}\tg x = 0 \)
				\item \( 3\sin x = 2\cos^2 x \)
				\item \( \sin^2 x + 2\cos x - 2 = 0 \)
			\end{enumerate}
		\end{multicols}
		\item Решить уравнения:
		\begin{multicols}{2}
			\begin{enumerate}[label=\asbuk*)]
				\item \( \sqrt{3}\sin x + \cos x = 0 \)
				\item \( 5\sin x + \cos x = 0 \)
				\item \( \sin^2 x + 3\sin x\cos x - 4\cos^2 x = 0\)
				\item \( \sqrt{2}\cos^2x=\sin\left( \dfrac{\pi}{2}+x \right) \)
			\end{enumerate}
		\end{multicols}
	\end{enumerate}
\end{document}