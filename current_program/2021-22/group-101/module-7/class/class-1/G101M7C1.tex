\documentclass[12pt, a4paper]{article}
\usepackage{../../../../../style}
\DeclareUnicodeCharacter{202F}{\,}
\begin{document}
	
\lhead{Группа 101} \chead{Модуль 7} \rhead{Школа <<Симметрия>>} \cfoot{}
\begin{center}
	\Large
	\textbf{Занятие №1}
\end{center}
\begin{center}
	\large
	Простейшие тригонометрические уравнения:
\end{center}
\begin{enumerate}[label=\arabic*)]
	\item \( \sin x = a \Leftrightarrow \left[
	\begin{array}{l}
		x_1 = \arcsin x + 2\pi n, n \in Z,\vspace{0,2cm}\\
		x_1 = \pi - \arcsin x + 2\pi n, n \in Z
	\end{array}
	\right. \)
	\item \( \cos x = a \Leftrightarrow x=\pm \arccos x + 2\pi n, n \in Z \)
	\item \( \tg x = a \Leftrightarrow x=\arctg x + \pi n, n \in Z \)
	\item \( \ctg x = a \Leftrightarrow x=\arcctg x + \pi n, n \in Z \)
\end{enumerate}
\begin{center}
	\large
	Однородные тригонометрические уравнения:
\end{center}
\[ a\cdot\sin x + b\cdot\cos x = 0 \]
\[ a\cdot\sin x + b\cdot\cos x = 0\hspace{0,5cm} |:\cos x, \cos x\neq0\]
\[ a\cdot\tg x + b = 0 \]
\[ a\cdot\tg x = -b \]
\[ \tg x = -\dfrac{b}{a} \]
\textbf{\large Задания:}
\begin{multicols}{2}
	\begin{enumerate}[label=\textbf{\arabic*.}]
		\item \( \tg x = \dfrac{\sqrt{3}}{3} \)
		\item \( \ctg x = - \dfrac{\sqrt{3}}{3} \)
		\item \( \tg 3x = \sqrt{3} \)
		\item \( \tg^2 x + \sqrt{2}\tg x = 0 \)
		\item \( 3\tg^2 x + \dfrac{1}{\tg^2 x -1} = 0,5\)
		\item \( \sin x - \sqrt{3}\cos x = 0 \)
		\item \( \sin x + 5\cos x = 0 \)
		\item \( \sin^2 x - 3\sin x\cos x + 2\cos^2 x = 0 \)
		\item \( 2\cos^2 x = \sqrt{3}\sin\left( \dfrac{3\pi}{2}+x \right) \)
		\item \( \sin 2x=\sqrt{2}\sin x \)
		\item \( \dfrac{1}{\cos^2 x}+\dfrac{3}{\sin\left( \dfrac{\pi}{2}+x \right)}+2=0 \)
		\item \( \sin^3 x - 7\sin x \cos^2 x + 6\cos^3 x = 0 \)
	\end{enumerate}
\end{multicols}
\end{document}