\documentclass[12pt, a4paper]{article}
\usepackage{../../../../../style}
\lhead{Для преподавателей} \chead{Встреча 1} \cfoot{} %\showanswers
\begin{document}
	\begin{listofex}
		\item (Настя) Представьте, что в тесте вы встретили следующий вопрос:\\
		
		Какова доля верных среди ответов на этот вопрос?
		\begin{enumerate}
			\item 50\%
			\item 25\%
			\item 0\%
			\item 50\%
		\end{enumerate}
		Что бы вы ответили и почему?
		\item (Ева) Сравните числа: \( \dfrac{\sin2016\degree}{\sin2017\degree} \) и \( \dfrac{\sin2018\degree}{\sin2019\degree} \)
		\item (Артем) Дан треугольник \( ABC \), в нем проведена высота \( AH \), точка \( O \) центр описанной около треугольника окружности. Требуется доказать равенство углов \( CAH \) и \( BAO \).
		\item (Геля) Решите неравенство: \( 0,5^{-\frac{x-2}{2x+4}}\cdot10^x\cdot x^{-2}\ge\dfrac{32^{-\frac{x-2}{2x+4}\cdot40^x}}{16x^2} \)
		\item (Глеб) Есть 1000 банок с кормом, одна из которых отравлена. Также есть 10 свиней, которым можно давать корм. Содержимое разных банок можно смешивать, а свиньи едят один раз в день и могут съесть хоть весь корм смешанный в одну кучу. Найдите минимальное количество дней, за которое можно узнать какая банка отравлена.
		\item (Антон) На сторонах выпуклого четырехугольника как на диаметрах построены четыре круга. Докажите, что они покрывают весь четырехугольник.
	\end{listofex}
\end{document}