\documentclass[12pt, a4paper]{article}
\usepackage{cmap} % Улучшенный поиск русских слов в полученном pdf-файле
\usepackage[T2A]{fontenc} % Поддержка русских букв
\usepackage[utf8]{inputenc} % Кодировка utf8
\usepackage[english, russian]{babel} % Языки: русский, английский
\usepackage{enumitem}
\usepackage{pscyr} % Нормальные шрифты
\usepackage{soulutf8}
\usepackage{amsmath}
\usepackage{amsthm}
\usepackage{amssymb}
\usepackage{scrextend}
\usepackage{titling}
\usepackage{indentfirst}
\usepackage{cancel}
\usepackage{soulutf8}
\usepackage{wrapfig}
\usepackage{gensymb}
\usepackage[dvipsnames,table,xcdraw]{xcolor}
\usepackage{tikz}

%Русские символы в списке
\makeatletter
\AddEnumerateCounter{\asbuk}{\russian@alph}{щ}
\makeatother

%Дублирование знаков при переносе
\newcommand*{\hm}[1]{#1\nobreak\discretionary{}%
	{\hbox{$\mathsurround=0pt #1$}}{}}

\usepackage{graphicx}
\graphicspath{{pic/}}
\DeclareGraphicsExtensions{.pdf,.png,.jpg}

%Изменеие параметров листа
\usepackage[left=15mm,right=15mm,
top=2cm,bottom=2cm,bindingoffset=0cm]{geometry}


\usepackage{fancyhdr}
\pagestyle{fancy}
\usepackage{multicol}

\setlength\parindent{1,5em}
\usepackage{indentfirst}

\begin{document}
	
	\lhead{Группа 82}
	\chead{Модуль 3 ДЗ№1}
	\rhead{Школа <<Симметрия>>}
	
	\section*{Домашняя работа №1}
	\begin{enumerate}
		\item \textit{(1 балл)} Вычислите
		
		\begin{multicols}{2}
			\begin{enumerate}[label=\asbuk*)]
				\item $\sqrt{49 \cdot 64 \cdot 100}$
				\item $\sqrt{250000}$
				\item $\sqrt{\left( -\dfrac{1}{3}\right)^2 }$
				\item $\sqrt{\dfrac{1}{9}} \cdot \sqrt{81}$
				\item $\sqrt{9}+\sqrt{4}$
				\item $\sqrt{49}:\sqrt{0,01}$
			\end{enumerate}
		\end{multicols}
		
		
		
		
		\item \textit{(1 балл)} Сравните 
		\begin{multicols}{2}
			\begin{enumerate}[label=\asbuk*)]
				\item $\sqrt{100}$ и $\sqrt{81}$
				\item $\sqrt{\dfrac{9}{16}}$ и $2$
				\item $\sqrt{\dfrac{1}{4}}$ и $\dfrac{1}{4}$
				\item $\sqrt{0,09}$ и $\sqrt{\dfrac{9}{16}}$
				\item $5\sqrt{2}$ и $2\sqrt{5}$
				\item $6\sqrt{3}$ и $5\sqrt{4}$
			\end{enumerate}
		\end{multicols}
		\item \textit{(2 балла)} Расположите в порядке возрастания числа
		\begin{enumerate}[label=\asbuk*)]
			\item $0,2\sqrt{48}, 0,9\sqrt{3}, \sqrt{3}, \sqrt{12}, 1\dfrac{1}{3}, \sqrt{3}$
		\end{enumerate}
		\item \textit{(3 балла)} Упростите выражение
		\begin{multicols}{2}
			\begin{enumerate}[label=\asbuk*)]
				\item $2\sqrt{2}+3\sqrt{2}$
				\item $2\sqrt{8}-3\sqrt{2}$
				\item $\sqrt{2}+8\sqrt{2}+\dfrac{1}{2}\cdot \sqrt{128}+5\sqrt{2}-\sqrt{200}$
				\item $\left( 7\sqrt{2}-5\sqrt{6}-3\sqrt{8}+4\sqrt{20}\right) \cdot 3\sqrt{2} $
				\item $\left( 7-\sqrt{3}\right) \left( 3+\sqrt{7}\right)  $
				\item $\left( \sqrt{20}-3\right)\left( 3+2\sqrt{5}\right)  $
			\end{enumerate}
		\end{multicols}
		\item \textit{(1 балл)} Упростите выражение
		$\left(m - \dfrac{1}{1+m} \right) \cdot \left( \dfrac{m+1}{1-m-m^2}\right)  $
		\item \textit{(2 балла)} Докажите, что для любого числа $a \geq 0$ выполняется равенство:
		
		$(\sqrt{a}-1)^2+4\sqrt{a}=(\sqrt{a}+1)^2$
	\end{enumerate}
\end{document}