\documentclass[12pt, a4paper]{article}
\usepackage{../../../../../style}
\begin{document}
\lhead{Группа 82} \chead{Модуль 6 Урок №1-2} \rhead{Школа <<Симметрия>>}
\cfoot{}
\begin{enumerate}
	\item Сторона треугольника равна $12$. Найдите отрезок, соединяющий середины медиан, проведенных к двум другим сторонам.
	\item Прямая, проходящая через общую точку $A$ двух окружностей, пересекает вторично эти окружности в точках $B$ и $C$ соответственно. Расстояние между проекциями центров окружностей на эту прямую равно $12$. Найдите $BC$, если известно, что точка $A$ лежит на отрезке $BC$.
	\item Докажите, что середины сторон любого четырехугольника являются вершинами параллелограмма.
	\item Острый угол $A$ ромба $ABCD$ равен $45\degree$, проекция стороны $AB$ на сторону $AD$ равна $12$. Найдите расстояние от центра ромба до стороны $CD$.
	\item В треугольник $ABC$ вписана окружность, касающаяся стороны $AB$ в точке $M$. Пусть $AM = x$, $BC = a$, полупериметр треугольника равен $p$. Докажите, что $x=p-a$.
	\item В треугольник со сторонами $6$, $10$ и $12$ вписана окружность. К окружности проведена касательная так, что она пересекает две большие стороны. Найдите периметр отсеченного треугольника.
	\item Расстояние от середины хорды $BC$ до диаметра $AB$ равно $1$. Найдите хорду $AC$, если $\angle BAC = 30 \degree$.
	\item Две прямые, проходящие через точку $C$, касаются окружности в точках $A$ и $B$. Может ли прямая, проходящая через середины отрезков $AC$ и $BC$, касаться этой окружности?
	\item Две окружности пересекаются в точках $A$ и $D$. Проведены диаметры $AB$ и $AC$ этих окружностей. Найдите $BD+DC$, если расстояние между центрами окружностей равно $4$ и центры окружностей лежат по разные стороны от общей хорды.
	\item Докажите, что середины двух противоположных сторон любого четырехугольника без параллельных сторон и середины его диагоналей являются вершинами параллелограмма.
	\item В выпуклом четырехугольнике $ABCD$ отрезок, соединяющий середины сторон $AB$ и $CD$, равен $1$. Прямые $BC$ и $AD$ перпендикулярны. Найдите отрезок, соединяющий середины диагоналей $AC$ и $BD$.
\end{enumerate}
\end{document}