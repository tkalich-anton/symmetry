\documentclass[12pt, a4paper]{article}
\usepackage{../../../../../style}
\begin{document}
	
\lhead{Группа 82} \chead{Модуль 6} \rhead{Школа <<Симметрия>>} \cfoot{}
\begin{center}
	\Large
	\textbf{Занятие №8}
\end{center}
\begin{enumerate}[label=\textbf{\arabic*.}]
	\item Решите уравнение: $$\dfrac{7}{x+1}-\dfrac{x+4}{2-2x}=\dfrac{3x^2-38}{x^2-1}$$
	\item Решите уравнения:
	\begin{enumerate}[label=\asbuk*)]
		\begin{multicols}{2}
		\item $x^2+4x+|x+3|+3=0$
		\item $\dfrac{x^2-2x}{4x-3}+5=\dfrac{16x-12}{x^2-2x}$
		\end{multicols}
	\end{enumerate}
	\item Диагональ равнобокой трапеции равна \( 10 \) и образует угол, равный \( 60\degree \), с основанием трапеции. Найдите среднюю линию трапеции.
	\item На прямую, проходящую через вершину \( A \) треугольника \( ABC \), опущены перпендикуляры \( BD \) и \( CE \). Докажите, что 
	середина стороны \( BC \) равноудалена от точек \( D \) и \( E \).
\end{enumerate}
\end{document}