\documentclass[12pt, a4paper]{article}
\usepackage{../../../../../style}
\begin{document}
	
	\lhead{Группа 82} \chead{Модуль 6} \rhead{Школа <<Симметрия>>}
	\begin{center}
		\Large
		\textbf{Занятие №6}
	\end{center}
	\begin{enumerate}
		\item Упростить выражение: $$\dfrac{x^2-y^2}{(x+y)^2}:\dfrac{6x-6y}{3x+3y}$$
		\item Упростить выражение: $$\dfrac{c^3-8}{x+3}:\left( \dfrac{c-2}{4c}\cdot\dfrac{8c^3}{c^2+3c} \right):\dfrac{c^2+2c-4}{2(3-c)}$$
		\item Решите уравнение: $$\dfrac{2}{x}+\dfrac{10}{x^2-2x}=\dfrac{1+2x}{x-2}$$
		\item Решите уравнение: $$\left(x+\dfrac{1}{x}\right)^2-4\dfrac{1}{2}\left(x+\dfrac{1}{x}\right)+5=0$$
		\item Докажите, что середины двух противоположных сторон любого четырехугольника без параллельных сторон и середины его диагоналей являются вершинами параллелограмма.
		\item В выпуклом четырехугольнике $ABCD$ отрезок, соединяющий середины сторон $AB$ и $CD$, равен $1$. Прямые $BC$ и $AD$ перпендикулярны. Найдите отрезок, соединяющий середины диагоналей $AC$ и $BD$.
		\item Один из углов прямоугольной трапеции равен \( 120\degree \), большее основание равно \( 12 \). Найдите отрезок, соединяющий середины диагоналей, если известно, что меньшая диагональ трапеции равна ее большему основанию.
	\end{enumerate}
\end{document}