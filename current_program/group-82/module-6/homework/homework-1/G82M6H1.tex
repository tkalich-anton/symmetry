\documentclass[12pt, a4paper]{article}
\usepackage{../../../../../style}
\begin{document}
	
	\lhead{Группа 82} \chead{Модуль 6} \rhead{Школа <<Симметрия>>} \cfoot{}
	\section*{Домашняя работа №1}
	\begin{enumerate}
		\item \textit{(1 балл)} \subimport{../../../../../exercises/arithmetic/coordinate_line/roots}{ex_2_2}
		\item \textit{(1 балл)} \subimport{../../../../../exercises/algebra/expressions/fractions}{ex_1}
		\item \textit{(1 балл)} Дан треугольник с периметром, равным $24$. Найдите периметр треугольника с вершинами в серединах сторон данного.
		\item \textit{(1 балл)} Найдите периметр четырехугольника с вершинами в серединах сторон прямоугольника с диагональю, равной $8$.
		\item \textit{(2 балла)} Найдите стороны и углы четырехугольника с вершинами в серединах сторон ромба, диагонали которого равны $6$ и $10$.
		\item \textit{(2 балла)} Расстояние между серединами взаимно перпендикулярных хорд $AC$ и $BC$ некоторой окружности равно $10$. Найдите расстояние от центра окружности до точки пересечения хорд.
		\item \textit{(2 балла)} В выпуклом четырехугольнике $ABCD$ отрезок, соединяющий середины диагоналей, равен отрезку, соединяющему середины сторон $AD$ и $BC$. Найдите угол, образованный продолжениями сторон $AB$ и $CD$.
	\end{enumerate}
\end{document}