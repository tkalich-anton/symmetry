\documentclass[12pt, a4paper]{article}
\usepackage{cmap} % Улучшенный поиск русских слов в полученном pdf-файле
\usepackage[T2A]{fontenc} % Поддержка русских букв
\usepackage[utf8]{inputenc} % Кодировка utf8
\usepackage[english, russian]{babel} % Языки: русский, английский
\usepackage{enumitem}
\usepackage{pscyr} % Нормальные шрифты
\usepackage{amsmath}
\usepackage{amsthm}
\usepackage{amssymb}
\usepackage{scrextend}
\usepackage{titling}
\usepackage{indentfirst}
\usepackage{cancel}
\usepackage{soulutf8}
\usepackage{wrapfig}
\usepackage{gensymb}
\usepackage[dvipsnames,table,xcdraw]{xcolor}
\usepackage{tikz}

%Русские символы в списке
\makeatletter
\AddEnumerateCounter{\asbuk}{\russian@alph}{щ}
\makeatother

%Дублирование знаков при переносе
\newcommand*{\hm}[1]{#1\nobreak\discretionary{}%
	{\hbox{$\mathsurround=0pt #1$}}{}}

\usepackage{graphicx}
\graphicspath{{pic/}}
\DeclareGraphicsExtensions{.pdf,.png,.jpg}

%Изменеие параметров листа
\usepackage[left=15mm,right=15mm,
top=2cm,bottom=2cm,bindingoffset=0cm]{geometry}

\usepackage{fancyhdr}
\pagestyle{fancy}
\usepackage{multicol}

\setlength\parindent{1,5em}
\usepackage{indentfirst}
\begin{document}
	
	\lhead{Группа 82}
	\chead{Модуль 6 Домашняя работа №1}
	\rhead{Школа <<Симметрия>>}
	\section*{Домашняя работа №1}
	\begin{enumerate}
		\item \textit{(2 балла)} Дан треугольник с периметром, равным $24$. Найдите периметр треугольника с вершинами в серединах сторон данного.
		\item \textit{(2 балла)} Найдите периметр четырехугольника с вершинами в серединах сторон прямоугольника с диагональю, равной $8$.
		\item \textit{(2 балла)} Найдите стороны и углы четырехугольника с вершинами в серединах сторон ромба, диагонали которого равны $6$ и $10$.
		\item \textit{(2 балла)} Расстояние между серединами взаимно перпендикулярных хорд $AC$ и $BC$ некоторой окружности равно $10$. Найдите расстояние от центра окружности до точки пересечения хорд.
		\item \textit{(2 балла)} В выпуклом четырехугольнике $ABCD$ отрезок, соединяющий середины диагоналей, равен отрезку, соединяющему середины сторон $AD$ и $BC$. Найдите угол, образованный продолжениями сторон $AB$ и $CD$.
	\end{enumerate}
\end{document}