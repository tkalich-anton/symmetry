\documentclass[12pt, a4paper]{article}
\usepackage{../../../../../style}
\begin{document}
\lhead{Группа 82} \chead{Модуль 6} \rhead{Школа <<Симметрия>>} \cfoot{}
\begin{center}
	\Large
	\textbf{Домашняя работа №2}
\end{center}
\begin{enumerate}
	\item \textit{(2 балла)} Решить уравнение \[ \dfrac{1-2x}{6x^2+3x}-\dfrac{2x+1}{14x^2-7x}=\dfrac{8}{12x^2-3} \]
	\item \textit{(2 балла)} Решить уравнение \[ x^2-|x-5|=5 \]
	\item \textit{(2 балла)} Решить уравнение \[ (2x-1)^4-(2x-1)^2-12=0 \]
	\item \textit{(2 балла)} Боковые стороны трапеции равны \( 7 \) и \( 11 \), а основания --- \( 5 \) и \( 15 \). Прямая, проведенная через вершину меньшего
	основания параллельно большей боковой стороне, отсекает от трапеции треугольник. Найдите его стороны.
	\item \textit{(2 балла)} Диагонали трапеции взаимно перпендикулярны. Одна из них равна \( 6 \), а вторая образует с основанием угол, равный \( 30\degree \). Найдите среднюю линию трапеции.
\end{enumerate}
\end{document}