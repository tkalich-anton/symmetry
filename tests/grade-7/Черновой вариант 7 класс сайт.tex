\documentclass[12pt, a4paper]{article}
\usepackage{cmap} % Улучшенный поиск русских слов в полученном pdf-файле
\usepackage[T2A]{fontenc} % Поддержка русских букв
\usepackage[utf8]{inputenc} % Кодировка utf8
\usepackage[english, russian]{babel} % Языки: русский, английский
\usepackage{enumitem}
\usepackage{pscyr} % Нормальные шрифты
\usepackage{amsmath}
\usepackage{amsthm}
\usepackage{amssymb}
\usepackage{scrextend}
\usepackage{titling}
\usepackage{indentfirst}
\usepackage{cancel}
\usepackage{soulutf8}
\usepackage{wrapfig}
\usepackage{gensymb}
\usepackage[dvipsnames,table,xcdraw]{xcolor}
\usepackage{tikz}

%Русские символы в списке
\makeatletter
\AddEnumerateCounter{\asbuk}{\russian@alph}{щ}
\makeatother

%Дублирование знаков при переносе
\newcommand*{\hm}[1]{#1\nobreak\discretionary{}%
	{\hbox{$\mathsurround=0pt #1$}}{}}

\usepackage{graphicx}
\graphicspath{{pic/}}
\DeclareGraphicsExtensions{.pdf,.png,.jpg}

%Изменеие параметров листа
\usepackage[left=15mm,right=15mm,
top=2cm,bottom=2cm,bindingoffset=0cm]{geometry}

\usepackage{fancyhdr}
\pagestyle{fancy}
\usepackage{multicol}

\setlength\parindent{1,5em}
\usepackage{indentfirst}
\begin{document}
	
	\chead{Тестирование 7 класс}
	\rhead{Школа <<Симметрия>>}
	\section*{Тестирование}
	\begin{enumerate}
		\item Выберите простые числа:
		
		$0; 1; 5; 12; 13; 14; 29; 30. $
		\item Разложите число 336 на простые множители.
		\item Найдите НОД 180 и 336.
		
		Найдите НОК 18 и 24.
		\item Вычислите:
		\begin{multicols}{2}
			\begin{enumerate}[label=\asbuk*)]
				\item $\dfrac{1}{3}+\dfrac{2}{7}$
				\item $\dfrac{5}{8}-\dfrac{5}{12}$
				\item $\left( \dfrac{9}{5}+3\dfrac{1}{5}\right) \cdot 0,2$ 
				\item $0,7 \cdot 3,09$
				\item $0,002 \cdot 100$
				\item $14-\dfrac{2}{3}$
				\item $-3\cdot 10 : (-5)$
				\item $120+(-14 \cdot 3)$
				\item $-\dfrac{3}{2}+1,5$
				\item $2^2+3^2$
				\item $3^2\cdot 3^3$
				\item $\left( \dfrac{1}{4}\right)^{16}:\left( \dfrac{1}{4}\right)^{14}$
				\item $\dfrac{(4^2)^6}{4^{13}}$
			\end{enumerate}
		\end{multicols}
		\item Решите уравнения:
			\begin{enumerate}[label=\asbuk*)]
				\item $2x=8$
				\item $x+16=31$
				\item $\dfrac{1}{3}x-10=\dfrac{2}{7}$ 
				\end{enumerate}
		\item \textit{}Найдите $\dfrac{2}{3}$ от $33$.
-		
		$34 – 17\%$ от загаданного числа. Найдите это число. 
		\item Найдите значение выражения $\dfrac{x+y}{2}-x$, если $x=10, y=12$.
		\item Масса сушёных яблок составляет $30\%$ массы свежих. Сколько кг сушёных яблок получится из $120$ кг свежих?
		\item Сравните $\dfrac{4}{7}$ и $\dfrac{5}{8}$.
		\item Сравните $1,02$ и $1,3$.
		\item Сравните $\dfrac{4}{7}$ и $0,56$.
		\item Расположите в порядке возрастания числа:	
		$0,56; 1,3; \dfrac{4}{7}; \dfrac{5}{8}; 1,02$
		\item Представьте $\dfrac{4}{3}$ в виде периодической дроби.
		\item 
		\begin{enumerate}[label=\asbuk*)]
			\item Переведите $1200$г в кг.
			\item Сколько мм в $3$дм
			\item Найдите периметр прямоугольника со сторонами $10$ и $2,3$
			\item Найдите площадь фигуры	(рис.1)
		\end{enumerate}	
		\item 
		 \begin{enumerate}[label=\asbuk*)]
			\item Решите пропорцию $\dfrac{30}{x}=\dfrac{5}{8}$
			\item Поезд, скорость которого $45$ км/ч, затратил на некоторый участок пути $4$ч. За сколько часов пройдёт этот же участок пути товарный поезд, если его скорость $40$км/ч?
		\end{enumerate}
		\item Вычислите:
		$|3-12|$
		\item Являются ли числа $5$ и $\dfrac{1}{5}$ противоположными? Запишите обратное число для $\dfrac{2}{3}$
		\item Равны ли числа:
		\begin{enumerate}[label=\asbuk*)]
			\item $-\dfrac{1}{3}$ и $\dfrac{-1}{3}$
			\item $-12$ и $-(-(-12)))$
			\item $-\dfrac{5}{6}$ и $\dfrac{-5}{-6}$
		\end{enumerate}
	\item Является ли равенство $5+3x=8x$ тождеством? Если нет, то почему?
	\item Преобразуйте:
	\begin{multicols}{2}
		\begin{enumerate}[label=\asbuk*)]
	\item  $8bc^3$
	\item $\left( -1\dfrac{1}{7}ab\right)^2 $
	\item $27a-3,1b+9a+3,1a+0,4b-a$
	\item $5a(a^2b-\dfrac{1}{3}+\dfrac{1}{12}a-\dfrac{1}{5})$
	\end{enumerate}
	\end{multicols}
	\item Соотнесите треугольники с их названиями
	(рис.2)
	\begin{enumerate}[label=\asbuk*)]
		\item Равнобедренный
		\item Тупоугольный
		\item Прямоугольный
		\item Правильный
	\end{enumerate}
		\item Соотнесите углы с их названиями
		(рис.2)
		\begin{enumerate}[label=\asbuk*)]
			\item Острый
			\item Тупой
			\item Соответственные углы
			\item Накрест лежащие углы
			\item Прямой
		\end{enumerate}
	\item Вставьте пропуски:
	Если две стороны и $\,\,\,\,\,\,\,\,\,\,\,\,\,\,\,\,\,\,\,\,\,\,\,\,\,\,\,\,\,\,\,\,\,\,$ одного треугольника соответственно равны $\,\,\,\,\,\,\,\,\,\,\,\,\,\,\,\,\,\,\,\,\,\,\,\,\,\,\,\,\,\,\,\,\,\,\,\,\,\,\,\,\,\,\,\,\,\,\,\,\,\,$ другого треугольника, то такие треугольники равны.\\\\
	Если $\,\,\,\,\,\,\,\,\,\,\,\,\,\,\,\,\,\,\,$ и два прилежащих к ней угла одного треугольника соответственно равны $\,\,\,\,\,\,\,\,\,\,\,\,\,\,\,\,\,\,\,\,\,\,\,\,\,\,\,\,\,\,\,$ другого треугольника, то такие треугольники равны.\\\\
	Если три $\,\,\,\,\,\,\,\,\,\,\,\,\,\,\,$ одного треугольника соответственно равны $\,\,\,\,\,\,\,\,\,\,$ другого треугольника, то такие треугольники равны.
	\item Перечислите свойства равнобедренных треугольников.
	\item Перечислите свойства равносторонних треугольников.
	\item Отрезки $MT$ и $AD$ пересекаются в точке $O$ так, что $MO=OT$; $AO=OD$. Чему равен $\angle OAD$ и $\angle OTA$, если $\angle DMP=22^{\circ}$, а $\angle MDO=57^{\circ}$\\
	(рис. 4)
	\item Найдите второй угол.\\
	(рис.5) % Смежные углы
	\item Медиана $AM$ треугольника $ABC$ перпендикулярна его биссектрисе $BK$. Найдите $AB$, если $BC=12$
	\end{enumerate}
\end{document}