\documentclass[10pt, a4paper]{article}
\usepackage{../../../style}
\lhead{\leftmark}
\setkeys{exercise}{showNumber=true,showAnswer=true,bankPath=../../bank}
\begin{document}
\section{Треугольники}
\subsection{Признаки равенства треугольников}
	\begin{enumerate}
		\item Медиана $AM$ треугольника $ABC$ перпендикулярна его биссектрисе $BK$. Найдите $AB$, если $BC = 12$.
		\item Прямая,  проведенная  через  вершину  $A$  треугольника $ABC$ перпендикулярно его медиане $BD$, 
		делит эту медиану пополам. Найдите отношение сторон $AB$ и $AC$.
		\item Стороны  равностороннего  треугольника  делятся  точками $K$, $L$, $M$ в одном и том же отношении (считая по часовой стрелке).  Докажите,  что  треугольник $KLM$  также  равносторонний.
		\item Докажите, что в равных треугольниках соответствующие медианы равны.
		\item Докажите, что в равных треугольниках соответствующие биссектрисы равны.
		\item Докажите, что биссектриса равнобедренного треугольника, проведенная из вершины, является также медианой
		и высотой.
		\item Медиана треугольника является также его высотой. Докажите, что такой треугольник равнобедренный.
		\item В треугольнике $ABC$ медиана $AM$ продолжена за точку $M$ на расстояние, равное $AM$. Найдите расстояние от полученной точки до вершин $B$ и $C$, если $AB = 7$, $AC = 11$.
		\item Биссектриса треугольника является его медианой. Докажите, что треугольник равнобедренный.
		\item Докажите признаки равенства прямоугольных тре-
		угольников:
		\begin{enumerate}[label=\asbuk*)]
			\item по двум катетам;
			\item по катету и гипотенузе;
			\item по катету и прилежащему острому углу;
			\item по гипотенузе и острому углу.
		\end{enumerate}
		\item Докажите, что серединный перпендикуляр к отрезку есть геометрическое место точек, равноудаленных от концов этого отрезка.
		\item Две различные окружности пересекаются в точках $A$ и $B$. Докажите, что прямая, проходящая через центры окружностей, делит отрезок $AB$ пополам и перпендикулярна ему.
		\item Две различные окружности с центрами в точках $O_1$ и $O_2$ пересекаются в точках $A$ и $B$. Прямая, проходящая через центры окружностей, пересекает отрезок $AB$ в точке $K$. Докажите, что треугольники $O_1KA$ и $O_1KB$ равны.
		\item Докажите признак равенства прямоугольных треугольников по катету и противолежащему углу.
		\item Докажите, что в равных треугольниках соответствующие высоты равны между собой.
		\item Докажите, что серединный перпендикуляр к отрезку является его осью симметрии.
		\item Докажите, что диагонали четырехугольника с равными сторонами взаимно перпендикулярны.
		\item Точки $M$ и $N$ --- середины равных сторон $AD$ и $BC$ четырехугольника $ABCD$. Серединные перпендикуляры к сторонам $AB$ и $CD$ пересекаются в точке $P$. Докажите, что серединный перпендикуляр к отрезку $MN$ проходит через точку $P$.
		\item Две высоты треугольника равны между собой. Докажите, что треугольник равнобедренный.
		\item Высоты треугольника $ABC$, проведенные из вершин $B$ и $C$, пересекаются в точке $M$. Известно, что $BM = CM$. Докажите, что треугольник ABC равнобедренный.
		\item Найдите геометрическое место внутренних точек угла, равноудаленных от его сторон.
		\item Докажите, что биссектриса угла является его осью симметрии.
		\item Через вершины $A$ и $C$ треугольника $ABC$ проведены прямые, перпендикулярные биссектрисе угла $ABC$, пересекающие прямые $CB$ и $BA$ в точках $K$ и $M$ соответственно. Найдите $AB$, если $BM = 8$, $KC = 1$.
		\item Через данную точку проведите прямую, пересекающую две данные прямые под равными углами.
		\item Площадь прямоугольника равна $24$. Найдите площадь четырехугольника с вершинами в серединах сторон прямоугольника.
		\item Средняя линия треугольника разбивает его на треугольник и четырехугольник. Какую часть составляет площадь полученного треугольника от площади исходного?
		\item Докажите, что медиана разбивает треугольник на два равновеликих треугольника.
		\item Точки, делящие сторону треугольника на $n$ равных частей, соединены отрезками с противоположной вершиной. Докажите, что при этом треугольник также разделился на $n$ равновеликих частей.
		\item Пусть $M$ — точка на стороне $AB$ треугольника $ABC$, причем $AM : MB = m : n$. Докажите, что площадь треугольника $CAM$ относится к площади треугольника $CBM$ как $m : n$.
		\item Докажите, что площадь выпуклого четырехугольника со взаимно перпендикулярными диагоналями равна половине произведения диагоналей.
		\item На сторонах $AB$ и $AC$ треугольника $ABC$, площадь которого равна $50$, взяты соответственно точки $M$ и $K$ так, что $AM : MB = 1 : 5$, а $AK : KC = 3 : 2$. Найдите площадь треугольника $AMK$.
		\item Вершины одного квадрата расположены на сторонах другого и делят эти стороны в отношении $1 : 2$, считая по часовой стрелке. Найдите отношение площадей квадратов.
	\end{enumerate}
	\exercise{755}
	\exercise{758}
	\exercise{759}
	\exercise{760}
	\exercise{761}
	\exercise{809}
	\exercise{810}
	\exercise{811}
	\exercise{812}
	\exercise{813}
	\exercise{814}
	\exercise{815}
	\exercise{816}
	\exercise{817}
	\exercise{818}
	\exercise{1052}
	\exercise{1053}
	\exercise{1054}
	\exercise{1055}
	\exercise{1056}
	\exercise{1057}
	\exercise{1058}
	\exercise{1059}
	\exercise{1060}
	\exercise{1061}
	\exercise{1062}
	\exercise{1063}
	\exercise{1149}
	\exercise{1150}
	\exercise{1151}
	\exercise{1152}
	\exercise{1153}
	\exercise{1154}
	\exercise{1155}
	\exercise{1156}
	\exercise{1157}
	\exercise{1158}
	\exercise{1159}
	\exercise{1160}
	\exercise{1161}
	\exercise{1162}
	\exercise{1199}
	\exercise{1200}
	\exercise{1201}
	\exercise{1202}
	\exercise{1203}
	\exercise{1204}
	\exercise{1205}
	\exercise{1206}
	\exercise{1207}
	\exercise{1270}
	\exercise{1271}
	\exercise{1278}
	\exercise{1279}
	\exercise{1280}
	\exercise{1281}
	\exercise{1282}
	\exercise{1283}
	\exercise{1284}
	\exercise{1285}
	\exercise{1286}
\subsection{Параллельность}
\subsection{Окружность}
\end{document}