\documentclass[10pt, a4paper]{article}
\usepackage{../../../style}

\begin{document}
		
\lhead{Планиметрия}
\rhead{Школа <<Симметрия>>}

\section{Треугольники}
\subsection{Признаки равенства треугольников}
	\begin{enumerate}
		\item \source{Гордин Р.К. Планиметрия, №1.40} Медиана $AM$ треугольника $ABC$ перпендикулярна его биссектрисе $BK$. Найдите $AB$, если $BC = 12$.
		\item \source{Гордин Р.К. Планиметрия, №1.41} Прямая,  проведенная  через  вершину  $A$  треугольника $ABC$ перпендикулярно его медиане $BD$, 
делит эту медиану пополам. Найдите отношение сторон $AB$ и $AC$.
		\item \source{Гордин Р.К. Планиметрия, №1.42} \setnumber{3} \setanswer{4}
\begin{ex}
	Какому из чисел $\dfrac{1}{6}$, $\dfrac{5}{6}$, $\dfrac{10}{6}$, $\dfrac{13}{6}$ соответствует точка $A$?
	\graphcenter{graphs/graph_3/graph_3}
	
	\selectanswer
	\begin{multicols}{4}
		\begin{enumerate}[label=\arabic*)]
			\item $\dfrac{5}{6}$
			\item $\dfrac{1}{6}$
			\item $\dfrac{10}{6}$
			\item $\dfrac{13}{6}$
		\end{enumerate}
	\end{multicols}
\end{ex}
		\item \source{Гордин Р.К. Планиметрия, №1.34} Докажите, что в равных треугольниках соответствующие медианы равны.
		\item \source{Гордин Р.К. Планиметрия, №1.35} Какое из чисел отмечено на координатной прямой точкой $A$?
\begin{center}
	\includegraphics[align=t,]{graphs/graph_5/graph_5}
\end{center}

\textit{В ответе укажите номер правильного варианта.}
\begin{multicols}{4}
	\begin{enumerate}[label=\arabic*)]
		\item $\sqrt{4}$
		\item $\sqrt{1}$
		\item $\sqrt{2}$
		\item $\sqrt{5}$
	\end{enumerate}
\end{multicols}
		\item \source{Гордин Р.К. Планиметрия, №1.37} Докажите, что биссектриса равнобедренного треугольника, проведенная из вершины, является также медианой
и высотой.
		\item \source{Гордин Р.К. Планиметрия, №1.38} \setnumber{7} \setanswer{3}
\begin{ex}
	Какому промежутку принадлежит число $\sqrt{37}$?
	
	\selectanswer
	\begin{multicols}{4}
		\begin{enumerate}[label=\arabic*)]
			\item $[4;5]$
			\item $[3;4]$
			\item $[6;7]$
			\item $[2;3]$
		\end{enumerate}
	\end{multicols}
\end{ex}
		\item \source{Гордин Р.К. Планиметрия, №1.46} В треугольнике $ABC$ медиана $AM$ продолжена за точку $M$ на расстояние, равное $AM$. Найдите расстояние от полученной точки до вершин $B$ и $C$, если $AB = 7$, $AC = 11$.
		\item \source{Гордин Р.К. Планиметрия, №1.47} \setnumber{9} \setanswer{4}
\begin{ex}
	Какому промежутку принадлежит число $3\sqrt{5}$?
	
	\selectanswer
	\begin{multicols}{4}
		\begin{enumerate}[label=\arabic*)]
			\item $[3;4]$
			\item $[5;6]$
			\item $[7;8]$
			\item $[6;7]$
		\end{enumerate}
	\end{multicols}
\end{ex}
		\item \source{Гордин Р.К. Планиметрия, №1.50} Докажите признаки равенства прямоугольных тре-
угольников:
\begin{enumerate}[label=\asbuk*)]
	\item по двум катетам;
	\item по катету и гипотенузе;
	\item по катету и прилежащему острому углу;
	\item по гипотенузе и острому углу.
\end{enumerate}
		\item \source{Гордин Р.К. Планиметрия, №1.51} \input{triangles/ex_11}
		\item \source{Гордин Р.К. Планиметрия, №1.52} Какому промежутку принадлежит число \(5\sqrt{5}\)?

\textit{В ответе укажите номер правильного варианта.}
\begin{multicols}{4}
	\begin{enumerate}[label=\arabic*)]
		\item $[8;9]$
		\item $[9;10]$
		\item $[11;12]$
		\item $[10;11]$
	\end{enumerate}
\end{multicols}
		\item \source{Ткалич А.А.} \input{triangles/ex_13}
		\item \source{Гордин Р.К. Планиметрия, №1.54} Докажите признак равенства прямоугольных треугольников по катету и противолежащему углу.
		\item \source{Гордин Р.К. Планиметрия, №1.57} Докажите, что в равных треугольниках соответствующие высоты равны между собой.
		\item \source{Гордин Р.К. Планиметрия, №1.58} \input{triangles/ex_16}
		\item \source{Гордин Р.К. Планиметрия, №1.52} Докажите, что диагонали четырехугольника с равными сторонами взаимно перпендикулярны.
		\item \source{Гордин Р.К. Планиметрия, №1.60} \input{triangles/ex_18}
		\item \source{Гордин Р.К. Планиметрия, №1.61} Две высоты треугольника равны между собой. Докажите, что треугольник равнобедренный.
		\item \source{Гордин Р.К. Планиметрия, №1.62} \input{triangles/ex_20}
		\item \source{Гордин Р.К. Планиметрия, №1.63} Найдите геометрическое место внутренних точек угла, равноудаленных от его сторон.
		\item \source{Гордин Р.К. Планиметрия, №1.64} Докажите, что биссектриса угла является его осью симметрии.
		\item \source{Гордин Р.К. Планиметрия, №1.65} Через вершины $A$ и $C$ треугольника $ABC$ проведены прямые, перпендикулярные биссектрисе угла $ABC$, пересекающие прямые $CB$ и $BA$ в точках $K$ и $M$ соответственно. Найдите $AB$, если $BM = 8$, $KC = 1$.
		\item \source{Гордин Р.К. Планиметрия, №1.66} Через данную точку проведите прямую, пересекающую две данные прямые под равными углами.
	\end{enumerate}
\subsection{Параллельность}
\subsection{Окружность}
\end{document}