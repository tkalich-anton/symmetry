\documentclass[10pt, a4paper]{article}
\usepackage{../../../style}
\lhead{\leftmark}
\setkeys{exercise}{showNumber=true,showAnswer=true,bankPath=../../bank}
\begin{document}
\section{Целые уравнения}
	\subsection{Линейные уравнения}
		\mexercise{_1}
		\mexercise{_2}
		\mexercise{_3}
		\mexercise{_4}
		\mexercise{_7}
		\mexercise{_5}
		\mexercise{_6}
		\mexercise{_8}
		\mexercise{_17}
		\mexercise{_18}
		\mexercise{_19}
%	\subsection{Квадратные уравнения}
%		\subsubsection{Неполные квадратные уравнения}
%			\mexercise{_10}
%			\mexercise{_12}
%			\mexercise{_11}
%			\mexercise{_13}
%			\mexercise{_14}
%			\mexercise{_15}
%		\subsubsection{Квадратные уравнения общего вида}
%			\mexercise{_9}
%			\mexercise{_16}
%			\mexercise{_20}
%			\mexercise{_21}
%	\subsection{Уравнения высших степеней}
%		\subsubsection{Биквадратные уравнения}
%			\mexercise{_22}
%			\exercise{34}
%		\subsubsection{Распадающиеся уравнения}
%			\mexercise{_23}
%			\mexercise{_24}
%			\mexercise{_56}
%			\mexercise{_55}
%			\mexercise{_57}
%		\subsubsection{Однородные уравнения}
%			\mexercise{_58}
%		\subsubsection{Симметрические уравнения}
%			\mexercise{_51}
%			\mexercise{_52}
%		\subsubsection{Другие замены}
%			\mexercise{_53}
%			\mexercise{_54}
%	\subsection{Целые уравнения с модулем}
%		\mexercise{_103}
\section{Дробные уравнения}
	\mexercise{_104}
%\section{Иррациональные уравнения}
%	\mexercise{_106}
%\section{Тригонометрические уравнения}
%	\mexercise{_105}
%\section{Показательные уравнения}
%	\mexercise{_108}
%\section{Логарифмические уравнения}
%	\mexercise{_29}
%	\mexercise{_107}
%\section{Смешанные уравнения}
%	\mexercise{_109} 
%\section{Применение свойств функций}
%	\subsection{Монотонность функций}
%	\mexercise{_110}
%	\subsection{Ограниченность функций}
%	\mexercise{_111}
%\section{Неразобранные задачи}
%	\exercise{3763}
%	\exercise{3764} 
\end{document}