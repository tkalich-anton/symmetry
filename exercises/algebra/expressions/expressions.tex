\documentclass[10pt, a4paper]{article}
\usepackage{../../../style}
\lhead{\leftmark}
\setkeys{exercise}{showNumber=true,showAnswer=true,bankPath=../../bank}
\begin{document}
\section{Целые выражения}
	\subsection{Разложение на множители}
		\exercise{664}
		\exercise{665}
		\exercise{666}
		\exercise{667}
		\exercise{656}
		\exercise{657}
		\exercise{658}
		\exercise{659}
		\exercise{660}
		\exercise{661}
		\exercise{662}
		\exercise{663}
	\subsection{Другие}
		\exercise{1226}
		\exercise{1228}
		\exercise{1292}
		\exercise{1304}
		\exercise{1310}
		\exercise{1311}
		\exercise{1312}
		\exercise{1321}
		\exercise{1322}
		\exercise{1323}
\section{Дробные выражения}
	\subsection{Упрощение алгебраической дроби}
		\mexercise{_30}
		\mexercise{_31}
		\mexercise{_32}
		\mexercise{_33}
		\mexercise{_34}
		\mexercise{_36}
	\subsection{Сложение и вычитание дробей с одинаковыми знаменателями}
		\mexercise{_35}
		\mexercise{_37}
		\mexercise{_38}
		\mexercise{_39}
	\subsection{Сложение и вычитание дробей с разными знаменателями}
		\mexercise{_40}
		\mexercise{_41}
		\mexercise{_42}
		\mexercise{_43}
		\mexercise{_44}
		\mexercise{_45}
		\mexercise{_46}
		\mexercise{_47}
	\subsection{Произведение дробей}
		\mexercise{_48}
	\subsection{Упрощение дробных выражений}
		\mexercise{_49}
		\mexercise{_50}
		\exercise{1225}
		\exercise{1093}
		\exercise{1094}
		\exercise{1095}
		\exercise{1096}
		\exercise{1114}
		\exercise{1115}
		\exercise{1223}
		\exercise{1225}
		\exercise{1302}
		\exercise{1303}
		\exercise{1308}
		\exercise{1309}
		\exercise{1314}
		\exercise{1315}
		\exercise{1316}
		\exercise{1317}
		\exercise{1318}
		\exercise{1319}
		\exercise{1320}
\section{Иррациональные выражения}
	\exercise{17}
	\exercise{775}
	\exercise{1105}
	\exercise{1106}
	\exercise{1107}
	\exercise{1108}
	\exercise{1227}
	\exercise{1327}
	\exercise{1328}
	\exercise{1333}
	\exercise{1334}
	\exercise{1336}
	\exercise{1337}
	\exercise{1338}
	\exercise{1339}
	\exercise{1335}
	\exercise{1101}
	\exercise{1102}
	\exercise{1103}
	\exercise{1104}
\section{Показательные выражения}
	\exercise{1113}
	\exercise{1224}
	\exercise{1229}
	\exercise{1230}
	\exercise{1231}
	\exercise{1290}
	\exercise{1291}
\section{Тригонометрические выражения}
	\exercise{1116}
	\exercise{1117}
	\exercise{1118}
	\exercise{1119}
	\exercise{1134}
	\exercise{1147}
	\exercise{1148}
\end{document}