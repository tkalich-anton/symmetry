%Группа 8-2 Модуль 1
\title{Занятие №1}
\begin{listofex}
	\item Упростить дробь:
	\begin{enumcols}[itemcolumns=3]
		\item \exercise{20}
		\item \exercise{49}
		\item \exercise{54}
		\item \exercise{55}
		\item \exercise{60}
		\item \exercise{61}
	\end{enumcols}
	\item Упростить дробь:
	\begin{enumcols}[itemcolumns=4]
		\item \exercise{63}
		\item \exercise{65}
		\item \exercise{67}
		\item \exercise{70}
	\end{enumcols}
	\item Упростить дробь:
	\begin{enumcols}[itemcolumns=3]
		\item \exercise{73}
		\item \exercise{74}
		\item \exercise{75}
		\item \exercise{78}
		\item \exercise{80}
	\end{enumcols}
	\item Упростить дробь:
	\begin{enumcols}[itemcolumns=5]
		\item \exercise{83}
		\item \exercise{86}
		\item \exercise{1346}
		\item \exercise{93}
		\item \exercise{97}
	\end{enumcols}
	\item Упростить дробь:
	\begin{enumcols}[itemcolumns=3]
		\item \exercise{99}
		\item \exercise{103}
		\item \exercise{107}
	\end{enumcols}
	\[ \begin{array}{cccc}
		\text{Разность квадратов}&(a+b)(a-b)& =&a^2-b^2,\\
		\text{Квадрат суммы}&(a+b)^2& =&a^2+2ab+b^2,\\
		\text{Квадрат разности}&(a-b)^2& =&a^2-2ab+b^2,\\
		\text{Сумма кубов}&(a+b)(a^2-ab+b^2)& =&a^3+b^3,\\
		\text{Разность кубов}&(a-b)(a^2+ab+b^2)& =&a^3-b^3,\\
		\text{Куб суммы}&(a+b)^3& =&a^3+3a^2b+3ab^2+b^3,\\
		\text{Куб разности}&(a-b)^3& =&a^3-3a^2b+3ab^2-b^3.\\
	\end{array} \]
	\item Упростить дробь:
	\begin{enumcols}[itemcolumns=3]
		\item \exercise{109}
		\item \exercise{113}
		\item \exercise{118}
		\item \exercise{121}
		\item \exercise{122}
		\item \exercise{126}
	\end{enumcols}
\end{listofex}
\newpage
\title{Занятие №2}
\[ \begin{array}{cccc}
	\text{Разность квадратов}&(a+b)(a-b)& =&a^2-b^2,\\
	\text{Квадрат суммы}&(a+b)^2& =&a^2+2ab+b^2,\\
	\text{Квадрат разности}&(a-b)^2& =&a^2-2ab+b^2,\\
	\text{Сумма кубов}&(a+b)(a^2-ab+b^2)& =&a^3+b^3,\\
	\text{Разность кубов}&(a-b)(a^2+ab+b^2)& =&a^3-b^3,\\
	\text{Куб суммы}&(a+b)^3& =&a^3+3a^2b+3ab^2+b^3,\\
	\text{Куб разности}&(a-b)^3& =&a^3-3a^2b+3ab^2-b^3.\\
\end{array} \]
\begin{listofex}
	\item Упростить дробь:
	\begin{enumcols}[itemcolumns=2]
		\item \exercise{51}
		\item \exercise{56}
		\item \exercise{57}
		\item \exercise{64}
		\item \exercise{76}
		\item \exercise{77}
	\end{enumcols}
	\item Упростить дробь:
	\begin{enumcols}[itemcolumns=2]
		\item \exercise{118}
		\item \exercise{121}
		\item \exercise{122}
		\item \exercise{126}
		\item \exercise{1351}
		\item \exercise{1353}
		\item \exercise{1371}
		\item \exercise{1374}
	\end{enumcols}
	\item Вычислить значение выражения:
	\item \exercise{642}
	\item Представить в виде несократимой дроби:
	\begin{enumcols}[itemcolumns=3]
		\item \exercise{130}
		\item \exercise{136}
		\item \exercise{827}
		\item \exercise{829}
		\item \exercise{835}
	\end{enumcols}
	\item Представить в виде несократимой дроби:
	\begin{enumcols}[itemcolumns=3]
		\item \exercise{837}
		\item \exercise{842}
		\item \exercise{844}
	\end{enumcols}
	%\item Представить в виде несократимой дроби:
	%\begin{enumcols}[itemcolumns=3]
	%	\item \exercise{848}
	%	\item \exercise{851}
	%	\item \exercise{855}
	%\end{enumcols}
	
	\newpage
	\item Представить в виде несократимой дроби:
	\begin{enumcols}[itemcolumns=2]
		\item \exercise{839}
		\item \exercise{846}
		\item \exercise{847}
		\item \exercise{850}
		\item \exercise{855}
		\item \exercise{856}
	\end{enumcols}
	\item Представить в виде несократимой дроби:
	\begin{enumcols}[itemcolumns=2]
		\item \exercise{858}
		\item \exercise{862}
		\item \exercise{864}
		\item \exercise{867}
		\item \exercise{871}
	\end{enumcols}
	\item Представить в виде несократимой дроби:
	\begin{enumcols}[itemcolumns=2]
		\item \exercise{874}
		\item \exercise{877}
		\item \exercise{879}
		\item \exercise{881}
		\item \exercise{884}
		\item \exercise{885}
		\item \exercise{888}
		\item \exercise{890}
	\end{enumcols}
	\item Представить в виде несократимой дроби:
	\begin{enumcols}[itemcolumns=3]
		\item \exercise{892}
		\item \exercise{894}
		\item \exercise{896}
	\end{enumcols}
\end{listofex}
\newpage
\title{Домашняя работа №1}
\begin{listofex}
	\item Упростить дробь:
	\begin{enumcols}[itemcolumns=2]
		\item \exercise{53}
		\item \exercise{59}
		\item \exercise{66}
		\item \exercise{69}
		\item \exercise{79}
		\item \exercise{81}
	\end{enumcols}
	\item \exercise{1223}
	\item Представить в виде несократимой дроби:
	\begin{enumcols}[itemcolumns=2]
		\item \exercise{826}
		\item \exercise{832}
		\item \exercise{836}
	\end{enumcols}
	\item Упростить дробь:
	\begin{enumcols}[itemcolumns=2]
		\item \exercise{1352}
		\item \exercise{1372}
		\item \exercise{1373}
		\item \exercise{1375}
		\item \exercise{1376}
		\item \exercise{1377}
	\end{enumcols}
	
\end{listofex}
\newpage

\title{Занятие №3}
\begin{listofex}
	\item 1
	
\end{listofex}
\newpage
\title{Занятие №4}
\begin{listofex}
	\item 1
	
\end{listofex}
\newpage
\title{Домашняя работа №2}
\begin{listofex}
	\item 1
	
\end{listofex}
\newpage
\title{Занятие №5}
\begin{listofex}
	\item 1
	
\end{listofex}
\newpage
\title{Занятие №6}
\begin{listofex}
	\item 1
	
\end{listofex}
\newpage
\title{Домашняя работа №3}
\begin{listofex}
	\item 1
	
\end{listofex}
\newpage
\title{Занятие №7}
\begin{listofex}
	\item 1
	
\end{listofex}
\newpage
\title{Занятие №8}
\begin{listofex}
	\item 1
	
\end{listofex}
\newpage
\title{Домашняя работа №4}
\begin{listofex}
	\item 1
	
\end{listofex}
\newpage
\title{Проверочная работа}
\begin{listofex}
	\item 1
	
\end{listofex}