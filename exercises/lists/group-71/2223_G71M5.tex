%
%===============>>  ГРУППА 7-1 МОДУЛЬ 5  <<=============
%
\setmodule{5}

%BEGIN_FOLD % ====>>_____ Занятие 1 _____<<====
\begin{class}[number=1]
	\begin{listofex}
		\item Найти значение выражения:
		\begin{itasks}[1]
			\task \( \dfrac{2}{3}(\dfrac{3}{5}+1)+(\mfrac{1}{1}{2}+\mfrac{3}{4}{5}) \)
			\task \( (5+\dfrac{2}{3})(-3-5)-(\mfrac{1}{1}{2}-5)^2 \)
			\task \( (1+\dfrac{6}{5})(\dfrac{1}{3}+\dfrac{6}{5})+(\mfrac{2}{3}{4}+1) \)
		\end{itasks}
		\item Преобразовать в многочлен стандартного вида:
		\begin{itasks}[1]
			\task \( (2x-3)^2-(x+2)(3x-1)-(4x+3)(3-4x) \)
			\task \( (3-4x)(2x+1)-(2x+5)(5-2x)-(2-7x)^2 \)
			\task \( (5-6x)(2x-1)-(4x+3)^2-(5x+1)(1-5x) \)
			\task \( (5x+3)(3x-5)-(x+2)(2x+3)-(3x-5)^2 \)
		\end{itasks}
		\item Решите уравнение:
		\begin{itasks}[1]
			\task \( (5x-1)(x-8)-(x-8)(5x-1)=0 \)
			\task \( (2x+3)(6x-5)-(3x+3)(4x-4)=0 \)
			\task \( (x-7)(x-9)-(x-9)(-x-5)=0 \)
			\task \( (3x+1)(8x-9)-(4x+1)(6x+5)=0 \)
			\task \( (5-x)(8x-1)=-(4x+1)(2x+1) \)
			\task \( (4-7x)(6x-1)=(3x+1)(3-14x) \)
		\end{itasks}
		
%		\item Решите уравнение:
%		\begin{itasks}[1]
%			\task \( (-\dfrac{2}{3}x+1)\dfrac{3}{2}x+\dfrac{1}{4}x=0 \)
%			
%			\task \( (\dfrac{4}{5}x-3)(\dfrac{2}{3}x+\dfrac{6}{7})-(\dfrac{1}{2}x-2) (-\dfrac{2}{3}x+1)=0 \)
%			
%			\task \( (\dfrac{4}{5}x+1)(5x+\dfrac{5}{2})-(\dfrac{1}{3}x+6) (\dfrac{3}{2}x-7)=0 \)
%			
%			\task \( (4-\dfrac{1}{6}x)(6x-\dfrac{2}{5})=(\dfrac{2}{28}x+1)(\mfrac{3}{4}{7}-14x) \)
%			
%			\task \( \dfrac{2}{3}x(8x-1)=\mfrac{2}{2}{3}(x-6)(2x+1) \)
%		\end{itasks}
	\end{listofex}
\end{class}
%END_FOLD

%BEGIN_FOLD % ====>>_____ Занятие 2 _____<<====
\begin{class}[number=2]
	\begin{listofex}
		\item Занятие 2
	\end{listofex}
\end{class}
%END_FOLD

%BEGIN_FOLD % ====>>_____ Занятие 3 _____<<====
\begin{class}[number=3]
	\begin{listofex}
		\item Занятие 3
	\end{listofex}
\end{class}
%END_FOLD

%BEGIN_FOLD % ====>>_____ Занятие 4 _____<<====
\begin{class}[number=4]
	\begin{listofex}
		\item Занятие 4
	\end{listofex}
\end{class}
%END_FOLD

%BEGIN_FOLD % ====>>_____ Занятие 5 _____<<====
\begin{class}[number=5]
	\begin{listofex}
		\item Занятие 5
	\end{listofex}
\end{class}
%END_FOLD

%BEGIN_FOLD % ====>>_____ Занятие 6 _____<<====
\begin{class}[number=6]
	\begin{listofex}
		\item Занятие 6
	\end{listofex}
\end{class}
%END_FOLD

%BEGIN_FOLD % ====>>_____ Занятие 7 _____<<====
\begin{class}[number=7]
	\begin{listofex}
		\item Занятие 7
	\end{listofex}
\end{class}
%END_FOLD

%BEGIN_FOLD % ====>>_ Домашняя работа 1 _<<====
\begin{homework}[number=1]
	\begin{listofex}
		%1
		\item Найти значение выражения:
		\begin{itasks}[1]
			\task \( \dfrac{1}{3}(\dfrac{4}{5}+3)-(\dfrac{1}{2}+\mfrac{3}{2}{3}) \)
			\task \( (1+\dfrac{1}{6})-(\dfrac{5}{2}-5)^2 \)
			\task \( (2+\dfrac{1}{2})(\dfrac{1}{4}+\dfrac{6}{5})+(\mfrac{1}{1}{4}-1) \)
			\task \( (1+\dfrac{2}{3})-(\dfrac{5}{2}-5)\dfrac{1}{3} \)
		\end{itasks}
		%2
		\item Решите уравнение:
		\begin{itasks}[1]
			\task \( (3x+4)^2=(3x-2)(2+3x) \)
			\task \( (1-6x)(2x-5)=(3x+4)(-4x-1) \)
			\task \( (2-5x)^2=(5x+4)(4-5x)\)
			\task \( (4x+1)^2=(16x-1)(x+2) \)
		\end{itasks}
		%3
		\item Преобразовать в многочлен стандартного вида:
		\begin{itasks}[1]
			\task \( (6x+5)(5-6x)-(x-6)^2-(x+7)(2x-7)=0 \)
			\task \( (4x-1)(x-3)-(4-5x)^2-(6x+5)(5x-6)=0 \)
			\task \( (7-2x)(x+2)-(6-x)(x+6)-(-2x-3)^2=0 \)
			\task \( (-4x+1)(1+4x)-(3x-2)(2x+3)-(-3x-5)^2=0 \)
		\end{itasks}
		%4
		\item Решите уравнение:
		\begin{itasks}[1]
			\task \( (\dfrac{1}{4}x+4)\dfrac{1}{2}x+\dfrac{1}{8}x=0 \)
			\task \( (-\dfrac{4}{5}x+1)(\dfrac{4}{5}x+\dfrac{1}{2})- \dfrac{1}{2}x (\dfrac{4}{3}x+12)=0 \)
			\task \( \dfrac{1}{2}x(-3-4x)=\dfrac{1}{2}(-x-6)(2x+1) \)
		\end{itasks} 
	\end{listofex}
\end{homework}
%END_FOLD

%BEGIN_FOLD % ====>>_ Домашняя работа 2 _<<====
\begin{homework}[number=2]
	\begin{listofex}
		\item ДЗ 2
	\end{listofex}
\end{homework}
%END_FOLD

%BEGIN_FOLD % ====>>_ Домашняя работа 3 _<<====
\begin{homework}[number=3]
	\begin{listofex}
		\item ДЗ 3
	\end{listofex}
\end{homework}
%END_FOLD

%BEGIN_FOLD % ====>>_ Проверочная работа _<<====
\begin{exam}
	\begin{listofex}
		\item Проверочная работа
	\end{listofex}
\end{exam}
%END_FOLD
