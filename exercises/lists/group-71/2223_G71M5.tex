%
%===============>>  ГРУППА 7-1 МОДУЛЬ 5  <<=============
%
\setmodule{5}
%BEGIN_FOLD % ====>>_____ Занятие 1 _____<<====
\begin{class}[number=1]
	\begin{listofex}
		\item БЫЛ РАЗБОР ПРОВЕРОЧНОЙ
	\end{listofex}
\end{class}
%END_FOLD
%BEGIN_FOLD % ====>>_____ Занятие 2 _____<<====
\begin{class}[number=2]
	\begin{listofex}
		\item Найти значение выражения:
		\begin{itasks}[1]
			\task \( \dfrac{2}{3}(\dfrac{3}{5}+1)+(\mfrac{1}{1}{2}+\mfrac{3}{4}{5}) \)
			\task \( (5+\dfrac{2}{3})(-3-5)-(\mfrac{1}{1}{2}-5)^2 \)
			\task \( (1+\dfrac{6}{5})(\dfrac{1}{3}+\dfrac{6}{5})+(\mfrac{2}{3}{4}+1) \)
		\end{itasks}
		\item Преобразовать в многочлен стандартного вида:
		\begin{itasks}[1]
			\task \( (2x-3)^2-(x+2)(3x-1)-(4x+3)(3-4x) \)
			\task \( (3-4x)(2x+1)-(2x+5)(5-2x)-(2-7x)^2 \)
			\task \( (5-6x)(2x-1)-(4x+3)^2-(5x+1)(1-5x) \)
			\task \( (5x+3)(3x-5)-(x+2)(2x+3)-(3x-5)^2 \)
		\end{itasks}
		\item Решите уравнение:
		\begin{itasks}[1]
			\task \( (5x-1)(x-8)-(x-8)(5x-1)=0 \)
			\task \( (2x+3)(6x-5)-(3x+3)(4x-4)=0 \)
			\task \( (x-7)(x-9)-(x-9)(-x-5)=0 \)
			\task \( (3x+1)(8x-9)-(4x+1)(6x+5)=0 \)
			\task \( (5-x)(8x-1)=-(4x+1)(2x+1) \)
			\task \( (4-7x)(6x-1)=(3x+1)(3-14x) \)
		\end{itasks}
	\end{listofex}
\end{class}
%END_FOLD

%BEGIN_FOLD % ====>>_ Домашняя работа 1 _<<====
\begin{homework}[number=1]
	\begin{listofex}
		\item Вычислить: \( \left(  1,5 : \dfrac{1}{3} - \dfrac{3}{8} : 0,25 \right) \cdot 3,2 - 3,2 \cdot \dfrac{5}{8} \)
		\item Применить формулу квадрата суммы:
		\begin{tasks}(2)
			\task \( (2x+5)^2 \)
			\task \( (3x^2+1)^2 \)
			\task \( \left( \dfrac{x}{3}+4y \right)^2 \)
			\task \( (0,25x+0,1)^2 \)
		\end{tasks}
		\item Решите уравнение: \( (3x+4)^2=(3x-2)(2+3x) \)
	\end{listofex}
\end{homework}
%END_FOLD

%BEGIN_FOLD % ====>>_____ Занятие 3 _____<<====
\begin{class}[number=3]
	\begin{listofex}
		\item Вычислить: \[ \left(  1,5 : \dfrac{1}{3} - \dfrac{3}{8} : 0,25 \right) \cdot 3,2 - 3,2 \cdot \dfrac{5}{8} \]
		\item Решите уравнение: \[ (x-7)(x-9)-(x-9)(-x-5)=0 \]
	\end{listofex}
	\textbf{Формулы квадрата суммы и квадрата разности:}\\
	\begin{tasks}(2)
		\task \( (a+b)^2=a^2+2ab+b^2 \)
		\task \( (a-b)^2=a^2-2ab+b^2 \)
	\end{tasks}
	\begin{listofex}[resume]
		\item Применить формулу квадрата суммы:
		\begin{tasks}(4)
			\task \( (x+1)^2 \)
			\task \( (2x+3)^2 \)
			\task \( (5x+10)^2 \)
			\task \( (2x+3y)^2 \)
			\task \( (12+4x)^2 \)
			\task \( (x^2+3)^2 \)
			\task \( \left( \dfrac{x}{2}+2y \right)^2 \)
			\task \( (0,1x+3,5)^2 \)
		\end{tasks}
		\item Применить формулу квадрата разности:
		\begin{tasks}(3)
			\task \( (x-7)^2 \)
			\task \( (4x-3y)^2 \)
			\task \( (15x-11)^2 \)
			\task \( (2x+3y)^2 \)
			\task \( \left( \dfrac{1}{5}x-y \right)^2 \)
			\task \( (0,25x-1,7)^2 \)
		\end{tasks}
		\item Применить формулу квадрата суммы/разности и привести подобные слагаемые:
		\begin{tasks}(2)
			\task \( (x+9)^2+3x^2-18x \)
			\task \( (2x-1)^2+(2x+1)^2 \)
			\task \( (12x-2y)^2-(5x+y)^2 \)
			\task \( (3x^3+1)^2-6x^3 \)
		\end{tasks}
	\end{listofex}
\end{class}
%END_FOLD

%BEGIN_FOLD % ====>>_____ Занятие 4 _____<<====
\begin{class}[number=4]
	\begin{listofex}
		\item Вычислить: \( \left(  3,6 \cdot \mfrac{2}{7}{9} + 1,125 + \mfrac{5}{2}{5} \cdot  \mfrac{2}{7}{9} - \mfrac{1}{1}{8} \right) : 2,5 \)
		\item Применить формулу квадрата суммы:
		\begin{tasks}(2)
			\task \( (2x-1)^2 \)
			\task \( (2x+5)^2 \)
			\task \( (6x-7)^2 \)
			\task \( (3x^2-2y^4)^2 \)
			\task \( (11+10xy^2)^2 \)
			\task \( (0,5x-2)^2 \)
			\task \( \left( \dfrac{x}{3}+4y \right)^2 \)
			\task \( (0,25x^2+4,8)^2 \)
		\end{tasks}
		\item Применить формулу квадрата суммы/разности и привести подобные слагаемые:
		\begin{tasks}(2)
			\task \( (4x+1)^2+4x^2-15x \)
			\task \( (3x-4)^2+(x+5)^2 \)
			\task \( (9x-y)^2-(4x+y)^2 \)
			\task \( (3x^3+4)^2-9x^6 \)
		\end{tasks}
		\item Упростить выражение:
		\[ (x-5)^2+(2x-3)^2-(2x+4)^2-x^2 \]
	\end{listofex}
\end{class}
%END_FOLD

%BEGIN_FOLD % ====>>_ Домашняя работа 2 _<<====
\begin{homework}[number=2]
	\begin{listofex}
		\item Вычислить: \( 20 : \mfrac{33}{1}{3} - \left( \mfrac{4}{7}{25} - 1,28 \right) : \left( 0,75 + \mfrac{3}{1}{4} \right) \cdot 0,2 \)
		\item Решите уравнение: \( (1-6x)(2x-5)=(3x+4)(-4x-1) \)
	\end{listofex}
\end{homework}
%END_FOLD

%BEGIN_FOLD % ====>>_____ Занятие 5 _____<<====
\begin{class}[number=5]
	\begin{listofex}
		\item Преобразуйте в многочлен стандартного вида:
		\begin{tasks}(2)
			\task \( (m+n)^2+(m-n)^2 \)
			\task \( 5(x-y)^2+(x-2y)^2 \)
			\task \( 2(a-1)^2+3(a-2)^2 \)
			\task \( 4(3x+4y)^2-7(2x-3y)^2 \)
			\task! \( 2(p-3q)^2-4(2p-q)^2-(2q-3p)(p+q) \)
		\end{tasks}
		\item Представьте в виде квадрата сумма или разности:
		\begin{tasks}(2)
			\task \( a^2-2ab+b^2 \)
			\task \( 9m^2-6m+1 \)
			\task \( 4x^2-4xy+y^2 \)
			\task \( 16+9x^6-24x^3 \)
		\end{tasks}
	\end{listofex}
	\textbf{Формула разности квадратов:}
	\[ a^2-b^2=(a-b)(a+b) \]
	\begin{listofex}
		\item Разложите на множители:
		\begin{tasks}(2)
			\task \( a^2-y^2 \)
			\task \( 4a^2-y^2 \)
			\task \( 9x^2-16 \)
			\task \( x^2-1 \)
			\task \( 16y^2-49x^2 \)
		\end{tasks}
		\item Представьте в виде многочлена:
		\begin{tasks}(2)
			\task \( (x+2y)(x-2y) \)
			\task \( (3m-n)(3m+n) \)
			\task \( (3m-n)(3m+n) \)
			\task \( (4p-1)(1+4p) \)
			\task \( (11a-13b)(11a+13b) \)
		\end{tasks}
		\item Вычислить: \( \dfrac{7}{40} : \mfrac{2}{11}{12} - 0,1 \cdot \left( 1,45 : \mfrac{2}{1}{3} - \dfrac{1}{20} : \mfrac{2}{1}{3} \right) \)
	\end{listofex}
\end{class}
%END_FOLD

%BEGIN_FOLD % ====>>_____ Занятие 6 _____<<====
\begin{class}[number=6]
	\begin{listofex}
		\item Преобразуйте в многочлен стандартного вида:
		\begin{tasks}(2)
			\task \( (2a+b)^2+(2a-b)^2 \)
			\task \( 3(x-y)^2+4(x-4y)^2 \)
			\task \( 2(a-1)^2-3(a-3)^2 \)
			\task \( (2x^2-3)^2+(3x+1)^2 \)
			\task \( (11x^2y^3+x^3)^2 \)
		\end{tasks}
		\item Преобразуйте в многочлен стандартного вида:
		\begin{tasks}(2)
			\task \( \left( x+\dfrac{1}{2} \right)^2 \)
			\task \( \left( \dfrac{a}{2}+\dfrac{b}{3} \right)^2 \)
			\task \( \left( \dfrac{1}{2}x^3-\dfrac{3}{5}y^2 \right)^2 \)
			\task \( (0,25x^2-1,2x)^2 \)
		\end{tasks}
		\item Примените формулу разности квадратов и разложите на множители выражение:
		\begin{tasks}(3)
			\task \( x^2-4 \)
			\task \( x^2-1 \)
			\task \( a^2b^2-9 \)
			\task \( (ab)^2-16 \)
			\task \( 4x^2-25 \)
			\task \( 16x^6-100y^2 \)
			\task \( 36x^{10}y^8-81a^4b^2 \)
			\task \( \dfrac{4}{9}-x^4 \)
			\task \( \dfrac{1}{16}m^2-\dfrac{25}{9}n^2 \)
		\end{tasks}
		\item Представьте в виде многочлена:
		\begin{tasks}(2)
			\task \( (x-2a)(x+2a) \)
			\task \( (2y-3)(3+2y) \)
			\task \( (x^2-12)(x^2+12) \)
			\task \( (21a-1)(1+21a) \)
			\task \( (0,2x-3,1)(0,2x+3,1) \)
			\task \( (1,45y^2+2,05)(1,45y^2-2,05) \)
			\task \( \left( \dfrac{1}{2}-2x \right)\left( \dfrac{1}{2}+2x \right) \)
		\end{tasks}
		\item Представьте в виде квадрата суммы или квадрата разности:
		\begin{tasks}(2)
			\task \( x^2+2xy+y^2 \)
			\task \( a^2-2ab+b^2 \)
			\task \( x^2+2x+1 \)
			\task \( 16+8p+p^2 \)
			\task \( x^2-4ax+4a^2 \)
			\task \( 16p^2+40pq+25q^2 \)
			\task \( a^6+2a^3b^3+b^6 \)
		\end{tasks}
		\item Подставьте вместо \( C \) и \( D \) такие одночлены, чтобы выполнялось равенство:
		\[ (a + C)^2 = a^2 + 6ab^2 + D \]
	\end{listofex}
\end{class}
%END_FOLD

%BEGIN_FOLD % ====>>_____ Занятие 7 _____<<====
\begin{class}[number=7]
	\begin{listofex}
		\item Занятие 7
	\end{listofex}
\end{class}
%END_FOLD

%BEGIN_FOLD % ====>>_ Домашняя работа 3 _<<====
\begin{homework}[number=3]
	\begin{listofex}
		\item Вычислить: \( \left( \mfrac{5}{3}{11} + \mfrac{3}{7}{22} - 8,25 \right) \cdot 1,76 : 1,875 + \left( \mfrac{1}{5}{6} + \mfrac{2}{4}{9} \right) \cdot \mfrac{2}{5}{11} : 5,25 \)
	\end{listofex}
\end{homework}
%END_FOLD

%BEGIN_FOLD % ====>>_ Проверочная работа _<<====
\begin{exam}
	\begin{listofex}
		\item Проверочная работа
	\end{listofex}
\end{exam}
%END_FOLD

%BEGIN_FOLD % ====>>_ Склад задач _<<====
\begin{consultation}
	\begin{listofex}
%		Не подходит под данную тему
		\item Найдите значение выражений \(x^2y+xy^2; x^2 + y^2; x^3 + y^3; x^4 + y^4; (x-y)^2\), если числа \(x\) и \(y\) таковы, что:
		\begin{itasks}[2]
			\task \(xy=-3; x+y=5\)
			\task \(xy=20; x+y=-3\)
			\task \(xy=-1; x+y=1\)
			\task \(xy=4; x+y=12\)
		\end{itasks}

%		\item Решите уравнение: \( (-\dfrac{2}{3}x+1)\dfrac{3}{2}x+\dfrac{1}{4}x=0 \)
%		\item Преобразовать в многочлен стандартного вида:
%		\begin{itasks}[1]
%			\task \( (6x+5)(5x-6)-(x-6)^2-(x+7)(2x-7) \)
%			\task \( (4x-1)(x-3)-(4-5x)^2-(6x+5)(5x-6) \)
%			\task \( (7-2x)(x+2)-(6-x)(x+6)-(-2x-3)^2 \)
%			\task \( (-4x+1)(1+4x)-(3x-2)(2x+3)-(3x-5)^2 \)
%		\end{itasks}
%		\item Решите уравнение:
%		\begin{itasks}[1]
%			\task \( (3x-5)(x+2)+(5-3x)(4x-1)=0 \)
%			\task \( (x-1)(5x-1)+(4x+5)(1-x)=0 \)
%			\task \( (3x+2)(x+6)-(3x-7)(-2-3x)=0 \)
%			\task \( (x-9)(-6x+5)-(6x-5)(6x-1)=0 \)
%			\task \( (9x-5)(3x-1)+(5-9x)(2x-3)=0 \)
%			\task \( (4x+7)(-2x-1)(2x+1)(-x+2)=0 \)
%		\end{itasks}

		
%		\item Разложите на множители:
%		\begin{itasks}[1]
%			\task \(2x+y+y^2-4x^2\)
%			\task \(a-3b+9b^2-a^2\)
%			\task \(a^3-ab-a^2b+a^2\)
%			\task \(x^2y-x^2-xy+x^3\)
%		\end{itasks}
%		\item Вычислить:
%		\begin{itasks}[1]
%			\task \( 4,5 + 0,5 \cdot (2,4 \cdot 1,375 - 1,64 : 0,8) : \mfrac{2}{1}{12} - \mfrac{1}{2}{7} \cdot 1,4 \)
%			\task \( \left(  8,96 : 0,8 + \mfrac{1}{1}{8} \cdot 0,8 \right) : 1,1 - \left( - \mfrac{2}{17}{36} + \mfrac{5}{7}{12} \right) \cdot 0,9 - \mfrac{4}{1}{3} : 2,6 \cdot 0,6 \)
%			\task \( 0,198 \cdot \mfrac{9}{1}{11} - \left(  2,56 + \dfrac{3}{4} - 2,56 - 0,125 \right) \cdot \mfrac{2}{2}{3} - \dfrac{1}{15} \)
%			\task \( \left( \mfrac{1}{2}{13} \cdot 0,42 + 0,78 \cdot \mfrac{1}{2}{13} \right) \cdot \mfrac{1}{4}{9} : 0,6 - 0,5 \cdot \mfrac{5}{2}{3} \)
%		\end{itasks}
%		\item Решите уравнение:
%		\begin{itasks}[1]
%			\task \( (-5x-1)^2 - (-3x+1)(2+x)-(5+2x)(2x-5) \)
%			\task \( (3x-1)^2=(2x-3)(1-3x) \)
%			\task \( (2x-5)(2-x)=(x-2)^2) \)
%			\task \( (-2x-3)^2=(2x+3)(2x-3) \)
%		\end{itasks}
%		\item Решите уравнение:
%		\begin{itasks}[1]
%			\task \( (\dfrac{1}{5}x+1)\dfrac{3}{4}x+\dfrac{1}{2}x=0 \)
%			\task \( (-\dfrac{2}{3}x+2)(\dfrac{1}{5}x-\dfrac{2}{3})+ \dfrac{2}{5}x (\dfrac{1}{3}x+2)=0 \)
%			\task \( \dfrac{3}{4}x(3-4x)=\dfrac{5}{6}(-3x-6)(2x+1) \)
%		\end{itasks}
%		\item Решите уравнение:
%		\begin{tasks}
%			\task \( (\dfrac{4}{5}x+1)(5x+\dfrac{5}{2})-(\dfrac{1}{3}x+6) (\dfrac{3}{2}x-7)=0 \)
%			\task \( (2-5x)^2=(5x+4)(4-5x)\)
%			\task \( (4x+1)^2=(16x-1)(x+2) \)
%		\end{tasks}
%		\item Преобразовать в многочлен стандартного вида:
%		\begin{itasks}[1]
%			\task \( (6x+5)(5-6x)-(x-6)(x-6)-(x+7)(2x-7)=0 \)
%			\task \( (4x-1)(x-3)-(4-5x)^2-(6x+5)(5x-6)=0 \)
%			\task \( (7-2x)(x+2)-(6-x)(x+6)-(-2x-3)^2=0 \)
%			\task \( (-4x+1)(1+4x)-(3x-2)(2x+3)-(-3x-5)^2=0 \)
%		\end{itasks}
%		\item Решите уравнение:
%		\begin{itasks}[1]
%			\task \( (\dfrac{1}{4}x+4)\dfrac{1}{2}x+\dfrac{1}{8}x=0 \)
%			\task \( (-\dfrac{4}{5}x+1)(\dfrac{4}{5}x+\dfrac{1}{2})- \dfrac{1}{2}x (\dfrac{4}{3}x+12)=0 \)
%			\task \( \dfrac{1}{2}x(-3-4x)=\dfrac{1}{2}(-x-6)(2x+1) \)
%		\end{itasks}
%		\item Решите уравнение:
%		\begin{itasks}[1]
%			\task \( (2-7x)(3x-5)-(-2x-3)(5-3x)=0 \)
%			\task \( (2x-9)(4x-5)+(-2x+9)(-5x+4)=0 \)
%			\task \( (2x-3)^2-(3-2x)(x-4)=0 \)
%			\task \( (2x-5)(x-3)-(3-x)^2=0 \)
%			\task \( (3x+2)(3x-2)=(-3x-2)^2 \)
%			\task \( (4-7x)(x-1)=(7x-4)(7x+4) \)
%		\end{itasks}
%		\item Разложите на множители:
%		\begin{itasks}[1]
%			\task \(3x+xy-x^2y-3y\)
%			\task \(a^2b-2b+ab^2-2a\)
%			\task \(2a^2-2b^2-a+b\)
%			\task \(x-y-3x^2+xy^2\)
%		\end{itasks}
%		\item Решите уравнение:
%		\begin{itasks}[1]
%			\task \( (\dfrac{5}{6}x-18=(15+\dfrac{2}{3}x)(-\dfrac{1}{6}x-5)) \)
%			\task \( (4-\dfrac{1}{6}x)(6x-\dfrac{2}{5})=(\dfrac{2}{28}x+1)(\mfrac{3}{4}{7}-14x) \)
%			\task \( \dfrac{2}{3}x(8x-1)=\mfrac{2}{2}{3}(x-6)(2x+1) \)
%		\end{itasks}
	\end{listofex}
\end{consultation}
%END_FOLD
