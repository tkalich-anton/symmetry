%Группа 71-1 Модуль 2
\title{Занятие №1}
\begin{listofex}
	\item На свой день рождения Алиса купила \( 560 \) кг фруктов (на весь класс). Из них \( 4/7 \) составляют яблоки, а остальное --- апельсины. Сколько килограммов апельсинов купила Алиса. Какую часть от всех фруктов составляют апельсины?
	\item В первый день турист прошел \( 42 \) км, что составляет \( 7/11 \) всего пути. Сколько километров осталось пройти туристу?
	\item Запишите произведение в виде степени, назовите основание и показатель степени:
	\begin{enumcols}[itemcolumns=3]
		\item \( 2\cdot2\cdot2 \)
		\item \( 10\cdot10\cdot10\cdot10\cdot10 \)
		\item \( 3\cdot3\cdot3\cdot3 \)
	\end{enumcols}
	\item Запишите произведение в виде степени числа \( 10 \):
	\begin{enumcols}[itemcolumns=3]
		\item \( 2\cdot5 \)
		\item \( 2\cdot5\cdot2\cdot5 \)
		\item \( 2\cdot2\cdot2\cdot2\cdot5\cdot5\cdot5\cdot5 \)
	\end{enumcols}
	\item Запишите произведение в виде степени:
	\begin{enumcols}[itemcolumns=3]
		\item \( 2^4\cdot2^3 \)
		\item \( 3^6\cdot3^2\cdot3^2 \)
		\item \( 5\cdot5^4\cdot5 \)
	\end{enumcols}
	\item Запишите в виде степени:
	\begin{enumcols}[itemcolumns=3]
		\item \( (11^9)^9 \)
		\item \( (2^{11})^{11} \)
		\item \( 7^5\cdot(7^2)^{10} \)
		\item \( (3^4)^5\cdot(3^5)^4\cdot(3^4)^4\cdot(3^5)^5 \)
	\end{enumcols}
	\item Какие числа называют простыми? Какие числа называют составными?
	\item Представьте число в виде произведения степеней простых чисел:
	\begin{enumcols}[itemcolumns=4]
		\item \( 16 \)
		\item \( 81 \)
		\item \( 1000 \)
		\item \( 196 \)
	\end{enumcols}
	\item Сколько градусов составляет \( 4/15 \) прямого угла? Сколько градусов составляет \( 7/20 \) развернутого угла?
	\item Рабочий за \( 4 \) дня окончил некоторую работу, сделав в первый день \( 3/20 \) всей работы, во второй день \( 7/40 \), а в третий --- \( 3/8 \). Какую часть работы он сделал в четвертый день?
\end{listofex}
\newpage
\title{Занятие №2}
\begin{listofex}
	\item Вася прочитал \( 13/15 \) книги. Сколько страниц прочитал Вася, если в книге \( 195 \) страниц?
	\item Запишите произведение в виде степени, назовите основание и показатель степени:
	\begin{enumcols}[itemcolumns=3]
		\item \( 7\cdot7\cdot7\cdot7 \)
		\item \( 11\cdot11\cdot11 \)
		\item \( 5\cdot5\cdot5\cdot5\cdot5\cdot5\cdot5 \)
	\end{enumcols}
	\item Запишите произведение в виде степени числа \( 6 \):
	\begin{enumcols}[itemcolumns=3]
		\item \( 2\cdot3 \)
		\item \( 2\cdot3\cdot2\cdot3\cdot2\cdot3 \)
		\item \( 2\cdot2\cdot2\cdot2\cdot2\cdot3\cdot3\cdot3\cdot3\cdot3 \)
	\end{enumcols}
	\item Запишите произведение в виде степени:
	\begin{enumcols}[itemcolumns=3]
		\item \( 3^2\cdot3^3 \)
		\item \( 4^9\cdot3^8\cdot3^7 \)
		\item \( 6\cdot6^7\cdot6 \)
	\end{enumcols}
	\item Запишите в виде степени:
	\begin{enumcols}[itemcolumns=2]
		\item \( (7^4)^7 \)
		\item \( (3^{99})^2 \)
		\item \( 2^7\cdot(2^6)^5 \)
		\item \( (11^2)^3\cdot(11^4)^5\cdot(11^6)^7 \)
	\end{enumcols}
	\item Представьте число в виде произведения степеней простых чисел:
	\begin{enumcols}[itemcolumns=4]
		\item \( 32 \)
		\item \( 36 \)
		\item \( 10000 \)
		\item \( 500 \)
	\end{enumcols}
	\item Федя читает книжку, в которой \( 720 \) страниц. За первый день он прочитал \( 5/12 \) всей книжки, а за второй --- \( 7/18 \) всей книжки. Сколько страниц ему осталось прочитать?
	\item Автомобиль проехал \( 575 \) км, что составляет \( 23/25 \) расстояния между двумя городами. Найдите расстояние между городами.
\end{listofex}
\newpage
\title{Домашняя работа №1}
\begin{listofex}
	\item Длина дороги \( 84 \) км. За первый день бригада рабочих отремонтировала \( 5/12 \) дороги, а за второй день --- \( 5/14 \) дороги. Сколько километров осталось отремонтировать?
	\item Заказанная работа была выполнена в \( 3 \) дня. В первый день было сделано \( 4/15 \) всей работы, во второй --- \( 5/12 \) всей работы. Какая часть работы была сделана в третий день?
	\item Вася прочитал \( 195 \) страниц, что составляет \( 13/15 \) всей книги. Сколько страниц в книге?
	\item Запишите произведение в виде степени, назовите основание и показатель степени:
	\begin{enumcols}[itemcolumns=3]
		\item \( 27\cdot27\cdot27 \)
		\item \( 4\cdot4\cdot4\cdot4\cdot4 \)
		\item \( 101\cdot101\cdot101\cdot101\cdot101\cdot101 \)
	\end{enumcols}
	\item Запишите произведение в виде степени числа \( 15 \):
	\begin{enumcols}[itemcolumns=3]
		\item \( 5\cdot3 \)
		\item \( 3\cdot5\cdot3\cdot5\cdot3\cdot5 \)
		\item \( 3\cdot3\cdot3\cdot3\cdot3\cdot5\cdot5\cdot5\cdot5\cdot5 \)
	\end{enumcols}
	\item Запишите произведение в виде степени:
	\begin{enumcols}[itemcolumns=3]
		\item \( 2^5\cdot2^9 \)
		\item \( 3^{11}\cdot3^{11}\cdot3^{11}\cdot3^{11} \)
		\item \( 54\cdot54^2\cdot54^3 \)
	\end{enumcols}
	\item Запишите в виде степени:
	\begin{enumcols}[itemcolumns=4]
		\item \( (2^5)^2 \)
		\item \( (3^4)^5 \)
		\item \( 5^2\cdot(5^3)^4 \)
		\item \( (4^3)^5\cdot(4^11)^2 \)
	\end{enumcols}
	\item Представьте число в виде произведения степеней простых чисел:
	\begin{enumcols}[itemcolumns=4]
		\item \( 64 \)
		\item \( 144 \)
		\item \( 4000 \)
		\item \( 504 \)
	\end{enumcols}
\end{listofex}
\newpage
\title{Занятие №3}
\begin{listofex}
	\item Какие алгебраические выражения называются одночленами? Какие --- многочленами?
	\item Представьте одночлен в стандартном виде:
	\begin{enumcols}[itemcolumns=3]
		\item \( 3x^215xx^3 \)
		\item \( 5x^44y^23x^3y^3 \)
		\item \( 3x^3(-2x^4) x^2x^2 \)
		\item \( 3x^2k^3(-4)xk^2 \)
		\item \( (-2)bbb^3b^2(-5b) \)
		\item \( \dfrac{1}{4}x^3\left( -\dfrac{16}{3} \right)x^4 \)
	\end{enumcols}
	\item Возведите одночлен в степень:
	\begin{enumcols}[itemcolumns=4]
		\item \( (2x)^2 \)
		\item \( (3x^2)^2 \)
		\item \( (2x^3)^5 \)
		\item \( (-6xy)^2 \)
		\item \( \left( \dfrac{1}{2}x^2 \right)^3 \)
		\item \( (0,01x^3y^4)^4 \)
		\item \( \left( \dfrac{1}{3}x^2x^3y^2y \right)^3 \)
		\item \( \left( (0,5x^5)^2 \right)^4 \)
		\item \( \left( \left( \dfrac{2}{5}x^2 \right)\cdot25x^3 \right)^3 \)
	\end{enumcols}
	\item Упростите выражение:
	\begin{enumcols}[itemcolumns=2]
		\item \( (2xy)^4+(3x^2y^2)^2 \)
		\item \( (-0,5x^3)^2+(2x^2)^3-\dfrac{1}{4}x^6 \)
		\item \( \left( (2yx^4) \right)^2-3y\left( x^2 \right)^2 \)
		\item \( 4a^6b^4c^3+2a^6c^2c(b^2)^2-\left( 4a^3b^2c \right)^2\cdot c \)
	\end{enumcols}
	\item Выполните умножение одночлена на одночлен:
	\begin{enumcols}[itemcolumns=2]
		\item \( 4x^6p^2c^3\cdot4x^2p^4c^2 \)
		\item \( \left( -2\dfrac{1}{4} \right)p^2x^2\cdot1\dfrac{1}{3}px^2 \)
		\item \( 1\dfrac{2}{3}k^3x^2\cdot\left( -1\dfrac{1}{5} \right)x^2k^2 \)
		\item \( 1,5x^2c^3\cdot\left( -\dfrac{3}{2} \right)x^3c^4 \)
	\end{enumcols}
	\item Представьте одночлен в виде квадрата или куба другого выражения:\\
	\textit{Примеры:} \( 4x^4y^2=(2x^2y)^2 \);\quad\( 8x^9y^3=(2x^3y)^3 \)
	\begin{enumcols}[itemcolumns=4]
		\item \( 16x^2 \)
		\item \( 25a^4b^2 \)
		\item \( \dfrac{1}{4}x^{10}y^{12} \)
		\item \( 27x^3 \)
		\item \( \dfrac{1}{64}y^{20}y^2 \)
		\item \( \dfrac{1}{125}x^6b^{12} \)
		\item \( 15\dfrac{5}{8}a^{18}y^9 \)
		\item \( 216x^{12}y^{99} \)
	\end{enumcols}
	\item Что такое подобные одночлены?
	\item Среди одночленов найдите подобные:\quad \( 3x^2y,\;2xy,\;-4yx^2,\;0,2xy^2,\;-x^2y,\;3x^2,\;9x^2y^2 \)
	\item Приведите подобные слагаемые:
	\begin{enumcols}[itemcolumns=2]
		\item \( 2x+3x-12x \)
		\item \( 14a^2+12a^2-6a^2 \)
		\item \( 43ax^2+(-12x)^2a+11x^a \)
		\item \( 12a^2b-11ab^2+3a^2b+14ab^2 \)
		\item \( 7,14xy^2+2,5xy^2-(-3,98y^2x) \)
		\item \( (2,1x^2y^2)-(2,1xy)^2 \)
	\end{enumcols}
\end{listofex}
\newpage
\title{Занятие №4}
\begin{listofex}
	\item Представьте одночлен в стандартном виде:
	\begin{enumcols}[itemcolumns=3]
		\item \( 5a^34a^3aa \)
		\item \( x^312y^3yxx^3 \)
		\item \( 12x^5y^3(-2x^3)xx^4y^2 \)
		\item \( 3a^24ab^3c7a^3c^5 \)
		\item \( \dfrac{3}{7}x^5x^4y^2\left( -\dfrac{28}{9}x \right)x^4 \)
	\end{enumcols}
	\item Возведите одночлен в степень:
	\begin{enumcols}[itemcolumns=4]
		\item \( (5xy)^2 \)
		\item \( (7x^3)^3 \)
		\item \( (0,2x^3a^2)^5 \)
		\item \( (-12x^5y)^2 \)
		\item \( \left( \dfrac{3}{2}x^3 \right)^3 \)
		\item \( (0,02x^2)^4 \)
		\item \( \left( \dfrac{1}{4}xx^2y^2y^3 \right)^3 \)
		\item \( \left( (0,55x^99)^2 \right)^2 \)
		\item \( \left( \left( \dfrac{3}{7}x^3 \right)\cdot7x^2 \right)^3 \)
	\end{enumcols}
	\item Упростите выражение:
	\begin{enumcols}[itemcolumns=2]
		\item \( (2xy)^6+(2x^23^3)^2 \)
		\item \( (-0,3x^5)^2-(2x^2)^5+\dfrac{1}{2}x^{10} \)
		\item \( \left( (3yx^2)^2 \right)^2-\dfrac{1}{2}y^4\left( 6x^4 \right)^2 \)
		\item \( 3a^4b^4c^4+2b(a^2)^2c^4(b)^3-(5a^2)^2\left( (bc)^2 \right)^2\cdot \)
	\end{enumcols}
	\item Представьте одночлен в виде квадрата или куба другого выражения:\\
	\textit{Примеры:} \( 4x^4y^2=(2x^2y)^2 \);\quad\( 8x^9y^3=(2x^3y)^3 \)
	\begin{enumcols}[itemcolumns=4]
		\item \( 100x^2 \)
		\item \( 81a^6b^4 \)
		\item \( \dfrac{1}{64}x^{8}c^4y^{8} \)
		\item \( 125x^9 \)
		\item \( \dfrac{1}{216}y^{21}x^3 \)
		\item \( \dfrac{1}{9}x^4b^{100} \)
		\item \( 2\dfrac{7}{9}a^{64}y^{58} \)
		\item \( 1000x^{1000}y^{1000} \)
	\end{enumcols}
	\item Приведите подобные слагаемые:
	\begin{enumcols}[itemcolumns=2]
		\item \( 10x-15x-12x \)
		\item \( 1,5x^2+1,3x^3-2,1x^2+4,02x^3 \)
		\item \( 12a^2x+(-5a)^2x+7xa^2 \)
		\item \( (0,01x)^3y^2-\left( \dfrac{1}{2}x \right)^3y^2+2,5x^3y^2 \)
	\end{enumcols}
	\item Длина дороги \( 40 \) км. За первый день бригада рабочих отремонтировала \( 3/10 \) дороги, а за второй день --- \( 11/20 \) дороги. Сколько километров осталось отремонтировать?
\end{listofex}
%\newpage
%\title{Домашняя работа №2}
%\begin{listofex}
%
%\end{listofex}
\newpage
\title{Занятие №5}
\begin{listofex}
	\item Представьте число в виде квадрата или куба:
	\begin{enumcols}[itemcolumns=4]
		\item \( 225 \)
		\item \( -27 \)
		\item \(0,064 \)
		\item \( -3\:\dfrac{3}{8} \)
	\end{enumcols}
	\item Представьте в виде степени:
	\begin{enumcols}[itemcolumns=3]
		\item \( 5^8\cdot25 \) с основанием \( 5 \)
		\item \( 2^9\cdot32 \) с основанием \( 2 \)
		\item \( 27\cdot81 \) с основанием \( 3 \)
	\end{enumcols}
	\item Вычислить:
	\begin{enumcols}[itemcolumns=4]
		\item \( \dfrac{8^6}{8^4} \)
		\item \( \dfrac{(-0,3)^5}{(-0,3)^3} \)
		\item \( \left( 1\:\dfrac{1}{2} \right)^4:\left( 1\:\dfrac{1}{2} \right)^2 \)
		\item \( \dfrac{2,13^{13}}{2,13^{11}} \)
	\end{enumcols}
	\item Вычислить:
	\begin{enumcols}[itemcolumns=3]
		\item \( \dfrac{7^9\cdot7^5}{7^{12}} \)
		\item \( \dfrac{3^{15}}{3^5\cdot3^6} \)
		\item \( \dfrac{0,6^{12}}{0,6^4\cdot0,6^5} \)
	\end{enumcols}
	\item Приведите подобные слагаемые:
	\begin{enumcols}[itemcolumns=1]
		\item \( 3xx^4+3xx^3-5x^2x^3-5x^2x \)
		\item \( 2a^2x^3-ax^3-a^4-a^2x^3+ax^3+2a^4 \)
	\end{enumcols}
	\item Найдите значение выражения:
	\begin{enumcols}[itemcolumns=1]
		\item \( 5x^6-3x^2+7-2x^6-3x^6+4x^2 \) при \( x=-10 \)
		\item \( 4x^6y^3-3x^6y^3+2x^2y^2-x^6y^3-x^2y^2+y\) при \( x=-2,\;y=-1 \)
	\end{enumcols}
	%\item Упростите выражение:
	%\begin{enumcols}[itemcolumns=3]
	%	\item \( (-x^2y^2)^4\cdot(-xy)^2 \)
	%	\item \( (-2x^3y^2)^3\cdot(-2y^2)^3 \)
	%	\item \( -\left( \dfrac{1}{3}xy^3 \right)^2\cdot(-3x)^3 \)
	%\end{enumcols}
	%\item Решите уравнение:
	%\begin{enumcols}[itemcolumns=3]
	%	\item \exercise{269}
	%	\item \exercise{292}
	%	\item \exercise{352}
	%	\item \exercise{296}
	%	\item \exercise{3596}
	%\end{enumcols}
\end{listofex}
\newpage
\title{Занятие №6}
\begin{listofex}
	\item Представьте одночлен в виде квадрата или куба другого выражения:
	\begin{enumcols}[itemcolumns=4]
		\item \( 64x^4 \)
		\item \( 25a^2b^4 \)
		\item \( \dfrac{1}{121}x^{10}c^4y^{4} \)
		\item \( 8a^6 \)
		\item \( \dfrac{8}{27}y^{6}x^9 \)
		\item \( \dfrac{16}{25}x^{44}b^{22} \)
		\item \( 2\dfrac{14}{25}x^{18}c^{20} \)
		\item \( 100x^{100}y^{100} \)
	\end{enumcols}
	\item Упростите выражение:
	\begin{enumcols}[itemcolumns=3]
		\item \( (-x^2y^2)^4\cdot(-xy)^2 \)
		\item \( (-2x^3y^2)^3\cdot(-2y^2)^3 \)
		\item \( -\left( \dfrac{1}{3}xy^3 \right)^2\cdot(-3x)^3 \)
	\end{enumcols}
	\item Вычислить:
	\begin{enumcols}[itemcolumns=4]
		\item \( \dfrac{3^9}{3^5} \)
		\item \( \dfrac{(-25)^5}{(-25)^3} \)
		\item \( \left( 2\:\dfrac{1}{5} \right)^{11}:\left( 2\:\dfrac{1}{5} \right)^9 \)
		\item \( \dfrac{5,55^{55}}{5,55^{53}} \)
	\end{enumcols}
	\item Вычислить:
	\begin{enumcols}[itemcolumns=3]
		\item \( \dfrac{9\cdot9^2\cdot9^3}{5^{5}} \)
		\item \( \dfrac{2\cdot3^5+5\cdot3^4}{22\cdot3^3} \)
		\item \( \dfrac{6^3\cdot3^9}{3^{10}\cdot2^2} \)
	\end{enumcols}
	\item Найдите значение выражения:
	\begin{enumcols}[itemcolumns=1]
		\item \( 5a^5-3a^2+7-2a^5+5a^2-3a^2 \) при \( a=-1,2 \)
		\item \( 2x^2y^2 + 3xy^2 -2(x^2y^2+y^2) + 3y^2 - 2xy^2\) при \( x=\dfrac{1}{3},\;y=0,5 \)
	\end{enumcols}
	\item Решите уравнение:
	\begin{enumcols}[itemcolumns=3]
		\item \exercise{269}
		\item \exercise{292}
		\item \exercise{352}
		\item \exercise{296}
		\item \exercise{3596}
	\end{enumcols}
\end{listofex}
\newpage
\title{Подготовка к проверочной работе}
\begin{listofex}
	\item В первый день турист прошел \( 21 \) км, что составляет \( 7/15 \) всего пути. Какой путь турист запланировал пройти? Сколько километров ему осталось пройти?
	\item Запишите произведение в виде степени:
	\begin{enumcols}[itemcolumns=2]
		\item \( 4\cdot4\cdot4\cdot4 \)
		\item \( 17\cdot17\cdot17\cdot17\cdot17 \)
	\end{enumcols}
	\item Запишите в виде числа в степени:
	\begin{enumcols}[itemcolumns=3]
		\item \( (4^9)^9 \)
		\item \( (2^{13})^{10} \)
		\item \( 7^3\cdot(7^3)^{5} \)
		\item \( (5^4)^3\cdot(5^5)^4\cdot(5^4)^2\cdot(5^2)^5 \)
	\end{enumcols}
	\item Какие числа называют простыми? Какие числа называют составными?
	\item Представьте число в виде произведения степеней простых чисел:
	\begin{enumcols}[itemcolumns=4]
		\item \( 25 \)
		\item \( 32 \)
		\item \( 100 \)
		\item \( 144 \)
	\end{enumcols}
	\item Сколько градусов составляет \( 13/30 \) прямого угла? Сколько градусов составляет \( 29/45 \) развернутого угла?
	\item Выполните умножение одночлена на одночлен:
	\begin{enumcols}[itemcolumns=2]
		\item \( 2x^2y^2c^3\cdot12x^3y^5c^4 \)
		\item \( 0,25a^2b^6\cdot1,5ab^2\cdot a^3 \)
		\item \( \left( -\dfrac{1}{3} \right)p^5x^7\cdot9p^6x^2 \)
		\item \( \left( -1\dfrac{2}{3} \right)k^2x^9\cdot\left( -\dfrac{6}{5} \right)x^9k^2 \)
		\item \( 1,1x^5\cdot\left( -1,1 \right)x^2c^4 \)
		\item \( 1,2a^3x^2\cdot\dfrac{10}{12}ax^3 \)
	\end{enumcols}
	\item Вычислить:
	\begin{enumcols}[itemcolumns=4]
		\item \( \dfrac{3^3}{3^2} \)
		\item \( \dfrac{(-2)^9}{(-2)^7} \)
		\item \( \left( \dfrac{5}{2} \right)^{17}:\left( 2,5 \right)^{15} \)
		\item \( \left( \dfrac{4,2^{12}}{4,2^{11}} \right)^2 \)
	\end{enumcols}
	\item Возведите одночлен в степень:
	\begin{enumcols}[itemcolumns=4]
		\item \( (3xy)^3 \)
		\item \( (2x^4)^3 \)
		\item \( (0,1x^5a^3)^2 \)
		\item \( (-10x^3y^2)^2 \)
		\item \( \left( \dfrac{1}{2}x^3 \right)^5 \)
		\item \( (0,04x^5)^3 \)
		\item \( \left( \dfrac{2}{3}x^3x^2y^5y^3 \right)^3 \)
		\item \( \left( \left( 0,1x^{33} \right)^2 \right)^2 \)
		\item \( \left( \left( \dfrac{2}{5}x^3 \right)^2\cdot25x^4 \right)^3 \)
	\end{enumcols}
	\item Упростите и найдите значение выражения:
	\begin{enumcols}[itemcolumns=1]
		\item \( 2x^3-2(x^3-2x^2)+3x^2 \) при \( x=-2 \)
		\item \( 5x^2y+2xy^2-4(x^2y+12)-x^2y+y\) при \( x=4,\;y=0,5 \)
	\end{enumcols}
\end{listofex}
\newpage
\title{Проверочная работа}
\begin{listofex}
	\item В первый день турист прошел \( 25 \) км, что составляет \( 5/13 \) всего пути. Какой путь турист запланировал пройти? Сколько километров ему осталось пройти?
	\item Запишите произведение в виде степени:
	\begin{enumcols}[itemcolumns=2]
		\item \( 3\cdot3\cdot3\cdot3\cdot3 \)
		\item \( 15^2\cdot15\cdot15\cdot15^2 \)
	\end{enumcols}
	\item Запишите в виде числа в степени:
	\begin{enumcols}[itemcolumns=3]
		\item \( (2^3)^8 \)
		\item \( (5^{19})^{3} \)
		\item \( 6^3\cdot(6^4)^{7} \)
		\item \( (3^8)^2\cdot(3^5)^3\cdot(3^2)^8\cdot(3^2)^9 \)
	\end{enumcols}
	\item Представьте число в виде произведения степеней простых чисел:
	\begin{enumcols}[itemcolumns=4]
		\item \( 16 \)
		\item \( 44 \)
		\item \( 36 \)
		\item \( 196 \)
	\end{enumcols}
	\item Сколько градусов составляет \( 29/45 \) прямого угла?
	\item Выполните умножение одночлена на одночлен:
	\begin{enumcols}[itemcolumns=2]
		\item \( 3x^5y^7c^3\cdot7x^2y^9c^3 \)
		\item \( 0,2a^6b^5\cdot2,1ab^5\cdot a^6 \)
		\item \( \left( -\dfrac{2}{3} \right)p^3x^2\cdot6p^2x^5 \)
		\item \( 2\dfrac{4}{5}k^2x^9\cdot\dfrac{5}{21}x^9k^2 \)
		\item \( 2x^3\cdot\left( -2,5 \right)xc^7 \)
		\item \( 1,3a^3x^9\cdot\dfrac{10}{13}a^2x^7 \)
	\end{enumcols}
	\item Вычислить:
	\begin{enumcols}[itemcolumns=4]
		\item \( \dfrac{4^5}{4^2} \)
		\item \( \dfrac{(-3)^{11}}{(-3)^9} \)
		\item \( \left( \dfrac{10}{4} \right)^{12}:\left( 2,5 \right)^{10} \)
		\item \( \left( \dfrac{3,7^{14}}{3,7^{13}} \right)^2 \)
	\end{enumcols}
	\item Возведите одночлен в степень:
	\begin{enumcols}[itemcolumns=4]
		\item \( (4xy)^3 \)
		\item \( (3x^9)^4 \)
		\item \( (0,01x^2a^4)^3 \)
		\item \( (-100x^4y^4)^2 \)
		\item \( \left( \dfrac{1}{2}x^7 \right)^4 \)
		\item \( \left( (0,01x^3)^2  \right)^2 \)
		\item \( \left( \dfrac{1}{2}x^2y^7y^4x^3 \right)^5 \)
		\item \( \left( \left( 0,1x^{11} \right)^2 \right)^3 \)
		\item \( \left( \left( \dfrac{2}{7}x^3 \right)^2\cdot49x^4 \right)^3 \)
	\end{enumcols}
	\item Упростите и найдите значение выражения:
	\begin{enumcols}[itemcolumns=1]
		\item \( 4x^5-4(x^5-2x^2)+4x^2 \) при \( x=-3 \)
		\item \( 17x^3y^2+2xy^2-10(x^3y^2+7x^2)-7x^3y^2+y\) при \( x=0,5,\;y=-25 \)
	\end{enumcols}
\end{listofex}