%
%===============>>  ГРУППА 7-1 МОДУЛЬ 8  <<=============
%
\setmodule{8}

%BEGIN_FOLD % ====>>_____ Занятие 1 _____<<====
\begin{class}[number=1]
	\begin{listofex}
		\item Разложите на множители: %305 а-е
		\begin{tasks}(2)
			\task \( 3x+xy-x^2y-3y \)
			\task \( a^2b-2b+ab^2-2a \)
			\task \( 2a^2-2b^2-a+b \)
			\task \( x-y-3x^2+3y^2 \)
			\task \( 2x+y+y^2-4x^2 \)
			\task \( a-3b+9b^2-a^2 \)
			\task \( x^4-3x^2+2 \)
			\task \( b^2c^2-4bc-b^2-c^2+1 \)
			\task \( y^2-10y+25-4x^2 \)
			\task \( (a+b)^3-a^3-b^3 \)
		\end{tasks}
		\item Примените формулы суммы и разности кубов:
		\begin{tasks}(2)
			\task \( x^3+(7y)^3 \)
			\task \( (2x)^3-(0,5y)^3 \)
			\task \( 8x^3-64y^3 \)
			\task \( (4x)^3+0,2^3 \)
		\end{tasks}
		\item Вычислите рациональным способом: %300 а-б 294
		\begin{tasks}(2)
			\task \( 67^2+23^2+2\cdot67\cdot23 \)
			\task \( 45^2-35^2+45\cdot70 \)
			\task \( \dfrac{ 32^3+17^3 }{ 49 }-32\cdot17 \)
			\task \( \dfrac{ 73^3-37^3 }{ 36 }-73\cdot37 \)
		\end{tasks}
		\item Решите уравнения:
		\begin{tasks}
			\task \( (3x+4)^2=(3x-5)(2+3x) \)
			\task \( (2+5x)^2=(5x+4)(4-5x) \)
			\task \( (6x-2)(1-x)=-(3x+5)(2x-0,5) \)
			\task \( (x-5)(10x+23)=5(-x-12)(2,5-2x) \)
			\task \( (4x+1)(1-4x)=-(16x-1)(x+2) \)
			\task \( (3x-2)(8x+1)=(6x+1)(4x-5) \)
		\end{tasks}
	\end{listofex}
\end{class}
%END_FOLD

%BEGIN_FOLD % ====>>_____ Занятие 2 _____<<====
\begin{class}[number=2]
	\begin{listofex}
		\item Разложите на множители: %305 ё-и
		\begin{tasks}
			\task \( a^3-ab-a^2b+a^2 \)
			\task \( x^2y-x^2-xy+x^3 \)
			\task \( 1-x^2+2xy-y^2 \)
			\task \( a^2-9b^2+a8bc-9c^2 \)
		\end{tasks}
		\item Подставьте вместо \(*\) одночлен так, чтобы полученное равенство было верным: %299 a б в г д е
		\begin{tasks}(2)
			\task \( (x-*)^2=*-4xa+* \)
			\task \( (*-3x)^2=*-12zx+* \)
			\task \( (*+5m^2)^2=*+20mn^4+* \)
			\task \( (3x-*)^2=*-30xa^4+* \)
			\task \( (7x+*)^2=*+42xc^6+* \)
			\task \( (*+6t)^2=121+*+* \)
			\task \( (*+3y)^3=*+9x^2y+*+*\)
			\task \( (*-5x)^3=4-*+*-* \)
		\end{tasks}
		\item Решите уравнения:
		\begin{tasks}
			\task \( (3x+4)^2=(3x-5)(2+3x) \)
			\task \( (2+5x)^2=(5x+4)(4+5x) \)
			\task \( (6x-2)(1-x)=-(3x+5)(2x-0,5) \)
			\task \( (x-5)(10x+23)=5(-x-12)(2,5-2x) \)
			\task \( (4x+1)(1-4x)=-(16x-1)(x+2) \)
			\task \( (3x-2)(8x+1)=(6x+1)(4x-5) \)
		\end{tasks}
	\end{listofex}
\end{class}
%END_FOLD

%BEGIN_FOLD % ====>>_ Домашняя работа 1 _<<====
\begin{homework}[number=1]
	\begin{listofex}
		\item Вычислите:
		\begin{tasks}(2)
			\task \( \dfrac{ 25^3+12^3 }{ 37 }-25\cdot12 \)
			\task \( \dfrac{ 15^3-6^3 }{ 9 }-15\cdot6 \)
		\end{tasks}
		\item Разложите на множители: %305 й-м
		\begin{tasks}(2)
			\task \( 2x^2-20xy+50y^2-2 \)
			\task \( 3a^2+12b^2+12ab-12 \)
			\task \( ac^4-c^4-ac^2+c^2 \)
			\task \( x^3y^2-xy^2-x^3+x \)
		\end{tasks}
		\item Вычислите:
		\begin{tasks}(3)
			\task \( -26-37 \)
			\task \( -83-22 \)
			\task \( -\dfrac{1}{3}+\dfrac{1}{2} \)
			\task \( -\mfrac{2}{1}{6}+11 \)
			\task \( -1,5-\dfrac{3}{7} \)
			\task \( \mfrac{3}{5}{6}-4 \)
			\task \( -1,18+\dfrac{33}{50} \)
			\task \( -0,8+4 \)
			\task \( 1,7-7,3 \)
			\task \( -2,4+3,6 \)
			%\task \( -8 \cdot (-7) \)
			\task \( -10 \cdot \dfrac{6}{5} \)
			%\task \( -11 \cdot (-12) \)
			\task \( -\dfrac{8}{3} \cdot (-15) \)
			%\task \( 0,7 \cdot (-8) \)
			\task \( -7 \cdot (-2,1) \)
			\task \( -9,8 \cdot (-50,6) \)
			\task \( -17,5 \cdot (-17,4) \)
			\task \( 3,08 \cdot (-4,05) \)
			\task \( -\dfrac{7}{9}\cdot 3 \)
			\task \( -0,125 \cdot (-6,4) \)
		\end{tasks}
		\item Решите уравнения:
		\begin{tasks}
			\task \( (x-2)^2=(0,5x-3)(4+2x) \)
			\task \( 4x^2=2(x-4)(1+2x) \)
			\task \( 2(6x-2,5)(1,6-x)=-(3x-0,2)(4x+3,5) \)
			\task \( (0,5x-5)(3x+23)=1,5(-x-5,5)(3,8-x) \)
		\end{tasks}
	\end{listofex}
\end{homework}
%END_FOLD

%BEGIN_FOLD % ====>>_____ Занятие 3 _____<<====
\begin{class}[number=3]
	\begin{listofex}
		\item Вычислите:
		\begin{tasks}(2)
			\task \( 15-21 \)
			\task \( 17-66 \)
			\task \( 100-143 \)
			\task \( 42-69 \)
			\task \( 85-98 \)
			\task \( 117-162 \)
			\task \( 31-67 \)
			\task \( 71-143 \)
			\task \( -\mfrac{2}{3}{7}-\mfrac{3}{5}{14} \)
			\task \( -\mfrac{4}{8}{9}+\mfrac{6}{7}{15} \)
			\task \( -3,3+\mfrac{4}{8}{7} \)
			\task \( -5,6+\mfrac{5}{3}{5} \)
			\task \( 9,9-\mfrac{11}{4}{15} \)
			\task \( -\mfrac{8}{4}{23}+\mfrac{9}{5}{46}+\left(-\mfrac{18}{13}{69}\right) \)
			\task \( -2,3 + \left(-\mfrac{1}{3}{5}\right) + 9,09 \)
			\task \( -\mfrac{6}{8}{12} + \left(-\dfrac{13}{24}\right) - (-7,11) \)
		\end{tasks}
		\item Решите уравнения: %придуманные (117)
		\begin{tasks}(2)
			\task \( (x-9)(x+7)=0 \)
			\task \( (x-4)(x+8)=0 \)
			\task \( (x-9)(x-11)=0 \)
			\task \( (x+9)(x+23)=0 \)
			\task \( x(-x-9,3)(-11+x)=0 \)
			\task \( (x+0)(x-11)=0 \)
			\task \( 4x^2-4x+1=0 \)
			\task \( 9x^2-18x+9=0 \)
			\task \( x^2-12x+36=0 \)
			\task \( (x^2+14x+49)(-x+3)=0 \)
			\task! \( -5,5x(-11x-4,6)(-2x-11,8)=0 \)
		\end{tasks}
		
	\end{listofex}
\end{class}
%END_FOLD

%BEGIN_FOLD % ====>>_____ Занятие 4 _____<<====
\begin{class}[number=4]
	\begin{listofex}
		\item Вычислите:
		\begin{tasks}(2)
			\task \( \left( -\dfrac{ 5 }{ 10 } \right)+4 \cdot (-5) \)
			\task \( (-12 \cdot 0,4) \cdot 0,5 - 11 \)
			\task \( (-25 \cdot \dfrac{ 1 }{ 5 } - 30) \cdot \dfrac{ 1 }{ 7 } \)
			\task \( -\mfrac{3}{5}{21}+\mfrac{3}{17}{42}+\left(-\mfrac{18}{45}{66}\right) \)
			\task \( -15,2 + \left(-\mfrac{2}{17}{25}\right) + 12 \)
			\task \( -\mfrac{4}{2}{3} + \left(-\dfrac{22}{24}\right) - (-6) \)
		\end{tasks}
		\item Решите уравнения: %117 ж-
		\begin{tasks}
			\task \( (3x-1)(2x+7)=0 \)
			\task \( (5-9x)(6x+8)=0 \)
			\task \( (5x+8)(6-12x)=0 \)
			\task \( (11x+1)(9x-12)=0 \)
			\task \( (7x+8)(8x+7)=0 \)
			\task \( (x+9)(5x+3)(4x-6)=0 \)
		\end{tasks}
		%G61M6L3 N3
		\item Найдите модуль:
		\begin{tasks}(3)
			\task \(  |3| \)
			\task \(  |-15| \)
			\task \( |-14+52|  \)
			\task \( |-17-5,7|  \)
			\task \(  \left|2-\mfrac{4}{5}{6}\right| \)
			\task \(  \left|\dfrac{-1}{4}\right| \)
			\task \(  \left|\dfrac{-8}{15}\right| \)
			\task \(  \left|-\mfrac{7}{8}{9} - \left|-\dfrac{27}{9}\right|\right| \)
		\end{tasks}
		%G61M6L3 N4
		\item Вычислите:
		\begin{tasks}(2)
			\task \(  |-5|+|-11| \)
			\task \(  -|-15|-|11,5| \)
			\task \( |-21+32| - \left|\dfrac{-3}{5}\right|  \)
			\task \( -|-9|+|15-7,3|-|-25|  \)
			\task \(  |-4,5|-\left|-\mfrac{2}{1}{2}\right| \)
			\task \(  -\left| \dfrac{-5}{8} \right|-\left|\dfrac{-3}{4}\right| \)
			\task \(  \left|\dfrac{-8}{15}\right| + \bigl|-3,5+|-12,2|| \)
			\task \(  \left|-\mfrac{3}{5}{6} - \left|-4+\dfrac{16}{9}\right|\right| \)
		\end{tasks}
		\item Решите уравнения:
		\begin{tasks}(2)
			\task \( |x|=5 \)
			\task \( |x-4|=6 \)
			\task \( |7-x|=12 \)
			\task \( |5x-2|=20 \)
		\end{tasks}
	\end{listofex}
\end{class}
%END_FOLD

%BEGIN_FOLD % ====>>_ Домашняя работа 2 _<<====
\begin{homework}[number=2]
	\begin{listofex}
		\item Решите уравнения:
		\begin{tasks}(2)
			\task \( (2x-5)(3x-4)=0 \)
			\task \( (7x+1)(-4,5+3x)=0 \)
			\task \( (15x-18)(11,3x-33,9)=0 \)
			\task \( (10x+29,5)\left( \dfrac{ 2x }{ 3 } + 18 \right)=0 \)
			\task! \( (15x-12) \cdot(-4x-18) \cdot(1,5-1,2x) \cdot\left( \dfrac{ -4 }{ 5 }-\dfrac{ x }{ 10 } \right)=0 \)
		\end{tasks}
		%G61M6L4 n2
		\item Найдите значение выражения:
		\begin{tasks}(2)
			\task \( |-8|-|-5| \)
			\task \( |-2,3|+|3,7|  \)
			\task \(  |-10|\cdot|-15| \)
			\task \( |-4,7|-|-1,9| \)
			\task \( \left|\dfrac{7}{9}\right|-\left|-\mfrac{1}{2}{3}\right| \)
			\task \( |-4|\cdot\left|-\mfrac{1}{3}{4}\right| \)
			\task \( \left|\mfrac{2}{3}{14}-\mfrac{1}{5}{7}\right|-\left|\dfrac{19}{21}\right| \)
			\task \( -\left|\mfrac{2}{5}{16}-\dfrac{17}{8}\right|+\left|-\dfrac{5}{4}\right| \)
		\end{tasks}
		\item Решите уравнения:
		\begin{tasks}(2)
			\task \( |x|=7,5 \)
			\task \( |x-15|=22 \)
			\task \( |-9-22|=1,5 \)
			\task \( |3x-5,7|=|-13,8| \)
		\end{tasks}
	\end{listofex}
\end{homework}
%END_FOLD

%BEGIN_FOLD % ====>>_____ Занятие 5 _____<<====
\begin{class}[number=5]
	\begin{listofex}
		\item Вычислите:
		\begin{tasks}(2)
			\task \( |-12|-|-5| \)
			\task \( |-10| \cdot |-15| \)
			
			\task \( |-4,7|-|-1,9| \)
			\task \( \left| -\mfrac{ 2 }{ 3 }{14} \right| + \left| -\mfrac{ 1 }{5}{ 7 } \right| - \left| -\dfrac{ 19 }{21  } \right| \)
			\task \( \left| -\mfrac{ 4}{5 }{8 }-\mfrac{ 2}{5 }{16 } + \dfrac{ 14 }{ 4 } \right| \)
			\task \( -|4|:\left| -\mfrac{ 1}{1 }{3 } \right| \)
			\task \( -|7|:\left| -\mfrac{ 1}{1 }{6 } \right| \)
			\task \( \left| \dfrac{ -5 }{ 12 } + (-15) \right| + |-7| \)
		\end{tasks}
		\item Решите уравнения:
		\begin{tasks}(2)
			\task \( |x|=10,25 \)
			\task \( |x-25|=5 \)
			\task \( |-15-4x|=\dfrac{ 8 }{ 10 } \)
			\task \( |32,8-5x|=|-\mfrac{5 }{3 }{ 5}| \)
			\task \( |0,2x-1,4|=2,23 \)
			\task \( |25x-14|=15 \)
			\task \( \left| \dfrac{ x }{ 5 } \right|=\dfrac{ |-5| }{ 25 } \)
			\task \( \dfrac{ |x| }{ -12,3 }=-\dfrac{ |3| }{ 15,3 } \)
		\end{tasks}
		\item Верно, или неверно утверждение? Если верно --- объясните, пожалуйста. Если неверно --- приведите контрпример, пожалуйста.
		\begin{tasks}(2)
			\task Если \( a=b \), то \( |a|=|b| \)
			\task Если \( a<b \), то \( |a|<|b| \)
			\task Если \(|a|<b \), то \( a<b \)
			\task Если \( |a|=|b| \), то \( a=b \)
			\task Если \( |a|<|b| \), то \( a<b \)
			\task Если \( a<|b| \), то \( a<b \)
		\end{tasks}
		\item Решите системы уравнений: %314 а-г
		\begin{tasks}(2)
			\task \( \begin{cases} x+y=-1 \\ x-y=1 \end{cases} \)
			\task \( \begin{cases} x+y=8 \\ x-y=-8 \end{cases} \)
			\task \( \begin{cases} x+y=7 \\ x-y=3 \end{cases} \)
			\task \( \begin{cases} x+y=11 \\ x-y=1 \end{cases} \)
			\task \( \begin{cases} 2x-5=1-y \\ y+11=4-x \end{cases} \)
			\task \( \begin{cases} 3y+2x=1 \\ 4-y=5+x \end{cases} \)
		\end{tasks}
	\end{listofex}
\end{class}
%END_FOLD

%BEGIN_FOLD % ====>>_____ Занятие 6 _____<<====
\begin{class}[number=6]
	\begin{listofex}
		\item Занятие 6
	\end{listofex}
\end{class}
%END_FOLD

%BEGIN_FOLD % ====>>_ Домашняя работа 3 _<<====
\begin{homework}[number=3]
	\begin{listofex}
		\item Домашняя работа 3
	\end{listofex}
\end{homework}
%END_FOLD

%BEGIN_FOLD % ====>>_____ Занятие 7 _____<<====
\begin{class}[number=7]
	\title{Подготовка к проверочной}
	\begin{listofex}
		\item Занятие 7
	\end{listofex}
\end{class}
%END_FOLD

=%BEGIN_FOLD % ====>>_ Проверочная работа _<<====
\begin{exam}
	\begin{listofex}
		\item Проверочная
	\end{listofex}
\end{exam}
%END_FOLD