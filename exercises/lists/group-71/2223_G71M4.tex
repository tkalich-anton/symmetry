%
%===============>>  ГРУППА 7-1 МОДУЛЬ 4  <<=============
%
\setmodule{4}
%
%===============>>  Занятие 1  <<===============
%
\begin{class}[number=1]
	\begin{listofex}
		\item Разделить число:
		\begin{enumcols}[itemcolumns=3]
			\item \( 15 \) в отношении \( 2:3 \)
			\item \( 120 \) в отношении \( 5:7 \)
			\item \( 54 \) в отношении \( 3:2:4 \)
		\end{enumcols}
		\item Точка \( K \) лежит на отрезке \( AB \) , \( AB=6 \) см, причём \( AK \) больше
		\( BK \) на \( 4,6 \) см. Найдите длины отрезков \( AK \) и \( BK \).
		\item Отрезок \( AB=14 \) см. Точка \( M \) делит этот отрезок в отношении \( AM:MB=3:4 \). Чему равны длины отрезков \( AM \) и \( MB \)?
		\item Точка \( C \) делит отрезок \( AB \) в отношении \( 4:5 \). Найдите длину \( AB \), если \( BC=10 \).
		\item Отрезок \( AB=14 \), а \( BC \) в \( \mfrac{1}{2}{7} \) раза больше отрезка \( AB \). Найдите \( AB+BC \).
		\item На прямой отложили отрезок \( AB=6 \) см. За точку \( B \) на прямой отметили точку \( C \) так, что \( AC \) в \( 4 \) раза больше, чем \( AB \). Найдите длину \( BC \).
		\item Точка \( B \) делит отрезок \( AC \) в отношении \( AB:BC=2:1 \). Точка \( D \) делит отрезок \( AB \) в отношении \( AD:DB=3:2 \). В каком отношении точка \( D \) делит отрезок \( AC \)?
		\item Решить уравнение:
		\begin{enumcols}[itemcolumns=3]
			\item \( 3x-12=4(x-1) \)
			\item \( 0,25(x-2)=15-0,75x \)
			\item \( 2x(x-15)+24=2x^2-6x \)
		\end{enumcols}
	\end{listofex}
\end{class}
%
%===============>>  Занятие 2  <<===============
%
\begin{class}[number=2]
	\begin{listofex}
		\item Разделить число:
		\begin{enumcols}[itemcolumns=3]
			\item \( 22 \) в отношении \( 5:6 \)
			\item \( 200 \) в отношении \( 13:7 \)
			\item \( 36 \) в отношении \( 7:3:2 \)
		\end{enumcols}
		\item Точки \( O \),\( K \),\( M \) лежат на одной прямой. Найти расстояние между
		точками \( O \) и \( M \) , если \( OK = 8,2 \) см, \( KM = 7,3 \) см. Указать все
		возможные решения.
		\item Боковая сторона равнобедренного треугольника равна \( 12 \), а периметр треугольника равен \( 44 \). Найдите длину основания.
		\item Точка \( K \) делит отрезок \( AB \) в отношении \( AK:BK=3:7 \). Найдите длину \( AB \), если \( BK=21 \).
		\item Отрезок \( AB=11 \), а \( BC \) в \( 1,7 \) раза больше отрезка \( AB \). Найдите \( BC-AB \).
		\item Сумма двух сторон равнобедренного треугольника равна \( 26 \) см, а
		периметр равен \( 36 \) см, какими могут быть стороны этого
		треугольника?
		\item Решить пропорцию:
		\begin{enumcols}[itemcolumns=2]
			\item \( 3:7,5=x:\mfrac{6}{1}{4}\)
			\item \( \dfrac{x}{1,8}=\dfrac{2,7}{0,09} \)
		\end{enumcols}
	\end{listofex}
\end{class}
%
%===============>>  Домашняя работа 1  <<===============
%
\begin{homework}[number=1]
	\begin{listofex}
		\item Разделить число:
		\begin{enumcols}[itemcolumns=3]
			\item \( 30 \) в отношении \( 4:2 \)
			\item \( 70 \) в отношении \( 3:4 \)
			\item \( 150 \) в отношении \( 9:4:2 \)
		\end{enumcols}
		\item Вычислить:
		\begin{enumcols}[itemcolumns=3]
			\item \( \dfrac{15^5}{3^4\cdot5^6} \)
			\item \( 3,5\cdot(8,68+1,136)-135,531:33,3 \)
		\end{enumcols}
		\item Решить пропорцию:
		\begin{enumcols}[itemcolumns=2]
			\item \( x:51,6=11,2:34,4 \)
			\item \( \dfrac{12,3}{6}=\dfrac{x}{4,2} \)
		\end{enumcols}
		\item Боковая сторона равнобедренного треугольника равна \( 13 \), а периметр треугольника равен \( 50 \). Найдите длину основания.
		\item Решить уравнение:
		\begin{enumcols}[itemcolumns=2]
			\item \( 17x+2=4(4x-23) \)
			\item \( 0,2(x-2,5)=7,5-1,8x \)
		\end{enumcols}
	\end{listofex}
\end{homework}
%
%===============>>  Занятие 3  <<===============
%
\begin{class}[number=3]
	\begin{listofex}
		\item Прямой угол поделили в отношении \( 7:3 \). Найдите величины получившихся частей.
		\item Развернутый угол поделили в отношении \( 1:2 \) и в каждой части провели биссектрису. Найдите угол между биссектрисами. Изменится ли результат, если отношение \( 1:2 \) заменить на \( 2:3 \)? Проверить и пояснить результат.
	\end{listofex}
	\begin{definit}
		Сумма углов в треугольнике равна \( 180\degree \).
	\end{definit}
	\begin{listofex}
		\item В треугольнике \( ABC \) угол \( \angle A = 80\degree \) и \( \angle B = 30\degree \). Найдите величину угла \( \angle C \).
		\item В треугольнике \( ABC \) угол \( \angle A \) в две раза меньше угла \( \angle B \) и в три раза меньше угла \( \angle C \). Найдите все углы треугольника \( ABC \).
		\item Два угла в треугольнике \( ABC \) в сумме составляют \( 150\degree \) и относятся друг к другу как \( 7:8 \). Найдите эти углы, а также третий угол треугольника \( ABC \).
		\item Все три угла в треугольнике \( ABC \) относятся друг к другу как \( 1:2:15 \). Найдите углы треугольника \( ABC \).
	\end{listofex}
	\begin{definit}
		В равнобедренном треугольнике прилегающие к основанию углы равны.
	\end{definit}
	\begin{listofex}[resume]
		\item Угол \( B \) при основании \( AB \) равнобедренного треугольника \( ABC \) равен \( 34\degree \). Найдите, чему равен угол при вершине треугольника \( ABC \).
		\item Угол при вершине равнобедренного треугольника в два раза меньше, чем угол при основании. Найдите углы треугольника.
		\item Решить уравнение:
		\begin{enumcols}[itemcolumns=2]
			\item \( 26x+2(x-1)=3(7x-10) \)
			\item \( 0,01(x-3,2)=1,034-0,12x \)
		\end{enumcols}
	\end{listofex}
\end{class}
%
%===============>>  Занятие 4  <<===============
%\begin{class}[number=4]
%	\begin{listofex}
%		\begin{enumcols}
%			\item Пусто
%		\end{enumcols}
%	\end{listofex}
%\end{class}
%
%===============>>  Домашняя работа 2  <<===============
%
\begin{homework}[number=2]
		\begin{listofex}
			\item В равнобедренном треугольнике \(ABC\) с основанием \(AC\) внешний угол при вершине \(C\) равен \(123 \degree \). Найдите величину угла \(ABC\). Ответ дайте в градусах.
			\item В треугольнике \( ABC \) угол \( \angle A \) в два раза больше угла \( \angle B \) и в три раза меньше угла \( \angle C \). Найдите все углы треугольника \( ABC \).
			\item Все три угла в треугольнике \( ABC \) относятся друг к другу как \( 2:7:11 \). Найдите углы треугольника \( ABC \).
			\item Два угла в треугольнике \( ABC \) в сумме составляют \( 135\degree \) и относятся друг к другу как \( 7:8 \). Найдите эти углы, а также третий угол треугольника \( ABC \).
			\item Развернутый угол поделили в отношении \( 6:4 \) и в каждой части провели биссектрису. Найдите угол между биссектрисами.
			\item Угол при вершине равнобедренного треугольника больше на \(20 \degree\), чем угол при основании. Найдите все углы треугольника.
			\item Решить уравнения:
			\begin{enumcols}[itemcolumns=2]
				\item \( x-2(2x-7)=2x \)
				\item \( 4x-2(3x+5)=0,5(4x-3) \)
				\item \( 0,6x=5 \cdot (0,1x-6) \)
				\item \( 0,01(8-2x)=1,5-0,7x \)
		\end{enumcols}
	\end{listofex}
\end{homework}
%
%===============>>  Занятие 5  <<===============
%
\begin{class}[number=5]
	\begin{definit}
		Периметр --- это сумма длин всех сторон фигуры.
	\end{definit}
	\begin{listofex}
		\item В треугольнике \(ABC\) \(AB = 5\), \(AC\) в два раза больше чем \( AB \), \(BC = 9\). Найдите периметр треугольника \( ABC \).
		\item Сумма двух сторон равнобедренного треугольника равна \( 20 \) см, а периметр равен \( 28 \) см, какими могут быть стороны этого треугольника.
		\item Найди длину прямоугольника, если его ширина \(7\) см, а периметр равен \(40\) см.
		\item Сторона квадрата \(18\) см. Найди длину прямоугольника с таким же периметром и шириной \(14\) см.
	\end{listofex}
	\begin{definit}
		Равными называются фигуры, которые при наложении совпадают.
	\end{definit}
	\begin{definit}
		Если две стороны и угол между ними одного треугольника соответственно равны двум сторонам и углу между ними другого треугольника, то такие треугольники равны.
	\end{definit}
	\begin{listofex}[resume]
		\item
		\begin{minipage}[t]{\bodywidth}
			Дан четырехугольник \( ARTV \). Известно, что \( RT=AV \) и \( \angle RTA = \angle TAV \). Докажите что треугольники \( ART \) и \( AVT \) равны.
		\end{minipage}
		\hspace{0.03\linewidth}
		\begin{minipage}[c]{\picwidth}
			\includegraphics[width=\linewidth]{\picpath/G71M4C5-3}
		\end{minipage}
		\item Докажите, что серединный перпендикуляр, проведенный к основанию равнобедренного треугольника, делит этот треугольник на два равных.
%		\item
%		\begin{minipage}[t]{0.73\linewidth}
%			Дан прямоугольник \(GHEF\), докажите что треугольники \(HEO\) и \(GOF\) равны, и \(GOH\) с \(EOF\).
%		\end{minipage}
%		\hspace{0.03\linewidth}
%		\begin{minipage}[c]{0.24\linewidth}
%			\includegraphics[width=0.7\linewidth]{pics/G71M4C5-1.jpg}
%		\end{minipage}
%		\item
%		\begin{minipage}[t]{0.45\linewidth}
%			Точка пересечения \( O \) делит каждый из отрезков \( AD \) и \( TM \) пополам. Найдите углы \(A\) и \(T\), если угол \(M = 22 \degree\), а угол \(D = 57 \degree\).
%		\end{minipage}
%		\hspace{0.05\linewidth}
%		\begin{minipage}[t]{0.5\linewidth}
%			\includegraphics[align=t, width=\textwidth]{pics/G71M4C5-2.jpg}
%		\end{minipage}
		\item Вычислить: \( \left( \mfrac{5}{7}{12}-\mfrac{3}{17}{36} \right)\cdot\mfrac{2}{1}{2}+\mfrac{4}{1}{3}\cdot\dfrac{3}{26}+\dfrac{1}{2} \)
		\item Сумма двух чисел равна \(2,4\), а их разность равна \(1,63\). Найдите эти числа.
	\end{listofex}
\end{class}
%
%===============>>  Домашняя работа 3  <<===============
%
%\begin{homework}[number=2]
%	\begin{listofex}
%
%	\end{listofex}
%\end{homework}
%\newpage
%\title{Подготовка к проверочной работе}
%\begin{listofex}
%	
%\end{listofex}
%
%===============>>  Занятие 6  <<===============
%
\begin{class}[number=6]
	\begin{listofex}
		\item Вычислите периметр равнобедренного треугольника \(АВС\), если \(D\) --- точка пересечения медианы угла \(A\) и основания треугольника, а периметр треугольника \(ADC\) равен \(18\) cм, и \(CD = 6\) cм и \(AD = BD\).
		\item Сторона \(CD\) треугольника \(CDE = 24\) см, сторона \(CE\) в \(3\) раза меньше стороны  \(CD\), а сторона \(DE\) на \(7\) см больше стороны \(CD\). Найти периметр треугольника  \(CDЕ\).
		\item Периметр треугольника равен \(62\) см, а одна из сторон равна \(21\) см. Найти две другие стороны, если одна из них на \(5\) см больше другой.
%		\item Может ли периметр треугольника быть равным \(19\), если одна из его сторон на \(1\) короче другой и на \(3\) длиннее третьей?
%		\item Диагонали \(AC\) и \(BD\) четырехугольника \(ABCD\) пересекаются в точке \(O\). Периметр треугольника \(ABC\) равен периметру треугольника \(ABD\), а периметр треугольника \(ACD\) --- периметру треугольника \(BCD\). Докажите, что \(AO = BO\).
		\item Докажите, что если диагонали четырёхугольника делят друг друга пополам, то противоположные стороны четырёхугольника - равны.
		\item Докажите, что в равных треугольниках соответствующие высоты равны между собой.
		\item Медиана треугольника делит его на два треугольника, периметры которых равны. Докажите, что треугольник равнобедренный.
		\item
		\begin{minipage}[t]{\bodywidth}
			Дан прямоугольник \(GHEF\), докажите что треугольники \(HEO\) и \(GOF\) равны, и \(GOH\) с \(EOF\).
		\end{minipage}
		\hspace{0.03\linewidth}
		\begin{minipage}[c]{\picwidth}
			\includegraphics[width=0.7\linewidth]{\picpath/G71M4C5-1.jpg}
		\end{minipage}
		\item Точка пересечения \( O \) делит каждый из отрезков \( AD \) и \( TM \) пополам. Найдите углы \(A\) и \(T\), если угол \(M = 22 \degree\), а угол \(D = 57 \degree\).
		\begin{center}
			\includegraphics[width=0.6\linewidth]{\picpath/G71M4C5-2.jpg}
		\end{center}
	\end{listofex}
\end{class}
%
%
%===============>>  Занятие 7  <<===============
%
%\begin{class}[number=7]
%	\begin{listofex}
%		
%	\end{listofex}
%\end{class}
%
%===============>>  Провечная работа  <<===============
%
\begin{exam}
	\begin{listofex}
		\item Угол \( A \) равен \( 30\degree \). Чему равен смежный с ним угол?
		\item Два вертикальных угла в сумме составляют \( 150\degree \). Чему равен каждый из них?
		\item В треугольнике \( ABC \) угол \( \angle  A = 35\degree \) и угол \( \angle B = 76\degree \). Чему равен угол \( C \)?
		\item Все три угла в треугольнике \( ABC \) относятся друг к другу как \( 3:5:10 \). Найдите углы треугольника \( ABC \).
		\item Найдите все стороны треугольника, если его периметр равен 19, а одна из его сторон на 1 короче другой и на 3 длиннее третьей?
		\item В треугольнике \( ABC \) \( AB = 7 \), \( BC \) больше \( AB \) в \( \mfrac{1}{4}{7} \) раза, а \( AC \) на \( 5 \) меньше \( BC \). Найдите периметр \( ABC \).
		\item Вычислить:
		\begin{tasks}(3)
			\task \( \dfrac{14^4}{7^6} \)
			\task \( \dfrac{6^3}{2^2\cdot3^3} \)
			\task \( \left( \dfrac{7}{5} \right)^9:(1,4)^7 \)
		\end{tasks}
		\item Решить уравнения:
		\begin{tasks}(2)
			\task \( 3(4-x)=6-(8x+3) \)
			\task \( \dfrac{x}{9}-\dfrac{x}{3}+\dfrac{x}{18}=-1 \)
			\task \( \dfrac{x-3}{5}=\dfrac{2x+1}{4} \)
		\end{tasks}
	\end{listofex}
\end{exam}