%
%===============>>  ГРУППА 7-1 МОДУЛЬ 6  <<=============
%
\setmodule{6}

%BEGIN_FOLD % ====>>_____ Занятие 1 _____<<====
\begin{class}[number=1]
	\begin{definit}
		Куб суммы: \( (a+b)^3=a^3+3a^2b+3ab^2+b^3 \)
	\end{definit}
	\begin{listofex}
		\item Раскройте скобки:
		\begin{tasks}(2)
			\task \( (2x+3y)^3 \)
			\task \( (4x+7y)^3 \)
			\task \( (1,5x+4)^3 \)
			\task \( (11x+8)^3 \)
		\end{tasks}
	\end{listofex}
		\begin{definit}
			Куб разности: \( (a-b)^3=a^3-3a^2b+3ab^2-b^3 \)
		\end{definit}
	\begin{listofex}[resume]
		\item Раскройте скобки:
		\begin{tasks}(2)
			\task \( (x-2y)^3 \)
			\task \( (3x-4y)^3 \)
			\task \( (5x-8)^3 \)
			\task \( (12x-5)^3 \)
		\end{tasks}
		
	\end{listofex}
	\begin{definit}
		Сумма кубов: \( a^3+b^3=(a+b)(a^2-ab+b^2) \) \\
		Разность кубов: \( a^3-b^3=(a-b)(a^2+ab+b^2) \)
	\end{definit}
	\begin{listofex}[resume]
		\item Примените формулы суммы и разности кубов:
		\begin{tasks}(2)
			\task \( x^3+4^3 \)
			\task \( 2^3-y^3 \)
			\task \( 8x^3-27y^3 \)
			\task \( (ab)^3+c^6 \)
		\end{tasks}
	\end{listofex}
		\begin{definit}
		Чтобы вынести общий множитель за скобки нужно выполнить следующие действия:
		\\ 1) Находим число, на которое делятся без остатка числовые коэффициенты каждого одночлена.
		\\ 2) Находим буквенные множители, которые повторяются в каждом одночлене. Выносим их за скобку в наименьшей степени.
		\end{definit}
	\begin{listofex}[resume]
		\item Найдите и вынесите общий множитель:
		\begin{tasks}(2)
			\task \( 19x-19y \)
			\task \( 6x^3-4 \)
			\task \( xy^2+x^3y \)
			\task \( 2x^4-6x^2 \)
			\task \( 3xy^2+6x^2 \)
			\task \( 9x^2+11x^4 \)
			\task \( 4x^2y^5+24xy^3 \)
			\task \( 16x^2y^2+8x^3y \)
			\task \( \dfrac{1}{4}y^5-\dfrac{1}{2}x^5 \)
			\task \( 6,8x^2y^4 + 6,9xy^5 \)
			\task \( 8,1x^3y^3 + 4,05x^{11}y \)
			\task \( 75y^3 - 105xy \)
		\end{tasks}
		\item Упростите выражение и приведите подобные слагаемые:
		\begin{tasks}(2)
			\task \( (a-b)^3+(b-2a)^2 \)
			\task \( (4a+2b)^3+(a-5b)^3 \)
			\task \( a^3-8b^3+(a-b)^3 \)
			\task \( a^6+b^3+(a^2-2b)^3 \)
			\task \( a^3-27b^3 - 2(a-3b)^3 \)
			\task \( (3a^2-4b)^3+(2a^2-8b)^3 \)
		\end{tasks}
	\end{listofex}
\end{class}
%END_FOLD

%BEGIN_FOLD % ====>>_____ Занятие 2 _____<<====
\begin{class}[number=2]
	\begin{listofex}
		\item Раскройте скобки:
		\begin{tasks}(2)
			\task \( (x+y)^3 \)
			\task \( (3x-4y)^3 \)
			\task \( (0,4x+3,5y)^3 \)
			\task \( (2x^2-0,8)^3 \)
		\end{tasks}
		\item Примените формулы суммы и разности кубов:
		\begin{tasks}(2)
			\task \( x^3+(7y)^3 \)
			\task \( (2x)^3-(0,5y)^3 \)
			\task \( 8x^3-64y^3 \)
			\task \( (4x)^3+0,2^3 \)
		\end{tasks}
		\item Найдите и вынесите общий множитель:
		\begin{tasks}(2)
			\task \( x^2y-5x \)
			\task \( 11x^3-44x^2 \)
			\task \( 27xy^2+6x^3y^2 \)
			\task \( 16x^4-6x^2 \)
		\end{tasks}
		\item Вычислите рациональным способом:
		\begin{tasks}(2)
			\task \( 78^2-77^2 \)
			\task \( 65^2-64^2 \)
			\task \( 123^2-124^2 \)
			\task \( 22^3-18^3 \)
			\task \( 95,5^3+4,5^3 \)
			\task \( 22^3-15,2^3 \)
			\task \( \dfrac{73^2-54^2}{19} \)
			\task \( \dfrac{65^2-91^2}{26} \)
			\task \( \dfrac{83^2-19^2}{39^2-25^2} \)
			\task \( \dfrac{57^2-33^2}{43^2-67^2} \)
			\task \( \dfrac{32^3+17^3}{49}-32\cdot17 \)
			\task \( \dfrac{73^3-37^3}{36}-37\cdot73 \)
		\end{tasks}
		\item Упростите выражение и найдите его значение:
		\begin{tasks}(1)
			\task \( \dfrac{(2a)^3(3a^2)^2}{(6a^4)^2} \), при \(a=\dfrac{5}{6}\).
			\task \( \dfrac{(7x-7y)(x^2+xy+y^2)}{14} \), при \( x=5, y=3 \)
		\end{tasks}
	\end{listofex}
\end{class}
%END_FOLD

%BEGIN_FOLD % ====>>_ Домашняя работа 1 _<<====
\begin{homework}[number=1]
	\begin{listofex}
		\item Раскройте скобки:
		\begin{tasks}(2)
			\task \( (2x+5y)^3 \)
			\task \( (4x+1)^3 \)
			\task \( (5x-4y)^3 \)
			\task \( (2x-3)^3 \)
		\end{tasks}
		\item Примените формулы суммы и разности кубов:
		\begin{tasks}(2)
			\task \( (3x)^3+y^3 \)
			\task \( 3^3+x^3 \)
		\end{tasks}
		\item Найдите и вынесите общий множитель:
		\begin{tasks}(2)
			\task \( x^2y^2-5xy^3 \)
			\task \( 2x^3-4x^2 \)
			\task \( 2,4xy^3+0,3x^3y^2 \)
			\task \( \dfrac{xy^4}{3}-\dfrac{x^2y^2}{6} \)
		\end{tasks}
		\item Вычислите рациональным способом:
		\begin{tasks}(2)
			\task \( 11^2-10^2 \)
			\task \( 12^2-23^2 \)
			\task \( \dfrac{18^2-12^2}{6} \)
			\task \( \dfrac{51^2-37^2}{15^2-1^2} \)
			\task \( \dfrac{28^3-12^3}{16}-14\cdot24 \)
		\end{tasks}
	\end{listofex}
\end{homework}
%END_FOLD

%BEGIN_FOLD % ====>>_____ Занятие 3 _____<<====
\begin{class}[number=3]
	\begin{definit}
		\textbf{Разложить на множители} --- преобразовать выражение так, чтобы последнее по порядку действие было умножение.
		
		Существует три способа разложить выражение на множители: \\
		1) Вынесение общего множителя за скобки \\
		2) Применение формул сокращенного умножения \\
		3) Метод группировки:\\ \( ab-2b+3a-6=(ab-2b)+(3a-6)=b(a-2)+3(a-2)=(b+3)(a-2) \)
	\end{definit}
	\begin{listofex}
		\item Вынесите общий множитель:
		\begin{tasks}(2)
			\task \( 2x+2y \)
			\task \( 3a+6b \)
			\task \( 10p-5pq \)
			\task \( -15ax-20ay \)
			\task \( a^3-ab \)
			\task \( x^3-x^2 \)
			\task \( 9m^4-6m^3 \)
			\task \( 30c^5x^2+15c^4x \)
		\end{tasks}
		\item Разложите на множители:
		\begin{tasks}(2)
			\task \( 4a^x+4ab+b^2 \)
			\task \( a^2-12a+36 \)
			\task \( 4x^2-9 \)
			\task \( x^4-8x^2+16 \)
			\task \( x^2-6xy+9y^2 \)
			\task \( 0,04x^2+0,4xy+y^2 \)
			\task \( x^3+3x^2y+3xy^2+y^3 \)
			\task \( b^3+9b^2+27b+27 \)
			\task \( a^3-6a^2+12a-8 \)
			\task \( x^3-6x^2y+12xy^2-27y^3 \)
		\end{tasks}
		\item Разложите на множители многочлен методом группировки:
		\begin{tasks}(2)
			\task \( mx+my+6x+6y \)
			\task \( 9x+ay+9y+ax \)
			\task \( 7a-7b+an-bn \)
			\task \( ax+ay-x-y \)
			\task \( 1-bx-x+b \)
			\task \( xy+2y-2x-4 \)
			
		
			\task \( 3m-mk+3k-k^2 \)
			\task \( x^2+7x-ax-7a \)
			\task \( xk-xy-x^2+yk \)
			\task \( x^3+x^2+x+1 \)
			\task \( y^5-y^3-y^2+1 \)
			\task \( a^4+2a^3-a-2 \)
			
		\end{tasks}
		\item Найдите значение выражения:
		\begin{tasks}(1)
			\task \( p^2q^2+pq-q^3-p^3 \) при \( p=0,5;q=-0,5 \)
			\task \( 3x^3-2y^3-6x^2y^2+xy \) при \( x=\dfrac{2}{3};y=0,5 \)
		\end{tasks}
		\item Разложите на множители многочлен %В ДЗ 717
		\begin{tasks}(2)
			\task \( ac^2-ad+c^3-cd-bc^2+bd \)
			\task \( ax^2+ay^2-bx^2-by^2+b-a \)
			\task \( an^2+cn^2-ap+ap^2-cp+cp \)
			\task \( xy^2-by^2-ax+ab+y^2-a \)
		\end{tasks}
	\end{listofex}
\end{class}
%END_FOLD

%BEGIN_FOLD % ====>>_____ Занятие 4 _____<<====
\begin{class}[number=4]
	\begin{listofex}
		\item Применить формулу квадрата суммы:
		\begin{tasks}(4)
			\task \( (a+b)^2 \)
			\task \( (2a+3)^2 \)
			\task \( (3a+2b)^2 \)
			\task \( (5a+0,2b)^2 \)
			\task \( (12+4x)^2 \)
			\task \( (x^2+3)^2 \)
			\task \( \left( \dfrac{x}{3}+2y \right)^2 \)
			\task \( (0,1x+3,5)^2 \)
		\end{tasks}
		\item Применить формулу квадрата разности:
		\begin{tasks}(3)
			\task \( (a-2b)^2 \)
			\task \( (4a-3b)^2 \)
			\task \( (a-11)^2 \)
			\task \( (0,5a+3b)^2 \)
			\task \( \left( \dfrac{1}{4}x-2y \right)^2 \)
			\task \( (0,25x-1,7)^2 \)
		\end{tasks}
		\item Примените формулу разности квадратов и разложите на множители выражение:
		\begin{tasks}(3)
			\task \( x^2-9 \)
			\task \( x^2-1 \)
			\task \( a^2b^2-4 \)
			\task \( (xy)^2-16 \)
			\task \( 4x^2-64 \)
			\task \( 16x^6-81y^2 \)
			\task \( 36x^{10}y^8-81a^4b^2 \)
			\task \( \dfrac{1}{36}-x^4 \)
			\task \( \dfrac{1}{16}m^2-\dfrac{25}{49}n^2 \)
		\end{tasks}
		\item Представьте в виде квадрата суммы или квадрата разности:
		\begin{tasks}(2)
			\task \( x^2+2xy+y^2 \)
			\task \( a^2-2ab+b^2 \)
			\task \( x^2+2x+1 \)
			\task \( 16+8p+p^2 \)
			\task \( x^2-4ax+4a^2 \)
			\task \( 16p^2+40pq+25q^2 \)
			\task \( a^6+2a^3b^3+b^6 \)
		\end{tasks}
		\item Представьте в виде произведения многочленов выражение:
		\begin{tasks}(2)
			\task \( x(b+c)+3b+3c \)
			\task \( y(a-c)+5a-5c \)
			\task \( p(c-d)+c-d \)
			\task \( a(p-q)+q-p \)
		%\end{tasks}
		%\item Разложите на множители многочлен методом группировки:
		%\begin{tasks}(2)
			\task \( ab-8a-bx+8x \)
			\task \( ax-b+bx-a \)
			\task \( ax-y+x-ay \)
			\task \( ax-2bx+ay-2by \)
			\task \( ax-y+x-ay \)
			\task \( mn-mk+xk-xn \)
		\end{tasks}
	\end{listofex}
\end{class}
%END_FOLD

%BEGIN_FOLD % ====>>_ Домашняя работа 2 _<<====
\begin{homework}[number=2]
	\begin{listofex}
		\item Представьте в виде квадрата суммы или квадрата разности:
		\begin{tasks}(2)
			\task \( a^2-2ab+b^2 \)
			\task \( x^2+4xy+4y^2 \)
			\task \( x^2+2x+1 \)
			\task \( 25+10p+p^2 \)
			\task \( 4a^2+8ab+4b^2 \)
			\task \( 25a^4+70a^2b^2+49b^4 \)
		\end{tasks}
		\item Чему равно значение выражения:
		\begin{tasks}(1)
			\task \( 2a+ac^2-a^2c-2c \) при \( a=\mfrac{1}{1}{3} \) и \( c=-\mfrac{1}{2}{3}  \),
			\task \( x^2y-y+xy^2-x \) при \( x=4 \) и \( y=0,25 \).
		\end{tasks}
		\item Разложите на множители многочлен:
		\begin{tasks}(2)
			\task \( b^6-3b^4-2b^2 \)
			\task \( a^2-ab-8a+8b \)
			\task \( ab-3b+b^2-3a \)
			\task \( 11x-xy+11y-x^2 \)
			\task \( kn-mn-n^2+mk \)
			\task \( x^2y+x+xy^2+y+2xy+2 \)
			\task \( x^2-xy+x-xy^2+y^3-y^2 \)
		\end{tasks}
		
		
	\end{listofex}
\end{homework}
%END_FOLD

%BEGIN_FOLD % ====>>_____ Занятие 5 _____<<====
\begin{class}[number=5]
	\begin{listofex}
		\item Вычислите рациональным способом:
		\begin{tasks}(2)
			\task \( 67^2+23^2+2 \cdot 67 \cdot 23 \)
			\task \( 45^2+35^2 + 2 \cdot 45 \cdot 35 \)
			\task \( -65^2 -27^2 + 27 \cdot 130 \)
			\task \( 27^2 + 39^2 + 39 \cdot 54 \)
			\task \( 58^2+23^2+46 \cdot 58 \)
			\task \( 87^2+12^2+87 \cdot 24 \)
		\end{tasks}
		\item Раскройте скобки:
		\begin{tasks}(2)
			\task \( (x+4y)^3 \)
			\task \( (5x+3y)^3 \)
			\task \( (10x+2,2)^3 \)
			\task \( (3y+11)^3 \)
			\task \( (3x-4y)^3 \)
			\task \( \left(1,5x-\dfrac{2}{3}y \right)^3 \)
			\task \( (14x-5)^3 \)
			\task \( (3,3-y)^3 \)
		\end{tasks}
		\item Вычислите:
		\begin{tasks}(3)
			\task \( \dfrac{5^{10} \cdot (5^3)^4}{5^{18}} \)
			\task \( \dfrac{3^{34}}{3^{17} \cdot (3^5)^2} \)
			\task \( \dfrac{7^{11}}{7^2 \cdot (7^4)^3} \)
			\task \( \dfrac{2^{21} \cdot (2^3)^5}{2^{40}} \)
			\task \( 2,5^{43} \cdot \left( \dfrac{2}{5} \right)^{41}  \)
			\task \( \left( \mfrac{2}{2}{3} \right)^{11} \cdot 0,375^8 \)
		\end{tasks}
		\item Примените формулы суммы и разности кубов:
		\begin{tasks}(2)
			\task \( (3x)^3+(2y)^3 \)
			\task \( x^3-(0,5y)^3 \)
			\task \( 27x^3-125y^3 \)
			\task \( (4x)^3+7^3 \)
		\end{tasks}
		\item Разложите на множители:
		\begin{tasks}(2)
			\task \( 3x+xy-x^2y-3y \)
			\task \( a^2b-2b+ab^2-2a \)
			\task \( 2a^2-2b^2-a+b \)
			\task \( x-y-3x^2+3y^2 \)
		\end{tasks}
		\item Преобразуйте в многочлен стандартного вида:
		\begin{tasks}(1)
			\task \( (m+n)^2+(m-n)^2 \)
			\task \( 4(3x+4y)^2-7(2x-3y)^2 \)
			\task \( (7-2x)(x+2)-(6-x)(x+6)-(-2x-3)^2 \)
			\task \( (-4x+1)(1+4x)-(3x-2)(2x+3)-(3x-5)^2 \)
		\end{tasks}
	\end{listofex}
\end{class}
%END_FOLD

%BEGIN_FOLD % ====>>_____ Занятие 6 _____<<====
\begin{class}[number=6]
	\begin{listofex}
		\item Подставьте вместо * одночлен так, чтобы полученное равенство было верным:
		\begin{tasks}(2)
			\task \( (x-*)^2=*-4xa+* \)
			\task \( (*-3x)^2=*-12zx+* \)
			\task \( (*+5m^2)^2=*+20nm^4+* \)
			\task \( (3x-*)^2=*-30xa^3+* \)
			\task \( (7y+*)^2=*+42zc^5+* \)
			\task \( (*+6t)^2=*+36b^7t+* \)
		\end{tasks}
		\item Вычислите рациональным способом:
		\begin{tasks}(2)
			\task \( 21^2+16^2 - 21 \cdot 32 \)
			\task \( 15^2 + 3^2 + 6 \cdot 15 \)
			\task \( 58^2+23^2+46 \cdot 58 \)
			\task \( 87^2+12^2+87 \cdot 24 \)
		\end{tasks}
		\item Найдите значение выражения при заданных значениях переменной:
		\begin{tasks}(2)
			\task \( \dfrac{(2x)^4}{(4x)^2} \) при \( x=-\dfrac{2}{3} \)
			\task \( \dfrac{(9y)^3}{(3y)^5} \) при \( y=\dfrac{1}{3} \)
			\task \( \dfrac{(2a)^2 \cdot (2a)^3}{(4a)^2} \) при \( a=-0,1 \)
			\task \( \dfrac{(4c)^5 \cdot (2c)^6}{(4c)^2} \) при \( c=-0,5 \)
		\end{tasks}
		\item Преобразуйте в многочлен стандартного вида:
		\begin{tasks}(1)
			\task \( 5(x-y)^2+(x-2y)^2 \)
			\task \( 2(a-1)^2+3(a+2)^2 \)
			\task \( 2(p-3q)^2-4(2p-q)^2-(2q-3p)(p+q) \)
			\task \( (4x-1)(x-3)-(4-5x)^2-(6x+5)(5x-6) \)
		\end{tasks}
		\item Вычислите:
		\begin{tasks}(2)
			\task \( \dfrac{5^{10} \cdot (5^3)^4}{5^{18}} \)
			\task \( \dfrac{3^{34}}{3^{17} \cdot (3^5)^2} \)
			\task \( \dfrac{2^{21} \cdot (2^3)^5}{2^{40}} \)
			\task \( \dfrac{7^{11}}{7^{2} \cdot (7^3)^4} \)
		\end{tasks}
		
	\end{listofex}
\end{class}
%END_FOLD

%BEGIN_FOLD % ====>>_ Домашняя работа 3 _<<====
\begin{homework}[number=3]
	\begin{listofex}
		\item Подставьте вместо * одночлен так, чтобы полученное равенство было верным:
		\begin{tasks}(2)
			\task \( (*-*)^2=*-5xa+a^2 \)
			\task \( (*-*)^2=x^2-3xb+* \)
			\task \( (*-*)^2=4x^{10}-*+b^2 \)
			\task \( (*)^2-(*)^2= 15 \cdot (10-*) \)
			\task \( (*+*)^3=8a^3+6 \cdot * \cdot * + * \cdot * \cdot b^4 + * \)
			%\task \( (*+*)^2=*+25x^4+2 \)
		\end{tasks}
		\item Вычислите:
		\begin{tasks}(2)
			\task \( (0,1)^{25} \cdot 10^{23} \)
			\task \( (2,5)^{43} \cdot 0,4^{41} \)
			\task \( \left( \mfrac{2}{2}{3} \right)^{11}  \cdot 0,375^8 \)
		\end{tasks}
		\item Вычислите рациональным способом:
		\begin{tasks}(2)
			\task \( \dfrac{18^2-54^2}{72} \)
			\task \( \dfrac{83^2-19^2}{39^2-25^2} \)
		\end{tasks}
		\item Решите уравнение:
		\begin{tasks}(2)
			\task \( (6x-8)(5x-8)-(3x-2)\cdot 10x=0 \)
			\task \( (9x+7)^2= (3x-2)(27x+7) \)
		\end{tasks}
	\end{listofex}
\end{homework}
%END_FOLD

%BEGIN_FOLD % ====>>_____ Занятие 7 _____<<====
\begin{class}[number=7]
	\title{Подготовка к проверочной}
	\begin{listofex}
		\item Применить формулы квадратов и кубов:
		\begin{tasks}(4)
			\task \( (a+4b)^2 \)
			\task \( (5a-23)^3 \)
			\task \( (3,5-11b)^2 \)
			\task \( (5+0,2b)^3 \)
			\task \( (6a-\dfrac{2}{3}b)^2 \)
			\task \( (12a-1)^3 \)
			\task \( (6a-1,5b)^2 \)
			\task \( \left( \dfrac{1}{5}x + 15y \right)^3 \)
		\end{tasks}
		\item Подставьте вместо * одночлен так, чтобы полученное равенство было верным:
		\begin{tasks}(2)
			\task \( (a+*)^2=*+14a+* \)
			\task \( (3x-*)^2=*-30xa^3+* \)
			\task \( (7a+*)^3=*+*+42ab^2+*  \)
			\task \( (*-2y)^3=*-6x^4y+*+* \)
		\end{tasks}
		\item Представьте в виде произведения:
		\begin{tasks}(2)
			\task \( 4xy+12y-4x-12 \)
			\task \( 60+ab-30b-12a \)
			\task \( -abc-5ac-4ab-20a \)
			\task \( a^3+a^2b+a^2+ab \)
			\task \( 45b+6a-3ab-90 \)
			\task \( -5xy-40y-15x-120 \)
		\end{tasks}
		\item Вычислите рациональным способом:
		\begin{tasks}(2)
			\task \( 75^2-73^2 \)
			\task \( 123^2-125^2 \)
			\task \( \dfrac{65^2-91^2}{26} \)
			\task \( \dfrac{22^2+15^2}{37} - 22 \cdot 15 \)
		\end{tasks}
		\item Решите уравнения:
		\begin{tasks}(1)
			\task \( (2x+1)^2= (4x-1,5)(x+3) \)
			\task \( (x-1)(6x-5)=(2x-0,5)(3x+4) \)
			%\task \( (4x+3)(-3x+2) - 2(6-2x)(3x+1,5)=0 \)
			%\task \( (8x-3)^2 - (16x-4)(4x+2,5)=15 \)
		\end{tasks}
		
	\end{listofex}
\end{class}
%END_FOLD

%BEGIN_FOLD % ====>>_ Проверочная работа _<<====
\begin{exam}
	\begin{listofex}
		\item Подставьте вместо * одночлен так, чтобы полученное равенство было верным:
		\begin{tasks}(2)
			\task \( (7+*)^2=36n^2+*+* \)
			\task \( (*+*)^2=25+*+16t^4 \)
			%\task \( (9+*)^2=4c+*+* \)
			\task \( (*+6t)^2=*+36b^7t+* \)
			%\task \( (*+*)^3= -a^3+*-3ab^4+* \)
			\task \( (*-2y)^3=8x^3-*+*-* \)
		\end{tasks}
		
		\item Представьте в виде произведения:
		\begin{tasks}(2)
			\task \( x^2-2xc+c^2-d^2 \)
			\task \( c^2+2c+1-a^2 \)
			\task \( ac^4-c^4+ac^3-c^3 \)
			\task \( x^3-x^2y+x^2-xy \)
		\end{tasks}
		\item Вычислите рациональным способом:
		\begin{tasks}(3)
			\task \( \dfrac{32^2-71^2}{78}  \)
			\task \( \dfrac{83^2-19^2}{39^2-52^2} \)
			%\task \( \dfrac{57^2-33^2}{43^2-67^2} \)
			\task \( (1,4)^{21} \cdot \left( \dfrac{5}{7} \right)^{19} \)
		\end{tasks}
		\item Решите уравнения:
		\begin{tasks}(1)
			\task \( (3x+5)^2= (3x-11)(3x+2) \)
			%\task \( (2x-1)(3x-6)=(2x-2)(3x+4) \)
			%\task \( (6x+3)(-7x+4) - 2(11-3x)(7x+5)=0 \)
			\task \( (6x-8)^2 - (7,2x+5)(5x+0,5)=4 \)
		\end{tasks}
	\end{listofex}
\end{exam}
%END_FOLD