%
%===============>>  ГРУППА 7-1 МОДУЛЬ 6  <<=============
%
\setmodule{6}

%BEGIN_FOLD % ====>>_____ Занятие 1 _____<<====
\begin{class}[number=1]
	\begin{definit}
		Куб суммы: \( (a+b)^3=a^3+3a^2b+3ab^2+b^3 \)
	\end{definit}
	\begin{listofex}
		\item Раскройте скобки:
		\begin{tasks}(2)
			\task \( (2x+3y)^3 \)
			\task \( (4x+7y)^3 \)
			\task \( (1,5x+4)^3 \)
			\task \( (11x+8)^3 \)
		\end{tasks}
	\end{listofex}
		\begin{definit}
			Куб разности: \( (a-b)^3=a^3-3a^2b+3ab^2-b^3 \)
		\end{definit}
	\begin{listofex}[resume]
		\item Раскройте скобки:
		\begin{tasks}(2)
			\task \( (x-2y)^3 \)
			\task \( (3x-4y)^3 \)
			\task \( (5x-8)^3 \)
			\task \( (12x-5)^3 \)
		\end{tasks}
		
	\end{listofex}
	\begin{definit}
		Сумма кубов: \( a^3+b^3=(a+b)(a^2-ab+b^2) \) \\
		Разность кубов: \( a^3-b^3=(a-b)(a^2+ab+b^2) \)
	\end{definit}
	\begin{listofex}[resume]
		\item Примените формулы суммы и разности кубов:
		\begin{tasks}(2)
			\task \( x^3+4^3 \)
			\task \( 2^3-y^3 \)
			\task \( 8x^3-27y^3 \)
			\task \( (ab)^3+c^6 \)
		\end{tasks}
	\end{listofex}
		\begin{definit}
		Чтобы вынести общий множитель за скобки нужно выполнить следующие действия:
		\\ 1) Находим число, на которое делятся без остатка числовые коэффициенты каждого одночлена.
		\\ 2) Находим буквенные множители, которые повторяются в каждом одночлене. Выносим их за скобку в наименьшей степени.
		\end{definit}
	\begin{listofex}[resume]
		\item Найдите и вынесите общий множитель:
		\begin{tasks}(2)
			\task \( 19x-19y \)
			\task \( 6x^3-4 \)
			\task \( xy^2+x^3y \)
			\task \( 2x^4-6x^2 \)
			\task \( 3xy^2+6x^2 \)
			\task \( 9x^2+11x^4 \)
			\task \( 4x^2y^5+24xy^3 \)
			\task \( 16x^2y^2+8x^3y \)
			\task \( \dfrac{1}{4}y^5-\dfrac{1}{2}x^5 \)
			\task \( 6,8x^2y^4 + 6,9xy^5 \)
			\task \( 8,1x^3y^3 + 4,05x^{11}y \)
			\task \( 75y^3 - 105xy \)
		\end{tasks}
		\item Упростите выражение и приведите подобные слагаемые:
		\begin{tasks}(2)
			\task \( (a-b)^3+(b-2a)^2 \)
			\task \( (4a+2b)^3+(a-5b)^3 \)
			\task \( a^3-8b^3+(a-b)^3 \)
			\task \( a^6+b^3+(a^2-2b)^3 \)
			\task \( a^3-27b^3 - 2(a-3b)^3 \)
			\task \( (3a^2-4b)^3+(2a^2-8b)^3 \)
		\end{tasks}
	\end{listofex}
\end{class}
%END_FOLD

%BEGIN_FOLD % ====>>_____ Занятие 2 _____<<====
\begin{class}[number=2]
	\begin{listofex}
		\item Раскройте скобки:
		\begin{tasks}(2)
			\task \( (x+y)^3 \)
			\task \( (3x-4y)^3 \)
			\task \( (0,4x+3,5y)^3 \)
			\task \( (2x^2-0,8)^3 \)
		\end{tasks}
		\item Примените формулы суммы и разности кубов:
		\begin{tasks}(2)
			\task \( x^3+(7y)^3 \)
			\task \( (2x)^3-(0,5y)^3 \)
			\task \( 8x^3-64y^3 \)
			\task \( (4x)^3+0,2^3 \)
		\end{tasks}
		\item Найдите и вынесите общий множитель:
		\begin{tasks}(2)
			\task \( x^2y-5x \)
			\task \( 11x^3-44x^2 \)
			\task \( 27xy^2+6x^3y^2 \)
			\task \( 16x^4-6x^2 \)
		\end{tasks}
		\item Вычислите рациональным способом:
		\begin{tasks}(2)
			\task \( 78^2-77^2 \)
			\task \( 65^2-64^2 \)
			\task \( 123^2-124^2 \)
			\task \( 22^3-18^3 \)
			\task \( 95,5^3+4,5^3 \)
			\task \( 22^3-15,2^3 \)
			\task \( \dfrac{73^2-54^2}{19} \)
			\task \( \dfrac{65^2-91^2}{26} \)
			\task \( \dfrac{83^2-19^2}{39^2-25^2} \)
			\task \( \dfrac{57^2-33^2}{43^2-67^2} \)
			\task \( \dfrac{32^3+17^3}{49}-32\cdot17 \)
			\task \( \dfrac{73^3-37^3}{36}-37\cdot73 \)
		\end{tasks}
		\item Упростите выражение и найдите его значение:
		\begin{tasks}(1)
			\task \( \dfrac{(2a)^3(3a^2)^2}{(6a^4)^2} \), при \(a=\dfrac{5}{6}\).
			\task \( \dfrac{(7x-7y)(x^2+xy+y^2)}{14} \), при \( x=5, y=3 \)
		\end{tasks}
	\end{listofex}
\end{class}
%END_FOLD

%BEGIN_FOLD % ====>>_ Домашняя работа 1 _<<====
\begin{homework}[number=1]
	\begin{listofex}
		\item Раскройте скобки:
		\begin{tasks}(2)
			\task \( (2x+5y)^3 \)
			\task \( (4x+1)^3 \)
			\task \( (5x-4y)^3 \)
			\task \( (2x-3)^3 \)
		\end{tasks}
		\item Примените формулы суммы и разности кубов:
		\begin{tasks}(2)
			\task \( (3x)^3+y^3 \)
			\task \( 3^3+x^3 \)
		\end{tasks}
		\item Найдите и вынесите общий множитель:
		\begin{tasks}(2)
			\task \( x^2y^2-5xy^3 \)
			\task \( 2x^3-4x^2 \)
			\task \( 2,4xy^3+0,3x^3y^2 \)
			\task \( \dfrac{xy^4}{3}-\dfrac{x^2y^2}{6} \)
		\end{tasks}
		\item Вычислите рациональным способом:
		\begin{tasks}(2)
			\task \( 11^2-10^2 \)
			\task \( 12^2-23^2 \)
			\task \( \dfrac{18^2-12^2}{6} \)
			\task \( \dfrac{51^2-37^2}{15^2-1^2} \)
			\task \( \dfrac{28^3-12^3}{16}-14\cdot24 \)
		\end{tasks}
	\end{listofex}
\end{homework}
%END_FOLD

%BEGIN_FOLD % ====>>_____ Занятие 3 _____<<====
\begin{class}[number=3]
	\begin{listofex}
		\item Занятие 3 
	\end{listofex}
\end{class}
%END_FOLD

%BEGIN_FOLD % ====>>_____ Занятие 4 _____<<====
\begin{class}[number=4]
	\begin{listofex}
		\item Занятие 4
	\end{listofex}
\end{class}
%END_FOLD

%BEGIN_FOLD % ====>>_ Домашняя работа 2 _<<====
\begin{homework}[number=2]
	\begin{listofex}
		\item Домашняя работа 2
	\end{listofex}
\end{homework}
%END_FOLD

%BEGIN_FOLD % ====>>_____ Занятие 5 _____<<====
\begin{class}[number=5]
	\begin{listofex}
		\item Занятие 5
	\end{listofex}
\end{class}
%END_FOLD

%BEGIN_FOLD % ====>>_____ Занятие 6 _____<<====
\begin{class}[number=6]
	\begin{listofex}
		\item Занятие 6
	\end{listofex}
\end{class}
%END_FOLD

%BEGIN_FOLD % ====>>_ Домашняя работа 3 _<<====
\begin{homework}[number=3]
	\begin{listofex}
		\item Домашняя работа 3
	\end{listofex}
\end{homework}
%END_FOLD

%BEGIN_FOLD % ====>>_____ Занятие 7 _____<<====
\begin{class}[number=7]
	\title{Подготовка к проверочной}
	\begin{listofex}
		\item Занятие 7
	\end{listofex}
\end{class}
%END_FOLD

%BEGIN_FOLD % ====>>_ Проверочная работа _<<====
\begin{exam}
	\begin{listofex}
		\item Проверочная
	\end{listofex}
\end{exam}
%END_FOLD