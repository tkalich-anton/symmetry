%Группа 101-1 Модуль 2
\title{Занятие №1}
\begin{listofex}
	\item Найдите:
	\begin{enumcols}[itemcolumns=3]
		\item \( \dfrac{2}{3} \) от \( 15 \)
		\item \( \dfrac{54}{101} \) от \( 505 \)
		\item \( \dfrac{17}{21} \) от \( 63 \)
	\end{enumcols}
	\item Найдите:
	\begin{enumcols}[itemcolumns=4]
		\item НОД\( (30;\;25) \)
		\item НОД\( (24;\;40) \)
		\item НОК\( (30;\;25) \)
		\item НОК\( (24;\;40) \)
	\end{enumcols}
	\item На свой день рождения Алиса купила \( 560 \) кг фруктов (на весь класс). Из них \( 4/7 \) составляют яблоки, а остальное --- апельсины. Сколько килограммов апельсинов купила Алиса. Какую часть от всех фруктов составляют апельсины?
	\item Сколько градусов составляет \( 4/15 \) прямого угла? Сколько градусов составляет \( 7/20 \) развернутого угла?
	\item Рабочий за \( 4 \) дня окончил некоторую работу, сделав в первый день \( 3/20 \) всей работы, во второй день \( 7/40 \), а в третий --- \( 3/8 \). Какую часть работы он сделал в четвертый день?
	\item В первый день турист прошел \( 42 \) км, что составляет \( 7/11 \) всего пути. Сколько километров осталось пройти туристу?
	\item Решить пропорцию:
	\begin{enumcols}[itemcolumns=4]
		\item \( \dfrac{5}{8}=\dfrac{15}{a} \)
		\item \( \dfrac{x}{7}=\dfrac{5}{14} \)
		\item \( 5:x=15:12 \)
		\item \( \dfrac{x}{7}=12:17 \)
	\end{enumcols}
\end{listofex}
\newpage
\title{Занятие №2}
\begin{listofex}
	\item Найдите:
	\begin{enumcols}[itemcolumns=3]
		\item \( \dfrac{2}{7} \) от \( 28 \)
		\item \( \dfrac{54}{60} \) от \( 420 \)
		\item \( \dfrac{17}{31} \) от \( 124 \)
	\end{enumcols}
	\item Вася прочитал \( 13/15 \) книги. Сколько страниц прочитал Вася, если в книге \( 195 \) страниц?
	\item Федя читает книжку, в которой \( 720 \) страниц. За первый день он прочитал \( 5/12 \) всей книжки, а за второй --- \( 7/18 \) всей книжки. Сколько страниц ему осталось прочитать?
	\item Автомобиль проехал \( 575 \) км, что составляет \( 23/25 \) расстояния между двумя городами. Найдите расстояние между городами.
	\item Решить пропорцию:
	\begin{enumcols}[itemcolumns=4]
		\item \( \dfrac{20}{8}=\dfrac{x}{6} \)
		\item \( \dfrac{x}{8}=\dfrac{9}{4} \)
		\item \( 2:x=5:25 \)
		\item \( \dfrac{x}{1}=2:7 \)
	\end{enumcols}
	\item Найдите:
	\begin{enumcols}[itemcolumns=4]
		\item НОД\( (45;\;60) \)
		\item НОД\( (64;\;96) \)
		\item НОД\( (120;\;260) \)
		\item НОД\( (30;\;150) \)
	\end{enumcols}
	\item Найдите:
	\begin{enumcols}[itemcolumns=4]
		\item НОК\( (16;\;24) \)
		\item НОК\( (45;\;60) \)
		\item НОК\( (27;\;36) \)
		\item НОК\( (125;\;75) \)
	\end{enumcols}
\end{listofex}
\newpage
\title{Домашняя работа №1}
\begin{listofex}
	\item Найдите:
	\begin{enumcols}[itemcolumns=3]
		\item \( \dfrac{5}{6} \) от \( 48 \)
		\item \( \dfrac{99}{100} \) от \( 900 \)
		\item \( \dfrac{31}{28} \) от \( 56 \)
	\end{enumcols}
	\item Длина дороги \( 84 \) км. За первый день бригада рабочих отремонтировала \( 5/12 \) дороги, а за второй день --- \( 5/14 \) дороги. Сколько километров осталось отремонтировать?
	\item Заказанная работа была выполнена в \( 3 \) дня. В первый день было сделано \( 4/15 \) всей работы, во второй --- \( 5/12 \) всей работы. Какая часть работы была сделана в третий день?
	\item Вася прочитал \( 195 \) страниц, что составляет \( 13/15 \) всей книги. Сколько страниц в книге?
	\item Решить пропорцию:
	\begin{enumcols}[itemcolumns=4]
		\item \( \dfrac{12}{8}=\dfrac{15}{a} \)
		\item \( \dfrac{x}{16}=\dfrac{5}{8} \)
		\item \( 35:x=14:6 \)
		\item \( \dfrac{x}{7}=12:17 \)
	\end{enumcols}
	\item Найдите:
	\begin{enumcols}[itemcolumns=4]
		\item НОД\( (48;\;72) \)
		\item НОД\( (36;\;42) \)
		\item НОК\( (48;\;72) \)
		\item НОК\( (36;\;42) \)
	\end{enumcols}
\end{listofex}
%\newpage
%\title{Занятие №3}
%\begin{listofex}
%
%\end{listofex}
%\newpage
%\title{Занятие №4}
%\begin{listofex}
%
%\end{listofex}
%\newpage
%\title{Домашняя работа №2}
%\begin{listofex}
%
%\end{listofex}
%\newpage
%\title{Занятие №5}
%\begin{listofex}
%
%\end{listofex}
%\newpage
%\title{Занятие №6}
%\begin{listofex}
%
%\end{listofex}
%\newpage
%\title{Занятие №7}
%\begin{listofex}
%
%\end{listofex}
%\newpage
%\title{Проверочная работа}
%\begin{listofex}
%
%\end{listofex}