%
%===============>>  ГРУППА 6-1 МОДУЛЬ 4  <<=============
%
\setmodule{4}
%
%===============>>  Занятие 1  <<===============
%
\begin{class}[number=1]
	\begin{definit}
		Чтобы поделить число на десятичную дробь, будем придерживаться следующего алгоритма:
		\begin{enumcols}[itemcolumns=1]
			\item Делить на десятичную дробь нельзя!
			\item \textbf{Для деления на десятичную дробь нужно перенести в делимом и в
			делителе запятую на столько цифр вправо, сколько их после запятой в
			делителе.}
			\item \textbf{Если в десятичной части делимого (справа от запятой) знаков меньше, чем в делителе, то во время переноса запятой добавляем в делимом нули для каждого недостающего разряда.}
			\item \textbf{Выполнить деление десятичной дроби на натуральное число.}
			\item Еще раз подчеркнем, что избавляться от запятой нужно только в
			делителе, в то время как в делимом запятая может остаться!
		\end{enumcols}
	\end{definit}
	\begin{listofex}
		\item Выполнить деление (в столбик):
		\begin{enumcols}[itemcolumns=4]
			\item \( 3,3:0,3 \)
			\item \( 2:0,5 \)
			\item \( 1:0,1 \)
			\item \( 5,1:0,17 \)
			\item \( 3:0,06 \)
			\item \( 25,2:0,4 \)
			\item \( 200,1:0,69 \)
			\item \( 6:0,0064 \)
		\end{enumcols}
		\item Сравнить:
		\begin{enumcols}[itemcolumns=2]
			\item \( \dfrac{4}{11} \) и \( 0,4 \)
			\item \( \mfrac{2}{4}{11} \) и \( 2,36 \)
		\end{enumcols}
	\end{listofex}
	\begin{definit}
		Чтобы найти число, которое составляет часть \( A \) от целого числа \( B \), нужно \( A\cdot B \)
	\end{definit}
	\begin{listofex}[resume]
		\item Найти часть от числа:
		\begin{enumcols}[itemcolumns=4]
			\item \( 0,3 \) от \( 15 \)
			\item \( 0,12 \) от \( 0,25 \)
			\item \( 0,8 \) от \( \dfrac{5}{64} \)
			\item \( \dfrac{1}{12} \) от \( 1,44 \)
		\end{enumcols}
	\end{listofex}
	\begin{definit}
	Чтобы найти целое число, часть \( M \) от которого составляет число \( K \), нужно \( K:M \).
	\end{definit}
	\begin{listofex}[resume]
		\item Найти целое, если:
		\begin{enumcols}[itemcolumns=2]
			\item \( 0,1 \) целого составляет \( 5 \)
			\item \( 0,7 \) целого составляет \( 49 \)
			\item \( 6,25 \) целого составляет \( 225 \)
			\item \( 0,032 \) целого составляет \( 11 \)
		\end{enumcols}
	\end{listofex}
	\begin{definit}
		Чтобы определить, какую часть от первого числа \( A \) составляет второе число \( B \), необходимо \( A:B \).
	\end{definit}
	\begin{listofex}[resume]
		\item Какую часть составляет первое число от второго? Ответ дайте в десятичной дроби:
		\begin{enumcols}[itemcolumns=4]
			\item \( 2 \) от \( 8 \)
			\item \( 3,6 \) от \( 12 \)
			\item \( 0,45 \) от \( 3,6 \)
			\item \( \mfrac{2}{2}{3} \) от \( \mfrac{5}{1}{3} \)
		\end{enumcols}
	\end{listofex}
	
\end{class}
%
%===============>>  Занятие 2  <<===============
%
\begin{class}[number=2]
	\begin{listofex}
		\item Выполнить деление (в столбик):
		\begin{enumcols}[itemcolumns=4]
			\item \( 2:0,4 \)
			\item \( 0,48:0,08 \)
			\item \( 70:1,75 \)
			\item \( 23,53:2,6 \)
			\item \( 0,09:0,001 \)
			\item \( 49,56:0,007 \)
			\item \( 56,58:0,0082 \)
			\item \( 648,432:0,0058 \)
		\end{enumcols}
		\item Сравнить:
		\begin{enumcols}[itemcolumns=2]
			\item \( \dfrac{1}{3} \) и \( 0,33 \)
			\item \( \mfrac{3}{13}{24} \) и \( 3,54167 \)
		\end{enumcols}
		\item Найти часть от числа:
		\begin{enumcols}[itemcolumns=4]
			\item \( 0,7 \) от \( 20 \)
			\item \( 0,24 \) от \( 1,15 \)
			\item \( 0,1 \) от \( \dfrac{30}{17} \)
			\item \( \dfrac{1}{20} \) от \( 2,2 \)
		\end{enumcols}
		\item Найти целое, если:
		\begin{enumcols}[itemcolumns=2]
			\item \( 0,5 \) целого составляет \( 14 \)
			\item \( 1,12 \) целого составляет \( 22,4 \)
			\item \( 1,25 \) целого составляет \( 6 \)
			\item \( 0,8 \) целого составляет \( 0,4 \)
		\end{enumcols}
		\item Какую часть составляет первое число от второго? Ответ дайте в десятичной дроби:
		\begin{enumcols}[itemcolumns=4]
			\item \( 7 \) от \( 35 \)
			\item \( 1,89 \) от \( 12,6 \)
			\item \( 1,425 \) от \( 5,7 \)
			\item \( \mfrac{2}{1}{2} \) от \( \mfrac{8}{1}{3} \)
		\end{enumcols}
		\item Найдите \( 0,73 \) числа, \( 0,21 \) которого равны \( 1,575 \).
		\item В двух ящиках было \( 38,25 \) кг гвоздей. Если из одного ящика переложить в другой \( 4,75 \) кг гвоздей, то в обоих ящиках гвоздей станет поровну. Сколько кг гвоздей было в каждом ящике?
	\end{listofex}
\end{class}
%
%===============>>  Домашняя работа 1  <<===============
%
\begin{homework}[number=1]
	\begin{listofex}
		\item Выполнить деление (в столбик):
		\begin{enumcols}[itemcolumns=4]
			\item \( 1,5:0,8 \)
			\item \( 2:0,2 \)
			\item \( 52,5:1,4 \)
			\item \( 3:0,06 \)
			\item \( 58,36:0,1 \)
			\item \( 85,69:41,8 \)
			\item \( 1,006:0,8 \)
			\item \( 397,5:0,53 \)
		\end{enumcols}
		\item Сравнить:
		\begin{enumcols}[itemcolumns=2]
			\item \( 0,3 \) и \( \dfrac{1}{3} \)
			\item \( 1,07 \) и \( \mfrac{1}{7}{101} \)
		\end{enumcols}
		\item Найти часть от числа:
		\begin{enumcols}[itemcolumns=4]
			\item \( 0,13 \) от \( 70 \)
			\item \( 1,2 \) от \( 2,25 \)
			\item \( 0,6 \) от \( \dfrac{5}{6} \)
			\item \( \dfrac{7}{55} \) от \( 5,5 \)
		\end{enumcols}
		\item Найти целое, если:
		\begin{enumcols}[itemcolumns=2]
			\item \( 0,3 \) целого составляет \( 12 \)
			\item \( 2,7 \) целого составляет \( 54 \)
			\item \( 0,75 \) целого составляет \( 15 \)
			\item \( 0,032 \) целого составляет \( 11 \)
		\end{enumcols}
		\item Какую часть составляет первое число от второго? Ответ дайте в десятичной дроби:
		\begin{enumcols}[itemcolumns=4]
			\item \( 4 \) от \( 100 \)
			\item \( 1,425 \) от \( 5,7 \)
			\item \( 1,95 \) от \( 1,3 \)
			\item \( \mfrac{3}{41}{100} \) от \( \mfrac{7}{3}{4} \)
		\end{enumcols}
		\item Найдите \( 0,89 \) числа, \( 0,37 \) которого равны \( 425,5 \).
	\end{listofex}
\end{homework}
%
%===============>>  Занятие 3  <<===============
%
%\begin{class}[number=3]
%	\begin{listofex}
%		\item Пусто
%	\end{listofex}
%\end{class}
%
%===============>>  Занятие 4  <<===============
% смещение на одно занятие с прошлого месяца
%\begin{class}[number=4]
%	\begin{listofex}
%		\item Пусто
%	\end{listofex}
%\end{class}
%
%===============>>  Домашняя работа 2  <<===============
%
%\begin{homework}[number=2]
%	\begin{listofex}
%
%	\end{listofex}
%\end{homework}
%
%===============>>  Занятие 5  <<===============
% смещение на одно занятие с прошлого месяца
%\begin{class}[number=5]
%	\begin{listofex}
%		\item Пусто
%	\end{listofex}
%\end{class}
%
%===============>>  Домашняя работа 3  <<===============
%
%\begin{homework}[number=2]
%	\begin{listofex}
%
%	\end{listofex}
%\end{homework}
%\newpage
%\title{Подготовка к проверочной работе}
%\begin{listofex}
%	
%\end{listofex}
%
%===============>>  Занятие 7  <<===============
%
%\begin{class}[number=7]
%	\begin{listofex}
%	
%	\end{listofex}
%\end{class}
%
%===============>>  Провечная работа  <<===============
%
%\begin{exam}
%	\begin{listofex}
%	
%	\end{listofex}
%\end{exam}