%
%===============>>  ГРУППА 6-2 МОДУЛЬ 4  <<=============
%
\setmodule{4}
%
%===============>>  Занятие 1  <<===============
%
\begin{class}[number=1]
	\begin{listofex}
		\item Вычислить:
		\begin{enumcols}[itemcolumns=4]
			\item \( 68\cdot\mfrac{2}{3}{34} \)
			\item \( \mfrac{4}{1}{2}\cdot\mfrac{2}{4}{5} \)
			\item \( \mfrac{3}{3}{11}\cdot\mfrac{7}{1}{3} \)
			\item \( \left( \mfrac{2}{1}{3} \right)^3 \)
		\end{enumcols}
		\item Вычислить: \( \mfrac{1}{1}{2}\cdot\mfrac{1}{1}{3}\cdot\mfrac{1}{1}{4}\cdot\mfrac{1}{1}{5} \)
		\item Вычислить: 
		\begin{enumcols}[itemcolumns=2]
			\item \( \left( \dfrac{5}{18}+\dfrac{7}{12}+\dfrac{4}{9} \right)\cdot\left( 1-\dfrac{20}{47} \right)\cdot\left( \mfrac{1}{1}{4}-\dfrac{17}{20} \right) \)
			\item \( 7\cdot\left( \mfrac{6}{8}{21}+\mfrac{4}{11}{14} \right)-11\cdot\left( \mfrac{3}{3}{22}-\mfrac{2}{37}{44} \right) \)
%			\item \( \left( \left( \dfrac{5}{12}+\dfrac{17}{30}+\dfrac{17}{20} \right)\cdot60-\mfrac{55}{3}{4} \right)\cdot\dfrac{4}{7}-31 \)
		\end{enumcols}
		\item Чему равна площадь комнаты, длина и ширина которой равны \( \mfrac{5}{1}{2} \) м и \( \mfrac{3}{1}{2} \) м?
	\end{listofex}
	\begin{definit}
		Два числа, произведение которых равно 1, называются \textbf{взаимно обратными}.
	\end{definit}
	\begin{listofex}[resume]
	\item Проверьте, являются ли числа взаимно обратными?
	\begin{enumcols}[itemcolumns=4]
		\item \( \dfrac{5}{6} \) и \( \mfrac{1}{1}{5} \)
		\item \( \mfrac{3}{2}{3} \) и \( \dfrac{3}{11} \)
		\item \( \mfrac{2}{1}{57} \) и \( \dfrac{57}{115} \)
		\item \( \dfrac{1}{57} \) и \( 57 \)
	\end{enumcols}
	Какую закономерность и способ для определения взаимно обратных чисел можно заметить?
	\item Найдите число, обратное данному:
	\begin{enumcols}[itemcolumns=4]
		\item \( 15 \)
		\item \( \dfrac{1}{4} \)
		\item \( \dfrac{23}{47} \)
		\item \( \mfrac{3}{4}{7} \)
	\end{enumcols}
	\end{listofex}
\end{class}
%
%===============>>  Занятие 2  <<===============
%
\begin{class}[number=2]
	\begin{definit}
		Чтобы поделить дробь \( \dfrac{a}{b} \) на дробь \( \dfrac{c}{d} \), нужно умножить \( \dfrac{a}{b} \) на дробь обратную \( \dfrac{c}{d} \), то есть:
		\[ \dfrac{a}{b}:\dfrac{c}{d}=\dfrac{a}{b}\cdot\dfrac{d}{c}=\dfrac{a\cdot d}{b\cdot c};\qquad\dfrac{a}{b}:c=\dfrac{a}{b}\cdot\dfrac{1}{c}=\dfrac{a}{b\cdot c};\qquad a:\dfrac{c}{d}=a\cdot\dfrac{d}{c}=\dfrac{a\cdot d}{c} \]
	\end{definit}
	\begin{listofex}
		\item Выполните деление:
		\begin{enumcols}[itemcolumns=4]
			\item \( \mfrac{40}{1}{2}:3 \)
			\item \( \mfrac{9}{2}{3}:2 \)
			\item \( \mfrac{3}{3}{8}:9 \)
			\item \( \dfrac{27}{32}:81 \)
		\end{enumcols}
		\item Выполните деление:
		\begin{enumcols}[itemcolumns=4]
			\item \( 2:3 \)
			\item \( 150:225 \)
			\item \( 21:28 \)
			\item \( 45:20 \)
		\end{enumcols}
		\item Выполните деление:
		\begin{enumcols}[itemcolumns=4]
			\item \( 10:\dfrac{1}{10} \)
			\item \( 20:\dfrac{1}{25} \)
			\item \( 2:\dfrac{2}{3} \)
			\item \( 3:\dfrac{3}{5} \)
			\item \( \mfrac{2}{2}{3}:\dfrac{2}{3} \)
			\item \( 6:\mfrac{1}{1}{2} \)
			\item \( 4:\mfrac{1}{1}{3} \)
			\item \( 18:\dfrac{54}{61} \)
		\end{enumcols}
		\item Выполните деление:
		\begin{enumcols}[itemcolumns=4]
			\item \( \dfrac{2}{3}:\dfrac{5}{7} \)
			\item \( \dfrac{5}{6}:\dfrac{7}{12} \)
			\item \( \dfrac{3}{5}:\dfrac{9}{25} \)
			\item \( \dfrac{15}{16}:\dfrac{3}{10} \)
		\end{enumcols}
		\item Скорость велосипедиста \( 12  \) км/ч. Сколько километров он проедет за \( \dfrac{1}{2} \) ч, \( \dfrac{1}{3} \) ч, \( \mfrac{3}{3}{4} \) ч?
		\item Два велосипедиста выехали одновременно из одного и того же пункта и двигались в
		одном и том же направлении. Скорость первого велосипедиста \( \mfrac{12}{3}{4} \) км/ч, а скорость второго в \( \mfrac{1}{1}{5} \) раза больше. Какое расстояние будет между ними через \( \mfrac{1}{1}{5} \) ч?
	\end{listofex}
\end{class}
%
%===============>>  Домашняя работа 1  <<===============
%
\begin{homework}[number=1]
	\begin{listofex}
		\item Вычислить:
		\begin{enumcols}[itemcolumns=4]
			\item \( 21\cdot\mfrac{3}{1}{35} \)
			\item \( 11\cdot\mfrac{3}{16}{22} \)
			\item \( \mfrac{2}{1}{2}\cdot\dfrac{3}{5} \)
			\item \( \mfrac{3}{3}{5}\cdot\mfrac{5}{5}{8} \)
		\end{enumcols}
		\item Проверьте, являются ли числа взаимно обратными?
		\begin{enumcols}[itemcolumns=3]
			\item \( \dfrac{2}{3} \) и \( \mfrac{1}{1}{2} \)
			\item \( 5 \) и \( \dfrac{1}{5} \)
			\item \( \mfrac{7}{4}{5} \) и \( \dfrac{5}{39} \)
		\end{enumcols}
		\item Найдите число, обратное данному:
		\begin{enumcols}[itemcolumns=5]
			\item \( 10 \)
			\item \( 1 \)
			\item \( \dfrac{1}{37} \)
			\item \( \dfrac{15}{67} \)
			\item \( \mfrac{10}{1}{4} \)
		\end{enumcols}
		\item Выполните деление:
		\begin{enumcols}[itemcolumns=4]
			\item \( 7:12 \)
			\item \( 120:200 \)
			\item \( 35:42 \)
			\item \( 33:77 \)
		\end{enumcols}
		\item Выполните деление:
		\begin{enumcols}[itemcolumns=6]
			\item \( \mfrac{23}{5}{7}:3 \)
			\item \( 99:\dfrac{1}{3} \)
			\item \( 24:\dfrac{4}{9} \)
			\item \( \dfrac{2}{9}:\dfrac{4}{3} \)
			\item \( \dfrac{7}{12}:\dfrac{21}{16} \)
			\item \( \mfrac{5}{1}{2}:\mfrac{3}{2}{3} \)
		\end{enumcols}
		\item Вычислить: \( \mfrac{22}{1}{5}\cdot\dfrac{5}{37}\cdot\dfrac{2}{3} \)
		\item Вычислить: \( 9\cdot\left( \mfrac{5}{11}{18}+\mfrac{4}{13}{27} \right) - 11\cdot\left( \mfrac{4}{3}{11}-\mfrac{3}{13}{33} \right) \)
	\end{listofex}
\end{homework}
%
%===============>>  Занятие 3  <<===============
%
\begin{class}[number=3]
	\begin{listofex}
		\item Сократить дробь:
		\begin{tasks}(3)
			\task \( \dfrac{128}{768} \)
			\task \( \dfrac{8\cdot9}{7\cdot9} \)
			\task \( \dfrac{13\cdot12\cdot11}{22\cdot24\cdot26} \)
		\end{tasks}
		\item Вычислить:
		\begin{tasks}(3)
			\task \( \mfrac{7}{2}{5}+\mfrac{4}{4}{5} \)
			\task \( \dfrac{7}{15}\cdot45 \)
			\task \( \mfrac{1}{15}{22}\cdot66 \)
			\task \( \mfrac{2}{1}{2}\cdot\mfrac{1}{3}{4}\cdot\mfrac{1}{1}{3}\mfrac{2}{2}{3} \)
			\task \( \left( \dfrac{2}{3} \right)^3 \)
		\end{tasks}
		\item Вычислить:\quad\( \left( \mfrac{3}{1}{2}:\mfrac{4}{2}{3}+\mfrac{4}{2}{3}:\mfrac{3}{1}{2} \right)\cdot\mfrac{4}{4}{5} \).
		\item Найти \( \dfrac{34}{49} \) от \( 980 \).
		\item Сколько минут в \( \dfrac{5}{12} \) часа?
		\item От дыни массой \( 3 \) кг \( 200 \) г отрезали \( \dfrac{3}{16} \) от всей дыни, а потом еще \( \dfrac{1}{8} \) от всей дыни. Чему равна масса
		каждого отрезанного куска?
		\item В книге \( 600 \) страниц. Мальчик прочитал в первый день половину всех страниц, а во второй --- треть \underline{оставшихся}. Сколько страниц ему осталось прочитать?
		\item Решить уравнение:
		\begin{tasks}(2)
			\task \( x-\dfrac{5}{12}=\dfrac{7}{8} \)
			\task \( \mfrac{3}{2}{7}-x=\mfrac{2}{7}{9} \)
		\end{tasks}
	\end{listofex}
\end{class}
%
%===============>>  Занятие 4  <<===============
% смещение на одно занятие с прошлого месяца
\begin{class}[number=4]
	\begin{listofex}
		\item Сократить дробь:
		\begin{tasks}(3)
			\task \( \dfrac{77}{242} \)
			\task \( \dfrac{34\cdot30}{17\cdot15} \)
			\task \( \dfrac{32\cdot33\cdot35}{14\cdot44\cdot8} \)
		\end{tasks}
		\item Вычислить:
		\begin{tasks}(3)
			\task \( \mfrac{13}{7}{12}-\mfrac{3}{5}{12} \)
			\task \( \dfrac{34}{100}\cdot200 \)
			\task \( \mfrac{5}{12}{19}\cdot38 \)
			\task \( \mfrac{2}{1}{2}\cdot\mfrac{5}{2}{5}\cdot\mfrac{2}{1}{11}\)
			\task \( \left( \mfrac{2}{1}{2} \right)^2 \)
		\end{tasks}
		\item Вычислить:\quad\( \mfrac{2}{3}{4}:\left( \mfrac{1}{1}{2}-\dfrac{2}{5} \right)+\left( \dfrac{3}{4}+\dfrac{5}{6} \right):\mfrac{3}{1}{6} \).
		\item Найти:\quad\( \dfrac{35}{57} \) от \( 114 \).
		\item Сколько сантиметров в \( \dfrac{3}{20} \) метра?
		\item Купили кусок ткани длиной \( 25 \) м \( 50 \) см и из \( \dfrac{1}{5} \) сшили платье. Сколько ткани ушло на платье и сколько осталось?
		\item В книге \( 300 \) страниц. В первый день мальчик прочитал треть всех страниц, а во второй --- четверть \underline{оставшихся}. Сколько страниц ему осталось прочитать?
		\item Решить уравнение:
		\begin{tasks}(2)
			\task \( x-\dfrac{5}{16}=\dfrac{3}{32} \)
			\task \( \mfrac{5}{5}{9}-x=\mfrac{3}{1}{9} \)
		\end{tasks}
	\end{listofex}
\end{class}
%
%===============>>  Домашняя работа 2  <<===============
%
\begin{homework}[number=2]
	\begin{listofex}
		\item Сократить дробь:
		\begin{tasks}(4)
			\task \( \dfrac{85}{255} \)
			\task \( \dfrac{11\cdot26}{121\cdot13} \)
			\task \( \dfrac{62\cdot18}{6\cdot31} \)
			\task \( \dfrac{15\cdot17\cdot21}{14\cdot51\cdot5} \)
		\end{tasks}
		\item Вычислить:
		\begin{tasks}(3)
			\task \( \mfrac{5}{6}{7}-\mfrac{2}{4}{7} \)
			\task \( \dfrac{6}{7}+\mfrac{2}{5}{14} \)
			\task \( \dfrac{6}{25}\cdot100 \)
			\task \( \mfrac{2}{3}{16}\cdot32 \)
			\task \( \dfrac{4}{7}\cdot\mfrac{2}{1}{5}\cdot\mfrac{3}{4}{6}\)
			\task \( \left( \mfrac{3}{2}{3} \right)^2 \)
		\end{tasks}
		\item Вычислить:
		\begin{tasks}(2)
			\task \( \mfrac{1}{3}{4}:\left( \mfrac{3}{4}{5}-\mfrac{1}{2}{5} \right)+\left( \dfrac{3}{4}-\dfrac{7}{11} \right):\mfrac{1}{1}{4} \).
			\task \( \left( \mfrac{4}{1}{2}\cdot\mfrac{2}{1}{3}+\mfrac{2}{2}{3}\cdot\mfrac{1}{1}{2} \right)\cdot\mfrac{4}{5}{6} \).
		\end{tasks}
		\item Найти: \( \dfrac{12}{33} \) от \( 132 \).
		\item Который сейчас час, если прошло  \( 60\% \) суток?
		\item \(120\) г золота сплавили с \(80\) г серебра. Найдите процентное содержание золота и серебра в полученном сплаве.
		\item Длина дороги 20 км. Заасфальтировали \( \dfrac{2}{5} \) дороги. Сколько километров дороги заасфальтировали? Сколько осталось заасфальтировать?
		\item Решить уравнение:
		\begin{tasks}(2)
			\task \( \dfrac{13}{16}-x=\dfrac{10}{32} \)
			\task \( \mfrac{1}{2}{3}+x=\mfrac{3}{1}{2}-\mfrac{1}{3}{9} \)
		\end{tasks}
	\end{listofex}
\end{homework}
%
%===============>>  Занятие 5  <<===============
% смещение на одно занятие с прошлого месяца
%\begin{class}[number=5]
%	\begin{listofex}
%		\item Пусто
%	\end{listofex}
%\end{class}
%
%===============>>  Домашняя работа 3  <<===============
%
%\begin{homework}[number=2]
%	\begin{listofex}
%
%	\end{listofex}
%\end{homework}
%\newpage
%\title{Подготовка к проверочной работе}
%\begin{listofex}
%	
%\end{listofex}
%
%===============>>  Занятие 7  <<===============
%
%\begin{class}[number=7]
%	\begin{listofex}
%	
%	\end{listofex}
%\end{class}
%
%===============>>  Провечная работа  <<===============
%
%\begin{exam}
%	\begin{listofex}
%	
%	\end{listofex}
%\end{exam}