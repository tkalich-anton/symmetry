%Группа 6-2 Модуль 3
\cheadbf{Модуль 3 Занятие 1}
\begin{definit}
	\textbf{Основное свойство дроби} --- Если числитель и знаменатель дроби увеличить или уменьшить в одно и тоже количество раз, то значение дроби не изменится.
\end{definit}
\begin{listofex}
	\item Привести дробь:
	\begin{enumcols}[itemcolumns=2]
		\item \( \dfrac{3}{4} \) к знаменателю \( 20 \)
		\item \( \dfrac{5}{7} \) к знаменателю \( 63 \)
		\item \( \dfrac{11}{12} \) к знаменателю \( 144 \)
		\item \( \dfrac{9}{20} \) к знаменателю \( 160 \)
		\item \( \dfrac{11}{9} \) к знаменателю \( 99 \)
		\item \( \dfrac{4}{15} \) к знаменателю \( 60 \)
		\item \( \dfrac{13}{14} \) к знаменателю \( 56 \)
	\end{enumcols}
	\item Сократить дробь:
	\begin{enumcols}[itemcolumns=6]
		\item \( \dfrac{3}{5} \)
		\item \( \dfrac{10}{14} \)
		\item \( \dfrac{15}{25} \)
		\item \( \dfrac{32}{48} \)
		\item \( \dfrac{32}{128} \)
		\item \( \dfrac{18}{27} \)
		\item \( \dfrac{17}{170} \)
		\item \( \dfrac{20}{36} \)
		\item \( \dfrac{15}{35} \)
		\item \( \dfrac{36}{92} \)
		\item \( \dfrac{42}{66} \)
		\item \( \dfrac{27}{63} \)
	\end{enumcols}
	\item Сократить дробь:
	\begin{enumcols}[itemcolumns=5]
		\item \( \dfrac{7\cdot3}{3\cdot14} \)
		\item \( \dfrac{14\cdot9}{6\cdot7\cdot3} \)
		\item \( \dfrac{25\cdot99}{81\cdot55} \)
		\item \( \dfrac{16\cdot45\cdot19}{81\cdot57\cdot4} \)
		\item \( \dfrac{3\cdot14\cdot62}{31\cdot10\cdot27} \)
	\end{enumcols}
\end{listofex}
\begin{definit}
	Чтобы найти часть \( \dfrac{a}{b} \) от числа \( c \), необходимо число \( c \) поделить на \( b \) и потом полученный результат умножить на \( a \).
\end{definit}
\begin{listofex}[resume]
	\item Потратили \( \dfrac{3}{8} \) от \( 400 \) руб. Сколько рублей потратили? Сколько еще осталось?
	\item Длина веревки \( 27 \) м. Отрезали \( \dfrac{2}{9} \) ее длины. Сколько метров веревки отрезали? Сколько осталось?
	\item Вычислить:
	\begin{enumcols}[itemcolumns=3]
		\item \( \dfrac{1}{4} \) от \( 64 \)
		\item \( \dfrac{17}{11} \) от \( 121 \)
		\item \( \dfrac{17}{15} \) от \( 75 \)
	\end{enumcols}
	\item Рабочий за \( 4 \) дня окончил некоторую работу, сделав в первый день \( 3/20 \) всей работы, во второй день \( 7/40 \), а в третий --- \( 3/8 \). Какую часть работы он сделал в четвертый день?
\end{listofex}
\newpage
\cheadbf{Модуль 3 Занятие 2}
\begin{listofex}
	\item Представьте неправильную дробь в виде смешанного числа:\\[1em]
	\textit{Пример:} \( \dfrac{16}{3}=\left[ \dfrac{3\cdot5+1}{3}=\dfrac{3\cdot5}{3}+\dfrac{1}{3} \right]=3\dfrac{1}{3} \)\\
	\begin{enumcols}[itemcolumns=7]
		\item \( \dfrac{12}{5} \)
		\item \( \dfrac{27}{2} \)
		\item \( \dfrac{39}{4} \)
		\item \( \dfrac{28}{9} \)
		\item \( \dfrac{54}{12} \)
		\item \( \dfrac{89}{2} \)
		\item \( \dfrac{112}{25} \)
	\end{enumcols}
	\item Представьте смешанное число в виде неправильной дроби:
	\begin{enumcols}[itemcolumns=4]
		\item \( \mfrac{2}{3}{2} \)
		\item \( \mfrac{7}{12}{15} \)
		\item \( \mfrac{10}{10}{9} \)
		\item \( \mfrac{9}{3}{5} \)
	\end{enumcols}
	\item Произвести сложение или вычитание дробей и, если возможно, сократить дробь:
	\begin{enumcols}[itemcolumns=4]
		\item \( \dfrac{12}{17}+\dfrac{3}{17} \)
		\item \( \dfrac{4}{9}+\dfrac{5}{9} \)
		\item \( \dfrac{15}{21}+\dfrac{16}{21} \)
		\item \( \dfrac{13}{50}+\dfrac{7}{50} \)
		\item \( \dfrac{15}{11}-\dfrac{4}{11} \)
		\item \( \dfrac{68}{30}-\dfrac{8}{30} \)
		\item \( \dfrac{112}{20}-\dfrac{2}{20} \)
		\item \( \dfrac{55}{42}-\dfrac{4}{42}-\dfrac{11}{42} \)
	\end{enumcols}
	\item Вычислить:
	\begin{enumcols}[itemcolumns=4]
		\item \( \mfrac{3}{14}{20}-\dfrac{12}{20} \)
		\item \( \mfrac{3}{3}{7}-\dfrac{5}{7} \)
		\item \( \mfrac{4}{2}{5}+5\:\dfrac{2}{5} \)
	\end{enumcols}
	\item Вычислить:
	\begin{enumcols}[itemcolumns=3]
		\item \( 8\:\dfrac{1}{9}+8\:\dfrac{7}{9}-3\:\dfrac{5}{9} \)
		\item \( 17\:\dfrac{15}{17}+5\:\dfrac{13}{17}+19\:\dfrac{11}{17} \)
		\item \( 5\:\dfrac{3}{8}-2\:\dfrac{5}{8} \)
		\item \( 6\:\dfrac{1}{3}-5\:\dfrac{2}{3} \)
		\item \( 4\:\dfrac{7}{12}-1\:\dfrac{5}{12}+2\:\dfrac{11}{12} \)
		\item \( 12\:\dfrac{3}{7}-4\:\dfrac{5}{7}-5\:\dfrac{4}{7} \)
	\end{enumcols}
	\item Привести к общему знаменателю:
	\begin{enumcols}[itemcolumns=3]
		\item \( \dfrac{4}{25} \) и \( \dfrac{1}{5} \)
		\item \( \dfrac{3}{17} \) и \( \dfrac{2}{34} \)
		\item \( \dfrac{10}{9} \) и \( \dfrac{5}{3} \)
		\item \( \dfrac{3}{24} \) и \( \dfrac{1}{12} \)
		\item \( \dfrac{5}{20} \) и \( \dfrac{13}{50} \)
		\item \( \dfrac{6}{25} \) и \( \dfrac{13}{75} \)
	\end{enumcols}
	\item Сравнить дроби:
	\begin{enumcols}[itemcolumns=4]
		\item \( \dfrac{5}{7} \) и \( \dfrac{2}{3} \)
		\item \( \dfrac{5}{12} \) и \( \dfrac{7}{16} \)
		\item \( \dfrac{33}{15} \) и \( \dfrac{23}{12} \)
		\item \( \dfrac{13}{21} \) и \( \dfrac{15}{28} \)
	\end{enumcols}
\end{listofex}
\newpage
\cheadbf{Модуль 3 Домашняя работа 1}
\begin{listofex}
	\item Привести к общему знаменателю:
	\begin{enumcols}[itemcolumns=6]
		\item \( \dfrac{15}{24} \) и \( \dfrac{16}{36} \)
		\item \( \dfrac{1}{33} \) и \( \dfrac{1}{55} \)
		\item \( \dfrac{4}{11} \) и \( \dfrac{16}{121} \)
		\item \( \dfrac{24}{100} \) и \( \dfrac{13}{4} \)
		\item \( \dfrac{11}{90} \) и \( \dfrac{33}{50} \)
		\item \( \dfrac{13}{250} \) и \( \dfrac{14}{350} \)
	\end{enumcols}
	\item Привести дробь:
	\begin{enumcols}[itemcolumns=2]
		\item \( \dfrac{13}{14} \) к знаменателю \( 56 \)
	\end{enumcols}
	\item Сравнить дроби:
	\begin{enumcols}[itemcolumns=4]
		\item \( \dfrac{131}{200} \) и \( \dfrac{54}{100} \)
		\item \( \dfrac{37}{50} \) и \( \dfrac{97}{150} \)
		\item \( \dfrac{33}{13} \) и \( \dfrac{45}{15} \)
		\item \( \dfrac{15}{70} \) и \( \dfrac{1}{30} \)
	\end{enumcols}
	\item Представьте неправильную дробь в виде смешанного числа:
	\begin{enumcols}[itemcolumns=6]
		\item \( \dfrac{27}{13} \)
		\item \( \dfrac{251}{2} \)
		\item \( \dfrac{542}{70} \)
		\item \( \dfrac{2002}{1000} \)
		\item \( \dfrac{145}{32} \)
		\item \( \dfrac{56}{3} \)
	\end{enumcols}
	\item Вычислить:
	\begin{enumcols}[itemcolumns=2]
		\item \( \dfrac{3}{5} \) от \( 25 \)
		\item \( \dfrac{5}{6} \) от \( 196 \)
	\end{enumcols}
	\item Вычислить:
	\begin{enumcols}[itemcolumns=4]
		\item \( 1-\dfrac{1}{2} \)
		\item \( 5\:\dfrac{6}{7}-5\:\dfrac{1}{7} \)
		\item \( 7\:\dfrac{56}{75}-7 \)
		\item \( 34\:\dfrac{7}{9}-6\:\dfrac{7}{9} \)
	\end{enumcols}
	\item Вычислить:
	\begin{enumcols}[itemcolumns=3]
		\item \( 6\:\dfrac{1}{3}-5\:\dfrac{2}{3} \)
		\item \( 4\:\dfrac{7}{12}-1\:\dfrac{5}{12}+2\:\dfrac{11}{12} \)
		\item \( 12\:\dfrac{3}{7}-4\:\dfrac{5}{7}-5\:\dfrac{4}{7} \)
	\end{enumcols}
\end{listofex}
%\newpage
%\title{Занятие №3}
%\begin{listofex}
%	
%\end{listofex}
%\newpage
%\title{Занятие №4}
%\begin{listofex}
%	
%\end{listofex}
\newpage
\cheadbf{Модуль 3 Домашняя работа 2}
\begin{listofex}
	\item Вычислить:
	\begin{enumcols}[itemcolumns=2]
		\item \( 4\:\dfrac{2}{3}-1\:\dfrac{1}{3} \)
		\item \( 4\:\dfrac{3}{4}-2\:\dfrac{1}{3}+4\:\dfrac{11}{12} \)
		\item \( 15\:\dfrac{5}{6}-4\:\dfrac{2}{3}-5\:\dfrac{1}{2} \)
		\item \( 11\:\dfrac{2}{3}+7\:\dfrac{1}{2}+2\:\dfrac{4}{9} \)
	\end{enumcols}
	\item Решить уравнение:
	\begin{enumcols}[itemcolumns=3]
		\item \( x+\mfrac{4}{1}{7}=\mfrac{9}{5}{7} \)
		\item \( \mfrac{9}{5}{12}-x=\mfrac{7}{20}{21} \)
		\item \( x-\mfrac{4}{3}{11}=\mfrac{2}{5}{22} \)
	\end{enumcols}
	\item Вычислить:
	\begin{enumcols}[itemcolumns=2]
		\item \( 3\cdot1,2+3\cdot1,5 \)
		\item \( 7\cdot3,5-2,8\cdot4,8 \)
		\item \( (1,33-0,6)\cdot(1,34+3,4)-5\cdot0,31 \)
		\item \( 9,8\cdot8,8\cdot2,5-0,05\cdot1312 \)
	\end{enumcols}
	\item Вычислить:
	\begin{enumcols}[itemcolumns=3]
		\item \( 3,3:1,65 \)
		\item \( 27:0,5 \)
		\item \( 32,25:0,15 \)
		\item \( 0,156\cdot1,7 \)
		\item \( 0,014\cdot0,03 \)
		\item \( 0,11\cdot 11 \)
	\end{enumcols}
	\item Сократить дробь:
	\begin{enumcols}[itemcolumns=6]
		\item \(\dfrac{14}{18}\)
		\item \(\dfrac{20}{70} \)
		\item \(\dfrac{11}{66} \)
		\item \(\dfrac{34}{51} \)
		\item \(\dfrac{68}{102} \)
		\item \(\dfrac{720}{640} \)
	\end{enumcols}
	\item Сократить дробь:
	\begin{enumcols}[itemcolumns=3]
		\item \(\dfrac{7 \cdot 3}{3 \cdot 14}\)
		\item \(\dfrac{3 \cdot 5 \cdot 28}{15 \cdot 49} \)
		\item \(\dfrac{49 \cdot 22 \cdot 25}{33 \cdot 28 \cdot 35} \)
	\end{enumcols}
	\item Найдите площадь прямоугольника, если ширина его \( 25 \) м, а длина в \( 5 \) раза меньше ширины.
\end{listofex}
%\newpage
%\title{Занятие №5}
%\begin{listofex}
%	
%\end{listofex}
%\newpage
%\title{Занятие №6}
%\begin{listofex}
%	
%\end{listofex}
%\newpage
%\title{Домашняя работа №3}
%\begin{listofex}
%	
%\end{listofex}
%\newpage
%\title{Подготовка к проверочной работе}
%\begin{listofex}
%	
%\end{listofex}
%\newpage
%\title{Проверочная работа}
%\title{Вариант 1}
%\begin{listofex}
%	
%\end{listofex}
%\newpage
%\title{Проверочная работа}
%\begin{listofex}
%	
%\end{listofex}