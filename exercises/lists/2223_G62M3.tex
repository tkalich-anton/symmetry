%Группа 6-2 Модуль 3
\cheadbf{Модуль 3 Занятие 1}
\begin{definit}
	\textbf{Основное свойство дроби} --- Если числитель и знаменатель дроби увеличить или уменьшить в одно и тоже количество раз, то значение дроби не изменится.
\end{definit}
\begin{listofex}
	\item Привести дробь:
	\begin{enumcols}[itemcolumns=2]
		\item \( \dfrac{3}{4} \) к знаменателю \( 20 \)
		\item \( \dfrac{5}{7} \) к знаменателю \( 63 \)
		\item \( \dfrac{11}{12} \) к знаменателю \( 144 \)
		\item \( \dfrac{9}{20} \) к знаменателю \( 160 \)
		\item \( \dfrac{11}{9} \) к знаменателю \( 99 \)
		\item \( \dfrac{4}{15} \) к знаменателю \( 60 \)
		\item \( \dfrac{13}{14} \) к знаменателю \( 56 \)
	\end{enumcols}
	\item Сократить дробь:
	\begin{enumcols}[itemcolumns=6]
		\item \( \dfrac{3}{5} \)
		\item \( \dfrac{10}{14} \)
		\item \( \dfrac{15}{25} \)
		\item \( \dfrac{32}{48} \)
		\item \( \dfrac{32}{128} \)
		\item \( \dfrac{18}{27} \)
		\item \( \dfrac{17}{170} \)
		\item \( \dfrac{20}{36} \)
		\item \( \dfrac{15}{35} \)
		\item \( \dfrac{36}{92} \)
		\item \( \dfrac{42}{66} \)
		\item \( \dfrac{27}{63} \)
	\end{enumcols}
	\item Сократить дробь:
	\begin{enumcols}[itemcolumns=5]
		\item \( \dfrac{7\cdot3}{3\cdot14} \)
		\item \( \dfrac{14\cdot9}{6\cdot7\cdot3} \)
		\item \( \dfrac{25\cdot99}{81\cdot55} \)
		\item \( \dfrac{16\cdot45\cdot19}{81\cdot57\cdot4} \)
		\item \( \dfrac{3\cdot14\cdot62}{31\cdot10\cdot27} \)
	\end{enumcols}
\end{listofex}
\begin{definit}
	Чтобы найти часть \( \dfrac{a}{b} \) от числа \( c \), необходимо число \( c \) поделить на \( b \) и потом полученный результат умножить на \( a \).
\end{definit}
\begin{listofex}[resume]
	\item Потратили \( \dfrac{3}{8} \) от \( 400 \) руб. Сколько рублей потратили? Сколько еще осталось?
	\item Длина веревки \( 27 \) м. Отрезали \( \dfrac{2}{9} \) ее длины. Сколько метров веревки отрезали? Сколько осталось?
	\item Вычислить:
	\begin{enumcols}[itemcolumns=3]
		\item \( \dfrac{1}{4} \) от \( 64 \)
		\item \( \dfrac{17}{11} \) от \( 121 \)
		\item \( \dfrac{17}{15} \) от \( 75 \)
	\end{enumcols}
	\item Рабочий за \( 4 \) дня окончил некоторую работу, сделав в первый день \( 3/20 \) всей работы, во второй день \( 7/40 \), а в третий --- \( 3/8 \). Какую часть работы он сделал в четвертый день?
\end{listofex}
\newpage
\cheadbf{Модуль 3 Занятие 2}
\begin{listofex}
	\item Представьте неправильную дробь в виде смешанного числа:\\[1em]
	\textit{Пример:} \( \dfrac{16}{3}=\left[ \dfrac{3\cdot5+1}{3}=\dfrac{3\cdot5}{3}+\dfrac{1}{3} \right]=3\dfrac{1}{3} \)\\
	\begin{enumcols}[itemcolumns=7]
		\item \( \dfrac{12}{5} \)
		\item \( \dfrac{27}{2} \)
		\item \( \dfrac{39}{4} \)
		\item \( \dfrac{28}{9} \)
		\item \( \dfrac{54}{12} \)
		\item \( \dfrac{89}{2} \)
		\item \( \dfrac{112}{25} \)
	\end{enumcols}
	\item Представьте смешанное число в виде неправильной дроби:
	\begin{enumcols}[itemcolumns=4]
		\item \( \mfrac{2}{3}{2} \)
		\item \( \mfrac{7}{12}{15} \)
		\item \( \mfrac{10}{10}{9} \)
		\item \( \mfrac{9}{3}{5} \)
	\end{enumcols}
	\item Произвести сложение или вычитание дробей и, если возможно, сократить дробь:
	\begin{enumcols}[itemcolumns=4]
		\item \( \dfrac{12}{17}+\dfrac{3}{17} \)
		\item \( \dfrac{4}{9}+\dfrac{5}{9} \)
		\item \( \dfrac{15}{21}+\dfrac{16}{21} \)
		\item \( \dfrac{13}{50}+\dfrac{7}{50} \)
		\item \( \dfrac{15}{11}-\dfrac{4}{11} \)
		\item \( \dfrac{68}{30}-\dfrac{8}{30} \)
		\item \( \dfrac{112}{20}-\dfrac{2}{20} \)
		\item \( \dfrac{55}{42}-\dfrac{4}{42}-\dfrac{11}{42} \)
	\end{enumcols}
	\item Вычислить:
	\begin{enumcols}[itemcolumns=4]
		\item \( \mfrac{3}{14}{20}-\dfrac{12}{20} \)
		\item \( \mfrac{3}{3}{7}-\dfrac{5}{7} \)
		\item \( \mfrac{4}{2}{5}+5\:\dfrac{2}{5} \)
	\end{enumcols}
	\item Вычислить:
	\begin{enumcols}[itemcolumns=3]
		\item \( 8\:\dfrac{1}{9}+8\:\dfrac{7}{9}-3\:\dfrac{5}{9} \)
		\item \( 17\:\dfrac{15}{17}+5\:\dfrac{13}{17}+19\:\dfrac{11}{17} \)
		\item \( 5\:\dfrac{3}{8}-2\:\dfrac{5}{8} \)
		\item \( 6\:\dfrac{1}{3}-5\:\dfrac{2}{3} \)
		\item \( 4\:\dfrac{7}{12}-1\:\dfrac{5}{12}+2\:\dfrac{11}{12} \)
		\item \( 12\:\dfrac{3}{7}-4\:\dfrac{5}{7}-5\:\dfrac{4}{7} \)
	\end{enumcols}
	\item Привести к общему знаменателю:
	\begin{enumcols}[itemcolumns=3]
		\item \( \dfrac{4}{25} \) и \( \dfrac{1}{5} \)
		\item \( \dfrac{3}{17} \) и \( \dfrac{2}{34} \)
		\item \( \dfrac{10}{9} \) и \( \dfrac{5}{3} \)
		\item \( \dfrac{3}{24} \) и \( \dfrac{1}{12} \)
		\item \( \dfrac{5}{20} \) и \( \dfrac{13}{50} \)
		\item \( \dfrac{6}{25} \) и \( \dfrac{13}{75} \)
	\end{enumcols}
	\item Сравнить дроби:
	\begin{enumcols}[itemcolumns=4]
		\item \( \dfrac{5}{7} \) и \( \dfrac{2}{3} \)
		\item \( \dfrac{5}{12} \) и \( \dfrac{7}{16} \)
		\item \( \dfrac{33}{15} \) и \( \dfrac{23}{12} \)
		\item \( \dfrac{13}{21} \) и \( \dfrac{15}{28} \)
	\end{enumcols}
\end{listofex}
\newpage
\cheadbf{Модуль 3 Домашняя работа 1}
\begin{listofex}
	\item Привести к общему знаменателю:
	\begin{enumcols}[itemcolumns=6]
		\item \( \dfrac{15}{24} \) и \( \dfrac{16}{36} \)
		\item \( \dfrac{1}{33} \) и \( \dfrac{1}{55} \)
		\item \( \dfrac{4}{11} \) и \( \dfrac{16}{121} \)
		\item \( \dfrac{24}{100} \) и \( \dfrac{13}{4} \)
		\item \( \dfrac{11}{90} \) и \( \dfrac{33}{50} \)
		\item \( \dfrac{13}{250} \) и \( \dfrac{14}{350} \)
	\end{enumcols}
	\item Привести дробь:
	\begin{enumcols}[itemcolumns=2]
		\item \( \dfrac{13}{14} \) к знаменателю \( 56 \)
	\end{enumcols}
	\item Сравнить дроби:
	\begin{enumcols}[itemcolumns=4]
		\item \( \dfrac{131}{200} \) и \( \dfrac{54}{100} \)
		\item \( \dfrac{37}{50} \) и \( \dfrac{97}{150} \)
		\item \( \dfrac{33}{13} \) и \( \dfrac{45}{15} \)
		\item \( \dfrac{15}{70} \) и \( \dfrac{1}{30} \)
	\end{enumcols}
	\item Представьте неправильную дробь в виде смешанного числа:
	\begin{enumcols}[itemcolumns=6]
		\item \( \dfrac{27}{13} \)
		\item \( \dfrac{251}{2} \)
		\item \( \dfrac{542}{70} \)
		\item \( \dfrac{2002}{1000} \)
		\item \( \dfrac{145}{32} \)
		\item \( \dfrac{56}{3} \)
	\end{enumcols}
	\item Вычислить:
	\begin{enumcols}[itemcolumns=2]
		\item \( \dfrac{3}{5} \) от \( 25 \)
		\item \( \dfrac{5}{6} \) от \( 196 \)
	\end{enumcols}
	\item Вычислить:
	\begin{enumcols}[itemcolumns=4]
		\item \( 1-\dfrac{1}{2} \)
		\item \( 5\:\dfrac{6}{7}-5\:\dfrac{1}{7} \)
		\item \( 7\:\dfrac{56}{75}-7 \)
		\item \( 34\:\dfrac{7}{9}-6\:\dfrac{7}{9} \)
	\end{enumcols}
	\item Вычислить:
	\begin{enumcols}[itemcolumns=3]
		\item \( 6\:\dfrac{1}{3}-5\:\dfrac{2}{3} \)
		\item \( 4\:\dfrac{7}{12}-1\:\dfrac{5}{12}+2\:\dfrac{11}{12} \)
		\item \( 12\:\dfrac{3}{7}-4\:\dfrac{5}{7}-5\:\dfrac{4}{7} \)
	\end{enumcols}
\end{listofex}
\newpage
\cheadbf{}
\title{Занятие 3}
\begin{listofex}
	\item Вычислить:
	\begin{enumcols}[itemcolumns=5]
		\item \( \mfrac{3}{2}{17}+\dfrac{4}{17} \)
		\item \( \mfrac{1}{5}{9}+\mfrac{3}{4}{9} \)
		\item \( \mfrac{3}{7}{12}+\mfrac{1}{5}{12} \)
		\item \( \mfrac{4}{1}{5}+\mfrac{2}{3}{5} \)
		\item \( \mfrac{6}{7}{9}+\dfrac{8}{9} \)
	\end{enumcols}
	\item Вычислить:
	\begin{enumcols}[itemcolumns=3]
		\item \( \dfrac{2}{17} \) от \( 34 \)
		\item \( \mfrac{3}{2}{7} \) от \( 35 \)
		\item \( \mfrac{9}{1}{9} \) от \( 36 \)
	\end{enumcols}
	\item Привести к общему знаменателю:
	\begin{enumcols}[itemcolumns=5]
		\item \( \dfrac{3}{15} \) и \( \dfrac{2}{5} \)
		\item \( \dfrac{5}{44} \) и \( \dfrac{5}{77} \)
		\item \( \dfrac{6}{27} \) и \( \dfrac{2}{36} \)
		\item \( \dfrac{1}{100} \) и \( \dfrac{1}{4} \)
		\item \( \dfrac{3}{100} \) и \( \dfrac{3}{8} \)
	\end{enumcols}
	\item Решить уравнение:
	\begin{enumcols}[itemcolumns=3]
		\item \( x-\dfrac{1}{14}=\dfrac{3}{14} \)
		\item \( x+\dfrac{13}{99}=\dfrac{14}{99} \)
		\item \( \mfrac{2}{1}{5}-x=\mfrac{1}{2}{5} \)
	\end{enumcols}
	\item Из пункта в пункт \( A \) в пункт \( B \), отстоящий от пункта \( А \) на \( 27 \) км, отправился пешеход со скоростью \( 5 \) км/ч. Через \( 36 \) минут после этого навстречу ему из \( B \) вышел другой пешеход со скоростью \( 3 \) км/ч. Через какое время после выхода второго пешехода они встретятся? Найдите расстояние от пункта \( B \) до места их встречи.
\end{listofex}
\title{Занятие 4}
\begin{listofex}
	\item Сократить дробь:
	\begin{enumcols}[itemcolumns=6]
		\item \( \dfrac{14}{18} \)
		\item \( \dfrac{20}{70} \)
		\item \( \dfrac{11}{66} \)
		\item \( \dfrac{34}{51} \)
		\item \( \dfrac{68}{102} \)
		\item \( \dfrac{720}{640} \)
	\end{enumcols}
	\item Сократить дробь:
	\begin{enumcols}[itemcolumns=3]
		\item \( \dfrac{7\cdot3}{3\cdot14} \)
		\item \( \dfrac{3\cdot5\cdot28}{15\cdot49} \)
		\item \( \dfrac{49\cdot22\cdot25}{33\cdot28\cdot35} \)
	\end{enumcols}
	\item Вычислить:
	\begin{enumcols}[itemcolumns=4]
		\item \( \dfrac{1}{4}+\dfrac{2}{3} \)
		\item \( \dfrac{1}{12}+\dfrac{2}{3} \)
		\item \( \dfrac{1}{20}+\dfrac{1}{4}+\dfrac{2}{5} \)
		\item \( \dfrac{1}{2}-\dfrac{1}{8} \)
		\item \( \dfrac{3}{4}-\dfrac{1}{2}+\dfrac{7}{8} \)
		\item \( \mfrac{3}{1}{3}-\mfrac{1}{1}{2} \)
		\item \( \mfrac{3}{13}{44}-\mfrac{1}{7}{33} \)
		\item \( \mfrac{7}{4}{25}-\mfrac{2}{3}{4} \)
	\end{enumcols}
	\item Решить уравнение:
	\begin{enumcols}[itemcolumns=3]
		\item \( x-\dfrac{5}{12}=\dfrac{7}{8} \)
		\item \( x-\dfrac{5}{18}=\dfrac{2}{27} \)
		\item \( \mfrac{3}{2}{7}-x=\mfrac{2}{7}{9} \)
	\end{enumcols}
	\item Два поезда вышли одновременно навстречу друг другу с двух станций. Один поезд проходит все расстояние между станциями за \( 3 \) часа, а другой -- за \( 4 \) часа. Какую часть пути им останется пройти до встречи спустя час после выхода?
\end{listofex}
%\newpage
%\title{Домашняя работа №2}
%\begin{listofex}
%	
%\end{listofex}
\setcounter{definit}{0}
\newpage
\title{Занятие 5}
\begin{listofex}
	\item Вычислить:
	\begin{enumcols}[itemcolumns=4]
		\item \( \mfrac{3}{2}{5}+\mfrac{2}{3}{5} \)
		\item \( \mfrac{7}{5}{9}-\mfrac{3}{7}{9} \)
		\item \( \mfrac{2}{3}{7}+\mfrac{4}{3}{14} \)
		\item \( \mfrac{15}{17}{20}-\mfrac{3}{4}{10} \)
	\end{enumcols}
\end{listofex}
\begin{definit}
	Чтобы перемножить натуральное число на дробь, нужно умножить натуральное число на числитель этой дроби и результат поделить на знаменатель.
	\[ a\cdot\dfrac{m}{k}=\dfrac{a\cdot m}{k} \]
\end{definit}
\begin{listofex}
	\item Умножить и, по возможности, сократить дробь:
	\begin{enumcols}[itemcolumns=5]
		\item \( 2\cdot\dfrac{2}{5} \)
		\item \( \dfrac{4}{9}\cdot9 \)
		\item \( \dfrac{3}{25}\cdot15 \)
		\item \( 12\cdot\dfrac{5}{18} \)
		\item \( 13\cdot\dfrac{7}{78} \)
	\end{enumcols}
	\item Умножить и, по возможности, сократить дробь:
	\begin{enumcols}[itemcolumns=4]
		\item \( \mfrac{2}{1}{3}\cdot2\)
		\item \( \mfrac{4}{2}{7}\cdot14 \)
		\item \( \mfrac{11}{3}{16}\cdot16 \)
		\item \( 75\cdot\mfrac{1}{7}{30} \)
	\end{enumcols}
	\item Вычислить: \( 10\cdot\left( \mfrac{3}{2}{15}-\mfrac{2}{5}{18} \right)+12\cdot\left( \mfrac{1}{5}{6}+\mfrac{5}{3}{4} \right) \)
\end{listofex}
\begin{definit}
	Чтобы перемножить две дроби друг на друга, нужно результат перемножения числителей поделить на результат перемножения знаменателей.
	\[ \dfrac{a}{b}\cdot\dfrac{m}{k}=\dfrac{a\cdot m}{b\cdot k} \]
\end{definit}
\begin{listofex}[resume]
	\item Вычислить:
	\begin{enumcols}[itemcolumns=4]
		\item \( \dfrac{1}{3}\cdot\dfrac{2}{7} \)
		\item \( \dfrac{4}{5}\cdot\dfrac{5}{7} \)
		\item \( \dfrac{3}{4}\cdot\dfrac{2}{9} \)
		\item \( \dfrac{12}{13}\cdot\dfrac{13}{14}\cdot\dfrac{14}{15} \)
	\end{enumcols}
\end{listofex}
\begin{listofex}[resume]
	\item Вычислить:
	\begin{enumcols}[itemcolumns=4]
		\item \( \mfrac{1}{1}{2}\cdot\dfrac{1}{2} \)
		\item \( \dfrac{1}{9}\cdot\mfrac{27}{9}{11} \)
		\item \( \mfrac{1}{1}{3}\cdot\mfrac{2}{1}{4} \)
		\item \( \mfrac{4}{1}{2}\cdot\mfrac{2}{4}{5} \)
	\end{enumcols}
\end{listofex}
\begin{definit}
	Чтобы найти \( \dfrac{a}{b} \) от числа \( с \), необходимо \( \dfrac{a}{b}\cdot c \).
\end{definit}
\begin{listofex}[resume]
	\item Вычислить:
	\begin{enumcols}[itemcolumns=4]
		\item \( \dfrac{3}{5} \) от \( \mfrac{6}{2}{3} \)
		\item \( \dfrac{7}{9} \) от \( \mfrac{4}{1}{2} \)
		\item \( \dfrac{11}{48} \) от \( \mfrac{13}{1}{11} \)
		\item \( \mfrac{2}{3}{4} \) от \( \mfrac{1}{2}{3} \)
	\end{enumcols}
	\item Из двух городов, расстояние между которыми \( 71 \) км, одновременно навстречу друг другу
	выехали два автомобиля, скорости которых равны \( 64 \) км/ч и \( 78 \) км/ч соответственно. Через
	какое время они встретятся?
\end{listofex}
\newpage
\title{Занятие 6}
\begin{listofex}
	\item Вычислить:
	\begin{enumcols}[itemcolumns=4]
		\item \( \dfrac{343}{600}-\dfrac{19}{75} \)
		\item \( \mfrac{192}{5}{6}-\mfrac{88}{5}{84} \)
		\item \( \mfrac{64}{1}{99}-\mfrac{3}{1}{121} \)
		\item \( \mfrac{1}{5}{8}+\mfrac{4}{8}{17}+\dfrac{9}{17}+\mfrac{2}{3}{8} \)
	\end{enumcols}
\end{listofex}
\begin{definit}
	Если знаменатели двух дробей равны, то больше та дробь, чей числитель больше.
\end{definit}
\begin{listofex}[resume]
	\item Приведите дроби к общему знаменателю и расположите в порядке возрастания:
	\[ \dfrac{7}{12},\;\dfrac{2}{6},\;\mfrac{1}{5}{24},\;\dfrac{10}{6},\;\dfrac{23}{12},\;\dfrac{3}{2} \]
\end{listofex}
\begin{definit}
	Если числители двух дробей равны, то больше та дробь, чей знаменатель меньше.
\end{definit}
\begin{listofex}[resume]
	\item Приведите дроби к общему числителю и расположите в порядке убывания:
	\[ \dfrac{1}{3},\;\dfrac{2}{35},\;\dfrac{3}{47},\;\dfrac{4}{47},\;\dfrac{6}{55} \]
	\item Вычислить: \[ \left( 20-\mfrac{19}{3}{4} \right)+\left( \mfrac{17}{2}{5}-16 \right)+\left( \mfrac{2}{1}{2}-\dfrac{17}{24} \right) \]
\end{listofex}
\setcounter{definit}{0}
\newpage
\title{Занятие 7}
\begin{listofex}
	\item Сократить дробь:
	\begin{enumcols}[itemcolumns=3]
		\item \( \dfrac{14\cdot15}{21\cdot20} \)
		\item \( \dfrac{33\cdot16\cdot45}{75\cdot44\cdot12} \)
		\item \( \dfrac{18\cdot35\cdot19}{95\cdot3\cdot42} \)
	\end{enumcols}
	\item Сократить дробь:
	\begin{enumcols}[itemcolumns=3]
		\item \( \dfrac{3^{10}}{3^9} \)
		\item \( \dfrac{2^7\cdot3^8}{2^5\cdot3^{11}} \)
		\item \( \dfrac{2^3\cdot3^6\cdot5^2}{2^2\cdot3^5\cdot5^4} \)
	\end{enumcols}
	\item Вычислить:
	\begin{enumcols}[itemcolumns=5]
		\item \( \dfrac{7}{9}\cdot9 \)
		\item \( 15\cdot\dfrac{13}{30} \)
		\item \( 30\cdot\dfrac{7}{90} \)
		\item \( \dfrac{1}{10}\cdot15 \)
		\item \( \dfrac{1}{57}\cdot57 \)
	\end{enumcols}
	\item Вычислить:
	\begin{enumcols}[itemcolumns=4]
		\item \( 5\cdot\mfrac{2}{3}{10} \)
		\item \( 3\cdot\mfrac{5}{4}{21} \)
		\item \( 11\cdot\mfrac{2}{23}{77} \)
		\item \( \mfrac{3}{3}{37}\cdot111 \)
	\end{enumcols}
	\item Вычислить: \( 7\cdot\left( \mfrac{6}{8}{21}+\mfrac{4}{11}{14} \right)-11\cdot\left( \mfrac{3}{3}{22}-\mfrac{2}{37}{44} \right) \)
	\item Вычислить:
	\begin{enumcols}[itemcolumns=5]
		\item \( \mfrac{4}{1}{2}\cdot\dfrac{14}{45} \)
		\item \( \mfrac{3}{3}{5}\cdot\mfrac{5}{5}{8} \)
		\item \( \mfrac{1}{1}{24}\cdot\mfrac{11}{1}{5} \)
		\item \( \mfrac{18}{1}{3}\cdot\mfrac{1}{2}{11} \)
		\item \( \left( \dfrac{1}{2} \right)^3 \)
	\end{enumcols}
	\item Токарю и его ученику нужно обработать \( 420 \) деталей. Токарь, работая один, может
	выполнить эту работу за \( 20 \) часов, а его ученик --- за \( 60 \) часов. За сколько часов выполнят эту работу токарь и его ученик, работая вдвоем?
\end{listofex}
\newpage
\title{Занятие 8}
\begin{definit}
	Чтобы поделить дробь на целое число, нужно числитель поделить на произведения знаменателя на целое число.
	\[ \dfrac{a}{b}:c=\dfrac{a}{b\cdot c} \]
\end{definit}
\begin{listofex}
	\item Выполнить деление и сократите дробь:
	\begin{enumcols}[itemcolumns=4]
		\item \( \dfrac{4}{5}:2 \)
		\item \( \dfrac{11}{13}:11 \)
		\item \( \dfrac{5}{11}:10 \)
		\item \( \dfrac{20}{27}:5 \)
		\item \( \mfrac{22}{1}{3}:67 \)
		\item \( \mfrac{5}{1}{3}:2 \)
		\item \( \mfrac{14}{2}{7}:3 \)
		\item \( \dfrac{27}{32}:81 \)
	\end{enumcols}
\end{listofex}
\begin{definit}
	Чтобы поделить целое число на дробь, нужно целое число умножить на знаменатель и результат поделить на числитель.
	\[ c:\dfrac{a}{b}=\dfrac{c\cdot b}{a} \]
\end{definit}
\begin{listofex}[resume]
	\item Выполнить деление и сократите дробь:
	\begin{enumcols}[itemcolumns=5]
		\item \( 1:\dfrac{1}{2} \)
		\item \( 11:\dfrac{1}{13} \)
		\item \( 33:\dfrac{3}{5} \)
		\item \( 77:\dfrac{11}{5} \)
		\item \( 5:\dfrac{10}{25} \)
		\item \( 18:\dfrac{54}{61} \)
		\item \( 24:\dfrac{4}{9} \)
		\item \( 15:\dfrac{5}{7} \)
		\item \( 15:\dfrac{4}{15} \)
		\item \( 10:\dfrac{8}{7} \)
	\end{enumcols}
	\item Выполнить деление и сократите дробь:
	\begin{enumcols}[itemcolumns=4]
		\item \( 2:\mfrac{3}{1}{3} \)
		\item \( 1:\mfrac{1}{1}{2} \)
		\item \( 120:\mfrac{1}{4}{5} \)
		\item \( 100:\mfrac{7}{1}{7} \)
	\end{enumcols}
	\item Вычислить: \( \left( \dfrac{5}{18}+\dfrac{7}{12}+\dfrac{4}{9} \right)\cdot\left( 1-\dfrac{20}{47} \right)\cdot\left( \mfrac{1}{1}{4}-\dfrac{17}{20} \right) \)
	\item Вычислить:
	\begin{enumcols}[itemcolumns=5]
		\item \( \dfrac{4}{5}\cdot\dfrac{3}{8}\cdot\dfrac{3}{5}\cdot\dfrac{3}{4}\cdot\dfrac{2}{3}\)
		\item \( \mfrac{3}{1}{2}\cdot\mfrac{8}{1}{3}\cdot\dfrac{3}{25}\cdot5\cdot\mfrac{6}{1}{4}\cdot16 \)
	\end{enumcols}
\end{listofex}
%\newpage
%\title{Проверочная работа}
%\begin{listofex}
%	
%\end{listofex}