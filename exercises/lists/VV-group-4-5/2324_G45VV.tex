%
%===============>>  ГРУППА 11-6 МОДУЛЬ 8  <<=============
%
\setmodule{Вспомнить всё}

%BEGIN_FOLD % ====>>_____ Занятие 1 _____<<====
\begin{class}[number=1]
	\begin{listofex}
		\item Вычислите и выполните проверку:
		\begin{tasks}(4)
			\task \( 460+320 \)
			\task \( 780-650 \)
			\task \( 84:7 \)
			\task \( 3\cdot19 \)
		\end{tasks}
		\item Решите примеры:
		\begin{tasks}(2)
			\task \( 100-56:(38-30) \)
			\task \( 96-48:8\cdot6 \)
		\end{tasks}
		\item Вычислите в столбик:
		\begin{tasks}(3)
			\task \( 1000001-963228 \)
			\task \( 100102-89456 \)
			\task \( 78000-9559 \)
		\end{tasks}
		\item В \( 8 \) одинаковых бочонках \( 72 \) кг мёда. Сколько мёда в \( 10 \) таких же бочонках?
		\item Найдите площадь квадрата со стороной \( 5 \) дм.
		\item Решите уравнения:
		\begin{tasks}(2)
			\task \( 430+x=570-70 \)
			\task \( 240+130=x-20 \)
			\task \( x:5=35 \)
			\task \( 96:x=6 \)
			\task \( x\cdot15=75 \)
			\task \( x\cdot19=57 \)
		\end{tasks}
		\item В двух одинаковых по массе корзинах \( 28 \) кг яблок, а в одном пакете --- \( 2 \) кг. Во сколько раз корзина с яблоками тяжелее, чем пакет с яблоками?
		\item Папе \( 30 \) лет, а сыну \( 6 \) лет. Во сколько раз папа старше сына? На сколько папа старше сына? На сколько лет папа будет старше сына через \( 3 \) года? Во сколько раз папа будет старше сына через \( 6 \) лет?
		\item Площадь прямоугольника \( 36 \) дм\( ^2 \). Найди длину его второй стороны, если первая --- \( 4 \) дм, и вырази её в сантиметрах.
	\end{listofex}
\end{class}
%END_FOLD

%BEGIN_FOLD % ====>>_ Домашняя работа 1 _<<====
\begin{homework}[number=1]
		\begin{listofex}
			\item Домашняя работа
		\end{listofex}
\end{homework}
%END_FOLD

%BEGIN_FOLD % ====>>_____ Занятие 2 _____<<====
\begin{class}[number=2]
	\begin{listofex}
		\item Пирог разделили на \( 6 \) равных частей и взяли одну такую часть. Это одна шестая доля пирога. Какие доли получатся, если разделить на \( 2 \) равные части каждую шестую долю пирога? Какую долю от пирога будут занимать \( 3 \) такие части?
		\item Масса сушёных грибов составляет одну десятую часть массы свежих грибов. Сколько килограммов сушёных грибов можно получить из \( 30 \) кг свежих? Сколько килограммов свежих грибов надо взять, чтобы получить \( 6 \) кг сушёных?
		\item Петя купил упаковку корма для попугая. В упаковке \( 27 \) пакетиков. На сколько недель хватит попугаю этого корма, если каждую неделю он съедает по \( 7 \) пакетиков? Сколько ещё пакетиков надо докупить, чтобы корма хватило на \( 7 \) недель?
		\item Масса одного ящика с мандаринами \( 8 \) кг. Найди массу \( 9 \) коробок с бананами, если одна коробка с бананами на \( 3 \) кг легче одного ящика с мандаринами.
		\item Из \( 30 \) кг подсолнечника получают \( 6 \) кг масла. Сколько килограммов масла можно получить из \( 25 \) кг семян подсолнечника?
%		\item Вычислите:
%		\begin{tasks}(2)
%			\task \( 78-(72-62)\cdot4 \)
%			\task \( 37+(25-15)\cdot3 \)
%			\task \( 54:9+8\cdot5 \)
%			\task \( 50:(10\cdot5) \)
%		\end{tasks}
		\item \( 24 \) л фруктового сока разлили в \( 8 \) банок поровну. Сколько надо таких банок, чтобы разлить \( 18 \) л сока? \( 21 \) л сока?
		\item Масса \( 8 \) мешков картофеля \( 400 \) кг. Сколько таких мешков потребуется, чтобы засыпать в них \( 500 \) кг картофеля?
		\item Поставь скобки, чтобы равенства стали верными:
		\begin{tasks}(2)
			\task \( 100-24:2=38 \)
			\task \( 360:6+3=40 \)
			\task \( 32\cdot2-2=0 \)
			\task \( 300+20\cdot2:10=96 \)
			\task \( 420:10-4:2=35 \)
			\task \( 4\cdot120-120:6=0 \)
		\end{tasks}
	\end{listofex}
		\title{Дополнительные задания для 5 класса}
		\begin{listofex}
		\item Вычислите:
			\begin{tasks}(2)
			\task  \( 365:73+252\cdot84 \)
			\task \( 390:65+224\cdot32 \)
			\task \( 273150:45 \)
			\task \( 899200:16 \)
			\task \( 90000-508\cdot173 \)
			\task \( 5001-272\cdot16 \)
			\end{tasks}
		\item Реши уравнения:
		\begin{tasks}(2)
			\task \( (x+287)\cdot2=486 \)
			\task \( (403-x)\cdot3=347 \)
			\task \( x:11-22=22033 \)
			\task \( 9000:(10\cdot x)=4 \)
		\end{tasks}
		\item Экскаватором за \( 1 \) ч можно выкопать канаву длиной \( 20 \) м. Одну канаву копали \( 10 \) ч, а другую --- \( 12 \) ч. Найдите общую длину канав, которые выкопали за это время. Реши задачу двумя способами и выбери самый удобный.
		\item Вычисли:
		\begin{tasks}(2)
			\task \( 408\cdot270+21008:808 \)
			\task! \( (31460+1040):(150-2400:120) \)
			\task \( (78213-75209)\cdot207-45\cdot308 \)
			\task \( 434280:517\cdot306+27449 \)
		\end{tasks}
	\end{listofex}
\end{class}
%END_FOLD

%BEGIN_FOLD % ====>>_ Домашняя работа 2 _<<====
\begin{homework}[number=2]
	\begin{listofex}
		\item Домашняя работа
	\end{listofex}
\end{homework}
%END_FOLD

%BEGIN_FOLD % ====>>_____ Занятие 3 _____<<====
\begin{class}[number=3]
	\begin{listofex}
		\item Теплоход проходит за \( 4 \) ч такое же расстояние, как и моторная лодка за \( 9 \) ч. Найди скорость моторной лодки, если известно, что скорость теплохода \( 36 \) км/ч?
		\item За \( 1 \) ч (\( 60 \) мин), двигаясь с одинаковой скоростью, машина проходит \( 60 \) км. Какое расстояние она пройдёт за \( 10 \) мин?
		\item Мама дала Саше \( 200 \) р. и попросила купить молоко, кефир и сметану. Саша решил купить \( 2 \) пакета молока по \( 32 \) рубля, \( 3 \) пакета кефира по \( 27 \) рублей. и банку сметаны за \( 28 \) рублей. Хватит ли ему денег? Если да, сколько рублей сдачи он должен получить? Хватит ли денег на то, чтобы купить ещё один пакет молока? А ещё одну банку сметаны?
		\item В мастерской в первый день сшили \( 19 \) одинаковых рюкзаков, во второй --- \( 23 \) таких рюкзака. На все эти рюкзаки пошло \( 84 \) м ткани. Сколько метров ткани расходовали каждый день?
		\item Вычислить:
		\begin{tasks}(2)
			\task \( 260\cdot403-(568\cdot5-1840) \)
			\task \( 671\cdot223+(6000-87\cdot40) \)
			\task \( (28084+9038):(2000-1954) \)
			\task \( (34001-28911)\cdot(3000-2924) \)
		\end{tasks}
		\item Во сколько раз сумма чисел \( 933 \) и \( 1167 \) больше частного чисел \( 21600 \) и \( 72 \)?
		\item На сколько произведение чисел \( 752 \) и \( 30 \) больше разности этих чисел?
		\item Вычисли и замени крупные величины более мелкими:
		\begin{tasks}(2)
			\task \( 9 \) т \( 385 \) кг \( + \) \( 6 \) т \( 135 \) кг
			\task \( 13 \) км \( 250 \) м \( - \) \( 8 \) км \( 480 \) м
		\end{tasks}
		\item За \( 7 \) дней завод изготовил \( 588 \) станков. Сколько станков изготовит завод за \( 24 \) дня, если каждый день станут выпускать на \( 1 \) станок больше?
		\item Туристы прошли \( 18 \) км, что составило третью часть всего их пути. Какое расстояние должны пройти туристы? Во сколько раз расстояние, которое они прошли, меньше оставшегося пути? Сколько времени они затратят на оставшийся путь, если будут идти со скоростью \( 4 \) км/ч?
	\end{listofex}
\end{class}
%END_FOLD

%BEGIN_FOLD % ====>>_ Домашняя работа 3 _<<====
\begin{homework}[number=3]
	\begin{listofex}
		\item Домашняя работа
	\end{listofex}
\end{homework}
%END_FOLD

%BEGIN_FOLD % ====>>_____ Занятие 4 _____<<====
\begin{class}[number=4]
	\begin{listofex}
		\item Пусто
	\end{listofex}
\end{class}
%END_FOLD