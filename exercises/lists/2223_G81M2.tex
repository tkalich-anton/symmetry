%Группа 8-1 Модуль 2
\title{Занятие №1}
\begin{listofex}
	\item \exercise{2350}
	\item \exercise{2390}
	\item \exercise{2380}
	\item Докажите, что внешний угол треугольника равен сумме двух внутренних углов не смежных с ним.
	\item \exercise{2385}
	\item \exercise{2381}
	\item \exercise{2393}
	\item \exercise{2387}
	\item \exercise{2407}
	\item \exercise{2400}
\end{listofex}
\newpage
\title{Занятие №2}
\begin{listofex}
	\item Дан треугольник \( ABC \), причем \( AB = AC \) и \( \angle A = 110\degree \). Внутри треугольника взята точка \( M \) такая, что \( \angle MBC = 30\degree \), а \( \angle MCB = 25\degree \). Найдите \( \angle AMC \).
	\item Докажите, что если медиана равна половине стороны, к которой она проведена, то треугольник прямоугольный.
	\item \exercise{2415}
	\item Докажите, что если треугольник вписан в окружность и одна из его сторон является диаметром этой окружности, то такой треугольник является прямоугольным.
	\item Докажите обратное, что если треугольник прямоугольный и вписан в окружность, то гипотенуза будет являться диаметром окружности.
	\item \exercise{2455}
	\item \exercise{2418}
	\item \exercise{2424}
\end{listofex}
\newpage
\title{Домашняя работа №1}
\begin{listofex}
	\item Вычислить:
	\begin{enumcols}[itemcolumns=2]
		\item \( 3^7\cdot3^9:3^{14} \)
		\item \( \dfrac{10^8}{2^9\cdot2^8} \)
	\end{enumcols}
	\item Упростить выражение:
	\begin{enumcols}[itemcolumns=2]
		\item \( (3x-y)^2-3x(3x+2y^2) \)
		\item \( (2x+1)^3-(2x-1)^3 \)
	\end{enumcols}
	\item Докажите, что в равных треугольниках соответствующие биссектрисы равны.
	\item В равностороннем треугольнике \( ABC \) биссектрисы \( CN \) и \( AM \) пересекаются в точке \( P \). Найдите \( \angle MPN \).
	\item \exercise{2347}
	\item \exercise{2423}
	\item \exercise{2456}
\end{listofex}
%\newpage
%\title{Занятие №3}
%\begin{listofex}
%
%\end{listofex}
%\newpage
%\title{Занятие №4}
%\begin{listofex}
%
%\end{listofex}
%\newpage
%\title{Домашняя работа №2}
%\begin{listofex}
%	\item \exercise{2412}
%\end{listofex}
%\newpage
%\title{Занятие №5}
%\begin{listofex}
%\item \exercise{2422}
%\item \exercise{2420}
%\end{listofex}
%\newpage
%\title{Занятие №6}
%\begin{listofex}
%
%\end{listofex}
%\newpage
%\title{Занятие №7}
%\begin{listofex}
%
%\end{listofex}
%\newpage
%\title{Проверочная работа}
%\begin{listofex}
%
%\end{listofex}
\newpage
\title{Консультация}
\begin{listofex}
	\item Постройте следующие точки в декартовой системе координат:
	\begin{enumcols}[itemcolumns=3]
		\item \( A(3;1) \)
		\item \( B(-2;4) \)
		\item \( C(7;-7) \)
		\item \( D(-2;-5) \)
		\item \( E(0;4) \)
		\item \( F(-5;0) \)
	\end{enumcols}
	Какие из этих точек лежат на оси абсцисс? Какие на оси ординат? Определите для остальных точек четверть, в которой они лежат.
	\item Подберите вторую координату так, чтобы точка:
	\begin{enumcols}[itemcolumns=1]
		\item \( A(*;4) \) лежала в 1 четверти;
		\item \( B(-1;*) \) лежала в 3 четверти;
		\item \( A(10;*) \) лежала в 4 четверти;
		\item \( A(*;5) \) лежала в 2 четверти.
	\end{enumcols}
	\item Найдите координаты точки, которая симметрична точке \( A(2;4) \) относительно оси \( OX \).
	\item Найдите координаты точки, которая симметрична точке \( A(-5;-5) \) относительно оси \( OY \).
	\item Даны точки \( A(2;1) \) и \( B(-5;1) \). Найдите координаты таких двух точек \( C \) и \( D \), чтобы соединив их получился квадрат.
	\item Постройте графики линейных функций:
	\begin{enumcols}[itemcolumns=2]
		\item \( y=2x-1 \)
		\item \( y=\dfrac{1}{2}x+4 \)
		\item \( y=0,25x-3 \)
		\item \( y=0,5x+0,5 \)
	\end{enumcols}
	\item Найдите уравнение прямой, которая проходит через начало координат и точку \( A(4;5) \)
\end{listofex}