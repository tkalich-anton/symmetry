%
%===============>>  ГРУППА 11-5 МОДУЛЬ 4  <<=============
%
\setmodule{4}
%
%===============>>  Занятие 1  <<===============
%
\begin{class}[number=1]
	\begin{listofex}
		\item Упростить:
		\begin{enumcols}[itemcolumns=4]
			\item \( \sin(90\degree-x) \)
			\item \( \cos(360\degree + x) \)
			\item \( \tg(180\degree+x) \)
			\item \( \cos(810\degree-x) \)
		\end{enumcols}
		\item Вычислить:
		\begin{enumcols}[itemcolumns=2]
			\item \( \dfrac{\sqrt{3}}{\sin60\degree}+\dfrac{3}{\sin30\degree} \)
			\item \( \dfrac{17\sin155\degree}{\sin25\degree} \)
			\item \( \dfrac{-2\sin105\degree}{\cos15\degree} \)
			\item \( \sin^215\degree-1+\cos^215 \)
			\item \( -\sqrt{27}\cos30\degree-\sqrt{2}\sin45\degree\ctg60\degree\tg60\degree\)
			\item \( \dfrac{9\sin45\degree\cos45\degree}{\cos^245\degree-\sin^245\degree} \)
		\end{enumcols}
		\item Вычислить:
		\begin{enumcols}[itemcolumns=2]
			\item \( \sin240\degree\sin150\degree\sin(-90)\degree\tg30\degree \)
			\item \( \cos(-300\degree)\sin(-120\degree)\tg(-150\degree) \)
		\end{enumcols}
		\item Упростить:
		\begin{enumcols}[itemcolumns=4]
			\item \( \sin\left( \dfrac{\pi}{2}+x \right) \)
			\item \( \cos(\pi+x) \)
			\item \( \tg\left( \dfrac{3\pi}{2}-x \right) \)
			\item \( \sin(-3,5\pi-x) \)
		\end{enumcols}
		\item Вычислить:
		\begin{enumcols}[itemcolumns=2]
			\item \( \sin\dfrac{5\pi}{4}\cos\dfrac{4\pi}{3}\tg\dfrac{2\pi}{3}\ctg\dfrac{3\pi}{4} \)
			\item \( \cos\left( -\dfrac{5\pi}{3} \right)\sin\left( -\dfrac{5\pi}{2} \right)\sin\dfrac{3\pi}{2} \)
		\end{enumcols}
		\item Вычислить:
		\begin{enumcols}[itemcolumns=2]
			\item \( \sin\dfrac{\pi}{4}\cos\dfrac{\pi}{6}\tg\dfrac{\pi}{3} \)
			\item \( \cos\left( -\dfrac{\pi}{2} \right)+\sqrt{3}\sin\left( -\dfrac{\pi}{3} \right) \)
			\item \( \sin(-2\pi)+0,23\cos\left( \dfrac{3\pi}{2} \right) \)
			\item \( \sin\left( \dfrac{3\pi}{4} \right)+\cos\left( -\dfrac{5\pi}{6} \right) \)
			\item \( \ctg\left( \dfrac{3\pi}{2} \right)+\dfrac{1}{\sqrt{2}}\sin\left( \dfrac{5\pi}{4} \right) \)
			\item \( \sin(-2,5\pi)-(3\cos(-\pi))^2 \)
		\end{enumcols}
		\item Упростить выражение:
		\begin{enumcols}[itemcolumns=1]
			\item \( \ctg\left( \dfrac{3\pi}{2}+x \right)\ctg(\pi-x)-\ctg\left( \dfrac{\pi}{2}+x \right)\tg(2\pi+x) \)
			\item \( \cos\left( \dfrac{3\pi}{2}+x \right)\sin x + \sin^2(3\pi+x)+\tg(5\pi+x)\ctg x \)
			\item \( \dfrac{\sin x}{1+\cos x}+\ctg x \)
		\end{enumcols}
	\end{listofex}
\end{class}
%
%===============>>  Занятие 2  <<===============
%
%\begin{class}[number=2]
%	\begin{listofex}
%		\item Пусто
%	\end{listofex}
%\end{class}
%
%===============>>  Домашняя работа 1  <<===============
%
\begin{homework}[number=1]
	\begin{listofex}
		\item Вычислить:
		\begin{enumcols}[itemcolumns=2]
			\item \( \dfrac{\sqrt{3}}{\tg60\degree}+\dfrac{9}{\sin30\degree} \)
			\item \( \dfrac{19\sin214\degree}{\sin34\degree} \)
			\item \( \dfrac{-10\sin115\degree}{\cos25\degree} \)
			\item \( \sin^2126\degree-2+\cos^2126\degree \)
		\end{enumcols}
		\item Вычислить с помощью формул синуса/косинуса двойного угла:
		\begin{enumcols}[itemcolumns=2]
			\item \( 2\sin\dfrac{\pi}{8}\cos\dfrac{\pi}{8} \)
			\item \( \cos^215\degree-\sin^215\degree \)
			\item \( 10\sin75\degree\cos75\degree \)
			\item \( \sqrt{27}\cos^2\dfrac{13\pi}{2}-\sqrt{27}\sin^2\dfrac{13\pi}{12} \)
		\end{enumcols}
		\item Вычислите с помощью метода приведения:
		\begin{enumcols}[itemcolumns=6]
			\item \( \sin600\degree \)
			\item \( \tg480\degree \)
			\item \( \cos330\degree \)
			\item \( \sin240\degree \)
			\item \( \cos\dfrac{9\pi}{4} \)
			\item \( \sin\dfrac{7\pi}{4} \)
			\item \( \cos\dfrac{3\pi}{2} \)
			\item \( \tg\dfrac{7\pi}{6} \)
		\end{enumcols}
	\item Одного рулона обоев хватает для оклейки полосы от пола до потолка шириной \( 1,3 \) м. Сколько рулонов обоев нужно купить для оклейки прямоугольной комнаты размерами \(  2,1 \) м на \( 5,9 \) м?
	\item Найдите значение выражения: \( \left( -\mfrac{2}{3}{4}-\dfrac{3}{8} \right)\cdot160 \)
	\end{listofex}
\end{homework}
%
%===============>>  Занятие 3  <<===============
%
%\begin{class}[number=3]
%	\begin{listofex}
%		\item Пусто
%	\end{listofex}
%\end{class}
%
%===============>>  Занятие 4  <<===============
% смещение на одно занятие с прошлого месяца
%\begin{class}[number=4]
%	\begin{listofex}
%		\item Пусто
%	\end{listofex}
%\end{class}
%
%===============>>  Домашняя работа 2  <<===============
%
%\begin{homework}[number=2]
%	\begin{listofex}
%
%	\end{listofex}
%\end{homework}
%
%===============>>  Занятие 5  <<===============
% смещение на одно занятие с прошлого месяца
%\begin{class}[number=5]
%	\begin{listofex}
%		\item Пусто
%	\end{listofex}
%\end{class}
%
%===============>>  Домашняя работа 3  <<===============
%
%\begin{homework}[number=2]
%	\begin{listofex}
%
%	\end{listofex}
%\end{homework}
%\newpage
%\title{Подготовка к проверочной работе}
%\begin{listofex}
%	
%\end{listofex}
%
%===============>>  Занятие 7  <<===============
%
%\begin{class}[number=7]
%	\begin{listofex}
%	
%	\end{listofex}
%\end{class}
%
%===============>>  Провечная работа  <<===============
%
%\begin{exam}
%	\begin{listofex}
%	
%	\end{listofex}
%\end{exam}