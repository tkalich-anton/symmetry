%Группа 11-4 Модуль 1
\title{Занятие №1}
\begin{listofex}
	\item Вычислить:
	\begin{enumcols}[itemcolumns=2]
		\item \exercise{1740}
		\item \exercise{1846}
		\item \exercise{569}
		\item \exercise{596}
		\item \exercise{1582}
		\item \exercise{1576}
	\end{enumcols}
	\item Вычислить:
	\begin{enumcols}[itemcolumns=2]
		\item \exercise{1140}
		\item \exercise{1138}
		\item \exercise{1142}
		\item \exercise{2964}
		\item \exercise{2966}
		\item \exercise{1797}
	\end{enumcols}
	\item Решить уравнение:
	\begin{enumcols}[itemcolumns=2]
		\item \exercise{645}
		\item \exercise{646}
		\item \exercise{1037}
		\item \exercise{998}
		\item \exercise{1002}
	\end{enumcols}
	\item Решить уравнение:
	\begin{enumcols}[itemcolumns=2]
		\item \exercise{31}
		\item \exercise{1082}
		\item \exercise{1084}
		\item \( x^2-10x+25+|x^2-9x+20|=0 \)
	\end{enumcols}
	\item Найдите три последовательных натуральных числа, если удвоенный квадрат
	первого из них на \( 26 \) больше произведения второго и третьего чисел.
\end{listofex}
\newpage
\title{Занятие №2}
\begin{listofex}
	\item Вычислить:
	\begin{enumcols}[itemcolumns=2]
		\item \exercise{1741}
		\item \exercise{1469}
		\item \exercise{570}
		\item \exercise{1578}
		\item \exercise{594}
		\item \exercise{1574}
	\end{enumcols}
	\item Вычислить:
	\begin{enumcols}[itemcolumns=2]
		\item \exercise{1141}
		\item \exercise{1137}
		\item \exercise{2990}
		\item \exercise{2991}
	\end{enumcols}
	\item Найти значение выражения:
	\begin{enumcols}[itemcolumns=2]
		\item \exercise{1322}
		\item \exercise{1564}
		\item \exercise{1108}
		\item \exercise{1859}
	\end{enumcols}
	\item Решить уравнение:
	\begin{enumcols}[itemcolumns=2]
		\item \exercise{972}
		\item \exercise{975}
		\item \exercise{995}
		\item \exercise{997}
		\item \exercise{1038}
	\end{enumcols}
	\item Решить уравнение:
	\begin{enumcols}[itemcolumns=2]
		\item \exercise{1860}
		\item \exercise{1861}
	\end{enumcols}
	\item Найдите четыре последовательных нечетных натуральных числа, если удвоенное
	произведение второго и третьего чисел на \( 107 \) больше произведения первого и четвертого
	чисел.
	\answer{\( 7;\;9;\;11;\;13 \)}
\end{listofex}
\newpage
\title{Домашняя работа №1}
\begin{listofex}
	\item Вычислить:
	\begin{enumcols}[itemcolumns=1]
		\item \exercise{1743}
		\item \exercise{1554}
		\item \exercise{1653}
		\item \exercise{572}
		\item \exercise{1587}
	\end{enumcols}
	\item Вычислить:
	\begin{enumcols}[itemcolumns=2]
		\item \exercise{1139}
		\item \exercise{2980}
		\item \exercise{2992}
		\item \exercise{1793}
	\end{enumcols}
	\item Найти значение выражения:
	\begin{enumcols}[itemcolumns=1]
		\item \exercise{1858}
		\item \exercise{1227}
	\end{enumcols}
	\item Решить уравнение:
	\begin{enumcols}[itemcolumns=2]
		\item \exercise{974}
		\item \exercise{976}
		\item \exercise{996}
		\item \exercise{1003}
		\item \exercise{1040}
	\end{enumcols}
	\item Решить уравнение:
	\begin{enumcols}[itemcolumns=2]
		\item \exercise{1862}
		\item \exercise{1863}
	\end{enumcols}
	\item Найдите три последовательных натуральных числа, если удвоенный квадрат большего из них на \( 79 \) больше суммы квадратов двух других чисел.
	\answer{\( 12;\;13;\;14 \)}
\end{listofex}
\newpage
\title{Занятие №3}
\begin{listofex}
	\item Решить уравнение:
	\begin{enumcols}[itemcolumns=2]
		\item \exercise{1004}
		\item \exercise{1040}
		\item \exercise{1077}
		%\item \exercise{1078}
		\item \exercise{1075}
	\end{enumcols}
	\item Решить уравнение:
	\begin{enumcols}[itemcolumns=2]
		\item \exercise{1163}
		\item \exercise{1171}
		\item \exercise{1172}
		\item \exercise{1173}
		\item \exercise{3397}
	\end{enumcols}
	\item Решить уравнение:
	\begin{enumcols}[itemcolumns=2]
		\item \exercise{3499}
		\item \exercise{3500}
		\item \exercise{3502}
		\item \exercise{3424}
		\item \exercise{3423}
		\item \exercise{3431}
		\item \exercise{3435}
	\end{enumcols}
\end{listofex}
\newpage
\title{Занятие №4}
\begin{listofex}
	\item Решить уравнение:
	\begin{enumcols}[itemcolumns=2]
		\item \exercise{3744}
		\item \exercise{3390}
		\item \exercise{3727}
		%\item \exercise{1078}
		\item \exercise{3741}
	\end{enumcols}
	\item Решить уравнение:
	\begin{enumcols}[itemcolumns=2]
		\item \( \sqrt{\dfrac{4}{2x-11}}=\dfrac{1}{3} \)
		\item \exercise{3403}
		\item \exercise{3404}
		%\item \exercise{3404}
		\item \exercise{3406}
		%\item \exercise{1173} на след урок
		%\item \exercise{3397} на след урок
	\end{enumcols}
	\item Решить уравнение:
	\begin{enumcols}[itemcolumns=2]
		\item \exercise{3514}
		\item \exercise{3497}
		\item \exercise{3502}
		\item \exercise{3424}
		\item \exercise{3423}
		%\item \exercise{3431} на след урок
		%\item \exercise{3435} на след урок
	\end{enumcols}
	\item Решить уравнение:
	\begin{enumcols}[itemcolumns=2]
		\item \exercise{3529}
		\item \exercise{3438}
		\item \exercise{3490}
		\item \exercise{3495}
		%\item \exercise{3440}
		%\item \exercise{3442}
		%\item \exercise{3443}
	\end{enumcols}
	%\item Решить уравнение:
	%\begin{enumcols}[itemcolumns=1]
	%	\item \exercise{3445}
	%	\item \exercise{3507}
	%	\item \exercise{3510}
	%\end{enumcols}
\end{listofex}
\newpage
\title{Домашняя работа №2}
\begin{listofex}
	\item Вычислить:
	\begin{enumcols}[itemcolumns=2]
		\item \exercise{1639}
		\item \exercise{1666}
	\end{enumcols}
	\item Решить уравнение:
	\begin{enumcols}[itemcolumns=1]
		\item \exercise{3745}
		\item \exercise{3766}
		\item \exercise{3714}
		\item \exercise{3719}
	\end{enumcols}
	\item Решить уравнение:
	\begin{enumcols}[itemcolumns=2]
		\item \exercise{3407}
		\item \exercise{3404}
		\item \exercise{3408}
	\end{enumcols}
	\item Решить уравнение:
	\begin{enumcols}[itemcolumns=2]
		\item \exercise{3427}
		\item \exercise{3511}
		\item \exercise{3422}
		\item \exercise{3519}
	\end{enumcols}
	\item \exercise{2954}
	\item Вычислить:
	\begin{enumcols}[itemcolumns=1]
		\item \exercise{2981}
		\item \exercise{1119}
		\item \exercise{2920}
		\item \exercise{2924}
	\end{enumcols}
\end{listofex}
\newpage
\title{Занятие №5}
\begin{listofex}
	\item Решить уравнение:
	\begin{enumcols}[itemcolumns=2]
		\item \exercise{3438}
		\item \exercise{3490}
		\item \exercise{3495}
		\item \( \lg x - x + x\lg x - 1 = 0 \)
		\item \( \log^2_2x-2\log_2 x - 3 = 0 \)
		\item \( \log_4(2\cdot4^{x-2}-1)=2x-4 \)
		\item \exercise{3440}
		\item \exercise{3442}
		\item \exercise{3443}
	\end{enumcols}
	\item \exercise{1173}
	\item \exercise{3435}
	\item Из пункта \( A \) в пункт \( B \) одновременно выехали два автомобиля. Первый проехал с постоянной скоростью весь путь. Второй проехал первую половину пути со скоростью \( 24 \)км/ч, а вторую половину пути – со скоростью, на \(16\) км/ч больше скорости первого, в результате чего прибыл в пункт \( B \) одновременно с первым автомобилем. Найдите скорость первого автомобиля. Ответ дайте в км/ч.
	\item Товарный поезд каждую минуту проезжает на \( 750 \) метров меньше, чем скорый, и на путь в \( 180 \) км тратит времени на \( 2 \) часа больше, чем скорый. Найдите скорость товарного поезда. Ответ дайте в км/ч.
	\item По двум параллельным железнодорожным путям друг навстречу другу следуют скорый и пассажирский поезда, скорости которых равны соответственно \( 75  \) км/ч и \( 60  \) км/ч. Длина пассажирского поезда равна \( 400  \) метрам. Найдите длину скорого поезда, если время, за которое он прошел мимо пассажирского поезда, равно \( 16  \) секундам. Ответ дайте в метрах.
	\item Лодка в \( 5:00  \) вышла из пункта \( A \) в пункт \( B \), расположенный в \( 30  \) км от \( A \). Пробыв в пункте \( B \) \( 2 \) часа, лодка отправилась назад и вернулась в пункт \( A \) в \( 23:00 \) того же дня. Определите (в км/ч) собственную скорость лодки, если известно, что скорость течения реки \( 1 \) км/ч.
	\item Первая труба пропускает на \(1\) литр воды в минуту меньше, чем вторая. Сколько литров воды в минуту пропускает первая труба, если резервуар объемом \(110\) литров она заполняет на \( 2 \) минуты дольше, чем вторая заполняет резервуар объемом \(99\) литров?
	\item Игорь и Паша красят забор за \( 15 \) часов. Паша и Володя красят этот же забор за \( 21 \) час, а Володя и Игорь – за \( 35 \) часов. За сколько часов мальчики покрасят забор, работая втроем?
\end{listofex}
\newpage
\title{Занятие №6}
\begin{listofex}
	\item Решить уравнение:
	\begin{enumcols}[itemcolumns=2]
		\item \exercise{3637}
		\item \( \dfrac{\lg x}{\log_x 2}-\lg10 + \log_2 x - \lg x = 0 \)
		\item \exercise{3439}
		\item \( \log_x 2 - \log_4 x \dfrac{7}{6} = 0 \)
		\item \( \log^2_4 x - \log_4\sqrt{x} - 1,5 = 0 \)
	\end{enumcols}
	\item \exercise{1173}
	\item \exercise{3426}
	\item Два гонщика участвуют в гонках. Им предстоит проехать \(46\) кругов по кольцевой трассе протяжённостью \(4\) км. Оба гонщика стартовали одновременно, а на финиш первый пришёл раньше второго на \(5\)минут. Чему равнялась средняя скорость второго гонщика, если известно, что первый гонщик в первый раз обогнал второго на круг через \(60\) минут? Ответ дайте в км/ч.
	\item Первая труба пропускает на \( 4 \) литра воды в минуту меньше, чем вторая. Сколько литров воды в минуту пропускает вторая труба, если резервуар объемом \(621 \) литр она заполняет на \( 2 \) минуты дольше, чем первая труба?
	\item Один мастер может выполнить заказ за \( 12 \) часов, а другой – за \( 6 \) часов. За сколько часов выполнят заказ оба мастера, работая вместе?
	\item Первый садовый насос перекачивает \( 5 \) литров воды за \( 2 \) минуты, второй насос перекачивает тот же объём воды за \( 3 \) минуты. Сколько минут эти два насоса должны работать совместно, чтобы перекачать \( 25 \) литров воды?
\end{listofex}
\newpage
\title{Домашняя работа №2}
\begin{listofex}
	\item Решить уравнение:
	\begin{enumcols}[itemcolumns=2]
		\item \( \log_4(2x^2-9x+10)=0 \)
		\item \( x^2+\log_5 x + \log_5\dfrac{25}{x}=11 \)
		\item \( \log_{11}(x^2-5x+6)=\log_{11}(x-2) \)
		\item \( \log^2_7 x - \log_7 x^3 + 2 = 0 \)
	\end{enumcols}
	\item Из пункта \( A \) в пункт \( B \), расстояние между которыми \(75\), км одновременно выехали автомобилист и велосипедист. Известно, что за час автомобилист проезжает на \(40\) км больше, чем велосипедист. Определите скорость велосипедиста, если известно, что он прибыл в пункт \(B\) на \(6\) часов позже автомобилиста. Ответ дайте в км/ч.
	\item Два мотоциклиста стартуют одновременно в одном направлении из двух диаметрально противоположных точек круговой трассы, длина которой равна \( 14  \) км. Через сколько минут мотоциклисты поравняются в первый раз, если скорость одного из них на \( 21  \) км/ч больше скорости другого?
	\item Первая труба пропускает на \(1\) литр воды в минуту меньше, чем вторая. Сколько литров воды в минуту пропускает первая труба, если резервуар объемом \(420\) литров она заполняет на \( 2 \) минуты дольше, чем вторая заполняет резервуар объемом \(399\) литров?
	\item Первый насос наполняет бак за \( 20 \) минут, второй – за \( 30 \) минут, а третий – за \( 1 \) час. За сколько минут наполнят бак три насоса, работая одновременно?
\end{listofex}
\newpage
\title{Занятие №7}
\begin{listofex}
	\item На изготовление \(667\) деталей первый рабочий тратит на \( 6 \) часов меньше, чем второй рабочий на изготовление \(754\) таких же деталей. Известно, что первый рабочий за час делает на \(3\) детали больше, чем второй. Сколько деталей в час делает первый рабочий?
	\item Двое рабочих, работая вместе, могут выполнить работу за \( 15 \) дней. За сколько дней, работая отдельно, выполнит эту работу первый  рабочий, если он за \(  2 \) дня выполняет такую же часть работы, какую второй – за \( 3 \) дня?
	\item Каждый из двух рабочих одинаковой квалификации может выполнить заказ за \( 15 \) часов. Через \( 3 \) часа после того, как один из них приступил к выполнению заказа, к нему присоединился второй рабочий, и работу над заказом они довели до конца уже вместе. Сколько часов потребовалось на выполнение всего заказа?
	\item Первый насос наполняет бак за \( 18 \) минут, второй – за \( 24 \) минуты, а третий – за \( 36 \) минут. За сколько минут наполнят бак три насоса, работая одновременно?
	\item  Плиточник планирует уложить \( 324 \) м\( ^2\) плитки. Если он будет укладывать на \( 6 \) м\( ^2 \) в день больше, чем запланировал, то закончит работу на \( 9 \) дней раньше. Сколько квадратных метров плитки в день планирует укладывать плиточник?
	\item Дима, Артем, Никита и Денис учредили компанию с уставным капиталом \( 100000 \) рублей. Дима внес \( 20\% \) уставного капитала, Артем –  \( 50000 \) рублей, Никита – \( 0,26 \) уставного капитала, а оставшуюся часть капитала внес Денис. Учредители договорились делить ежегодную прибыль пропорционально внесенному в уставной капитал вкладу. Какая сумма от прибыли \( 700000 \) рублей причитается Денису? Ответ дайте в рублях.
	\item Имеется два сплава. Первый содержит \( 15\% \) никеля, второй – \( 35\% \) никеля. Из этих двух сплавов получили третий сплав массой \( 140 \) кг, содержащий \( 30\%  \) никеля. На сколько килограммов масса первого сплава была меньше массы второго?
	\item Изюм получается в процессе сушки винограда. Сколько килограммов винограда потребуется для получения \( 56 \) килограммов изюма, если виноград содержит \( 90\% \) воды, а изюм содержит \( 5\% \) воды?
	\item По двум параллельным железнодорожным путям друг навстречу другу следуют скорый и пассажирский поезда, скорости которых равны соответственно \( 75  \) км/ч и \( 60  \) км/ч. Длина пассажирского поезда равна \( 400  \) метрам. Найдите длину скорого поезда, если время, за которое он прошел мимо пассажирского поезда, равно \( 16  \) секундам. Ответ дайте в метрах.
\end{listofex}
\newpage
\title{Проверочная работа}
\begin{listofex}
	\item Вычислить:
	\begin{enumcols}[itemcolumns=2]
		\item \exercise{1775}
		\item \exercise{1765}
		\item \exercise{1741}
		\item \exercise{1558}
		\item \exercise{1469}
		\item \exercise{570}
		\item \exercise{1578}
		\item \exercise{594}
	\end{enumcols}
	\item Решить уравнение:
	\begin{enumcols}[itemcolumns=2]
		\item \exercise{3745}
		\item \exercise{3714}
		\item \exercise{3744}
		\item \( \sqrt{\dfrac{4}{2x-11}}=\dfrac{1}{3} \)
	\end{enumcols}
	\item Решить уравнение:
	\begin{enumcols}[itemcolumns=3]
		\item \exercise{3514}
		\item \exercise{3424}
		\item \exercise{3423}
	\end{enumcols}
	\item Решить уравнение:
	\begin{enumcols}[itemcolumns=2]
		\item \exercise{3438}
		\item \exercise{3490}
		\item \exercise{3495}
		\item \exercise{3529}
		\item \( \lg x - x + x\lg x - 1 = 0 \)
		\item \( \log^2_2x-2\log_2 x - 3 = 0 \)
	\end{enumcols}
	\item Двое рабочих, работая вместе, могут выполнить работу за \( 15 \) дней. За сколько дней, работая отдельно, выполнит эту работу первый  рабочий, если он за \(  2 \) дня выполняет такую же часть работы, какую второй – за \( 3 \) дня?
	\item Имеется два сплава. Первый содержит \( 15\% \) никеля, второй – \( 35\% \) никеля. Из этих двух сплавов получили третий сплав массой \( 140 \) кг, содержащий \( 30\%  \) никеля. На сколько килограммов масса первого сплава была меньше массы второго?
	\item Два гонщика участвуют в гонках. Им предстоит проехать \(46\) кругов по кольцевой трассе протяженностью \(4\) км. Оба гонщика стартовали одновременно, а на финиш первый пришел раньше второго на \(5\)минут. Чему равнялась средняя скорость второго гонщика, если известно, что первый гонщик в первый раз обогнал второго на круг через \(60\) минут? Ответ дайте в км/ч.
\end{listofex}