%
%===============>>  ГРУППА 11-4 МОДУЛЬ 8  <<=============
%
\setmodule{8}

%BEGIN_FOLD % ====>>_____ Занятие 1 _____<<====
\begin{class}[number=1]
	\begin{listofex}
		\item Занятие 1
	\end{listofex}
\end{class}
%END_FOLD

%BEGIN_FOLD % ====>>_____ Занятие 2 _____<<====
\begin{class}[number=2]
	\begin{listofex}
		\item Найдите все значения \(a\), при каждом из которых система \[ \begin{cases} yx^2+y^2=2y+63-7x^2, \\ x \ge -3, \\ x+y=a  \end{cases} \] имеет ровно два различных решения.
		\item Найдите все значения \(a\), при каждом из которых система \[ \begin{cases} x^2-2x+|y|-15=0 \\ x^2+(y-a)(y+a)=2 (x-0,5)  \end{cases} \] имеет ровно шесть решений.
		\item Найдите все значения \(a\), при каждом из которых система уравнений \[ \begin{cases} |x-4|+3|y|=2 \\ 9y^2+x^2-8x+4(a+3)=0 \end{cases} \] имеет ровно четыре решения.
	\end{listofex}
\end{class}
%END_FOLD

%BEGIN_FOLD % ====>>_ Домашняя работа 1 _<<====
\begin{homework}[number=1]
	\begin{listofex}
		\item Имеются два сосуда. Первый содержит \( 30 \) кг, а второй --- \( 20 \) кг раствора кислоты различной концентрации. Если эти растворы смешать, то получится раствор, содержащий \( 68\% \) кислоты. Если же смешать равные массы этих растворов, то получится раствор, содержащий \( 70\% \) кислоты. Сколько килограммов кислоты содержится в первом сосуде?
		\item Найдите наибольшее значение функции \( y=x+\dfrac{9}{x} \) на отрезке \( [-4; -1] \)
		\item 
		\begin{tasks}(1)
			\task Решите уравнение: \( \log_4\left( 2^{2x}-\sqrt{3}\cos x-\sin2x \right)=x \).
			\task Найжите все корни этого уравнения, принадлежащие отрезку \( \left[ 2\pi; \dfrac{7\pi}{2} \right]  \).
		\end{tasks}
		\item Решите неравенство \( \dfrac{0,2^{|x^2-4x+2|}-0,04}{3-x}\le0 \).
		\item Найдите все значения параметра \( a \), при каждом из которых система уравнений
		\[ \begin{cases}
			3|x-2|+|y|-3=0,\\
			ax-y+2a+2=0
		\end{cases} \]
		имеет ровно \( 2 \) решения.
		\item Найдите все значения параметра \( a \), при каждом из которых система уравнений
		\[ \begin{cases}
			\dfrac{xy^2-2xy-4y+8}{\sqrt{x+4}}=0,\\
			y=ax
		\end{cases} \]
		имеет ровно \( 2 \) различных решения.
		\item Найдите все значения параметра \( a \), при каждом из которых система уравнений
		\[ \begin{cases}
			x^2+12x+|y|+27=0,\\
			x^2+(y-a)(y+a)=-12(x+3)
		\end{cases} \]
		имеет ровно \( 4 \) решения.
	\end{listofex}
\end{homework}
%END_FOLD

%BEGIN_FOLD % ====>>_____ Занятие 3 _____<<====
\begin{class}[number=3]
	\begin{listofex}
		\item Занятие 3 
	\end{listofex}
\end{class}
%END_FOLD

%BEGIN_FOLD % ====>>_____ Занятие 4 _____<<====
\begin{class}[number=4]
	\begin{listofex}
		\item Занятие 4
	\end{listofex}
\end{class}
%END_FOLD

%BEGIN_FOLD % ====>>_ Домашняя работа 2 _<<====
\begin{homework}[number=2]
	\begin{listofex}
		\item Домашняя работа 2
	\end{listofex}
\end{homework}
%END_FOLD

%BEGIN_FOLD % ====>>_____ Занятие 5 _____<<====
\begin{class}[number=5]
	\begin{listofex}
		\item Занятие 5
	\end{listofex}
\end{class}
%END_FOLD

%BEGIN_FOLD % ====>>_____ Занятие 6 _____<<====
\begin{class}[number=6]
	\begin{listofex}
		\item Занятие 6
	\end{listofex}
\end{class}
%END_FOLD

%BEGIN_FOLD % ====>>_ Домашняя работа 3 _<<====
\begin{homework}[number=3]
	\begin{listofex}
		\item Домашняя работа 3
	\end{listofex}
\end{homework}
%END_FOLD

%BEGIN_FOLD % ====>>_____ Занятие 7 _____<<====
\begin{class}[number=7]
	\title{Подготовка к проверочной}
	\begin{listofex}
		\item Занятие 7
	\end{listofex}
\end{class}
%END_FOLD

%BEGIN_FOLD % ====>>_ Проверочная работа _<<====
\begin{exam}
	\begin{listofex}
		\item Проверочная
	\end{listofex}
\end{exam}
%END_FOLD

%BEGIN_FOLD % ====>>_ Консультация_<<====
\begin{consultation}
	\begin{listofex}
		\item В июле \( 2016 \) года планируется взять кредит в размере \( 4,2 \) млн. руб. Условия возврата таковы: 
		\begin{tasks}(1)
			\task каждый январь долг возрастает на \( r\% \) по сравнению с концом предыдущего года.
			\task с февраля по июнь необходимо выплатить часть долга.
			\task в июле \( 2017 \), \( 2018 \) и \( 2019 \) годов долг остается равным \( 4,2 \) млн. руб. 
			\task суммы выплат \( 2020 \) и \( 2021 \) годов равны.
		\end{tasks}		
		Найдите \( r \), если в \( 2021 \) году долг будет выплачен полностью и общие выплаты составят \( 6,1 \)  млн. рублей.
		\item Георгий взял кредит в банке на сумму \( 804 000 \) рублей. Схема выплата кредита такова: в конце каждого года банк увеличивает на \( 10 \) процентов оставшуюся сумму долга, а затем Георгий переводит в банк свой очередной платеж. Известно, что Георгий погасил кредит за три года, причем каждый его следующий платеж был ровно вдвое меньше предыдущего. Какую сумму Георгий заплатил в третий раз? Ответ дайте в рублях.
		\item В июле \( 2020 \) года планируется взять кредит в банке на сумму \( 147 000 \) рублей. Условия его возврата таковы:
		\begin{tasks}(1)
			\task каждый январь долг увеличивается на \( 10\% \) по сравнению с концом предыдущего года;
			\task с февраля по июнь каждого года необходимо выплатить одним платежом часть долга.
		\end{tasks}	
		Сколько рублей будет выплачено банку, если известно, что кредит будет полностью погашен двумя равными платежами, то есть за два года.
		\item Андрей Петрович взял кредит на несколько лет и выплатил его равными ежегодными платежами по \( 200 000 \) руб. При этом в начале каждого года сумма кредита увеличивалась на \( 10\% \), а в конце года производился платёж. Если бы Андрей Петрович не делал платежей, то за это время вследствие начисления процентов сумма кредита составила бы \( 928 200 \) руб. На сколько лет был взят кредит?
	\end{listofex}
\end{consultation}
%END_FOLD