%
%===============>>  ГРУППА 11-4 МОДУЛЬ 8  <<=============
%
\setmodule{8}

%BEGIN_FOLD % ====>>_____ Занятие 1 _____<<====
\begin{class}[number=1]
	\begin{listofex}
		\item Постройте график уравнения:
		\begin{tasks}(2)
			\task \( xy-2=2x-y \)
			\task \( y\sqrt{x}-1=y-\sqrt{x} \)
			\task \( (x^2+4)(y^2+1)=8xy \)
			\task \( (y-2)^2=(x+1)^2 \)
			\task \( |y|=2-x \)
			\task \( |y|=9-x^2 \)
			\task \( x|y|=-2 \)
			\task \( |y|(x+1)=1 \)
			\task \( |y|=2|x|-x^2 \)
			\task \( |y|=x^2-4|x|+3 \)
			\task \( |y|=|x^2-4x+3| \)
			\task \( x^2-6x+9=y^4 \)
		\end{tasks}
		\item Изобразите на координатной плоскости множество точек, координаты которых \( (x; y) \) удовлетворяют неравенству:
		\begin{tasks}(2)
			\task \( (x-1)(y+2)\ge0 \)
			\task \( (x-1)(|y|+2)\le0 \)
			\task \( x^2\ge y^2 \)
			\task \( y>3|x|-2 \)
			\task \( y\le|2-\sqrt{x+1}| \)
			\task \( |x|+|y|\le3 \)
		\end{tasks}
%		\item Найдите все значения параметра \( a \), при каждом из которых система уравнений
%		\[ \left\{
%		\begin{array}{l}
%			x^2-2x+|y|-15=0,\\
%			x^2+(y-a)(y+a)=2\left( x-\dfrac{1}{2} \right)
%		\end{array}
%		\right. \]
%		имеет ровно 6 решений.
		\item Найдите все значения параметра \( a \), при каждом из которых система уравнений
		\[ \left\{
		\begin{array}{l}
			(y^2-xy+x-3y+2)\sqrt{x+3}=0,\\
			a-x-y=0
		\end{array}
		\right. \]
		имеет ровно два различных решения.
		\item Найдите все значения параметра \( a \), при каждом из которых система уравнений
		\[ \left\{
		\begin{array}{l}
			4|y-3|=12-3|x|,\\
			y^2-a^2=3(2y-3)-x^2
		\end{array}
		\right. \]
		имеет ровно четыре решения.
		\item Для каждого значения параметра \( a \) решить уравнение:
		\[ |x+a|+|x-a|=2 \]
	\end{listofex}
\end{class}
%END_FOLD

%BEGIN_FOLD % ====>>_____ Занятие 2 _____<<====
\begin{class}[number=2]
	\begin{listofex}
		\item Найдите все значения \(a\), при каждом из которых система \[ \begin{cases} yx^2+y^2=2y+63-7x^2, \\ x \ge -3, \\ x+y=a  \end{cases} \] имеет ровно два различных решения.
		\item Найдите все значения \(a\), при каждом из которых система \[ \begin{cases} x^2-2x+|y|-15=0 \\ x^2+(y-a)(y+a)=2 (x-0,5)  \end{cases} \] имеет ровно шесть решений.
		\item Найдите все значения \(a\), при каждом из которых система уравнений \[ \begin{cases} |x-4|+3|y|=2 \\ 9y^2+x^2-8x+4(a+3)=0 \end{cases} \] имеет ровно четыре решения.
	\end{listofex}
\end{class}
%END_FOLD

%BEGIN_FOLD % ====>>_ Домашняя работа 1 _<<====
\begin{homework}[number=1]
	\begin{listofex}
		\item Имеются два сосуда. Первый содержит \( 30 \) кг, а второй --- \( 20 \) кг раствора кислоты различной концентрации. Если эти растворы смешать, то получится раствор, содержащий \( 68\% \) кислоты. Если же смешать равные массы этих растворов, то получится раствор, содержащий \( 70\% \) кислоты. Сколько килограммов кислоты содержится в первом сосуде?
		\item Найдите наибольшее значение функции \( y=x+\dfrac{9}{x} \) на отрезке \( [-4; -1] \)
		\item 
		\begin{tasks}(1)
			\task Решите уравнение: \( \log_4\left( 2^{2x}-\sqrt{3}\cos x-\sin2x \right)=x \).
			\task Найжите все корни этого уравнения, принадлежащие отрезку \( \left[ 2\pi; \dfrac{7\pi}{2} \right]  \).
		\end{tasks}
		\item Решите неравенство \( \dfrac{0,2^{|x^2-4x+2|}-0,04}{3-x}\le0 \).
		\item Найдите все значения параметра \( a \), при каждом из которых система уравнений
		\[ \begin{cases}
			3|x-2|+|y|-3=0,\\
			ax-y+2a+2=0
		\end{cases} \]
		имеет ровно \( 2 \) решения.
		\item Найдите все значения параметра \( a \), при каждом из которых система уравнений
		\[ \begin{cases}
			\dfrac{xy^2-2xy-4y+8}{\sqrt{x+4}}=0,\\
			y=ax
		\end{cases} \]
		имеет ровно \( 2 \) различных решения.
		\item Найдите все значения параметра \( a \), при каждом из которых система уравнений
		\[ \begin{cases}
			x^2+12x+|y|+27=0,\\
			x^2+(y-a)(y+a)=-12(x+3)
		\end{cases} \]
		имеет ровно \( 4 \) решения.
	\end{listofex}
\end{homework}
%END_FOLD

%BEGIN_FOLD % ====>>_____ Занятие 3 _____<<====
\begin{class}[number=3]
	\begin{listofex}
		\item
		\begin{minipage}[t]{\bodywidth}
			Биссектриса тупого угла параллелограмма делит противоположную сторону в отношении \( 4 : 3 \), считая от вершины острого угла. Найдите большую сторону параллелограмма, если его периметр равен \( 88 \).
			\foranswer
		\end{minipage}
		\gapwidth
		\begin{minipage}[t]{\picwidth}
			\includegraphics[align=t, width=\linewidth]{\picpath/prob_3_1}
		\end{minipage}
		\item
		\begin{minipage}[t]{\bodywidth}
			Площадь боковой поверхности конуса в два раза больше площади основания. Найдите угол между образующей конуса и плоскостью основания. Ответ дайте в градусах.
			\foranswer
		\end{minipage}
		\gapwidth
		\begin{minipage}[t]{\picwidth}
			\includegraphics[align=t, width=\linewidth]{\picpath/prob_3_2}
		\end{minipage}
		\item У Дины в копилке лежит \( 7 \) рублёвых, \( 5 \) двухрублёвых, \( 6 \) пятирублёвых и \( 2 \) десятирублёвых
		монеты. Дина наугад достаёт из копилки одну монету. Найдите вероятность того, что оставшаяся в
		копилке сумма составит менее \( 60 \) рублей.
		\foranswer
		\item Чтобы пройти в следующий круг соревнований, футбольной команде нужно набрать хотя бы \( 4 \)
		очка в двух играх. Если команда выигрывает, она получает \( 3 \) очка, в случае ничьей --- \( 1 \) очко, если
		проигрывает --- \( 0 \) очков.
		Найдите вероятность того, что команде удастся выйти в следующий круг
		соревнований.
		Считайте, что в каждой игре вероятности выигрыша и проигрыша одинаковы и равны \( 0,4 \).
		\foranswer
		\item Решите уравнение \( \sin \dfrac{\pi x}{3}=0,5 \). В ответе напишите наименьший положительный корень.
		\foranswer
		\item Найдите значение выражения \( 0,8^{1/7}\cdot5^{2/7}\cdot20^{6/7} \).
		\item
		\begin{minipage}[t]{\bodywidth}
			На рисунке изображены график функции \( y=f(x) \) и касательная к этому графику, проведенная в точке \( x_0 \). Уравнение касательной показано на рисунке. Найдите значение функции \( g(x)=(f`(x)-0,5)\cdot6 \) в точке \( x_0 \).
			\foranswer
		\end{minipage}
		\gapwidth
		\begin{minipage}[t]{\picwidth}
			\includegraphics[align=t, width=\linewidth]{\picpath/prob_3_3}
		\end{minipage}
	\item Автомобиль, масса которого равна \( m=2160 \) кг, начинает двигаться с ускорением, которое в
	течение \( t \) секунд остаeтся неизменным, и проходит за это время путь \( S=500 \) метров. Значение силы
	(в ньютонах), приложенной в это время к автомобилю, равно \( F=\dfrac{2mS}{t^2} \). Определите наибольшее
	время после начала движения автомобиля, за которое он пройдeт указанный путь, если известно, что
	сила \( F \), приложенная к автомобилю, не меньше \( 2400 \) H. Ответ выразите в секундах.
	\foranswer
	\item Петя и Ваня выполняют одинаковый тест. Петя отвечает за час на 8 вопросов теста, а Ваня —
	на 9. Они одновременно начали отвечать на вопросы теста, и Петя закончил свой тест позже Вани на
	20 минут. Сколько вопросов содержит тест?
	\foranswer
	\item
	\begin{minipage}[t]{\bodywidth}
		На рисунке изображён график функции вида \( f(x)=\dfrac{a}{x+b}+c \), где числа \( a \), \( b \), \( c \) -- целые числа. Найдите \( f\left( \dfrac{4}{3} \right) \).
		\foranswer
	\end{minipage}
	\gapwidth
	\begin{minipage}[t]{\picwidth}
		\includegraphics[align=t, width=\linewidth]{\picpath/prob_3_4}
	\end{minipage}
	\item Найдите точку минимума функции \( y=2x-\ln(x+3)+7 \).
	\foranswer
	\item Найдите все значения \( a \), для каждого из которых уравнение \[ \log_{1-x}(a-x+2)=2 \] имеет хотя бы один корень, принадлежащий промежутку \( [-1;1) \).
	\end{listofex}
	\newpage
	\begin{listofex}
	\item Найдите все значения параметра \( a \), при каждом из которых уравнение
	\[ x^2-x-a=0 \]
	имеет хотя бы одно решение, удовлетворяющее неравенству \( x>\dfrac{1}{2} \).
	\item При каких значениях параметра \( a \) все решения уравнения
	\[ 3|x+2a|-3a+x-15=0 \]
	удовлетворяют неравенству \( 4\le x \le 6 \)?
	\item Найдите все значения параметра a, при каждом из которых на интервале \( (1;2) \) существует хотя бы одно число \( x \), неудовлетворяющее неравенству
	\[ a + \sqrt{a^2-2ax+x^2}\le3x-x^2. \]
	\item Найдите все значения \( а \), при каждом из которых система неравенств
	\[ \left\{
	\begin{array}{l}
		ax\ge2,\\
		\sqrt{x-1}>a,\\
		3x\le2a+11
	\end{array}
	\right. \]
	имеет хотя бы одно решение на отрезке \( [3;4] \).
	\item Найдите все положительные значения \( a \), при каждом из которых множеством решений неравенства
	\[ \dfrac{x-2}{ax^2-(a^2+1)x+a}\ge0 \]
	является некоторый луч.
	\end{listofex}
\end{class}
%END_FOLD

%BEGIN_FOLD % ====>>_____ Занятие 4 _____<<====
\begin{class}[number=4]
	\begin{listofex}
		\item Найдите все значения \( a \), при каждом из которых система неравенств
		\[ \left\{
		\begin{array}{l}
			(a+7x+4)(a-2x+4)\le0,\\
			a+3x\ge x^2
		\end{array}
		\right. \]
		имеет хотя бы одно решение.
		\item Найдите все значения \( a \), при каждом из которых неравенство
		\[ \left|\dfrac{x^2+x-2a}{x+a}-1\right|\le2 \]
		не имеет решений на интервале \( (1;2) \).
		\item Решите неравенство: \( \dfrac{9^x-3^x-90}{3^x-82}\le1 \).
		\item Решите неравенство: \( \dfrac{3^{|x^2-2x-1|}}{x}\ge0 \).
	\end{listofex}
\end{class}
%END_FOLD

%BEGIN_FOLD % ====>>_ Домашняя работа 2 _<<====
\begin{homework}[number=2]
	\begin{listofex}
		\item Найдите наибольшее значение функции \( y=\ln(x+5)^5-5x \) на отрезке \( [-4,5;0] \).
		\item
		а) Решите уравнение \( |2\tg x - 5| - |2\tg x - 1|=2 \)\\
		б) Укажите корни этого уравнения, принадлежащие отрезку \( \left[ -\dfrac{\pi}{2};\dfrac{\pi}{2} \right] \).
		\item Решите неравенство \( \sqrt{3\cdot4^x-5\cdot2^{x+1}+3}\ge2^x-3 \).
		\item Найдите все значения параметра \( a \), при каждом из которых система неравенств
		\[ \left\{
		\begin{array}{l}
			a+3x\le12,\\
			a+4x\ge x^2,\\
			a\le x
		\end{array}
		\right. \]
		имеет хотя бы одно решение.
		\item Найдите все значения параметра a, при каждом из которых хотя бы одно решение неравенства
		\( x^2+a+|x-a-1|+1\le3x \) принадлежит отрезку \( [0;1] \).
		\item Найдите все решения параметра \( a \), при которых уравнение
		\[ x^2-4x-2|x-a|+2+a=0 \]
		имеет ровно два решения.
		\item Найдите все значения параметра \( a \), при каждом из которых неравенство
		\[ \dfrac{x-3a-1}{x+2a-2}\le0 \]
		выполняется для всех \( x \) из промежутка \( 2\le x\le3 \).
	\end{listofex}
\end{homework}
%END_FOLD

%BEGIN_FOLD % ====>>_____ Занятие 5 _____<<====
\begin{class}[number=5]
	\begin{listofex}
		\item Найдите все значения параметра \( a \), при каждом из которых уравнение
		\[ |x^2+a^2-5x-4a|=x+a \]
		имеет \( 4 \) решения.
		\rightlabel{Реальный ЕГЭ 02.06.2022}
		\item Найдите все значения параметра \( a \), при каждом из которых уравнение
		\[ |x^2-a^2|=|x+a|\cdot\sqrt{x^2-ax+4a} \]
		имеет два различных корня.
		\rightlabel{Реальный ЕГЭ 07.06.2021}
		\item Найдите все значения параметра \( a \), при каждом из которых уравнение
		\[ \dfrac{|4x|-x-3-a}{x^2-x-a}=0 \]
		имеет ровно два различных корня.
		\rightlabel{Реальный ЕГЭ 2019 г.}
		\item Найдите все значения параметра \( a \), при каждом из которых система уравнений
		\[ \left\{
		\begin{array}{l}
			x^2+y^2-4(a+1)x-2ay+5a^2+8a+3=0,\\
			y^2=x^2
		\end{array}
		\right. \]
		имеет ровно четыре различных решения.
		\rightlabel{Реальный ЕГЭ 2018 г.}
		\item Найдите все значения параметра \( a \), при каждом из которых уравнение
		\[\ln(3a-x)\ln(2x+2a-5)=\ln(3a-x)\ln(x-a) \]
		имеет ровно один корень на отрезке \( [0;2] \).
		\rightlabel{Реальный ЕГЭ 2017 г.}
		\item Найдите все значения параметра \( a \), при каждом из которых система уравнений
		\[ \left\{
		\begin{array}{l}
			(x-3)(y+3x-9)=|x-3|^3,\\
			y=x+a
		\end{array}
		\right. \]
		имеет ровно четыре различных решения.
		\rightlabel{Реальный ЕГЭ 2016 г.}
		\item Найдите все значения параметра \( a \), при каждом из которых система уравнений
		\[ \left\{
		\begin{array}{l}
			y^2+x-2=|x^2+x-2|,\\
			x-y=a
		\end{array}
		\right. \]
		имеет более двух решений.
		\rightlabel{Реальный ЕГЭ 2015 г.}
		\item Найдите все значения параметра \( a \), при каждом из которых система уравнений
		\[ \left\{
		\begin{array}{l}
			\sqrt{16-y^2}=\sqrt{16-a^2x^2},\\
			x^2+y^2=8x+4y
		\end{array}
		\right. \]
		имеет ровно два различных корня.
		\rightlabel{Реальный ЕГЭ 10.07.2020}
	\end{listofex}
\end{class}
%END_FOLD

%BEGIN_FOLD % ====>>_____ Занятие 6 _____<<====
\begin{class}[number=6]
	\begin{listofex}
		\item
		а) Решите уравнения \( \cos2x+\sqrt{2}\cos\left( \dfrac{\pi}{2}-x \right)-1=0 \);\\
		б) Укажите корни этого уравнения, принадлежащие отрезку \( \left[ \dfrac{5\pi}{2};4\pi \right] \).
		\item Решить неравенство:
		\[ x^2\cdot\log_{625}(3-x)\le\log_5(x^2-6x+9) \]
		\item Найдите все значения параметра \( a \), при каждом из которых система уравнений
		\[ \left\{
		\begin{array}{l}
			\log_3(a-x^2)=\log_3(a-y^2),\\
			x^2+y^2=4x+6y
		\end{array}
		\right. \]
		имеет ровно два различных решения.
		\item Найдите все значения параметра \( a \), при каждом из которых система уравнений
		\[ \left\{
		\begin{array}{l}
			(x-3)(y+3x-9)=|x-3|^3,\\
			y=x+a
		\end{array}
		\right. \]
		имеет ровно четыре различных решения.
		\rightlabel{Реальный ЕГЭ 2016 г.}
		\item 15-го января планируется взять кредит в банке на 19 месяцев. Условия его возврата таковы:\\
		--- 1-го числа каждого месяца долг возрастёт на \( r\% \) по сравнению с концом предыдущего месяца;\\
		--- со 2-го по 14-е число каждого месяца необходимо выплатить часть долга;\\
		--- 15-го числа каждого месяца долг должен быть на одну и ту же сумму меньше долга на 15-е число предыдущего месяца.\\
		Известно, что общая сумма выплат после полного погашения кредита на 30\% больше суммы, взятой в кредит. Найдите \( r \).
		\item 31 декабря 2013 года Сергей взял в банке \( 9\;930\;000 \) рублей в кредит под \( 10\% \) годовых. Схема выплаты кредита следующая: 31 декабря каждого следующего года банк начисляет проценты на оставшуюся сумму долга (то есть увеличивает долг на \( 10\% \)), затем Сергей переводит в банк определённую сумму ежегодного платежа. Какой должна быть сумма ежегодного платежа, чтобы Сергей выплатил долг тремя равными ежегодными платежами?
	\end{listofex}
\end{class}
%END_FOLD

%BEGIN_FOLD % ====>>_ Домашняя работа 3 _<<====
\begin{homework}[number=3]
	\begin{listofex}
		\item
		а) Решите уравнения
		\( 2\sin^2(x+\pi)-\cos\left( \dfrac{\pi}{2}-x \right)=0 \);\\
		б) Укажите корни этого уравнения, принадлежащие отрезку
		\( \left[ -\dfrac{5\pi}{2};-\pi \right] \).
		\item Решить неравенство:
		\[ x^2\cdot\log_{243}(x+6)\le\log_3(x^2+12x+36) \]
		\item Решить неравенство:
		\[ \dfrac{9^x-3^{x+1}-19}{3^x-6}+\dfrac{9^{x+1}-3^{x+4}+2}{3^x-9}\le10\cdot3^x+3 \]
		\item Найдите все значения параметра \( a \), при каждом из которых система уравнений
		\[ \left\{
		\begin{array}{l}
			\sqrt{36-y^2}=\sqrt{36-a^2x^2},\\
			x^2+y^2=6x+8y
		\end{array}
		\right. \]
		имеет ровно два различных решения.
		\item 15‐го января планируется взять кредит в банке на 14 месяцев. Условия его возврата таковы:\\
		— 1-го числа каждого месяца долг возрастает на \( r\% \) по сравнению с концом предыдущего месяца;\\
		— со 2-го по 14-е число каждого месяца необходимо выплатить часть долга;\\
		— 15-го числа каждого месяца долг должен быть на одну и ту же сумму меньше долга на 15 число предыдущего месяца.\\
		Известно, что общая сумма выплат после полного погашения кредита на \( 15\% \) больше суммы, взятой в кредит. Найдите \( r \).
		\item 31 декабря 2014 года Алексей взял в банке \( 6 902 000 \) рублей в кредит под \( 12,5\% \) годовых. Схема выплаты кредита следующая  — 31 декабря каждого следующего года банк начисляет проценты на оставшуюся сумму долга (то есть увеличивает долг на \( 12,5\% \)), затем Алексей переводит в банк \( X \) рублей. Какой должна быть сумма \( X \), чтобы Алексей выплатил долг четырьмя равными платежами (то есть за четыре года)?
	\end{listofex}
\end{homework}
%END_FOLD

%BEGIN_FOLD % ====>>_____ Занятие 7 _____<<====
\begin{class}[number=7]
	\begin{listofex}
		\item
		а) Решите уравнения \( \cos2x+\sqrt{2}\cos\left( \dfrac{\pi}{2}-x \right)-1=0 \);\\
		б) Укажите корни этого уравнения, принадлежащие отрезку \( \left[ \dfrac{5\pi}{2};4\pi \right] \).
		\item Решить неравенство:
		\[ x^2\cdot\log_{625}(3-x)\le\log_5(x^2-6x+9) \]
		\item Найдите все значения параметра \( a \), при каждом из которых система уравнений
		\[ \left\{
		\begin{array}{l}
			\log_3(a-x^2)=\log_3(a-y^2),\\
			x^2+y^2=4x+6y
		\end{array}
		\right. \]
		имеет ровно два различных решения.
		\item Найдите все значения параметра \( a \), при каждом из которых система уравнений
		\[ \left\{
		\begin{array}{l}
			(x-3)(y+3x-9)=|x-3|^3,\\
			y=x+a
		\end{array}
		\right. \]
		имеет ровно четыре различных решения.
		\rightlabel{Реальный ЕГЭ 2016 г.}
		\item 15-го января планируется взять кредит в банке на 19 месяцев. Условия его возврата таковы:\\
		--- 1-го числа каждого месяца долг возрастёт на \( r\% \) по сравнению с концом предыдущего месяца;\\
		--- со 2-го по 14-е число каждого месяца необходимо выплатить часть долга;\\
		--- 15-го числа каждого месяца долг должен быть на одну и ту же сумму меньше долга на 15-е число предыдущего месяца.\\
		Известно, что общая сумма выплат после полного погашения кредита на 30\% больше суммы, взятой в кредит. Найдите \( r \).
		\item 31 декабря 2013 года Сергей взял в банке \( 9\;930\;000 \) рублей в кредит под \( 10\% \) годовых. Схема выплаты кредита следующая: 31 декабря каждого следующего года банк начисляет проценты на оставшуюся сумму долга (то есть увеличивает долг на \( 10\% \)), затем Сергей переводит в банк определённую сумму ежегодного платежа. Какой должна быть сумма ежегодного платежа, чтобы Сергей выплатил долг тремя равными ежегодными платежами?
	\end{listofex}
\end{class}
%END_FOLD

%BEGIN_FOLD % ====>>_ Занятие 8 _<<====
\begin{class}[number=8]
	\begin{listofex}
		\item Найдите все значения параметра \( a \), при каждом из которых система уравнений
		\[ \left\{
		\begin{array}{l}
			x^2+5x+y^2-y-|x-5y+5|=52,\\
			y-2=a(x-5)
		\end{array}
		\right. \]
		имеет ровно два различных решения.
		\item Алексей взял кредит в банке на срок 12 месяцев.
		По договору Алексей должен вернуть кредит ежемесячными платежами.
		В конце каждого месяца к оставшейся сумме долга добавляется \( r\% \) этой суммы и своим ежемесячным платежом Алексей погашает эти добавленные проценты и уменьшает сумму долга.
		Ежемесячные платежи подбираются так, чтобы долг уменьшался на одну и ту же величину каждый месяц (на практике такая схема называется «схемой с дифференцированными платежами»).
		Известно, что общая сумма, выплаченная Алексеем банку за весь срок кредитования, оказалась на \( 13\% \) больше, чем сумма, взятая им в кредит. Найдите \( r \).
		\item Георгий взял кредит в банке на сумму \( 804\:000 \) рублей.
		Схема выплата кредита такова: в конце каждого года банк увеличивает на 10 процентов оставшуюся сумму долга,
		а затем Георгий переводит в банк свой очередной платеж.
		Известно, что Георгий погасил кредит за три года, причем каждый его следующий платеж был ровно вдвое меньше предыдущего.
		Какую сумму Георгий заплатил в третий раз?
		Ответ дайте в рублях.
	\end{listofex}
\end{class}
%END_FOLD

%BEGIN_FOLD % ====>>_____ ЧелноГриг 1_____<<====
\begin{consultation}
	\begin{listofex}
		\item Решить систему неравенств:
		\[ \left\{
		\begin{array}{l}
			5^{x-59}\le125,\\
			9^{x+59}>\dfrac{1}{81}.
		\end{array}
		\right. \]
		\item Решить неравенство:
		\[ 3^x\cdot\left( \dfrac{1}{81} \right)^{2x+3}<9 \]
		\item Решить неравенство:
		\[ 2^x+\dfrac{1}{2^{x-5}}<33 \]
		\item Решить неравенство:
		\[ 4^{x+1}+4^{x-0,5}-2^{2x-4}\le284 \]
		\item Решить неравенство:
		\[ 7^{x^2}\le343\cdot7^x \]
		\item Решить неравенство:
		\[ 23^{x^2-4}<24^{x^2-4} \]
		\item Решите неравенство:
		\[ \dfrac{2}{5^x-1}+\dfrac{5^x-2}{5^x-3}\ge2 \]
		\item Решите неравенство:
		\[ \dfrac{1}{3^{x-1}}+\dfrac{1}{3^x}+\dfrac{1}{3^{x+1}}<52 \]
		\item Решите неравенство:
		\[ 6^x-4\cdot3^x-2^x+4\le0 \]
	\end{listofex}
	\newpage
	\title{Домашняя работа}
	\begin{listofex}
		\item Решить неравенство: \[ 2^{3x-4} +2^{3x+1} \ge 66 \]
		\item Решить неравенство: \[ \dfrac{15^x-225}{x^2+8x+12} \ge 0 \]
		\item Решить неравенство: \[ 3^x + \dfrac{27}{3^x} > 28 \]
		\item Решить неравенство: \[ x^2 \cdot 4^x + 16 < 4x^2 + 4^{x+1} \]
		\item Решить систему неравенств: 
		\[ \left\{
		\begin{array}{l}
			4^x-6 \cdot 2^x +8 >0, \\
			\dfrac{6x-5}{x-6} < 1
		\end{array}
		\right. \]
		\item Решить систему неравенств: 
		\[ \left\{
		\begin{array}{l}
			9^x - 10 \cdot 3^x + 9 \ge 0, \\
			25^{0,5x^2-5} < 0,2
		\end{array}
		\right. \]
		\item Решите неравенство:
		\[ \dfrac{3^x-1}{3^x-3}\le1+\dfrac{1}{3^x-2} \]
		\item Решите неравенство:
		\[ 9^x-31\cdot3^x+108\le0 \]
		\item Решите неравенство:
		\[ 6^x-4\cdot3^x-3\cdot2^x+12\le0 \]
	\end{listofex}
\end{consultation}
%END_FOLD