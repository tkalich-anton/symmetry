%
%===============>>  ГРУППА 11-4 МОДУЛЬ 5  <<=============
%
\setmodule{5}

%BEGIN_FOLD % ====>>_____ Занятие 1 _____<<====
\begin{class}[number=1]
	\begin{listofex}
		\item Площадь поверхности куба равна \( 18 \). Найдите его диагональ.
		\item Если каждое ребро куба увеличить на \( 1 \), то его объем увеличится на \( 19 \). Найдите ребро куба.
		\item Объем первого куба в \( 8 \) раз больше объема второго куба.
		Во сколько раз площадь поверхности первого куба больше площади поверхности второго куба?
		\item Прямоугольный параллелепипед описан около сферы радиуса \( 1 \). Найдите его площадь поверхности.
		\item Два ребра прямоугольного параллелепипеда, выходящие из одной вершины, равны \( 1 \), \( 2 \).
		Площадь поверхности параллелепипеда равна \( 16 \). Найдите его диагональ.
		\item Площадь грани прямоугольного параллелепипеда равна 12. Ребро, перпендикулярное этой грани, равно 4. Найдите объем параллелепипеда.
		\item Объем параллелепипеда \( ABCDA_1B_1C_1D_1 \) равен \( 4,5 \). Найдите объем треугольной пирамиды \( AD_1CB_1 \).
		\item В сосуд, имеющий форму правильной треугольной призмы, налили воду. Уровень воды достигает \( 80 \) см. На какой высоте будет находиться уровень воды, если ее перелить в другой такой же сосуд, у которого сторона основания в \( 4 \) раза больше, чем у первого? Ответ выразите в см.
		\item Найдите площадь боковой поверхности правильной шестиугольной призмы, сторона основания которой равна \( 5 \), а высота  --- \( 10 \).
		\item Через среднюю линию основания треугольной призмы, объем которой равен \( 32 \), проведена плоскость, параллельная боковому ребру.
		Найдите объем отсеченной треугольной призмы.
		\item В основании прямой призмы лежит ромб с диагоналями, равными \( 6 \) и \( 8 \). Площадь ее поверхности равна \( 248 \).
		Найдите боковое ребро этой призмы.
		\item Найдите объем многогранника, вершинами которого являются точки \( A \), \( B \), \( C \), \( A_1 \), \( B_1 \), \( C_1 \) правильной шестиугольной призмы \( ABCDEFA_1B_1C_1D_1E_1F_1 \), площадь основания которой равна \( 6 \), а боковое ребро равно \( 3 \).
		\item В правильной треугольной пирамиде \( SABC \) медианы основания \( ABC \) пересекаются в точке \( O \). Площадь треугольника ABC равна \( 9 \); объем пирамиды равен \( 6 \). Найдите длину отрезка \( OS \).
		\item В правильной треугольной пирамиде \( SABC \) точка \( M \) --- середина ребра \( AB \), \( S \) --- вершина.
		Известно, что \( BC = 3 \), а площадь боковой поверхности пирамиды равна \( 45 \). Найдите длину отрезка \( SM \).
		\item В цилиндрическом сосуде уровень жидкости достигает \( 16 \) см. На какой высоте будет находиться уровень жидкости, если ее перелить во второй сосуд, диаметр которого в \( 2 \) раза больше первого? Ответ дайте в сантиметрах.
		\item Одна цилиндрическая кружка вдвое выше второй, зато вторая в полтора раза шире. Найдите отношение объема второй кружки к объему первой.
		\item Площадь боковой поверхности цилиндра равна \( 2 \pi \), а диаметр основания  --- \( 1 \). Найдите высоту цилиндра.
		\item Во сколько раз уменьшится объем конуса, если его высота уменьшится в \( 3 \) раза, а радиус основания останется прежним?
		\item Конус получается при вращении равнобедренного прямоугольного треугольника \( ABC \) вокруг катета, равного \( 6 \). Найдите его объем, деленный на \( \pi \).
		\item Найдите все значения параметра \( a \), при которых наименьшее значение функции
		\[ y=|x+4|+|2x-a| \]
		меньше \( 3 \).
	\end{listofex}
\end{class}
%END_FOLD

%BEGIN_FOLD % ====>>_____ Занятие 2 _____<<====
\begin{class}[number=2]
	\begin{listofex}
		\item Занятие 2
	\end{listofex}
\end{class}
%END_FOLD

%BEGIN_FOLD % ====>>_____ Занятие 3 _____<<====
\begin{class}[number=3]
	\begin{listofex}
		\item Занятие 3
	\end{listofex}
\end{class}
%END_FOLD

%BEGIN_FOLD % ====>>_____ Занятие 4 _____<<====
\begin{class}[number=4]
	\begin{listofex}
		\item Занятие 4
	\end{listofex}
\end{class}
%END_FOLD

%BEGIN_FOLD % ====>>_____ Занятие 5 _____<<====
\begin{class}[number=5]
	\begin{listofex}
		\item Занятие 5
	\end{listofex}
\end{class}
%END_FOLD

%BEGIN_FOLD % ====>>_____ Занятие 6 _____<<====
\begin{class}[number=6]
	\begin{listofex}
		\item Занятие 6
	\end{listofex}
\end{class}
%END_FOLD

%BEGIN_FOLD % ====>>_____ Занятие 7 _____<<====
\begin{class}[number=7]
	\begin{listofex}
		\item Занятие 7
	\end{listofex}
\end{class}
%END_FOLD

%BEGIN_FOLD % ====>>_ Домашняя работа 1 _<<====
\begin{homework}[number=1]
	\begin{listofex}
		\item Площадь поверхности куба равна 24. Найдите его объем.
%		\item Если каждое ребро куба увеличить на 1, то его площадь поверхности увеличится на 54. Найдите ребро куба.
%		\item Объем одного куба в 729 раз больше объема другого куба. Во сколько раз площадь поверхности первого куба больше площади поверхности второго куба?
%		\item Два ребра прямоугольного параллелепипеда, выходящие из одной вершины, равны 3 и 4. Площадь поверхности этого параллелепипеда равна 94. Найдите третье ребро, выходящее из той же вершины.
%		\item Объем прямоугольного параллелепипеда равен \( 24 \). Одно из его ребер равно \( 3 \). Найдите площадь грани параллелепипеда, перпендикулярной этому ребру.
%		\item Найдите объем многогранника, вершинами которого являются точки A, D, A1, B, C, B1 прямоугольного параллелепипеда \( ABCDA_1B_1C_1D_1 \), у которого AB = 3, AD = 4, AA_1 = 5.
%		\item Через среднюю линию основания треугольной призмы проведена плоскость, параллельная боковому ребру. Найдите объём этой призмы, если объём отсеченной треугольной призмы равен 5.
%		\item Основанием прямой треугольной призмы служит прямоугольный треугольник с катетами 6 и 8, высота призмы равна 10. Найдите площадь ее поверхности.
%		\item Найдите объем многогранника, вершинами которого являются точки A, B, D, E, A_1, B_1, D_1, E_1 правильной шестиугольной призмы ABCDEFA_1B_1C_1D_1E_1F_1, площадь основания которой равна 6, а боковое ребро равно 2.
%		\item В правильной треугольной пирамиде SABC медианы основания ABC пересекаются в точке O. Площадь треугольника ABC равна 2; объем пирамиды равен 5. Найдите длину отрезка OS.
%		\item Длина окружности основания цилиндра равна 3, высота равна 2. Найдите площадь боковой поверхности цилиндра.
%		\item Площадь боковой поверхности цилиндра равна \( 2 \pi \) , а диаметр основания  --- \( 1 \). Найдите диаметр цилиндра.
%		\item Во сколько раз увеличится объем конуса, если радиус его основания увеличится в \( 1,5 \) раза, а высота останется прежней?
%		\item Высота конуса равна \( 6 \), образующая равна \( 10 \). Найдите его объем, деленный на \( \pi \) .
	\end{listofex}
\end{homework}
%END_FOLD

%BEGIN_FOLD % ====>>_ Домашняя работа 2 _<<====
\begin{homework}[number=2]
	\begin{listofex}
		\item ДЗ 2
	\end{listofex}
\end{homework}
%END_FOLD

%BEGIN_FOLD % ====>>_ Домашняя работа 3 _<<====
\begin{homework}[number=3]
	\begin{listofex}
		\item ДЗ 3
	\end{listofex}
\end{homework}
%END_FOLD

%BEGIN_FOLD % ====>>_ Проверочная работа _<<====
\begin{exam}
	\begin{listofex}
		\item Проверочная работа
	\end{listofex}
\end{exam}
%END_FOLD
