%
%===============>>  ГРУППА 11-4 МОДУЛЬ 6  <<=============
%
\setmodule{6}

%BEGIN_FOLD % ====>>_____ Занятие 1 _____<<====
\begin{class}[number=1]
	\begin{listofex}
		\item Решить неравенство: 
		\( |\lg|x-1|+2|\ge3 \)
		\item Решить неравенство: 
		\( 9\log_{12}(x^2-3x-4)\le10+\log_{12}\dfrac{(x+1)^9}{x-4} \)
		\item Решить систему неравенств: 
		\[ \left\{
		\begin{array}{l}
			4^{x+1}-33\cdot2^x+8\le0, \\
			2\log_2\dfrac{x-1}{x+1,2}+\log_2(x+1,2)^2\ge2
		\end{array}
		\right. \]
		\item Решить неравенство: 
		\( \log_{6x^2-x-1}(2x^2-5x+3)\ge0 \)
		\item Решить неравенство: 
		\( \log_{3-x}(x+1)\cdot\log_(x+2)(4-x)\le0 \)
		\item Решить неравенство: 
		\( \log_{x+2}(7x^2+11x-6)<2 \)
		\item Решить неравенство: 
		\( \log_{\frac{25-x^2}{16}}{\dfrac{24+2x-x^2}{14}}>1 \)
		\item Решить неравенство: 
		\( \log_{-5x^2-6x}(6^x)>0 \)
		\item Решить неравенство: 
		\( \log_{x+5}(4x^2-5x+1)\le\log_{\frac{10x+41}{10x+43}}1 \)
		\item Решить неравенство: 
		\( \left( \dfrac{1}{2} \right)^{\log_3(\log_{1/5}(x^2-4/5))}>1 \)
		\item Решить неравенство: 
		\( \log_{x^2+1}\dfrac{2\cdot4^x-15\cdot2^x+23}{4^x-9\cdot2^x+14}\ge0 \)
		\item Решить систему неравенств: 
		\[ \left\{
		\begin{array}{l}
			4^x-12\cdot2^x+32\ge0, \\
			\log_{\frac{2x^2-7x+6}{x-6}}(x-1)\ge0
		\end{array}
		\right. \]
	\end{listofex}
\end{class}
%END_FOLD

%BEGIN_FOLD % ====>>_____ Занятие 2 _____<<====
\begin{class}[number=2]
	\begin{listofex}
		\item Решить неравенство: 
		\( |\lg|x-3|+1|\ge4 \)
		\item Решить неравенство: 
		\( \dfrac{\sqrt{3x-4}}{5x-2}\le\dfrac{\sqrt{3x-4}}{2x+7} \)
		\item Решить неравенство: 
		\( \log_{1-x}(x+3)\cdot\log_{x+4}(2-x)\le0 \)
		\item Решить неравенство: 
		\( \log_{x+1}(6x^2+x-5)<2 \)
		\item Решить неравенство: 
		\( \log_3(x^2-1)\cdot(\log_9(x-1)+\log_9(x+1))\ge2 \)
		\item Решить неравенство: 
		\( \log_x \left( \dfrac{4x+5}{6-5x} \right)< -1 \)
		\item Решить неравенство: 
		\( \log_{x+5}(4x^2-5x+1)\le\log_{\frac{10x+41}{10x+43}}1 \)
		\item Решить неравенство: 
		\( \left( \dfrac{1}{2} \right)^{\log_3(\log_{1/5}(x^2-4/5))}>1 \)
	\end{listofex}
\end{class}
%END_FOLD

%BEGIN_FOLD % ====>>_ Домашняя работа 1 _<<====
\begin{homework}[number=1]
	\begin{listofex}
		\item Решить неравенство: 
		\( |x^2-6x+3|\le x-3 \)
		\item Решить неравенство: 
		\( \log_{6x^2-x-1}(3x^2-7x+3)\ge0 \)
		\item Решить неравенство: 
		\( \dfrac{1-2x}{4x+1}+5\sqrt{\dfrac{2x-1}{4x+1}}>6 \)
		\item Решить систему неравенств: 
		\[ \left\{
		\begin{array}{l}
			3\cdot26^x\le2\cdot39^x, \\
			11\cdot15^x\ge5\cdot33^x
		\end{array}
		\right. \]
		\item Решить неравенство: 
		\( \dfrac{3^x-1}{3^x-3}\le1+\dfrac{1}{3^x-2} \)
		\item Решить неравенство: 
		\( \log_{x-1}(5-x)\cdot\log_(x-1)x\ge0 \)
		\item Решить неравенство: 
		\( \log_{x+6}(5x^2-6x+1)\le\log_{\frac{10x+57}{10x+59}}1 \)
		\item Решить неравенство: 
		\( \log_4\dfrac{3-x}{x-7}+\log_{0,25}(x-3)\ge\log_{1/4}(x-7)^2 \)
	\end{listofex}
\end{homework}
%END_FOLD

%BEGIN_FOLD % ====>>_____ Занятие 3 _____<<====
\begin{class}[number=3]
	\begin{listofex}
		\item Решить неравенство: 
		\[ 5^x+\dfrac{125}{5^x-126}\ge0 \]
		\begin{flushright}
			\textit{(Основная волна ЕГЭ 2022)}
		\end{flushright}
		\item Решить неравенство:
		\[ \dfrac{2}{5^x+75}\ge\dfrac{1}{5^x-25} \]
		\begin{flushright}
			\textit{(Основная волна ЕГЭ 02.06.2022)}
		\end{flushright}
		\item Решить неравенство:
		\[ 16^{\tfrac{1}{x}-1}-4^{\tfrac{1}{x}-1}-2\ge0 \]
		\begin{flushright}
			\textit{(Основная волна ЕГЭ 07.06.2021)}
		\end{flushright}
		\item Решить неравенство:
		\[ x^2\log_{343}(x+3)\le\log_7(x^2+6x+9) \]
		\begin{flushright}
			\textit{(Основная волна ЕГЭ 2020)}
		\end{flushright}
		\item Решить неравенство:
		\[ 1+\dfrac{6}{\log_3x-3}+\dfrac{5}{\log^2_3x-\log_3(27x^6)+12}\ge0 \]
		\begin{flushright}
			\textit{(Досрочный ЕГЭ 2022)}
		\end{flushright}
		\item Решить неравенство:
		\[ \log_3(x+7)+\dfrac{1}{6}\log_3(x+1)^6\ge2 \]
		\begin{flushright}
			\textit{(Основная волна ЕГЭ 2021 СПб)}
		\end{flushright}
		\item Решить неравенство:
		\[ \dfrac{1}{2^x-1}+\dfrac{4^{x+\tfrac{1}{2}}-2^{x+5}+4}{2^x-16}\ge2^{x+1} \]
		\begin{flushright}
			\textit{(Резервный ЕГЭ 29.06.2021)}
		\end{flushright}
		\item Решить неравенство:
		\[ \dfrac{1}{\log_3x+4}+\dfrac{2}{\log_3(3x)}\cdot\left( \dfrac{2}{\log_3x+4}-1 \right)\le0 \]
		\begin{flushright}
			\textit{(Досрочный ЕГЭ 29.04.2021)}
		\end{flushright}
		\item Решить неравенство:
		\[ \log_5\left( (3-x)(x^2+2) \right)\ge\log_5(x^2-7x+12)+\log_5(5-x) \]
		\begin{flushright}
			\textit{(Резервный ЕГЭ 24.06.2019)}
		\end{flushright}
		\item Решить неравенство:
		\[ 27\cdot45^x-27^{x+1}-12\cdot15^x+12\cdot9^x+5^x-3^x\le0 \]
		\begin{flushright}
			\textit{(Резервный ЕГЭ 24.07.2020)}
		\end{flushright}
		\item Решить неравенство:
		\[ 2^{x+1}+\dfrac{9}{x}-\dfrac{3\cdot2^x}{x}\ge6 \]
		\begin{flushright}
			\textit{(Пробный ЕГЭ 2020 СПб)}
		\end{flushright}
		\item Решить неравенство:
		\[ \log_3(4-4x)\ge\log_3(x^2-4x+3)+\log_3(x+2) \]
		\begin{flushright}
			\textit{(Основная волна ЕГЭ 29.05.2019)}
		\end{flushright}
		\item Решить неравенство:
		\[ \dfrac{9^x+2\cdot3^x-117}{3^x-27}\le1 \]
		\begin{flushright}
			\textit{(Досрочный ЕГЭ 29.03.2019)}
		\end{flushright}
	\end{listofex}
\end{class}
%END_FOLD

%BEGIN_FOLD % ====>>_____ Занятие 4 _____<<====
\begin{class}[number=4]
	\begin{listofex}
		\item Решить неравенство:
		\[ 27\cdot45^x-27^{x+1}-12\cdot15^x+12\cdot9^x+5^x-3^x\le0 \]
		\begin{flushright}
			\textit{(Резервный ЕГЭ 24.07.2020)}
		\end{flushright}
		\item Решить неравенство:
		\[ \dfrac{9^x+2\cdot3^x-117}{3^x-27}\le1 \]
		\begin{flushright}
			\textit{(Досрочный ЕГЭ 29.03.2019)}
		\end{flushright}
		\item Решить неравенство:
		\[ \log_2(x-1)(x^2+2)\le1+\log_2(x+3x-4)-\log_2x \]
		\begin{flushright}
			\textit{(Резервный ЕГЭ 24.06.2019)}
		\end{flushright}
		\item Решить неравенство:
		\[ \log_2\left( \dfrac{1}{x}-1 \right)+\log_2\left( \dfrac{1}{x}+1 \right)\le\log_2(27x-1) \]
		\begin{flushright}
			\textit{(Резервный ЕГЭ 25.06.2018)}
		\end{flushright}
		\item Решить неравенство:
		\[ \log_{11}(8x^2+7)-\log_{11}(x^2+x+1)\ge\log_{11}\left( \dfrac{x}{x+5}+7 \right) \]
		\begin{flushright}
			\textit{(Реальный ЕГЭ 01.06.2018)}
		\end{flushright}
		\item Решить неравенство:
		\[ 3^{x^2}\cdot5^{x-1}\ge3 \]
		\begin{flushright}
			\textit{(Досрочный ЕГЭ 11.04.2018)}
		\end{flushright}
	\end{listofex}
\end{class}
%END_FOLD

%BEGIN_FOLD % ====>>_ Домашняя работа 2 _<<====
\begin{homework}[number=2]
	\begin{listofex}
		\item Решите неравенство:
		\[\dfrac{1}{8}\log_2(x-2)^8+\log_2(x+4)\geq3\]
		\item Решите неравенство:
		\[x^2\log_{625}(6-x)\leq\log_5(x^2-12x+36)\]
		\item Решите неравенство:
		\[x^2\log_{512}(9-x)\leq\log_2(x^2-18x+81)\]
		\item Решите неравенство:
		\[\log_5\left( (3-x)(x^2+2) \right)\geq\log_5(x^2-7x+12)+\log_5(5-x)\]
	\end{listofex}
\end{homework}
%END_FOLD

%BEGIN_FOLD % ====>>_____ Занятие 5 _____<<====
\begin{class}[number=5]
	\begin{listofex}
		\item Занятие 5
	\end{listofex}
\end{class}
%END_FOLD

%BEGIN_FOLD % ====>>_____ Занятие 6 _____<<====
\begin{class}[number=6]
	\begin{listofex}
		\item Занятие 6
	\end{listofex}
\end{class}
%END_FOLD

%BEGIN_FOLD % ====>>_ Домашняя работа 3 _<<====
\begin{homework}[number=3]
	\begin{listofex}
		\item Решите уравнение:
		\[2\cos x\cdot\left( \cos x+\cos\dfrac{5\pi}{4} \right)+\cos x+\cos\dfrac{3\pi}{4}=0\]
		\item Решите уравнение:
		\[2(\sin x+\cos x)=\ctg x+1\]
		\item Решите уравнение:
		\[\cos\left( x-\dfrac{3\pi}{2} \right)=\sin2x\]
		\item Решите уравнение:
		\[\dfrac{\sin x}{\cos x+1}=\cos x-1\]
		\item Решите уравнение:
		\[\sin^2x-2\cos x=2\sin^3\left( x-\dfrac{\pi}{2} \right)\]
		\item Решите уравнение:
		\[4\sqrt{3}\sin x-\sin2x=2\sqrt{3}\sin^2x-4\cos x\]
		\item Решите уравнение:
		\[2\cos\left( \dfrac{\pi}{4}-2x \right)-\sqrt{2}\sin x=\sqrt{2}\sin2x+\sqrt{2}\]
		\item Решите уравнение:
		\[8\cos^4x+3\cos2x-6=0\]
		\item Решите уравнение:
		\[\left( \frac{1}{10} \right)^{\sqrt{3}\sin\left( \dfrac{\pi}{2}-x \right)}=10^{\sin(2\pi-x)}\]
	\end{listofex}
\end{homework}
%END_FOLD

%BEGIN_FOLD % ====>>_____ Занятие 7 _____<<====
\begin{class}[number=7]
	\title{Подготовка к проверочной}
	\begin{listofex}
		\item Занятие 7
	\end{listofex}
\end{class}
%END_FOLD

%BEGIN_FOLD % ====>>_ Проверочная работа _<<====
\begin{class}[number=8]
	\begin{listofex}
		\item а) Решить уравнение: \( 9^{\sin x}+9^{-\sin x}=\dfrac{10}{3} \)\\
		б) Укажите корни этого уравнения, принадлежащие отрезку \( \left[ -\dfrac{7\pi}{2};-2\pi \right] \)
		\item При каких значениях параметра \( a \) уравнение
		\[ \dfrac{x^2-2x+a^2-4a}{x^2-a}=0 \]
		имеет ровно \( 2 \) различных решения.
		\item При каких значениях параметра \( a \) уравнение
		\[ \dfrac{|4x|-x-3-a}{x^2-x-a}=0 \]
		имеет ровно \( 2 \) различных решения.
	\end{listofex}
\end{class}
%END_FOLD