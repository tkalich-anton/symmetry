%
%===============>>  ГРУППА 11-4 МОДУЛЬ 7  <<=============
%
\setmodule{7}

%BEGIN_FOLD % ====>>_____ Занятие 1 _____<<====
\begin{class}[number=1]
	\begin{listofex}
		\item Найдите все значения параметра \( a \), при каждом из которых уравнение:
		\[ (x-a)(x^2-x-a)=0 \]
		а) имеет два различных корня;\\
		б) имеет три различных корня.
		\item Найдите все значения параметра \( a \), при каждом из которых уравнение:
		\[ \dfrac{x^2-x-a}{x-a}=0 \]
		а) имеет два различных корня;\\
		б) имеет один корень.
		\item Найдите все значения параметра \( a \), при каждом из которых уравнение:
		\[ (4x-a)(3,75|x|-x^2+a)=0 \]
		имеет два или три различных корня.
		\item При каких значениях \( a \) уравнение
		\[ 2|x-a|+a-4+x=0 \]
		имеет решение и все решения удовлетворяют неравенству \( 0\le x \le 4 \)?
		\item Найдите все значения параметра \( a \), при каждом из которых уравнение
		\[ x^2-2(a^2-4a+1)x+4=0 \]
		имеет два различных отрицательных корня.
		\item Найдите все значения параметра \( a \), при каждом из которых уравнение
		\[ \dfrac{x^2-2x-1}{x^2-2x+2}=a \]
		имеет хотя бы один корень.
		\item Найдите все значения параметра \( a \), при каждом из которых уравнение
		\[ 121^x+(3a^2-a+4)\cdot11^x-5a-2=0 \]
		имеет единственный корень.
	\end{listofex}
\end{class}
%END_FOLD

%BEGIN_FOLD % ====>>_____ Занятие 2 _____<<====
\begin{class}[number=2]
	\begin{listofex}
		\item Найдите все значения параметра \( a \), при каждом из которых уравнение
		\[ 121^x+(3a^2-a+4)\cdot11^x-5a-2=0 \]
		имеет единственный корень.
		\item Найдите все значения параметра \( a \), при каждом из которых уравнение
		\[ x^2-2(a^2-4a+1)x+4=0 \]
		имеет два различных отрицательных корня.
		\item Найдите все значения параметра \( a \), при каждом из которых уравнение
		\[ (a+3)x^2-2(a+3)x-5=0 \] имеет единственный корень.
		\item Найдите все значения параметра a, при каждом из которых
		уравнение
		\[ (ax^2-(a^2+9)+9a)\sqrt{x+5}=0 \]
		имеет ровно два различных корня.
		\item Найдите все значения параметра a, при каждом из которых
		уравнение
		\[ x^2+2(a^2+7a+3)x+9=0 \]
		имеет два различных положительных корня.
		\item Найдите все значения параметра \( a \), при каждом из которых
		уравнение
		\[ 4\cos^43x-4(a-3)\cos^23x-2a+5=0 \]
		имеет хотя бы один корень.
	\end{listofex}
\end{class}
%END_FOLD

%BEGIN_FOLD % ====>>_ Домашняя работа 1 _<<====
\begin{homework}[number=1]
	\begin{listofex}
		\item Домашняя работа 1
	\end{listofex}
\end{homework}
%END_FOLD

%BEGIN_FOLD % ====>>_____ Занятие 3 _____<<====
\begin{class}[number=3]
	\begin{listofex}
		\item Занятие 3 
	\end{listofex}
\end{class}
%END_FOLD

%BEGIN_FOLD % ====>>_____ Занятие 4 _____<<====
\begin{class}[number=4]
	\begin{listofex}
		\item Занятие 4
	\end{listofex}
\end{class}
%END_FOLD

%BEGIN_FOLD % ====>>_ Домашняя работа 2 _<<====
\begin{homework}[number=2]
	\begin{listofex}
		\item Домашняя работа 2
	\end{listofex}
\end{homework}
%END_FOLD

%BEGIN_FOLD % ====>>_____ Занятие 5 _____<<====
\begin{class}[number=5]
	\begin{listofex}
		\item Занятие 5
	\end{listofex}
\end{class}
%END_FOLD

%BEGIN_FOLD % ====>>_____ Занятие 6 _____<<====
\begin{class}[number=6]
	\begin{listofex}
		\item Занятие 6
	\end{listofex}
\end{class}
%END_FOLD

%BEGIN_FOLD % ====>>_ Домашняя работа 3 _<<====
\begin{homework}[number=3]
	\begin{listofex}
		\item При каких значениях \( a \) уравнение \( 2x^2+x-a=0 \) имеет хотя бы один общий корень с уравнением \( 2x^2-7x+6=0 \)?
		\item При каком значении параметра \( a \) уравнение \\ \( (a+4x-x^2-1)(a+1-|x-2|)=0 \) имеет три корня?
	\end{listofex}
\end{homework}
%END_FOLD

%BEGIN_FOLD % ====>>_____ Занятие 7 _____<<====
\begin{class}[number=7]
	\title{Подготовка к проверочной}
	\begin{listofex}
		\item Занятие 7
	\end{listofex}
\end{class}
%END_FOLD

=%BEGIN_FOLD % ====>>_ Проверочная работа _<<====
\begin{exam}
	\begin{listofex}
		\item Проверочная
	\end{listofex}
\end{exam}
%END_FOLD

%BEGIN_FOLD % ====>>_ Консультация _<<====
\begin{consultation}
	\begin{listofex}
		\item Найдите все значения параметра \( a \), при каждом из которых уравнение \( (a-2)x^2-4ax+a-1=0 \) имеет два корня разных знаков.
		\item Найдите все значения параметра \( a \), при каждом из которых уравнение \( (a+3)x^2+2(a-3)x-a+3=0 \) имеет хотя бы один корень больше \( -1 \).
		\item Найдите все значения параметра \( a \), при каждом из которых уравнение \( (a-2)x^2-4(a+1)x+2a+2=0 \) имеет хотя бы один корень, меньший \( 2 \).
	\end{listofex}
\end{consultation}
%END_FOLD