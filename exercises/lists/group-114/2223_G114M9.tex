%
%===============>>  ГРУППА 11-4 МОДУЛЬ 9  <<=============
%
\setmodule{9}

%BEGIN_FOLD % ====>>_____ Занятие 1 _____<<====
\begin{class}[number=1]
	\begin{listofex}
		\item Решите неравенство: \(  \dfrac{x-3\sqrt{x-1}+1}{4\sqrt{x-1}-2-x}\le0 \)
		\item Найдите все значения параметра \( a \), при которых неравенство
		\[ (x-a)(x-a-2)>0 \]
		является следствием неравенства
		\[ (x-1)(x-3)<0 \]
		\item Найти все значения параметра \( a \), при которых неравенство
		\[ 2x+2|x-a|+|x-1| > 3 \]
		выполняется для всех \( x\in R \).
		\item Найти все значения параметра \( a \), при которых уравнение
		\[ \log_2(4^x-a)=x \]
		имеет ровно два решения.
		\item В июле планируется взять кредит в банке на сумму 18 млн рублей на некоторый срок (целое число лет).
		Условия его возврата таковы:\\
		--- каждый январь долг возрастает на \( 10\% \) по сравнению с концом предыдущего года;\\
		--- с февраля по июнь каждого года необходимо выплатить часть долга;\\
		--- в июле каждого года долг должен быть на одну и ту же сумму меньше долга на июль предыдущего года.\\
		На сколько лет был взят кредит, если общая сумма выплат после полного погашения кредита составила \( 27 \) млн рублей?
		\item В июле 2018 года планируется взять кредит в банке. Условия его возврата таковы:\\
		--- каждый январь долг увеличивается на 20\% по сравнению с концом предыдущего года;\\
		---с февраля по июнь каждого года необходимо выплатить одним платежом часть долга.\\
		Сколько рублей необходимо взять в банке, если известно,
		что кредит будет полностью погашен четырьмя равными платежами,
		и банку будет выплачено \( 311\:040 \) рублей?
		\item Светлана взяла кредит в банке на 4 года на сумму \( 4\:420\:000 \) рублей.
		Условия возврата кредита таковы: в конце каждого года банк увеличивает текущую сумму долга на \( 10\% \).
		Светлана хочет выплатить весь долг двумя равными платежами --- в конце второго и четвертого годов.
		При этом платежи в каждом случае выплачиваются после начисления процентов.
		Сколько рублей составит каждый из этих платежей?
		\item В июле 2026 года планируется взять кредит на пять лет в размере \( 220 \) тысяч рублей.
		Условия его возврата таковы:\\
		--- каждый январь долг возрастает на \( r\% \) по сравнению с концом предыдущею года;\\
		--- с февраля по июнь каждого года необходимо выплатить одним платежом часть долга;
		--- в июле \( 2027 \), \( 2028 \) и \( 2029 \) годов долг остаётся равным 220 тысяч рублей;\\
		--- выплаты в \( 2030 \) и \( 2031 \) годах равны;\\
		--- к июлю \( 2031 \) года долг будет выплачен полностью.\\
		Найдите \( r \), если известно, что долг будет выплачен полностью и общий размер выплат составит \( 420 \) тысяч рублей.
	\end{listofex}
\end{class}
%END_FOLD

%BEGIN_FOLD % ====>>_____ Занятие 2 _____<<====
\begin{class}[number=2]
	\begin{listofex}
		\item Найти все значения параметра \( a \), при которых система неравенств:
		\[ \begin{cases}
			x^2+2x+a\le0,\\
			x^2-4x-6a\le0
		\end{cases} \]
		имеет единственное решение.
		\item Найти все значения параметра \( a \), при которых неравенство
		\[ |x^2+4x-a| > 6 \]
		не имеет решений на отрезке \( [-3;0] \).
	\end{listofex}
\end{class}
%END_FOLD

%BEGIN_FOLD % ====>>_ Домашняя работа 1 _<<====
\begin{homework}[number=1]
	\begin{listofex}
		\item Найти все значения параметра \( a \), при которых неравенство
		\[ (x-3a)(x-a-3)<0 \]
		выполняется при всех \( x \), таких, что \( 1\le x \le 3 \).
		\item Найти все значения параметра \( a \), при которых неравенство
		\[ x^2+|x-a|-3\ge0 \]
		имеет хотя бы одно положительное решение.
		\item Найти все значения параметра \( a \), при которых неравенство
		\[ |x^2-2x+a| > 5 \]
		не имеет решений на отрезке \( [-1;2] \).
		\item Найти все значения параметра \( a \), при которых уравнение
		\[ \sqrt{x+a}=x \]
		имеет решение, принадлежащее отрезку \( [0;1] \).
	\end{listofex}
\end{homework}
%END_FOLD

%BEGIN_FOLD % ====>>_____ Занятие 3 _____<<====
\begin{class}[number=3]
	\begin{listofex}
		\item 3
	\end{listofex}
\end{class}
%END_FOLD

%BEGIN_FOLD % ====>>_____ Занятие 4 _____<<====
\begin{class}[number=4]
	\begin{listofex}
		\item 4
	\end{listofex}
\end{class}
%END_FOLD

%BEGIN_FOLD % ====>>_ Домашняя работа 2 _<<====
\begin{homework}[number=2]
	\begin{listofex}
		\item Найти все значения параметра \( a \), при каждом из которых неравенство
		\[ 2 > |x+a|+x^2 \]
		имеет хотя бы одно положительное решение.
	\end{listofex}
\end{homework}
%END_FOLD

%BEGIN_FOLD % ====>>_____ Занятие 5 _____<<====
\begin{class}[number=5]
	\begin{listofex}
		\item 5
	\end{listofex}
\end{class}
%END_FOLD

%BEGIN_FOLD % ====>>_____ Занятие 6 _____<<====
\begin{class}[number=6]
	\begin{listofex}
		\item 6
	\end{listofex}
\end{class}
%END_FOLD

%BEGIN_FOLD % ====>>_ Домашняя работа 3 _<<====
\begin{homework}[number=2]
	\begin{listofex}
		\item 3
	\end{listofex}
\end{homework}
%END_FOLD

%BEGIN_FOLD % ====>>_____ Занятие 7 _____<<====
\begin{class}[number=7]
	\begin{listofex}
		\item 7
	\end{listofex}
\end{class}
%END_FOLD

%BEGIN_FOLD % ====>>_____ Занятие 8 _____<<====
\begin{class}[number=8]
	\begin{listofex}
		\item 8
	\end{listofex}
\end{class}
%END_FOLD