\setmodule{9}

%BEGIN_FOLD % ====>>_____ Занятие 1 _____<<====
\begin{class}[number=1]
	\begin{listofex}
		\item Занятие 1
	\end{listofex}
\end{class}
%END_FOLD

%BEGIN_FOLD % ====>>_____ Занятие 2 _____<<====
\begin{class}[number=2]
	\begin{listofex}
		\item Занятие 2
	\end{listofex}
\end{class}
%END_FOLD

%BEGIN_FOLD % ====>>_ Домашняя работа 1 _<<====
\begin{homework}[number=1]
	\begin{listofex}
		\item Домашняя работа 1
	\end{listofex}
\end{homework}
%END_FOLD

%BEGIN_FOLD % ====>>_____ Занятие 3 _____<<====
\begin{class}[number=3]
	\begin{listofex}
		\item Занятие 3 
	\end{listofex}
\end{class}
%END_FOLD

%BEGIN_FOLD % ====>>_____ Занятие 4 _____<<====
\begin{class}[number=4]
	\begin{listofex}
		\item 1
	\end{listofex}
\end{class}
%END_FOLD

%BEGIN_FOLD % ====>>_ Домашняя работа 2 _<<====
\begin{homework}[number=2]
	\begin{listofex}
		\item Объем шара равен \( 288\pi \). Найдите площадь его поверхности, деленную на \( \pi \).
		\item Игральный кубик бросают дважды. Известно, что в сумме выпало \( 8 \) очков. Найдите вероятность того, что во второй раз выпало \( 3 \) очка.
		\item Игральную кость бросили один или несколько раз. Оказалось, что сумма всех выпавших очков равна \( 3 \). Какова вероятность того, что было сделано два броска? Ответ округлите до сотых.
		\item Маша коллекционирует принцесс из Киндер-сюрпризов. Всего в коллекции \( 10 \) разных принцесс, и они равномерно распределены, то есть в каждом очередном Киндер-сюрпризе может с равными вероятностями оказаться любая из \( 10 \) принцесс. У Маши уже есть две разные принцессы из коллекции. Какова вероятность того, что для получения следующей принцессы Маше придётся купить ещё \( 2 \) или \( 3 \) шоколадных яйца?
		\item 
		\begin{minipage}[t]{\bodywidth}
			На рисунке изображены график функции \( y=f(x) \) и касательная к этому графику, проведённая в точке \( x_0=2 \). Найдите значение производной функции \( g(x)=x^2-f(x)+1 \) в точке \( x_0 \).
		\end{minipage}
		\gapwidth
		\begin{minipage}[t]{\picwidth}
			\includegraphics[align=t, width=\linewidth]{\picpath/G113M9H2}
		\end{minipage}
		\item Прямая \( y=7x-5 \) параллельна касательной к графику функции \( y=x^2+6x-8 \). Найдите абсциссу точки касания.
		\item Игорь и Паша красят забор за \( 9 \) часов. Паша и Володя красят этот же забор за \( 12 \) часов, а Володя и Игорь --- за \( 18 \) часов. За сколько часов мальчики покрасят забор, работая втроем?
		\item Найдите наименьшее значение функции \( y=2x^2-5x+\ln x-3 \) на отрезке \( \left[ \dfrac{5}{6}; \dfrac{7}{6} \right] \)
		\item \begin{tasks}(1)
			\task Решите уравнение: \( 2^{4\sin^2x+1}+2^{4\cos^2x}=18 \).
			\task Найдите все корни этого уравнения, принадлежащие отрезку \( \left[ 2\pi; \dfrac{7\pi}{2} \right] \).
		\end{tasks}
		\item Решите неравенство: \( \log_2(x^2+4x)+\log_{0,5}\dfrac{x}{4}+2\ge\log_2(x^2+3x-4)\).
		\item Решите неравенство: \( \log_x(x^3-8)\le\log_x(x^3+2x-13) \).
		\item \( 15 \)-го декабря планируется взять кредит в банке на сумму \( 300 \) тысяч рублей на \( 21 \) месяц. Условия возврата таковы:\\		
		--- \( 1 \)-го числа каждого месяца долг возрастает на \( 2\% \) по сравнению с концом предыдущего месяца;\\
		--- со \( 2 \)-го по \( 14 \)-е число каждого месяца необходимо выплатить часть долга;\\
		--- \( 15 \)-го числа каждого месяца с \( 1 \)-го по \( 20 \)-й долг должен быть на одну и ту же сумму меньше долга на \( 15 \)-е число предыдущего месяца;\\		
		--- \( 15 \)-го числа \( 20 \)-го месяца долг составит \( 100 \) тысяч рублей;\\
		--- к \( 15 \)-му числу \( 21 \)-го месяца кредит должен быть полностью погашен.\\
		Найдите общую сумму выплат после полного погашения кредита.
		\item Найдите все значения параметра \( a \) при каждом из которых уравнение
		\[x^2+a^2-2x-6a=|6x-2a|\]
		имеет ровно два различных корня.
	\end{listofex}
\end{homework}
%END_FOLD

%BEGIN_FOLD % ====>>_____ Занятие 5 _____<<====
\begin{class}[number=5]
	\begin{listofex}
		\item Занятие 5
	\end{listofex}
\end{class}
%END_FOLD

%BEGIN_FOLD % ====>>_____ Занятие 6 _____<<====
\begin{class}[number=6]
	\begin{listofex}
		\item Занятие 6
	\end{listofex}
\end{class}
%END_FOLD

%BEGIN_FOLD % ====>>_ Домашняя работа 3 _<<====
\begin{homework}[number=3]
	\begin{listofex}
		\item \( 31 \) декабря \( 2014 \) года Тимофей взял в банке \( 7 007 000  \) рублей в кредит под \( 20\% \) годовых. Схема выплаты кредита следующая: \( 31 \) декабря каждого следующего года банк начисляет проценты на оставшуюся сумму долга (то есть увеличивает долг на \( 20\% \)), затем Тимофей переводит в банк платёж. Весь долг Тимофей выплатил за \( 3 \) равных платежа. На сколько рублей меньше он бы отдал банку, если бы смог выплатить долг за \( 2 \) равных платежа?
		\item В июле планируется взять кредит на сумму \( 1 342 000 \) рублей. Условия его возврата таковы:\\
		--- каждый январь долг возрастает на \( 20\% \) по сравнению с концом предыдущего года;\\		
		--- с февраля по июнь каждого года необходимо выплатить некоторую часть долга.\\		
		На сколько рублей больше придётся отдать в случае, если кредит будет полностью погашен четырьмя равными платежами (то есть за \( 4 \) года), по сравнению со случаем, если кредит будет полностью погашен двумя равными платежами (то есть за \( 2 \) года)?
		\item В июле \( 2016 \) года планируется взять кредит в банке на четыре года в размере \( S \) млн рублей, где \( S \) --- целое число. Условия его возврата таковы:\\		
		--- каждый январь долг увеличивается на \( 15\% \) по сравнению с концом предыдущего года;\\		
		--- с февраля по июнь каждого года необходимо выплатить часть долга;\\		
		--- в июле каждого года долг должен составлять часть кредита в соответствии со следующей таблицей.
		\begin{table}[h]
			\begin{tabular}{llllll}
				\hline
				\multicolumn{1}{|c|}{Месяц и год}         & \multicolumn{1}{c|}{Июль 2016} & \multicolumn{1}{c|}{Июль 2017} & \multicolumn{1}{c|}{Июль 2018} & \multicolumn{1}{c|}{Июль 2019} & \multicolumn{1}{c|}{Июль 2020} \\ \hline
				\multicolumn{1}{|c|}{Долг (в млн рублей)} & \multicolumn{1}{c|}{S}         & \multicolumn{1}{c|}{0,8S}      & \multicolumn{1}{c|}{0,5S}      & \multicolumn{1}{c|}{0,1S}      & \multicolumn{1}{c|}{0}         \\ \hline
			\end{tabular}
		\end{table}
		Найдите наибольшее значение \( S \), при котором общая сумма выплат будет меньше \( 50 \) млн рублей.
	\end{listofex}
\end{homework}
%END_FOLD

%BEGIN_FOLD % ====>>_____ Занятие 7 _____<<====
\begin{class}[number=7]
	\title{Подготовка к проверочной}
	\begin{listofex}
		\item Занятие 7
	\end{listofex}
\end{class}
%END_FOLD

%BEGIN_FOLD % ====>>_ Проверочная работа _<<====
\begin{exam}
	\begin{listofex}
		\item Проверочная
	\end{listofex}
\end{exam}
%END_FOLD