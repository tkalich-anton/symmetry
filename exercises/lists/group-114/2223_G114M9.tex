\setmodule{9}

%BEGIN_FOLD % ====>>_____ Занятие 1 _____<<====
\begin{class}[number=1]
	\begin{listofex}
		\item Занятие 1
	\end{listofex}
\end{class}
%END_FOLD

%BEGIN_FOLD % ====>>_____ Занятие 2 _____<<====
\begin{class}[number=2]
	\begin{listofex}
		\item Занятие 2
	\end{listofex}
\end{class}
%END_FOLD

%BEGIN_FOLD % ====>>_ Домашняя работа 1 _<<====
\begin{homework}[number=1]
	\begin{listofex}
		\item Домашняя работа 1
	\end{listofex}
\end{homework}
%END_FOLD

%BEGIN_FOLD % ====>>_____ Занятие 3 _____<<====
\begin{class}[number=3]
	\begin{listofex}
		\item Занятие 3 
	\end{listofex}
\end{class}
%END_FOLD

%BEGIN_FOLD % ====>>_____ Занятие 4 _____<<====
\begin{class}[number=4]
	\begin{listofex}
		\item Весной катер идёт против течения реки в \(\mfrac{1 }{2 }{ 3}\) раза медленнее, чем по течению. Летом течение становится на \(1\) км/ч медленнее. Поэтому летом катер идёт против течения в \(\mfrac{1 }{1 }{2 }\) раза медленнее, чем по течению. Найдите скорость течения весной (в км/ч).
		\item Каждый из двух рабочих одинаковой квалификации может выполнить заказ за \(15\) часов. Через \(3\) часа после того, как один из них приступил к выполнению заказа, к нему присоединился второй рабочий, и работу над заказом они довели до конца уже вместе. Сколько часов потребовалось на выполнение всего заказа?
		\item Материальная точка движется прямолинейно по закону \(x(t)=\dfrac{ 1 }{ 3 }t^3-3t^2-5t+3\) (где \(x\) --- расстояние от точки отсчета в метрах, \(t\) --- время в секундах, измеренное с начала движения). В какой момент времени (в секундах) ее скорость была равна \(2\) м/с?
		\item В торговом центре два одинаковых автомата продают кофе. Вероятность того, что к концу дня в автомате закончится кофе, равна \(0,35\). Вероятность того, что кофе закончится в обоих автоматах, равна \(0,2\). Найдите вероятность того, что к концу дня кофе останется в обоих автоматах.
		\item Найдите наименьшее значение функции \(y=e^{2x}-6e^x+3\) на отрезке \([1;2]\).
		\item 
		\begin{tasks}
			\task Решите уравнение \( \dfrac{ 16^{\sin^2x}-4^{\sin x} }{ \sqrt{\cos x}-1 }=0 \)
			\task Укажите корни этого уравнения, принадлежащие отрезку \(\left[ -\dfrac{ 3\pi }{ 4 }; \pi \right] \)
		\end{tasks}
		\item Решите неравенство: \(\log_{\log_x 2x}(9x-4) \ge 0 \).
		\item \(15\)-го января планируется взять кредит в банке на сумму \(2,4\) млн рублей на \(24\) месяца. Условия его возврата таковы: \\
		--- \(1\)-го числа каждого месяца долг возрастает на \(3\%\) по сравнению с концом предыдущего месяца; \\
		--- со \(2\)-го по \(14\)-е число каждого месяца необходимо выплатить часть долга; \\
		--- \(15\)-го числа каждого месяца долг должен быть на одну и ту же величину меньше долга на \(15\)-е число предыдущего месяца. \\
		Какую сумму нужно выплатить банку в первые \(12\) месяцев?
		
		\item Найдите все значения a, при каждом из которых уравнение
		\[\dfrac{ x^2-10x+a^2 }{ \sqrt{(a-x+8)(a+x-3)}=0 }\]
		имеет ровно один корень на отрезке \([2; 6]\).
	\end{listofex}
\end{class}
%END_FOLD

%BEGIN_FOLD % ====>>_ Домашняя работа 2 _<<====
\begin{homework}[number=2]
	\begin{listofex}
		\item Домашняя работа 2
	\end{listofex}
\end{homework}
%END_FOLD

%BEGIN_FOLD % ====>>_____ Занятие 5 _____<<====
\begin{class}[number=5]
	\begin{listofex}
		\item Занятие 5
	\end{listofex}
\end{class}
%END_FOLD

%BEGIN_FOLD % ====>>_____ Занятие 6 _____<<====
\begin{class}[number=6]
	\begin{listofex}
		\item Занятие 6
	\end{listofex}
\end{class}
%END_FOLD

%BEGIN_FOLD % ====>>_ Домашняя работа 3 _<<====
\begin{homework}[number=3]
	\begin{listofex}
		\item Домашняя работа 3
	\end{listofex}
\end{homework}
%END_FOLD

%BEGIN_FOLD % ====>>_____ Занятие 7 _____<<====
\begin{class}[number=7]
	\title{Подготовка к проверочной}
	\begin{listofex}
		\item Занятие 7
	\end{listofex}
\end{class}
%END_FOLD

%BEGIN_FOLD % ====>>_ Проверочная работа _<<====
\begin{exam}
	\begin{listofex}
		\item Проверочная
	\end{listofex}
\end{exam}
%END_FOLD