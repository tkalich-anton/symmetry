%
%===============>>  ГРУППА 5-1 МОДУЛЬ 4  <<=============
%
\setmodule{4}
%
%===============>>  Занятие 1  <<===============
%
%\begin{class}[number=1]
%	\begin{listofex}
%		\item Пусто
%	\end{listofex}
%\end{class}
%
%===============>>  Занятие 2  <<===============
%
\begin{class}[number=2]
	\begin{listofex}
		\item Вычислить:
			\begin{enumcols}[itemcolumns=4]
				\item \( \dfrac{1}{2}+\dfrac{5}{6} \)
				\item \( \dfrac{2}{3}-\dfrac{1}{2} \)
				\item \( \dfrac{8}{9}+\dfrac{1}{12} \)
				\item \( \dfrac{5}{8}-\dfrac{1}{20} \)
				\item \( \mfrac{1}{2}{3}+\mfrac{2}{1}{4} \)
				\item \( \mfrac{5}{1}{4}-\mfrac{4}{2}{5} \)
				\item \( \mfrac{10}{8}{21}+\mfrac{3}{13}{21} \)
				\item \( \mfrac{5}{1}{4}-\mfrac{3}{8}{9} \)
			\end{enumcols}
		\item Вычислить:
		\begin{enumcols}[itemcolumns=4]
			\item \( \dfrac{1}{8}\cdot\dfrac{3}{7} \)
			\item \( \dfrac{2}{5}\cdot\dfrac{1}{10} \)
			\item \( \dfrac{3}{10}\cdot\dfrac{10}{3} \)
			\item \( \dfrac{10}{7}\cdot\dfrac{14}{12} \)
			\item \( \mfrac{1}{1}{2}\cdot\mfrac{3}{1}{6} \)
			\item \( \mfrac{2}{4}{17}\cdot\mfrac{2}{5}{16} \)
			\item \( \mfrac{120}{45}{49}\cdot\dfrac{3}{5} \)
			\item \( \dfrac{4}{85}\cdot\mfrac{8}{9}{10} \)
			\end{enumcols}
		\end{listofex}
		\begin{definit}
			Чтобы поделить дробь на целое число, нужно числитель поделить на произведение знаменателя на целое число.
			\[ \dfrac{a}{b}:c=\dfrac{a}{b\cdot c} \]
		\end{definit}
	\begin{listofex}[resume]
		\item Выполнить деление и сократите дробь:
		\begin{enumcols}[itemcolumns=4]
			\item \( \dfrac{4}{5}:2 \)
			\item \( \dfrac{11}{13}:11 \)
			\item \( \dfrac{5}{11}:10 \)
			\item \( \dfrac{20}{27}:5 \)
			\item \( \mfrac{22}{1}{3}:67 \)
			\item \( \mfrac{5}{1}{3}:2 \)
			\item \( \mfrac{14}{2}{7}:3 \)
			\item \( \dfrac{27}{32}:81 \)
		\end{enumcols}
	\end{listofex}
		\begin{definit}
			Чтобы поделить целое число на дробь, нужно целое число умножить на знаменатель и результат поделить на числитель.
			\[ c:\dfrac{a}{b}=\dfrac{c\cdot b}{a} \]
		\end{definit}
	\begin{listofex}[resume]
			\item Выполнить деление и сократите дробь:
			\begin{enumcols}[itemcolumns=5]
				\item \( 1:\dfrac{1}{2} \)
				\item \( 11:\dfrac{1}{13} \)
				\item \( 33:\dfrac{3}{5} \)
				\item \( 77:\dfrac{11}{5} \)
				\item \( 5:\dfrac{10}{25} \)
				\item \( 18:\dfrac{54}{61} \)
				\item \( 24:\dfrac{4}{9} \)
				\item \( 15:\dfrac{5}{7} \)
				\item \( 15:\dfrac{4}{15} \)
				\item \( 10:\dfrac{8}{7} \)
			\end{enumcols}
			\item Выполнить деление и сократите дробь:
		\begin{enumcols}[itemcolumns=4]
			\item \( 2:\mfrac{3}{1}{3} \)
			\item \( 1:\mfrac{1}{1}{2} \)
			\item \( 120:\mfrac{1}{4}{5} \)
			\item \( 100:\mfrac{7}{1}{7} \)
		\end{enumcols}
	\item Вычислить: \( \left( \dfrac{5}{18}+\dfrac{7}{12}+\dfrac{4}{9} \right)\cdot\left( 1-\dfrac{20}{47} \right)\cdot\left( \mfrac{1}{1}{4}-\dfrac{17}{20} \right) \)
	\end{listofex}
	

\end{class}
%
%===============>>  Домашняя работа 1  <<===============
%
\begin{homework}[number=1]
	\begin{listofex}
		\item Проверьте, являются ли числа взаимно обратными?
		\begin{tasks}(4)
			\task \( 11 \) и \( \dfrac{1}{11} \)
			\task \( \mfrac{1}{2}{11} \) и \( \dfrac{11}{13} \)
			\task \( \mfrac{2}{1}{57} \) и \( \dfrac{57}{115} \)
			\task \( \mfrac{10}{10}{11} \) и \( \dfrac{11}{120} \)
		\end{tasks}
		\item Вычислите:
		\begin{tasks}(4)
			\task \( \dfrac{8}{11}:4 \)
			\task \( \dfrac{3}{5}:2 \)
			\task \( \mfrac{4}{2}{3}:7 \)
			\task \( \mfrac{15}{3}{7}:3 \)
			\task \( 20:\dfrac{1}{25} \)
			\task \( 3:\dfrac{1}{19} \)
			\task \( 2:\mfrac{3}{1}{3} \)
			\task \( 120:\mfrac{1}{4}{5} \)
		\end{tasks}
		\item Вычислите:
		\begin{tasks}(4)
			\task \( 2:\mfrac{3}{1}{3} \)
			\task \( 1:\mfrac{1}{1}{2} \)
			\task \( 120:\mfrac{1}{4}{5} \)
			\task \( 100:\mfrac{7}{1}{7} \)
		\end{tasks}
		\item Вычислите:
		\begin{tasks}(4)
			\task \( \mfrac{3}{3}{11}:\dfrac{27}{44} \)
			\task \( \mfrac{14}{1}{2}:\mfrac{4}{1}{9} \)
			\task \( \mfrac{2}{1}{4}:\mfrac{1}{1}{8} \)
			\task \( \mfrac{15}{7}{24}:\mfrac{3}{7}{120} \)
		\end{tasks}
		\item Два велосипедиста выехали одновременно из одного и того же пункта и двигались в одном и том же направлении. Скорость первого велосипедиста \( \mfrac{12}{3}{4} \) км/ч, а скорость второго в \( \mfrac{1}{1}{5} \) раза больше. Какое расстояние будет между ними через \( \mfrac{1}{1}{5} \) ч?
	\end{listofex}
\end{homework}
%
%===============>>  Занятие 3  <<===============
%
\begin{class}[number=3]
	\begin{definit}
		Два числа \( A \) и \( B \), произведение которых равно \( 1 \), называются \textbf{взаимно обратными}, т.е.
		\[ A \cdot B = 1 \]
	\end{definit}
	\begin{listofex}
		\item Проверьте, являются ли числа взаимно обратными?
		\begin{enumcols}[itemcolumns=4]
			\item \( \dfrac{5}{6} \) и \( \mfrac{1}{1}{5} \)
			\item \( \mfrac{3}{2}{3} \) и \( \dfrac{3}{11} \)
			\item \( \mfrac{2}{1}{57} \) и \( \dfrac{57}{115} \)
			\item \( \dfrac{1}{57} \) и \( 57 \)
		\end{enumcols}
		Какую закономерность и способ для определения взаимно обратных чисел можно заметить?
		\item Найдите число, обратное данному:
		\begin{enumcols}[itemcolumns=4]
			\item \( 15 \)
			\item \( \dfrac{1}{4} \)
			\item \( \dfrac{23}{47} \)
			\item \( \mfrac{3}{4}{7} \)
		\end{enumcols}
	\end{listofex}
	\begin{definit}
		Чтобы поделить число \( M \) на число \( P \), можно заменить деление эквивалентным умножением и число \( M \) умножить на число, обратное числу \( P \), то есть:
		\[ M:P=M\cdot\dfrac{1}{P}. \]
		Будем применять это правило в этих случаях:
		\[ 
			\dfrac{a}{b}:\dfrac{d}{e}=\dfrac{a \cdot e}{b \cdot d};
			\qquad
			\dfrac{a}{b}:c=\dfrac{a}{b}\cdot\dfrac{1}{c}=\dfrac{a}{b \cdot c};
			\qquad
			c:\dfrac{a}{b}=c\cdot\dfrac{b}{a}=\dfrac{c \cdot b}{a}.
		\]
	\end{definit}
%	\textbf{Памятка:}
%	\[ 
%		\dfrac{a}{b}\cdot c=\dfrac{a\cdot c}{b};
%		\qquad
%		\dfrac{a}{b}\cdot\dfrac{d}{e}=\dfrac{a\cdot d}{b \cdot e};
%		\qquad
%		c:\dfrac{a}{b}=\dfrac{c \cdot b}{a};
%		\qquad
%		\dfrac{a}{b}:c=\dfrac{a}{b \cdot c};
%		\qquad
%		\dfrac{a}{b}:\dfrac{d}{e}=\dfrac{a \cdot e}{b \cdot e}.
%	\]
	\begin{listofex}
		\item Выполните деление:
		\begin{enumcols}[itemcolumns=4]
			\item \( \mfrac{40}{1}{2}:3 \)
			\item \( \mfrac{9}{2}{3}:2 \)
			\item \( \mfrac{3}{3}{8}:9 \)
			\item \( \dfrac{27}{32}:81 \)
		\end{enumcols}
		\item Выполните деление:
		\begin{enumcols}[itemcolumns=4]
			\item \( 2:3 \)
			\item \( 150:225 \)
			\item \( 21:28 \)
			\item \( 45:20 \)
		\end{enumcols}
		\item Выполните деление:
		\begin{enumcols}[itemcolumns=4]
			\item \( 10:\dfrac{1}{10} \)
			\item \( 20:\dfrac{1}{25} \)
			\item \( 2:\dfrac{2}{3} \)
			\item \( 3:\dfrac{3}{5} \)
			\item \( \mfrac{2}{2}{3}:\dfrac{2}{3} \)
			\item \( 6:\mfrac{1}{1}{2} \)
			\item \( 4:\mfrac{1}{1}{3} \)
			\item \( 18:\dfrac{54}{61} \)
		\end{enumcols}
		\item Выполните деление:
		\begin{enumcols}[itemcolumns=4]
			\item \( \dfrac{2}{3}:\dfrac{5}{7} \)
			\item \( \dfrac{5}{6}:\dfrac{7}{12} \)
			\item \( \dfrac{3}{5}:\dfrac{9}{25} \)
			\item \( \dfrac{15}{16}:\dfrac{3}{10} \)
			\item \( \dfrac{4}{15}:\dfrac{12}{23} \)
			\item \( \mfrac{12}{3}{5}:9 \)
			\item \( \dfrac{27}{64}:18 \)
			\item \( \mfrac{3}{7}{39}:\mfrac{1}{5}{31} \)
		\end{enumcols}
		\item Вычислить:\quad\( 17:\left( \dfrac{3}{5}+\dfrac{1}{4} \right)+\left( \dfrac{7}{8}-\dfrac{1}{4} \right)\cdot\left( \dfrac{4}{5} \right)^2 \)
	\end{listofex}
\end{class}
%
%===============>>  Занятие 4  <<===============
\begin{class}[number=4]
	\begin{listofex}
		\item Выполните деление:
		\begin{enumcols}[itemcolumns=4]
			\item \( \dfrac{2}{3}:\dfrac{5}{7} \)
			\item \( \dfrac{5}{6}:\dfrac{7}{12} \)
			\item \( \dfrac{3}{5}:\dfrac{9}{25} \)
			\item \( \dfrac{15}{16}:\dfrac{3}{10} \)
			\item \( \dfrac{4}{15}:\dfrac{12}{23} \)
			\item \( \mfrac{12}{3}{5}:9 \)
			\item \( \dfrac{27}{64}:18 \)
			\item \( \mfrac{3}{7}{39}:\mfrac{1}{5}{31} \)
		\end{enumcols}
		\item Вычислить: \( 17:\left( \dfrac{3}{5}+\dfrac{1}{4} \right)+\left( \dfrac{7}{8}-\dfrac{1}{4} \right)\cdot\left( \dfrac{4}{5} \right)^2 \)
		\item С какой скоростью нужно ехать, чтобы преодолеть \( \mfrac{6}{8}{12} \) км за \( \dfrac{1}{3} \) часа?
		\item Чему равна площадь комнаты, размеры которой \( \mfrac{5}{1}{2} \) м и \( \mfrac{3}{1}{2} \) м?
		\item За \( 1 \) ч велосипедист проезжает \( 12 \) км. Сколько километров он проедет за \( \dfrac{1}{2} \) ч, \( \dfrac{1}{3} \) ч, \( \dfrac{3}{4} \) ч, \( 3 \) ч, \( \mfrac{2}{1}{3} \) ч, \( \mfrac{1}{1}{4} \) ч, \( \mfrac{3}{3}{4} \) ч?
		\item Найдите массу металлической детали, объём которой равен \( \mfrac{3}{1}{3} \)дм\( ^3 \), если масса \( 1 \) дм\( ^3\) этого металла равна \( \mfrac{7}{4}{5} \) кг.
		\item Вычислите:
		\begin{enumcols}[itemcolumns=2]
			\item \( \left( \dfrac{3}{4}+\dfrac{1}{6} \right)\cdot3+\left( \dfrac{5}{6}-\dfrac{1}{2} \right):\dfrac{2}{9} \)
			\item \( \left( \mfrac{6}{1}{2}-\mfrac{4}{1}{4} \right):\mfrac{2}{1}{2} \)
			\item \( \dfrac{9}{10}\cdot\mfrac{1}{1}{14}:\mfrac{2}{4}{7}\cdot24-\mfrac{2}{4}{15}:\left( \mfrac{1}{1}{5}-\dfrac{2}{3} \right) \)
		\end{enumcols}
	\end{listofex}
\end{class}

%
%===============>>  Домашняя работа 2  <<===============
%
%\begin{homework}[number=2]
%	\begin{listofex}
%
%	\end{listofex}
%\end{homework}

%
%===============>>  Проверочная работа  <<===============
%
\begin{exam}
	\title{Проверочная работа}
	\begin{listofex}
		\item Вычислите:
		\begin{tasks}(4)
			\task \( 4+\dfrac{1}{2} \)
			\task \( \dfrac{2}{5}+\dfrac{9}{10} \)
			\task \( \dfrac{97}{100}-\dfrac{3}{50} \)
			\task \( \mfrac{5}{3}{4}-\mfrac{1}{3}{8} \)
		\end{tasks}
		\item Вычислите:
		\begin{tasks}(4)
			\task \( \dfrac{2}{3}\cdot\dfrac{5}{6} \)
			\task \( \dfrac{12}{63}\cdot\dfrac{21}{24} \)
			\task \( \mfrac{5}{1}{2}\cdot\mfrac{3}{2}{7} \)
			\task \( \mfrac{100}{1}{20}\cdot\mfrac{2}{198}{2001} \)
		\end{tasks}
		\item Вычислите:
		\begin{tasks}(2)
			\task \( \dfrac{5}{12}:\dfrac{5}{4} \)
			\task \( \mfrac{1}{3}{7}:\dfrac{5}{7} \)
		\end{tasks}
		\item Найти:
		\begin{tasks}(2)
			\task \( \dfrac{3}{5} \) от \( 75 \)
			\task \( \dfrac{6}{7} \) от \( \dfrac{21}{8} \)
		\end{tasks}
		\item Являются ли следующие числа взаимно обратными? Почему?
		\begin{tasks}(5)
			\task \( 7 \) и \( \dfrac{1}{7} \)
			\task \( \dfrac{6}{7} \) и \( \dfrac{14}{12} \)
			\task \( \dfrac{1}{2} \) и \( \dfrac{2}{4} \)
			\task \( \mfrac{2}{1}{6} \) и \( \dfrac{6}{13} \)
			\task \( \mfrac{1}{4}{5} \) и \( \dfrac{25}{45} \)
		\end{tasks}
		\item Сколько времени затратил пешеход, который прошёл \( \mfrac{5}{1}{2} \) км со скоростью \( \mfrac{4}{2}{5} \) км/ч?
		\item Вычислить: \( \left( \cfrac{3}{20}\cdot\left( \cfrac{7}{12}-\cfrac{1}{2} \right)+\cfrac{79}{80} \right):\left( \cfrac{13}{24}:\left( \cfrac{7}{12}+\cfrac{1}{2} \right)-\cfrac{1}{4} \right) \)
	\end{listofex}
\end{exam}

%
%===============>>  Занятие 6  <<===============
%
\begin{class}[number=6]
	\begin{listofex}
		\item Начертите:
		\begin{enumcols}[itemcolumns=1]
		\item \( 3 \) разных острых угла в тетради
		\item \( 3 \) разных тупых угла в тетради
		\item прямой угол в тетради
		\item развёрнутый угол в тетради
		\item угол, градусная величина которого равна \( 75\degree \)
		\item угол, градусная величина которого равна \( 186\degree \)
			\end{enumcols}
		\item Какой угол образуют часовая и минутная стрелка
		\begin{enumcols}[itemcolumns=1]
			\item в \( 3 \) часа утра
			\item в \( 20 \) часов
			\item в \( 18 \) часов \( 30 \) минут
		\end{enumcols}
	\item Начертите угол \( ABC=30\degree \). Проведите луч \( BD \) так, чтобы 
	\begin{enumcols}[itemcolumns=1]
		\item угол \( ABD \) была равен \( 90\degree \), а угол \( CBD \) равен \( 120\degree \)
		\item угол \( ABD \) была равен \( 90\degree \), а угол \( CBD \) равен \( 60\degree \)
	\end{enumcols}
	\item За \( 1 \) ч велосипедист проезжает \( 12 \) км. Сколько километров он проедет за \( \dfrac{1}{2} \) ч, \( \dfrac{1}{3} \) ч, \( \dfrac{3}{4} \) ч, \( 3 \) ч, \( \mfrac{2}{1}{3} \) ч, \( \mfrac{1}{1}{4} \) ч, \( \mfrac{3}{3}{4} \) ч?
	\item Найдите массу металлической детали, объём которой равен \( \mfrac{3}{1}{3} \)дм\( ^3 \), если масса \( 1 \) дм\( ^3\) этого металла равна \( \mfrac{7}{4}{5} \) кг.
	\item Вычислите:
	\begin{enumcols}[itemcolumns=2]
		\item \( \left( \dfrac{3}{4}+\dfrac{1}{6} \right)\cdot3+\left( \dfrac{5}{6}-\dfrac{1}{2} \right):\dfrac{2}{9} \)
		\item \( \left( \mfrac{6}{1}{2}-\mfrac{4}{1}{4} \right):\mfrac{2}{1}{2} \)
		\item \( \dfrac{9}{10}\cdot\mfrac{1}{1}{14}:\mfrac{2}{4}{7}\cdot24-\mfrac{2}{4}{15}:\left( \mfrac{1}{1}{5}-\dfrac{2}{3} \right) \)
	\end{enumcols}
	\end{listofex}
\end{class}
%
%===============>>  Домашняя работа 3  <<===============
%
%\begin{homework}[number=2]
%	\begin{listofex}
%
%	\end{listofex}
%\end{homework}

%
%===============>>  Занятие 7  <<===============
%
%\begin{class}[number=7]
%	\begin{listofex}
%	
%	\end{listofex}
%\end{class}
<<<<<<< HEAD
%===============>>  Проверочная работа  <<===============
%
\begin{exam}
	\title{Проверочная работа}
	\begin{listofex}
		\item Вычислите:
		\begin{enumcols}[itemcolumns=4]
			\item \( 4+\dfrac{1}{2} \)
			\item \( \dfrac{2}{5}+\dfrac{9}{10} \)
			\item \( \mfrac{5}{3}{4}-\mfrac{1}{3}{8} \)
			\item \( \dfrac{97}{100}-\dfrac{3}{50} \)
		\end{enumcols}
		\item Вычислите:
		\begin{enumcols}[itemcolumns=4]
			\item \( \dfrac{2}{3}\cdot\dfrac{5}{6} \)
			\item \( \dfrac{12}{63}\cdot\dfrac{21}{24} \)
			\item \( \mfrac{5}{1}{2}\cdot\mfrac{3}{2}{7} \)
			\item \( \mfrac{100}{1}{20}\cdot\mfrac{2}{198}{2001} \)
		\end{enumcols}
			\item Являются ли следующие числа взаимно обратными? Почему?
			\begin{enumcols}[itemcolumns=6]
				\item \( 7 \) и \( \dfrac{1}{7} \)
				\item \( \dfrac{6}{7} \) и \( \dfrac{14}{12} \)
				\item \( \dfrac{1}{2} \) и \( \dfrac{2}{4} \)
				\item \( \mfrac{2}{1}{6} \) и \( \dfrac{6}{13} \)
				\item \( \mfrac{1}{4}{5} \) и \( \dfrac{25}{45} \)
				\item \( \mfrac{7}{1}{2} \) и \( \mfrac{2}{1}{7} \)
			\end{enumcols}			
			\item Сколько времени затратил пешеход, который прошёл \( \mfrac{5}{1}{2} \) км со скоростью \( \mfrac{4}{2}{5} \) км/ч?
			\item Вычислить: \( \left( \cfrac{3}{20}\cdot\left( \cfrac{7}{12}-\cfrac{1}{2} \right)+\cfrac{79}{80} \right):\left( \cfrac{13}{24}:\left( \cfrac{7}{12}+\cfrac{1}{2} \right)-\cfrac{1}{4} \right) \)
	\end{listofex}
\end{exam}
		\newpage
		\title{Консультация}
	\begin{listofex}
			\item Товарный поезд прошёл \( \mfrac{31}{1}{2} \) км за \( \dfrac{3}{4} \) часа. Какова его часовая скорость?
			\item Куплено \( \mfrac{7}{1}{2} \) м ткани по \( \mfrac{5}{1}{2} \) руб. и \( 5 \) м по \( \mfrac{3}{3}{4} \) за \( 1 \) м. Сколько стоит вся покупка?
			\item Группа туристов совершила переход в \( 27 \) км за \( \mfrac{6}{3}{4} \) часа. Какую часть пути проходили туристы за \( 1 \) час и сколько километров проходили они за \( 1 \) час?
			\item Лес, пашня и луга занимают \( 600 \) га. Из них лес занимает \( \dfrac{1}{5} \) всей земли, пашня -- \( \dfrac{2}{3} \), остальное -- луга. Сколько гектаров занимают в отдельности лес, пашня и луга?
			\item На сколько участков можно разбить огород в \( \mfrac{1}{3}{4} \) га, если в каждом участке должно быть по \( \dfrac{1}{8} \) га?
			\item За сколько времени можно пройти \( \mfrac{7}{1}{8} \) км, если идти со скоростью \( \mfrac{4}{3}{4} \) км/ч?
			\item Трава при высыхании потеряла \( \dfrac{2}{3} \) своего веса. Сколько сена получили из \( \mfrac{7}{1}{4} \) т травы?
	\end{listofex}
=======
>>>>>>> daf1f12962b69f918e5d198977b665f188bb0943
