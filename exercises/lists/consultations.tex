%Индивидуальные консультации
\title{Консультация}
Некое утверждение будет справедливым для натурального значения \( n \) тогда, когда

1) Оно будет верно при \( n=1 \) (база индукции)

2) из того, что это выражение справедливо для произвольного натурального \( n=k \), следует, что оно будет верно и при \( n=k+1 \) (шаг индукции)
\begin{listofex}
	\item Докажите методом математической индукции:
	\begin{enumcols}[itemcolumns=2]
		\item \( 1+2+3+\dots+n=\dfrac{n(n+1)}{2} \)
		\item \( 1+3+5+\dots+(2n-1)=n^2 \)
		\item \( 2+4+6+\dots+2n=n(n+1) \)
		\item \( 1^2+2^2+3^2+\dots+n^2=\dfrac{n(n+1)(2n+1)}{6} \)
		\item \( 3+12+\dots+3\cdot4^{n-1}=4^n-1 \)
	\end{enumcols}
	
	
\end{listofex}