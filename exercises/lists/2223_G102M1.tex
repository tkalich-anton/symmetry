%Группа 10-2 Модуль 1
\title{Занятие №1}
\begin{listofex}
	\item \exercise{1402}
	\item Из формулы \( \dfrac{1}{F}=\dfrac{1}{f}+\dfrac{1}{d} \) выразить: а) \( F \); б) \( d \)
	\item Из формулы \( F=\gamma\cdot\dfrac{m_1m_2}{r^2} \) выразить \( r \). Все величины положительны.
	\item Найти значение выражения \( x^2+\dfrac{1}{x^2} \), если известно, что \( x-\dfrac{1}{x}=5 \)
	\item \exercise{645}
	\item \exercise{646}
	\item Упростить выражение \( \dfrac{p\cdot q}{p+q}\cdot\left( \dfrac{q}{p}-\dfrac{p}{q} \right) \) и найдите значение выражения, если \( p=3-2\sqrt{2} \) и \( q=-2\sqrt{2} \)
	\item Вычислить:
	\begin{enumcols}[itemcolumns=3]
		\item \( \sqrt{77\cdot24\cdot33\cdot14} \)
		\item \( \sqrt{21}\cdot\sqrt{3\dfrac{6}{7}} \)
		\item \( \dfrac{(3\sqrt{5})^2}{15} \)
	\end{enumcols}
	\item Найти значение выражения \( 3x^2-2x-1 \), если \( x=\dfrac{1-\sqrt{2}}{3} \)
	\item Упростить выражение:
	\begin{enumcols}[itemcolumns=2]
		\item \( \dfrac{a}{a-1}-\dfrac{\sqrt{a}}{\sqrt{a}+1} \)
		\item \( \left( \dfrac{\sqrt{a}-5}{\sqrt{a}+5}+\dfrac{20\sqrt{a}}{a-25} \right):\dfrac{\sqrt{a}+5}{a-5\sqrt{a}} \)
	\end{enumcols}

	\item Известно, что \( \sqrt{8-x}+\sqrt{x+3}=4 \). Найдите значение выражения \( \sqrt{(8-x)(x+3)} \)
	
	\item Найдите три последовательных натуральных числа, если удвоенный квадрат
	первого из них на \( 26 \) больше произведения второго и третьего чисел.
\end{listofex}
\newpage
\title{Занятие №2}
\begin{listofex}
	\item \exercise{1420}
	\item Из формулы \( S_n=\dfrac{2a_1+d(n+1)}{2}\cdot n \) выразить: а) \( a_1 \); б) \( d \)
	\item Из формулы \( P=\dfrac{U^2}{R} \) выразить \( U \). Все величины положительны.
	\item Найти значение выражения \( 4x^2+\dfrac{1}{x^2} \), если известно, что \( 2x+\dfrac{1}{x}=7 \)
	\item \exercise{972}
	\item \exercise{975}
	\item Вычислить:
	\begin{enumcols}[itemcolumns=3]
		\item \( \sqrt{5\cdot6\cdot8\cdot20\cdot27} \)
		\item \( \sqrt{15}\cdot\sqrt{6\dfrac{2}{3}} \)
		\item \( \dfrac{6}{(2\sqrt{3})^2} \)
	\end{enumcols}
	\item Найти значение выражения \( 2x^2-6x+3 \), если \( x=\dfrac{3-\sqrt{5}}{2} \)
	\item Упростить выражение:
	\begin{enumcols}[itemcolumns=2]
		\item \( \dfrac{c}{c-4}-\dfrac{\sqrt{c}}{\sqrt{c}-2} \)
		\item \( \left( \dfrac{\sqrt{y}+7}{\sqrt{y}-7}-\dfrac{28\sqrt{y}}{y-49} \right):\dfrac{\sqrt{y}-7}{y+7\sqrt{y}} \)
	\end{enumcols}
	\item Известно, что \( \sqrt{y-1}+\sqrt{8-y}=2 \). Найдите значение выражения \( \sqrt{(y-1)(8-y)} \)
	\item Найдите четыре последовательных нечетных натуральных числа, если удвоенное
	произведение второго и третьего чисел на \( 107 \) больше произведения первого и четвертого
	чисел.
\end{listofex}
\newpage
\title{Домашняя работа №1}
\begin{listofex}
	\item Упростить выражение:
	\begin{enumcols}[itemcolumns=1]
		\item \exercise{1471}
		\item \exercise{1431}
	\end{enumcols}
	\item Из формулы \( S=\dfrac{abc}{4R} \) выразить: а) \( c \); б) \( R \)
	\item Из формулы \( Q=I^2Rt \) выразить \( I \). Все величины положительны.
	\item Найти значение выражения \( 25x^2+\dfrac{1}{x^2} \), если известно, что \( 5x+\dfrac{1}{x}=4 \)
	\item \exercise{974}
	\item \exercise{976}
	\item Вычислить:
	\begin{enumcols}[itemcolumns=3]
		\item \( \sqrt{21\cdot65\cdot39\cdot35} \)
		\item \( \sqrt{12}\cdot\sqrt{5\dfrac{1}{3}} \)
		\item \( \dfrac{(5\sqrt{7})^2}{35} \)
	\end{enumcols}
	\item Найти значение выражения \( a^2-6\sqrt{5}-1 \), если \( a=\sqrt{5}+4 \)
	\item Упростить выражение:
	\begin{enumcols}[itemcolumns=2]
		\item \( \dfrac{x}{x-16}-\dfrac{\sqrt{x}}{\sqrt{x}+4} \)
		\item \( \left( \dfrac{\sqrt{m}-2}{\sqrt{m}+2}+\dfrac{8\sqrt{m}}{m-4} \right):\dfrac{\sqrt{m}+2}{m-2\sqrt{m}} \)
	\end{enumcols}
	\item Известно, что \( \sqrt{7-x}+\sqrt{x-2}=3 \). Найдите значение выражения \( \sqrt{(7-x)(x-2)} \)
	\item Найдите три последовательных натуральных числа, если удвоенный квадрат второго из
	них на \( 56 \) меньше удвоенного произведения первого и третьего чисел.
\end{listofex}
\newpage
\title{Занятие №3}
\begin{listofex}
	\item Упростить выражение:
	\begin{enumcols}[itemcolumns=1]
		\item \exercise{1456}
		\item \exercise{1453}
	\end{enumcols}
	\item Вычислить:
	\begin{enumcols}[itemcolumns=3]
		\item \( \dfrac{7!}{5!} \)
		\item \( \dfrac{2000!}{1999!} \)
		\item \( \dfrac{5!+6!+7!}{8!-7!} \)
	\end{enumcols}
	\item Докажите, что для любого натурального \( n \) верно равенство:
	\begin{enumcols}[itemcolumns=2]
		\item \( n!+(n+1)! = n!(n+2) \)
		\item \( (n-1)!+n!+(n+1)! = (n+1)^2(n-1)! \)
	\end{enumcols}
	\item Запишите в виде дроби:
	\begin{enumcols}[itemcolumns=2]
		\item \( \dfrac{1}{(n+1)!}-\dfrac{n^2+5n}{(n+3)!} \)
		\item \( \dfrac{1}{(k-1)!}-\dfrac{k}{(k+1)!} \)
	\end{enumcols}
	\item Множество, состоящее из шести элементов \( A_1 \), \( A_2 \), \( A_3 \), \( A_4 \), \( A_5 \), \( A_6 \), упорядочили всеми возможными способами. Сколько таких способов? В скольких случаях:
	\begin{enumcols}
		\item элемент \( A_1 \) будет первым по порядку;
		\item элемент \( A_1 \) не будет ни первым ни последним;
		\item элемент \( A_1 \) будет первым, а \( A_6 \) будет последним.
	\end{enumcols}
	\item Сколькими различными способами можно усадить в ряд трех мальчиков и трех девочек так, чтобы никакие два мальчика и никакие две девочки не оказались рядом?
	\item Вычислить \( P_{12}:P_{10} \)
\end{listofex}
\newpage
\title{Занятие №4}
\begin{listofex}
	\item Упростить выражение:
	\begin{enumcols}[itemcolumns=2]
		\item \exercise{1433}
		\item \exercise{1383}
	\end{enumcols}
	\item Вычислить:
	\begin{enumcols}[itemcolumns=3]
		\item \( \dfrac{8!}{5!} \)\answer{\( 336 \)}
		\item \( \dfrac{500!}{498!} \)\answer{\( 249500 \)}
		\item \( \dfrac{3!+5!+6!}{141\cdot4!-282\cdot3!} \)\answer{\( 0,5 \)}
	\end{enumcols}
	\item Докажите, что для любого натурального \( n \) верно равенство:
	\begin{enumcols}[itemcolumns=2]
		\item \( (n+1)!-n! = n!n \)
		\item \( \dfrac{(n+1)!}{(n-1)!}=n^2+n \)
	\end{enumcols}
	\item \exercise{1864}
	\item Вычислить:
	\begin{enumcols}[itemcolumns=2]
		\item \exercise{1098}
		\item \exercise{1775}
		\item \exercise{1756}
	\end{enumcols}
	\item Вычислить:
	\begin{enumcols}[itemcolumns=2]
		\item \exercise{1743}
		\item \exercise{1759}
	\end{enumcols}
	\item Решить уравнения:
	\begin{enumcols}[itemcolumns=2]
		\item \exercise{1003}
		\item \exercise{996}
	\end{enumcols}
\end{listofex}
\newpage
\title{Домашняя работа №2}
\begin{listofex}
	\item Упростить выражение:
	\begin{enumcols}[itemcolumns=2]
		\item \exercise{1472}
		\item \exercise{1512}
		\item \exercise{1467}
	\end{enumcols}
	\item Вычислить:
	\begin{enumcols}[itemcolumns=3]
		\item \( \dfrac{20!}{22!} \) \answer{\( \dfrac{1}{462} \)}
		\item \( \dfrac{15!}{10!\cdot5!} \) \answer{\( 3003 \)}
		\item \( \dfrac{18!-17\cdot17!-16\cdot16!}{17!-16!} \) \answer{\( \dfrac{579}{16} \)}
	\end{enumcols}
	\item Докажите, что для любого натурального \( n \) верно равенство:
	\begin{enumcols}[itemcolumns=2]
		\item \( (n+1)!-n!+(n-1)! = (n^2+1)(n-1)! \)
		\item \( \dfrac{(n-1)!}{n!}-\dfrac{n!}{(n+1)!}=\dfrac{1}{n(n+1)} \)
	\end{enumcols}
	\item \exercise{1865}
	\item Вычислить:
	\begin{enumcols}[itemcolumns=2]
		\item \exercise{1215}
		\item \exercise{1776}
		\item \exercise{1765}
		\item \exercise{1686}
	\end{enumcols}
	\item Вычислить:
	\begin{enumcols}[itemcolumns=2]
		\item \exercise{1741}
		\item \exercise{1758}
	\end{enumcols}
	\item Решить уравнения:
	\begin{enumcols}[itemcolumns=2]
		\item \exercise{997}
		\item \exercise{1003}
	\end{enumcols}
\end{listofex}
\newpage
\title{Занятие №5}
\begin{listofex}
	\item За круглый стол на \( 9 \) стульев в случайном порядке рассаживаются \( 7 \) мальчиков и \( 2 \) девочки. Найдите вероятность того, что обе девочки будут сидеть рядом.
	\item Сколько пятизначных чисел можно составить из цифр от 1 до 5? От 1 до 6? От 1 до 7? Так, чтобы цифры в числе не повторялись.
	\item Решить уравнение:
	\begin{enumcols}[itemcolumns=2]
		\item \exercise{3588}
		\item \exercise{3595}
		\item \exercise{3623}
		\item \exercise{3585}
	\end{enumcols}
	\item Решить уравнение:
	\begin{enumcols}[itemcolumns=3]
		\item \exercise{455}
		\item \exercise{468}
		\item \exercise{477}
	\end{enumcols}
	\item Решить уравнение:
	\begin{enumcols}[itemcolumns=2]
		\item \exercise{491}
		\item \exercise{496}
	\end{enumcols}
	\item Решить уравнение:
	\begin{enumcols}[itemcolumns=3]
		\item \exercise{41}
		\item \exercise{543}
		\item \exercise{549}
	\end{enumcols}
	\item Решить уравнение:
	\begin{enumcols}[itemcolumns=2]
		\item \exercise{509}
		\item \exercise{498}
	\end{enumcols}
	\item Решить уравнение:
	\begin{enumcols}[itemcolumns=3]
		\item \exercise{32}
		\item \exercise{16}
		\item \exercise{3670}
	\end{enumcols}
	\item Решить уравнение:
	\begin{enumcols}[itemcolumns=2]
		\item \exercise{3627}
		\item \exercise{3629}
	\end{enumcols}
\end{listofex}