%Группа 9-1 Модуль 2
\title{Занятие №1}
\begin{listofex}
	\item \exercise{2350}
	\item \exercise{2095}
	\item \exercise{2390}
	\item \exercise{2380}
	\item \exercise{2385}
	\item \exercise{2381}
	\item \exercise{2393}
	\item \exercise{2559}
	\item \exercise{2040}
\end{listofex}
\newpage
\title{Занятие №2}
\begin{listofex}
	\item Докажите, что если медиана равна половине стороны, к которой она проведена, то треугольник прямоугольный.
	\item \exercise{2415}
	\item Докажите, что если треугольник вписан в окружность и одна из его сторон является диаметром этой окружности, то такой треугольник является прямоугольным.
	\item Докажите обратное, что если треугольник прямоугольный и вписан в окружность, то гипотенуза будет являться диаметром окружности.
	\item \exercise{2455}
	\item \exercise{2418}
	\item \exercise{2424}
\end{listofex}
\newpage
\title{Домашняя работа №1}
\begin{listofex}
	\item Упростить выражение:
	\begin{enumcols}[itemcolumns=2]
		\item \exercise{748}
		\item \exercise{1500}
	\end{enumcols}
	\item Вычислить:
	\begin{enumcols}[itemcolumns=2]
		\item \exercise{1649}
		\item \exercise{1682}
	\end{enumcols}
	\item Докажите, что в равных треугольниках соответствующие биссектрисы равны.
	\item \exercise{2423}
	\item \exercise{2456}
	\item \exercise{2412}
\end{listofex}
\newpage
\title{Занятие №3}
\begin{listofex}
	\item Докажите следующие свойства окружности:
	\begin{enumcols}[itemcolumns=1]
		\item диаметр, перпендикулярный хорде, делит ее пополам;
		\item диаметр, проходящий через середину хорды, не являющейся диаметром, перпендикулярен этой хорде;
		\item хорды, удаленные от центра окружности на равные расстояния, равны.
	\end{enumcols}
	\item \exercise{2437}
	\item \exercise{2439}
	\item \exercise{2442}
	\item \exercise{2444}
	\item \exercise{2445}
	\item \exercise{2422}
	\item \exercise{2420}
\end{listofex}
\newpage
\title{Занятие №4}
\begin{listofex}
	\item Внутренние углы треугольника \( ABC \) относятся как \( 10:5:3 \). Найдите внутренние и внешние углы треугольника \( ABC \) и вычислите разницу самого наибольшего и наименьшего внешних углов. \answer{ внутренние:\( 100;\;50;\;30 \), внешние: \( 80;\;130;\;100; \), разница: \( 50 \) }
	\item В треугольнике \( ABC \) углы \( B \) и \( C \) равны \( 30 \) и \( 40 \) соответственно. Сторону \( AB \) продлили за вершину \( A \) и из это вершины провели высоту и биссектрису внешнего угла. Найдите угол между высотой и биссектрисой. \answer{ \( 85 \) }
	\item Две параллельные прямые пересечены третьей. Найдите угол между биссектрисами внутренних односторонних углов.
	\item \exercise{2438}
	\item \exercise{2441}
	\item \exercise{2354}
	\item \exercise{2514}
	\item Решить уравнение:
	\begin{enumcols}[itemcolumns=2]
		\item \exercise{1015}
		\item \exercise{1036}
	\end{enumcols}
\end{listofex}
\newpage
\title{Домашняя работа №2}
\begin{listofex}
		\item \exercise{2436}
	\item \exercise{2440}
	\item \exercise{2454}
	\item \exercise{2457}
	\item \exercise{2459}
	\item \exercise{2412}
	\item В треугольнике \( ABC \) угол \( \angle B = 80 \). Найдите угол между высотами проведенными из двух других углов. \answer{ \( 100 \) }
	\item Решить уравнение:
	\begin{enumcols}[itemcolumns=2]
		\item \exercise{3391}
		\item \exercise{1034}
\end{enumcols}
\end{listofex}
\newpage
\title{Занятие №5}
\begin{listofex}
	\item \exercise{2472}
	\item \exercise{2473}
	\item \exercise{2474}
	\item \exercise{2476}
	\item \exercise{2483}
	\item \exercise{2493}
	\item \exercise{1608}
	\item \exercise{3664}
\end{listofex}
\newpage
\title{Занятие №6}
\begin{listofex}
	\item \exercise{2477}
	\item \exercise{2481}
	\item \exercise{2484}
	\item \exercise{2486}
	\item \exercise{2479}
	\item \exercise{2500}
	\item \exercise{2502}
	\item \exercise{1115}
\end{listofex}
\newpage
\title{Домашняя работа №3}
\begin{listofex}
	\item \exercise{2475}
	\item \exercise{2478}
	\item \exercise{2485}
	\item \exercise{2480}
\end{listofex}
%\newpage
%\title{Занятие №7}
%\begin{listofex}
%
%\end{listofex}
%\newpage
%\title{Проверочная работа}
%\begin{listofex}
%
%\end{listofex}
\newpage
\title{Консультация}
\begin{listofex}
	\item Построить график функции:
	\begin{enumcols}[itemcolumns=2]
		\item \( y=x^2 \)
		\item \( y=(x+2)^2 \)
		\item \( y=(x-3)^2-2 \)
		\item \( y=x^2-4x+6 \)
	\end{enumcols}
	\item Найдите координаты точек пересечения прямой \( y=2x-7 \) и параболы \( y=x^2+8x+1 \)
	\item Построить график функции:\quad\( y=\dfrac{(x+3)^2(x-1)}{x+3}-2 \)
\end{listofex}
\newpage
\title{Консультация}
\begin{listofex}
	\item Построить график функции:
	\begin{enumcols}[itemcolumns=2]
		\item \( y=(x+2)^2 \)
		\item \( y=2x^3-1 \)
		\item \( y=\sqrt{x+3}-1 \)
		\item \( y=\dfrac{1}{x+2}+1 \)
		\item \( y=x^2-4x+6 \)
	\end{enumcols}
	\item Построить график функции, представив ее в виде \( y=(x-d)^3+h \):
	\begin{enumcols}[itemcolumns=2]
		\item \( y=x^3+3x^2+3x+1 \)
		\item \( y=x^3+3x^2+3x+3 \)
		\item \( y=x^3-9x^2+27x-24 \)
	\end{enumcols}
	\item Построить график функции:\quad\( y=\dfrac{x+3}{(x+3)(x-1)}-2 \)
\end{listofex}