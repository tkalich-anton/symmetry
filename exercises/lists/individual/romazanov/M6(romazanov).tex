%
%===============>>  Ромазанов Модуль 6 <<=============
%
\setmodule{6}

%BEGIN_FOLD % ====>>_____ Занятие 1 _____<<====
\begin{class}[number=1]
	\begin{listofex}
		\item В прямоугольном треугольнике \( \angle C=90\degree \), \( AB=8 \), \( \angle A=30\degree \). Найдите угол \( B \). Чему будет равен катет \( CB \)?
		\item В прямоугольном треугольнике \( ABC \) катет \( AС \) равен половине гипотенузы \( AB \). Найдите градусную меру всех углов треугольника.
		\item Известно, что в прямоугольном треугольнике (\( \angle C=90\degree \)) угол \( B=60\degree \). Найдите прилежащий к этому углу катет, если гипотенуза равна \( 50 \).
		\item Найдите площадь квадрата, если его сторона равна \( 6 \).
		\item Найдите площадь параллелограмма, если одна из его сторон равна \( 8 \), а опущенная на неё высота равна \( 4 \).
		\item Найдите площадь ромба, если одна из его сторон равна \( 3 \), а высота равна \( 2 \).
		\item Стороны квадрата и ромба равны. Известно, что площадь квадрата равна \( 64 \), а острый угол ромба равен \( 30\degree \). Найдите площадь ромба.
	\end{listofex}
\end{class}
%END_FOLD

%BEGIN_FOLD % ====>>_____ Занятие 2 _____<<====
\begin{class}[number=2]
	\begin{listofex}
		\item Сформулируйте три признака равенства треугольников.
		\item Медиана треугольника делит его на два треугольника, периметры которых равны. Докажите, что треугольник равнобедренный.
		\item Докажите, что в равных треугольниках соответствующие медианы равны.
		\item Докажите, что биссектриса равнобедренного треугольника, проведенная из вершины, является также медианой и высотой.
		\item Медиана треугольника является также его высотой. Докажите, что такой треугольник равнобедренный.
	\end{listofex}
	\newpage
	\title{Домашняя работа}
	\begin{listofex}
		\item Докажите, что в равных треугольниках соответствующие биссектрисы равны.
		\item Медиана треугольника является также его высотой. Докажите, что такой треугольник равнобедренный.
	\end{listofex}
\end{class}
%END_FOLD

%BEGIN_FOLD % ====>>_ Домашняя работа 1 _<<====
\begin{homework}[number=1]
	\begin{listofex}
		\item В прямоугольном треугольнике \( ABC \) \( \angle C=90\degree \), \( BC=4 \), \( \angle A=30\degree \). Найдите гипотенузу.
		\item В прямоугольном треугольнике один угол равен \( 60\degree \). Чем будет равен прилежащий к этому углу катет, если гипотенуза равна \( 12 \)?
		\item Найдите площадь параллелограмма, если одна из его сторон равна \( 30 \), а высота, опущенная на неё, равна \( 20 \).
		\item Стороны квадрата и ромба равны. Известно, что площадь квадрата равна \( 36 \), а острый угол ромба равен \( 30\degree \). Найдите площадь ромба.
	\end{listofex}
\end{homework}
%END_FOLD

%BEGIN_FOLD % ====>>_____ Занятие 3 _____<<====
\begin{class}[number=3]
	\begin{listofex}
		\item Занятие 3 
	\end{listofex}
\end{class}
%END_FOLD

%BEGIN_FOLD % ====>>_____ Занятие 4 _____<<====
\begin{class}[number=4]
	\begin{listofex}
		\item Занятие 4
	\end{listofex}
\end{class}
%END_FOLD

%BEGIN_FOLD % ====>>_ Домашняя работа 2 _<<====
\begin{homework}[number=2]
	\begin{listofex}
		\item Докажите, что в равных треугольниках соответствующие биссектрисы равны.
		\item Медиана треугольника является также его высотой. Докажите, что такой треугольник равнобедренный.
	\end{listofex}
\end{homework}
%END_FOLD

%BEGIN_FOLD % ====>>_____ Занятие 5 _____<<====
\begin{class}[number=5]
	\begin{listofex}
		\item Занятие 5
	\end{listofex}
	\end{class}
	%END_FOLD
	
	%BEGIN_FOLD % ====>>_____ Занятие 6 _____<<====
	\begin{class}[number=6]
		\begin{listofex}
			\item Занятие 6
		\end{listofex}
	\end{class}
	%END_FOLD
	
	%BEGIN_FOLD % ====>>_ Домашняя работа 3 _<<====
	\begin{homework}[number=3]
		\begin{listofex}
			\item Домашняя работа 3
		\end{listofex}
	\end{homework}
	%END_FOLD
	
	%BEGIN_FOLD % ====>>_____ Занятие 7 _____<<====
	\begin{class}[number=7]
		\title{Подготовка к проверочной}
		\begin{listofex}
			\item Занятие 7
		\end{listofex}
	\end{class}
	%END_FOLD
	
	%BEGIN_FOLD % ====>>_ Проверочная работа _<<====
	\begin{exam}
		\begin{listofex}
			\item Проверочная
		\end{listofex}
	\end{exam}
	%END_FOLD