%
%===============>>  Ромазанов Модуль 6 <<=============
%
\setmodule{6}

%BEGIN_FOLD % ====>>_____ Занятие 1 _____<<====
\begin{class}[number=1]
	\begin{listofex}
		\item В прямоугольном треугольнике \( \angle C=90\degree \), \( AB=8 \), \( \angle A=30\degree \). Найдите угол \( B \). Чему будет равен катет \( CB \)?
		\item В прямоугольном треугольнике \( ABC \) катет \( AС \) равен половине гипотенузы \( AB \). Найдите градусную меру всех углов треугольника.
		\item Известно, что в прямоугольном треугольнике (\( \angle C=90\degree \)) угол \( B=60\degree \). Найдите прилежащий к этому углу катет, если гипотенуза равна \( 50 \).
		\item Найдите площадь квадрата, если его сторона равна \( 6 \).
		\item Найдите площадь параллелограмма, если одна из его сторон равна \( 8 \), а опущенная на неё высота равна \( 4 \).
		\item Найдите площадь ромба, если одна из его сторон равна \( 3 \), а высота равна \( 2 \).
		\item Стороны квадрата и ромба равны. Известно, что площадь квадрата равна \( 64 \), а острый угол ромба равен \( 30\degree \). Найдите площадь ромба.
	\end{listofex}
\end{class}
%END_FOLD

%BEGIN_FOLD % ====>>_____ Занятие 2 _____<<====
\begin{class}[number=2]
	\begin{listofex}
		\item Сформулируйте три признака равенства треугольников.
		\item Медиана треугольника делит его на два треугольника, периметры которых равны. Докажите, что треугольник равнобедренный.
		\item Докажите, что в равных треугольниках соответствующие медианы равны.
		\item Докажите, что биссектриса равнобедренного треугольника, проведенная из вершины, является также медианой и высотой.
		\item Медиана треугольника является также его высотой. Докажите, что такой треугольник равнобедренный.
	\end{listofex}
	\newpage
	\title{Домашняя работа}
	\begin{listofex}
		\item Докажите, что в равных треугольниках соответствующие биссектрисы равны.
		\item Медиана треугольника является также его высотой. Докажите, что такой треугольник равнобедренный.
	\end{listofex}
\end{class}
%END_FOLD

%BEGIN_FOLD % ====>>_ Домашняя работа 1 _<<====
\begin{homework}[number=1]
	\begin{listofex}
		\item В прямоугольном треугольнике \( ABC \) \( \angle C=90\degree \), \( BC=4 \), \( \angle A=30\degree \). Найдите гипотенузу.
		\item В прямоугольном треугольнике один угол равен \( 60\degree \). Чем будет равен прилежащий к этому углу катет, если гипотенуза равна \( 12 \)?
		\item Найдите площадь параллелограмма, если одна из его сторон равна \( 30 \), а высота, опущенная на неё, равна \( 20 \).
		\item Стороны квадрата и ромба равны. Известно, что площадь квадрата равна \( 36 \), а острый угол ромба равен \( 30\degree \). Найдите площадь ромба.
	\end{listofex}
\end{homework}
%END_FOLD

%BEGIN_FOLD % ====>>_____ Занятие 3 _____<<====
\begin{class}[number=3]
	\begin{listofex}
		\item Один угол треугольника равен \( 26\degree \), а второй в три раза больше. Найдите третий угол.
		\item Один внутренний угол треугольника в два, а второй в три раза больше третьего, найдите все углы треугольника.
		\item Один внешний угол равен \( 140\degree \), а второй --- \( 100\degree \). Чему равны внутренние углы треугольника?
		\item Угол треугольника равен \( 30\degree \), второй угол в \( 3 \) раза больше первого. Чему равны внешние углы при каждой вершине? Чему равна сумма внешних углов?
		\item Углы треугольника относятся как \(2 : 3 : 4\). Найдите отношение внешних углов треугольника.
		\item Сумма накрест лежащих углов при пересечении двух параллельных прямых секущей равна \(210 \degree \). Найдите эти углы.
		\item Найдите все углы, образованные при пересечении параллельных прямых \(a\) и \(b\) с секущей \(c\), если один из углов равен \( 150 \degree \).
		\item Через вершину \(B\) треугольника \(ABC\) проведена прямая, параллельная прямой \(AC\). Образовавшиеся при этом три угла с вершиной B относятся как \(3 : 10 : 5\). Найдите углы треугольника \(ABC\).
		
	\end{listofex}
	\newpage
	\title{Домашняя работа}
	\begin{listofex}
		\item Угол треугольника равен \( 36\degree \), второй угол в \( 2 \) раза меньше первого. Чему равны внешние углы при каждой вершине? Чему равна сумма внешних углов?
		\item Сумма соответственных углов при пересечении двух параллельных прямых секущей равна \(190\degree \). Найдите эти углы.
		\item Накрест лежащие углы, образованные при пересечении двух параллельных прямых третьей, в сумме составляют \(80\) градусов. Найдите все углы, образовавшиеся при пересечении параллельных прямых секущей.
		\item Найдите все углы, образованные при пересечении параллельных прямых \(a\) и \(b\) с секущей \(c\), если один из углов равен \( 115 \degree \).
	\end{listofex}
\end{class}
%END_FOLD

%BEGIN_FOLD % ====>>_____ Занятие 4 _____<<====
\begin{class}[number=4]
	\begin{listofex}
		\item Занятие 4
	\end{listofex}
\end{class}
%END_FOLD

%BEGIN_FOLD % ====>>_ Домашняя работа 2 _<<====
\begin{homework}[number=2]
	\begin{listofex}
		\item Докажите, что в равных треугольниках соответствующие биссектрисы равны.
		\item Медиана треугольника является также его высотой. Докажите, что такой треугольник равнобедренный.
	\end{listofex}
\end{homework}
%END_FOLD

%BEGIN_FOLD % ====>>_____ Занятие 5 _____<<====
\begin{class}[number=5]
	\begin{listofex}
		\item Занятие 5
	\end{listofex}
	\end{class}
	%END_FOLD
	
	%BEGIN_FOLD % ====>>_____ Занятие 6 _____<<====
	\begin{class}[number=6]
		\begin{listofex}
			\item Занятие 6
		\end{listofex}
	\end{class}
	%END_FOLD
	
	%BEGIN_FOLD % ====>>_ Домашняя работа 3 _<<====
	\begin{homework}[number=3]
		\begin{listofex}
			\item Домашняя работа 3
		\end{listofex}
	\end{homework}
	%END_FOLD
	
	%BEGIN_FOLD % ====>>_____ Занятие 7 _____<<====
	\begin{class}[number=7]
		\title{Подготовка к проверочной}
		\begin{listofex}
			\item Занятие 7
		\end{listofex}
	\end{class}
	%END_FOLD
	
	%BEGIN_FOLD % ====>>_ Проверочная работа _<<====
	\begin{exam}
		\begin{listofex}
			\item Проверочная
		\end{listofex}
	\end{exam}
	%END_FOLD