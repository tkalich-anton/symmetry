%
%===============>>  Сорокин Модуль 5 <<=============
%
\setmodule{5}
%
%BEGIN_FOLD % ====>>_____ Занятие 1 _____<<====
\begin{class}[number=1]
	\begin{listofex}
		\item Докажите, что сумма углов треугольника равна \(180 \degree\).
		\item Найдите все углы, образованные при пересечении параллельных прямых a и b с секущей c, если:
		\begin{tasks}(1)
			\task один из углов равен \(150 \degree\),
			\task один из углов на \(70 \degree\) больше другого.
		\end{tasks}
		\item Через вершину \(B\) треугольника \(ABC\) проведена прямая, параллельная прямой \(AC\). Образовавшиеся при этом три угла с вершиной \(B\) относятся как \(3:10:5\). Найдите углы треугольника \(ABC\).
		\item В параллелограмме \(ABCD\) проведены биссектрисы \(AK\) и \(BN\), которые пересекаются в точке \(Q\). Найти \(\angle AQB\).
		\item В параллелограмме \(ABCD\) углы \(BAC\) и \(CAD\) равны соответственно \(35 \degree\) и \(25 \degree\). Найдите углы \(BAC, ABC, BCA, CDB\).
	\end{listofex}
\end{class}
%END_FOLD

%BEGIN_FOLD % ====>>_____ Занятие 2 _____<<====
\begin{class}[number=2]
	\begin{listofex}
		\item 
	\end{listofex}
\end{class}
%END_FOLD

%BEGIN_FOLD % ====>>_ Домашняя работа 1 _<<====
\begin{homework}[number=1]
	\begin{listofex}
		\item 
	\end{listofex}
\end{homework}
%END_FOLD

%BEGIN_FOLD % ====>>_____ Занятие 3 _____<<====
\begin{class}[number=3]
	\begin{listofex}
		\item 
	\end{listofex}
\end{class}
%END_FOLD

%BEGIN_FOLD % ====>>_____ Занятие 4 _____<<====
\begin{class}[number=4]
	\begin{listofex}
		\item Занятие 4
	\end{listofex}
\end{class}
%END_FOLD

%BEGIN_FOLD % ====>>_ Домашняя работа 2 _<<====
\begin{homework}[number=2]
	\begin{listofex}
		\item ДЗ 2
	\end{listofex}
\end{homework}
%END_FOLD

%BEGIN_FOLD % ====>>_____ Занятие 5 _____<<====
\begin{class}[number=5]
	\begin{listofex}
		\item Занятие 5
	\end{listofex}
\end{class}
%END_FOLD

%BEGIN_FOLD % ====>>_____ Занятие 6 _____<<====
\begin{class}[number=6]
	\begin{listofex}
		\item Занятие 6
	\end{listofex}
\end{class}
%END_FOLD

%BEGIN_FOLD % ====>>_ Домашняя работа 3 _<<====
\begin{homework}[number=3]
	\begin{listofex}
		\item ДЗ 3
	\end{listofex}
\end{homework}
%END_FOLD

%BEGIN_FOLD % ====>>_____ Занятие 7 _____<<====
\begin{class}[number=7]
	\begin{listofex}
		\item Занятие 7
	\end{listofex}
\end{class}
%END_FOLD

%BEGIN_FOLD % ====>>_ Проверочная работа _<<====
\begin{exam}
	\begin{listofex}
		\item Проверочная работа
	\end{listofex}
\end{exam}
%END_FOLD
