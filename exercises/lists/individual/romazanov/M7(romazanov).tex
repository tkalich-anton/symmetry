%
%===============>>  Ромазанов Модуль 7 <<=============
%
\setmodule{7}

%BEGIN_FOLD % ====>>_____ Занятие 1 _____<<====
\begin{class}[number=1]
	\begin{listofex}
		\item Два угла трапеции равны \( 32\degree \) и \( 143\degree \). Найдите два других её угла.
		\item Найдите углы равнобокой трапеции, если разность её противолежащих углов равна \( 86\degree \).
		\item В прямоугольной трапеции тупой угол в \( 5 \) раз больше острого. Найдите углы трапеции.
		\item Высота равнобокой трапеции, проведённая из вершины тупого угла, образует с боковой стороной угол \( 17\degree \). Найдите углы трапеции.
		\item Найдите среднюю линию трапеции, если её основания равны \( 6 \) см и \( 11 \) см.
		\item Одно из оснований трапеции равно \( 7 \) см, а средняя линия --- \( 11 \) см. Найдите второе основание трапеции.
		\item Средняя линия трапеции равна \( 19 \) см, а одно из оснований меньше другого на \( 6 \) см. Найдите основания трапеции.
		\item Средняя линия прямоугольной трапеции равна \( 14 \) см, а её высота, проведённая из вершины тупого угла, делит основание в отношении \( 3:1 \), считая от вершины прямого угла. Найдите основания трапеции.
	\end{listofex}
	\newpage
	\title{Домашняя работа}
	\begin{listofex}
		\item Два угла трапеции равны \( 36\degree \) и \( 62\degree \). Найдите два других её угла.
		\item Найдите среднюю линию трапеции, если её основания равны \( 12 \) см и \( 14 \) см.
		\item Средняя линия трапеции равна \( 24 \) см, а её основания относятся как \( 3:5 \). Найдите основания трапеции.
	\end{listofex}
\end{class}
%END_FOLD

%BEGIN_FOLD % ====>>_____ Занятие 2 _____<<====
\begin{class}[number=2]
	\begin{listofex}
		\item   В треугольнике \( ABC \) угол \( C \) равен \( 90\degree \), \( AC=6 \), \( AB=10 \). Найдите \( \sin B \).
		\item В треугольнике \( ABC \) угол \( C \) равен \( 90\degree \), \( BC=14 \), \( AB=50 \). Найдите \( \cos B \).
		\item В треугольнике \( ABC \) угол \( C \) равен  \( 90\degree \), \( \sin B=\dfrac{3}{7} \), \( AB=21 \). Найдите \( AC \).
		\item В треугольнике \( ABC \) угол \( C \) равен  \( 90\degree \), \( \cos B=\dfrac{3}{8} \), \( AB=64 \). Найдите \( BC \).
		\item Синус острого угла \( A \) треугольника \( ABC \) равен \( \dfrac{\sqrt{21}}{5} \). Найдите \( \cos A \).
		\item Косинус острого угла \( A \) треугольника \( ABC \) равен \( \dfrac{3\sqrt{11}}{5} \). Найдите \( \sin A \).
		\item В треугольнике \( ABC \) известно, что \( AB=6 \), \( BC=10 \), \( \sin\angle ABC=\dfrac{1}{3} \). Найдите площадь треугольника \( ABC \).
	\end{listofex}
	\newpage
	\title{Домашняя работа}
	\begin{listofex}
		\item В треугольнике \( ABC \) угол \( C \) равен \( 90\degree \), \( AC=7 \), \( AB=25 \). Найдите \( \sin B \).
		\item В треугольнике \( ABC \) угол \( C \) равен \( 90\degree \), \( BC=14 \), \( AB=20 \). Найдите \( \cos B \).
		\item Косинус острого угла \( A \) треугольника \( ABC \) равен \( \dfrac{3\sqrt{7}}{8} \). Найдите \( \sin A \).
	\end{listofex}
\end{class}
%END_FOLD

%BEGIN_FOLD % ====>>_____ Занятие 3 _____<<====
\begin{class}[number=3]
	\begin{listofex}
		\item Найдите стороны прямоугольника, если: %454
		\begin{tasks}
			\task его площадь равна \( 250 \) см\(^2\), а одна сторона в \(2,5\) раза больше другой;
			\task его площадь равна \( 9 \) м\(^2\), а периметр равен \(12\)м.
		\end{tasks}
		\item Найдите сторону квадрата, площадь которого равна площади прямоугольника со смежными сторонами \(8\) м и \(18\) м. %457
		\item Два участка земли огорожены заборами одинаковой длины. Первый участок имеет форму прямоугольника со сторонами \(220\) м и \(160\) м, а второй имеет форму квадрата. Площадь какого участка больше и на сколько? %458
		\item Диагональ параллелограмма, равная \(13\) см, перпендикулярна к стороне параллелограмма, равной \(12\) см. Найдите площадь параллелограмма. %460
		\item Сторона ромба равна \(6\) см, а один из углов равен \(150 \degree\). Найдите площадь ромба. %462
	\end{listofex}
\end{class}
%END_FOLD

%BEGIN_FOLD % ====>>_____ Занятие 4 _____<<====
\begin{class}[number=4]
	\begin{listofex}
		\item Занятие 4
	\end{listofex}
\end{class}
%END_FOLD
