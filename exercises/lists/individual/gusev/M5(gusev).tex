%
%===============>>  Киселев Модуль 4 <<=============
%
\setmodule{5}
%
%===============>>  Занятие 1  <<===============
%
\begin{class}[number=1]
	\title{Занятие 1}
	\begin{listofex}
		\item Выделите целую часть. Запишите ответ в виде смешанного числа:
		\begin{tasks}(3)
			\task \( \dfrac{98}{7} \)
			\task \( \dfrac{563}{12} \)
			\task \( \dfrac{1874}{31} \)
		\end{tasks}
		\item Представьте в виде несократимой дроби:
		\begin{tasks}(3)
			\task \( \dfrac{42}{63} \)
			\task \( \dfrac{34}{85} \)
			\task \( \dfrac{1000}{3175} \)
		\end{tasks}
		\item Вычислить:
		\begin{tasks}(3)
			\task \( \dfrac{21}{30}+\dfrac{4}{30} \)
			\task \( \dfrac{11}{27}+\dfrac{9}{27} \)
			\task \( \dfrac{15}{100}+\dfrac{16}{100} \)
		\end{tasks}
		\item Вычислить:
		\begin{tasks}(5)
			\task \( \mfrac{8}{1}{9}+3 \)
			\task \( \mfrac{3}{2}{5}+\dfrac{1}{5} \)
			\task \( \mfrac{2}{7}{17}+\dfrac{16}{17} \)
			\task \( \mfrac{2}{9}{10}+\dfrac{5}{10} \)
			\task \( \mfrac{13}{13}{14}+\dfrac{5}{14} \)
		\end{tasks}
		\item Вычислить:
		\begin{tasks}(4)
			\task \( \mfrac{3}{3}{5}+\mfrac{4}{1}{5} \)
			\task \( \mfrac{1}{2}{13}+\mfrac{5}{5}{13} \)
			\task \( \mfrac{6}{1}{7}+\mfrac{3}{4}{7} \)
			\task \( \mfrac{3}{5}{10}+\mfrac{1}{5}{10} \)
		\end{tasks}
		\item Вычислить:
		\begin{tasks}(3)
			\task \( \mfrac{1}{6}{7}+\mfrac{2}{5}{7} \)
			\task \( \mfrac{3}{2}{9}+\mfrac{16}{8}{9} \)
			\task \( \mfrac{3}{13}{15}+\mfrac{4}{12}{15} \)
		\end{tasks}
	\end{listofex}
	\newpage
	\title{Домашняя работа}
	\begin{listofex}
		\item Выделите целую часть. Запишите ответ в виде смешанного числа:
		\begin{tasks}(3)
			\task \( \dfrac{132}{7} \)
			\task \( \dfrac{439}{23} \)
			\task \( \dfrac{145}{10} \)
		\end{tasks}
		\item Вычислить:
		\begin{tasks}(3)
			\task \( \dfrac{13}{20}+\dfrac{2}{20} \)
			\task \( \dfrac{4}{55}+\dfrac{44}{55} \)
			\task \( \dfrac{77}{200}+\dfrac{23}{200} \)
		\end{tasks}
		\item Представьте в виде несократимой дроби:
		\begin{tasks}(3)
			\task \( \dfrac{45}{75} \)
			\task \( \dfrac{81}{90} \)
			\task \( \dfrac{60}{144} \)
		\end{tasks}
		\item Вычислить:
		\begin{tasks}(5)
			\task \( \mfrac{3}{5}{7}+4 \)
			\task \( \mfrac{1}{5}{16}+\dfrac{3}{16} \)
			\task \( \mfrac{4}{5}{7}+\dfrac{4}{7} \)
			\task \( \mfrac{11}{8}{9}+\dfrac{3}{9} \)
			\task \( \mfrac{4}{51}{54}+\dfrac{21}{54} \)
		\end{tasks}
		\item Вычислить:
		\begin{tasks}(4)
			\task \( \mfrac{2}{5}{6}+\mfrac{1}{4}{6} \)
			\task \( \mfrac{12}{4}{9}+\mfrac{17}{1}{9} \)
			\task \( \mfrac{1}{3}{15}+\mfrac{11}{3}{15} \)
			\task \( \mfrac{4}{16}{20}+\mfrac{13}{3}{20} \)
		\end{tasks}
		\item Вычислить:
		\begin{tasks}(3)
			\task \( \mfrac{1}{5}{8}+\mfrac{3}{7}{8} \)
			\task \( \mfrac{4}{1}{9}+\mfrac{5}{8}{9} \)
			\task \( \mfrac{3}{13}{17}+\mfrac{5}{14}{17} \)
		\end{tasks}
	\end{listofex}
\end{class}
%
%===============>>  Занятие 2  <<===============
%
%\begin{class}[number=2]
%	\begin{listofex}
%		\item Пусто
%	\end{listofex}
%\end{class}
%
%===============>>  Домашняя работа 1  <<===============
%
%\begin{homework}[number=1]
%	
%\end{homework}
%
%===============>>  Занятие 3  <<===============
%
%\begin{class}[number=3]
%	\begin{listofex}
%		\item Пусто
%	\end{listofex}
%\end{class}
%
%===============>>  Занятие 4  <<===============
%
%\begin{class}[number=4]
%	\begin{listofex}
%		\item Пусто
%	\end{listofex}
%\end{class}
%
%===============>>  Домашняя работа 2  <<===============
%
%\begin{homework}[number=2]
%	\begin{listofex}
%		\item Пусто
%	\end{listofex}
%\end{homework}
%
%===============>>  Занятие 5  <<===============
%
%\begin{class}[number=6]
%	\begin{listofex}
%		\item Пусто
%	\end{listofex}
%\end{class}
%
%===============>>  Занятие 6  <<===============
%
\begin{class}[number=6]
	\begin{listofex}
		\item Вычислить:
		\begin{tasks}(4)
			\task \( \mfrac{3}{1}{8}+\mfrac{5}{7}{8} \)
			\task \( \mfrac{5}{4}{5}+\mfrac{6}{7}{10} \)
			\task \( \mfrac{10}{7}{9}+\mfrac{12}{77}{81} \)
			\task \( \mfrac{8}{3}{20}+\mfrac{7}{7}{10}\)
		\end{tasks}
		\item Приведите к общему знаменателю:
		\begin{tasks}(4)
			\task \( \dfrac{3}{20} \) и \( \dfrac{1}{30} \)
			\task \( \dfrac{7}{8} \) и \( \dfrac{5}{12} \)
			\task \( \dfrac{10}{21} \) и \( \dfrac{4}{49} \)
			\task \( \dfrac{5}{36} \) и \( \dfrac{5}{24} \)
		\end{tasks}
		\item Решить уравнения:
		\begin{tasks}(4)
			\task \( \mfrac{5}{1}{20}+x=\mfrac{6}{3}{20} \)
			\task \( \mfrac{8}{8}{9}-x=\mfrac{3}{5}{9} \)
			\task \( x+\mfrac{4}{4}{5}=\mfrac{5}{1}{10} \)
			\task \( \mfrac{9}{1}{12}-x=\mfrac{7}{5}{6} \)
		\end{tasks}
	\end{listofex}
\end{class}
%
%===============>>  Домашняя работа 3  <<===============
%
%\begin{homework}[number=3]
%	\begin{listofex}
%		\item Пусто
%	\end{listofex}
%\end{homework}
%
%===============>>  Занятие 7  <<===============
%
%\begin{class}[number=7]
%	\begin{listofex}
%		\item Пусто
%	\end{listofex}
%\end{class}
%
%===============>>  Проверочная работа  <<===============
%
%\begin{exam}
%	\begin{listofex}
%		\item Пусто
%	\end{listofex}
%\end{exam}