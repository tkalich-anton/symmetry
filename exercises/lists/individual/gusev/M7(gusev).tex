%
%===============>>  Гусев Модуль 7 <<=============
%
\setmodule{7}

%BEGIN_FOLD % ====>>_____ Занятие 1 _____<<====
\begin{class}[number=1]
	\begin{listofex}
		\item Вычислить:
		\begin{tasks}(3)
			\task \( \dfrac{7}{10}+15,3 \)
			\task \( 2,4+\dfrac{13}{100} \)
			\task \( \dfrac{1}{2}+16,7 \)
			\task \( \dfrac{3}{4}-0,1 \)
			\task \( \mfrac{3}{1}{50}+4,98 \)
			\task \( \mfrac{5}{13}{25}-4,12 \)
		\end{tasks}
		\item Вычислите:
		\begin{tasks}(4)
			\task \( 0,5\cdot10 \)
			\task \( 0,15\cdot10000 \)
			\task \( 50,265\cdot10 \)
			\task \( 21,598\cdot1000 \)
			\task \( 9,56\cdot50 \)
			\task \( 8,532\cdot1200 \)
			\task \( 9,123\cdot15000 \)
			\task \( 24,1\cdot130000 \)
		\end{tasks}
		\item Вычислите:
		\begin{tasks}(4)
			\task \( 25,36\cdot1,2 \)
			\task \( 3,5\cdot3,8 \)
			\task \( 63,25\cdot0,3 \)
			\task \( 5,369\cdot0,05 \)
			\task \( 12,03\cdot4,6 \)
			\task \( 23,71\cdot1,8 \)
			\task \( 0,24\cdot5,5 \)
			\task \( 0,7\cdot0,14 \)
		\end{tasks}
	\end{listofex}
\end{class}
%END_FOLD

%BEGIN_FOLD % ====>>_ Домашняя работа 1 _<<====
\begin{homework}[number=1]
	\begin{listofex}
		\item Вычислить:
		\begin{tasks}(4)
			\task \( \dfrac{9}{10}+1,75 \)
			\task \( \dfrac{5}{8}+9,262 \)
			\task \( \mfrac{1}{4}{25}+10,78 \)
			\task \( 15,1-\mfrac{4}{7}{50} \)
		\end{tasks}
		\item Вычислить:
		\begin{tasks}(4)
			\task \( 14,007\cdot10 \)
			\task \( 15,41\cdot10000 \)
			\task \( 78,963\cdot140 \)
			\task \( 555,599\cdot700 \)
		\end{tasks}
	\end{listofex}
\end{homework}
%END_FOLD

%BEGIN_FOLD % ====>>_____ Занятие 2 _____<<====
\begin{class}[number=2]
	\begin{listofex}
		\item Джотаро Куджо вступил в схватку с Дио Брандо. Способность стэнда Дио --- The World: остановка времени. Дио остановил время на \( 15 \) секунд: \( \dfrac{1}{3} \) времени он рассказывал о своём стэнде, ещё \( \dfrac{2}{5} \) от всего времени --- вспоминал семейство Джостаров, а всё оставшееся время он удивлялся, что Джотаро Куджо смог пошевелиться. Сколько секунд Дио был в изумлении?
		\item Большой русский кот Шлёпа весил \( 4 \) кг, когда был котёнком. Когда он повзрослел, его вес увеличился в \( 5 \) раз, а когда его начали откармливать, то ещё в \( 1,5\) раза. Сколько Шлёпа весит теперь?
		\item Вычислите:
		\begin{tasks}(4)
			\task \( 25,36\cdot1,2 \)
			\task \( 3,5\cdot3,8 \)
			\task \( 63,25\cdot0,3 \)
			\task \( 5,369\cdot0,05 \)
			\task \( 12,03\cdot4,6 \)
			\task \( 23,71\cdot1,8 \)
			\task \( 0,24\cdot5,5 \)
			\task \( 0,7\cdot0,14 \)
		\end{tasks}
		\item Округлите дроби до сотых:
		\begin{tasks}(4)
			\task \( 1,561 \)
			\task \( 15,012 \)
			\task \( 140,778 \)
			\task \( 89,369 \)
			\task \( 15,6248 \)
			\task \( 47,1459 \)
			\task \( 62,8955 \)
			\task \( 36,9876 \)
		\end{tasks}
	\end{listofex}
\end{class}
%END_FOLD

%BEGIN_FOLD % ====>>_ Домашняя работа 2 _<<====
\begin{homework}[number=2]
	\begin{listofex}
		\item Количество детей, играющих в Five Nights at Freddy's на момент выхода игры составляло \( 5 \) тысяч человек. Через полгода количество игроков увеличилось в \( 10 \) раз, а ещё через месяц --- в \( 7,5 \) раз. Сколько тысяч человек играло в игру на тот момент?
		\item Вычислите:
		\begin{tasks}(4)
			\task \( 1,5\cdot2,8 \)
			\task \( 3,18\cdot5,96 \)
			\task \( 2,103\cdot1,7 \)
			\task \( 5,78\cdot3,02 \)
			\task \( 5,99\cdot1,1 \)
			\task \( 4,36\cdot7,202 \)
			\task \( 9,863\cdot2,125 \)
			\task \( 10,27\cdot35,116 \)
		\end{tasks}
	\end{listofex}
\end{homework}
%END_FOLD

%BEGIN_FOLD % ====>>_____ Занятие 3 _____<<====
\begin{class}[number=3]
	\begin{listofex}
		\item Округлите дроби до сотых:
		\begin{tasks}(4)
			\task \( 1,561 \)
			\task \( 15,012 \)
			\task \( 140,778 \)
			\task \( 89,369 \)
			\task \( 15,6248 \)
			\task \( 47,1459 \)
			\task \( 62,8955 \)
			\task \( 36,9876 \)
		\end{tasks}
		\item Выполните деление:
		\begin{tasks}(4)
			\task \( 6,9:3 \)
			\task \( 10,2:2 \)
			\task \( 20,7:9 \)
			\task \( 243,2:8 \)
			\task \( 88,298:7 \)
			\task \( 772,8:12 \)
			\task \( 93,15:23 \)
			\task \( 0,644:92 \)
		\end{tasks}
		\item Рельс длиной \( 8,14 \) м разрезали на две части, одна из которых на \( 0,96 \) м длиннее другой. Определите длину каждой из них.
	\end{listofex}
\end{class}
%END_FOLD

%BEGIN_FOLD % ====>>_ Домашняя работа 3 _<<====
\begin{homework}[number=3]
	\begin{listofex}
		\item Округлите дроби до сотых:
		\begin{tasks}(4)
			\task \( 9,1234 \)
			\task \( 0,5667 \)
			\task \( 12,3661 \)
			\task \( 0,0078 \)
		\end{tasks}
		\item Выполните деление:
		\begin{tasks}(4)
			\task \( 0,644:92 \)
			\task \( 1,15:5 \)
			\task \( 19,2:8 \)
			\task \( 2,88:4 \)
		\end{tasks}
		\item Рельс длиной \( 8,14 \) м разрезали на две части, одна из которых на \( 0,96 \) м длиннее другой. Определите длину каждой из них.
	\end{listofex}
\end{homework}
%END_FOLD

%BEGIN_FOLD % ====>>_____ Занятие 4 _____<<====
\begin{class}[number=4]
	\begin{listofex}
		\item Выполните действия
		\begin{tasks}
			\task \( 4,735\cdot0,5+14,95\cdot1,3+2,121\cdot0,7 \)
			\task \( (0,578+2,172)\cdot(1,823+0,117)-1,711\cdot(0,418+1,382) \)
			\task \( 3,006-0,0417\cdot3-0,875\cdot0,4 \)
		\end{tasks}
		\item Ширина прямоугольника \( 5,15 \) см, а длина в \( 2 \) раза больше. Найдите площадь и периметр данного прямоугольника?
		\item Площадь прямоугольника \( 42 \) см\( ^{2} \), ширина его равна \( 6 \) см. Чему равна длина прямоугольника?
	\end{listofex}
\end{class}
%END_FOLD


%BEGIN_FOLD % ====>>_ Проверочная работа _<<====
\begin{exam}
	\begin{listofex}
		\item Проверочная
	\end{listofex}
\end{exam}
%END_FOLD