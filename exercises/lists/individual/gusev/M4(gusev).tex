%
%===============>>  Киселев Модуль 4 <<=============
%
\setmodule{4}
%
%===============>>  Занятие 1  <<===============
%
\begin{class}[number=1]
	\title{Занятие 1}
	\begin{listofex}
		\item Вычислить:
		\begin{tasks}(1)
			\task \( (42\cdot124+2430):38\cdot202-(3008:94+527\cdot8):72 \)
			\task \( \left( \left( 16000:32 - 1640:82 \right):15\cdot7000 - 192000 \right):40 \)
		\end{tasks}
		\item Вычислить рациональным способом:
		\begin{tasks}(3)
			\task \( 27+123+16+234 \)
			\task \( (37+89)-37 \)
			\task \( 66507+(83000-56507) \)
		\end{tasks}
	\end{listofex}
	\begin{definit}
		Распределительный закон умножения:
		\[
		a\cdot b + a\cdot c = a\cdot(b+c)
		\qquad
		a \cdot b - a \cdot c = a\cdot(b-c)\]
	\end{definit}
	\begin{listofex}[resume]
		\item Вычислить рациональным способом:
		\begin{tasks}(3)
			\task \( 43\cdot16+43\cdot84 \)
			\task \( 63\cdot7-7\cdot33 \)
			\task \( 57\cdot123+57\cdot34-57\cdot157 \)
		\end{tasks}
		\item Решить уравнение:
		\begin{tasks}(3)
			\task \( 57+x=75 \)
			\task \( x-2091=468 \)
			\task \( 825-(x+176)=493 \)
		\end{tasks}
		\item Расположите числа в порядке возрастания:
		\[ 50057,\;507,\;5757,\;77755,\;75057,\;7557,\;55577,\;7057,\;570 \]
		\item Сравните числа, не производя вычислений:
		\begin{tasks}(2)
			\task \( 98478-12345 \) и \( 98478-1234 \)
			\task \( 12554:459 \) и \( 12554:353 \)
			\task \( 532\cdot618-532\cdot436 \) и \( 532\cdot(618-436) \)
			\task \( 359\cdot(764-547) \) и \( 359\cdot766-359\cdot549 \)
		\end{tasks}
	\end{listofex}
	\newpage
	\title{Домашняя работа}
	\begin{listofex}
		\item Вычислить:
		\begin{tasks}(1)
			\task \( (64\cdot125+128\cdot75):800\cdot5000-(300\cdot400+5107\cdot800):70 \)
			\task \( \left( \left( 24\cdot250+18\cdot350 \right):60\cdot400+\left( 44\cdot4500+108\cdot1500 \right):20 \right):400 \)
		\end{tasks}
		\item Вычислить рациональным способом:
		\begin{tasks}(3)
			\task \( 48+215+132+35 \)
			\task \( (65+111)-65 \)
			\task \( 3453+(6748-453) \)
		\end{tasks}
		\item Вычислить рациональным способом:
		\begin{tasks}(3)
			\task \( 85\cdot44+44\cdot15 \)
			\task \( 90\cdot25+10\cdot25 \)
			\task \( 31\cdot145+31\cdot75+31\cdot80 \)
		\end{tasks}
		\item Решить уравнение:
		\begin{tasks}(3)
			\task \( 33+x=112 \)
			\task \( 1741-x=599 \)
			\task \( (524-x)-133=207 \)
		\end{tasks}
		\item Расположите числа в порядке убывания:
		\[ 8004,\;408,\;44888,\;84848,\;4008,\;804,\;8040,\;840,\;88444,\;48484 \]
		\item Сравните числа, не производя вычислений:
		\begin{tasks}(2)
			\task \( 6387+76345 \) и \( 6387+73645 \)
			\task \( 13673:121 \) и \( 13673:113 \)
			\task \( 258\cdot764 + 258\cdot545 \) и \( 258\cdot(764+548) \)
			\task \( 496\cdot(862-715) \) и \( 496\cdot860-496\cdot715 \)
		\end{tasks}
	\end{listofex}
\end{class}
%
%===============>>  Занятие 2  <<===============
%
%\begin{class}[number=2]
%	\begin{listofex}
%		\item Пусто
%	\end{listofex}
%\end{class}
%
%===============>>  Домашняя работа 1  <<===============
%
%\begin{homework}[number=1]
%	
%\end{homework}
%
%===============>>  Занятие 3  <<===============
%
%\begin{class}[number=3]
%	\begin{listofex}
%		\item Пусто
%	\end{listofex}
%\end{class}
%
%===============>>  Занятие 4  <<===============
%
%\begin{class}[number=4]
%	\begin{listofex}
%		\item Пусто
%	\end{listofex}
%\end{class}
%
%===============>>  Домашняя работа 2  <<===============
%
%\begin{homework}[number=2]
%	\begin{listofex}
%		\item Пусто
%	\end{listofex}
%\end{homework}
%
%===============>>  Занятие 5  <<===============
%
%\begin{class}[number=6]
%	\begin{listofex}
%		\item Пусто
%	\end{listofex}
%\end{class}
%
%===============>>  Занятие 6  <<===============
%
%\begin{class}[number=6]
%	\begin{listofex}
%		\item Пусто
%	\end{listofex}
%\end{class}
%
%===============>>  Домашняя работа 3  <<===============
%
%\begin{homework}[number=3]
%	\begin{listofex}
%		\item Пусто
%	\end{listofex}
%\end{homework}
%
%===============>>  Занятие 7  <<===============
%
%\begin{class}[number=7]
%	\begin{listofex}
%		\item Пусто
%	\end{listofex}
%\end{class}
%
%===============>>  Проверочная работа  <<===============
%
%\begin{exam}
%	\begin{listofex}
%		\item Пусто
%	\end{listofex}
%\end{exam}