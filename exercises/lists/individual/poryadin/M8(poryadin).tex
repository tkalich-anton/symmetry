%
%===============>>  Порядин Модуль 8 <<=============
%=
\setmodule{8}

%BEGIN_FOLD % ====>>_____ Занятие 1 _____<<====
\begin{class}[number=1]
	\begin{listofex}
		\item Решите уравнения: 
			\begin{tasks}(2)
			\task \( x^3=6x^2+7x \)
			\task \( x^3=x^2-7+7 \)
			\task! \( (x-2)(x-3)(x-4)=(x-3)(x-4)(x-5) \)
			\task \( x(x^{2}+4x+4)=3(x+2) \)
			\task \( (x+5)^{3}=25(x+5)\)
		\end{tasks} 
		\item Решите неравенства:
		\begin{tasks}(2)
			\task \( \dfrac{(x-3)(x-4)}{x-5}\ge0 \)
			\task \( \dfrac{(x+1)(x-1)}{x-3}<0 \)
			\task \( \dfrac{(x-0)(x-3)}{x+4}>0 \)
			\task \( \dfrac{x}{(x+1)(x-8)}\le0 \)
		\end{tasks}
		\item Постройте график функции \( y=5-\dfrac{x^4-x^3}{x^2-x} \) и определите, при каких значениях \( m \) прямая \( y=m \) имеет с графиком ровно две общие точки.
	\end{listofex}
\end{class}
%END_FOLD

%BEGIN_FOLD % ====>>_____ Занятие 2 _____<<====
\begin{class}[number=2]
	\begin{listofex}
		\item .
	\end{listofex}
\end{class}
%END_FOLD

%BEGIN_FOLD % ====>>_ Домашняя работа 1 _<<====
\begin{homework}[number=1]
	\begin{listofex}
		\item Решите неравенства:
		\begin{tasks}(2)
			\task \( \dfrac{(x+3)(x+4)}{x+5}\le0 \)
			\task \( \dfrac{x}{(x-3)(x+4)}>0 \)
			\end{tasks}
		\item Постройте график функции \( y=\dfrac{(x^2-4)(x^2-4x+3)}{x^2-3x+2} \) и определите, при каких значениях \( m \) прямая \( y=m \) имеет с графиком ровно одну общую точку.
	\end{listofex}
\end{homework}
%END_FOLD

%BEGIN_FOLD % ====>>_____ Занятие 3 _____<<====
\begin{class}[number=3]
	\begin{listofex}
		\item На экзамене \( 25 \) билетов, Сергей не выучил \( 3 \) из них. Найдите вероятность того, что ему попадётся выученный билет.
		\item Телевизор у Саши сломался и показывает только один случайный канал. Саша включает телевизор. В это время по пятнадцати каналам из пятидесяти показывают кинокомедии. Найдите вероятность того, что Саша попадет на канал, где комедия не идет.
		\item В фирме такси в данный момент свободно \( 15 \) машин: \( 3 \) чёрных, \( 6 \) жёлтых и \( 6 \) зелёных. По вызову выехала одна из машин, случайно оказавшаяся ближе всего к заказчику. Найдите вероятность того, что к нему приедет жёлтое такси.
		\item В каждой пятой банке кофе согласно условиям акции есть приз. Призы распределены по банкам случайно. Галя покупает банку кофе в надежде выиграть приз. Найдите вероятность того, что Галя не найдёт приз в своей банке.
		\item В среднем из каждых \( 80 \) поступивших в продажу аккумуляторов \( 76 \) аккумуляторов заряжены. Найдите вероятность того, что купленный аккумулятор не заряжен.
		\item Для экзамена подготовили билеты с номерами от \( 1 \) до \( 25 \). Какова вероятность того, что наугад взятый учеником билет имеет номер, являющийся двузначным числом?
		\item В мешке содержатся жетоны с номерами от \( 2 \) до \( 51 \) включительно. Какова вероятность, того, что номер извлеченного наугад из мешка жетона является однозначным числом?
		\item В денежно-вещевой лотерее на \( 100000 \) билетов разыгрывается \( 1300 \) вещевых и \( 850 \) денежных выигрышей. Какова вероятность получить вещевой выигрыш?
		\item В чемпионате по футболу участвуют \( 16 \) команд, которые жеребьевкой распределяются на \( 4 \) группы: \( A \), \( B \), \( C \) и \( D \). Какова вероятность того, что команда России не попадает в группу \( A \)?
		\item В группе из \( 20 \) российских туристов несколько человек владеют иностранными языками. Из них пятеро говорят только по-английски, трое только по-французски, двое по-французски и по-английски. Какова вероятность того, что случайно выбранный турист говорит по-французски?
		\item Стас, Денис, Костя, Маша, Дима бросили жребий --- кому начинать игру. Найдите вероятность того, что начинать игру должна будет девочка.
		\item В лыжных гонках участвуют \( 11 \) спортсменов из России, \( 6 \) спортсменов из Норвегии и \( 3 \) спортсмена из Швеции. Порядок, в котором спортсмены стартуют, определяется жребием. Найдите вероятность того, что первым будет стартовать спортсмен из России.
		\item Из каждых \( 1000 \) электрических лампочек \( 5 \) бракованных. Какова вероятность купить исправную лампочку?
		\item Определите вероятность того, что при бросании кубика выпало число очков, не большее \( 3 \).
		\item Игральную кость бросают дважды. Найдите вероятность того, что оба раза выпало число, большее \( 3 \).
		\item Игральную кость бросают дважды. Найдите вероятность того, что сумма двух выпавших чисел равна \( 4 \) или \( 7 \).
	\end{listofex}
\end{class}
%END_FOLD

%BEGIN_FOLD % ====>>_____ Занятие 4 _____<<====
\begin{class}[number=4]
	\begin{listofex}
		\item Биссектрисы углов \( N \) и \( M \) треугольника \( MNP \) пересекаются в точке \( A \). Найдите \( \angle NAM \), если \( \angle N=84\degree \), а \( \angle M=42\degree \).
		\item Диагональ прямоугольника образует угол \( 51\degree \) с одной из его сторон. Найдите острый угол между диагоналями этого прямоугольника. Ответ дайте в градусах.
		\item
		\begin{minipage}[t]{\bodywidth}
			Углы, отмеченные на рисунке одной дугой, равны. Найдите угол \( \alpha \). Ответ дайте в градусах.
		\end{minipage}
		\hspace{0.02\linewidth}
		\begin{minipage}[t]{\picwidth}
			\includegraphics[align=t, width=0.8\linewidth]{../../../../../exercises/lists/pics/leontevaM7H2-1}
		\end{minipage}
		\item
		\begin{minipage}[t]{\bodywidth}
			Углы, отмеченные на рисунке одной дугой, равны. Найдите угол \( \alpha \). Ответ дайте в градусах.
		\end{minipage}
		\hspace{0.02\linewidth}
		\begin{minipage}[t]{\picwidth}
			\includegraphics[align=t, width=0.8\linewidth]{../../../../../exercises/lists/pics/leontevaM7H2-2}
		\end{minipage}
		\item 
		\begin{minipage}[t]{\bodywidth}
			На плоскости даны четыре прямые. Известно, что \( \angle 1 = 120^{\circ} \), \( \angle 2 = 60^{\circ} \), \( \angle 3 = 55^{\circ} \). Найдите \( \angle 4 \). Ответ дайте в градусах.
		\end{minipage}
		\hspace{0.02\linewidth}
		\begin{minipage}[t]{\picwidth}
			\includegraphics[align=t, width=0.8\linewidth]{../../../../../exercises/lists/pics/leontevaM7H2-3}
		\end{minipage}		
		\item На сторонах угла \( BAC \) и на его биссектрисе отложены равные отрезки \( AB \), \( AC \) и \( AD \). Величина угла \( BDC \) равна \( 160\degree \). Определите величину угла \( BAC \).
		\item В треугольнике \( ABC \) углы \( A \) и \( C \) равны \( 40\degree \) и \( 60\degree \) соответственно. Найдите угол между высотой \( BH \) и биссектрисой \( BD \).
		\item Центральный угол \( AOB \) опирается на хорду \( AB \) длиной \( 6 \). При этом угол \( OAB \) равен \( 60\degree \). Найдите радиус окружности.
		\item В окружности с центром в точке \( O \) проведены диаметры \( AD \) и \( BC \), угол \( OCD \) равен \( 30\degree \). Найдите величину угла \( OAB \).
		\item Найдите градусную меру тупого центрального угла \( MON \), если известно, \( NP \) --- диаметр, а градусная мера угла \( MNP \) равна \( 18\degree \).
		\item Найдите вписанный угол \( DEF \), если градусные меры дуг \( DE \) и \( EF \) равны \( 150\degree \) и \( 68\degree \) соответственно.
		\item Найдите градусную меру угла \( ACB \), если известно, что \( BC \) является диаметром окружности, а градусная мера центрального угла \( AOC \) равна \( 96\degree \).
		\item В окружности с центром \( O \) \( AC \) и \( BD \) --- диаметры. Угол \( ACB \) равен \( 26\degree \). Найдите угол \( AOD \). Ответ дайте в градусах.
		\item Прямоугольный треугольник с катетами \( 5 \) см и \( 12 \) см вписан в окружность. Чему равен радиус этой окружности?
		\item Точки \( A \) и \( B \) делят окружность на две дуги, длины которых относятся как \( 9:11 \). Найдите величину центрального угла, опирающегося на меньшую из дуг. Ответ дайте в градусах.
		\item Величина центрального угла \( AOD \) равна \( 110\degree \). Найдите величину вписанного угла \( ACB \). Ответ дайте в градусах.
		\item Точки \( A \), \( B \), \( C \) и \( D \) лежат на одной окружности так, что хорды \( AB \) и \( CD \) взаимно перпендикулярны, а угол \( BDC=25\degree \) . Найдите величину угла \( ACD \).
	\end{listofex}
\end{class}
%END_FOLD

%BEGIN_FOLD % ====>>_ Домашняя работа 2 _<<====
\begin{homework}[number=2]
	\begin{listofex}
		\item Решите уравнения:
	\end{listofex}
\end{homework}
%END_FOLD

%BEGIN_FOLD % ====>>_____ Занятие 5 _____<<====
\begin{class}[number=5]
	\begin{listofex}
		\item Занятие 5
	\end{listofex}
\end{class}
%END_FOLD

%BEGIN_FOLD % ====>>_____ Занятие 6 _____<<====
\begin{class}[number=6]
	\begin{listofex}
		\item Занятие 6
	\end{listofex}
\end{class}
%END_FOLD

%BEGIN_FOLD % ====>>_ Домашняя работа 3 _<<====
\begin{homework}[number=3]
	\begin{listofex}
		\item .
	\end{listofex}
\end{homework}
%END_FOLD

%BEGIN_FOLD % ====>>_____ Занятие 7 _____<<====
\begin{class}[number=7]
	\title{Подготовка к проверочной}
	\begin{listofex}
		\item Занятие 7
	\end{listofex}
\end{class}
%END_FOLD

%BEGIN_FOLD % ====>>_ Проверочная работа _<<====
\begin{exam}
	\begin{listofex}
		\item Проверочная
	\end{listofex}
\end{exam}
%END_FOLD