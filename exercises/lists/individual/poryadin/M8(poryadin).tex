%
%===============>>  Порядин Модуль 8 <<=============
%=
\setmodule{8}

%BEGIN_FOLD % ====>>_____ Занятие 1 _____<<====
\begin{class}[number=1]
	\begin{listofex}
		\item Решите уравнения: 
			\begin{tasks}(2)
			\task \( x^3=6x^2+7x \)
			\task \( x^3=x^2-7+7 \)
			\task! \( (x-2)(x-3)(x-4)=(x-3)(x-4)(x-5) \)
			\task \( x(x^{2}+4x+4)=3(x+2) \)
			\task \( (x+5)^{3}=25(x+5)\)
		\end{tasks} 
		\item Решите неравенства:
		\begin{tasks}(2)
			\task \( \dfrac{(x-3)(x-4)}{x-5}\ge0 \)
			\task \( \dfrac{(x+1)(x-1)}{x-3}<0 \)
			\task \( \dfrac{(x-0)(x-3)}{x+4}>0 \)
			\task \( \dfrac{x}{(x+1)(x-8)}\le0 \)
		\end{tasks}
		\item Постройте график функции \( y=5-\dfrac{x^4-x^3}{x^2-x} \) и определите, при каких значениях \( m \) прямая \( y=m \) имеет с графиком ровно две общие точки.
	\end{listofex}
\end{class}
%END_FOLD

%BEGIN_FOLD % ====>>_____ Занятие 2 _____<<====
\begin{class}[number=2]
	\begin{listofex}
		\item .
	\end{listofex}
\end{class}
%END_FOLD

%BEGIN_FOLD % ====>>_ Домашняя работа 1 _<<====
\begin{homework}[number=1]
	\begin{listofex}
		\item Решите неравенства:
		\begin{tasks}(2)
			\task \( \dfrac{(x+3)(x+4)}{x+5}\le0 \)
			\task \( \dfrac{x}{(x-3)(x+4)}>0 \)
			\end{tasks}
		\item Постройте график функции \( y=\dfrac{(x^2-4)(x^2-4x+3)}{x^2-3x+2} \) и определите, при каких значениях \( m \) прямая \( y=m \) имеет с графиком ровно одну общую точку.
	\end{listofex}
\end{homework}
%END_FOLD

%BEGIN_FOLD % ====>>_____ Занятие 3 _____<<====
\begin{class}[number=3]
	\begin{listofex}
		\item Занятие 3 
	\end{listofex}
\end{class}
%END_FOLD

%BEGIN_FOLD % ====>>_____ Занятие 4 _____<<====
\begin{class}[number=4]
	\begin{listofex}
		\item .
	\end{listofex}
\end{class}
%END_FOLD

%BEGIN_FOLD % ====>>_ Домашняя работа 2 _<<====
\begin{homework}[number=2]
	\begin{listofex}
		\item Решите уравнения:
	\end{listofex}
\end{homework}
%END_FOLD

%BEGIN_FOLD % ====>>_____ Занятие 5 _____<<====
\begin{class}[number=5]
	\begin{listofex}
		\item Занятие 5
	\end{listofex}
\end{class}
%END_FOLD

%BEGIN_FOLD % ====>>_____ Занятие 6 _____<<====
\begin{class}[number=6]
	\begin{listofex}
		\item Занятие 6
	\end{listofex}
\end{class}
%END_FOLD

%BEGIN_FOLD % ====>>_ Домашняя работа 3 _<<====
\begin{homework}[number=3]
	\begin{listofex}
		\item .
	\end{listofex}
\end{homework}
%END_FOLD

%BEGIN_FOLD % ====>>_____ Занятие 7 _____<<====
\begin{class}[number=7]
	\title{Подготовка к проверочной}
	\begin{listofex}
		\item Занятие 7
	\end{listofex}
\end{class}
%END_FOLD

%BEGIN_FOLD % ====>>_ Проверочная работа _<<====
\begin{exam}
	\begin{listofex}
		\item Проверочная
	\end{listofex}
\end{exam}
%END_FOLD