%
%===============>>  Порядин Модуль 6 <<=============
%=
\setmodule{6}

%BEGIN_FOLD % ====>>_____ Занятие 1 _____<<====
\begin{class}[number=1]
	\begin{listofex}
		\item Сократите дробь:
		\begin{tasks}(2)
			\task \( \dfrac{7x+7y}{x^2+2xy+y^2} \)
			\task \( \dfrac{az+at-bz-bt}{az-at-bz+bt} \)
			\task \( \dfrac{a^3-a^2-a+1}{a^2-2a^2+1} \)
			\task \( \dfrac{-15x^2+4yz-10xz+6xy}{15x^2+2yz-5xz-6xy} \)
			\task \( \dfrac{9x^3-2bc^2-xc^2+18x^2b}{3x^2-2bc-xc+6xb} \)
		\end{tasks}
		\item Выполните умножение:
		\begin{tasks}(1)
			\task \( \dfrac{4x^2-6xy+9y^2}{2x-3y}\cdot\dfrac{9y^2-4x^2}{8x^3+27y^3} \)
			\task \( \dfrac{3-6x}{2x^2+4x+8}\cdot\dfrac{2x+1}{x^2+4-4x}\cdot\dfrac{8-x^3}{4x^2-1} \)
		\end{tasks}
		\item Выполните деление:
		\begin{tasks}(1)
				\task \( \dfrac{2a^2+6ac-ab-3bc}{2ab-4a^2+bc-2ac}:\dfrac{2ac+ab+3bc+6c^2}{2ab+bc-4ac-2c^2} \)
				\task \( \dfrac{x^4-2x^2+1}{x^3-27}:\dfrac{x^2-1}{x^2+3x+9} \)
		\end{tasks}
	\end{listofex}
\end{class}
%END_FOLD

%BEGIN_FOLD % ====>>_ Домашняя работа 1 _<<====
\begin{homework}[number=1]
	\begin{listofex}
		\item Выполните умножение:
		\begin{tasks}(1)
			\task \( \dfrac{a^2+ab}{5a-a^2+b^2-5b}\cdot\dfrac{a^2-b^2+25-10a}{a^2-b^2} \)
			\task \( \dfrac{a+b}{a^2-4b+4a-b^2}\cdot\dfrac{16-b^2-a^2-2ab}{a^2+ab} \)
		\end{tasks}
		\item Выполните деление:
		\begin{tasks}(1)
			\task \( \dfrac{27a^3-64b^3}{b^2-4}:\dfrac{9a^2+12ab+16b^2}{b^2+4b+4} \)
			\task \( \dfrac{6c(a^6-b^{12})}{a^2+ab^2+b^4}:(c^3(2a+2b^2)(3a^2-3ab^2+3b^4)) \)
		\end{tasks}
		\item Решить уравнение:
		\begin{tasks}(2)
			\task \( 3x^2-4x-7=0 \)
			\task \( 6x^2+3x+1=0 \)
			\task \( 11x^2-9x-1=0 \)
			\task \( x^2+12x+36=0 \)
		\end{tasks}
		\item Решить уравнение:
		\begin{tasks}(2)
			\task \( -x(x+7)=(x-2)(x+2) \)
			\task \( (3x-1)(x+3)=x(1+6x) \)
		\end{tasks}
		\end{listofex}
\end{homework}
%END_FOLD

%BEGIN_FOLD % ====>>_____ Занятие 2 _____<<====
\begin{class}[number=2]
	\begin{listofex}
		\item Решите уравнения:
		\begin{tasks}(2)
			\task \( 2x^2+x=0 \)
			\task \( -3x^2-12x=0 \)
			\task \( 7x^2+14x=0 \)
			\task \( 8x^2+3x=0 \)
			\task \( x^2=64 \)
			\task \( x^2=36 \)
			\task \( -x^2+25=0 \)
			\task \( 100-x^2=0 \)
			\task \( x^2-8x+12=0 \)
			\task \( 15-2x-x^2=0 \)
			\task \( x^2-4x+3=0 \)
			\task \( x^2-4x+4=0 \)
			\task \( 10x^2-17x+34=7x^2-26x+28 \)
			\task \( (x-6)^2=-24x \)
			\task \( x^2+9=(x+9)^2 \)
			\task \( (2x+7)^2=(2x-1)^2 \)
		\end{tasks}
		\item Упростите выражения:
		\begin{tasks}(2)
			\task \( \dfrac{b-6}{4-b^2}+\dfrac{2}{2b-b^2} \)
			\task \( \dfrac{b}{ab-5a^2}-\dfrac{15b-25a}{b^2-25a^2} \)
			\task \( \dfrac{x-12a}{x^2-16a^2}-\dfrac{4a}{4ax-x^2} \)
			\task \( \dfrac{a-30y}{a^2-100y^2}-\dfrac{10y}{10ay-a^2} \)
		\end{tasks}
	\end{listofex}
\end{class}
%END_FOLD

%BEGIN_FOLD % ====>>_____ Занятие 3 _____<<====
\begin{class}[number=3]
	\begin{listofex}
		\item Упростите выражения:
		\begin{tasks}(2)
			\task \( \dfrac{b-6}{4-b^2}+\dfrac{2}{2b-b^2} \)
			\task \( \dfrac{b}{ab-5a^2}-\dfrac{15b-25a}{b^2-25a^2} \)
			\task \( \dfrac{x-12a}{x^2-16a^2}-\dfrac{4a}{4ax-x^2} \)
			\task \( \dfrac{a-30y}{a^2-100y^2}-\dfrac{10y}{10ay-a^2} \)
		\end{tasks}
		\item Упростите выражения:
		\begin{itasks}[2]
			\task \( \dfrac{1}{2x-2}+\dfrac{2}{5x-5} \)
			\task \( \dfrac{2m}{ax+bx}+\dfrac{3y}{ay+by} \)
			\task \( \dfrac{y}{ax-bx}-\dfrac{x}{ay-by} \)
			\task \( \dfrac{3b}{2a^3b-8a^2b^2}-\dfrac{5a}{12a^3b-3a^4} \)
		\end{itasks}
		\item Упростите выражение: \[ \left( \dfrac{4}{a+1}+\dfrac{2a}{a^2-1}+\dfrac{-1}{a-1} \right)\cdot(a^2+2a+1) \]
		\item Упростите выражения:
		\begin{tasks}(1)
			\task \( \left( a+\dfrac{2+a^2}{1-a} \right)\cdot\dfrac{1-2a+a^2}{a+2} \)
			\task \( \dfrac{7-5m}{m-4}+\dfrac{4m}{m+4}\cdot\dfrac{m^2-16}{4m}+\dfrac{9m-23}{m-4} \)
			\task \( \left( \dfrac{16b}{16-b^2}+\dfrac{4-b}{4+b} \right):\dfrac{3a+16}{a^2-36}+\dfrac{6(a-6)}{a+6} \)
		\end{tasks}
	\end{listofex}
\end{class}
%END_FOLD

%BEGIN_FOLD % ====>>_____ Занятие 4 _____<<====
\begin{class}[number=4]
	\begin{listofex}
		\item Решите биквадратные уравнения:
		\begin{tasks}(2)
			\task \( x^4+20x^2+64=0 \)
			\task \( 4x^4-41x^2+100=0 \)
			\task \( 6c^4-35=11c^2 \)
			\task \( 10p^4-21=p^2 \)
			\task \( 9x^4=9x^2-1 \)
			\task \( 3x^4+21=4x^2 \)
		\end{tasks}
		\item Решите уравнения:
		\begin{tasks}(2)
			\task \( (2x+3)^2=3(2x+3)-2 \)
			\task \( (x^2-2x)^2-4(x^2-2x)+3=0 \)
		\end{tasks}
		\item Выполните деление:
		\begin{tasks}(1)
			\task \( \dfrac{2a^2+6ac-ab-3bc}{2ab-4a^2+bc-2ac}:\dfrac{2ac+ab+3bc+6c^2}{2ab+bc-4ac-2c^2} \)
			\task \( \dfrac{x^4-2x^2+1}{x^3-27}:\dfrac{x^2-1}{x^2+3x+9} \)
		\end{tasks}
	\end{listofex}
\end{class}
%END_FOLD

%BEGIN_FOLD % ====>>_____ Занятие 5 _____<<====
\begin{class}[number=5]
	\begin{listofex}
		\item Решите уравнения:
		\begin{tasks}(1)
			\task \( (x^2+x)^2-8x^2-8x+12=0 \)
			\task \( \dfrac{2x-1}{x}+\dfrac{4x}{2x-1}=5 \)
			\task \( 7\left( x+\dfrac{1}{x} \right)+2\left( x^2+\dfrac{1}{x^2} \right)+9=0 \)
			\task \( \dfrac{4}{9x^2-9x+2}-\dfrac{8}{9x^2-9x+8}=1 \)
			\task \( \dfrac{x^2-x}{x^2-x+1}-\dfrac{x^2-x+2}{x^2-x-2}=1 \)
		\end{tasks}
	\end{listofex}
\end{class}
%END_FOLD

%BEGIN_FOLD % ====>>_ Домашняя работа 2 _<<====
\begin{homework}[number=2]
	\begin{listofex}
		\item Решите биквадратные уравнения:
		\begin{tasks}(2)
			\task \( -3x^4-21x^2-36=0 \)
			\task \( x^4+4x^2+3=0 \)
			\task \( 4x^4+20x^2+24=0 \)
		\end{tasks}
		\item Упростите выражение: \[ \dfrac{x^4-2x^2+1}{x^3-27}:\dfrac{x^2-1}{x^2+3x+9} \]
		\item Решите уравнения:
		\begin{tasks}(1)
			\task \( (x^4+3x+1)(x^2+3x+3)+1=0 \)
			\task \( \dfrac{x^2}{(2x+3)^2}-\dfrac{2x^2}{2x+3}+1=0 \)
		\end{tasks}
	\end{listofex}
\end{homework}
%END_FOLD

%BEGIN_FOLD % ====>>_____ Занятие 6 _____<<====
\begin{class}[number=5]
	\begin{listofex}
		\item Решите уравнения:
		\begin{tasks}(1)
			\task \( \dfrac{x^2-x}{x^2-x+1}-\dfrac{x^2-x+2}{x^2-x-2}=1 \)
			\task \( x^2(x^2-1)(x^2-2)(x^2-3)=24 \)
		\end{tasks}
		\item Постройте график функции \( y=2x-5 \)
		\begin{tasks}(1)
			\task Проверьте \textit{(графическим, а затем аналитическим способом)}, принадлежит ли точка с координатами \( (4;3) \) графику этой функции?
			\task Найдите абсциссу точки на графике функции, ордината которой равна \( 217 \).
			\task Найдите координаты точек пересечения графика данной функции с графиком функции \( y=4x-1 \).
			\task Найдите уравнение прямой, которая параллельна исходной и проходит через начало координат.
			\task Найдите уравнение прямой, которая параллельна исходной и проходит точку \( (-1;1) \).
		\end{tasks}
		\item Постройте график функции \(y=x^2\) и найдите координаты точек пересечения с прямой \( y=2x \) графическим, а затем аналитическим способом.
		\item Найдите аналитическим способом точки пересечения графиков функций \(f(x)=x^2+3x-10\)	и \( g(x)=-3x^2-9x-19 \).
		\item Парабола вида \( y=ax^2 \) проходит через точку \( (-1;3) \). Найдите \( a \).
		\item Найдите такие значения переменной \( x \), при которых значение функции \( y=6x^2-x-12 \) равнялось нулю.
	\end{listofex}
\end{class}
%END_FOLD

%BEGIN_FOLD % ====>>_____ Занятие 7 _____<<====
\begin{class}[number=6]
	\begin{listofex}
			\item Постройте график функции \(y=x^2\) и найдите координаты точек пересечения с прямой \( y=2x \) графическим, а затем аналитическим способом.
		\item Найдите аналитическим способом точки пересечения графиков функций \(f(x)=x^2+3x-10\)	и \( g(x)=-3x^2-9x-19 \).
		\item Парабола вида \( y=ax^2 \) проходит через точку \( (-1;3) \). Найдите \( a \).
		\item Найдите такие значения переменной \( x \), при которых значение функции \( y=6x^2-x-12 \) равнялось нулю.
		\item Упростите выражения:
		\begin{tasks}(2)
			\task \( \dfrac{b-6}{4-b^2}+\dfrac{2}{2b-b^2} \)
			\task \( \dfrac{b}{ab-5a^2}-\dfrac{15b-25a}{b^2-25a^2} \)
			\task \( \dfrac{x-12a}{x^2-16a^2}-\dfrac{4a}{4ax-x^2} \)
			\task \( \dfrac{a-30y}{a^2-100y^2}-\dfrac{10y}{10ay-a^2} \)
		\end{tasks}
	\end{listofex}
\end{class}
%END_FOLD

%BEGIN_FOLD % ====>>_ Домашняя работа 3 _<<====
\begin{homework}[number=3]
	\begin{listofex}
		\item Найдите такие значения переменной \( x \), при которых значении функции \( y=-7x^2+19x+6 \) равнялось нулю.
		\item Постройте график функции \( y=-2x^2 \) и найдите координаты точек пересечения с прямой \( y=-3x+1 \) графическим, а затем аналитическим способом.
		\item Найдите аналитическим способом точки пересечения графиков функций \(f(x)=-3x^2+2x+1\)	и \( g(x)=x^2-6x+1 \).
		\item Упростите выражение: \[ \left( \dfrac{4}{a+1}+\dfrac{2a}{a^2-1}+\dfrac{-1}{a-1} \right)\cdot(a^2+2a+1) \]
	\end{listofex}
\end{homework}
%END_FOLD

%BEGIN_FOLD % ====>>_____ Занятие 7 _____<<====
\begin{class}[number=7]
	\begin{listofex}
		\item Постройте график функции \( y=3x+1 \)
		\begin{tasks}(1)
			\task Проверьте \textit{(графическим, а потом аналитическим способом)}, принадлежит ли точка с координатами \( (2;7) \) графику этой функции?
			\task Найдите точку на графике функции, ордината которой равна \( 214 \).
			\task Найдите точку пересечения графика данной функции с графиком функции \( y=7x-2 \).
			\task Найдите уравнение прямой, которая параллельна исходной и проходит через начало координат.
			\task Найдите уравнение прямой, которая параллельна исходной и проходит точку \( (-1;1) \).
		\end{tasks}
		\item Постройте график функции \( y=3x^2 \):
		\begin{tasks}(1)
			\task Определите промежутки возрастания и убывания функции;
			\task Найдите область определения и область значений функции;
			\task Найдите точки пересечения графика данной функции с графиком функции \( y=7x-2 \).
			\end{tasks}
	\end{listofex}
\end{class}
%END_FOLD

%BEGIN_FOLD % ====>>_____ Занятие 8 _____<<====
\begin{class}[number=8]
	\begin{listofex}
		\item  Постройте график функции
		\[y=	 \left\{
		\begin{array}{l}
			2x+1, \quad x<0,\\
			-1,5x+1, \quad 0\leq x<2,\\
			x-4, \quad x\geq 2
		\end{array}
		\right. \]
		и определите, при каких значениях \( m \) прямая \( y=m \) имеет с графиком ровно две общие точки.
		\item Постройте график функции
		\[y=	 \left\{
		\begin{array}{l}
			x^2+2x+3, \quad x\leq-3,\\
			x+9, \quad x>-3
		\end{array}
		\right. \]
		и определите, при каких значениях \( m \) прямая \( y=m \) имеет с графиком ровно две общие точки.
		\item Построить график функции \( y=x-|2x+1|-2 \). Найти точки пересечения данного графика с графиком функции \( y=x-5 \).
		\item Постройте график функции \( y=x^2-3|x|-x \)  и определите, при каких значениях \( c \)  прямая \( y=c \)  имеет с графиком три общие точки.
	\end{listofex}
\end{class}
%END_FOLD

%BEGIN_FOLD % ====>>_____ Занятие 9 _____<<====
\begin{class}[number=9]
	\begin{listofex}
		\item Решите неравенства:
		\begin{tasks}(2)
			\task \( 2(3x-7)-5x\leq3x-12 \)
			\task \( x-\dfrac{x-3}{4}+\dfrac{x+1}{8}>2 \)
		\end{tasks}
		\item Упростите выражение:
		\begin{tasks}(2)
			\task \( \sqrt{4+2\sqrt{3}} \)
			\task \( \sqrt{9-4\sqrt{5}} \)
			\task \( \sqrt{10+2\sqrt{21}} \)
			\task \( \sqrt{14+6\sqrt{5}}+\sqrt{14-6\sqrt{5}} \)
		\end{tasks}
		\item Упростите выражение и найдите его значение при \( b=-\dfrac{15}{16} \):
		\[\dfrac{64b^2+128b+64}{b}:\left( \dfrac{4}{b}+4 \right)\]
		\item Вычислить, используя алгоритм вычисления квадратного корня:
		\begin{tasks}(3)
			\task \( \sqrt{4096} \)
			\task \( \sqrt{101761} \)
			\task \( \sqrt{632,5225} \)
		\end{tasks}
		\item Найдите значение выражения:
		\[|\sqrt{68}-4|+|\sqrt{68}-4\sqrt{5}|+|\sqrt{80}-10|\]
		\item Докажите тождество:
		\[\left( \dfrac{1}{a^2-4a}+\dfrac{a+3}{a^2-16} \right)\cdot\dfrac{4a-a^2}{a+2}+\dfrac{a+8}{a+4}=\dfrac{6}{a+4}\]
		\item  Постройте график функции
		\[y=	 \left\{
		\begin{array}{l}
			-\dfrac{16}{x}, \quad x<-4\\
			x^2+4x+4, \quad x\geq-4
		\end{array}
		\right. \]
	\end{listofex}
\end{class}
%END_FOLD

%BEGIN_FOLD % ====>>_ Домашняя работа 3 _<<====
\begin{homework}[number=3]
	\begin{listofex}
		\item Упростите выражение:
		\[\dfrac{y^3-9x^2y+x}{xy^2-9x^3}+(1-3x-y)\cdot\left( \dfrac{3x+y+1}{9x^2-y^2}-\dfrac{3x+y}{9x^2-3x+y-y^2} \right)\]
		\item Постройте график:
		\[ y=	 \left\{
		\begin{array}{l}
			1,5x-3, \quad x<2,\\
			-1,5x+3, \quad 2\leq x\leq3,\\
			3x-10,5, \quad x>3.
		\end{array}
		\right. \]
		\item Найдите значение выражения:
		\[|\sqrt{70}-8|+|\sqrt{70}-4\sqrt{5}|+|\sqrt{80}-9|\]
		\item Упростить выражение:
		\[\sqrt{7+4\sqrt{3}}+\sqrt{7-4\sqrt{3}}\]
	\end{listofex}
\end{homework}
%END_FOLD