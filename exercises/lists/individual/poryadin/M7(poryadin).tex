%
%===============>>  Порядин Модуль 7 <<=============
%=
\setmodule{7}

%BEGIN_FOLD % ====>>_____ Занятие 1 _____<<====
\begin{class}[number=1]
	\begin{listofex}
		\item Постройте график функции
		\[y=	 \left\{
		\begin{array}{l}
			2x-2, \quad x<3\\
			-3x+13 \quad 3\leq x\leq4\\
			1,5x-7, \quad x>4.
		\end{array}
		\right. \]
		Определите, при каких значениях \( m \) прямая \( y=m \) имеет с графиком ровно две общие точки.
		\item Постройте график функции
		\[y=	 \left\{
		\begin{array}{l}
			x^2, \quad |x|\leq1\\
			-\dfrac{1}{x}, \quad |x|>1.
		\end{array}
		\right. \]
		При каких значениях параметра \( c \) прямая \( y=c \) имеет с графиком ровно одну общую точку.
		\item Постройте график функции
		\[y=	 \left\{
		\begin{array}{l}
			-x^2-4x-4, \quad x<-1\\
			1-|x-1|, \quad x\geq-1.
		\end{array}
		\right. \]
		При каких значениях параметра \( c \) прямая \( y=c \) имеет с графиком ровно две общие точки.
	\end{listofex}
\end{class}
%END_FOLD

%BEGIN_FOLD % ====>>_____ Занятие 2 _____<<====
\begin{class}[number=2]
	\begin{listofex}
		\item При каких значениях \( m \) вершины парабол \( y =-x^2-6mx+m \) и \( y= x^2-4mx-2 \) расположены по одну сторону от оси \( x \)?		
		\item При каких значениях \( p \) вершины парабол \( y= -x^2+2px+3 \) и\( y= x^2-6px+p \) расположены по разные стороны от оси \( x \)?
		\item При каких значениях \( m \) вершины парабол \( y= x^2-4mx+m \) и \(y=-x^2+8mx+4 \) расположены по одну сторону от оси \( x \)?
	\end{listofex}
\end{class}
%END_FOLD

%BEGIN_FOLD % ====>>_ Домашняя работа 1 _<<====
\begin{homework}[number=1]
	\begin{listofex}
		\item При каких значениях \( m \) вершины парабол \( y=x^2+4mx+2m \) и \( y=-x^2+2mx+4 \) расположены по одну сторону от оси \( x \)?
		\item Упростите выражение и найдите его значение при \( a=-2 \): \quad \( \dfrac{a^2+4a}{a^2+8a+16} \)
	\end{listofex}
\end{homework}
%END_FOLD

%BEGIN_FOLD % ====>>_____ Занятие 3 _____<<====
\begin{class}[number=3]
	\begin{listofex}
		\item Занятие 3 
	\end{listofex}
\end{class}
%END_FOLD

%BEGIN_FOLD % ====>>_____ Занятие 4 _____<<====
\begin{class}[number=4]
	\begin{listofex}
		\item Решите квадратные неравенства:
		\begin{tasks}(2)
			\task \( (x+8)(x-3)>0 \)
			\task \( (x-6)(x-\dfrac{9}{11})\ge0 \)
			\task \( (x+\dfrac{2}{3})(x-\dfrac{5}{6})\le0 \)
			\task \( (-x-0,2)(x+0,6)<0 \)
			\task \( x^2-3x<0 \)
			\task \( x^2-4x>0 \)
			\task \( x^2+2x\ge0 \)
			\task \( x^2+1\le0 \)
		\end{tasks}
		\item Решите квадратные неравенства:
		\begin{tasks}(2)
			\task \( x^2+x-30<0 \)
			\task \( x^2+10x+25\ge0 \)
			\task \( 9x^2-6x+1\le0 \)
			\task \( -4x^2+4x-x\le0 \)
		\end{tasks}
		\item Постройте график функции \( y=-4-\dfrac{x^4-x^3}{x^2-x} \) и определите, при каких значениях \( m \) прямая \( y=m \) имеет с графиком ровно две общие точки.
	\end{listofex}
\end{class}
%END_FOLD

%BEGIN_FOLD % ====>>_ Домашняя работа 2 _<<====
\begin{homework}[number=2]
	\begin{listofex}
			\item Постройте график функции \( y=\dfrac{1-2x}{2x^2-x} \) и определите, при каких значениях \( k \) прямая \( y=kx \) имеет с графиком ровно одну общую точку.
		\item Постройте график функции \( y=3-\dfrac{x+5}{x^2+5x} \) и определите, при каких значениях \( m \) прямая \( y=m \) не имеет с графиком ни одной общей точки.
	\end{listofex}
\end{homework}
%END_FOLD

%BEGIN_FOLD % ====>>_____ Занятие 5 _____<<====
\begin{class}[number=5]
	\begin{listofex}
		\item Занятие 5
		\end{listofex}
\end{class}
%END_FOLD

%BEGIN_FOLD % ====>>_____ Занятие 6 _____<<====
\begin{class}[number=6]
	\begin{listofex}
		\item Занятие 6
	\end{listofex}
\end{class}
%END_FOLD

%BEGIN_FOLD % ====>>_ Домашняя работа 3 _<<====
\begin{homework}[number=3]
	\begin{listofex}
		\item Решите уравнения:
		\begin{tasks}(2)
			\task \( x^3-5x^2-16x+80=0 \)
			\task \( x^3+2x^2-49x-98=0 \)
			\task \( x^4-2x^3+x-2=0 \)
			\task \( x^4-4x^3+5x^2-4x+4=0 \)
		\end{tasks}
		\item Постройте график функции \( y=\dfrac{(x^2+2,25)(x-1)}{1-x} \).
		\item Постройте график функции
		\[y= \left\{
		\begin{array}{l}
			x-3, \quad x<3,\\
			-1,5x+4,5,\quad 3\le x\le4,\\
			1,5x-7,5, \quad x>4,
		\end{array}
		\right.\]
		и определите, при каких значениях \( m \) прямая \( y=m \) имеет с графиком ровно две общие точки.
	\end{listofex}
\end{homework}
%END_FOLD

%BEGIN_FOLD % ====>>_____ Занятие 7 _____<<====
\begin{class}[number=7]
	\title{Подготовка к проверочной}
	\begin{listofex}
		\item Занятие 7
	\end{listofex}
\end{class}
%END_FOLD

%BEGIN_FOLD % ====>>_ Проверочная работа _<<====
\begin{exam}
	\begin{listofex}
		\item Проверочная
	\end{listofex}
\end{exam}
%END_FOLD