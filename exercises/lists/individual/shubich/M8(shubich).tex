%
%===============>>  Шубич Модуль 8 <<=============
%
\setmodule{8}

%BEGIN_FOLD % ====>>_____ Занятие 1 _____<<====
\begin{class}[number=1]
	\begin{listofex}
		\item Вычислите:
		\begin{tasks}(4)
			\task \( \dfrac{4}{25}+\dfrac{15}{4} \)
			\task \( \dfrac{9}{4}+\dfrac{8}{5} \)
			\task \( \dfrac{19}{2}-\dfrac{7}{25} \)
			\task \( \dfrac{3}{2}-\dfrac{9}{5} \)
			\task \( \dfrac{9}{5}\cdot\dfrac{2}{3} \)
			\task \( \dfrac{21}{5}\cdot\dfrac{3}{7} \)
			\task \( \dfrac{14}{5}:\dfrac{7}{2} \)
			\task \( \dfrac{3}{5}:\dfrac{4}{35} \)
		\end{tasks}
		\item Вычислите:
		\begin{tasks}(4)
			\task \( \mfrac{4}{1}{2}-\mfrac{3}{1}{2} \)
			\task \( \mfrac{5}{7}{8}-\mfrac{3}{1}{4} \)
			\task \( \mfrac{6}{7}{9}-\mfrac{2}{1}{3} \)
			\task \( \mfrac{8}{5}{7}-\mfrac{1}{1}{14} \)
			\task \( \mfrac{2}{1}{3}-\mfrac{1}{1}{2} \)
			\task \( \mfrac{7}{1}{8}-\mfrac{2}{3}{4} \)
			\task \( \mfrac{10}{2}{9}-\mfrac{7}{2}{3} \)
			\task \( \mfrac{12}{1}{13}-\mfrac{10}{11}{26} \)
			\task \( 3-\mfrac{2}{1}{2} \)
			\task \( 7-\mfrac{5}{2}{7} \)
			\task \( 10-\mfrac{5}{2}{9} \)
			\task \( 14-\mfrac{10}{2}{3} \)
		\end{tasks}
		\item Вычислите:
		\begin{tasks}(4)
			\task \( \mfrac{2}{1}{3}\cdot\mfrac{7}{2}{5} \)
			\task \( \mfrac{5}{6}{7}\cdot\mfrac{1}{7}{8} \)
			\task \( \mfrac{4}{2}{11}\cdot\mfrac{12}{1}{23} \)
			\task \( \mfrac{5}{1}{6}\cdot\mfrac{8}{2}{13} \)
			\task \( \mfrac{1}{2}{3}:\mfrac{1}{4}{5} \)
			\task \( \mfrac{9}{7}{8}:\mfrac{6}{3}{4} \)
			\task \( \mfrac{10}{5}{6}:\mfrac{6}{1}{3} \)
			\task \( \mfrac{2}{9}{10}:\mfrac{1}{3}{10} \)
		\end{tasks}
		\item Вычислите:
		\begin{tasks}(2)
			\task \( \left( \dfrac{19}{8}+\dfrac{11}{12} \right):\dfrac{5}{48} \)
			\task \( \left( \dfrac{14}{11}+\dfrac{17}{10} \right)\cdot\dfrac{11}{15} \)
			\task \( \left( \mfrac{2}{3}{4}+\mfrac{2}{1}{5} \right)\cdot16 \)
			\task \( \mfrac{1}{8}{17}:\left( \dfrac{12}{17}+\mfrac{2}{7}{11} \right) \)
		\end{tasks}
	\end{listofex}
\end{class}
%END_FOLD

%BEGIN_FOLD % ====>>_____ Занятие 2 _____<<====
\begin{class}[number=2]
	\begin{listofex}
		\item Вычислите:
		\begin{tasks}(4)
			\task \( \mfrac{2}{1}{3}\cdot\mfrac{7}{2}{5} \)
			\task \( \mfrac{5}{6}{7}\cdot\mfrac{1}{7}{8} \)
			\task \( \mfrac{4}{2}{11}\cdot\mfrac{12}{1}{23} \)
			\task \( \mfrac{5}{1}{6}\cdot\mfrac{8}{2}{13} \)
			\task \( \mfrac{1}{2}{3}:\mfrac{1}{4}{5} \)
			\task \( \mfrac{9}{7}{8}:\mfrac{6}{3}{4} \)
			\task \( \mfrac{10}{5}{6}:\mfrac{6}{1}{3} \)
			\task \( \mfrac{2}{9}{10}:\mfrac{1}{3}{10} \)
		\end{tasks}
		\item Вычислите:
		\begin{tasks}(2)
			\task \( \left( \dfrac{19}{8}+\dfrac{11}{12} \right):\dfrac{5}{48} \)
			\task \( \left( \dfrac{14}{11}+\dfrac{17}{10} \right)\cdot\dfrac{11}{15} \)
			\task \( \left( \mfrac{2}{3}{4}+\mfrac{2}{1}{5} \right)\cdot16 \)
			\task \( \mfrac{1}{8}{17}:\left( \dfrac{12}{17}+\mfrac{2}{7}{11} \right) \)
		\end{tasks}
	\end{listofex}
\end{class}
%END_FOLD

%BEGIN_FOLD % ====>>_ Домашняя работа 1 _<<====
\begin{homework}[number=1]
	\begin{listofex}
		\item Вычислите:
		\begin{tasks}(4)
			\task \( 10-\mfrac{4}{5}{12} \)
			\task \( 15-\mfrac{11}{4}{5} \)
			\task \( \mfrac{8}{3}{17}-\mfrac{4}{11}{17} \)
			\task \( \mfrac{4}{1}{9}-\mfrac{2}{5}{9} \)
		\end{tasks}
		\item Вычислите:
		\begin{tasks}(2)
			\task \( \left( \dfrac{19}{8}+\dfrac{11}{12} \right):\dfrac{5}{48} \)
			\task \( \left( \dfrac{14}{11}+\dfrac{17}{10} \right)\cdot\dfrac{11}{15} \)
			\task \( \left( \mfrac{2}{3}{4}+\mfrac{2}{1}{5} \right)\cdot16 \)
			\task \( \mfrac{1}{8}{17}:\left( \dfrac{12}{17}+\mfrac{2}{7}{11} \right) \)
		\end{tasks}
	\end{listofex}
\end{homework}
%END_FOLD

%BEGIN_FOLD % ====>>_____ Занятие 3 _____<<====
\begin{class}[number=3]
	\begin{listofex}
		\item Вычислите:
		\begin{tasks}(2)
			\task \( \left( \dfrac{17}{8}-\dfrac{11}{20} \right):\dfrac{5}{46} \)
			\task \( \mfrac{4}{3}{4}:\left( \mfrac{1}{1}{15}+\dfrac{3}{5} \right) \)
			\task \( \mfrac{1}{1}{12}:\left( \mfrac{1}{13}{18}-\mfrac{2}{5}{9} \right) \)
			\task \( \left( \mfrac{1}{11}{16}-\mfrac{3}{7}{8} \right)\cdot4 \)
			\task \( \left( \dfrac{5}{6}+\mfrac{1}{1}{10} \right)\cdot24 \)
			\task \( \left( \dfrac{10}{13}+\dfrac{15}{4} \right)\cdot\dfrac{26}{5} \)
		\end{tasks}
		\item Выполните сложение:
		\begin{tasks}(4)
			\task \( 0,3+0,2 \)
			\task \( 0,12+0,6 \)
			\task \( 2,101+0,009 \)
			\task \( 9,123+1,2 \)
		\end{tasks}
		\item Выполните вычитание:
		\begin{tasks}(4)
			\task \( 1,23-0,23 \)
			\task \( 2,55-1,63 \)
			\task \( 2,05-1,48 \)
			\task \( 13,9-12,693 \)
		\end{tasks}
		\item Выполните умножение:
		\begin{tasks}(4)
			\task \( 0,6\cdot0,48 \)
			\task \( 1,5\cdot8,99 \)
			\task \( 9,3\cdot7,1 \)
			\task \( 7,01\cdot150,02 \)
		\end{tasks}
		\item Ширина прямоугольника равна \( 5,2 \) см, а его длина на \( 2,15 \) см больше ширины. Периметр прямоугольника на \( 9,7 \) см больше периметра квадрата. Найдите сторону квадрата.
		\item Длина первого отрезка \( 12,9 \) см. Он в \( 3 \) раза длиннее, чем второй, а третий на \( 6,24 \) см короче, чем первый и второй вместе. Найдите длину каждого отрезка.
		\item Если сторону квадрата, периметр которого \( 36,9 \) см, уменьшить в 3 раза, то получится ширина прямоугольника, периметр которого \( 42,8 \) см. Найдите длину этого прямоугольника и вычислите его площадь.
	\end{listofex}
\end{class}
%END_FOLD

%BEGIN_FOLD % ====>>_____ Занятие 4 _____<<====
\begin{class}[number=4]
		\begin{listofex}
			\item Вычислить:
			\begin{tasks}(4)
				\task \( 20,7:9 \)
				\task \( 243,2:8 \)
				\task \( 7,368:24 \)
				\task \( 25:125 \)
				\task \( 1:8 \)
				\task \( 72,57:59 \)
				\task \( 0,7:25 \)
				\task \( 6,78:26 \)
			\end{tasks}
			\item Вычислить:
			\begin{tasks}(4)
				\task \( 45,5:10 \)
				\task \( 45,5:1000 \)
				\task \( 45,5:10000 \)
				\task \( 89:10 \)
				\task \( 89:100 \)
				\task \( 32,2:10 \)
				\task \( 7,98:10 \)
				\task \( 47,7:1000 \)
				\task \( 0,911:1000 \)
			\end{tasks}
			\item Вычислить:
			\begin{tasks}(4)
				\task \( 2:0,4 \)
				\task \( 70:1,75 \)
				\task \( 24:0,2 \)
				\task \( 2:0,5 \)
				\task \( 45:0,05 \)
				\task \( 125:2,5 \)
				\task \( 484:0,004 \)
				\task \( 5,1:0,17 \)
				\task \( 25,2:0,4 \)
				\task \( 200,1:0,69 \)
			\end{tasks}
		\item Длина первого отрезка \( 12,9 \) см. Он в \( 3 \) раза длиннее, чем второй, а третий на \( 6,24 \) см короче, чем первый и второй вместе. Найдите длину каждого отрезка.
		\item Если сторону квадрата, периметр которого \( 36,9 \) см, уменьшить в 3 раза, то получится ширина прямоугольника, периметр которого \( 42,8 \) см. Найдите длину этого прямоугольника и вычислите его площадь.
		\item Периметр квадрата равен \( 17,2 \). Найдите его сторону и площадь.
		\item Периметр треугольника равен \( 72,9 \), а его стороны относятся как \( 2:3:4 \). Найдите длины его сторон.
		\item Периметр квадрата равен \( 17,2 \). Найдите его сторону и площадь.
		\item Периметр треугольника равен \( 72,9 \), а его стороны относятся как \( 2:3:4 \). Найдите длины его сторон.
		\item Выполните действия:
		\begin{tasks}(1)
			\task \( 11,47+(3,89-2,11)-4,416+3,711 \)
			\task \( 3,16+(7,84-4,181)-3,11+14,816 \)
			\task \( 1,49+(6,13-4,12)-0,5+7,289 \)
		\end{tasks}
	\end{listofex}
\end{class}
%END_FOLD

%BEGIN_FOLD % ====>>_ Домашняя работа 2 _<<====
\begin{homework}[number=2]
	\begin{listofex}
		\item 
		\begin{tasks}(1)
			\task \( 8,102+0,97 \)
			\task \( 3,89+4,91 \)
			\task \( 17,22+13,25 \)
			\task \( 9,102+6,898 \)
			\task \( 78,5-74,123 \)
			\task \( 10-9,621 \)
			\task \( 64,56-9,123 \)
			\task \( 30,02-16,88 \)
			\task \( 100,23\cdot8,96 \)
			\task \( 6,12\cdot7,36 \)
			\task \( 2,39\cdot7,12 \)
			\task \( 19,03\cdot0,002 \)
		\end{tasks}
	\end{listofex}
\end{homework}
%END_FOLD

%BEGIN_FOLD % ====>>_____ Занятие 5 _____<<====
\begin{class}[number=5]
	\begin{listofex}
		\item Занятие 5
	\end{listofex}
\end{class}
%END_FOLD

%BEGIN_FOLD % ====>>_____ Занятие 6 _____<<====
\begin{class}[number=6]
	\begin{listofex}
		\item Занятие 6
	\end{listofex}
\end{class}
%END_FOLD

%BEGIN_FOLD % ====>>_ Домашняя работа 3 _<<====
\begin{homework}[number=3]
	\begin{listofex}
		\item Домашняя работа 3
	\end{listofex}
\end{homework}
%END_FOLD

%BEGIN_FOLD % ====>>_____ Занятие 7 _____<<====
\begin{class}[number=7]
	\title{Подготовка к проверочной}
	\begin{listofex}
		\item Занятие 7
	\end{listofex}
\end{class}
%END_FOLD

=%BEGIN_FOLD % ====>>_ Проверочная работа _<<====
\begin{exam}
	\begin{listofex}
		\item Проверочная
	\end{listofex}
\end{exam}
%END_FOLD