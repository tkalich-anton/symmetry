%
%===============>>  Симонов Модуль 8 <<=============
%
\setmodule{8}

%BEGIN_FOLD % ====>>_____ Занятие 1 _____<<====
\begin{class}[number=1]
	\begin{listofex}
		\item Упростите выражение:
		\[\dfrac{\sqrt{\sqrt{10}-2}\cdot\sqrt{\sqrt{10}+2}}{\sqrt{24}}\]
		\item Сократите дроби:
		\begin{tasks}(2)
			\task \( \dfrac{5^{n+1}-5^{n-1}}{2\cdot5^n} \)
			\task \( \dfrac{18^{n+3}}{3^{2n+5}\cdot2^{n-2}} \)
		\end{tasks}
		\item Моторная лодка прошла \( 36 \) км по течению реки и вернулась обратно, потратив на весь путь \( 5 \) часов. Скорость течения реки равна \( 3 \) км/ч. Найдите скорость лодки в неподвижной воде.
		\item Баржа прошла по течению реки \( 48 \) км и, повернув обратно, прошла ещё \( 36 \) км, затратив на весь путь \( 6 \) часов. Найдите собственную скорость баржи, если скорость течения реки равна \( 5 \) км/ч.
		\item Теплоход проходит по течению реки до пункта назначения \( 280 \) км и после стоянки возвращается в пункт отправления. Найдите скорость теплохода в неподвижной воде, если скорость течения равна \( 4 \) км/ч, стоянка длится \( 15 \) часов, а в пункт отправления теплоход возвращается через \( 39 \) часов после отплытия из него.
		\item Из пунктов \( A \) и \( B \), расстояние между которыми \( 19 \) км, вышли одновременно навстречу друг другу два пешехода и встретились в \( 9 \) км от \( A \). Найдите скорость пешехода, шедшего из \( A \), если известно, что он шёл со скоростью, на \( 1 \) км/ч большей, чем пешеход, шедший из \( B \), и сделал в пути получасовую остановку.
		\item Расстояние между городами \( A \) и \( B \) равно \( 750 \) км. Из города \( A \) в город \( B \) со скоростью \( 50 \) км/ч выехал первый автомобиль, а через три часа после этого навстречу ему из города \( B \) выехал со скоростью \( 70 \) км/ч второй автомобиль. На каком расстоянии от города \( A \) автомобили встретятся?
	\end{listofex}
\end{class}
%END_FOLD

%BEGIN_FOLD % ====>>_____ Занятие 2 _____<<====
\begin{class}[number=2]
	\begin{listofex}
	\item Сократите дроби:
		\begin{tasks}(2)
		\task \( \dfrac{2^{n+2}\cdot21^{n+3}}{6^{n+1}\cdot7^{n+2}} \)
		\task \( \dfrac{18^{n+3}}{3^{2n+5}\cdot2^{n-2}} \)
		\end{tasks}
	\item Туристы проплыли на лодке от лагеря некоторое расстояние вверх по течению реки, затем причалили к берегу и, погуляв \( 2 \) часа, вернулись обратно через 6 часов от начала путешествия. На какое расстояние от лагеря они отплыли, если скорость течения реки равна \( 3 \) км/ч, а собственная скорость лодки \( 6 \) км/ч?
	\item Рыболов проплыл на лодке от пристани некоторое расстояние вверх по течению реки, затем бросил якорь, \( 2 \) часа ловил рыбу и вернулся обратно через \( 5 \) часов от начала путешествия. На какое расстояние от пристани он отплыл, если скорость течения реки равна \( 2 \) км/ч, а собственная скорость лодки \( 6 \) км/ч?
	\item Теплоход проходит по течению реки до пункта назначения \( 280 \) км и после стоянки возвращается в пункт отправления. Найдите скорость теплохода в неподвижной воде, если скорость течения равна \( 4 \) км/ч, стоянка длится \( 15 \) часов, а в пункт отправления теплоход возвращается через \( 39 \) часов после отплытия из него.
	\item От пристани \( A \) к пристани \( B \), расстояние между которыми равно \( 70 \) км, отправился с постоянной скоростью первый теплоход, а через \( 1 \) час после этого следом за ним, со скоростью, на \( 8 \) км/ч большей, отправился второй. Найдите скорость первого теплохода, если в пункт \( B \) оба теплохода прибыли одновременно.
	\item Катер прошёл от одной пристани до другой, расстояние между которыми по реке равно \( 48 \) км, сделал стоянку на \( 20 \) мин и вернулся обратно через \( \mfrac{2}{1}{3} \) ч после начала поездки. Найдите скорость течения реки, если известно, что скорость катера в стоячей воде равна \( 20 \) км/ч.	
	\end{listofex}
\end{class}
%END_FOLD

%BEGIN_FOLD % ====>>_ Домашняя работа 1 _<<====
\begin{homework}[number=1]
	\begin{listofex}
		\item Миша с папой решили покататься на колесе обозрения. Всего на колесе двадцать четыре кабинки, из них \( 5 \) --- синие, \( 7 \) --- зеленые, остальные --- красные. Кабинки по очереди подходят к платформе для посадки. Найдите вероятность того, что Миша прокатится в красной кабинке.
		\item В лыжных гонках участвуют \( 7 \) спортсменов из России, \( 1 \) спортсмен из Швеции и \( 2 \) спортсмена из Норвегии. Порядок, в котором спортсмены стартуют, определяется жребием. Найдите вероятность того, что спортсмен из Швеции будет стартовать последним.
		\item Сократите дробь: \[\dfrac{100^n}{5^{2n-1}\cdot4^{n-2}}\]
		\item Теплоход проходит по течению реки до пункта назначения \( 140 \) км и после стоянки возвращается в пункт отправления. Найдите скорость течения, если скорость теплохода в неподвижной воде равна \( 15 \) км/ч, стоянка длится \( 11 \) часов, а в пункт отправления теплоход возвращается через \( 32 \) часа после отплытия из него.
		\item Моторная лодка прошла против течения реки \( 132 \) км и вернулась в пункт отправления, затратив на обратный путь на \( 5 \) часов меньше, чем на путь против течения. Найдите скорость лодки в неподвижной воде, если скорость течения реки равна \( 5 \) км/ч.
	\end{listofex}
\end{homework}
%END_FOLD

%BEGIN_FOLD % ====>>_____ Занятие 3 _____<<====
\begin{class}[number=3]
	\begin{listofex}
		\item Решите уравнение:
		\((2x-2)^2(x-2)=(2x-2)(x-2)^2\)
		\item Решите уравнения: 
		\begin{tasks}(2)
			\task \((x^2-4)+(x^2-6x-16)^2=0\)
			\task \( (x-1)(x^2+6x+9)=5(x+3) \)
		\end{tasks}
		\item Решите уравнение: \quad \( (x-2)^4+3(x-2)^2-10=0 \)
		\item Решите уравнение: \quad \( |x^2-3x-18|+|x-6|=0 \)
		\item Из пунктов \( A \) и \( B \), расстояние между которыми \( 19 \) км, вышли одновременно навстречу друг другу два пешехода и встретились в \( 9 \) км от \( A \). Найдите скорость пешехода, шедшего из \( A \), если известно, что он шёл со скоростью, на \( 1 \) км/ч большей, чем пешеход, шедший из \( B \), и сделал в пути получасовую остановку.
		\item Расстояние между городами \( A \) и \( B \) равно \( 750 \) км. Из города \( A \) в город \( B \) со скоростью \( 50 \) км/ч выехал первый автомобиль, а через три часа после этого навстречу ему из города \( B \) выехал со скоростью \( 70 \) км/ч второй автомобиль. На каком расстоянии от города \( A \) автомобили встретятся?
	\end{listofex}
\end{class}
%END_FOLD

%BEGIN_FOLD % ====>>_____ Занятие 4 _____<<====
\begin{class}[number=4]
	\begin{listofex}
		\item Решите уравнения:
		\begin{tasks}(1)
			\task \( (x-4)(x-5)(x-6)=(x-2)(x-5)(x-6) \)
			\task \( (x-2)(x-3)(x-4)=(x-2)(x-4)(x-5) \)
		\end{tasks}
		\item Решите уравнения:
		\begin{tasks}(2)
			\task \( (x+1)^4-(x+1)^2-6=0 \)
			\task \( (x-4)^4-4(x-4)^2-21=0 \)
			\task \( \dfrac{1}{x^2}-\dfrac{1}{x}-6=0 \)
			\task \( \dfrac{1}{(x-1)^2}+\dfrac{2}{x-1}-3=0 \)
		\end{tasks}
		\item Решите уравнения:
		\begin{tasks}(2)
			\task \( \dfrac{2x^2+7x-4}{x^2-16}=1 \)
			\task \( \dfrac{2x^2+3x-2}{x^2-4}=1 \)
		\end{tasks}
	\end{listofex}
\end{class}
%END_FOLD

%BEGIN_FOLD % ====>>_ Домашняя работа 2 _<<====
\begin{homework}[number=2]
	\begin{listofex}
		\item Найдите значение выражения: \[ \dfrac{25n^{\frac{1}{3}}}{n^{\frac{1}{4}}\cdot n^{\frac{1}{12}}} \]
		\item Туристы проплыли на лодке от лагеря некоторое расстояние вверх по течению реки, затем причалили к берегу и, погуляв \( 3 \) часа, вернулись обратно через \( 5 \) часов от начала путешествия. На какое расстояние от лагеря они отплыли, если скорость течения реки равна \( 3 \) км/ч, а собственная скорость лодки \( 6 \) км/ч?
		\item Теплоход проходит по течению реки до пункта назначения \( 140 \) км и после стоянки возвращается в пункт отправления. Найдите скорость течения, если скорость теплохода в неподвижной воде равна \( 15 \) км/ч, стоянка длится \( 11 \) часов, а в пункт отправления теплоход возвращается через \( 32 \) часа после отплытия из него.
	\end{listofex}
\end{homework}
%END_FOLD

%BEGIN_FOLD % ====>>_____ Занятие 5 _____<<====
\begin{class}[number=5]
	\begin{listofex}
		\item Занятие 5
	\end{listofex}
\end{class}
%END_FOLD

%BEGIN_FOLD % ====>>_____ Занятие 6 _____<<====
\begin{class}[number=6]
	\begin{listofex}
		\item Занятие 6
	\end{listofex}
\end{class}
%END_FOLD

%BEGIN_FOLD % ====>>_ Домашняя работа 4 _<<====
\begin{homework}[number=4]
	\begin{listofex}
<<<<<<< HEAD
		\item  Решите уравнения:
		\begin{tasks}(2)
			\task \( x^4-5x^2+4=0 \)
			\task \( (x^2-25)^2+(x^2+3x-10)^2=0 \)
			\task! \( (x-3)(x-4)(x-5)=(x-2)(x-4)(x-5) \)
			\task \( (x+3)^2+2(x+3)^2-8=0 \)
			\task \( |x^2-49|+|x^2+4x-21|=0 \)
		\end{tasks}
		\item Рабочие прокладывают тоннель длиной \( 500 \) метров, ежедневно увеличивая норму прокладки на одно и то же число метров. Известно, что за первый день рабочие проложили \( 3 \) метра тоннеля. Определите, сколько метров тоннеля проложили рабочие в последний день, если вся работа была выполнена за \( 10 \) дней.
		\item В первом ряду кинозала \( 24 \) места, а в каждом следующем на \( 2 \) больше, чем в предыдущем. Сколько мест в восьмом ряду?
		\item Бактерия, попав в живой организм, к концу \( 20 \)-й минуты делится на две бактерии, каждая из них к концу следующих \( 20 \) минут делится опять на две и т. д. Сколько бактерий окажется в организме через \( 5 \) часов, если по истечении четвертого часа в организм из окружающей среды попала еще одна бактерия?
=======
		\item Решите уравнение:
		\begin{tasks}(2)
			\task \( \dfrac{3}{x-19}=\dfrac{19}{x-3} \)
			\task \( \dfrac{x-4}{x-6}=2 \)
		\end{tasks}
		\item Решите уравнение:
		\begin{tasks}(1)
			\task \( (x-2)^{2}(x-3)=20(x-2) \)
			\task \( (x-3)(x-4)(x-5)=(x-2)(x-3)(x-5) \)
		\end{tasks}
		\item Решите уравнение:
		\begin{tasks}(2)
			\task \( (x+3)^{4}+2(x+3)^{2}-8=0 \)
			\task \( (x+1)^{4}+(x+1)^{2}-6=0 \)
		\end{tasks}
>>>>>>> bb27d0be6fe3f76ff3588dce56c92fa594401d3f
	\end{listofex}
\end{homework}
%END_FOLD

%BEGIN_FOLD % ====>>_____ Занятие 7 _____<<====
\begin{class}[number=7]
	\title{Подготовка к проверочной}
	\begin{listofex}
		\item Занятие 7
	\end{listofex}
\end{class}
%END_FOLD

%BEGIN_FOLD % ====>>_ Проверочная работа _<<====
\begin{exam}
	\begin{listofex}
		\item Проверочная
	\end{listofex}
\end{exam}
%END_FOLD