%
%===============>>  Гусев Модуль 7 <<=============
%
\setmodule{7}

%BEGIN_FOLD % ====>>_____ Занятие 1 _____<<====
\begin{class}[number=1]
	\begin{definit}
	\textbf{Синусом} острого угла прямоугольного треугольника называется отношение противолежащего катета к гипотенузе.
\end{definit}
\begin{definit}
	\textbf{Косинусом} острого угла прямоугольного треугольника называется отношение прилежащего катета к гипотенузе.
\end{definit}
\begin{definit}
	\textbf{Тангенсом} острого угла прямоугольного треугольника называется отношение противолежащего катета к прилежащему катету.
\end{definit}
\begin{definit}
	\textbf{Основное тригонометрическое тождество:} \[\sin^2\alpha+\cos^2\alpha=1\]
\end{definit}
\begin{listofex}
	\item В треугольнике \( ABC \) угол \( C \) равен \( 90 \) градусов, \( AC=6 \), \( AB=20 \). Найдите \( \sin B \).
	\item  В треугольнике \( ABC \) угол \( C \) равен \( 90 \) градусов, \( BC=9 \), \( AB=20 \). Найдите \( \cos B \).
	\item В треугольнике \( ABC \) угол \( C \) равен \( 90 \) градусов, \( BC=9 \), \( AC=27 \). Найдите \( \tg B \).
	\item Найдите синус, косинус и тангенс углов \( A \) и \( B \) треугольника \( ABC \) с прямым углом \( C \), если:
	\begin{tasks}(2)
		\task \( BC=8 \), \( AB=17 \)
		\task \( BC=21 \), \( AC=20 \)
		\task \( BC=1 \), \( AC=2 \)
	\end{tasks}
	\item Найдите:
	\begin{tasks}(1)
		\task \( \sin\alpha \) и \( \tg\alpha \), если \( \cos\alpha=\dfrac{1}{2} \)
		\task \( \cos\alpha \) и \( \tg\alpha \), если \( \sin\alpha=\dfrac{\sqrt{3}}{2} \)
	\end{tasks}
	\item В треугольнике \( ABC \) угол \( C \) прямой, \( BC=8 \), \( \sin A=0,4 \). Найдите \( AB \).
	\item В треугольнике \( ABC \) угол \( C \) прямой, \( AC=15 \), \( \cos A=\dfrac{5}{7} \). Найдите \( AB \).
	\item В треугольнике \( ABC \) угол \( C \) равен \( 90\degree \), \( BC=12 \), \( \sin A=\dfrac{4}{11} \). Найдите \( AB \).
	\item Катеты прямоугольного треугольника равны \( \sqrt{15} \) и \( 1 \). Найдите синус наименьшего угла этого треугольника.
	\item Площадь прямоугольного треугольника равна \( 32\sqrt{3} \). Один из острых углов равен \( 30\degree \). Найдите длину гипотенузы.
	\item В треугольнике \( ABC \) угол \( C \) равен \( 90\degree \), \( AC=12 \), \( \tg A=\dfrac{2\sqrt{10}}{3} \).  Найдите \( AB \).
	\item В треугольнике \( ABC \) угол \( C \) равен \( 90\degree \), \( \sin A=\dfrac{4}{5} \), \( AC=9 \). Найдите \( AB \).
	\item Найдите синус меньшего острого угла между	диагональю прямоугольника и его стороной, если периметр прямоугольника равен \( 34 \) см, а одна из сторон --- \( 12 \) см. 
	\item Тангенс острого угла прямоугольного треугольника равен \( \dfrac{2}{5} \), а один из катетов на \( 6 \) см больше другого. Найдите площадь треугольника. 
	\item Основание равнобедренного треугольника равно \( 8 \) см, тангенс угла при основании равен \( 2 \). Найдите площадь треугольника. 
	\item Периметр равнобедренного треугольника равен \( 64 \) см, косинус угла при основании равен \( 0,6 \). Найдите площадь треугольника.
	\end{listofex}
\end{class}
%END_FOLD

%BEGIN_FOLD % ====>>_ Домашняя работа 1 _<<====
\begin{homework}[number=1]
	\begin{listofex}
		\item ,
	\end{listofex}
\end{homework}
%END_FOLD

%BEGIN_FOLD % ====>>_____ Занятие 2 _____<<====
\begin{class}[number=2]
	\begin{listofex}
		\item Площадь прямоугольного треугольника равна \( 32\sqrt{3} \). Один из острых углов равен \( 30\degree \). Найдите длину гипотенузы.
		\item В треугольнике \( ABC \) угол \( C \) равен \( 90\degree \), \( AC=12 \), \( \tg A=\dfrac{2\sqrt{10}}{3} \).  Найдите \( AB \).
		\item В треугольнике \( ABC \) угол \( C \) равен \( 90\degree \), \( \sin A=\dfrac{4}{5} \), \( AC=9 \). Найдите \( AB \).
		\item Найдите синус меньшего острого угла между	диагональю прямоугольника и его стороной, если периметр прямоугольника равен \( 34 \) см, а одна из сторон --- \( 12 \) см. 
		\item Тангенс острого угла прямоугольного треугольника равен \( \dfrac{2}{5} \), а один из катетов на \( 6 \) см больше другого. Найдите площадь треугольника. 
		\item Основание равнобедренного треугольника равно \( 8 \) см, тангенс угла при основании равен \( 2 \). Найдите площадь треугольника. 
		\item Периметр равнобедренного треугольника равен \( 64 \) см, косинус угла при основании равен \( 0,6 \). Найдите площадь треугольника.
		\item Найдите значение выражения:
		\begin{tasks}(2)
			\task \( \dfrac{3^8\cdot3^5}{3^9} \)
			\task \( \dfrac{24^4}{3^2\cdot8^3} \)
			\task \( (3\sqrt{2})^2 \)
			\task \( 4^{-10}\cdot(4^3)^4 \)
			\task \( 3\cdot10^{-1}+1\cdot10^{-2}+5\cdot10^{-4} \)
			\task \( \dfrac{1}{4^{-10}}\cdot\dfrac{1}{4^{9}} \)
		\end{tasks}
		\item Решите уравнение:
			\[\dfrac{2x-3x^2}{5}-\dfrac{7x^2-x}{4}=\dfrac{x^2}{2} \]
	\end{listofex}
\end{class}
%END_FOLD

%BEGIN_FOLD % ====>>_ Домашняя работа 2 _<<====
\begin{homework}[number=2]
	\begin{listofex}
		\item В треугольнике \( ABC \) угол \( C \) прямой, \( BC=12 \), \( \sin A=\dfrac{6}{13} \). Найдите \( AB \).
		\item В треугольнике \( ABC \) угол \( C \) прямой, \( BC=15 \), \( \cos B=\dfrac{18}{30} \). Найдите \( AB \).
		\item Найдите синус, косинус и тангенс углов \( A \) и \( B \) треугольника \( ABC \) с прямым углом \( C \), если:
		\begin{tasks}(2)
			\task \( BC=4 \), \( AB=12 \)
			\task \( BC=3 \), \( AC=5 \)
			\task \( BC=13 \), \( AC=24 \)
		\end{tasks}
		\item Вычислить \( \sin \alpha \) и \( \tg\alpha \), если \( \cos \alpha =\dfrac{3}{5} \).
	\end{listofex}
\end{homework}
%END_FOLD

%BEGIN_FOLD % ====>>_____ Занятие 3 _____<<====
\begin{class}[number=3]
	\begin{listofex}
		\item 
	\end{listofex}
\end{class}
%END_FOLD

%BEGIN_FOLD % ====>>_ Домашняя работа 3 _<<====
\begin{homework}[number=3]
	\begin{listofex}
		\item В треугольнике \( ABC \) угол \( C \) равен \( 90\degree \), \( AC=9 \), \( AB=25 \). Найдите \( \sin B \).
		\item В треугольнике \( ABC \) угол \( C \) равен \( 90\degree \), \( BC=4 \), \( AC=28 \). Найдите \( \tg B \).
		\item В треугольнике \( ABC \) угол \( C \) равен \( 90\degree \), \( \sin B=\dfrac{5}{17} \), \( AB=51 \). Найдите \( AC \).
		\item В треугольнике\( ABC \) угол \( C \) равен \( 90\degree \), \( \tg B=\dfrac{8}{5} \), \( BC=20 \). Найдите \( AC \).
		\item Синус острого угла \( A \) треугольника \( ABC \) равен \( \dfrac{2\sqrt{6}}{5} \). Найдите \( \cos A \).
		\item В равнобедренном треугольнике \( ABC \) \( AB=BC \). Угол \( B \) равен \( 120\degree \), а высота, опущенная на основание равна \( \dfrac{\sqrt{3}}{2} \). Найдите площадь треугольника.
		\item Вычилсите:
		\begin{tasks}(2)
			\task \( \dfrac{125^6}{25^8} \)
			\task \( \dfrac{1}{5^{-8}}\cdot\dfrac{1}{5^6} \)
		\end{tasks}
	\end{listofex}
\end{homework}
%END_FOLD

%BEGIN_FOLD % ====>>_____ Занятие 4 _____<<====
\begin{class}[number=4]
	\begin{listofex}
		\item Пусто
	\end{listofex}
\end{class}
%END_FOLD


%BEGIN_FOLD % ====>>_ Проверочная работа _<<====
\begin{exam}
	\begin{listofex}
		\item Проверочная
	\end{listofex}
\end{exam}
%END_FOLD