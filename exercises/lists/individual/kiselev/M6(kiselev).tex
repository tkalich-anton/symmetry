%
%===============>>  Киселев Модуль 6 <<=============
%
\setmodule{6}

%BEGIN_FOLD % ====>>_____ Занятие 1 _____<<====
\begin{class}[number=1]
	\begin{listofex}
		\item Вычислить:
		\begin{tasks}(1)
			\task \( \dfrac{\left( 7\dfrac{1}{3} \right)^2-\left( 2\dfrac{2}{3} \right)^2}{\left( 5\dfrac{7}{9} \right)^2-\left( 4\dfrac{2}{9} \right)^2} \)
			\task \( \dfrac{42,5904:6,08-1,245}{(18,2^2-5,6^2+23,8\cdot7,4):5,95+35,2} \)
		\end{tasks}
		\item В фирме такси в наличии \( 50 \) легковых автомобилей; \( 27 \) из них чёрного цвета с жёлтыми надписями на бортах, остальные --- жёлтого цвета с чёрными надписями. Найдите вероятность того, что на случайный вызов приедет машина жёлтого цвета с чёрными надписями.
		\item Перед началом первого тура чемпионата по бадминтону участников разбивают на игровые пары случайным образом с помощью жребия. Всего в чемпионате участвует \( 26 \) бадминтонистов, среди которых \( 16 \) спортсменов из России, в том числе Тарас Селезнёв. Найдите вероятность того, что в первом туре Тарас Селезнёв будет играть с каким-либо бадминтонистом из России.
		\item В кармане у Саши было четыре конфеты --- «Маска», «Василёк», «Взлётная» и «Коровка», а так же ключи от квартиры. Вынимая ключи, Саша случайно выронил из кармана одну конфету. Найдите вероятность того, что потерялась конфета «Василёк».
		\item Какова вероятность того, что случайно выбранный телефонный номер оканчивается двумя чётными цифрами?
		\item Из районного центра в деревню ежедневно ходит автобус. Вероятность того, что в понедельник в автобусе окажется меньше \( 18 \) пассажиров, равна \( 0,82 \). Вероятность того, что окажется меньше \( 10 \) пассажиров, равна \( 0,51 \). Найдите вероятность того, что число пассажиров будет от \( 10 \) до \( 17 \).
		\item Вероятность того, что новый электрический чайник прослужит больше года, равна \( 0,97 \). Вероятность того, что он прослужит больше двух лет, равна \( 0,89 \). Найдите вероятность того, что он прослужит меньше двух лет, но больше года.
		\item Биатлонист пять раз стреляет по мишеням. Вероятность попадания в мишень при одном выстреле равна \( 0,8 \). Найдите вероятность того, что биатлонист первые три раза попал в мишени, а последние два промахнулся. Результат округлите до сотых.
		\item В магазине три продавца. Каждый из них занят с клиентом с вероятностью \( 0,3 \). Найдите вероятность того, что в случайный момент времени все три продавца заняты одновременно (считайте, что клиенты заходят независимо друг от друга).
		\item На экзамене по геометрии школьник отвечает на один вопрос из списка экзаменационных вопросов. Вероятность того, что это вопрос по теме «Вписанная окружность», равна \( 0,2 \). Вероятность того, что это вопрос по теме «Параллелограмм», равна \( 0,15 \). Вопросов, которые одновременно относятся к этим двум темам, нет. Найдите вероятность того, что на экзамене школьнику достанется вопрос по одной из этих двух тем.
		\item Два велосипедиста одновременно отправились в \( 154 \)-километровый пробег. Первый ехал со скоростью, на \( 3 \) км/ч большей, чем скорость второго, и прибыл к финишу на \( 3 \) часа раньше второго. Найти скорость велосипедиста, пришедшего к финишу вторым. Ответ дайте в км/ч. 
		
	\end{listofex}
\end{class}
%END_FOLD

%BEGIN_FOLD % ====>>_____ Занятие 2 _____<<====
\begin{class}[number=2]
	\begin{listofex}
		\item Занятие 2
	\end{listofex}
\end{class}
%END_FOLD

%BEGIN_FOLD % ====>>_ Домашняя работа 1 _<<====
\begin{homework}[number=1]
	\begin{listofex}
		\item Поезд, двигаясь равномерно со скоростью \( 60 \) км/ч, проезжает мимо придорожного столба за \( 57 \) секунд. Найдите длину поезда в метрах.
		\item Из двух городов, расстояние между которыми \( 720 \) км, по параллельным путям отправляются навстречу друг другу два поезда и встречаются на середине пути. Второй поезд вышел на \( 1 \) ч позже первого со скоростью, на \( 4 \) км/ч большей, чем скорость первого поезда. Найдите скорость второго поезда. Ответ дайте в км/ч.
	\end{listofex}
\end{homework}
%END_FOLD

%BEGIN_FOLD % ====>>_____ Занятие 3 _____<<====
\begin{class}[number=3]
	\begin{listofex}
		\item Занятие 3 
	\end{listofex}
\end{class}
%END_FOLD

%BEGIN_FOLD % ====>>_____ Занятие 4 _____<<====
\begin{class}[number=4]
	\begin{listofex}
		\item Занятие 4
	\end{listofex}
\end{class}
%END_FOLD

%BEGIN_FOLD % ====>>_ Домашняя работа 2 _<<====
\begin{homework}[number=2]
	\begin{listofex}
		\item Домашняя работа 2
	\end{listofex}
\end{homework}
%END_FOLD

%BEGIN_FOLD % ====>>_____ Занятие 5 _____<<====
\begin{class}[number=5]
	\begin{listofex}
		\item Занятие 5
	\end{listofex}
	\end{class}
	%END_FOLD
	
	%BEGIN_FOLD % ====>>_____ Занятие 6 _____<<====
	\begin{class}[number=6]
		\begin{listofex}
			\item Занятие 6
		\end{listofex}
	\end{class}
	%END_FOLD
	
	%BEGIN_FOLD % ====>>_ Домашняя работа 3 _<<====
	\begin{homework}[number=3]
		\begin{listofex}
			\item Домашняя работа 3
		\end{listofex}
	\end{homework}
	%END_FOLD
	
	%BEGIN_FOLD % ====>>_____ Занятие 7 _____<<====
	\begin{class}[number=7]
		\title{Подготовка к проверочной}
		\begin{listofex}
			\item Занятие 7
		\end{listofex}
	\end{class}
	%END_FOLD
	
	%BEGIN_FOLD % ====>>_ Проверочная работа _<<====
	\begin{exam}
		\begin{listofex}
			\item Проверочная
		\end{listofex}
	\end{exam}
	%END_FOLD