%
%===============>>  Киселев Модуль 7 <<=============
%
\setmodule{7}

%BEGIN_FOLD % ====>>_____ Занятие 1 _____<<====
\begin{class}[number=1]
	\begin{listofex}
		\item Занятие 1
	\end{listofex}
\end{class}
%END_FOLD

%BEGIN_FOLD % ====>>_____ Занятие 2 _____<<====
\begin{class}[number=2]
	\begin{listofex}
		%11.1v2v3 po 4-7
		\item Найдите точку максимума функции \( y=x^3-3x^2+2 \).
		\item Найдите точку минимума функции \( y=2x^3-5x^2+25 \).
		\item Найдите наименьшее значение функции \(y=x^3-3x^2+2\) на отрезке \( [1;4] \).
		\item Найдите наибольшее значение функции \(y=x^3-6x^2\) на отрезке \( [-3;3] \).
		\item Найдите точку максимума функции \( y=x^3-48x+17 \).
		\item Найдите наименьшее значение функции \( y=x^3-27 \) на отрезке \([0;4]\).
		\item Найдите наибольшее значение функции \( y=x^3-3x+4 \) на отрезке \([-2;0]\).
		
		%5-8 analog 282861
		\item Найдите наименьшее значение функции \( y=(x-3)^2(x-6)-1 \) на отрезке \( [4;6] \).
		\item Найдите наименьшее значение функции \( y=(x-6)^2(x+4)-3 \) на отрезке \( [1;11] \).
		\item Найдите наименьшее значение функции \( y=(x-7)^2(x-6)+6 \) на отрезке \( [6,5;19] \).
		\item Найдите наименьшее значение функции \( y=(x-10)^2(x+4)+7 \) на отрезке \( [2;14] \).
		\item Найдите наименьшее значение функции \( y=(x+3)^2(x+5)-1 \) на отрезке \([-4;-1]\).
		\item Найдите наименьшее значение функции \( y=(x-7)^2(x+6) \) на отрезке \([-1;20]\).
		\item Найдите наименьшее значение функции \( y=(x-8)^2(x-1)+10 \) на отрезке \([6;14]\).
		\item Найдите наибольшее значение функции \( y=(x-2)^2(x-4)+5 \) на отрезке \([1;3]\).
		
		\item Найдите наибольшее значение функции \( y=\dfrac{ x^2+25 }{ x } \) на отрезке \( [-10;-1] \).
		\item Найдите точку максимума функции \( y=\dfrac{ 16 }{ x }+x+3 \).
		\item Найдите минимума функции функции \( y=\dfrac{ 25 }{ x }+x+25 \).
		\item Найдите наименьшее значение функции \( y=x+\dfrac{ 36 }{ x } \) на отрезке \( [1;9] \).
		\item Найдите точку максимума функции \( y=-\dfrac{ x^2+289 }{ x } \).
		\item Найдите наименьшее значение функции \( y=\dfrac{x^2+25  }{ x } \) на отрезке \([1;10]\).
		\item Найдите точку минимума функции \( y=-\dfrac{ x^2+1 }{ x } \).
		
		\item Найдите точку минимума функции \( y=(3-x)e^{3-x} \).
		\item Найдите точку максимума функции \( y=(x+16)e^{16-x} \).
		\item Найдите точку минимума функции \( y=(3x^2-36x+36)e^{x-36} \).
		\item Найдите точку максимума функции \( y=(x^2-12x+12)e^{x+12} \).
		\item Найдите наименьшее значение функции \( y=(x-8)e^{x-7} \) на отрезке \([6;8]\).
		\item Найдите точку минимума функции \( y=(x+16)e^{x-16} \).
		\item Найдите точку максимума функции \( y=(9-x)e^{x+9} \).
	\end{listofex}
\end{class}
%END_FOLD

%BEGIN_FOLD % ====>>_ Домашняя работа 1 _<<====
\begin{homework}[number=1]
	\begin{listofex}
		\item Вычислите:
		\begin{tasks}(3)
			\task \( 24\sqrt{3}\cos(-750\degree) \)
			\task \( -12\sqrt{2}\sin(225\degree) \)
			\task \( \dfrac{6\cos59\degree}{\sin31\degree} \)
			\task \( \dfrac{43\cos64\degree}{\sin36\degree} \)
			\task \( \dfrac{25}{\sin\left( -\frac{25\pi}{4} \right)\cos\left( \frac{25\pi}{4} \right)} \)
			\task \( \dfrac{47}{\sin\left( -\frac{35pi}{4} \right)\cos\left( \frac{35\pi}{4} \right)} \)
		\end{tasks}
		\item Моторная лодка прошла против течения реки \( 255 \) км и вернулась в пункт отправления, затратив на обратный путь на \( 2 \) часа меньше. Найдите скорость лодки в неподвижной воде, если скорость течения равна \( 1 \) км/ч. Ответ дайте в км/ч.
		\item В треугольнике \( ABC \) \( AC=BC=18 \), \( \sin B=\dfrac{\sqrt{15}}{4} \). Найдите \( AB \).
		\item В треугольнике \( ABC \) \( AC=BC \), \( AB=8 \), \( \tg A\dfrac{\sqrt{3}}{\sqrt{3}} \).  Найдите \( AC \).
		\item В треугольнике \( ABC \) \( AC=BC \), \( AH \) --- высота, \( AB=8 \), \( \cos BAC=0,5 \). Найдите \( BH \).
		\item В равнобедренном треугольнике \( ABC \) с основанием \( AB \) боковая сторона равна \( 16\sqrt{7} \), \( \sin\angle BAC=0,75 \). Найдите длину высоты \( AH \).
		\item Площадь параллелограмма равна \( 54 \), две его стороны равны \( 18 \) и \( 36 \). Найдите большую высоту этого параллелограмма.
		\item В ромбе \( ABCD \) угол \( DBA \) равен \( 13\degree \). Найдите угол \( BCD \). Ответ дайте в градусах.
	\end{listofex}
\end{homework}
%END_FOLD

%BEGIN_FOLD % ====>>_____ Занятие 3 _____<<====
\begin{class}[number=3]
	\begin{listofex}
		\item Основания равнобедренной трапеции равны \(51\) и \(65\). Боковые стороны равны \(25\). Найдите синус острого угла трапеции.
		\item Основания равнобедренной трапеции равны \(43\) и \(73\). Косинус острого угла трапеции равен \(\dfrac{ 5 }{ 7 }\).  Найдите боковую сторону.
		\item Большее основание равнобедренной трапеции равно \(34\). Боковая сторона равна \(14\). Синус острого угла равен \(\dfrac{ 2\sqrt{10} }{ 7 }\).  Найдите меньшее основание.
		\item Основания равнобедренной трапеции равны \(7\) и \(51\). Тангенс острого угла равен \(\dfrac{ 5 }{ 11 }\).  Найдите высоту трапеции.
		\item Меньшее основание равнобедренной трапеции равно \(23\). Высота трапеции равна \(39\). Тангенс острого угла равен \(\dfrac{ 13 }{ 8 }\). Найдите большее основание.
		\item Основания равнобедренной трапеции равны \(17\) и \(87\). Высота трапеции равна \(14\). Найдите тангенс острого угла.
		\item Основания равнобедренной трапеции равны \(14\) и \(26\), а ее периметр равен \(60\). Найдите площадь трапеции.
		\item Основания равнобедренной трапеции равны \(7\) и \(13\), а ее площадь равна \(40\). Найдите периметр трапеции.
		\item Найдите площадь прямоугольной трапеции, основания которой равны \(6\) и \(2\), большая боковая сторона составляет с основанием угол \(45\degree\).
		\item Основания прямоугольной трапеции равны \(12\) и \(4\). Ее площадь равна \(64\). Найдите острый угол этой трапеции. Ответ дайте в градусах.
		\item Основания трапеции равны \(18\) и \(6\), боковая сторона, равная \(7\), образует с одним из оснований трапеции угол \(150\degree\). Найдите площадь трапеции.
		\item Основания трапеции равны \(27\) и \(9\), боковая сторона равна \(8\). Площадь трапеции равна \(72\). Найдите острый угол трапеции, прилежащий к данной боковой стороне. Ответ выразите в градусах.
		\item Средняя линия трапеции равна \(28\), а меньшее основание равно \(18\). Найдите большее основание трапеции.
		\item Основания трапеции равны \(4\) и \(10\). Найдите больший из отрезков, на которые делит среднюю линию этой трапеции одна из ее диагоналей.
		\item Основания трапеции равны \(3\) и \(2\). Найдите отрезок, соединяющий середины диагоналей трапеции.
	\end{listofex}
\end{class}
%END_FOLD

%BEGIN_FOLD % ====>>_____ Занятие 4 _____<<====
\begin{class}[number=4]
	\begin{listofex}
		\item Занятие 4
	\end{listofex}
\end{class}
%END_FOLD

%BEGIN_FOLD % ====>>_ Домашняя работа 2 _<<====
\begin{homework}[number=2]
	\begin{listofex}
		\item Домашняя работа 2
	\end{listofex}
\end{homework}
%END_FOLD

%BEGIN_FOLD % ====>>_____ Занятие 5 _____<<====
\begin{class}[number=5]
	\begin{listofex}
		\item Занятие 5
	\end{listofex}
\end{class}
%END_FOLD

%BEGIN_FOLD % ====>>_____ Занятие 6 _____<<====
\begin{class}[number=6]
	\begin{listofex}
		\item Занятие 6
	\end{listofex}
\end{class}
%END_FOLD

%BEGIN_FOLD % ====>>_ Домашняя работа 3 _<<====
\begin{homework}[number=3]
	\begin{listofex}
		\item Основания равнобедренной трапеции равны \( 4 \) и \( 34 \). Боковые стороны равны \( 25 \). Найдите синус острого угла трапеции.
		\item Основания равнобедренной трапеции равны \( 47 \) и \( 19 \). Тангенс острого угла равен \( \dfrac{9}{14} \).  Найдите высоту трапеции.
		\item Основания равнобедренной трапеции равны \( 14 \) и \( 24 \), а ее площадь равна \( 228 \). Найдите периметр трапеции.
		\item Основания прямоугольной трапеции равны \( 15 \) и \( 23 \). Ее площадь равна \( 152 \). Найдите острый угол этой трапеции. Ответ дайте в градусах.
		\item Груз массой \( 0,15 \) кг колеблется на пружине. Его скорость \( v \) меняется по закону \( v=v_0\sin\dfrac{2\pi t}{T} \),  где \( t \) --- время с момента начала колебаний, \( T=16 \) с --- период колебаний,  \( v_0=0,4 \) м/с. Кинетическая энергия \( E \) (в джоулях) груза вычисляется по формуле \( E=\dfrac{mv^2}{2} \),  где \( m \) --- масса груза в килограммах, \( v \) --- скорость груза в м/с. Найдите кинетическую энергию груза через \( 2 \) секунды после начала колебаний. Ответ дайте в джоулях.
	\end{listofex}
\end{homework}
%END_FOLD

%BEGIN_FOLD % ====>>_____ Занятие 7 _____<<====
\begin{class}[number=7]
	\title{Подготовка к проверочной}
	\begin{listofex}
		\item Занятие 7
	\end{listofex}
\end{class}
%END_FOLD

=%BEGIN_FOLD % ====>>_ Проверочная работа _<<====
\begin{exam}
	\begin{listofex}
		\item Проверочная
	\end{listofex}
\end{exam}
%END_FOLD