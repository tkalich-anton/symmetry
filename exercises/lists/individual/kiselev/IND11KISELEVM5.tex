%
%===============>>  Киселев Модуль 5 <<=============
%=
\setmodule{5}

%BEGIN_FOLD % ====>>_____ Занятие 1 _____<<====
\begin{class}[number=1]
	
	\begin{listofex}
		\item   Занятие 1
	\end{listofex}
\end{class}
%END_FOLD

%BEGIN_FOLD % ====>>_____ Занятие 2 _____<<====
\begin{class}[number=2]
	\begin{listofex}
		\item Занятие 2
	\end{listofex}
\end{class}
%END_FOLD

%BEGIN_FOLD % ====>>_ Домашняя работа 1 _<<====
\begin{homework}[number=1]
	\begin{listofex}
		\item Решить уравнения:
		\begin{tasks}(2)
			\task \( \log_2(3+x)=7 \)
			\task \( \log_{11}(16+x)=\log_{11}12 \)
			\task \( \log_6(x+4)=\log_6(6x-6) \)
			\task \( \log_x27=3 \)
		\end{tasks}
		\item Найти значение выражения, \( \log_a(a^5b^9) \), \quad если \( \log_ba=\dfrac{3}{4} \)
		\item Водолазный колокол, содержащий в начальный момент времени \(  v =3 \) моля воздуха объeмом \( V_1=40 \) л, медленно опускают на дно водоeма. При этом происходит изотермическое сжатие воздуха до конечного объeма \( V_2 \). Работа, совершаемая водой при сжатии воздуха, определяется выражением \( A=\alpha vT\log_2\dfrac{V_1}{V_2} \) (Дж), где  \( \alpha=11,5  \) постоянная, а \( T = 300 \) К --- температура воздуха. Какой объeм \( V_2 \) (в литрах) станет занимать воздух, если при сжатии газа была совершена работа в \( 10350 \) Дж?
	\end{listofex}
\end{homework}
%END_FOLD

%BEGIN_FOLD % ====>>_____ Занятие 3 _____<<====
\begin{class}[number=3]
	\begin{listofex}
		\item Моторная лодка прошла против течения реки \( 112  \) км и вернулась в пункт отправления, затратив на обратный путь на \( 6 \) часов меньше. Найдите скорость течения, если скорость лодки в неподвижной воде равна \( 11 \) км/ч. Ответ дайте в км/ч.
		\item Теплоход проходит по течению реки до пункта назначения \( 200 \) км и после стоянки возвращается в пункт отправления. Найдите скорость течения, если скорость теплохода в неподвижной воде равна \( 15 \) км/ч, стоянка длится \( 10 \) часов, а в пункт отправления теплоход возвращается через \( 40 \) часов после отплытия из него. Ответ дайте в км/ч.
		\item Пристани \( A \) и \( B \) расположены на озере, расстояние между ними 390 км. Баржа отправилась с постоянной скоростью из \( A \) в \( B \). На следующий день после прибытия она отправилась обратно со скоростью на \( 3 \) км/ч больше прежней, сделав по пути остановку на \( 9 \) часов. В результате она затратила на обратный путь столько же времени, сколько на путь из \( A \) в \( B \). Найдите скорость баржи на пути из \( A \) в \( B \). Ответ дайте в км/ч.
		\item Из пункта \( A \) в пункт \( B \), расстояние между которыми \( 75 \) км, одновременно выехали автомобилист и велосипедист. Известно, что за час автомобилист проезжает на \( 40 \) км больше, чем велосипедист. Определите скорость велосипедиста, если известно, что он прибыл в пункт \( B \) на \( 6 \) часов позже автомобилиста. Ответ дайте в км/ч.
		\item Велосипедист выехал с постоянной скоростью из города \( A \) в город \( B \), расстояние между которыми равно \( 70 \) км. На следующий день он отправился обратно в \( A \) со скоростью на \( 3 \) км/ч больше прежней. По дороге он сделал остановку на \( 3 \) часа. В результате велосипедист затратил на обратный путь столько же времени, сколько на путь из \( A \) в \( B \). Найдите скорость велосипедиста на пути из \( B \) в \( A \). Ответ дайте в км/ч.
		\item Моторная лодка прошла против течения реки \( 255 \) км и вернулась в пункт отправления, затратив на обратный путь на \( 2 \) часа меньше. Найдите скорость лодки в неподвижной воде, если скорость течения равна \( 1 \) км/ч. Ответ дайте в км/ч.
		\item Теплоход проходит по течению реки до пункта назначения \( 255 \) км и после стоянки возвращается в пункт отправления. Найдите скорость теплохода в неподвижной воде, если скорость течения равна \( 1 \) км/ч, стоянка длится \( 2 \) часа, а в пункт отправления теплоход возвращается через \( 34 \) часа после отплытия из него. Ответ дайте в км/ч.
		\item Баржа в \( 10:00 \) вышла из пункта \( A \) в пункт \( B \), расположенный в \( 15 \) км от \( A \). Пробыв в пункте B \( 1 \) час \( 20 \) минут, баржа отправилась назад и вернулась в пункт \( A \) в \( 16:00 \) того же дня. Определите (в км/час) скорость течения реки, если известно, что собственная скорость баржи равна \( 7 \) км/ч.
		\item Моторная лодка в \( 10:00 \) вышла из пункта \( А \) в пункт \( В \), расположенный в \( 30 \) км от \( А \). Пробыв в пункте \( В \) \( 2 \) часа \( 30 \) минут, лодка отправилась назад и вернулась в пункт \( А \) в \( 18:00 \) того же дня. Определите (в км/ч) собственную скорость лодки, если известно, что скорость течения реки \( 1 \) км/ч.
	\end{listofex}
\end{class}
%END_FOLD

%BEGIN_FOLD % ====>>_____ Занятие 4 _____<<====
\begin{class}[number=4]
	\begin{listofex}
		\item 
	\end{listofex}
\end{class}
%END_FOLD

%BEGIN_FOLD % ====>>_ Домашняя работа 2 _<<====
\begin{homework}[number=2]
	\begin{listofex}
		\item Решить уравнение: \quad \( \log_3(-2-x)=2 \)
		\item Первые \( 190 \) км автомобиль ехал со скоростью \( 50 \) км/ч, следующие \( 180 \) км -- со скоростью \( 90 \) км/ч, а затем \( 170 \) км -- со скоростью \( 100 \) км/ч. Найдите среднюю скорость автомобиля на протяжении всего пути. Ответ дайте в км/ч.
		\item Первые два часа автомобиль ехал со скоростью \( 50 \) км/ч, следующий час -- со скоростью \( 100 \) км/ч, а затем два часа -- со скоростью \( 75 \) км/ч. Найдите среднюю скорость автомобиля на протяжении всего пути. Ответ дайте в км/ч.
		\item Из городов \( A \) и \( B \), расстояние между которыми равно \( 330 \) км, навстречу друг другу одновременно выехали два автомобиля и встретились через \( 3 \) часа на расстоянии \( 180 \) км от города \( B \). Найдите скорость автомобиля, выехавшего из города \( A \). Ответ дайте в км/ч.
		\item Два велосипедиста одновременно отправились в \( 88 \)-километровый пробег. Первый ехал со скоростью, на \( 3 \) км/ч большей, чем скорость второго, и прибыл к финишу на \( 3 \) часа раньше второго. Найти скорость велосипедиста, пришедшего к финишу вторым. Ответ дайте в км/ч.
	\end{listofex}
\end{homework}
%END_FOLD

%BEGIN_FOLD % ====>>_____ Занятие 5 _____<<====
\begin{class}[number=5]
	\begin{listofex}
		\item Занятие 5
	\end{listofex}
\end{class}
%END_FOLD

%BEGIN_FOLD % ====>>_____ Занятие 6 _____<<====
\begin{class}[number=6]
	\begin{listofex}
		\item Занятие 6
	\end{listofex}
\end{class}
%END_FOLD

%BEGIN_FOLD % ====>>_ Домашняя работа 3 _<<====
\begin{homework}[number=3]
	\begin{listofex}
		\item ДЗ 3
	\end{listofex}
\end{homework}
%END_FOLD

%BEGIN_FOLD % ====>>_____ Занятие 7 _____<<====
\begin{class}[number=7]
	\begin{listofex}
		\item Занятие 7
	\end{listofex}
\end{class}
%END_FOLD

%BEGIN_FOLD % ====>>_ Проверочная работа _<<====
\begin{exam}
	\begin{listofex}
		\item Проверочная работа
	\end{listofex}
\end{exam}
%END_FOLD