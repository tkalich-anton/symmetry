%
%===============>>  Киселев Модуль 7 <<=============
%
\setmodule{7}

%BEGIN_FOLD % ====>>_____ Занятие 1 _____<<====
\begin{class}[number=1]
	\begin{listofex}
		\item .
	\end{listofex}
\end{class}
%END_FOLD

%BEGIN_FOLD % ====>>_____ Занятие 2 _____<<====
\begin{class}[number=2]
	\begin{listofex}
		\item .
	\end{listofex}
\end{class}
%END_FOLD

%BEGIN_FOLD % ====>>_ Домашняя работа 1 _<<====
\begin{homework}[number=1]
	\begin{listofex}
		\item .
	\end{listofex}
\end{homework}
%END_FOLD

%BEGIN_FOLD % ====>>_____ Занятие 3 _____<<====
\begin{class}[number=3]
		\begin{definit}
		\textbf{Синусом} острого угла прямоугольного треугольника называется отношение противолежащего катета к гипотенузе.
		\end{definit}
		\begin{definit}
		\textbf{Косинусом} острого угла прямоугольного треугольника называется отношение прилежащего катета к гипотенузе.
		\end{definit}
		\begin{definit}
		\textbf{Тангенсом} острого угла прямоугольного треугольника называется отношение противолежащего катета к прилежащему катету.
		\end{definit}
		\begin{definit}
		\textbf{Основное тригонометрическое тождество:} \[\sin^2\alpha+\cos^2\alpha=1\]
		\end{definit}
		\begin{listofex}
		\item В треугольнике \( ABC \) угол \( C \) равен \( 90 \) градусов, \( AC=6 \), \( AB=20 \). Найдите \( \sin B \).
		\item  В треугольнике \( ABC \) угол \( C \) равен \( 90 \) градусов, \( BC=9 \), \( AB=20 \). Найдите \( \cos B \).
		\item В треугольнике \( ABC \) угол \( C \) равен \( 90 \) градусов, \( BC=9 \), \( AC=27 \). Найдите \( \tg B \).
		\item Найдите синус, косинус и тангенс углов \( A \) и \( B \) треугольника \( ABC \) с прямым углом \( C \), если:
		\begin{tasks}(2)
			\task \( BC=8 \), \( AB=17 \)
			\task \( BC=21 \), \( AC=20 \)
			\task \( BC=1 \), \( AC=2 \)
		\end{tasks}
		\item Найдите:
		\begin{tasks}(1)
			\task \( \sin\alpha \) и \( \tg\alpha \), если \( \cos\alpha=\dfrac{1}{2} \)
			\task \( \cos\alpha \) и \( \tg\alpha \), если \( \sin\alpha=\dfrac{\sqrt{3}}{2} \)
		\end{tasks}
		\item В треугольнике \( ABC \) угол \( C \) прямой, \( BC=8 \), \( \sin A=0,4 \). Найдите \( AB \).
		\item В треугольнике \( ABC \) угол \( C \) прямой, \( AC=15 \), \( \cos A=\dfrac{5}{7} \). Найдите \( AB \).
		\item В треугольнике \( ABC \) угол \( C \) равен \( 90\degree \), \( BC=12 \), \( \sin A=\dfrac{4}{11} \). Найдите \( AB \).
		\item В треугольнике \( ABC \) угол \( C \) равен \( 90\degree \), \( AC = 4,8 \),  \( \sin A = \dfrac{7}{25} \).  Найдите \( AB \).
		\item В треугольнике \( ABC \) угол \( C \) равен \( 90\degree \),  \( \tg A = \dfrac{33}{4\sqrt{33}} \),  \( AC =  4 \). Найдите \( AB \).
		\item В треугольнике \( АВС \) угол \( С \) равен \( 90\degree \), высота \( CH \) равна \( 7 \), \( BH = 24 \). Найдите  \( \cos A \).
		\item В треугольнике \( ABC \) \( AC=BC=5\),  \( \sin A = \dfrac{7}{25} \).  Найдите \(AB\).
		\item В треугольнике \( ABC \) \( AC = BC = 8 \),  \( \cos A = 0,5 \). Найдите \(AB\).
		\item В треугольнике \( ABC \) угол \( C \) равен \( 90\degree \), \( BC=5 \), \( \sin A=\dfrac{7}{25} \).  Найдите высоту \( CH \).
		\item В треугольнике \( ABC \) угол \( C \) равен \( 90\degree \), \( CH \) --- высота, \( BC=3 \), \( \cos A=\dfrac{\sqrt{35}}{6} \).  Найдите \( AH \).
		\item В треугольнике \( ABC \) угол \( C \) равен \( 90\degree \), \( CH \) --- высота, \( BC=5 \),  \( \cos A=\dfrac{7}{25} \).  Найдите \( BH \).
		\item В треугольнике \( ABC \) \( AC = BC \), \( AH \) --- высота, \( AB = 8 \),  \( \cos BAC = 0,5 \). Найдите \( BH \).
	\end{listofex}
\end{class}
%END_FOLD

%BEGIN_FOLD % ====>>_____ Занятие 4 _____<<====
\begin{class}[number=4]
	\begin{listofex}
		\item .
	\end{listofex}
\end{class}
%END_FOLD

%BEGIN_FOLD % ====>>_ Домашняя работа 2 _<<====
\begin{homework}[number=2]
	\begin{listofex}
		\item В треугольнике \( ABC \) угол \( C \) равен \( 90\degree \), \( BC=6 \), \( \sin A=0,3 \). Найдите \( AB \).
		\item Катеты прямоугольного треугольника равны  \( \sqrt{21} \) и \( 2 \). Найдите синус наименьшего угла этого треугольника.
		\item В треугольнике \( ABC \) угол \( C \) равен \( 90\degree \), \( \tg A=\dfrac{2\sqrt{10}}{3} \),  \( AB=28 \). Найдите \( AC \).
		\item В треугольнике \( ABC \) угол \( C \) равен \( 90\degree \), \( AC=24 \), \( BC=7 \). Найдите \( \sin A \).
		\item В треугольнике \( ABC \) угол \( C \) равен \( 90 \) градусов, \( BC=15 \), \( \cos A=\dfrac{12}{13} \).  Найдите \( AC \).
		\item В треугольнике \( ABC \) угол \( C \) равен \( 156\degree \), \( AC=BC \). Найдите угол \( A \). Ответ дайте в градусах.
		\item В треугольнике \( ABC \) угол \( A \) равен \( 30\degree \), угол \( B \) --- тупой, \( CH \) --- высота, \( \angle BCH=22\degree \). Найдите угол \( ACB \). Ответ дайте в градусах.
	\end{listofex}
\end{homework}
%END_FOLD

%BEGIN_FOLD % ====>>_____ Занятие 5 _____<<====
\begin{class}[number=5]
	\begin{listofex}
		\item .
	\end{listofex}
\end{class}
%END_FOLD

%BEGIN_FOLD % ====>>_____ Занятие 6 _____<<====
\begin{class}[number=6]
	\begin{listofex}
		\item .
	\end{listofex}
\end{class}
%END_FOLD

%BEGIN_FOLD % ====>>_ Домашняя работа 3 _<<====
\begin{homework}[number=3]
	\begin{listofex}
		\item .
	\end{listofex}
\end{homework}
%END_FOLD

%BEGIN_FOLD % ====>>_____ Занятие 7 _____<<====
\begin{class}[number=7]
	\title{Подготовка к проверочной}
	\begin{listofex}
		\item Занятие 7
	\end{listofex}
\end{class}
%END_FOLD

%BEGIN_FOLD % ====>>_ Проверочная работа _<<====
\begin{exam}
	\begin{listofex}
		\item .
	\end{listofex}
\end{exam}
%END_FOLD