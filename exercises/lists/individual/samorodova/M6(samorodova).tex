%
%===============>>  Самородова Модуль 6 <<=============
%=
\setmodule{6}

%BEGIN_FOLD % ====>>_____ Занятие 4 _____<<====
\begin{class}[number=4]
	\begin{listofex}
		\item Вычислить:
		\begin{tasks}(1)
			\task \( \mfrac{10}{5}{9}-\mfrac{1}{7}{32}\cdot\left( \mfrac{4}{14}{15}+\mfrac{3}{1}{15} \right) \)
			\task \( \dfrac{1}{13}\cdot\left( \mfrac{2}{3}{8}-\mfrac{1}{5}{6} \right)\cdot\mfrac{2}{2}{5}+\dfrac{9}{10} \)
		\end{tasks}
		\item В магазине куртки продавались по цене \( 8 000 \) руб. за одну куртку. Летом на эту цену стала действовать скидка в \( 20\% \). Сколько рублей составляет скидка?
		\item Мама получила премию \( 50 000 \) руб. На подарок дочери она потратила \( 1\% \) этой премии. Сколько рублей стоит подарок?
		\item В палатку завезли \( 850 \) кг огурцов. Покупатель взял для соления \( 3\% \) всех огурцов. Сколько килограммов огурцов было куплено?
		\item На поле, площадь которого \( 620 \) га, работали хлопкоуборочные машины. За сутки они убрали \( 15\% \) всего поля. Сколько гектаров хлопка они не убрали за сутки?
		\item Сколько человек было в кино, если \( 1\% \) всех зрителей составляет \( 7 \) человек?
		\item Ученик прочитал \( 138 \) страниц, что составляет \( 23\% \) числа всех страниц в книге. Сколько страниц в книге?
	\end{listofex}
	\newpage
	\title{Домашняя работа}
	\begin{listofex}
		\item Вычислите:
		\[\left( \dfrac{1}{4}\dfrac{7}{18} \right)\cdot\mfrac{1}{1}{23}-1\]
		\item На поле, площадь которого \( 620 \) га, работали хлопкоуборочные машины. За сутки они убрали \( 15\% \) всего поля. Сколько гектаров хлопка они не убрали за сутки?
		\item Мотоциклист за день проехал некоторое расстояние. \( 1\% \) пути он ехал по просёлочной дороге, что составило \( 3,2 \) км. Сколько километров проехал мотоциклист за день?
	\end{listofex}
\end{class}
%END_FOLD

%BEGIN_FOLD % ====>>_ Домашняя работа 1 _<<====
\begin{homework}[number=3]
	\begin{listofex}
		\item 
	\end{listofex}
\end{homework}
%END_FOLD

%BEGIN_FOLD % ====>>_____ Занятие 5 _____<<====
\begin{class}[number=5]
	\begin{listofex}
		\item Вычислите:
		\[\left( \mfrac{5}{3}{8}+\mfrac{18}{1}{2}-\mfrac{7}{5}{24} \right):\mfrac{16}{2}{3}\]
		\item Витя пошел в магазин, взяв с собой \( 400 \) рублей. Он купил тетрадь за \( 24 \) рубля. Сколько процентов всех денег он потратил?
		\item Цена на куртку понизилась сначала на \( 25\% \), а затем ещё на \( 10\% \). Какая цена на куртку была изначально, если теперь она стоит \( 3000 \) руб?
		\item Настя прочитала сначала \( 50\% \) книги, а потом \( 20\% \) от оставшейся части, что составило \( 50 \) страниц. Сколько страниц осталось прочитать Насте?
		\item В январе завод выпустил \( 200 \) холодильников, а в феврале --- \( 212 \). На сколько процентов выросло производство холодильников на заводе в феврале по сравнению с январём?
	\end{listofex}
		\newpage
		\title{Домашняя работа}
		\begin{listofex}
		\item Вычислите:
		\[\left( \mfrac{4}{1}{12}\cdot\mfrac{1}{5}{7}-\mfrac{5}{2}{9} \right)\cdot\dfrac{3}{8}\]
		\item В школе \( 800 \) учеников. Из них \( 120 \) человек приняли участие в лыжной гонке. сколько процентов всех учеников школы приняло участие в гонке?
		\item После увеличения стоимости брюк на \( 10\% \), а потом ещё на \( 20\% \) они стали стоить \( 2310 \) руб. Какова была их начальная стоимость?
		\end{listofex}
\end{class}
%END_FOLD

%BEGIN_FOLD % ====>>_ Домашняя работа 2 _<<====
\begin{homework}[number=2]
	\begin{listofex}
		\item Домашняя работа
	\end{listofex}
\end{homework}
%END_FOLD

%BEGIN_FOLD % ====>>_____ Занятие 3 _____<<====
\begin{class}[number=3]
	\begin{listofex}
		\item Занятие 3
	\end{listofex}
\end{class}
%END_FOLD

%BEGIN_FOLD % ====>>_ Домашняя работа 3 _<<====
\begin{homework}[number=3]
	\begin{listofex}
		\item Домашняя работа
	\end{listofex}
\end{homework}
%END_FOLD

%BEGIN_FOLD % ====>>_____ Занятие 4 _____<<====
\begin{class}[number=4]
	\begin{listofex}
		\item Пусто
	\end{listofex}
\end{class}
%END_FOLD


%BEGIN_FOLD % ====>>_ Проверочная работа _<<====
\begin{exam}
	\begin{listofex}
		\item Проверочная
	\end{listofex}
\end{exam}
%END_FOLD