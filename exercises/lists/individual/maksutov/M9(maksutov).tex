\setmodule{9}

%BEGIN_FOLD % ====>>_____ Занятие 1 _____<<====
\begin{class}[number=1]
	\begin{listofex}
		\item Вычислите:
		\begin{tasks}(2)
			\task \( (\sqrt{13}-\sqrt{7})(\sqrt{13}+\sqrt{7}) \)
			\task \( \dfrac{ \sqrt{2,8} \cdot \sqrt{4,2} }{ \sqrt{0,24} } \)
			\task \( 7^{\tfrac{ 4 }{ 9 }} \cdot 49^{\tfrac{ 5 }{ 18 }} \)
			\task \( \dfrac{ 2^{3,5} \cdot 3^{5,5} }{ 6^{4,5} } \)
			\task \( \log_5 0,2 + \log_{0,5}4 \)
			\task \( \dfrac{ \log_3 25 }{ \log_3 5 } \)
		\end{tasks}
		%3
		\item
		\begin{minipage}[t]{0.45\linewidth}
			На рисунке изображен график функции \( y = f(x)\), определенной на интервале \((-5; 5)\). Найдите количество точек, в которых касательная к графику функции параллельна прямой \(y  =  6\) или совпадает с ней.
		\end{minipage}
		\hspace{0.02\linewidth}
		\begin{minipage}[t]{0.5\linewidth}
			\includegraphics[align=t, width=\linewidth]{../\picpath/G111M5L2-1}
		\end{minipage}
		%4
		\item
		\begin{minipage}[t]{0.45\linewidth}
			На рисунке изображен график производной функции \(f(x)\), определенной на интервале \((-10; 2)\). Найдите количество точек, в которых касательная к графику функции \(f(x)\) параллельна прямой \(y = -2x - 11\) или совпадает с ней.
		\end{minipage}
		\hspace{0.02\linewidth}
		\begin{minipage}[t]{0.5\linewidth}
			\includegraphics[align=t, width=\linewidth]{../\picpath/G111M5L2-2}
		\end{minipage}
		%5
		\item
		\begin{minipage}[t]{0.45\linewidth}
			На рисунке изображен график производной функции \(f(x)\), определенной на интервале \( (-6;6) \). Найдите промежутки возрастания функции \(f(x)\). В ответе укажите сумму целых точек, входящих в эти промежутки.
		\end{minipage}
		\hspace{0.02\linewidth}
		\begin{minipage}[t]{0.5\linewidth}
			\includegraphics[align=t, width=\linewidth]{../\picpath/G111M5L2-3}
		\end{minipage}
		%6
		\item
		\begin{minipage}[t]{0.45\linewidth}
			На рисунке изображен график функции \(y = f(x)\), определенной на интервале \((-6; 8)\). Определите количество целых точек, в которых производная функции положительна.
		\end{minipage}
		\hspace{0.02\linewidth}
		\begin{minipage}[t]{0.5\linewidth}
			\includegraphics[align=t, width=\linewidth]{../\picpath/G111M5L2-4}
		\end{minipage}
		%7
		\item Найдите:
		\begin{tasks}(1)
			\task точку максимума функции \(y = 2x^3 -22x-15\)
			\task наименьшее значение функции \(y = 3x^3 - 54x\) на отрезке \([0;4]\)
			\task точку максимума функции \(y = x^3 - 5x^2 + 7x -5\)
			\task наименьшее значение функции \(y=(x-2)^2(x-4)+5\) на отрезке \( [1;3] \)
			\task наименьшее значение функции \(y=(x+3)^2(x+5)-1\) на отрезке \( [-4;-1] \)
			\task наибольшее точку минимума функции \(y=(x-2)^2(x-4)+5\)
			\task точку максимума функции \(y = -\dfrac{x^2+289}{x}\)
			\task наименьшее значение функции \(y = \dfrac{x^2+25}{x}\) на отрезке \([1;10]\)
			\task наименьшее точку максимума функции \(y=(x+16)e^{x-16}\)
			\task наименьшее точку минимума функции \(y=(9-x)e^{x+9}\)
			\task наименьшее значение функции \(y=4x-4\ln(x+7)+6\) на отрезке \( -6,5; 0 \)
			%\task наименьшее значение функции \( 12\cos x + 6 \sqrt{3}x-2\sqrt{3} + 0 \) на отрезке \( \left[ 0; \dfrac{ \pi }{ 2 } \right]  \)
			\task точку максимума функции \(y=\ln(x+5)^5-5x\)
			\task точку максимума функции \(y=3x-\ln(x+3)^3\)
			
			\task наименьшее значение функции \( (2x-3)\cos x - 2 \sin x + 5 \) на отрезке \( \left( 0; \dfrac{ \pi }{ 2 } \right) \)
			\task точку максимума функции \( y=(2x-3)\cos x - 2\sin x + 5 \), принадлежащую отрезку \( \left( 0;\dfrac{ \pi }{ 2 } \right) \)
			
		\end{tasks}
	\end{listofex}
\end{class}
%END_FOLD

%BEGIN_FOLD % ====>>_____ Занятие 2 _____<<====
\begin{class}[number=2]
	\begin{listofex}
		%11.4 10-13 И 11.5 5-8 И 11.6 1-3
		\item Найдите точку минимума функции \( y=(x-2)^2 e^{ x-5} \).
		\item Найдите наименьшее значение функции \( y=(x+6)^2 e^{4-x} \) на отрезке \([-7;-5]\).
		\item Найдите точку минимума функции \( y=(x+3)^2e^{2-x} \).
		\item Найдите максимальное значение функции \(y=(x^2-8x+8)e^{6-x}\) на отрезке \([3;9]\).
		\item Найдите наибольшее значение функции \( y=2x^2-13x+9\ln x + 8 \) на отрезке \(\left[ \dfrac{ 13 }{ 14 }; \dfrac{15  }{14  } \right] \).
		\item Найдите наименьшее значение функции \( y=2x^2-5x+\ln x-3 \) на отрезке \(\left[ \dfrac{ 5 }{ 6 }; \dfrac{ 7 }{ 6 } \right] \).
		\item Найдите наименьшее значение функции \( y=9x-\ln(9x)+3 \) на отрезке \(\left[ \dfrac{1  }{ 18 }; \dfrac{ 5 }{ 18 } \right] \).
		\item Найдите наибольшее значение функции \( y=\ln(11x)-11x+9 \) на отрезке \(\left[ \dfrac{ 1 }{ 22}; \dfrac{ 5 }{ 22 } \right] \).
		%11-trigon 1-3
		\item Найдите наименьшее значение функции \(y=5 \cos x - 6x + 4 \) на отрезке \( \left[ -\dfrac{ 3\pi }{ 2 }; 0 \right] \).
		\item Найдите наибольшее значение функции \(y=12\cos x + 6\sqrt{3}x - 2 \sqrt{3} \pi +6 \) на отрезке \( \left[ 0; \dfrac{ \pi }{ 2 } \right] \).
		\item Найдите наименьшее значение функции \(y=3+\dfrac{ 5\pi }{ 4 } -5x - 5\sqrt{2} \cos x \) на отрезке \( \left[ 0; \dfrac{ \pi }{ 2 } \right] \).
		\item Найдите наибольшее значение функции \(y=15 x + 3\sin x + 5 \) на отрезке \( \left[ -\dfrac{ \pi }{ 2 }; 0 \right] \).
		\item Найдите наименьшее значение функции \(y=7\sin x -8x+9 \) на отрезке \( \left[ -\dfrac{ 3\pi }{ 2 }; 0 \right] \).
		\item Прямая \(y=7x-5\) параллельна касательной к графику функции \(y=x^2+6x-8\). Найдите абсциссу точки касания.
		\item Материальная точка движется прямолинейно по закону \(x(t)=6t^2-48t+17\) (где \(x\) --- расстояние от точки отсчета в метрах, \(t\) --- время в секундах, измеренное с начала движения). Найдите ее скорость (в м/с) в момент времени \(t  =  9\) с.
		\item Материальная точка движется прямолинейно по закону \(x(t)=\dfrac{ 1 }{ 3 }t^3-3t^2-5t+3\) (где \(x\) --- расстояние от точки отсчета в метрах, \(t\) --- время в секундах, измеренное с начала движения). В какой момент времени (в секундах) ее скорость была равна \(2\) м/с?
		%KuznetsovM8L2
		\item На рисунке изображен график производной функции \( f(x) \), определенной на интервале \( (-13; 10) \). Найдите количество точек минимума функции \( f(x) \) на отрезке \( [-9;9] \).
		\begin{center}
			\includegraphics[align=t, width=0.8\linewidth]{../\picpath/KuznetsovM8L2}
		\end{center}
		\item 
		\begin{minipage}[t]{0.45\linewidth}
			На рисунке изображены график функции \( y=f(x) \) и касательная к этому графику, проведённая в точке \( x_0=2 \). Найдите значение производной функции \( g(x)=x^2-f(x)+1 \) в точке \( x_0 \).
		\end{minipage}
		\begin{minipage}[t]{0.5\linewidth}
			\includegraphics[align=t, width=\textwidth]{../\picpath/KuznetsovM8L2_1}
		\end{minipage}
		%9641
		\item
		\begin{minipage}[t]{0.45\linewidth}
			На рисунке изображён график функции \(y=f(x)\) и касательная к нему в точке с абсциссой \(x_0\). Найдите значение производной функции \(f(x)\) в точке \(x_0\).
		\end{minipage}
		\hspace{0.02\linewidth}
		\begin{minipage}[t]{0.5\linewidth}
			\includegraphics[align=t, width=\linewidth]{../\picpath/maksutovM9L2-1}
		\end{minipage}
		%525446
		\item
		\begin{minipage}[t]{0.45\linewidth}
			На рисунке изображён график функции \(y=f(x)\) и касательная к нему в точке с абсциссой \(x_0\). Найдите значение производной функции \(f(x)\) в точке \(x_0\).
		\end{minipage}
		\hspace{0.02\linewidth}
		\begin{minipage}[t]{0.5\linewidth}
			\includegraphics[align=t, width=\linewidth]{../\picpath/maksutovM9L2-3}
		\end{minipage}
		%525401
		\item
		\begin{minipage}[t]{0.45\linewidth}
			На рисунке изображён график функции \(y=f(x)\) и касательная к нему в точке с абсциссой \(x_0\). Найдите значение производной функции \(f(x)\) в точке \(x_0\).
		\end{minipage}
		\hspace{0.02\linewidth}
		\begin{minipage}[t]{0.5\linewidth}
			\includegraphics[align=t, width=\linewidth]{../\picpath/maksutovM9L2-2}
		\end{minipage}
		
	\end{listofex}
\end{class}
%END_FOLD

%BEGIN_FOLD % ====>>_ Домашняя работа 1 _<<====
\begin{homework}[number=1]
	\begin{listofex}
		\item Домашняя работа 1
	\end{listofex}
\end{homework}
%END_FOLD

%BEGIN_FOLD % ====>>_____ Занятие 3 _____<<====
\begin{class}[number=3]
	\begin{listofex}
		\item Один острый угол прямоугольного треугольника на \( 32\degree \) больше другого. Найдите больший острый угол. Ответ дайте в градусах.
		\item Один острый угол прямоугольного треугольника в \( 4 \) раза больше другого. Найдите больший острый угол. Ответ дайте в градусах.
		\item В треугольнике \(ABC\) угол \(C\) равен \(90 \degree \), \(AC=4,8\), \(\sin{A} = \dfrac{7}{25}\). Найдите \(AB\).
		\item В треугольнике \(ABC\) угол \(C\) равен \(90 \degree \), \(AC=2\), \(\cos{A} = \dfrac{\sqrt{17}}{17}\). Найдите \(AC\).
		\item В треугольнике \(ABC\) угол \(C\) равен \(90 \degree \), \(AC=8\), \(\tg{A} = 0,5\). Найдите \(AB\).
		\item В треугольнике \(ABC\) угол \(C\) равен \(90 \degree \), \(CH\) --- высота, \(AB=13\), \(\tan{A} = \dfrac{1}{5}\). Найдите \(AH\).
		\item В треугольнике \(ABC\) угол \(C\) равен \(90 \degree \), \(CH\) --- высота, \(BC=3\), \(\sin{A} = \dfrac{1}{6}\). Найдите \(AH\).
		\item В треугольник \(ABC, AC = BC, AB = 9,6\), \( \sin{A}= \dfrac{7}{25} \). Найдите \(AC\).
		\item В треугольник \(ABC, AC = BC = 8\), \( \cos{A}= 0,5 \). Найдите \(AB\).
		\item В треугольник \(ABC, AC = BC, AB = 8\), \( \tg{A}= \dfrac{33}{4 \cdot \sqrt[]{33}} \). Найдите \(AC\).
		\item Основания равнобедренной трапеции равны \(7\) и \(51\). Тангенс острого угла равен \( \dfrac{ 5 }{ 11 } \).  Найдите высоту трапеции.
		\item В параллелограмме \( ABCD: \) \( AB  =  3, AD  =  21, \sin A = \dfrac{ 6 }{ 7 } \). Найдите большую высоту параллелограмма.
		\item
		\begin{minipage}[t]{\bodywidth}
			Найдите сторону правильного шестиугольника, описанного около окружности, радиус которой равен \(\sqrt{3}\).
		\end{minipage}
		\hspace{0.02\linewidth}
		\begin{minipage}[t]{\picwidth}
			\includegraphics[align=t, width=\linewidth]{../\picpath/MECGERM9H3-1}
		\end{minipage}
		\item
		\begin{minipage}[t]{\bodywidth}
			Сторона правильного треугольника равна \(\sqrt{3}\). Найдите радиус окружности, вписанной в этот треугольник.
		\end{minipage}
		\hspace{0.02\linewidth}
		\begin{minipage}[t]{\picwidth}
			\includegraphics[align=t, width=\linewidth]{../\picpath/MECGERM9H3-2}
		\end{minipage}
		%Первые 5 вписанные окружности
		\item Площадь треугольника равна \(24\), а радиус вписанной окружности равен \(2\). Найдите периметр этого треугольника.
		\item Около окружности, радиус которой равен \(3\), описан пятиугольник, периметр которого равен \(20\). Найдите его площадь.
		\item Найдите радиус окружности, вписанной в правильный треугольник, высота которого равна \(6\).
		\item Радиус окружности, вписанной в правильный треугольник, равен \(6\). Найдите высоту этого треугольника.
		%Первые 3 описанные окружности
		\item Точки \(A, B, C,\) расположенные на окружности, делят ее на три дуги, градусные величины которых относятся как \(1 : 3 : 5\). Найдите больший угол треугольника \(ABC\). Ответ дайте в градусах.
		\item Угол \(A\) четырехугольника \(ABCD\), вписанного в окружность, равен \(58\degree\). Найдите угол \(C\) этого четырехугольника. Ответ дайте в градусах.
		\item Стороны четырехугольника \(ABCD \) \( AB, BC, CD, AD\) стягивают дуги описанной окружности, градусные величины которых равны соответственно \(95\) градусов, \(49\) градусов, \(71\) градусов, \(145\) градусов. Найдите угол \(B\) этого четырехугольника. Ответ дайте в градусах.
	\end{listofex}
\end{class}
%END_FOLD

%BEGIN_FOLD % ====>>_____ Занятие 4 _____<<====
\begin{class}[number=4]
	\begin{listofex}
		\item Занятие 4
	\end{listofex}
\end{class}
%END_FOLD

%BEGIN_FOLD % ====>>_ Домашняя работа 2 _<<====
\begin{homework}[number=2]
	\begin{listofex}
		\item Домашняя работа 2
	\end{listofex}
\end{homework}
%END_FOLD

%BEGIN_FOLD % ====>>_____ Занятие 5 _____<<====
\begin{class}[number=5]
	\begin{listofex}
		\item Занятие 5
	\end{listofex}
\end{class}
%END_FOLD

%BEGIN_FOLD % ====>>_____ Занятие 6 _____<<====
\begin{class}[number=6]
	\begin{listofex}
		\item Занятие 6
	\end{listofex}
\end{class}
%END_FOLD

%BEGIN_FOLD % ====>>_ Домашняя работа 3 _<<====
\begin{homework}[number=3]
	\begin{listofex}
		\item Домашняя работа 3
	\end{listofex}
\end{homework}
%END_FOLD

%BEGIN_FOLD % ====>>_____ Занятие 7 _____<<====
\begin{class}[number=7]
	\title{Подготовка к проверочной}
	\begin{listofex}
		\item Занятие 7
	\end{listofex}
\end{class}
%END_FOLD

%BEGIN_FOLD % ====>>_ Проверочная работа _<<====
\begin{exam}
	\begin{listofex}
		\item Проверочная
	\end{listofex}
\end{exam}
%END_FOLD