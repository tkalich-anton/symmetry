\setmodule{9}

%BEGIN_FOLD % ====>>_____ Занятие 1 _____<<====
\begin{class}[number=1]
	\begin{listofex}
		\item Вычислите:
		\begin{tasks}(1)
			\task \( 3^{\log_{\sqrt[3]{9}}4} +2^{\tfrac{1}{\log_{16} 4}} \)
			\task \( \dfrac{\log_3 135}{\log_{15} 3} - \dfrac{\log_3 5}{\log_{405} 3} \)
			\task \( \dfrac{3+\log_{12}27}{3-\log_{12} 27} \cdot \log_6 16 \)
			\task \( \log_3 27 - \log_{\sqrt{3}}27 - \log_{\frac{1}{3}} 27 - \log_{\frac{\sqrt{3}}{2}} \left( \dfrac{64}{27} \right)  \)
		\end{tasks}
		%1
		\item Материальная точка движется прямолинейно по закону \(x(t) = 6t^2 - 48t + 17\) (где \(x\)  — расстояние от точки отсчета в метрах, \(t\)  — время в секундах, измеренное с начала движения). Найдите ее скорость (в м/с) в момент времени \(t  =  9\) с.
		%2
		\item Материальная точка движется прямолинейно по закону \(x(t) = -t^4 + 6t^3 + 5t + 23\) (где \(x\)  — расстояние от точки отсчета в метрах, \(t\)  — время в секундах, измеренное с начала движения). Найдите ее скорость в (м/с) в момент времени \(t = 3\) с.
		%3
		\item
		\begin{minipage}[t]{0.45\linewidth}
			На рисунке изображен график функции \( y = f(x)\), определенной на интервале \((-5; 5)\). Найдите количество точек, в которых касательная к графику функции параллельна прямой \(y  =  6\) или совпадает с ней.
		\end{minipage}
		\hspace{0.02\linewidth}
		\begin{minipage}[t]{0.5\linewidth}
			\includegraphics[align=t, width=\linewidth]{../\picpath/G111M5L2-1}
		\end{minipage}
		%4
		\item
		\begin{minipage}[t]{0.45\linewidth}
			На рисунке изображен график производной функции \(f(x)\), определенной на интервале \((-10; 2)\). Найдите количество точек, в которых касательная к графику функции \(f(x)\) параллельна прямой \(y = -2x - 11\) или совпадает с ней.
		\end{minipage}
		\hspace{0.02\linewidth}
		\begin{minipage}[t]{0.5\linewidth}
			\includegraphics[align=t, width=\linewidth]{../\picpath/G111M5L2-2}
		\end{minipage}
		%5
		\item
		\begin{minipage}[t]{0.45\linewidth}
			На рисунке изображен график производной функции \(f(x)\), определенной на интервале \( (-6;6) \). Найдите промежутки возрастания функции \(f(x)\). В ответе укажите сумму целых точек, входящих в эти промежутки.
		\end{minipage}
		\hspace{0.02\linewidth}
		\begin{minipage}[t]{0.5\linewidth}
			\includegraphics[align=t, width=\linewidth]{../\picpath/G111M5L2-3}
		\end{minipage}
		%6
		\item
		\begin{minipage}[t]{0.45\linewidth}
			На рисунке изображен график функции \(y = f(x)\), определенной на интервале \((-6; 8)\). Определите количество целых точек, в которых производная функции положительна.
		\end{minipage}
		\hspace{0.02\linewidth}
		\begin{minipage}[t]{0.5\linewidth}
			\includegraphics[align=t, width=\linewidth]{../\picpath/G111M5L2-4}
		\end{minipage}
		%7
		\item Найдите:
		\begin{tasks}(1)
			\task точку максимума функции \(y = 2x^3 -22x-15\)
			\task наименьшее значение функции \(y = 3x^3 - 54x\) на отрезке \([0;4]\)
			\task точку максимума функции \(y = x^3 - 5x^2 + 7x -5\)
			\task точку максимума функции \(y = -\dfrac{x^2+289}{x}\)
			\task наименьшее значение функции \(y = \dfrac{x^2+25}{x}\) на отрезке \([1;10]\)
			\task наименьшее значение функции \(y=(x-2)(x-4)+5\) на отрезке \( [1;3] \)
			\task наименьшее значение функции \(y=(x+3)(x+5)-1\) на отрезке \( [-4;-1] \)
			\task наибольшее точку минимума функции \(y=(x-2)^2(x-4)+5\)
			%\task наименьшее значение функции \( 12\cos x + 6 \sqrt{3}x-2\sqrt{3} + 0 \) на отрезке \( \left[ 0; \dfrac{ \pi }{ 2 } \right]  \)
			\task наименьшее значение функции \( (2x-3)\cos x - 2 \sin x + 5 \) на отрезке \( \left( 0; \dfrac{ \pi }{ 2 } \right) \)
			\task точку максимума функции \( y=(2x-3)\cos x - 2\sin x + 5 \), принадлежащую отрезку \( \left( 0;\dfrac{ \pi }{ 2 } \right) \)
			%\task наименьшее значение функции \(\) на отрезке \(  \)
		\end{tasks}
	\end{listofex}
\end{class}
%END_FOLD

%BEGIN_FOLD % ====>>_____ Занятие 2 _____<<====
\begin{class}[number=2]
	\begin{listofex}
		\item Занятие 2
	\end{listofex}
\end{class}
%END_FOLD

%BEGIN_FOLD % ====>>_ Домашняя работа 1 _<<====
\begin{homework}[number=1]
	\begin{listofex}
		\item Домашняя работа 1
	\end{listofex}
\end{homework}
%END_FOLD

%BEGIN_FOLD % ====>>_____ Занятие 3 _____<<====
\begin{class}[number=3]
	\begin{listofex}
		\item Занятие 3 
	\end{listofex}
\end{class}
%END_FOLD

%BEGIN_FOLD % ====>>_____ Занятие 4 _____<<====
\begin{class}[number=4]
	\begin{listofex}
		\item Занятие 4
	\end{listofex}
\end{class}
%END_FOLD

%BEGIN_FOLD % ====>>_ Домашняя работа 2 _<<====
\begin{homework}[number=2]
	\begin{listofex}
		\item Домашняя работа 2
	\end{listofex}
\end{homework}
%END_FOLD

%BEGIN_FOLD % ====>>_____ Занятие 5 _____<<====
\begin{class}[number=5]
	\begin{listofex}
		\item Занятие 5
	\end{listofex}
\end{class}
%END_FOLD

%BEGIN_FOLD % ====>>_____ Занятие 6 _____<<====
\begin{class}[number=6]
	\begin{listofex}
		\item Занятие 6
	\end{listofex}
\end{class}
%END_FOLD

%BEGIN_FOLD % ====>>_ Домашняя работа 3 _<<====
\begin{homework}[number=3]
	\begin{listofex}
		\item Домашняя работа 3
	\end{listofex}
\end{homework}
%END_FOLD

%BEGIN_FOLD % ====>>_____ Занятие 7 _____<<====
\begin{class}[number=7]
	\title{Подготовка к проверочной}
	\begin{listofex}
		\item Занятие 7
	\end{listofex}
\end{class}
%END_FOLD

%BEGIN_FOLD % ====>>_ Проверочная работа _<<====
\begin{exam}
	\begin{listofex}
		\item Проверочная
	\end{listofex}
\end{exam}
%END_FOLD