%
%===============>>  Порядин Модуль 8 <<=============
%=
\setmodule{8}

%BEGIN_FOLD % ====>>_____ Занятие 1 _____<<====
\begin{class}[number=1]
	\begin{listofex}
		\item .
	\end{listofex}
\end{class}
%END_FOLD

%BEGIN_FOLD % ====>>_____ Занятие 2 _____<<====
\begin{class}[number=2]
	\begin{listofex}
		\item .
	\end{listofex}
\end{class}
%END_FOLD

%BEGIN_FOLD % ====>>_ Домашняя работа 1 _<<====
\begin{homework}[number=1]
	\begin{listofex}
		\item .
	\end{listofex}
\end{homework}
%END_FOLD

%BEGIN_FOLD % ====>>_____ Занятие 3 _____<<====
\begin{class}[number=3]
	\begin{listofex}
		\item .
	\end{listofex}
\end{class}
%END_FOLD

%BEGIN_FOLD % ====>>_____ Занятие 4 _____<<====
\begin{class}[number=4]
	\begin{listofex}
		\item .
	\end{listofex}
\end{class}
%END_FOLD

%BEGIN_FOLD % ====>>_ Домашняя работа 2 _<<====
\begin{homework}[number=2]
	\begin{listofex}
		\item .
	\end{listofex}
\end{homework}
%END_FOLD

%BEGIN_FOLD % ====>>_____ Занятие 5 _____<<====
\begin{class}[number=5]
	\begin{listofex}
		\item Занятие 5
	\end{listofex}
\end{class}
%END_FOLD

%BEGIN_FOLD % ====>>_____ Занятие 6 _____<<====
\begin{class}[number=6]
	\begin{listofex}
		\item Занятие 6
	\end{listofex}
\end{class}
%END_FOLD

%BEGIN_FOLD % ====>>_ Домашняя работа 3 _<<====
\begin{homework}[number=3]
	\begin{listofex}
		\item .
	\end{listofex}
\end{homework}
%END_FOLD

%BEGIN_FOLD % ====>>_____ Занятие 7 _____<<====
\begin{class}[number=7]
	\begin{listofex}
		\item Вычислите:
		\begin{tasks}(3)
			\task \( 4 + 5 \cdot (9 + 11) \)
			\task \( 124\cdot4 \)
			\task \( 7 + 3 \cdot (8 + 12) \)
			\task \( 17 + 3 \cdot 3 - 18 \)
			\task \( 53 - 2\cdot 8 + 12 \)
			\task \( 15015 : 5 - 230 \cdot 3 \)
		\end{tasks}
		\item У Тани есть 1500 рублей, и ей нужно купить 2 кг капусты, 1 кг перца, 1 кг моркови и 2 килограмма помидоров. Какое наибольшее число лукошек клубники может купить Таня на оставшиеся деньги?
		\begin{figure}[h]
			\center{\includegraphics[align=t, width=0.5\linewidth]{../../../../../exercises/lists/pics/poryadin4M8L7}}
		\end{figure}
		
		\item Электричка из Ростова-на-Дону в Краснодар отправилась в 7 часов 40 минут и прибыла в 12 часов 25 минут. Сколько времени занимает дорога из Ростова-на-Дону в Краснодар, если ехать этой электричкой? Ответ вырази в минутах.
		\item Сивый мерин догоняет пони с расстояния 23 520 метров. Через 86 минут между ними оставалось 42 метра. Сколько сивый мерин может проскакать за 2 минуты, если пони скачет 95 м/мин.
		\item Синий трактор едет навстречу телевизору на колёсиках со скоростью в 4 раза быстрее чем у телевизора. Телевизор проезжает в среднем за 42 часа 2436 метров. Встретятся ли они через 79  часов, если расстояние между ними было 22 км?
		\item Осьминог и каракатица испугались друг дружку и поплыли в разные стороны. Через 37 секунд между ними оказалось 2 516 дециметров. Сколько каракатица проплывает за 201 секунду, если осьминог плавает на 14 дм/с быстрее.
	\end{listofex}
\end{class}
%END_FOLD

%BEGIN_FOLD % ====>>_____ Занятие 8 _____<<====
\begin{class}[number=8]
	\begin{listofex}
		\item Занятие 8
	\end{listofex}
\end{class}
%END_FOLD

%BEGIN_FOLD % ====>>_____ Домашняя работа 4 _____<<====
\begin{homework}[number=4]
	\begin{listofex}
		\item Вычислите:
		\begin{tasks}(3)
			\task \( 20 + 20 : 5 - 17 \)
			\task \( 24 - 4 \cdot 2 + 15 \)
			\task \( 18 + 4 \cdot 3 - 11 \)
			\task \( 49 - 8 \cdot 3 - 16 \)
			\task \( 6 + 27 : 3 + 19 \)
			\task \( 12 + 24 : 4 + 23 \)
		\end{tasks}
		\item Масса восьми одинаковых ящиков с черносливом равна 100 кг. Масса пустого ящика равна 500 грамм. Чему равна масса чернослива в одном ящике?
		\item Сивый мерин догоняет пони с расстояния 23 520 метров. Через 86 минут между ними оставалось 42 метра. Найдите сколько сивый мерин может проскакать за 2 минуты, если пони скачет 
		95 м/мин.
		\item Синий трактор едет навстречу телевизору на колёсиках со скоростью в 4 раза быстрее чем у телевизора. Телевизор проезжает в среднем за 42 часа 2436 метров. Встретятся ли они через 79  часов, если расстояние между ними было 22 км?
		\item Осьминог и каракатица испугались друг дружку и поплыли в разные стороны. Через 37 секунд между ними оказалось 2 516 дециметров. Найдите сколько каракатица проплывает за 201 секунду, если осьминог плавает на 14 дм/с быстрее.
	\end{listofex}
\end{homework}
%END_FOLD