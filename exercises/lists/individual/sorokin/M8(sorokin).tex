%
%===============>>  Сорокин Модуль 8 <<=============
%
\setmodule{8}

%BEGIN_FOLD % ====>>_____ Занятие 1 _____<<====
\begin{class}[number=1]
	\begin{listofex}
		\item Решите систему уравнений:
		\begin{tasks}(2)
			\task \( \begin{cases}
				 2x+y=7 \\\
				 4x-y=5 
			\end{cases} \)
			\task \( \begin{cases}
				 2x+3=5 \\\
				 3x-2y=1 
			\end{cases} \)
		\end{tasks}
		\item Решите системы уравнений
		\begin{tasks}(2)
			\task \( \begin{cases}
				\dfrac{1}{2}x+\dfrac{1}{3}y=1 \\\
				4x-0,5 y=-11
			\end{cases} \)
			\task \( \begin{cases}
				\dfrac{3}{4}x-\dfrac{2}{3}y=-5\\\
				5(x-2)+30=3-y
			\end{cases} \)
			\task \( \begin{cases}
				0,5(x-4y)-16=x+3y \\\
				4(x+3y)+39=\dfrac{1}{3}(6x+y)
			\end{cases} \)
		\end{tasks}
		\item Для кормления \( 5 \) лошадей и \( 25 \) коров ежедневно отпускают \(  220 \) кг сена, для \( 3 \) лошадей и \( 35 \) коров отпускают \( 272 \) кг сена. Найдите дневную норму сена для лошади и для коровы.
		\item Прямая \( y = kx + b \) проходит через точки \( A (2; 7) \) и \( B (–1; –2) \). Найдите величины \( k \) и \( b \).
		\item Пять досок и шесть брусьев весят \( 107 \) кг. Четыре доски тяжелее двух брусьев на \( 4 \) кг. Сколько весит одна доска и один брус?
		\item За \( 5 \) кг огурцов и \( 4 \) кг помидоров заплатили \( 220 \) р. Сколько стоит килограмм огурцов и сколько стоит килограмм помидоров, если \( 4 \) кг огурцов дороже киллограмма помидоров на \( 50 \) р.?
		\item Масса \( 2 \) слитков олова и \( 5 \) слитков свинца равна \( 33 \) кг. Какова масса слитка олова и какова масса слитка свинца, если масса \( 6 \) слитков олова на \( 19 \) кг больше массы слитка свинца?
		\item Кассир разменял \( 500 \)-рублевую купюру на \( 50 \)-рублевые и \( 10 \)-рублевые, всего \( 22 \) купюры. Сколько было выдано кассиром \( 50 \)-рублевых и \( 10 \)-рублевых купюр в отдельности?
		 \item  Два велосипедиста выехали одновременно навстречу друг другу из двух городов, расстояние между которыми 90 км. Через 3 ч они встретились, причем первый велосипедист проехал на 6 км больше второго. Найди скорость каждого селосипедиста.
		 \item  При каком значении \( a \) система уравнений \( \begin{cases}
		 	3x+ay=4 \\\
		 	6x-2y=8
		 \end{cases} \) имеет бесконечно много решений?
	\end{listofex}
\end{class}
%END_FOLD

%BEGIN_FOLD % ====>>_ Домашняя работа 1 _<<====
\begin{homework}[number=1]
	\begin{listofex}
		\item Решите систему уравнений \begin{tasks}(2)
			\task \( \begin{cases}
				x+5y=0 \\\
				3x+7y-16=0
			\end{cases} \)
			\task \( \begin{cases}
				7x-y=0 \\\
				3x-y+12=0
			\end{cases} \)
			\task \( \begin{cases}
				x-y-2=0 \\\
				x+y-6=0
			\end{cases} \)
			\task \( \begin{cases}
				x-2y-3=0 \\\
				5x+y-4=0
			\end{cases} \)
			\task \( \begin{cases}
				x+4y-2=0 \\\
				3x+8y-2=0
			\end{cases} \)
			\task \( \begin{cases}
				3x-2y-4=0 \\\
				x+5y-7=0
			\end{cases} \)
			\task \( \begin{cases}
				7x-2y-6=0 \\\
				x+4y+12=0
			\end{cases} \)
			\task \( \begin{cases}
				2x+3y-3=0 \\\
				x-y+6=0
			\end{cases} \)
		\end{tasks}
		
	\end{listofex}
\end{homework}
%END_FOLD

%BEGIN_FOLD % ====>>_____ Занятие 3 _____<<====
\begin{class}[number=3]
	\begin{listofex}
		\item Занятие 3 
	\end{listofex}
\end{class}
%END_FOLD

%BEGIN_FOLD % ====>>_____ Занятие 4 _____<<====
\begin{class}[number=4]
	\begin{listofex}
		\item Занятие 4
	\end{listofex}
\end{class}
%END_FOLD

%BEGIN_FOLD % ====>>_ Домашняя работа 2 _<<====
\begin{homework}[number=2]
	\begin{listofex}
		\item Домашняя работа 2
	\end{listofex}
\end{homework}
%END_FOLD

%BEGIN_FOLD % ====>>_____ Занятие 5 _____<<====
\begin{class}[number=5]
	\begin{listofex}
		\item Занятие 5
	\end{listofex}
\end{class}
%END_FOLD

%BEGIN_FOLD % ====>>_____ Занятие 6 _____<<====
\begin{class}[number=6]
	\begin{listofex}
		\item Занятие 6
	\end{listofex}
\end{class}
%END_FOLD

%BEGIN_FOLD % ====>>_ Домашняя работа 3 _<<====
\begin{homework}[number=3]
	\begin{listofex}
		\item Домашняя работа 3
	\end{listofex}
\end{homework}
%END_FOLD

%BEGIN_FOLD % ====>>_____ Занятие 7 _____<<====
\begin{class}[number=7]
	\title{Подготовка к проверочной}
	\begin{listofex}
		\item Занятие 7
	\end{listofex}
\end{class}
%END_FOLD

=%BEGIN_FOLD % ====>>_ Проверочная работа _<<====
\begin{exam}
	\begin{listofex}
		\item Проверочная
	\end{listofex}
\end{exam}
%END_FOLD