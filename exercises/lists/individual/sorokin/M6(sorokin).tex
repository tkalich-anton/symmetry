%
%===============>>  Сорокин Модуль 6 <<=============
%=
\setmodule{6}

%BEGIN_FOLD % ====>>_____ Занятие 1 _____<<====
\begin{class}[number=1]
	\begin{definit}
		Сумма внутренних углов в треугольнике равна \( 180\degree \).
	\end{definit}
	\begin{definit}
		\textbf{Внешний угол} --- угол между стороной треугольника и продолжением другой стороны. Внешний угол является смежным с одним из внутренних.
	\end{definit}
	\begin{listofex}
		\item В треугольнике \( ABC \) два угла равны \( 50\) и \( 70 \) градусов. Найдите третий угол.
		\item Один угол треугольника равен \( 26\degree \), а второй в три раза больше. Найдите третий угол.
		\item Один внутренний угол треугольника в два, а второй в три раза больше третьего, найдите все углы треугольника.
		\item Один внешний угол равен \( 40\degree \), а второй --- \( 100\degree \). Чему равны внутренние углы треугольника?
		\item Угол треугольника равен \( 30\degree \), второй угол в \( 3 \) раза больше первого. Чему равны внешние углы при каждой вершине? Чему равна сумма внешних углов?
		\item В прямоугольном треугольнике один угол равен \( 40 \) градусов. Найдите сумму наибольшего и наименьшего угла.
		\item В прямоугольном треугольнике один острый угол на \( 17 \) градусов больше другого. Найдите углы треугольника.
		\item В прямоугольном треугольнике два острых угла равны. Какая у них градусная мера?
	\end{listofex}
	\begin{definit}
		Если секущая пересекает две параллельные прямые, то:
		\begin{tasks}(1)
			\task внутренние накрест лежащие углы равны: \( \angle 4 \) и \( \angle5 \), \( \angle 3 \) и \( \angle 6\);
			\task сумма внутренних односторонних углов равна  \(180\degree\): \( \angle 3 \) и \( \angle 5 \), \( \angle 4 \) и \( \angle 6 \);
			\task соответственные углы равны: \( \angle 1 \) и \( \angle 5 \), \( \angle 2 \) и \( \angle 6 \), \( \angle 3 \) и \( \angle 7 \), \( \angle 4 \) и \( \angle 8 \);
			\task внешние накрест лежащие углы равны: \( \angle 2 \) и \( \angle 7 \), \( \angle 1 \) и \( \angle 8 \);
			\task сумма внешних односторонних углов равна  \(180\degree\): \( \angle 1 \) и \( \angle 7 \), \( \angle 2 \) и \( \angle 8 \).
		\end{tasks}
	\begin{minipage}[c]{0.9\linewidth}
		\includegraphics[align=t, width=\linewidth]{../\picpath/sorokinM6L1-1}
	\end{minipage}
	\end{definit}
	\begin{listofex}[resume]
		\item Сумма накрест лежащих углов при пересечении двух параллельных прямых секущей равна \(210 \degree \). Найдите эти углы.
		\item Найдите все углы, образованные при пересечении параллельных прямых \(a\) и \(b\) с секущей \(c\), если один из углов равен \( 150 \degree \).
		\item Через вершину \(C\) треугольника \(ABC\) проведена прямая, параллельная биссектрисе \(BD\) угла \(ABC\). Эта прямая пересекает прямую \(AB\) в точке \(K\). Найдите углы треугольника \(BKC\), если \(\angle ABC = 130 \degree\).
		\item Отрезки \(AB\) и \(CD\) пересекаются в точке \(O\) и делятся этой точкой пополам. Докажите, что \(AC \parallel BD\) и \(AD \parallel BC\).
		
	\end{listofex}
\end{class}
%END_FOLD

%BEGIN_FOLD % ====>>_ Домашняя работа 1 _<<====
\begin{homework}[number=1]
	\begin{listofex}
		\item Домашняя работа
	\end{listofex}
\end{homework}
%END_FOLD

%BEGIN_FOLD % ====>>_____ Занятие 2 _____<<====
\begin{class}[number=2]
	\begin{listofex}
		\item Занятие 2
	\end{listofex}
\end{class}
%END_FOLD

%BEGIN_FOLD % ====>>_ Домашняя работа 2 _<<====
\begin{homework}[number=2]
	\begin{listofex}
		\item Домашняя работа
	\end{listofex}
\end{homework}
%END_FOLD

%BEGIN_FOLD % ====>>_____ Занятие 3 _____<<====
\begin{class}[number=3]
	\begin{listofex}
		\item Занятие 3
	\end{listofex}
\end{class}
%END_FOLD

%BEGIN_FOLD % ====>>_ Домашняя работа 3 _<<====
\begin{homework}[number=3]
	\begin{listofex}
		\item Домашняя работа
	\end{listofex}
\end{homework}
%END_FOLD

%BEGIN_FOLD % ====>>_____ Занятие 4 _____<<====
\begin{class}[number=4]
	\begin{listofex}
		\item Пусто
	\end{listofex}
\end{class}
%END_FOLD


%BEGIN_FOLD % ====>>_ Проверочная работа _<<====
\begin{exam}
	\begin{listofex}
		\item Проверочная
	\end{listofex}
\end{exam}
%END_FOLD