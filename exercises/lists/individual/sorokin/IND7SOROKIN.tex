%
%===============>>  Сорокин Модуль 5 <<=============
%
\setmodule{5}
%
%BEGIN_FOLD % ====>>_____ Занятие 1 _____<<====
\begin{class}[number=1]
	\begin{listofex}
		\item Вычислить:
		\begin{tasks}(1)
			\task \( \left( 1,6-\mfrac{2}{1}{6}+\dfrac{41}{90} \right)\cdot\mfrac{3}{3}{5}-0,25:1,25 \)
			\task \( 12:\mfrac{7}{1}{2}+7,5:12+\dfrac{1}{4}:0,4\cdot(5,1-3,86) \)
			\task \( \dfrac{12,8\cdot\mfrac{3}{3}{4}-\mfrac{4}{4}{11}\cdot4,125}{\mfrac{2}{4}{7}:\dfrac{3}{35}} \)
		\end{tasks}
		\item Вычислить:
		\begin{tasks}(1)
			\task \( (5a+3)(2a-4)-2a^2+7a \) при \( a=2 \)
			\task \( (x+1)(2x^2-1)-3x^3 \) при \( x=\mfrac{3}{2}{5} \)
			\task \( (3x-2y)^2-(2x-y)^2 \) при \( x=2,35,\,y=-1,65 \)
		\end{tasks}
		\item Сравнить:
		\begin{tasks}(2)
			\task \( \left( \dfrac{1}{2} \right)^5 \) и \( \left( \dfrac{3}{4} \right)^3 \)
			\task \( \left( -\dfrac{2}{5} \right)^3 \) и \( -\left( \dfrac{1}{2} \right)^5 \)
		\end{tasks}
		\item Расположите в порядке возрастания:
		\[ \left( \dfrac{1}{3} \right)^2,\,\dfrac{2}{3},\,\left( -\dfrac{1}{3} \right)^3,\,\left( \mfrac{3}{1}{2} \right)^2,\,\dfrac{5}{9} \]
		\item Решить уравнение:
		\[ 2x(x-1)-2x^2=4x-12 \]
	\end{listofex}
\end{class}
%END_FOLD

%BEGIN_FOLD % ====>>_____ Занятие 2 _____<<====
\begin{class}[number=2]
	\begin{listofex}
		\item Вычислить: 
		\begin{tasks}(1)
			\task \( \left(  1,5 : \dfrac{1}{3} - \dfrac{3}{8} : 0,25 \right) \cdot 3,2 - 3,2 \cdot \dfrac{5}{8} \)
			\task \( \left(  3,6 \cdot \mfrac{2}{7}{9} + 1,125 + \mfrac{5}{2}{5} \cdot  \mfrac{2}{7}{9} - \mfrac{1}{1}{8} \right) : 2,5 \)
		\end{tasks}
		\item Применить формулу квадрата суммы/разности и привести подобные слагаемые:
		\begin{tasks}(2)
			\task \( (x+5)^2+(3x)^2 \)
			\task \( (10x-15)^2-(6-9x)^2 \)
			\task \( (0,5x+5,2)^2-(6x+0,12)^2 \)
			\task \( (x-y)^2-2xy+(x^2+5)^2 \)
			\task \( (x+y^2)^2 + (y-6)^2 + (x^2+y)^2 \)
			\task  \( (xy-5)^2 + (x-0,2y)^2 \)
		\end{tasks}
		\item Решите уравнения: 
		\begin{tasks}(1)
			\task \( (3x+4)^2=(3x-2)(2+3x) \)
			\task \( (1-6x)(2x-5)=(3x+4)(-4x-1) \)
		\end{tasks}
		\item Сравнить: 
		\begin{tasks}(2)
			\task \( \left( \dfrac{1}{4} \right)^2\) и \( \left( \dfrac{1}{2} \right)^3\)
			\task \( \left( \dfrac{6}{7} \right)^4\) и \( \left( \dfrac{4}{9} \right)^3\)
		\end{tasks}
		
	\end{listofex}
\end{class}
%END_FOLD

%BEGIN_FOLD % ====>>_ Домашняя работа 1 _<<====
\begin{homework}[number=1]
	\begin{listofex}
		\item Вычислите: \(12:\mfrac{7}{1}{2}+7,5:12+\dfrac{1}{4}:0,4\cdot(5,1-3,86)\)
		\item Вычислите: \((3x-2y)^2-(2x-y)^2\), при \(x=2,35, y=-1,65\)
		\item Сравнить: \(- \left( \dfrac{2}{5} \right)^3\) и \(- \left( \dfrac{1}{2} \right)^5\)
		\item Решить уравнение: \quad \( -3x^2+4x-7=-x^2+5x-(-1+2x^2) \)
		\item Применить формулу квадрата суммы/разности и привести подобные слагаемые:
		\begin{tasks}(2)
			\task \( (4x+1)^2+4x^2-15x \)
			\task \( (0,3x-4)^2+(x+1,5)^2 \)
			\task \( (9,9x-3y)^2-(4x+0,12y)^2 \)
			\task \( (3x^3+4)^2-9x^6 \)
		\end{tasks}\
	\end{listofex}
\end{homework}
%END_FOLD

%BEGIN_FOLD % ====>>_____ Занятие 3 _____<<====
\begin{class}[number=3]
	\begin{listofex}
		\item Вычислите: \[ \dfrac{7}{40} : \mfrac{2}{11}{12} - 0,1 \cdot \left( 1,45 : \mfrac{2}{1}{3} - \dfrac{1}{20} : \mfrac{2}{1}{3} \right) \]
			%\task \( 20 : \mfrac{33}{1}{3} - \left( \mfrac{4}{7}{25} - 1,28 \right) : \left( 0,75 + \mfrac{3}{1}{4} \right) \cdot 0,2 \)
		\item Примените формулы разности квадратов:
		\begin{tasks}(2)
			\task \( x^2-y^2 \)
			\task \( 0,01a^2-9b^2 \)
			\task \( 121x^4-64y^6 \)
			\task \( \dfrac{1}{9}a^4-\dfrac{25}{49}b^9 \)
			\task \( \dfrac{9}{4}x^{16}-81y^6 \)
		\end{tasks}
		\item Сравните: \( \left( \dfrac{7}{12} \right)^2\) и \( \left( \dfrac{8}{11} \right)^2\)
		\item Применить формулу квадрата суммы/разности и привести подобные слагаемые:
		\begin{tasks}(2)
			\task \( (a+b)^2+(2a-3b)^2 \)
			\task \( 3(4a-2b)^2+\dfrac{(a-4b)^2}{2} \)
			\task \( 2(a-0,5b)^2-3(0,2a-3b)^2 \)
			\task \( (2a^2-3)^2+(3b+1)^2 \)
		\end{tasks}
		\item Решите уравнения:
		\begin{tasks}(1)
			\task \( (x-9)(6x+5)-(x-5)(6x-1)=0 \)
			\task \( (9x-5)(3x-1)+(5-9x)(3x-3)=0 \)
			%\task 
		\end{tasks}
	\end{listofex}
\end{class}
%END_FOLD

%BEGIN_FOLD % ====>>_____ Занятие 4 _____<<====
\begin{class}[number=4]
	\begin{listofex}
		\item Занятие 4
	\end{listofex}
\end{class}
%END_FOLD

%BEGIN_FOLD % ====>>_ Домашняя работа 2 _<<====
\begin{homework}[number=2]
	\begin{listofex}
		\item Вычислите: \( 20 : \mfrac{33}{1}{3} - \left( \mfrac{4}{7}{25} - 1,28 \right) : \left( 0,75 + \mfrac{3}{1}{4} \right) \cdot 0,2 \)
		\item Примените формулы разности квадратов:
		\begin{tasks}(3)
			\task \( x^4-y^2 \)
			\task \( 49a^{12}-\dfrac{25}{9}b^4 \)
			\task \( \dfrac{1}{4}a^2-0,01b^{0,25} \)
		\end{tasks}
		\item Решите уравнения: 
		\begin{tasks}(1)
			\task \( (4x+7)(-2x-1)+(8x+1)(x+2)=0 \)
			\task \( (25x-17)(x+5)=(5x+4)^2 \)
		\end{tasks}
		\item Примените формулу квадрата суммы/разности и привести подобные слагаемые:
		\begin{tasks}(2)
			\task \( (7a-2b)^2+(5a-0,2b)^2 \)
			\task \( 3\left(\dfrac{1}{2}a-2b\right)^2+(0,6a-4b)^2 \)
			\task \( (6a+8b)^2 - (0,5b-4a)^2 \)
			\task \( \left(\dfrac{4}{5}a+4b \right)^2+(0,2a^2-5b)^2 \)
		\end{tasks}
		\item Вычислить:
		\begin{tasks}(4)
			\task \( 0,8 \cdot 1,5 \)
			\task \( 1,1 \cdot 2,8 \)
			\task \( 1,25 \cdot 8,4 \)
			\task \( 15,5 \cdot 0,07 \)
			\task \( 1,779 \cdot 1,25 \)
			\task \( 0,042 \cdot 10,99 \)
			\task \( 7,452 \cdot 15,117 \)
			\task \( 250,4 \cdot 0,008 \)
		\end{tasks}
	\end{listofex}
\end{homework}
%END_FOLD

%BEGIN_FOLD % ====>>_____ Занятие 5 _____<<====
\begin{class}[number=5]
	\begin{listofex}
		\item Занятие 5
	\end{listofex}
\end{class}
%END_FOLD

%BEGIN_FOLD % ====>>_____ Занятие 6 _____<<====
\begin{class}[number=6]
	\begin{listofex}
		\item Занятие 6
	\end{listofex}
\end{class}
%END_FOLD

%BEGIN_FOLD % ====>>_ Домашняя работа 3 _<<====
\begin{homework}[number=3]
	\begin{listofex}
		\item ДЗ 3
	\end{listofex}
\end{homework}
%END_FOLD

%BEGIN_FOLD % ====>>_____ Занятие 7 _____<<====
\begin{class}[number=7]
	\begin{listofex}
		\item Занятие 7
	\end{listofex}
\end{class}
%END_FOLD

%BEGIN_FOLD % ====>>_ Проверочная работа _<<====
\begin{exam}
	\begin{listofex}
		\item Проверочная работа
	\end{listofex}
\end{exam}
%END_FOLD
