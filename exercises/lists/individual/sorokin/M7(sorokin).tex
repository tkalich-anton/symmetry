%
%===============>>  Сорокин Модуль 7 <<=============
%
\setmodule{7}

%BEGIN_FOLD % ====>>_____ Занятие 1 _____<<====
\begin{class}[number=1]
	\begin{listofex}
		\item В треугольнике \( ABC \) \( AD \)  — биссектриса, угол \( C \) равен \( 50^{\circ} \), угол \( CAD \) равен \( 28^{\circ} \). Найдите угол \( B \). Ответ дайте в градусах.
		\item В остроугольном треугольнике \( ABC \) угол \( A \) равен \( 65^{\circ} \). \( BD \) и \( CE \)  — высоты, пересекающиеся в точке \( O \). Найдите угол \( DOE \). Ответ дайте в градусах.
		\item В треугольнике \( ABC \) проведена биссектриса \( AD \) и \( AB = AD = CD \). Найдите меньший угол треугольника \( ABC \). Ответ дайте в градусах.
		\item Один из углов треугольника в \( 2 \) раза больше второго, а третий угол равен \(  33^{\circ} \). Определите два неизвестных угла треугольника.
		\item
		\begin{minipage}[t]{\bodywidth}
			Найдите угол \( \alpha \) (если на чертеже необходимо выделить четыре или более углов, то их отмечают одной дугой).
		\end{minipage}
		\hspace{0.02\linewidth}
		\begin{minipage}[t]{\picwidth}
			\includegraphics[align=t, width=0.8\linewidth]{../../../../../exercises/lists/pics/sorokinM7L1-2}
		\end{minipage}
		\item В треугольнике \( ABC \) углы \( A \) и \( C \) равны \( 20^{\circ } \) и \( 60^{\circ} \) соответственно. Найдите угол между высотой \( BH \) и биссектрисой \( BD \). 
		\item В треугольнике \( ABC \) угол \( A \) равен \( 30^{\circ} \), угол \( B \)  — тупой, \( CH \)  — высота, угол \( BCH  \) равен \( 22^{\circ} \). Найдите угол \( ACB \). Ответ дайте в градусах.
	\end{listofex}
\end{class}
%END_FOLD

%BEGIN_FOLD % ====>>_ Домашняя работа 1 _<<====
\begin{homework}[number=1]
	\begin{listofex}
		
		\item В прямоугольном треугольнике угол между высотой и медианой, проведенными из вершины прямого угла, равен \( 40^{\circ} \). Найдите больший из острых углов этого треугольника.
		\item В треугольнике \( ABC \) углы \( A \) и \( C \) равны \( 30^{\circ} \) и \( 50^{\circ} \) соответственно. Найдите угол между высотой \( BH \) и биссектрисой \( BD \).
		\item В треугольнике \( ABC \) угол \( A \) равен \( 38^{\circ} \), угол \( B \) - тупой, \( CH \)  — высота, угол \( BCH \) равен \( 35^{\circ} \). Найдите угол \( ACB \). Ответ дайте в градусах.
		\item В треугольнике \( ABC \) угол \(  A \) равен \( 7^{\circ} \), внешний угол при вершине \( B \) равен \( 22^{\circ} \). Найдите угол \( C \).
	\end{listofex}
\end{homework}
%END_FOLD

%BEGIN_FOLD % ====>>_____ Занятие 2 _____<<====
\begin{class}[number=2]
	\begin{listofex}
		\item Один из углов треугольника в \( 4 \) раза больше третьего, а второй угол угол равен \(  25^{\circ} \). Определите два неизвестных угла треугольника.
		\item 
		\begin{minipage}[t]{\bodywidth}
			В треугольнике \( ABC \)  \( AB = BC \), \( BAD=105\degree \). Найдите угол \( MCN \).
		\end{minipage}
		\hspace{0.02\linewidth}
		\begin{minipage}[t]{\picwidth}
			\includegraphics[align=t, width=1.3\linewidth]{../../../../../exercises/lists/pics/sorokinM7L2-1}
		\end{minipage} 
		\item 
		\begin{minipage}[t]{\bodywidth}
			В треугольнике \( ABC \) \( \angle A = 40^{\circ} \), \( 
			\angle C = 60^{\circ} \), \( AK = AB \), \( CM = CB \). Найдите угол \( KBM \).
		\end{minipage}
		\hspace{0.02\linewidth}
		\begin{minipage}[t]{\picwidth}
			\includegraphics[align=t, width=1.3\linewidth]{../../../../../exercises/lists/pics/sorokinM7L1-1}
		\end{minipage} 
		\item Углы треугольника относятся как \(  2:8:35 \). Найдите меньший из них. Ответ дайте в градусах.
		\item Углы треугольника относятся как \( 2:3:4 \). Найдите отношение соответствующих внешних углов треугольника,
		взятых по одному при каждой вершине.
		\item Два угла треугольника равны \( 58^{\circ} \) и \( 72^{\circ} \). Найдите тупой угол, который образуют высоты треугольника, выходящие из вершин этих углов. Ответ дайте в градусах.
		\item В треугольнике \( ABC \) \( CH \)  — высота, \( AD \)  — биссектриса, \( O \) — точка пересечения прямых \( CH \) и \( AD \), угол \( BAD \) равен \( 26^{\circ} \). Найдите угол \( AOC \). Ответ дайте в градусах.
		\item В треугольнике \( ABC \) угол \( A \) равен \( 30^{\circ} \), угол \( B \) равен \( 86^{\circ} \), \( CD \)  — биссектриса внешнего угла при вершине \( C \), причем точка \( D \) лежит на прямой \( AB \). На продолжении стороны \( AC \) за точку \( C  \) выбрана такая точка \( E \), что \( CE  =  CB \). Найдите угол \( BDE \). Ответ дайте в градусах.
	\end{listofex}
\end{class}
%END_FOLD

%BEGIN_FOLD % ====>>_ Домашняя работа 2 _<<====
\begin{homework}[number=2]
	\begin{listofex}
		\item Домашняя работа
	\end{listofex}
\end{homework}
%END_FOLD

%BEGIN_FOLD % ====>>_____ Занятие 3 _____<<====
\begin{class}[number=3]
	\begin{listofex}
		\item Занятие 3
	\end{listofex}
\end{class}
%END_FOLD

%BEGIN_FOLD % ====>>_ Домашняя работа 3 _<<====
\begin{homework}[number=3]
	\begin{listofex}
		\item Домашняя работа
	\end{listofex}
\end{homework}
%END_FOLD

%BEGIN_FOLD % ====>>_____ Занятие 4 _____<<====
\begin{class}[number=4]
	\begin{listofex}
		\item Пусто
	\end{listofex}
\end{class}
%END_FOLD


%BEGIN_FOLD % ====>>_ Проверочная работа _<<====
\begin{exam}
	\begin{listofex}
		\item Проверочная
	\end{listofex}
\end{exam}
%END_FOLD