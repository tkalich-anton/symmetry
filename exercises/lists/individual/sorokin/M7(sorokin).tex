%
%===============>>  Сорокин Модуль 7 <<=============
%
\setmodule{7}

%BEGIN_FOLD % ====>>_____ Занятие 1 _____<<====
\begin{class}[number=1]
	\begin{listofex}
		\item В треугольнике \( ABC \) \( AD \)  — биссектриса, угол \( C \) равен \( 50^{\circ} \), угол \( CAD \) равен \( 28^{\circ} \). Найдите угол \( B \). Ответ дайте в градусах.
		\item В остроугольном треугольнике \( ABC \) угол \( A \) равен \( 65^{\circ} \). \( BD \) и \( CE \)  — высоты, пересекающиеся в точке \( O \). Найдите угол \( DOE \). Ответ дайте в градусах.
		\item В треугольнике \( ABC \) проведена биссектриса \( AD \) и \( AB = AD = CD \). Найдите меньший угол треугольника \( ABC \). Ответ дайте в градусах.
		\item Один из углов треугольника в \( 2 \) раза больше второго, а третий угол равен \(  33^{\circ} \). Определите два неизвестных угла треугольника.
		\item
		\begin{minipage}[t]{\bodywidth}
			Найдите угол \( \alpha \) (если на чертеже необходимо выделить четыре или более углов, то их отмечают одной дугой).
		\end{minipage}
		\hspace{0.02\linewidth}
		\begin{minipage}[t]{\picwidth}
			\includegraphics[align=t, width=0.8\linewidth]{../../../../../exercises/lists/pics/sorokinM7L1-2}
		\end{minipage}
		\item В треугольнике \( ABC \) углы \( A \) и \( C \) равны \( 20^{\circ } \) и \( 60^{\circ} \) соответственно. Найдите угол между высотой \( BH \) и биссектрисой \( BD \). 
		\item В треугольнике \( ABC \) угол \( A \) равен \( 30^{\circ} \), угол \( B \)  — тупой, \( CH \)  — высота, угол \( BCH  \) равен \( 22^{\circ} \). Найдите угол \( ACB \). Ответ дайте в градусах.
	\end{listofex}
\end{class}
%END_FOLD

%BEGIN_FOLD % ====>>_ Домашняя работа 1 _<<====
\begin{homework}[number=1]
	\begin{listofex}
		
		\item В прямоугольном треугольнике угол между высотой и медианой, проведенными из вершины прямого угла, равен \( 40^{\circ} \). Найдите больший из острых углов этого треугольника.
		\item В треугольнике \( ABC \) углы \( A \) и \( C \) равны \( 30^{\circ} \) и \( 50^{\circ} \) соответственно. Найдите угол между высотой \( BH \) и биссектрисой \( BD \).
		\item В треугольнике \( ABC \) угол \( A \) равен \( 38^{\circ} \), угол \( B \) - тупой, \( CH \)  — высота, угол \( BCH \) равен \( 35^{\circ} \). Найдите угол \( ACB \). Ответ дайте в градусах.
		\item В треугольнике \( ABC \) угол \(  A \) равен \( 7^{\circ} \), внешний угол при вершине \( B \) равен \( 22^{\circ} \). Найдите угол \( C \).
	\end{listofex}
\end{homework}
%END_FOLD

%BEGIN_FOLD % ====>>_____ Занятие 2 _____<<====
\begin{class}[number=2]
	\begin{listofex}
		\item Один из углов треугольника в \( 4 \) раза больше третьего, а второй угол угол равен \(  25^{\circ} \). Определите два неизвестных угла треугольника.
		\item 
		\begin{minipage}[t]{\bodywidth}
			В треугольнике \( ABC \)  \( AB = BC \), \( BAD=105\degree \). Найдите угол \( MCN \).
		\end{minipage}
		\hspace{0.02\linewidth}
		\begin{minipage}[t]{\picwidth}
			\includegraphics[align=t, width=1.3\linewidth]{../../../../../exercises/lists/pics/sorokinM7L2-1}
		\end{minipage} 
		\item 
		\begin{minipage}[t]{\bodywidth}
			В треугольнике \( ABC \) \( \angle A = 40^{\circ} \), \( 
			\angle C = 60^{\circ} \), \( AK = AB \), \( CM = CB \). Найдите угол \( KBM \).
		\end{minipage}
		\hspace{0.02\linewidth}
		\begin{minipage}[t]{\picwidth}
			\includegraphics[align=t, width=1.3\linewidth]{../../../../../exercises/lists/pics/sorokinM7L1-1}
		\end{minipage} 
		\item Известно, что  \( \angle BAC = 68\degree \), \( AM \) — биссектриса угла \( BAC \),\( AK \) — биссектриса угла \(  MAC \). Найдите градусную меру угла \( BAK \).
		\item После того как один из смежных углов увеличили на \( 40\% \), другой угол уменьшился на \( 60 \% \). Найдите, какими были по величине первоначально два данных смежных угла.
		\item Углы треугольника относятся как \(  2:8:35 \). Найдите меньший из них. Ответ дайте в градусах.
		\item Углы треугольника относятся как \( 2:3:4 \). Найдите отношение соответствующих внешних углов треугольника,
		взятых по одному при каждой вершине.
		\item Два угла треугольника равны \( 58^{\circ} \) и \( 72^{\circ} \). Найдите тупой угол, который образуют высоты треугольника, выходящие из вершин этих углов. Ответ дайте в градусах.
		\item В треугольнике \( ABC \) \( CH \)  — высота, \( AD \)  — биссектриса, \( O \) — точка пересечения прямых \( CH \) и \( AD \), угол \( BAD \) равен \( 26^{\circ} \). Найдите угол \( AOC \). Ответ дайте в градусах.
		\item В треугольнике \( ABC \) угол \( A \) равен \( 30^{\circ} \), угол \( B \) равен \( 86^{\circ} \), \( CD \)  — биссектриса внешнего угла при вершине \( C \), причем точка \( D \) лежит на прямой \( AB \). На продолжении стороны \( AC \) за точку \( C  \) выбрана такая точка \( E \), что \( CE  =  CB \). Найдите угол \( BDE \). Ответ дайте в градусах.
		\item Внутри угла \( BAC \), равного \( 114^{\circ} \), из его вершины проведен луч \( AE \). Угол BAE в \( 2 \) раза больше угла \( EAC \). Найти величину угла \( BAE \). 
	\end{listofex}
\end{class}
%END_FOLD

%BEGIN_FOLD % ====>>_ Домашняя работа 2 _<<====
\begin{homework}[number=2]
	\begin{listofex}
		\item Известно, что  \( \angle BAC = 52\degree \), \( AM \) — биссектриса угла \( BAC \),\( AK \) — биссектриса угла \(  MAC \). Найдите градусную меру угла \( BAK \).
		\item Один из внешних углов равнобедренного треугольника равен \( 84\degree \). Найдите все углы данного треугольника.
		\item Углы треугольника относятся как \(  4:13:19 \). Найдите сумму самого большого и самого маленького угла. Ответ дайте в градусах.
		\item В треугольнике \( ABC \) \( CH \)  — высота, \( AD \)  —  биссектриса, \( O \) — точка пересечения прямых \( CH \) и \( AD \), угол \( BAD \) равен \( 31^{\circ} \). Найдите угол \( HCD \). Ответ дайте в градусах.
		\item Две пересекающиеся прямые образовали углы. Один из них равен \( 46\degree \). Найдите остальные углы треугольника. Ответ дайте в градусах.
	\end{listofex}
\end{homework}
%END_FOLD

%BEGIN_FOLD % ====>>_____ Занятие 3 _____<<====
\begin{class}[number=3]
	\begin{listofex}
		\item Точки \(A\) и \(C\) лежат по одну сторону от прямой \(a\). Перпендикуляры \(AB\) и \(CD\) к прямой \(a\) равны. % 105
		\begin{tasks}
			\task Докажите, что \( \angle ABD = \angle CDB \),
			\task Найдите  \( \angle ABC \), если \( \angle ADB =44 \degree \).
		\end{tasks}
		\item 
		\begin{minipage}[t]{\bodywidth}
			Докажите, что треугольники \(TCO\) и \( BOP \) равны.
		\end{minipage}
		\hspace{0.02\linewidth}
		\begin{minipage}[t]{\picwidth}
			\includegraphics[align=t, width=\linewidth]{../../../../../exercises/lists/pics/G71M7L6-1}
		\end{minipage}
		\item %187
		\begin{minipage}[t]{\bodywidth}
			По данным рисунка докажите, что \(AB \parallel DE\).
		\end{minipage}
		\hspace{0.02\linewidth}
		\begin{minipage}[t]{\picwidth}
			\includegraphics[align=t, width=\linewidth]{../../../../../exercises/lists/pics/G71M7L6-2}
		\end{minipage}
		\item В треугольнике \(ABC\) угол \(A\) равен \(40 \degree\), а угол \(BCE\), смежный с углом \(ABC\), равен \(80 \degree\). Докажите, что биссектриса угла \(BCE\) параллельна \(AB\). %192
		
		%191
		\item Отрезок \(BK\) --- биссектриса треугольника \(ABC\). Через точку \(K\) проведена прямая, пересекающая сторону \(BC\) в точке \(M\) так, что \(BM=MK\). Докажите, что \(KM \parallel AB\). 
		\item На сторонах вертикальных углов отложены от его вершины равные отрезки \(OA, OB, OC\) и \(OD\). Укажите пары равных треугольников с вершинами в точках \(O, A, B, C\) и \(D\).
		%190
		\item 
		\begin{minipage}[t]{\bodywidth}
			На рисунке \(AB=BC, AD=DE, \angle C = 70 \degree, \angle EAC = 35 \degree\). Докажите, что \(DE \parallel AC\)
		\end{minipage}
		\hspace{0.02\linewidth}
		\begin{minipage}[t]{\picwidth}
			\includegraphics[align=t, width=\linewidth]{../../../../../exercises/lists/pics/G71M7L6-3}
		\end{minipage}
	\end{listofex}
\end{class}
%END_FOLD

%BEGIN_FOLD % ====>>_ Домашняя работа 3 _<<====
\begin{homework}[number=3]
	\begin{listofex}
		\item  Угол \( ABC\) равен \( 70\degree \), а угол \( BCD \) равен \( 110\degree \). Могут ли прямые \( AB \) и \( CD \) быть:
		\begin{tasks}(1)
			\task[a)] параллельными
			\task[б)] пересекающимися?
		\end{tasks}
		\item Периметр равнобедренного треугольника равен \( 1 \) м, а основание равно \( 0,4 \) м найдите длину боковой стороны.
		\item Отрезки \( AC \) и \( BD \) перескекаются в точке \( O \). Докажите равенство треугольников \( BAO \) и \( DCO \), если известно, что угол \( BAO \) равен углу \( DCO \) и \( AO=CO \).
		\item 
		\begin{minipage}[t]{\bodywidth}
			Дано: \\\ \( AC \) ‒ биссектриса \( \angle A \), \( AB=AD \).\\\ Доказать: \( BC = CD \).
		\end{minipage}
		\hspace{0.02\linewidth}
		\begin{minipage}[t]{\picwidth}
			\includegraphics[align=t, width=\linewidth]{../../../../../exercises/lists/pics/sorokinM7H3-1}
		\end{minipage}
		\item На сторонах угла \( BAC \) отложили равные отрезки \( AM \) и \( AN \). На биссектрисе угла \( A \) взяли точку \( D \) и соединили с \( M \) и \( N \). Доказать, что \( DM=DN \).
	\end{listofex}
\end{homework}
%END_FOLD

%BEGIN_FOLD % ====>>_____ Занятие 4 _____<<====
\begin{class}[number=4]
	\begin{listofex}
		\item 
		\begin{minipage}[t]{\bodywidth}
			Дано: \( a || b \), \( c \) --- секущая, \( \angle 1 + \angle 2 = 102\degree \). Найти все образовавшиеся углы.
		\end{minipage}
		\hspace{0.02\linewidth}
		\begin{minipage}[t]{\picwidth}
			\includegraphics[align=t, width=\linewidth]{../../../../../exercises/lists/pics/sorokinM7L4-1}
		\end{minipage}
		\item 
		\begin{minipage}[t]{\bodywidth}
			Дано: \( \angle 1 = \angle 2\), \(  \angle 3 = 120\degree \). Найти: \( \angle 4 \).
		\end{minipage}
		\hspace{0.02\linewidth}
		\begin{minipage}[t]{\picwidth}
			\includegraphics[align=t, width=\linewidth]{../../../../../exercises/lists/pics/sorokinM7L4-2}
		\end{minipage}
		\item Отрезок \( AD \) --- биссектриса треугольника \( ABC \). Через точку \( D \) проведена прямая, параллельная стороне \( AB \) и пересекающая сторону \( AC \) в точке \( F \). Найти углы треугольника \( ADF \), если \( \angle BAC = 72\degree \).
		\item Прямая \( EK \) является секущей для прямых \( CD \) и \( MN  \) \( (E \in CD \), \(K \in MN) \). \( \angle DEK  \) равен \( 65\degree \). При каком значении угла \( NKE  \) прямые \( CD  \) и \( MN  \) могут быть параллельными?
		\item Через точку на плоскости провели \( 10 \) прямых, после
		чего плоскость разрезали по этим прямым на углы. Докажите,
		что хотя бы один из этих углов меньше \( 20\degree \).
		\item Докажите, что прямая, проходящая через середины
		боковых сторон равнобедренного треугольника, параллельна ос-
		нованию.
		\item Прямая, проходящая через вершину \( A \) треугольни-
		ка \( ABC \), пересекает сторону \( BC \) в точке \( M \) . При этом \( BM
		= AB \), \( \angle BAM = 35\degree \), \( \angle CAM = 15\degree \). Найдите углы треугольника \( ABC \).
		\item 
		\begin{minipage}[t]{\bodywidth}
			На рисунке точка \( O \) --- середина отрезков \( AB \) и \( CD \), \( OC= 5 \)см, \( AC=6 \)см, \( OB =  2 \) см. Найдите периметр треугольника \( OBD \).
		\end{minipage}
		\hspace{0.02\linewidth}
		\begin{minipage}[t]{\picwidth}
			\includegraphics[align=t, width=\linewidth]{../../../../../exercises/lists/pics/sorokinM7L4-3}
		\end{minipage}
		\item \begin{minipage}[t]{\bodywidth}
			Дано \( CAD =  ACB \), \( AD = BC \). Докажите, что \( AB = CD \).
		\end{minipage}
		\hspace{0.02\linewidth}
		\begin{minipage}[t]{\picwidth}
			\includegraphics[align=t, width=1.5\linewidth]{../../../../../exercises/lists/pics/sorokinM7L4-4}
		\end{minipage}
		\item \begin{minipage}[t]{\bodywidth}
			Дано \(  ABE =  ECD \), \( BE = CE \), \( BK = LC \),  \( BKE= 110\degree \)  Доказать, что треугольники \( BEK \) и \( ELC  \) равны.
		\end{minipage}
		\hspace{0.02\linewidth}
		\begin{minipage}[t]{\picwidth}
			\includegraphics[align=t, width=\linewidth]{../../../../../exercises/lists/pics/sorokinM7L4-5}
		\end{minipage}
		\item Медиана треугольника является также его высотой. Докажите, что такой треугольник равнобедренный.
		\item В равнобедренном треугольнике боковая сторона в \( 2,5 \) раза больше основания. Найдите стороны треугольника, если его периметр \( 48  \) м.
		\item На сторонах угла \( A \) отмечены точки \( M \) и \( K \) так, что \( AM = AK \), точка \( P \) лежит внутри угла \( A \) и \( PK = PM \). Докажите, что луч \( AP \) --- биссектриса угла \( MAK \).
		\item Высоты треугольника \( ABC \), проведенные из вершин \( B \)
		и \( C \), пересекаются в точке \( M \) . Известно, что \( BM = CM \) . Дока-
		жите, что треугольник \( ABC \) равнобедренный.
		
		
	\end{listofex}
\end{class}
%END_FOLD

%BEGIN_FOLD % ====>>_ Домашняя работа 3 _<<====
\begin{homework}[number=4]
	\begin{listofex}
		\item 
		\begin{minipage}[t]{\bodywidth}
			Докажите равенство треугольников \( ABF \) и \( CBD \), если  \( AB=BC \) и \( BF=BD \).
		\end{minipage}
		\hspace{0.02\linewidth}
		\begin{minipage}[t]{\picwidth}
			\includegraphics[align=t, width=\linewidth]{../../../../../exercises/lists/pics/sorokinM7H4-1}
		\end{minipage}
		\item  Найдите стороны равнобедренного треугольника, если его периметр равен \( 33 \) см, а основание на \( 3 \) см меньше боковой стороны.
		\item На боковых сторонах \( AB \) и \( BC \) равнобедренного треугольника \( ABC \) отметили соответственно точки \( D \) и  \( E \) так, что \( \angle ACD = \angle CAE \). Докажите, что \( AD=CE \).
		\item \begin{minipage}[t]{\bodywidth}
			Известно, что \( EK = FK \) и \( EC = FC \). Докажите, что \( \angle EMK = \angle FMK \).
		\end{minipage}
		\hspace{0.02\linewidth}
		\begin{minipage}[t]{\picwidth}
			\includegraphics[align=t, width=\linewidth]{../../../../../exercises/lists/pics/sorokinM7H4-2}
		\end{minipage}
		\item Серединный перпендикуляр стороны \( AB \) треугольника \( ABC \) пересекает его сторону \( AC \) в точке \( M \). Найдите сторону \( AC \) треугольника \( ABC \), если \( BC=8 \) см, а периметр треугольника \( MBC \) равен \( 25 \) см.
	\end{listofex}
\end{homework}
%END_FOLD



%BEGIN_FOLD % ====>>_ Проверочная работа _<<====
\begin{exam}
	\begin{listofex}
		\item Проверочная
	\end{listofex}
\end{exam}
%END_FOLD