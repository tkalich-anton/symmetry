%
%===============>>  Сорокин Модуль 8 <<=============
%
\setmodule{9}

%BEGIN_FOLD % ====>>_____ Занятие 1 _____<<====
\begin{class}[number=1]
	\begin{listofex}
		\item Решите систему уравнений:
		\begin{tasks}(2)
			\task \( \begin{cases}
				 5x+y-7=0 \\\
				  x-3y-11=0 
			\end{cases} \)
			\task \( \begin{cases}
				 2x+y-1=0 \\\
				  3x+2y+5=0 
			\end{cases} \)
			\task \( \begin{cases}
				 2x+y-7=0 \\\
				  x-2y+4=0 
			\end{cases} \)
			\task \( \begin{cases}
				 3x+y+5=0 \\\
				  x-3y-5=0 
			\end{cases} \)
			\task \( \begin{cases}
				 x+2y-4=0 \\\
				  3x+y+3=0 
			\end{cases} \)
			\task \( \begin{cases}
				 5x+y-15=0 \\\
				  x-2y-14=0 
			\end{cases} \)
		\end{tasks}
		\item Решите систему уравнений:
		\begin{tasks}(2)
			\task \( \begin{cases}
				 2x-3y+7=0 \\\
				  3x+4y-1=0 
			\end{cases} \)
			\task \( \begin{cases}
				 3x-3y-5=0 \\\
				  6x+8y+11=0 
			\end{cases} \)
		\end{tasks}
		\item Решите систему уравнений:
		\begin{tasks}(2)
			\task \( \begin{cases}
				 x+2y-3=0 \\\
				  x+y+1=0 
			\end{cases} \)
			\task \( \begin{cases}
				 x-3y+3=0 \\\
				  x+y-1=0 
			\end{cases} \)
			\task \( \begin{cases}
				 4x+y-2=0 \\\
				  3x+y+3=0 
			\end{cases} \)
			\task \( \begin{cases}
				 x-y-7=0 \\\
				  3x-y+1=0 
			\end{cases} \)
		\end{tasks}
		\item Решите систему уравнений:
		\begin{tasks}(2)
			\task \( \begin{cases}
				 x+3y-1=0 \\\
				  -x+4y+8=0 
			\end{cases} \)
			\task \( \begin{cases}
				 x-2y+3=0 \\\
				  -x+3y-2=0 
			\end{cases} \)
			\task \( \begin{cases}
				 x-y+2=0 \\\
				  3x+y-4=0 
			\end{cases} \)
			\task \( \begin{cases}
				 2x-y-3=0 \\\
				  -x-y+4=0 
			\end{cases} \)
		\end{tasks}
		\item Сумма двух чисел равна \( 36 \). Одно из них в \(  2 \) раза больше другого. Найдите эти числа. \item  Три тетради и две ручки стоят \( 99 \) руб. Ручка дороже тетради на \( 42 \) руб. Сколько стоит \( 1 \) ручка и одна тетрадь? 
		\item   У Лены \( 8  \) монет по \(  10 \) руб. и по \( 5 \) руб. Сколько у нее десятирублевых  и сколько пятирублевых монет, если всего у нее \( 65 \) руб.? 
		\item  Тетрадь  стоит  \( 16 \)  руб,  а  карандаш  \( 4 \)  руб.  Саша  купил  несколько  тетрадей  и  карандашей, заплатив за всю покупку \( 88 \) руб. Сколько тетрадей и сколько карандашей купил Саша, если за тетради он заплатил на \( 8 \) руб. больше, чем  за карандаши?
		
	\end{listofex}
\end{class}
%END_FOLD

%BEGIN_FOLD % ====>>_ Домашняя работа 1 _<<====
\begin{homework}[number=1]
	\begin{listofex}
		\item Решите системы уравнений
		\begin{tasks}(2)
			\task \( \begin{cases}
				2x+5y-15=0 \\\
				3x+2y-6=0 
			\end{cases} \)
			\task \( \begin{cases}
				4x-5y-3=0 \\\
				3x-2y-11=0 
			\end{cases} \)
			\task \( \begin{cases}
				2x+4y-6=0 \\\
				3x-2y-25=0 
			\end{cases} \)
			\task \( \begin{cases}
				5x+3y-7=0 \\\
				3x-5y-45=0 
			\end{cases} \)
			
		\end{tasks}
		\item Три яблока и две груши весят вместе \( 1 \) кг \( 200 \) г, а два яблока и три груши весят \( 1 \) кг \( 300 \) г. Сколько весит яблоко и сколько весит груша?
		\item Семь альбомов и две тетради стоят вместе \( 111 \) руб, а пять альбомов и три тетради стоят \( 84 \) руб. Сколько стоит один альбом и сколько стоит одна тетрадь?
		\item Мать старше дочери на \( 23 \) года, а вместе им \( 51 \) год. Сколько лет дочери?
	\end{listofex}
\end{homework}
%END_FOLD

%BEGIN_FOLD % ====>>_____ Занятие 3 _____<<====
\begin{class}[number=3]
	\begin{listofex}
		\item Решите систему уравнений:
		\begin{tasks}(2)
			\task \( \begin{cases}
				x+2y-3=0 \\\
				2x-3y+8=0
			\end{cases} \)
			\task \( \begin{cases}
				2x+y-8=0 \\\
				3x+4y-7=0 
			\end{cases} \)
			\task \( \begin{cases}
				-6x+2y+6=0 \\\
				5x-y-17=0 
			\end{cases} \)
			\task \( \begin{cases}
				5x+3y-7=0 \\\
				2x-y-5=0 
			\end{cases} \)
			\task \( \begin{cases}
				2x+5y-15=0 \\\
				3x+2y-6=0 
			\end{cases} \)
			\task \( \begin{cases}
				4x-5y-3=0 \\\
				3x-2y-11=0 
			\end{cases} \)
		\end{tasks}
		
		
	\end{listofex}
\end{class}
%END_FOLD

%BEGIN_FOLD % ====>>_____ Занятие 4 _____<<====
\begin{class}[number=4]
	\begin{listofex}
		\item Занятие 4
	\end{listofex}
\end{class}
%END_FOLD

%BEGIN_FOLD % ====>>_ Домашняя работа 2 _<<====
\begin{homework}[number=2]
	\begin{listofex}
		\item Домашняя работа 2
	\end{listofex}
\end{homework}
%END_FOLD

%BEGIN_FOLD % ====>>_____ Занятие 5 _____<<====
\begin{class}[number=5]
	\begin{listofex}
		\item Занятие 5
	\end{listofex}
\end{class}
%END_FOLD

%BEGIN_FOLD % ====>>_____ Занятие 6 _____<<====
\begin{class}[number=6]
	\begin{listofex}
		\item Занятие 6
	\end{listofex}
\end{class}
%END_FOLD

%BEGIN_FOLD % ====>>_ Домашняя работа 3 _<<====
\begin{homework}[number=3]
	\begin{listofex}
		\item Домашняя работа 3
	\end{listofex}
\end{homework}
%END_FOLD

%BEGIN_FOLD % ====>>_____ Занятие 7 _____<<====
\begin{class}[number=7]
	\title{Подготовка к проверочной}
	\begin{listofex}
		\item Занятие 7
	\end{listofex}
\end{class}
%END_FOLD

=%BEGIN_FOLD % ====>>_ Проверочная работа _<<====
\begin{exam}
	\begin{listofex}
		\item Проверочная
	\end{listofex}
\end{exam}
%END_FOLD