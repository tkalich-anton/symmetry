%
%===============>>  Чернов Модуль 5 <<=============
%=
\setmodule{5}

%BEGIN_FOLD % ====>>_____ Занятие 1 _____<<====
\begin{class}[number=1]
	
	\begin{listofex}
		\item   Занятие 1
	\end{listofex}
\end{class}
%END_FOLD

%BEGIN_FOLD % ====>>_____ Занятие 2 _____<<====
\begin{class}[number=2]
	\begin{listofex}
		\item Занятие 2
	\end{listofex}
\end{class}
%END_FOLD

%BEGIN_FOLD % ====>>_ Домашняя работа 1 _<<====
\begin{homework}[number=1]
	\begin{listofex}
		\item Упростите выражение: \( \dfrac{3x^2+4x}{x^2-2x}-\dfrac{2x-7}{x}-\dfrac{x-8}{x-2} \)
		\item Сократите дробь: 
		\begin{tasks}(2)
			\task \(\dfrac{x^3-3x^2-4x+12}{(x-3)(x+2)}\)
			\task \( \dfrac{a^2-9}{ab+3b-2a-6} \)
		\end{tasks}
		\item Решите уравнения: 
		\begin{tasks}(2)
			\task \((x^2-4)+(x^2-6x-16)^2=0\)
			\task \( (x-1)(x^2+6x+9)=5(x+3) \)
		\end{tasks}
		\item Решите неравенства: 
		\begin{tasks}(2)
			\task \( \dfrac{8x-9}{5} \geq \dfrac{x^2}{3} \)
			\task \( x^2(-x^2-4) \leq 4(-x^2-4) \)
		\end{tasks}
		\item Постройте график функции \(y=x^2-4|x|+2x\) и определите, при каких значениях c прямая \(y=c\) имеет с графиком ровно две общие точки.
	\end{listofex}
\end{homework}
%END_FOLD

%BEGIN_FOLD % ====>>_____ Занятие 3 _____<<====
\begin{class}[number=3]
	\begin{listofex}
		\item Занятие 3
	\end{listofex}
\end{class}
%END_FOLD

%BEGIN_FOLD % ====>>_____ Занятие 4 _____<<====
\begin{class}[number=4]
	\begin{listofex}
		\item Занятие 4
	\end{listofex}
\end{class}
%END_FOLD

%BEGIN_FOLD % ====>>_ Домашняя работа 2 _<<====
\begin{homework}[number=2]
	\begin{listofex}
		\item ДЗ 2
	\end{listofex}
\end{homework}
%END_FOLD

%BEGIN_FOLD % ====>>_____ Занятие 5 _____<<====
\begin{class}[number=5]
	\begin{listofex}
		\item Занятие 5
	\end{listofex}
\end{class}
%END_FOLD

%BEGIN_FOLD % ====>>_____ Занятие 6 _____<<====
\begin{class}[number=6]
	\begin{listofex}
		\item Занятие 6
	\end{listofex}
\end{class}
%END_FOLD

%BEGIN_FOLD % ====>>_ Домашняя работа 3 _<<====
\begin{homework}[number=3]
	\begin{listofex}
		\item ДЗ 3
	\end{listofex}
\end{homework}
%END_FOLD

%BEGIN_FOLD % ====>>_____ Занятие 7 _____<<====
\begin{class}[number=7]
	\begin{listofex}
		\item Занятие 7
	\end{listofex}
\end{class}
%END_FOLD

%BEGIN_FOLD % ====>>_ Проверочная работа _<<====
\begin{exam}
	\begin{listofex}
		\item Проверочная работа
	\end{listofex}
\end{exam}
%END_FOLD