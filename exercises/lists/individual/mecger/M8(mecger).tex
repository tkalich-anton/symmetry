%
%===============>>  Мецгер Модуль 6 <<=============
%=
\setmodule{7}

%BEGIN_FOLD % ====>>_____ Занятие 1 _____<<====
\begin{class}[number=1]
	\begin{listofex}
		\item 
	\end{listofex}
\end{class}
%END_FOLD

%BEGIN_FOLD % ====>>_ Домашняя работа 1 _<<====
\begin{homework}[number=1]
	\begin{listofex}
		\item Смешали некоторое количество \( 14 \)-процентного раствора некоторого вещества с таким же количеством \( 18 \)-процентного раствора этого вещества. Сколько процентов составляет концентрация получившегося раствора?
		\item Смешали \( 4 \) литра \( 20 \)-процентного водного раствора некоторого вещества с \( 6 \) литрами \( 35 \)-процентного водного раствора этого же вещества. Сколько процентов составляет концентрация получившегося раствора?
		\item Первый сплав содержит \( 5\% \) меди, второй --- \( 11\% \) меди. Масса второго сплава больше массы первого на \( 4 \) кг. Из этих двух сплавов получили третий сплав, содержащий \( 10\% \) меди. Найдите массу третьего сплава.
		\item Имеется два сплава. Первый содержит \( 10\% \) никеля, второй --- \( 35\% \) никеля. Из этих двух сплавов получили третий сплав массой \( 250 \) кг, содержащий \( 25\% \) никеля. На сколько килограммов масса первого сплава была меньше массы второго?
		\item Смешав \( 41 \)-процентный и \( 63 \)-процентный растворы кислоты и добавив \( 10 \) кг чистой воды, получили \( 49 \)-процентный раствор кислоты. Если бы вместо \( 10 \) кг воды добавили \( 10 \) кг \( 50 \)-процентного раствора той же кислоты, то получили бы \( 54 \)-процентный раствор кислоты. Сколько килограммов \( 41 \)-процентного раствора использовали для получения смеси?
	\end{listofex}
\end{homework}
%END_FOLD

%BEGIN_FOLD % ====>>_____ Занятие 2 _____<<====
\begin{class}[number=2]
	\begin{listofex}
		\item Занятие 2
	\end{listofex}
\end{class}
%END_FOLD

%BEGIN_FOLD % ====>>_ Домашняя работа 2 _<<====
\begin{homework}[number=2]
	\begin{listofex}
		\item 
	\end{listofex}
\end{homework}
%END_FOLD

%BEGIN_FOLD % ====>>_____ Занятие 3 _____<<====
\begin{class}[number=3]
	\begin{listofex}
		\item Занятие 3
	\end{listofex}
\end{class}
%END_FOLD

%BEGIN_FOLD % ====>>_ Домашняя работа 3 _<<====
\begin{homework}[number=3]
	\begin{listofex}
		\item 
	\end{listofex}
\end{homework}
%END_FOLD

%BEGIN_FOLD % ====>>_____ Занятие 4 _____<<====
\begin{class}[number=4]
	\begin{listofex}
		\item Пусто
	\end{listofex}
\end{class}
%END_FOLD


%BEGIN_FOLD % ====>>_ Проверочная работа _<<====
\begin{exam}
	\begin{listofex}
		\item Проверочная
	\end{listofex}
\end{exam}
%END_FOLD