\setmodule{7}

%BEGIN_FOLD % ====>>_____ Занятие 1 _____<<====
\begin{class}[number=1]
	\begin{listofex}
		\item Занятие 1
	\end{listofex}
\end{class}
%END_FOLD

%BEGIN_FOLD % ====>>_____ Занятие 2 _____<<====
\begin{class}[number=2]
	\begin{listofex}
		\item Занятие 2
	\end{listofex}
\end{class}
%END_FOLD

%BEGIN_FOLD % ====>>_ Домашняя работа 1 _<<====
\begin{homework}[number=1]
	\begin{listofex}
		\item В параллелограмме \(ABCD\): \(AB=4, AD=4, \sin A = 0,25 \).  Найдите большую высоту параллелограмма.
		\item Параллелограмм и прямоугольник имеют одинаковые стороны. Найдите острый угол параллелограмма, если его площадь равна половине площади прямоугольника. Ответ дайте в градусах.
		\item Стороны параллелограмма равны \(9\) и \(15\). Высота, опущенная на первую сторону, равна \(10\). Найдите высоту, опущенную на вторую сторону параллелограмма.
		\item Площадь параллелограмма равна \(40\), две его стороны равны \(5\) и \(10\). Найдите большую высоту этого параллелограмма.
		
		\item В треугольнике \(ABC\): \(AC=BC=27, AH\) --- высота, \(\sin BAC = \dfrac{ 2 }{ 3 }\). Найдите \(BH\).
		\item В треугольнике \(ABC\): \(AC = BC = 4\sqrt{15} \), \(\cos BAC = 0,25\). Найдите высоту \(AH\).
		\item Два угла треугольника равны \(58 \degree\) и \(72\degree \). Найдите тупой угол, который образуют высоты треугольника, выходящие из вершин этих углов. Ответ дайте в градусах.
		\item В треугольнике \(ABC\): \(CH\) --- высота, \(AD\) --- биссектриса, \(O\) --- точка пересечения прямых \(CH\) и \(AD\), \(\angle BAD = 26 \degree\). Найдите угол \(AOC\). Ответ дайте в градусах.
		\item Рабочие прокладывают тоннель длиной \(87\) метров, ежедневно увеличивая норму прокладки на одно и то же число метров. Известно, что за первый день рабочие проложили \(7\) метров туннеля. Определите, сколько метров туннеля проложили рабочие в последний день, если вся работа была выполнена за \(6\) дней.
		\item Бригада маляров красит забор длиной \(150\) метров, ежедневно увеличивая норму покраски на одно и то же число метров. Известно, что за первый и последний день в сумме бригада покрасила \(75\) метров забора. Определите, сколько дней бригада маляров красила весь забор.
		\item Теплоход проходит по течению реки до пункта назначения \(255\) км и после стоянки возвращается в пункт отправления. Найдите скорость теплохода в неподвижной воде, если скорость течения равна \(1\) км/ч, стоянка длится \(2\) часа, а в пункт отправления теплоход возвращается через \(34\) часа после отплытия из него. Ответ дайте в км/ч.
		\item Моторная лодка в \(10:00\) вышла из пункта \(A\) в пункт \(B\), расположенный в \(30\) км от \(A\). Пробыв в пункте \(B\) \(2\) часа \(30\) минут, лодка отправилась назад и вернулась в пункт \(A\) в \(18:00\). Определите (в км/ч) собственную скорость лодки, если известно, что скорость течения реки \(1\) км/ч.
	\end{listofex}
\end{homework}
%END_FOLD

%BEGIN_FOLD % ====>>_____ Занятие 3 _____<<====
\begin{class}[number=3]
	\begin{listofex}
		\item Занятие 3 
	\end{listofex}
\end{class}
%END_FOLD

%BEGIN_FOLD % ====>>_____ Занятие 4 _____<<====
\begin{class}[number=4]
	\begin{listofex}
		\item Занятие 4
	\end{listofex}
\end{class}
%END_FOLD

%BEGIN_FOLD % ====>>_ Домашняя работа 2 _<<====
\begin{homework}[number=2]
	\begin{listofex}
		\item
		\begin{minipage}[t]{\bodywidth}
			У треугольника со сторонами \(9\) и \(6\) проведены высоты к этим сторонам. Высота, проведенная к первой стороне, равна \(4\). Чему равна высота, проведенная ко второй стороне?
		\end{minipage}
		\hspace{0.02\linewidth}
		\begin{minipage}[t]{\picwidth}
			\includegraphics[align=t, width=\linewidth]{../\picpath/MECGERM9H2-1}
		\end{minipage}
		\item
		\begin{minipage}[t]{\bodywidth}
			В треугольнике \(ABC\) \(CH\) --- высота, \(AD\) --- биссектриса, \(O\) --- точка пересечения прямых \(CH\) и \(AD\), угол \(BAD\) равен \(26\degree \). Найдите угол \(AOC\). Ответ дайте в градусах.
		\end{minipage}
		\hspace{0.02\linewidth}
		\begin{minipage}[t]{\picwidth}
			\includegraphics[align=t, width=\linewidth]{../\picpath/MECGERM9H2-2}
		\end{minipage}
		\item
		\begin{minipage}[t]{\bodywidth}
			В треугольнике \(ABC\) угол \(B\) равен \(45\degree \), угол \(C\) равен \(85\degree \), \(AD\) --- биссектриса, \(E\) --- такая точка на \(AB\), что \(AE  =  AC\). Найдите угол \(BDE\). Ответ дайте в градусах.
		\end{minipage}
		\hspace{0.02\linewidth}
		\begin{minipage}[t]{\picwidth}
			\includegraphics[align=t, width=\linewidth]{../\picpath/MECGERM9H2-3}
		\end{minipage}
		\item Диагональ прямоугольника вдвое больше одной из его сторон. Найдите больший из углов, который образует диагональ со сторонами прямоугольника? Ответ выразите в градусах.
		\item
		\begin{minipage}[t]{\bodywidth}
			Диагонали четырехугольника равны \(57\) и \(8\). Найдите периметр четырехугольника, вершинами которого являются середины сторон данного четырехугольника.
		\end{minipage}
		\hspace{0.02\linewidth}
		\begin{minipage}[t]{\picwidth}
			\includegraphics[align=t, width=\linewidth]{../\picpath/MECGERM9H2-4}
		\end{minipage}
		\item Площадь параллелограмма \(ABCD\) равна \(153\). Найдите площадь параллелограмма \(A'B'C'D'\), вершинами которого являются середины сторон данного параллелограмма.
		\item
		\begin{minipage}[t]{\bodywidth}
			Через концы \(A, B\) дуги окружности в \(62\degree \) проведены касательные \(AC\) и \(BC\). Найдите угол \(ACB\). Ответ дайте в градусах.
		\end{minipage}
		\hspace{0.02\linewidth}
		\begin{minipage}[t]{\picwidth}
			\includegraphics[align=t, width=\linewidth]{../\picpath/MECGERM9H2-5}
		\end{minipage}
		\item К окружности, вписанной в квадрат со стороной, равной \(a\), проведена касательная, пересекающая две его стороны. Найдите периметр отсеченного треугольника.
		\item Прямая касается окружности с центром \(O\) в точка \(A\). Точка \(C\) на этой прямой и точка \(D\) на окружности расположены по разные стороны от прямой \(OA\). Найдите угол \(CAD\), если угол \(AOD\) равен \(110 \degree\).
	\end{listofex}
\end{homework}
%END_FOLD

%BEGIN_FOLD % ====>>_____ Занятие 5 _____<<====
\begin{class}[number=5]
	\begin{listofex}
		\item Занятие 5
	\end{listofex}
\end{class}
%END_FOLD

%BEGIN_FOLD % ====>>_____ Занятие 6 _____<<====
\begin{class}[number=6]
	\begin{listofex}
		\item Занятие 6
	\end{listofex}
\end{class}
%END_FOLD

%BEGIN_FOLD % ====>>_ Домашняя работа 3 _<<====
\begin{homework}[number=3]
	\begin{listofex}
		\item
		\begin{minipage}[t]{\bodywidth}
			Найдите сторону правильного шестиугольника, описанного около окружности, радиус которой равен \(\sqrt{3}\).
		\end{minipage}
		\hspace{0.02\linewidth}
		\begin{minipage}[t]{\picwidth}
			\includegraphics[align=t, width=\linewidth]{../\picpath/MECGERM9H3-1}
		\end{minipage}
		\item
		\begin{minipage}[t]{\bodywidth}
			Сторона правильного треугольника равна \(\sqrt{3}\). Найдите радиус окружности, вписанной в этот треугольник.
		\end{minipage}
		\hspace{0.02\linewidth}
		\begin{minipage}[t]{\picwidth}
			\includegraphics[align=t, width=\linewidth]{../\picpath/MECGERM9H3-2}
		\end{minipage}
		\item Две равные окружности касаются изнутри третьей и касаются
		между собой. Соединив три центра, получим треугольник с периметром, равным \(18\). Найдите радиус большей окружности.
		\item В большей из двух концентрических окружностей (имеющих общий центр) проведена хорда, равная \(32\) и касающаяся меньшей окружности. Найдите радиус каждой из окружностей, если ширина образовавшегося кольца равна \(8\).
		\item На основании равнобедренного треугольника, равном \(8\), как на хорде построена окружность, касающаяся боковых сторон треугольника. Найдите радиус окружности, если высота, опущенная на основание треугольника, равна \(3\).
	\end{listofex}
\end{homework}
%END_FOLD

%BEGIN_FOLD % ====>>_____ Занятие 7 _____<<====
\begin{class}[number=7]
	\title{Подготовка к проверочной}
	\begin{listofex}
		\item Занятие 7
	\end{listofex}
\end{class}
%END_FOLD

%BEGIN_FOLD % ====>>_ ДЗ 4 _<<====
\begin{homework}[number=4]
	\begin{listofex}
		\item Две окружности радиусов \(4\) и \(3\) с центрами в точках \(O_1\) и \(O_2\) касаются некоторой прямой в точках \(M_1\) и \(M_2\) соответственно и лежат по разные стороны от этой прямой. Отношение отрезков \(O_1O_2\) и \(M_1M_2\). Найдите \(O_1O_2\).
		\item \(BC\) --- касательная к окружности с центром \(O\) (\(C\) --- точка касания). Отрезок \(OB\) пересекает окружность в точке \(A\). Известно, что \(CA=AB\). Найдите \( \angle COA \). Ответ дайте в градусах.
		\item В прямоугольный треугольник вписана окружность, касающаяся его гипотенузы \(AB\) в точке \(K\). Известно, что \(AK=3\) и \(BK=4\). Найдите площадь треугольника \(ABC\).
	\end{listofex}
\end{homework}
%END_FOLD