%
%===============>>  Челнокова Модуль 5 <<=============
%
\setmodule{5}

%BEGIN_FOLD % ====>>_____ Занятие 1 _____<<====
\begin{class}[number=1]
	\begin{listofex}
		\item Занятие 1
	\end{listofex}
\end{class}
%END_FOLD

%BEGIN_FOLD % ====>>_____ Занятие 2 _____<<====
\begin{class}[number=2]
	\begin{listofex}
		\item \textbf{Формулы:}
		\begin{tasks}(2)
			\task \( \sin(-x) = -\sin x \);
			\task \( \cos(-x) = \cos x \);
			\task \( \sin(180 - x) = \sin x \);
			\task \( \cos(180 - x) = -\cos x \);
			\task \( \sin(180+x) = -\sin x \);
			\task \( \cos(180+x) = -\cos x \).
		\end{tasks}
		 \item \textbf{Основное тригонометрическое тождество:}
		 \[\sin^2x+\cos^2x=1\]
		 \item \textbf{Метод приведения аргумента тригонометрических функций:}
		\begin{tasks}(1)
			\task Выносим минус за знак аргумента;
			\task "Убираем"{ }полные круги из аргумента \textit{(в будущем не обязательно);}
			\task Представляем аргумент в виде суммы/разности так, чтобы одно слагаемое было кратно \( 90 \), а другое было табличным значением (\( 30\degree;\;45\degree;\;60\degree \));
			\task Определяем четверть аргумента \textit{(меньшее слагаемое всегда принимаем за острый угол);}
			\task Определяем знак функции в этой четверти;
			\task Меняем или оставляем название тригонометрической функции (\( 0\degree;\;180\degree \) --- не меняем название функции; \( 90\degree;\;270\degree \) --- меняем название функции на противоположное).
		\end{tasks}
		\item Вычислить с помощью метода приведения:
		\[ \sin135\degree;\;\cos240\degree;\;\tg150\degree;\;\ctg220\degree;\;\sin(-220\degree);\;\tg840\degree;\;\cos(-240\degree);\;\sin315\degree \]
		\item Вычислить:
		\begin{tasks}(1)
			\task \( \dfrac{\sqrt{3}}{\sin60\degree}+\dfrac{3}{\sin30\degree} \)
			\task \( \dfrac{17\sin155\degree}{\sin25\degree} \)
			\task \( \dfrac{-2\sin105\degree}{\cos15\degree} \)
			\task \( \sin^215\degree-1+\cos^215 \)
			\task \( -\sqrt{27}\cos30\degree-\sqrt{2}\sin45\degree\ctg60\degree\tg60\degree\)
		\end{tasks}
			\end{listofex}
		\begin{definit}
			Радиан --- центральный угол, который опирается на дугу, равную радиусу данной окружности.
		\end{definit}
		\begin{definit}
			Число \( \pi \) --- отношение длины окружности к ее диаметру. Или иначе отношение половины длины окружности к ее радиусу.
		\end{definit}
		Таким образом можно сделать вывод, что в половине окружности радиус умещается \( \pi \) раз, а значит развернутый угол равен \( \pi \) радиан (т.е. \( \pi \) радиан \( = 180\degree \)).
		\begin{tasks}(1)
			\task \( 1 \) градус \( = \dfrac{\pi}{180} \) радиан;
			\task \( 1 \) радиан \( = \dfrac{180}{\pi}\) градусов (по факту всегда вместо \( \pi \) подставляем \( 180\degree \)).
		\end{tasks}
		\begin{listofex}[resume]
			\item Вычислить с помощью метода приведения:
			\[ \cos\dfrac{5\pi}{4};\;\sin\dfrac{7\pi}{3};\;\sin\dfrac{3\pi}{2};\;\sin\left( -\dfrac{5\pi}{3} \right);\;\cos\dfrac{7\pi}{6};\;\sin\dfrac{13\pi}{4};\;\sin\left( -\dfrac{7\pi}{6}  \right);\;\cos\dfrac{21\pi}{4} \]
		\item Вычислить:
		\begin{tasks}(2)
			\task \( \dfrac{5\cos29\degree}{\sin61\degree} \)
			\task \( -4\sqrt{3}\cos(-750\degree) \)
			\task \( \dfrac{4\cos146\degree}{\cos34\degree} \)
			\task \( 7\tg13\degree\cdot\tg77\degree \)
			\task \( \dfrac{12}{\sin^227\degree+\cos^2207\degree} \)
			\task \( \dfrac{5\sin98\degree}{\sin49\degree\cdot\sin41\degree} \)
			\task \( -50\tg9\degree\cdot\tg81\degree+31 \)
		\end{tasks}
%		\item Найти:
%		\begin{tasks}(1)
%			\task \( 5\sin\alpha \), \quad если \( \cos\alpha=\dfrac{2\sqrt{6}}{5} \) и \( \alpha\in\left( \dfrac{3\pi}{2}; 2\pi \right) \);
%			\task \( 3\cos\alpha \), \quad если \( \sin\alpha=-\dfrac{2\sqrt{2}}{3} \) и \( \alpha\in\left( \dfrac{3\pi}{2}; 2\pi \right) \);
%			\task \( 24\cos\alpha \), \quad если \( \sin\alpha=-0,2 \);
%			\task \( \sin\left( \dfrac{7\pi}{2}-\alpha \right) \), \quad если \( \sin\alpha=0,8 \) и \( \alpha\in\left( \dfrac{\pi}{2}; \pi \right) \);			
%			\task \( \dfrac{3\cos\alpha-4\sin\alpha}{2\sin\alpa-5\cos\alpha} \), \quad если \( \tg\alpha=3 \);
%			\task \( \tg\alpha \), \quad если \( \dfrac{7\sin\alpha+13\cos\alpha}{5\sin\alpha-17\cos\alpha}=3 \).
%		\end{tasks}
		\end{listofex}
\end{class}
%END_FOLD

%BEGIN_FOLD % ====>>_ Домашняя работа 1 _<<====
\begin{homework}[number=1]
	\begin{listofex}
		\item Домашняя работа 1
	\end{listofex}
\end{homework}
%END_FOLD

%BEGIN_FOLD % ====>>_____ Занятие 3 _____<<====
\begin{class}[number=3]
	\begin{listofex}
		\item Найти:
		\begin{tasks}(1)
			\task \( 5\sin\alpha \), \quad если \( \cos\alpha=\dfrac{2\sqrt{6}}{5} \) и \( \alpha\in\left( \dfrac{3\pi}{2}; 2\pi \right) \);
			\task \( 3\cos\alpha \), \quad если \( \sin\alpha=-\dfrac{2\sqrt{2}}{3} \) и \( \alpha\in\left( \dfrac{3\pi}{2}; 2\pi \right) \);
			\task \( 24\cos\alpha \), \quad если \( \sin\alpha=-0,2 \);
			\task \( \sin\left( \dfrac{7\pi}{2}-\alpha \right) \), \quad если \( \sin\alpha=0,8 \) и \( \alpha\in\left( \dfrac{\pi}{2}; \pi \right) \);			
			\task \( \dfrac{3\cos\alpha-4\sin\alpha}{2\sin\alpha-5\cos\alpha} \), \quad если \( \tg\alpha=3 \);
			\task \( \tg\alpha \), \quad если \( \dfrac{7\sin\alpha+13\cos\alpha}{5\sin\alpha-17\cos\alpha}=3 \).
		\end{tasks}
		\item \textbf{Формулы двойного угла:}
		\begin{tasks}(2)
			\task \( \sin2x=2\sin x\cos x \);
			\task \( \cos2x=\cos^2x-\sin^2x \);
			\task \( \tg2x=\dfrac{2\tg x}{1-\tg^2x} \);
			\task \( \ctg2x=\dfrac{\ctg^2x-1}{2\ctg x} \)
		\end{tasks}
		\item Вывести другие формулы нахождения косинуса двойного угла.
		\item Вычислить:
		\begin{tasks}(2)
			\task \( \dfrac{12\sin11\degree\cdot\cos11\degree}{\sin22\degree} \)
			\task \( \dfrac{24(\sin^217-\cos^217)}{\cos34\degree} \)
			\task \( \dfrac{-6\sin142\degree}{\sin71\degree\cdot\sin19\degree} \)
			\task \( \dfrac{36\sin102\degree\cdot\cos102\degree}{\sin204\degree} \)
			\task \( 7\sqrt{2}\cos\dfrac{15\pi}{8}\sin\dfrac{15\pi}{8} \)
			\task \( 4\sqrt{3}\cos^2\dfrac{7\pi}{12}-2\sqrt{3} \)
			\task \( 2\sqrt{3}-4\sqrt{3}\sin^2\dfrac{7\pi}{12} \)
			\task \( \sqrt{72}-\sqrt{288}\sin^2\dfrac{21\pi}{8} \)
		\end{tasks}
	\item Вычислить:
	\begin{tasks}(1)
		\task \( \ctg\left( \dfrac{3\pi}{2}+x \right)\ctg(\pi-x)-\ctg\left( \dfrac{\pi}{2}+x \right)\tg(2\pi+x) \)
		\task \( \cos\left( \dfrac{3\pi}{2}+x \right)\sin x + \sin^2(3\pi+x)+\tg(5\pi+x)\ctg x \)
		\task \( \dfrac{\sin x}{1+\cos x}+\ctg x \)
	\end{tasks}
	\end{listofex}
\end{class}
%END_FOLD

%BEGIN_FOLD % ====>>_____ Занятие 4 _____<<====
\begin{class}[number=4]
	\begin{listofex}
		\item Занятие 4
	\end{listofex}
\end{class}
%END_FOLD

%BEGIN_FOLD % ====>>_ Домашняя работа 2 _<<====
\begin{homework}[number=2]
	\begin{listofex}
		\item Вычислить с помощью метода приведения: \[ \sin150\degree;\;\cos135\degree;\;\sin225\degree;\;\cos(-120\degree);\;\tg(-150\degree);\;\sin(-225\degree);\;\cos300\degree;\;\sin(-315\degree) \]
		\item Найти \( 26\cos\left( \dfrac{3\pi}{2}+\alpha \right) \), \quad если \( \cos\alpha=\dfrac{12}{13} \) и \( \alpha\in\left( \dfrac{3\pi}{2}; 2\pi \right) \)
		\item Вычислить:
		\begin{tasks}(1)
			\task \( \cos(-300\degree)\sin(-120\degree)\tg(-150\degree) \)
			\task \( \sin\dfrac{5\pi}{4}\cos\dfrac{4\pi}{3}\tg\dfrac{2\pi}{3}\ctg\dfrac{3\pi}{4} \)
		\end{tasks}
		\item Вычислить:
		\begin{tasks}(2)
			\task \( \dfrac{4\sin16\degree\cdot\cos16\degree}{\sin32\degree} \)
			\task \( \dfrac{7(\sin^274\degree-\cos^274\degree)}{\cos148\degree} \)
			\task \( 5\sin\dfrac{11\pi}{12}\cdot\cos\dfrac{11\pi}{12} \)
			\task \( \sqrt{32}\cos^2\dfrac{13\pi}{8}-\sqrt{8} \)
		\end{tasks}
	\end{listofex}
\end{homework}
%END_FOLD

%BEGIN_FOLD % ====>>_____ Занятие 5 _____<<====
\begin{class}[number=5]
	\begin{listofex}
		\item Занятие 5
	\end{listofex}
	\end{class}
	%END_FOLD
	
	%BEGIN_FOLD % ====>>_____ Занятие 6 _____<<====
	\begin{class}[number=6]
		\begin{listofex}
			\item Занятие 6
		\end{listofex}
	\end{class}
	%END_FOLD
	
	%BEGIN_FOLD % ====>>_ Домашняя работа 3 _<<====
	\begin{homework}[number=3]
		\begin{listofex}
			\item Вычислить:
			\begin{tasks}(1)
				\task \( \sin240\degree\sin150\degree\sin(-90)\degree\tg30\degree \)
				\task \( \cos\left( -\dfrac{5\pi}{3} \right)\sin\left( -\dfrac{5\pi}{2} \right)\sin\dfrac{3\pi}{2} \)
			\end{tasks}
		\end{listofex}
	\end{homework}
	%END_FOLD
	
	%BEGIN_FOLD % ====>>_____ Занятие 7 _____<<====
	\begin{class}[number=7]
		\title{Подготовка к проверочной}
		\begin{listofex}
			\item Занятие 7
		\end{listofex}
	\end{class}
	%END_FOLD
	
	%BEGIN_FOLD % ====>>_ Проверочная работа _<<====
	\begin{exam}
		\begin{listofex}
			\item Проверочная
		\end{listofex}
	\end{exam}
	%END_FOLD