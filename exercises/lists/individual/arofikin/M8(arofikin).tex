%
%===============>>  Киселев Модуль 7 <<=============
%
\setmodule{8}

%BEGIN_FOLD % ====>>_____ Занятие 1 _____<<====
\begin{class}[number=1]
	\begin{listofex}
		\item Занятие 1
	\end{listofex}
\end{class}
%END_FOLD

%BEGIN_FOLD % ====>>_____ Занятие 2 _____<<====
\begin{class}[number=2]
	\begin{listofex}
		\item Занятие 2
	\end{listofex}
\end{class}
%END_FOLD

%BEGIN_FOLD % ====>>_ Домашняя работа 1 _<<====
\begin{homework}[number=1]
	\begin{listofex}
		\item Домашняя работа 1
	\end{listofex}
\end{homework}
%END_FOLD

%BEGIN_FOLD % ====>>_____ Занятие 3 _____<<====
\begin{class}[number=3]
	\begin{listofex}
		\item Занятие 3
	\end{listofex}
\end{class}
%END_FOLD

%BEGIN_FOLD % ====>>_____ Занятие 4 _____<<====
\begin{class}[number=4]
	\begin{listofex}
		\item Занятие 4
	\end{listofex}
\end{class}
%END_FOLD

%BEGIN_FOLD % ====>>_ Домашняя работа 2 _<<====
\begin{homework}[number=2]
	\begin{listofex}
		\item Домашняя работа 2
	\end{listofex}
\end{homework}
%END_FOLD

%BEGIN_FOLD % ====>>_____ Занятие 5 _____<<====
\begin{class}[number=5]
	\begin{listofex}
		\item Занятие 5
	\end{listofex}
\end{class}
%END_FOLD

%BEGIN_FOLD % ====>>_____ Занятие 6 _____<<====
\begin{class}[number=6]
	\begin{listofex}
			\item  Вычислить:
		\begin{tasks}(2)
			\task \( \dfrac{16}{25}-\dfrac{1}{5}\cdot\dfrac{4}{5} \)
			\task  \(\dfrac{2}{16}+\left( \mfrac{5}{1}{4}-\mfrac{3}{3}{4}+\dfrac{1}{4} \right)\cdot\dfrac{3}{4}\)
			\task \( \left( \mfrac{63}{2}{8}+\mfrac{3}{1}{8} \right)-\left( 13-\mfrac{10}{5}{8} \right) \)
			\task \( \mfrac{2}{1}{2}\cdot\mfrac{3}{1}{2}-\mfrac{8}{1}{4} \)
			\task!\( \left( \dfrac{2}{3}\cdot\mfrac{4}{1}{3}\cdot\dfrac{1}{2}-1 \right):1-\dfrac{1}{3}\cdot\mfrac{1}{1}{2}\cdot\dfrac{2}{3} \)
			\task \( \mfrac{2}{7}{12}+\mfrac{4}{5}{12}-\left( \mfrac{20}{3}{12}-\mfrac{19}{5}{12} \right) \)
		\end{tasks}
		
	\end{listofex}
\end{class}
%END_FOLD

%BEGIN_FOLD % ====>>_ Домашняя работа 3 _<<====
\begin{homework}[number=3]
	\begin{listofex}
		\item Вычислите:
		\begin{tasks}(2)
			\task \( \mfrac{23}{3}{7}+\dfrac{2}{7} \)
			\task \( \dfrac{7}{10} - \dfrac{2}{5}\cdot\dfrac{1}{2}\)
			\task \( \mfrac{4}{2}{9}-\mfrac{2}{6}{9} \)
			\task \( \dfrac{2}{3}\cdot\dfrac{1}{3}+\dfrac{1}{9} \)
			\task \( \mfrac{4}{7}{9}\cdot\mfrac{1}{2}{3}-\dfrac{3}{27} \)
			\task \( \dfrac{42}{45}-\dfrac{33}{45} \)
			\task \( \mfrac{1}{1}{11}\cdot2+3\)
			\task \( 4+\dfrac{3}{8}-\mfrac{2}{2}{8} \)
		\end{tasks}
		\item Длина дороги \( 84 \) км. За первый день бригада рабочих отремонтировала \(\dfrac{ 3}{14} \) дороги, а за второй день --- \( \dfrac{5}{14} \) дороги. Сколько километров осталось отремонтировать? 
	\end{listofex}
\end{homework}
%END_FOLD

%BEGIN_FOLD % ====>>_____ Занятие 7 _____<<====
\begin{class}[number=7]
	\begin{listofex}
		\item Вычислить:
		\begin{tasks}
			\task \( 4,735\cdot0,5+14,95\cdot1,3+2,121\cdot0,7 \)
			\task \( (0,578+2,172)\cdot(1,823+0,117)-1,711\cdot(0,418+1,382) \)
			\task \( 3,006-0,0417\cdot3-0,875\cdot0,4 \)
		\end{tasks}
		\item Вычислить:
		\begin{tasks}(4)
			\task \( 1,8\cdot5,5 \)
			\task \( 41,7\cdot6,05 \)
			\task \( 8,42\cdot9,9 \)
			\task \( 6,703\cdot2,45 \)
			\task \( 55,3\cdot3,81 \)
			\task \( 6,321\cdot7,8 \)
			\task \( 32,4\cdot103,5 \)
			\task \( 8,05\cdot8,05 \)
		\end{tasks}
		\item Вычислите:
		\begin{tasks}(4)
			\task \( 0,5\cdot10 \)
			\task \( 0,15\cdot10000 \)
			\task \( 50,265\cdot10 \)
			\task \( 21,598\cdot1000 \)
			\task \( 9,56\cdot50 \)
			\task \( 8,532\cdot1200 \)
			\task \( 9,123\cdot15000 \)
			\task \( 24,1\cdot130000 \)
		\end{tasks}

	\end{listofex}
\end{class}
%END_FOLD

%BEGIN_FOLD % ====>>_ Проверочная работа _<<====
\begin{class}[number=8]
	\begin{listofex}
		\item Вычислите: 
		\begin{tasks}(1)
			\task \( \left( 13-\mfrac{8}{5}{12} \right)+\left( \mfrac{17}{1}{12}-\mfrac{16}{4}{12} \right) \)
			\task \( \left( \mfrac{63}{2}{9}+\mfrac{3}{8}{9} \right)-\left( 13-\mfrac{10}{5}{9} \right) \)
			\task \( \mfrac{2}{1}{3}\cdot\mfrac{4}{1}{3}-\mfrac{8}{4}{3}:3 \)
			\task \( \left( \dfrac{5}{4}\cdot\mfrac{2}{1}{6}\cdot\dfrac{5}{2}-1 \right):1-\dfrac{1}{2}\cdot\mfrac{1}{3}{8}\cdot\dfrac{2}{3} \)
		\end{tasks}
		\item Выполните действия:
		\begin{tasks}(1)
			\task \( 11,47+(3,89-2,11)-4,416+3,711 \)
			\task \( 3,16+(7,84-4,181)-3,11+14,816 \)
			\task \( 1,49+(6,13-4,12)-0,5+7,289 \)
		\end{tasks}
		\item Вычислите:
		\begin{tasks}(3)
			\task \( 23,4\cdot100 \)
			\task\( 4,5 \cdot100 \)
			\task \( 2,345\cdot100 \)
			\task\( 8,09\cdot0,001 \) 
			\task\( 25,043 \cdot0,001  \)
			\task\( 29,01\cdot0,001   \)
		\end{tasks}
		\item Вычислите:
		\begin{tasks}(4)
			\task \( 25,36\cdot1,2 \)
			\task \( 3,5\cdot3,8 \)
			\task \( 63,25\cdot0,3 \)
			\task \( 5,369\cdot0,05 \)
			\task \( 12,03\cdot4,6 \)
			\task \( 23,71\cdot1,8 \)
			\task \( 0,24\cdot5,5 \)
			\task \( 0,7\cdot0,14 \)
		\end{tasks}
		\item Ширина прямоугольника \( 5,15 \) см, а длина в \( 2 \) раза больше. Найдите площадь и периметр данного прямоугольника?
		\item Площадь прямоугольника \( 42 \) см\( ^{2} \), ширина его равна \( 6 \) см. Чему равна длина прямоугольника?
		\item Бассейн имеет форму прямоугольника со сторонами \( 5,32 \) м и \( 4,74 \) м. Чему равна его площадь?
		\item Все стороны семиугольника имеют длину \( 9,47 \) см. Найдите периметр фигуры.
	\end{listofex}
\end{class}
%END_FOLD

%BEGIN_FOLD % ====>>_ Домашняя работа 4 _<<====
\begin{homework}[number=4]
	\begin{listofex}
		\item Вычислите:
		\begin{tasks}(2)
			\task \( 0,763-0,321+5,8 \)
			\task \( 10,5-6,957+11,87 \)
		\end{tasks}
		\item Вычислите 
		\begin{tasks}(4)
			\task \( 1,6\cdot10 \)
			\task \( 5,8\cdot100 \)
			\task \( 15,3\cdot200 \)
			\task \( 87,701\cdot 140 \)
		\end{tasks}
		\item Вычислите 
		\begin{tasks}(3)
			\task \( 440,8\cdot0,001 \)
			\task \( 79,22\cdot31,59 \)
			\task \( 84,102\cdot13,4 \)
		\end{tasks}
		\item Бассейн имеет форму прямоугольника со сторонами \( 5,32 \) м и \( 4,74 \) м. Чему равна его площадь?
		\item Все стороны семиугольника имеют длину \( 9,47 \) см. Найдите периметр фигуры.
	\end{listofex}
\end{homework}
%END_FOLD