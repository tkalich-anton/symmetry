%
%===============>>  Киселев Модуль 7 <<=============
%
\setmodule{8}

%BEGIN_FOLD % ====>>_____ Занятие 1 _____<<====
\begin{class}[number=1]
	\begin{listofex}
		\item Занятие 1
	\end{listofex}
\end{class}
%END_FOLD

%BEGIN_FOLD % ====>>_____ Занятие 2 _____<<====
\begin{class}[number=2]
	\begin{listofex}
		\item В магазин привезли яблоки и груши, причём яблок было \( \dfrac{7}{20} \) тонны, а груш --- на \( \dfrac{3}{20}\) тонны меньше, чем яблок. Сколько всего тонн фруктов привезли в магазин?
		\item Пятачок принёс для Винни два бочонка с мёдом. Масса одного бочонка \( \mfrac{5}{4}{7} \) кг, а масса второго \( \mfrac{3}{2}{7} \) кг. Сколько мёда было в двух бочонках?
		\item Периметр треугольника \( MNK \) равен \( 30 \) см. Найдите длину стороны \( MK \), если длина \( MN \) равна \( \mfrac{8}{5}{7} \) см, а \( NK \) на \( \dfrac{4}{7} \) см длиннее \( MN \).
		\item В первый день бригада рабочих заасфальтировала \( \mfrac{20}{5}{18} \) м дороги, а во второй день --- \( \mfrac{15}{7}{18} \) м дороги. Сколько метров дороги заасфальтировала бригада за два дня?
		\item В одном ящике \( \mfrac{15}{3}{8} \) кг слив, а во втором ящике на \( \mfrac{2}{7}{8} \) кг больше. Сколько килограммов слив в двух ящиках?
		\item Фермер завез на рынок \( \mfrac{42}{7}{17} \)  кг зелени --- петрушки, укропа и сельдерея. Петрушки и укропа вместе было \( \mfrac{29}{4}{17} \) кг, петрушки и сельдерея \( \mfrac{28}{1}{17} \) кг. Сколько килограмм каждого вида зелени привез фермер на рынок?
	\end{listofex}
\end{class}
%END_FOLD

%BEGIN_FOLD % ====>>_ Домашняя работа 1 _<<====
\begin{homework}[number=1]
	\begin{listofex}
		\item Домашняя работа 1
	\end{listofex}
\end{homework}
%END_FOLD

%BEGIN_FOLD % ====>>_____ Занятие 3 _____<<====
\begin{class}[number=3]
	\begin{listofex}
		\item Занятие 3
	\end{listofex}
\end{class}
%END_FOLD

%BEGIN_FOLD % ====>>_____ Занятие 4 _____<<====
\begin{class}[number=4]
	\begin{listofex}
		\item Занятие 4
	\end{listofex}
\end{class}
%END_FOLD

%BEGIN_FOLD % ====>>_ Домашняя работа 2 _<<====
\begin{homework}[number=2]
	\begin{listofex}
		\item Домашняя работа 2
	\end{listofex}
\end{homework}
%END_FOLD

%BEGIN_FOLD % ====>>_____ Занятие 5 _____<<====
\begin{class}[number=5]
	\begin{listofex}
		\item Занятие 5
	\end{listofex}
\end{class}
%END_FOLD

%BEGIN_FOLD % ====>>_____ Занятие 6 _____<<====
\begin{class}[number=6]
	\begin{listofex}
		\item Занятие 6
	\end{listofex}
\end{class}
%END_FOLD

%BEGIN_FOLD % ====>>_ Домашняя работа 3 _<<====
\begin{homework}[number=3]
	\begin{listofex}
		\item Домашняя работа 3
	\end{listofex}
\end{homework}
%END_FOLD

%BEGIN_FOLD % ====>>_____ Занятие 7 _____<<====
\begin{class}[number=7]
	\title{Подготовка к проверочной}
	\begin{listofex}
		\item Занятие 7
	\end{listofex}
\end{class}
%END_FOLD

%BEGIN_FOLD % ====>>_ Проверочная работа _<<====
\begin{exam}
	\begin{listofex}
		\item Проверочная
	\end{listofex}
\end{exam}
%END_FOLD