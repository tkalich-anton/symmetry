%Группа =10-2 Модуль 2
\title{Занятие №1}
\begin{listofex}
	\item \exercise{2350}
	\item \exercise{2095}
	\item \exercise{2390}
	\item \exercise{2381}
	\item \exercise{2380}
	\item \exercise{2393}
	\item \exercise{2559}
	\item \exercise{2040}
	\item \exercise{2424}
	\item \exercise{2369}
\end{listofex}
\newpage
\title{Занятие №2}
\begin{listofex}
	\item \exercise{1115}
	\item Докажите, что если медиана равна половине стороны, к которой она проведена, то треугольник прямоугольный.
	\item \exercise{2415}
	\item Докажите, что если треугольник вписан в окружность и одна из его сторон является диаметром этой окружности, то такой треугольник является прямоугольным.
	\item Докажите обратное, что если треугольник прямоугольный и вписан в окружность, то гипотенуза будет являться диаметром окружности.
	\item \exercise{2455}
	\item \exercise{2418}
	\item \exercise{2422}
	%\item \exercise{2365}
	%\item \exercise{2562}
	%\item \exercise{2385}
	%\item \exercise{2685}
	%\item \exercise{2391}
	%\item \exercise{2396}
\end{listofex}
\newpage
\title{Домашняя работа №1}
\begin{listofex}
	\item \exercise{1484}
	\item \exercise{1114}
	\item \exercise{4140}
	\item \exercise{2408}
	\item \exercise{2389}
	\item \exercise{2401}

\end{listofex}
\newpage
\title{Занятие №3}
\begin{listofex}
	\item Докажите следующие свойства окружности:
	\begin{enumcols}[itemcolumns=1]
		\item диаметр, перпендикулярный хорде, делит ее пополам;
		\item диаметр, проходящий через середину хорды, не являющейся диаметром, перпендикулярен этой хорде;
		\item хорды, удаленные от центра окружности на равные расстояния, равны.
	\end{enumcols}
	\item \exercise{2437}
	\item \exercise{2439}
	\item \exercise{2442}
	\item \exercise{2444}
	\item \exercise{2445}
	\item Докажите, что если треугольник вписан в окружность и одна из его сторон является диаметром этой окружности, то такой треугольник является прямоугольным.
	\item Центр окружности, описанной около треугольника, симметричен центру окружности, вписанной в этот треугольник, относительно одной из сторон. Найдите углы треугольника.
	\item Через точку \( A \) проведена прямая, пересекающая
	окружность с диаметром \( AB \) в точке \( K \), отличной от \( A \), а
	окружность с центром \( B \) --- в точках \( M \) и \( N \). Докажите, что \( MK = KN \).
\end{listofex}
\newpage
\title{Занятие №4}
\begin{listofex}
	\item Внутренние углы треугольника \( ABC \) относятся как \( 10:5:3 \). Найдите внутренние и внешние углы треугольника \( ABC \) и вычислите разницу самого наибольшего и наименьшего внешних углов. \answer{ внутренние:\( 100;\;50;\;30 \), внешние: \( 80;\;130;\;100; \), разница: \( 50 \) }
	\item В треугольнике \( ABC \) углы \( B \) и \( C \) равны \( 30 \) и \( 40 \) соответственно. Сторону \( AB \) продлили за вершину \( A \) и из это вершины провели высоту и биссектрису внешнего угла. Найдите угол между высотой и биссектрисой. \answer{ \( 85 \) }
	\item Две параллельные прямые пересечены третьей. Найдите угол между биссектрисами внутренних односторонних углов.
	\item \exercise{2438}
	\item \exercise{2441}
	\item \exercise{2354}
	\item \exercise{2514}
	\item Решить уравнение:
	\begin{enumcols}[itemcolumns=2]
		\item \exercise{1015}
		\item \exercise{1036}
	\end{enumcols}
\end{listofex}
\newpage
\title{Домашняя работа №2}
\begin{listofex}
	\item \exercise{1522}
	\item \exercise{1317}
	\item \exercise{2436}
	\item \exercise{2440}
	\item \exercise{2454}
	\item \exercise{2457}
	\item \exercise{2459}
\end{listofex}
\newpage
\title{Занятие №5}
\begin{listofex}
	\item \exercise{2472}
	\item \exercise{2473}
	\item \exercise{2474}
	\item \exercise{2483}
	\item \exercise{2493}
	\item \exercise{2508}
	\item \exercise{1608}
	\item \exercise{3664}
\end{listofex}
\newpage
\title{Занятие №6}
\begin{listofex}
	\item \exercise{2477}
	\item \exercise{2481}
	\item \exercise{2484}
	\item \exercise{2486}
	\item \exercise{2479}
	\item \exercise{2500}
	\item \exercise{2502}
	\item \exercise{2506}
	\item \exercise{1339}
\end{listofex}
\newpage
\title{Домашняя работа №3}
\begin{listofex}
	\item \exercise{2475}
	\item \exercise{2476}
	\item \exercise{2478}
	\item \exercise{2485}
	\item \exercise{2502}
	\item \exercise{2480}
	\item \exercise{2501}
\end{listofex}
\newpage
\title{Подготовка к проверочной работе}
\begin{listofex}
	\item Чему равен угол между биссектрисами двух смежных углов?
	\item Чему равен угол между биссектрисами двух внутренних односторонних углов при параллельных прямых? Докажите это.
	\item Сформулируйте и докажите теорему о внешнем угле треугольника.
	\item Чему равна сумма всех внешних углов треугольника?
	\item Докажите, что биссектриса внешнего угла при вершине равнобедренного треугольника, параллельна основанию.
	\item Докажите, что если в треугольнике один угол равен сумме двух других, то такое треугольник прямоугольный.
	\item Докажите, что если медиана равна половине стороны, к которой она проведена, то такой треугольник прямоугольный.
	\item Докажите, что если треугольник вписан в окружность и одна из его сторон является диаметром этой окружности, то такой треугольник прямоугольный.
	\item Сформулируйте теорему об угле в \( 30\degree \) в прямоугольном треугольнике. Сформулируйте обратную теорему.
	\item Сформулируйте теорему о диаметре, перпендикулярном хорде.
	\item Сформулируйте теорему о диаметре, проходящем через середину хорды.
	\item Где лежит центр вписанной в треугольник окружности? Где лежит центр описанной окружности?
	\item Сформулируйте теорему о двух касательных, проведенных из одной точки к окружности.
	\item \exercise{2472}
	\item Угол между биссектрисами двух углов треугольника равен \( 120\degree \). Чему равен третий угол треугольника?
	\item Угол треугольника равен \( 50\degree \). Найдите угол между высотами, проведенными из двух других углов.
	\item В треугольнике \( ABC \) угол \( \angle B=60\degree \). Найдите угол между биссектрисами двух других внешних углов.
	\item \exercise{2456}
	\item \exercise{2459}
	\item \exercise{2475}
	\item \exercise{2478}
	\item \exercise{2485}
	\item \exercise{2480}
	\item \exercise{2464}
	\item В треугольнике \( ABC \) медиана \( AM \) продолжена за точку \( M \) на расстояние, равное \( AM \). Найдите расстояние от полученной точки до вершин \( B  \) и \( C\), если \( AB = 5\), \( AC = 12\).
	\item \exercise{2441}
	\item \exercise{2514}
	\item \exercise{2470}
\end{listofex}
\newpage
\title{Проверочная работа}
\title{Вариант 1}
\begin{listofex}
	\item
	\begin{enumcols}[itemcolumns=1]
		\item Чему равен угол между биссектрисами двух смежных углов?
		\item Сформулируйте и докажите теорему о внешнем угле треугольника.
		\item Докажите, что биссектриса внешнего угла при вершине равнобедренного треугольника, параллельна основанию.
		\item Докажите, что если медиана равна половине стороны, к которой она проведена, то такой треугольник прямоугольный.
		\item Докажите, что если треугольник вписан в окружность и одна из его сторон является диаметром этой окружности, то такой треугольник прямоугольный.
		\item Сформулируйте теорему об угле в \( 30\degree \) в прямоугольном треугольнике. Сформулируйте обратную теорему.
		\item Сформулируйте теорему о диаметре, проходящем через середину хорды.
		\item Где лежит центр вписанной в треугольник окружности?
	\end{enumcols}
	\item В треугольнике \( ABC \) обе стороны \( AB \) и \( BC \) равны \( 15 \). Чему равна сторона \( AC \), если \( \angle BAC = 60 \degree \)?
	\item Угол между биссектрисами двух углов треугольника равен \( 100\degree \). Чему равен третий угол треугольника?
	\item \exercise{2456}
	\item \exercise{2478}
	\item \exercise{2485}
	\item В треугольнике \( ABC \) медиана \( AM \) продолжена за точку \( M \) на расстояние, равное \( AM \). Найдите расстояние от полученной точки до вершин \( B  \) и \( C\), если \( AB = 5\), \( AC = 12\).
	\item \exercise{2441}
	\item \exercise{2514}
	\item \exercise{1115}
	\item \exercise{3664}
\end{listofex}
\newpage
\title{Проверочная работа}
\title{Вариант 2}
\begin{listofex}
	\item
	\begin{enumcols}[itemcolumns=1]
		\item Чему равен угол между биссектрисами двух внутренних односторонних углов при параллельных прямых?
		\item Сформулируйте и докажите теорему о внешнем угле треугольника.
		\item Докажите, что если в треугольнике один угол равен сумме двух других, то такое треугольник прямоугольный.
		\item Докажите, что если треугольник вписан в окружность и одна из его сторон является диаметром этой окружности, то такой треугольник прямоугольный.
		\item Сформулируйте теорему об угле в \( 30\degree \) в прямоугольном треугольнике. Сформулируйте обратную теорему.
		\item Сформулируйте теорему о диаметре, перпендикулярном хорде.
		\item Сформулируйте теорему о двух касательных, проведенных из одной точки к окружности.
	\end{enumcols}
	\item В треугольнике \( ABC \) обе стороны \( AB \) и \( BC \) равны \( 30 \). Чему равна сторона \( AC \), если \( \angle BAC = 60 \degree \)?
	\item Угол треугольника равен \( 80\degree \). Найдите угол между высотами, проведенными из двух других углов.
	\item \exercise{2456}
	\item \exercise{2478}
	\item \exercise{2485}
	\item \exercise{2441}
	\item В треугольнике \( ABC \) медиана \( AM \) продолжена за точку \( M \) на расстояние, равное \( AM \). Найдите расстояние от полученной точки до вершин \( B  \) и \( C\), если \( AB = 6\), \( AC = 17\).
	\item \exercise{2514}
	\item \exercise{1115}
	\item \exercise{3664}
\end{listofex}
\newpage
\title{Консультация}
\textbf{Математическая индукция} --- метод математического доказательства, который применяется, чтобы доказать истинность некого утверждения для всех натуральных чисел. Некоторое утверждение будет справедливым для натурального значения \( n \) тогда, и только тогда, когда:
\begin{enumcols}[itemcolumns=1]
	\item Оно будет верно при \( n=1 \) \textbf{(база индукции)}
	\item Предположительно справедливо для произвольного натурального \( n=k \) \textbf{(предположение индукции)}
	\item И окажется верным при \( n=k+1 \) \textbf{(шаг индукции)}
\end{enumcols}
\begin{listofex}
	\item Докажите методом математической индукции:
	\begin{enumcols}[itemcolumns=2]
		\item \( 1+2+3+\dots+n=\dfrac{n(n+1)}{2} \)
		\item \( 1+3+5+\dots+(2n-1)=n^2 \)
		\item \( 2+4+6+\dots+2n=n(n+1) \)
		\item \( 1^2+2^2+3^2+\dots+n^2=\dfrac{n(n+1)(2n+1)}{6} \)
	\end{enumcols}
	\item В треугольнике \( ABC \) сторона \( AB=12 \), \( BC=4 \) и \( \angle CBA=45\degree \). Найдите площадь треугольника.
	\item Радиус описанной вокруг равностороннего треугольника \( ABC \) окружности равен \( 9 \). Найдите площадь сторону и площадь треугольника \( ABC \).
\end{listofex}
\newpage
\title{Консультация}
\begin{listofex}
	\item Решить уравнение:
	\begin{enumcols}[itemcolumns=1]
		\item \exercise{997}
		\item \exercise{1003}
		\item \exercise{1014}
		\item \exercise{1037}
		\item \exercise{1003}
		\item \exercise{3746}
		\item \exercise{3762}
	\end{enumcols}
\end{listofex}
\newpage
\title{Консультация}
\begin{listofex}
	\item Найдите область определения функции:
	\begin{enumcols}[itemcolumns=2]
		\item \( y=\dfrac{x-7}{x^2-6x+8} \)
		\item \( y=\sqrt{x^2+6x-16} \)
		\item \( y=\sqrt{\dfrac{x+11}{x^2+14x+33}} \)
		\item \( y=\dfrac{1-\sqrt{-x^2-7x+8}}{1+\sqrt{x+9}} \)
	\end{enumcols}
	\item Найдите область значений функции:
	\begin{enumcols}[itemcolumns=2]
		\item \( y=2x-1 \)
		\item \( y=2x^2-3 \)
		\item \( y=-3x^2-12x+1,\;x\in[-6;1) \)
		\item \( y=1-\dfrac{3}{x} \)
		\item \( y=\dfrac{x-1}{x+1} \)
	\end{enumcols}
	\item Найдите промежутки монотонности:
	\begin{enumcols}[itemcolumns=2]
		\item \( y=x^2-9x+20 \)
		\item \( y=(x+3)^2-12 \)
	\end{enumcols}
	\item Пусть функция \( y=f(x) \) определена и возрастает на \( R \). Решите уравнение: \[ f\left( \dfrac{24}{x} \right)=f\left( 1+\dfrac{17-x}{x-1} \right) \]
	\item Найдите область определения функции и исследуйте ее на четность и нечетность: \[ y=\dfrac{x^2}{1+x}+\dfrac{x^2}{1-x} \]
	\item Являются ли функции \( y=f(x) \) и \( y=g(x) \) взаимно обратными, если \( f(x)=3x+5 \) и \( g(x)=\dfrac{1}{3}x-\dfrac{5}{3} \)?
	\item Найдите функцию, обратную \( y=\dfrac{x+7}{2x-5} \).
\end{listofex}
\newpage
\title{Консультация}
\begin{listofex}
		\item Решить уравнение:
	\begin{enumcols}[itemcolumns=2]
		\item \exercise{543}
		\item \exercise{503}
		\item \( (x^2+6x)^2+2(x+3)^2=81 \) \answer{\( -7;\;-3;\;1 \)}
		\item \exercise{975}
	\end{enumcols}
	\item Сколько пятизначных чисел можно получить из цифр \( 1;\;3;\;5;\;7;\;9 \)?
	\item Сколько трехзначных чисел можно получить из цифр \( 1;\;3;\;5;\;7;\;9 \)?
	\item Сколько есть способов поставить в ряд (последовательность не важна) 3 человек из 8?
	\item \exercise{1453}
	\item Решить неравенство:
	\begin{enumcols}[itemcolumns=2]
		\item \( (x-1)(x+5)\ge0 \) \answer{\( [-12;0,5] \)}
		\item \( x^2-6x+5\ge0 \) \answer{\( (-\infty;1]\cup[18;+\infty) \)}
		\item \( (3x^2-8x+4)(5x^2-8x-4)\le0 \)\answer{ \( \left[ -\dfrac{2}{5};\dfrac{2}{3} \right]\cup\{2\} \) }
	\end{enumcols}
\end{listofex}
\newpage
\title{Консультация}
\begin{listofex}
	\item \exercise{3744}
	\item \exercise{1171}
	\item \exercise{3407}
	\item \exercise{650}
	\item \exercise{3785}
	\item Окружность, построенная на биссектрисе \( AD \) треугольника \( ABC \) как на диаметре, пересекает стороны \( AB \) и \( AC \) соответственно в точках \( M \) и \( N \), отличных от \( A \). Докажите, что \( AM = AN \).
	\item Докажите, что отличная от A точка пересечения окружностей, построенных на сторонах \( AB \) и \( AC \) треугольника \( ABC \) как на диаметрах, лежит на прямой \( BC \).
\end{listofex}