%Группа 101-1 Модуль 2
\title{Занятие №1}
\begin{listofex}
	\item Вычислить:
	\begin{enumcols}[itemcolumns=3]
		\item \exercise{562}
		\item \exercise{564}
		\item \exercise{569}
		\item \exercise{571}
		\item \exercise{579}
	\end{enumcols}
	\item Вычислить:
	\begin{enumcols}[itemcolumns=3]
		\item \exercise{1577}
		\item \exercise{1578}
		\item \exercise{1579}
	\end{enumcols}
	\item Вычислить:
	\begin{enumcols}[itemcolumns=3]
		\item \exercise{1572}
		\item \exercise{1565}
		\item \exercise{1566}
		\item \exercise{1573}
		\item \exercise{1567}
		\item \exercise{1575}
		\item \exercise{1594}
	\end{enumcols}
	\item Вычислить:
	\begin{enumcols}[itemcolumns=2]
		\item \exercise{1569}
		\item \exercise{1570}
		\item \exercise{1571}
		\item \exercise{1574}
	\end{enumcols}
	\item Вычислить:
	\begin{enumcols}[itemcolumns=2]
		\item \exercise{1583}
		\item \exercise{1584}
		\item \exercise{1580}
		\item \exercise{1585}
		\item \exercise{1586}
		\item \exercise{595}
	\end{enumcols}
	\item Решить уравнение:
	\begin{enumcols}[itemcolumns=2]
		\item \( \log_2(4-x)=7 \)
		\item \( \log_{1/7}(7-2x)=-2 \)
		\item \( \log_4(x+3)=\log_4(4x-15) \)
		\item \( \log_5(7-x)=\log_5(3-x)+1 \)
		\item \( \log_8 2^{8x-4}=4 \)
		\item \( \log_5(x^2+13x)=\log_5(9x+5) \)
	\end{enumcols}
\end{listofex}
\newpage
\title{Занятие №2}
\begin{listofex}
	\item Вычислить:
	\begin{enumcols}[itemcolumns=3]
		\item \exercise{563}
		\item \exercise{569}
		\item \exercise{581}
		\item \exercise{1589}
		\item \exercise{1595}
	\end{enumcols}
	\item Вычислить:
	\begin{enumcols}[itemcolumns=2]
		\item \exercise{582}
		\item \exercise{584}
		\item \exercise{591}
		\item \exercise{594}
		\item \exercise{1596}
	\end{enumcols}
	\item Вычислить:
	\begin{enumcols}[itemcolumns=2]
		\item \exercise{592}
		\item \exercise{1597}
	\end{enumcols}
	\item Вычислить:
	\begin{enumcols}[itemcolumns=3]
		\item \exercise{1578}
		\item \exercise{1580}
		\item \exercise{1581}
	\end{enumcols}
	\item Решить уравнение:
	\begin{enumcols}[itemcolumns=2]
		\item \( \log_2(4-x)=7 \) \answer{ \( -124 \) }
		\item \( \log_{1/7}(7-2x)=-2 \) \answer{ \( -21 \) }
		\item \( \log_4(x+3)=\log_4(4x-15) \) \answer{ \( 6 \) }
		\item \( \log_5(7-x)=\log_5(3-x)+1 \) \answer{ \( 2 \) }
		\item \( \log_8 2^{8x-4}=4 \) \answer{ \( 2 \) }
		\item \( \log_5(x^2+13x)=\log_5(9x+5) \) \answer{ \( 1 \) }
	\end{enumcols}
\end{listofex}
\newpage
\title{Домашняя работа №1}
\begin{listofex}
	\item Вычислить:
	\begin{enumcols}[itemcolumns=4]
		\item \exercise{588}
		\item \exercise{1590}
		\item \exercise{1293}
		\item \exercise{567}
		\item \exercise{580}
		\item \exercise{570}
		\item \exercise{573}
		\item \exercise{572}
	\end{enumcols}
	\item Вычислить:
	\begin{enumcols}[itemcolumns=2]
		\item \exercise{582}
		\item \exercise{590}
		\item \exercise{595}
		\item \exercise{1587}
	\end{enumcols}
	\item Вычислить:
	\begin{enumcols}[itemcolumns=3]
		\item \exercise{1568}
		%\item \exercise{1576}
		\item \exercise{1573}
		\item \exercise{1588}
	\end{enumcols}
	\item Вычислить:
	\begin{enumcols}[itemcolumns=2]
		\item \exercise{586}
		\item \exercise{1583}
		\item \exercise{595}
		\item \exercise{1585}
	\end{enumcols}
	\item Решить уравнение:
	\begin{enumcols}[itemcolumns=2]
		\item \exercise{506}
		\item \exercise{3653}
		\item \exercise{1164}
		\item \exercise{3529}
		\item \exercise{606}
		\item \exercise{613}
		\item \exercise{601}
	\end{enumcols}
\end{listofex}
\newpage
\title{Занятие №3}
\begin{listofex}
	\item Вычислить значения синуса и косинуса \( 30\degree \), \( 45\degree \), \( 60\degree \).
	\item Вычислить значения тангенса и котангенса с теми же самыми аргументами.
	\item Доказать следующие факты:\\
	ОТТ: \( \sin^2x+\cos^2x=1 \); \( \tg x = \dfrac{\sin x}{\cos x} \); \( \ctg x = \dfrac{\cos x}{\sin x} \) и \( \tg x \cdot \ctg x = 1 \)\\[1em]
	\fbox{%
		\begin{minipage}[t]{0,9\textwidth}
			\textit{Расширенное понятие синуса и косинуса.}\vspace{1em}
			
			\textbf{Косинус угла \boldmath\( {\alpha} \)} --- абсцисса точки на единичной окружности, соответствующей углу \( \alpha \).
			
			\textbf{Синус угла \boldmath\( \alpha \)} --- ордината точки на единичной окружности, соответствующей углу \( \alpha \).
		\end{minipage}
	}
	\item \exercise{2807}
	\item \fbox{%
		\parbox[t]{0.9\textwidth}{
			\begin{minipage}[t]{0,3\textwidth}
				\( \sin(x+360\degree\cdot n) = \sin x \)
				
				\( \cos(x+360\degree\cdot n) = \cos x \)
			\end{minipage}
			\begin{minipage}[t]{0,3\textwidth}
				
				\( \tg(x+360\degree\cdot n) = \tg x \)
				
				\( \ctg(x+360\degree\cdot n) = \ctg x \)
				
				\vspace{1em}
			\end{minipage}
			\begin{minipage}[t]{0,25\textwidth}
				
				\( \sin(-x) = -\sin x \)
				
				\( \cos(-x) = \cos x \)
			\end{minipage}
			\begin{minipage}[t]{0,25\textwidth}
				\( \sin(180 - x) = \sin x \)
				
				\( \cos(180 - x) = -\cos x \)
			\end{minipage}
			\begin{minipage}[t]{0,25\textwidth}
				
				\( \sin(180+x) = -\sin x \)
				
				\( \cos(180+x) = -\cos x \)
			\end{minipage}
		}
	}
	\item Вычислить:
	\begin{enumcols}[itemcolumns=1]
		\item \exercise{2808}
		%\item \exercise{2809}
		\item \exercise{2810}
		\item \exercise{2811}
	\end{enumcols}
\end{listofex}
\newpage
\title{Занятие №4}
\begin{listofex}
	\item Вычислить через определение координат точки на окружности:
	\[ \sin90\degree;\;\sin270\degree;\;\sin180\degree;\;\cos0\degree;\;\cos360\degree;\;\sin(-90\degree);\;\tg270\degree;\;\ctg(-90\degree);\;\sin720\degree;\;\cos540\degree \]
	\item Вычислить:
	\begin{enumcols}[itemcolumns=1]
		\item \( \cos180\degree(\sin90\degree-\sin30\degree)+\sin30\degree(\cos45\degree+\ctg30\degree) \) \answer{ \( \dfrac{2\sqrt{3}+\sqrt{2}-2}{4} \) }
		\item \( \dfrac{\sin90\degree+\cos30\degree-\sin(-30\degree)}{(\cos30\degree-\sin30\degree\cdot\tg(-45\degree))\cdot\cos(-30\degree)} \) \answer{ \( 2 \) }
	\end{enumcols}
	\item С помощью формул:\quad\( \sin(x+y)=\sin x\cos y + \sin y \cos x \) и \( \cos(x+y)=\cos x \sin y - \sin x \sin y \) выразить следующие формулы:
	\begin{enumcols}[itemcolumns=4]
		\item \( \sin(x-y) \)
		\item \( \cos(x-y) \)
		\item \( \sin2x \)
		\item \( \cos2x \)
	\end{enumcols}
	\item \textbf{Метод приведения аргумента тригонометрических функций:}
	\begin{enumcols}[itemcolumns=1]
		\item[0)] Обязательно сначала вынести минус за знак аргумента;
		\item "Убрать" полные круги из аргумента ;
		\item Представить аргумент в виде суммы или разности;
		\item Определить четверть аргумента;
		\item Определить занк функции в этой четверти;
		\item Поменять или оставить название тригонометрической функции.
	\end{enumcols}
	\item \exercise{2808}
	\item \exercise{2814}
	\item \exercise{2815}
	\item Вычислить:
	\begin{enumcols}[itemcolumns=2]
		\item \exercise{1137}
		\item \exercise{1143}
		\item \exercise{1144}
		\item \exercise{1145}
		\item \exercise{1146}
	\end{enumcols}
\end{listofex}
\newpage
\title{Занятие №5}
\begin{listofex}
	\item \textbf{Тригонометрические формулы:}
	\begin{enumcols}[itemcolumns=2]
		\item \( \sin(x+y)=\sin x\cos y + \sin y \cos x \)
		\item \( \sin(x-y)=\sin x\cos y - \sin y \cos x \)
		\item \( \cos(x+y)=\cos x\cos y - \sin y \sin x \)
		\item \( \cos(x-y)=\cos x\cos y + \sin y \sin x \)
		\item \( \sin(-x)=-\sin x \)
		\item \( \cos(-x)=\cos x \)
		\item \( \tg(-x)=-\tg x \)
		\item \( \ctg(-x)=-\ctg x \)
	\end{enumcols}
	\item \textbf{Метод приведения аргумента тригонометрических функций:}
	\begin{enumcols}
		\item[0)] Вынести минус за знак аргумента;
		\item "Убрать" полные круги из аргумента ;
		\item Представить аргумент в виде суммы/разности;
		\item Определить четверть аргумента;
		\item Определить занк функции в этой четверти;
		\item Поменять/оставить название тригонометрической функции.
	\end{enumcols}
	\item Вычислить по координатам точки на окружности:
	\begin{enumcols}[itemcolumns=1]
		\item \( \cos90\degree;\;\cos270\degree;\;\sin 180\degree;\;\cos360\degree;\;\cos720\degree;\;\sin(-180\degree);\;\tg(-180)\degree; \)
		\item \( \ctg(-90\degree);\;\sin1170\degree;\;\cos(990)\degree;\;\cos(-1710\degree) \)
	\end{enumcols}
	\item Вычислить через формулы суммы/разности:
	\[ \sin150\degree;\;\cos135\degree;\;\sin235\degree;\;\cos(-120\degree);\;\cos330\degree;\;\tg(-150\degree);\;\sin(-225\degree);\;\cos300\degree;\;\sin(-315\degree) \]
	\item Вычислить с помощью метода приведения:
	\[ \sin135\degree;\;\cos240\degree;\;\sin390\degree;\;\tg150\degree;\;\ctg220\degree;\;\sin(-220\degree) \]
	\item Вычислить:
	\begin{enumcols}[itemcolumns=3]
		\item \exercise{2977}
		\item \exercise{2978}
		\item \exercise{2981}
		\item \exercise{2973}
		\item \exercise{2958}
	\end{enumcols}
	\item Вычислить удобным для вас способом:
	\[ \cos\dfrac{5\pi}{4};\;\sin\dfrac{7\pi}{3};\;\sin\dfrac{3\pi}{2};\;\sin\left( -\dfrac{5\pi}{3} \right);\;\cos\dfrac{7\pi}{6};\;\sin\dfrac{13\pi}{4};\;\sin\left( -\dfrac{7\pi}{6}  \right);\;\cos\dfrac{21\pi}{4};\;\tg\dfrac{16\pi}{6};\;\ctg\dfrac{11\pi}{4} \]
	\item Вычислить:
	\begin{enumcols}[itemcolumns=2]
		\item \exercise{1807}
		\item \exercise{2965}
	\end{enumcols}
\end{listofex}
\newpage
\title{Занятие №6}
\begin{listofex}
	\item Вычислить значение:
	\begin{enumcols}[itemcolumns=3]
		\item \exercise{1137}
		\item \exercise{2969}
		\item \exercise{1143}
		\item \exercise{1144}
		\item \exercise{1145}
		\item \exercise{2962}
	\end{enumcols}
\item Вычислить значение:
\begin{enumcols}[itemcolumns=1]
	\item \exercise{1793}
	\item \exercise{1794}
	\item \exercise{1796}
\end{enumcols}
	\item Вычислить значение:
	\begin{enumcols}[itemcolumns=2]
		\item \exercise{2967}
		\item \exercise{2966}
		\item \exercise{2983}
		\item \exercise{2987}
		\item \exercise{2990}
	\end{enumcols}
	\item Вычислить значение:
	\begin{enumcols}[itemcolumns=2]
		\item \exercise{1117}
		\item \exercise{1118}
		\item \exercise{1841}
	\end{enumcols}
	\item \exercise{1828}
	\item \exercise{1820}
	\item \exercise{1804}
\end{listofex}
\newpage
\title{Занятие №7}
\begin{listofex}
	\item Вычислить:
	\begin{enumcols}[itemcolumns=2]
		\item \( 12\sin150\degree\cdot\cos120\degree \) \answer{ \( -3 \) }
		\item \( \dfrac{12\sin407\degree}{\sin47\degree} \)\answer{ \( 12 \) }
		\item \( \dfrac{5\sin10\degree\cdot\cos10\degree}{\sin20\degree} \)
		\item \( \dfrac{2\sqrt{3}\sin60\degree\cdot\cos60\degree}{\cos^2 30\degree - \sin^2 30\degree} \)\answer{ \( \sqrt{3} \) }
	\end{enumcols}
	\item Вычислить:
	\begin{enumcols}[itemcolumns=2]
		\item \( \dfrac{3\cos39\degree}{\sin51\degree}+\dfrac{2\cos31\degree}{\sin59\degree} \) \answer{ \( 5 \) }
		\item \( \dfrac{2\sin388\degree}{\cos242\degree} \) \answer{ \( -2 \) }
		\item \( \dfrac{6\sin33\degree\cos33\degree}{\sin66\degree}+\dfrac{\sin88\degree}{6\sin44\degree\cos44\degree} \) \answer{ \( 3\dfrac{1}{3} \) }
		\item \( \dfrac{10(\sin^2 32\degree-\cos^2 32\degree)}{-4\cos64\degree}+11 \) \answer{ \( 13,5 \) }
	\end{enumcols}
	\item Вычислить:
	\begin{enumcols}[itemcolumns=2]
		\item \( -4\sqrt{3}\sin\left( -\dfrac{7\pi}{3} \right) \) \answer{ \( 6 \) }
		\item \( 2\sqrt{3}\tg\left( -\dfrac{13\pi}{6} \right) \) \answer{ \( -2 \) }
		\item \( (3\sqrt{3})^2\tg\left( \dfrac{\vphantom{7}\pi}{12} \right)\cdot\tg\left( \dfrac{7\pi}{12} \right) \) \answer{ \( -6 \) }
		\item \( \dfrac{7}{\cos^2\left( \dfrac{\vphantom{9}\pi}{16} \right)+\cos^2\left( \dfrac{9\pi}{16} \right)} \) \answer{ \( 7 \) }
		\item \( \sqrt{3}-\sqrt{12}\sin^2\dfrac{10\pi}{12} \) \answer{ \( \dfrac{\sqrt{3}}{2} \) }
		\item \( \dfrac{25}{\sin^2\dfrac{11\pi}{24}+1+\sin^2\dfrac{23\pi}{24}} \)
	\end{enumcols}
	\item Вычислить:
	\begin{enumcols}[itemcolumns=4]
		\item \( \log_4 16 \)
		\item \( \log_{1/5}5\sqrt{5} \)
		\item \( \log_{\sqrt[5]{2}}32 \)
		\item \( \log_{1/7}^2 49 \)
	\end{enumcols}
	\item Вычислить:
	\begin{enumcols}[itemcolumns=1]
		\item \exercise{1095}
		\item \exercise{1093}
	\end{enumcols}
	\item \exercise{1423}
\end{listofex}
\newpage
\title{Домашняя работа №2}
\begin{listofex}
	\item Вычислить:
	\begin{enumcols}[itemcolumns=1]
		\item \( 4\sqrt{3}\cos150\degree\cdot\sin210\degree \) \answer{ \( 3 \) }
		\item \( \dfrac{15\cos395\degree}{\cos35\degree} \)
		\item \( \cos240\degree(\sin45\degree+\sin135\degree)-\sin60\degree(\cos180\degree+\ctg45\degree) \)
	\end{enumcols}
	\item Вычислить:
	\begin{enumcols}[itemcolumns=1]
		\item \( \left( \dfrac{4\tg120\degree\cdot\cos210\degree-\sin270\degree}{2\cos240\degree-3\sqrt{3}\sin210\degree} \right)\cdot\dfrac{5}{3\sqrt{3}+2}-\dfrac{1}{23} \)\answer{ \( 3 \) }
		\item \( \dfrac{\sqrt{8}\sin\left( -\dfrac{\pi}{4} \right)+\sqrt{27}\cos\left( \dfrac{\pi}{3} \right)-4\sin\left( -\dfrac{\pi}{6} \right)}{6\sqrt{3}} \) \answer{ \( 0,25 \) }
		\item \( 4\cos\left( \dfrac{2\pi}{3} \right)-\left( \sqrt{3}+1 \right)\left( \ctg\left( \dfrac{7\pi}{6} \right)-1 \right) \) \answer{ \( -4 \) }
		\item \( \left( 4 - \sin\left( -\dfrac{10\pi}{3} \right) \right)^2+4\tg\left( \dfrac{\pi}{3} \right) \) \answer{ \( 16,75 \) }
	\end{enumcols}
	\item Вычислить:
	\begin{enumcols}[itemcolumns=2]
		\item \( 4\sqrt{2}\tg\dfrac{\pi}{4}\cos\dfrac{7\pi}{3}+11 \)
		\item \( \dfrac{8}{\sin\left( -\dfrac{27\pi}{4}\right)\cos\left( \dfrac{31\pi}{4} \right) } \)
	\end{enumcols}
	\item Вычислить:
	\begin{enumcols}[itemcolumns=2]
		\item \( \dfrac{4\sin22\degree\cos22\degree}{\cos66\degree}+\dfrac{\sin100}{4\sin50\degree\cos50\degree} \)
		\item \( \dfrac{22(\sin^2 16\degree-\cos^2 16\degree)}{\cos32\degree}+5 \)
	\end{enumcols}
	\item Найдите значение выражения \( 5\tg(5\pi-x)-\tg(-x) \), если \( \tg x = 7 \)
	\item Вычислить:
	\begin{enumcols}[itemcolumns=2]
		\item \exercise{1588}
		\item \exercise{1567}
		\item \exercise{1294}
		\item \exercise{590}
		\item \exercise{1569}
	\end{enumcols}
	\item Расстояние между городами \( A \) и \( B \) равно \( 435 \) км. Из города \( A \) в город \( B \) со скоростью \( 60 \) км/ч выехал первый автомобиль, а через час после этого навстречу ему из города B выехал со скоростью \( 65 \) км/ч второй автомобиль. На каком расстоянии от города A автомобили встретятся? Ответ дайте в километрах.
\end{listofex}
\newpage
\title{Проверочная работа}
\begin{listofex}
	\item Вычислить:
	\begin{enumcols}[itemcolumns=3]
		\item \exercise{562}
		\item \exercise{571}
		\item \exercise{1570}
		\item \exercise{1573}
		\item \exercise{1583}
		\item \exercise{595}
		\item \exercise{1594}
	\end{enumcols}
	\item Решить уравнение:
	\begin{enumcols}[itemcolumns=2]
		\item \( \log_{1/7}(5-4x)=-1 \)
		\item \( \log_4(3x+3)=\log_4(2x-11) \)
		\item \( \log_5(7-x)=\log_5(3-x)+1 \)
		\item \( \log_4 2^{8x-4}=2 \)
		\item \exercise{606}
		\item \( \log_5(x^2+13x)=\log_5(9x+5) \)
	\end{enumcols}
	\item Вычислить с помощью метода приведения:
	\[ \cos135\degree;\;\cos225\degree;\;\sin405\degree;\;\tg120\degree;\sin(-150\degree) \]
	\item Вычислить:
	\begin{enumcols}[itemcolumns=3]
		\item \exercise{1137}
		\item \exercise{1143}
		\item \exercise{1145}
		\item \exercise{1146}
		\item \exercise{2962}
	\end{enumcols}
	\item Вычислить значение:
	\begin{enumcols}[itemcolumns=3]
		\item \exercise{2967}
		\item \( \dfrac{13}{4\sin^237\degree+4\sin^2127\degree} \)
		\item \exercise{2987}
	\end{enumcols}
	\item Вычислить:
	\begin{enumcols}[itemcolumns=2]
		\item \( -4\sqrt{3}\sin\left( -\dfrac{4\pi}{3} \right) \)
		\item \( (2\sqrt{5})^2\tg\left( \dfrac{\vphantom{3}\pi}{4} \right)\cdot\tg\left( \dfrac{\pi}{4} \right) \)
		\item \( \dfrac{7}{\cos^2\left( \dfrac{\vphantom{9}\pi}{8} \right)+\cos^2\left( \dfrac{5\pi}{8} \right)} \)
		\item \( \sqrt{3}-\sqrt{12}\sin^2\dfrac{7\pi}{12} \)
	\end{enumcols}
	\item \exercise{1117}
\end{listofex}
\newpage
\title{Консультация}
\begin{listofex}
	\item Вычислить:
	\begin{enumcols}[itemcolumns=2]
		\item \exercise{573}
		\item \exercise{586}
		\item \exercise{1577}
		\item \exercise{1590}
		\item \exercise{1595}
	\end{enumcols}
	\item Вычислить:
	\begin{enumcols}[itemcolumns=2]
		\item \exercise{595}
		\item \exercise{1584}
	\end{enumcols}
	\item Вычислить значение выражения:
	\begin{enumcols}[itemcolumns=2]
		\item \exercise{597}
		\item \exercise{599}
	\end{enumcols}
	\item Решить уравнение:
	\begin{enumcols}[itemcolumns=2]
		\item \exercise{607}
		\item \exercise{609}
		\item \exercise{611}
		\item \exercise{777}
		\item \exercise{780}
		\item \exercise{783}
		\item \exercise{786}
	\end{enumcols}
\end{listofex}