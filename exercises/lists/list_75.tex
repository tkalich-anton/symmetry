%Азарных Андрей Занятие №1
\begin{enumcols}[label=\textbf{\arabic*.}]
	\item Упростить выражение:
	\begin{enumcols}[itemcolumns=2]
		\item \( \sqrt{2}+3\sqrt{32}+\dfrac{1}{2}\sqrt{128}-6\sqrt{18} \)
		\item \( (3-\sqrt{2})(2+3\sqrt{2}) \)
		\item \( (8+3\sqrt{5})(2-\sqrt{5}) \)
		\item \( (7-\sqrt{3})(\sqrt{3}+7) \)
		\item \( (\sqrt{2}+1)^2+(\sqrt{2}-1)^2 \)
		\item \( (\sqrt{7}-2)^2+4\sqrt{7} \)
	\end{enumcols}
	\item Сократить дробь:
	\begin{enumcols}[itemcolumns=2]
		\item \( \dfrac{3\sqrt{2}+2\sqrt{2}}{\sqrt{200}} \)
		\item \( \dfrac{\sqrt{5}+5}{\sqrt{5}} \)
	\end{enumcols}
	\item Освободитесь от иррациональности в знаменателе:
	\begin{enumcols}[itemcolumns=3]
		\item \( \dfrac{1}{\sqrt{2}-1} \)
		\item \( \dfrac{2}{\sqrt{3}-1} \)
		\item \( \dfrac{\sqrt{5}-\sqrt{3}}{\sqrt{5}+\sqrt{3}} \)
	\end{enumcols}
	\item Сократить дробь:
	\begin{enumcols}[itemcolumns=4]
		\item \exercise{20}
		\item \exercise{51}
		\item \exercise{55}
		\item \exercise{63}
		\item \exercise{67}
		\item \exercise{68}
		\item \exercise{78}
		\item \exercise{91}
		\item \exercise{92}
		\item \exercise{97}
		\item \exercise{106}
		\item \exercise{119}
		\item \exercise{123}
		\item \exercise{125}
		\item \exercise{118}
		\item \exercise{128}
	\end{enumcols}
	\item Представить в виде несократимой дроби:
	\begin{enumcols}[itemcolumns=3]
		\item \exercise{130}
		\item \exercise{137}
		\item \exercise{139}
		\item \exercise{826}
		\item \exercise{830}
		\item \exercise{833}
		\item \exercise{837}
		\item \exercise{841}
		\item \exercise{844}
		\item \exercise{848}
		\item \exercise{857}
	\end{enumcols}
	\item Представить в виде несократимой дроби:
	\begin{enumcols}[itemcolumns=2]
		\item \exercise{874}
		\item \exercise{876}
		\item \exercise{878}
		\item \exercise{880}
		\item \exercise{883}
		\item \exercise{890}
	\end{enumcols}
\end{enumcols}