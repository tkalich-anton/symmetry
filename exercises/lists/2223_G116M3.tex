%
%===============>> ГРУППА 11-6 МОДУЛЬ 3 <<=============
%
\setmodule{3}
%
%===============>>  Занятие 1  <<===============
%
%\begin{class}[number=1]
%	\begin{listofex}
%		
%	\end{listofex}
%\end{class}
%
%===============>>  Домашняя работа 1  <<===============
%
\begin{homework}[number=1]
	\begin{listofex}
	\item 	\exercise{2807}
	\item \exercise{2808}
	\item \exercise{2809}
	\item На счёте Настиного мобильного телефона было \( 82 \) рубля, а после разговора с Лерой осталось \( 40 \) рублей. Известно, что разговор длился целое число минут, а одна минута разговора стоит \( 3 \) рубля \( 50 \) копеек. Сколько минут длился разговор с Лерой?
	\item В доме, в котором живет Вася, один подъезд. На каждом этаже находится по 4 квартиры. Вася живет в квартире \( 71 \). На каком этаже живет Вася?
	\item На день рождения полагается дарить букет из нечётного числа цветов. Ромашки стоят \( 20 \) рублей за штуку. У Вани есть \( 90 \) рублей. Из какого наибольшего числа ромашек он может купить букет Маше на день рождения?
	\item Для ремонта требуется купить \( 23 \) лампочки. Каждая лампочка стоит \( 37 \) рублей. Сколько рублей сдачи получит покупатель, давший кассиру \( 1000 \) рублей за такую покупку?
	\item В летнем лагере \( 172 \) ребёнка и \( 24 \) воспитателя. В одном автобусе можно перевозить не более \( 30 \) пассажиров. Какое наименьшее количество таких автобусов понадобится, чтобы за один раз перевезти всех из лагеря в город?
	\item Шоколадка стоит \( 40 \) рублей. В воскресенье в супермаркете действует специальное предложение: заплатив за две шоколадки, покупатель получает три (одну в подарок). Сколько шоколадок можно получить на \( 320 \) рублей в воскресенье?
	\item Для ремонта требуется \( 57 \) рулонов обоев. Какое наименьшее количество пачек обойного клея нужно для такого ремонта, если \( 1 \) пачка клея рассчитана на \( 5 \) рулонов?
	\item Спидометр автомобиля показывает скорость в милях в час. Какую скорость (в милях в час) показывает спидометр, если автомобиль движется со скоростью \( 120 \) км в час? (Считайте, что \( 1  \) миля равна \( 1,6  \) км.)
	\item На автозаправке клиент отдал кассиру \( 1000 \) рублей и попросил залить бензин до полного бака. Цена бензина -- \( 32 \) рубля за литр. Клиент получил \( 72 \) рубля сдачи. Сколько литров бензина было залито в бак?
	\item В среднем за день во время конференции расходуется \( 70 \) пакетиков чая. Конференция длится \(  4 \) дня. В пачке чая \( 100 \) пакетиков. Какого наименьшего количества пачек чая хватит на все дни конференции?
	\item Больному прописано лекарство, которое нужно пить по \( 0.5 \) г \( 4 \) раза в день в течение \( 3 \) дней. В одной упаковке \( 10 \) таблеток лекарства по \( 0.5 \) г. Какого наименьшего количества упаковок хватит на весь курс лечения?
	\item Сырок стоит \( 16 \) рублей \( 70 \) копеек. Какое наибольшее число сырков можно купить на \( 120 \) рублей?
	\end{listofex}
\end{homework}
%
%
%===============>>  Занятие 4  <<===============
%
%\begin{class}[number=4]
%	\begin{listofex}
%		
%	\end{listofex}
%\end{class}
%
%===============>>  Домашняя работа 2  <<===============
%
%\begin{homework}[number=2]
%	\begin{listofex}
%	
%	\end{listofex}
%\end{homework}
%
%===============>>  Занятие 5  <<===============
%
%\begin{class}[number=5]
%	\begin{listofex}
%	
%	\end{listofex}
%\end{class}
%
%===============>>  Занятие 6  <<===============
%
%\begin{class}[number=6]
%	\begin{listofex}
%	
%	\end{listofex}
%\end{class}
%
%===============>>  Домашняя работа 3  <<===============
%
%\begin{homework}[number=3]
%	\begin{listofex}
%
%	\end{listofex}
%\end{homework}
%\newpage
%\title{Подготовка к проверочной работе}
%\begin{listofex}
%	
%\end{listofex}
%
%===============>>  Занятие 7  <<===============
%
%\begin{class}[number=7]
%	\begin{listofex}
%	
%	\end{listofex}
%\end{class}
%
%===============>>  Проверочная работа  <<===============
%
%\begin{exam}
%	\begin{listofex}
%	
%	\end{listofex}
%\end{exam}
