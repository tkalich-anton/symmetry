%===============>> ГРУППА 7-1 МОДУЛЬ 3 <<=============
\setmodule{3}
%===============>>     Занятие 1     <<===============
\begin{class}[type=class, number=1]
	\begin{listofex}
		\item Вычислить: \( \left( 6\dfrac{5}{9}-3\dfrac{1}{4} \right)\cdot2\dfrac{2}{17} \)
		\item Раскройте скобки и приведите подобные слагаемые:
		\begin{enumcols}[itemcolumns=2]
			\item \( 2(4x+1)+5(2x+6) \)
			\item \( 7(x+2y)+6(y-x) \)
			\item \( 2,1(2x-y)+4,2(x+3y)+1,2(x-4y) \)
			\item \( 2,5(1,2x-4y)+3(3y+x)-x\)
			\item \( 2\dfrac{1}{7}(3,5a+7b-14)+4(2a-b+5) \)
			\item \( 3(2a-b+7)+1\:\dfrac{8}{9}\left( \dfrac{9}{17}a+9b-6 \right) \)
		\end{enumcols}
		\item Раскройте скобки cо знаком "минус":
		\begin{enumcols}[itemcolumns=3]
			\item \( -(a-b) \)
			\item \( -(a+b) \)
			\item \( -(2x-3y+6a) \)
			\item \( -2(x-y+5a) \)
			\item \( -\left( 3x-5y-2\:\dfrac{1}{3}+8 \right) \)
			\item \( -11(7x-0,11y-2) \)
		\end{enumcols}
		\item Раскройте скобки и приведите подобные слагаемые:
		\begin{enumcols}[itemcolumns=2]
			\item \( k-(y-c)+(d-c-y)+(-k+b) \)
			\item \( 9-2(-c+5) \)
			\item \( -2(d+3)+3(2-d) \)
			\item \( -2(10x-5y+4)+3(2x-6y+5) \)
			\item \( -12\left( \dfrac{1}{2}x-\dfrac{1}{3}+\dfrac{1}{4}z-1 \right)+2(3x-4y-5) \)
		\end{enumcols}
		\item Раскройте скобки и приведите подобные слагаемые:
		\begin{enumcols}[itemcolumns=2]
			\item \( (2a^2b-10b^3)-(4a^2b-12b^3) \)
			\item \( \left( \dfrac{1}{2}x^2y^2-\dfrac{2}{3}ab-\dfrac{5}{6}a^2b \right)-\left( a^2b-\dfrac{1}{3}x^2y^2+\dfrac{1}{2}ab \right) \)
		\end{enumcols}
		\item Найдите значение выражения: \( \dfrac{(2x)^4}{(4x)^2} \) при \( x=-\dfrac{2}{3} \)
	\end{listofex}
\end{class}
%===============>>     Занятие 2     <<===============
\begin{class}[type=class, number=2]
	\begin{listofex}
		\item Раскройте скобки и приведите подобные слагаемые:
		\begin{enumcols}[itemcolumns=2]
			\item \( -7(x^2+4)+5(2x^2+6) \)
			\item \( 12(x+2y)+6(9-x)-24y \)
			\item \( 3,1(5x-2y)+2,1(x+2,2y)+3,7y \)
			\item \( 2,5(4x^3-4y)+3(3y+x^3)-13x^3-y\)
			\item \( \dfrac{1}{2}\left( 2x^2+4a^2 \right)-\dfrac{3}{2}\left( \dfrac{2}{3}x^2-2a^2 \right) \)
		\end{enumcols}
		\item Раскройте скобки cо знаком "минус":
		\begin{enumcols}[itemcolumns=3]
			\item \( -(4a-2x) \)
			\item \( -(-x-3y) \)
			\item \( -(5x^2-7x+4) \)
			\item \( -12(x-y+4a) \)
			\item \( -\left( \dfrac{5}{6}x^4+13x^2 \right) \)
		\end{enumcols}
		\item Замените звездочку так, чтобы равенство было верным:
		\begin{enumcols}[itemcolumns=2]
			\item \( *\:(3x^2-5)=9x^2-\:* \)
			\item \( -5(*-4x^4)=-25a^2+\:* \)
		\end{enumcols}
		\item Раскройте скобки и приведите подобные слагаемые:
		\begin{enumcols}[itemcolumns=2]
			\item \( (23a^3b-11x^3)-(17a^3b-10x^3)+x^3 \)
			\item \( 2(x^3-12x^3a^3)+(3ax)^3-x^3 \)
		\end{enumcols}
		\item Найдите значение выражения: \( \dfrac{(9x)^7}{3(3x)^5} \) при \( x=-0,3 \)
		\item Раскрыть скобки и привести подобные слагаемые:
		\begin{enumcols}[itemcolumns=3]
			\item \( (x-2)(2x+1) \)
			\item \( (a-1)(a+1) \)
			\item \( (4a-12)(3a^2+5) \)
			\item \( (x^2+4)(x^2+0,25) \)
			\item \( (0,4x+3x^2)(2x^2-4) \)
			\item \( \left( \dfrac{2}{7}x^2+2 \right)\left( \dfrac{7}{2}x-7 \right) \)
		\end{enumcols}
	\end{listofex}
\end{class}
%===============>> Домашняя работа 1 <<===============
\begin{class}[type=homework, number=1]
	\begin{listofex}
		\item Раскройте скобки и приведите подобные слагаемые:
		\begin{enumcols}[itemcolumns=2]
			\item \( -6(x^3+7)+5(3x^3+1) \)
			\item \( 10(x-12y)+10(9-x)+119y \)
			\item \( 2,05(3x-15y)+3,5(x+2,07y) \)
			\item \( 3,5\left( \dfrac{10}{7}x^5+\dfrac{5}{7}a^2 \right)-2,8\left( \dfrac{1}{14}x^5+\dfrac{3}{28}a^2 \right) \)
		\end{enumcols}
		\item Замените звездочку так, чтобы равенство было верным:
		\begin{enumcols}[itemcolumns=2]
			\item \( -3(*-7x^2)=-24x^6+\:* \)
			\item \( *\:(2x^4+5)=-8x^4-\:* \)
		\end{enumcols}
		\item Найдите значение выражения: \( 13x^3-4(2x^3-2x^4)-2(2x^2)^2 \) при \( x=-3 \)
		\item Найдите значение выражения: \( \dfrac{(5x)^9}{(5x)^8} \) при \( x=0,5 \)
		\item Вычислить:
		\begin{enumcols}[itemcolumns=4]
			\item \( \dfrac{15^9}{15^7} \)
			\item \( \dfrac{(-0,1)^{22}}{(-0,1)^{20}} \)
			\item \( \left( \dfrac{14}{5} \right)^{9}:\left( 2,8 \right)^{7} \)
			\item \( \left( \dfrac{25^{19}}{25^{18}} \right)^2 \)
		\end{enumcols}
		\item Раскрыть скобки и привести подобные слагаемые:
		\begin{enumcols}[itemcolumns=3]
			\item \( (a-1)(3a+1) \)
			\item \( (4x^2-13)(2x^2+x) \)
			\item \( (0,5a-10)(2a^2+0,1a^4) \)
			\item \( (x^2+1)(x^2-1) \)
			\item \( \left( \dfrac{5}{9}x^2+5 \right)\left( \dfrac{9}{5}x+9 \right) \)
			\item \( \left( \dfrac{12}{5}x-0,36 \right)\left( \dfrac{1}{24}x^4-\dfrac{100}{36} \right) \)
			\item \( (0,1x^2-0,01)(0,02x+0,2) \)
		\end{enumcols}
	\end{listofex}
\end{class}
%===============>>     Занятие 3     <<===============
\begin{class}[type=class, number=3]
	\begin{listofex}
		\item Выполните умножение:
		\begin{enumcols}[itemcolumns=2]
			\item \( (a+1)(a+1) \)
			\item \( (5m+7n)(2n+4m) \)
			\item \( (-a-b)(2a-3b) \)
			\item \( (mn^3-m^2)(m-1) \)
			\item \( (8x-3)(4x+5) \)
			\item \( (8x-3)\cdot4x+5 \)
			\item \( (1,2x-a)(1,2a+x) \)
			\item \( (3,5x-2y+3a)(2,2a+3x) \)
		\end{enumcols}
		\item Выполните умножение и запишите многочлен в стандартном виде:
		\begin{enumcols}[itemcolumns=3]
			\item \( (x+1)(x^2-x+1) \)
			\item \( (x^3+2x-3)(2-3x) \)
			\item \( (a-b-c)(a-1) \)
			\item \( (5m^2-3mn+n^2)(2n-m^2) \)
			\item \( (c^2-cd-d^2)(c+d) \)
			\item \( (a^2-2a+3)(a-1) \)
		\end{enumcols}
		\item Выполните умножение и запишите многочлен в стандартном виде:
		\begin{enumcols}[itemcolumns=2]
			\item \( \left( \dfrac{1}{3}-m \right)\left( \dfrac{1}{2}m-3 \right) \)
			\item \( (0,05y-2,3x)(y-0,2x) \)
			\item \( \left( \mfrac{1}{1}{2}x-y \right)\left( \mfrac{2}{1}{3}y-\dfrac{1}{3}x \right) \)
			\item \( (0,25x^2-1,2x+5)(x-0,5x^2-1) \)
		\end{enumcols}
		\item Докажите равенство:\quad\( (a+b)(a+c)=a^2+(b+c)a+bc \)
		\item Упростить выражение:
		\begin{enumcols}[itemcolumns=2]
			\item \( (3b-2)(5-2b)+6b^2 \)
			\item \( x^3-(x^2-3x)(x+3) \)
		\end{enumcols}
		\item Выполните умножение:\quad\( (x+1)(x+2)(x+3) \)
	\end{listofex}
\end{class}

%\newpage
%\title{Занятие №4}
%\begin{listofex}
%	
%\end{listofex}
%\newpage
%\title{Домашняя работа №2}
%\begin{listofex}
%	
%\end{listofex}
%\newpage
%\title{Занятие №5}
%\begin{listofex}
%	
%\end{listofex}
%\newpage
%\title{Занятие №6}
%\begin{listofex}
%	
%\end{listofex}
%\newpage
%\title{Домашняя работа №3}
%\begin{listofex}
%	
%\end{listofex}
%\newpage
%\title{Подготовка к проверочной работе}
%\begin{listofex}
%	
%\end{listofex}
%\newpage
%\title{Проверочная работа}
%\title{Вариант 1}
%\begin{listofex}
%	
%\end{listofex}
%\newpage
%\title{Проверочная работа}
%\begin{listofex}
%	
%\end{listofex}