%===============>> ГРУППА 10-2 МОДУЛЬ 3 <<=============
\setmodule{3}
%===============>>     Занятие 1     <<===============
\begin{class}[number=1]
	\begin{listofex}
		\item Рассмотрите прямоугольный треугольник с острым углом, равным \( x \), и гипотенузой, равной \( 1 \):
		\begin{enumcols}
			\item Чему равны катеты такого треугольника?
			\item Чему будут равны катеты, если гипотенуза будет равна \( c \)?
			\item Запишите теорему Пифагора для данного треугольника с гипотенузой, равной \( 1 \) (Основное тригонометрическое тождество);
			\item Убедитесь, что если гипотенуза будет равна \( c \), то ОТТ (основное тригонометрическое тождество) выполняется;
			\item Убедитесь, что при любом значении гипотенузы: \( \tg x = \dfrac{\sin x}{\cos x} \) и \( \ctg x = \dfrac{\cos x}{\sin x} \).
		\end{enumcols}
		\item Рассмотрите прямоугольный треугольник с углом \( 30\degree \) и гипотенузой равной \( 1 \):
		\begin{enumcols}
			\item Найдите катеты этого треугольника;
			\item Вычислите \( \sin,\:\cos,\:\tg,\:\ctg \) углов \( 30\degree \) и \( 60\degree \);
			\item Сделайте то же самое для треугольника с углом \( 30\degree \) и гипотенузой равной \( 3 \). Что можно сказать про \( \sin,\:\cos,\:\tg,\:\ctg \) углов \( 30\degree \) и \( 60\degree \)?
		\end{enumcols}
		\item Проделать те же действия для прямоугольного треугольника с углом \( 45\degree \) и гипотенузой равной \( 1 \).
		\item Вычислить значения тангенса и котангенса с теми же самыми аргументами.
		\item Записать все получившиеся значения для \( \sin,\:\cos,\:\tg,\:\ctg \) углов \( 30\degree \), \( 45\degree \) и \( 60\degree \) в таблицу.
		\item \textit{Расширенное понятие синуса и косинуса:}\\[0.5em]
		\fbox{%
			\begin{minipage}[t]{0,92\textwidth}
				
				\textbf{\boldmath\( {\cos x} \)} --- абсцисса точки на единичной окружности, соответствующей углу \( x \).
				
				\textbf{\boldmath\( \sin x \)} --- ордината точки на единичной окружности, соответствующей углу \( x \).
			\end{minipage}
		}
		\item \exercise{2807}
		\item Выяснить, почему при \( n\in Z \):
		\begin{enumcols}[itemcolumns=2]
			\item \( \sin(x+360\degree\cdot n) = \sin x \);
			\item \( \cos(x+360\degree\cdot n) = \cos x \);
			\item \( \tg(x+360\degree\cdot n) = \tg x \);
			\item \( \ctg(x+360\degree\cdot n) = \ctg x \).
		\end{enumcols}
		\item Доказать геометрическим способом, что:
		\begin{enumcols}[itemcolumns=3]
			\item \( \sin(-x) = -\sin x \);
			\item \( \cos(-x) = \cos x \);
			\item \( \sin(180 - x) = \sin x \);
			\item \( \cos(180 - x) = -\cos x \);
			\item \( \sin(180+x) = -\sin x \);
			\item \( \cos(180+x) = -\cos x \).
		\end{enumcols}
	\end{listofex}
\end{class}
%===============>>     Занятие 2     <<===============
\begin{class}[number=2]
	\begin{listofex}
		\item \textbf{Формулы с прошлого урока:}
		\begin{enumcols}[itemcolumns=3]
			\item \( \sin(-x) = -\sin x \);
			\item \( \cos(-x) = \cos x \);
			\item \( \sin(180 - x) = \sin x \);
			\item \( \cos(180 - x) = -\cos x \);
			\item \( \sin(180+x) = -\sin x \);
			\item \( \cos(180+x) = -\cos x \).
		\end{enumcols}
		\item Вычислить:
		\begin{enumcols}[itemcolumns=5]
			\item \( \cos120\degree \)
			\item \( \cos150\degree \)
			\item \( \sin225\degree \)
			\item \( \sin(-135\degree )\)
			\item \( \cos225\degree \)
			\item \( \tg(-120\degree) \)
			\item \( \cos405\degree \)
			\item \( \sin540\degree \)
			\item \( \cos(-510\degree) \)
			\item \( \sin(-450\degree) \)
		\end{enumcols}
		\item \textbf{Формулы суммы/разности синуса или косинуса:}
		\begin{enumcols}[itemcolumns=2]
			\item \( \sin(x+y)=\sin x\cos y + \sin y \cos x \)
			\item \( \sin(x-y)=\sin x\cos y - \sin y \cos x \)
			\item \( \cos(x+y)=\cos x \cos y - \sin x \sin y \)
			\item \( \cos(x-y)=\cos x \cos y + \sin x \sin y \)
		\end{enumcols}
		\item Упростить с помощью данных формул:
		\begin{enumcols}[itemcolumns=4]
			\item \( \sin(90+x)\)
			\item \( \sin(90-x)\)
			\item \( \sin(180+x)\)
			\item \( \sin(180-x)\)
			\item \( \sin(270+x)\)
			\item \( \sin(270-x)\)
			\item \( \sin(360+x)\)
			\item \( \sin(360-x)\)
		\end{enumcols}
		\item Упростить с помощью данных формул:
		\begin{enumcols}[itemcolumns=4]
			\item \( \cos(90+x)\)
			\item \( \cos(90-x)\)
			\item \( \cos(180+x)\)
			\item \( \cos(180-x)\)
			\item \( \cos(270+x)\)
			\item \( \cos(270-x)\)
			\item \( \cos(360+x)\)
			\item \( \cos(360-x)\)
		\end{enumcols}
		\item Вычислить:
		\begin{enumcols}[itemcolumns=5]
			\item \( \sin 300\degree \)
			\item \( \cos 240\degree \)
			\item \( \tg 330\degree \)
			\item \( \cos 120\degree \)
			\item \( \sin 390\degree \)
			\item \( \cos 495\degree \)
			\item \( \cos (-780\degree) \)
			\item \( \sin (-300\degree) \)
			\item \( \tg (-225\degree) \)
			\item \( \sin (-1200\degree) \)
		\end{enumcols}
		\item Вычислить:
		\begin{enumcols}[itemcolumns=4]
			\item \exercise{1137}
			\item \exercise{1143}
			\item \exercise{2973}
			\item \exercise{2980}
		\end{enumcols}
		\item Вычислить:
		\begin{enumcols}[itemcolumns=2]
			\item \( \dfrac{51\cos 4\degree}{\sin 86\degree}+\dfrac{\sqrt{3}}{2}\cdot\dfrac{\sin 60\degree}{3} \)
			\item \( \dfrac{32\cos116\degree}{\sin 64\degree}+\dfrac{25\cos 29\degree}{\sin 61\degree} \)
		\end{enumcols}
		\item При температуре \( 0^{\circ}  \) рельс имеет длину \( l_0=12,5 \)м. При возрастании температуры происходит тепловое расширение рельса, и его длина, выраженная в метрах, меняется по закону \( l(t^{\circ})=l_0(1+\alpha\cdot t^{\circ}) \), где  \( \alpha=1,2\cdot 10^{-5}(^{\circ}C)^{-1} \) – коэффициент теплового расширения, \( t^{\circ} \) – температура (в градусах Цельсия). При какой температуре рельс удлинится на \( 6 \) мм? Ответ выразите в градусах Цельсия.
		\item  Из пункта \( A \) в пункт \( B \) одновременно выехали два автомобиля. Первый проехал с постоянной скоростью весь путь. Второй проехал первую половину пути со скоростью \( 24 \)км/ч, а вторую половину пути – со скоростью, на \(16\) км/ч больше скорости первого, в результате чего прибыл в пункт \( B \) одновременно с первым автомобилем. Найдите скорость первого автомобиля. Ответ дайте в км/ч.
	\end{listofex}
\end{class}
%===============>> Домашняя работа 1 <<===============
\begin{homework}[number=1]
	\begin{listofex}
		\item Вычислить:
		\begin{enumcols}[itemcolumns=4]
			\item \( \cos90\degree \)
			\item \( \sin90\degree \)
			\item \( \cos(135\degree )\)
			\item \( \sin225\degree \)
			\item \( \tg(-135\degree) \)
			\item \( \ctg(-120\degree) \)
			\item \( \cos540\degree \)
			\item \( \sin495\degree \)
			\item \( \sin(-1125)\degree \)
			\item \( \tg(-960\degree) \)
			\item \( \ctg(750\degree) \)
			\item \( \tg1620\degree \)
		\end{enumcols}
		\item Вычислить:
		\begin{enumcols}[itemcolumns=3]
			\item \exercise{1138}
			\item \exercise{1144}
			\item \exercise{2974}
			\item \exercise{2981}
			\item \exercise{2977}
			\item \exercise{2985}
		\end{enumcols}
		\item Вычислить:
		\begin{enumcols}[itemcolumns=2]
			\item \( \dfrac{100,5\cdot\cos 10\degree}{\sin 80\degree}+\dfrac{\sin 45\degree}{2}\cdot\sqrt[]{2} \)
			\item \( \dfrac{20\cos140\degree}{\sin 50\degree}+\dfrac{10\cos 3\degree}{\sin 87\degree} \)
		\end{enumcols}
		\item По закону Ома для полной цепи сила тока, измеряемая в амперах, равна \( I=\dfrac{\sigma}{R+r} \), где \(\sigma\) – ЭДС источника (в вольтах), \(r=2\) Ом – его внутреннее сопротивление, \(R\) – сопротивление цепи (в омах). При каком наименьшем сопротивлении цепи сила тока будет составлять не более \(40\% \) от силы тока короткого замыкания \( I_{кз} =\dfrac{\sigma}{r}\)? (Ответ выразите в омах).
		\item Из пункта \( A \) в пункт \( B \) одновременно выехали два автомобиля. Первый проехал с постоянной скоростью весь путь. Второй проехал первую половину пути со скоростью \( 72 \)км/ч, а вторую половину пути – со скоростью, на \(10\) км/ч больше скорости первого, в результате чего прибыл в пункт \( B \) одновременно с первым автомобилем. Найдите скорость первого автомобиля. Ответ дайте в км/ч.
		\item Вычислить:
		\begin{enumcols}[itemcolumns=2]
			\item \exercise{1744}
			\item \exercise{1740}
		\end{enumcols}
	\end{listofex}
\end{homework}
%===============>>     Занятие 3     <<===============
\begin{class}[number=3]
	\begin{listofex}
		\item \textbf{Формулы суммы/разности синуса или косинуса:}
		\begin{enumcols}[itemcolumns=2]
			\item \( \sin(x+y)=\sin x\cos y + \sin y \cos x \)
			\item \( \sin(x-y)=\sin x\cos y - \sin y \cos x \)
			\item \( \cos(x+y)=\cos x \cos y - \sin x \sin y \)
			\item \( \cos(x-y)=\cos x \cos y + \sin x \sin y \)
		\end{enumcols}
		\item Вычислить через формулы суммы/разности:
		\[ \sin150\degree;\;\cos135\degree;\;\sin225\degree;\;\cos(-120\degree);\;\cos330\degree;\;\tg(-150\degree);\;\sin(-225\degree);\;\cos300\degree;\;\sin(-315\degree) \]
		\item \textbf{Метод приведения аргумента тригонометрических функций:}
		\begin{enumcols}
			\item[0)] Выносим минус за знак аргумента;
			\item "Убираем"{ }полные круги из аргумента \textit{(в будущем не обязательно);}
			\item Представляем аргумент в виде суммы/разности так, чтобы одно слагаемое было кратно \( 90 \), а другое было табличным значением (\( 30\degree;\;45\degree;\;60\degree \));
			\item Определяем четверть аргумента \textit{(меньшее слагаемое всегда принимаем за острый угол);}
			\item Определяем знак функции в этой четверти;
			\item Меняем или оставляем название тригонометрической функции (\( 0\degree;\;180\degree \) --- не меняем название функции; \( 90\degree;\;270\degree \) --- меняем название функции на противоположное).
		\end{enumcols}
		\item Вычислить с помощью метода приведения:
		\[ \sin135\degree;\;\cos240\degree;\;\sin390\degree;\;\tg150\degree;\;\ctg220\degree;\;\sin(-220\degree);\;\tg840\degree;\;\cos(-240\degree);\;\sin315\degree \]
	\end{listofex}
	\begin{definit}
		Радиан --- центральный угол, который опирается на дугу, равную радиусу данной окружности.
	\end{definit}
	\begin{definit}
		Число \( \pi \) --- отношение длины окружности к ее диаметру. Или иначе отношение половины длины окружности к ее радиусу.
	\end{definit}
	Таким образом можно сделать вывод, что в половине окружности радиус умещается \( \pi \) раз, а значит развернутый угол равен \( \pi \) радиан (т.е. \( \pi \) радиан \( = 180\degree \)).
	\begin{enumcols}
		\item \( 1 \) градус \( = \dfrac{\pi}{180} \) радиан;
		\item \( 1 \) радиан \( = \dfrac{180}{\pi}\) градусов (по факту всегда вместо \( \pi \) подставляем \( 180\degree \)).
	\end{enumcols}
	\begin{listofex}[resume]
		\item Перевести градусы в радианы:
		\begin{enumcols}[itemcolumns=5]
			\item \( 90\degree \)
			\item \( 120\degree \)
			\item \( 60\degree \)
			\item \( 45\degree \)
			\item \( 30\degree \)
			\item \( 210\degree \)
			\item \( 270\degree \)
			\item \( 360\degree \)
			\item \( 225\degree \)
			\item \( 330\degree \)
			\item \( 390\degree \)
			\item \( 150\degree \)
			\item \( 810\degree \)
			\item \( 210\degree \)
			\item \( 300\degree \)
		\end{enumcols}
		\newpage
		\item Перевести радианы в градусы:
		\begin{enumcols}[itemcolumns=5]
			\item \( \dfrac{\pi}{2} \)
			\item \( \dfrac{3\pi}{2} \)
			\item \( \dfrac{5\pi}{4} \)
			\item \( \dfrac{7\pi}{6} \)
			\item \( \dfrac{14\pi}{2} \)
			\item \( \dfrac{36\pi}{9} \)
			\item \( \dfrac{11\pi}{3} \)
			\item \( \dfrac{5\pi}{3} \)
			\item \( \dfrac{9\pi}{3} \)
			\item \( \dfrac{45\pi}{6} \)
			\item \( \dfrac{7\pi}{4} \)
			\item \( \dfrac{13\pi}{6} \)
			\item \( \dfrac{55\pi}{4} \)
			\item \( \dfrac{15\pi}{5} \)
			\item \( \dfrac{21\pi}{4} \)
		\end{enumcols}
		\item Вычислить с помощью метода приведения:
		\[ \cos\dfrac{5\pi}{4};\;\sin\dfrac{7\pi}{3};\;\sin\dfrac{3\pi}{2};\;\sin\left( -\dfrac{5\pi}{3} \right);\;\cos\dfrac{7\pi}{6};\;\sin\dfrac{13\pi}{4};\;\sin\left( -\dfrac{7\pi}{6}  \right);\;\cos\dfrac{21\pi}{4};\;\tg\dfrac{16\pi}{6};\;\ctg\dfrac{11\pi}{4} \]
	\end{listofex}
\end{class}
%
%===============>>  Занятие 4  <<===============
%
\begin{class}[number=4]
	\begin{listofex}
		\item \textbf{Формулы суммы/разности синуса или косинуса:}
		\begin{enumcols}[itemcolumns=2]
			\item \( \sin(x+y)=\sin x\cos y + \sin y \cos x \)
			\item \( \sin(x-y)=\sin x\cos y - \sin y \cos x \)
			\item \( \cos(x+y)=\cos x \cos y - \sin x \sin y \)
			\item \( \cos(x-y)=\cos x \cos y + \sin x \sin y \)
		\end{enumcols}
		\item Вычислить через формулы суммы/разности:
		\[ \sin150\degree;\;\cos135\degree;\;\sin225\degree;\;\cos(-120\degree);\;\cos330\degree;\;\tg(-150\degree);\;\sin(-225\degree);\;\cos300\degree;\;\sin(-315\degree) \]
		\item \textbf{Метод приведения аргумента тригонометрических функций:}
		\begin{enumcols}
			\item[0)] Выносим минус за знак аргумента;
			\item "Убираем"{ }полные круги из аргумента \textit{(в будущем не обязательно);}
			\item Представляем аргумент в виде суммы/разности так, чтобы одно слагаемое было кратно \( 90 \), а другое было табличным значением (\( 30\degree;\;45\degree;\;60\degree \));
			\item Определяем четверть аргумента \textit{(меньшее слагаемое всегда принимаем за острый угол);}
			\item Определяем знак функции в этой четверти;
			\item Меняем или оставляем название тригонометрической функции (\( 0\degree;\;180\degree \) --- не меняем название функции; \( 90\degree;\;270\degree \) --- меняем название функции на противоположное).
		\end{enumcols}
		\item Вычислить с помощью метода приведения:
		\[ \sin135\degree;\;\cos240\degree;\;\sin390\degree;\;\tg150\degree;\;\ctg220\degree;\;\sin(-220\degree);\;\tg840\degree;\;\cos(-240\degree);\;\sin315\degree \]
	\end{listofex}
	\begin{definit}
		Радиан --- центральный угол, который опирается на дугу, равную радиусу данной окружности.
	\end{definit}
	\begin{definit}
		Число \( \pi \) --- отношение длины окружности к ее диаметру. Или иначе отношение половины длины окружности к ее радиусу.
	\end{definit}
	Таким образом можно сделать вывод, что в половине окружности радиус умещается \( \pi \) раз, а значит развернутый угол равен \( \pi \) радиан (т.е. \( \pi \) радиан \( = 180\degree \)).
	\begin{enumcols}
		\item \( 1 \) градус \( = \dfrac{\pi}{180} \) радиан;
		\item \( 1 \) радиан \( = \dfrac{180}{\pi}\) градусов (по факту всегда вместо \( \pi \) подставляем \( 180\degree \)).
	\end{enumcols}
	\begin{listofex}[resume]
		\item Перевести градусы в радианы:
		\begin{enumcols}[itemcolumns=5]
			\item \( 90\degree \)
			\item \( 120\degree \)
			\item \( 60\degree \)
			\item \( 45\degree \)
			\item \( 30\degree \)
			\item \( 210\degree \)
			\item \( 270\degree \)
			\item \( 360\degree \)
			\item \( 225\degree \)
			\item \( 330\degree \)
			\item \( 390\degree \)
			\item \( 150\degree \)
			\item \( 810\degree \)
			\item \( 210\degree \)
			\item \( 300\degree \)
		\end{enumcols}
		\newpage
		\item Перевести радианы в градусы:
		\begin{enumcols}[itemcolumns=5]
			\item \( \dfrac{\pi}{2} \)
			\item \( \dfrac{3\pi}{2} \)
			\item \( \dfrac{5\pi}{4} \)
			\item \( \dfrac{7\pi}{6} \)
			\item \( \dfrac{14\pi}{2} \)
			\item \( \dfrac{36\pi}{9} \)
			\item \( \dfrac{11\pi}{3} \)
			\item \( \dfrac{5\pi}{3} \)
			\item \( \dfrac{9\pi}{3} \)
			\item \( \dfrac{45\pi}{6} \)
			\item \( \dfrac{7\pi}{4} \)
			\item \( \dfrac{13\pi}{6} \)
			\item \( \dfrac{55\pi}{4} \)
			\item \( \dfrac{15\pi}{5} \)
			\item \( \dfrac{21\pi}{4} \)
		\end{enumcols}
		\item Вычислить с помощью метода приведения:
		\[ \cos\dfrac{5\pi}{4};\;\sin\dfrac{7\pi}{3};\;\sin\dfrac{3\pi}{2};\;\sin\left( -\dfrac{5\pi}{3} \right);\;\cos\dfrac{7\pi}{6};\;\sin\dfrac{13\pi}{4};\;\sin\left( -\dfrac{7\pi}{6}  \right);\;\cos\dfrac{21\pi}{4};\;\tg\dfrac{16\pi}{6};\;\ctg\dfrac{11\pi}{4} \]
	\end{listofex}
\end{class}
%
%===============>>  Домашняя работа 2  <<===============
%
%\begin{homework}[number=2]
%	\begin{listofex}
%
%	\end{listofex}
%\end{homework}
%
%===============>>  Занятие 5  <<===============
%
\begin{class}[number=5]
	\begin{listofex}
		\item Вычислить:
		\begin{enumcols}[itemcolumns=2]
			\item \( 2\sin30\degree-\sqrt{3}\sin60\degree\ctg45\degree\tg30\degree\)
			\item \( \dfrac{6\sin30\degree\cos30\degree}{\cos^230\degree-\sin^230\degree} \)
			\item \( -2\cos(-90\degree)+3\sin(-270\degree) \)
			\item \( \dfrac{\sqrt{3}}{\sin60\degree}+\dfrac{3}{\sin30\degree} \)
			% Занятие 6 Галицкий стр. 187 13.1 б) 13.2 б) формат задач с Решу ЕГЭ
		\end{enumcols}
		\item Вычислить:
		\begin{enumcols}[itemcolumns=2]
			\item \( \dfrac{-13\sin126\degree}{\sin54\degree} \)
			\item \( \cos^2(-46\degree)+\sin^2(-46\degree) \)
			\item \( \sin^223\degree+9+\cos^223 \)
			\item \( \dfrac{2\sin^221\degree+2\cos^221\degree}{4} \)
		\end{enumcols}
		\item Вычислить:
		\begin{enumcols}[itemcolumns=2]
			\item \( \sin\dfrac{\pi}{3}\cos\dfrac{\pi}{4}\tg\dfrac{\pi}{6} \)
			\item \( \cos\left( -\dfrac{\pi}{4} \right)+\sqrt{3}\sin\left( -\dfrac{\pi}{6} \right) \)
			\item \( -\sin(-\pi)+0,5\cos\left( \dfrac{\pi}{2} \right) \)
			\item \( \sin\left( \dfrac{5\pi}{6} \right)+\cos\left( -\dfrac{2\pi}{3} \right) \)
			\item \( \tg(-3\pi)+\dfrac{1}{2}\sin\left( \dfrac{7\pi}{4} \right) \)
			\item \( \sin(-2\pi)+2\cos^2(-\pi)+\tg(\pi) \)
		\end{enumcols}
		\item Вычислить:
		\begin{enumcols}[itemcolumns=2]
			\item \( \sin225\degree\cos120\degree\tg330\degree\ctg240\degree \)
			\item \( \sin\dfrac{7\pi}{4}\cos\dfrac{7\pi}{6}\tg\dfrac{5\pi}{3}\ctg\dfrac{4\pi}{3} \)
			\item \( \sin(-300\degree)\cos(-135\degree)\tg(-210\degree) \)
			\item \( \cos\left( \dfrac{7\pi}{3} \right)\sin\left( -\dfrac{4\pi}{3} \right)\sin\dfrac{3\pi}{2} \)
			% Галицкий 13.31, 32
		\end{enumcols}
		\item Упростить выражение:
		\begin{enumcols}[itemcolumns=1]
			\item \( \tg\left( \dfrac{3\pi}{2}-x \right)\tg(\pi+x)-\cos\left( \dfrac{\pi}{2}+x \right)\sin(\pi+x) \)
			\item \( \cos(3\pi-x)+\ctg(3.5\pi-x)+\cos\left( \dfrac{3\pi}{2}+x \right)\ctg(\pi+x) \)
			\item \( \dfrac{\cos x}{1+\sin x}+\tg x \)
			% Галицкий 13.37, 38, 39
		\end{enumcols}
		\item \exercise{2965}
		\item \exercise{2874}
	\end{listofex}
\end{class}
%
%===============>>  Занятие 6  <<===============
%
\begin{class}[number=6]
	\begin{listofex}
		\item Вычислить:
		\begin{enumcols}[itemcolumns=2]
			\item \( \dfrac{\sqrt{3}}{\sin60\degree}+\dfrac{3}{\sin30\degree} \)
			\item \( \dfrac{17\sin155\degree}{\sin25\degree} \)
			\item \( \dfrac{-2\sin105\degree}{\cos15\degree} \)
			\item \( \sin^215\degree-1+\cos^215 \)
			\item \( -\sqrt{27}\cos30\degree-\sqrt{2}\sin45\degree\ctg60\degree\tg60\degree\)
			\item \( \dfrac{9\sin45\degree\cos45\degree}{\cos^245\degree-\sin^245\degree} \)
			% Занятие 6 Галицкий стр. 187 13.1 б) 13.2 б) формат задач с Решу ЕГЭ
		\end{enumcols}
		\item Вычислить:
		\begin{enumcols}[itemcolumns=2]
			\item \( \sin240\degree\sin150\degree\sin(-90)\degree\tg30\degree \)
			\item \( \cos(-300\degree)\sin(-120\degree)\tg(-150\degree) \)
			\item \( \sin\dfrac{5\pi}{4}\cos\dfrac{4\pi}{3}\tg\dfrac{2\pi}{3}\ctg\dfrac{3\pi}{4} \)
			\item \( \cos\left( -\dfrac{5\pi}{3} \right)\sin\left( -\dfrac{5\pi}{2} \right)\sin\dfrac{3\pi}{2} \)
			% Галицкий 13.31, 32
		\end{enumcols}
		\item Вычислить:
		\begin{enumcols}[itemcolumns=2]
			\item \( \sin\dfrac{\pi}{4}\cos\dfrac{\pi}{6}\tg\dfrac{\pi}{3} \)
			\item \( \cos\left( -\dfrac{\pi}{2} \right)+\sqrt{3}\sin\left( -\dfrac{\pi}{3} \right) \)
			\item \( \sin(-2\pi)+0,23\cos\left( \dfrac{3\pi}{2} \right) \)
			\item \( \sin\left( \dfrac{3\pi}{4} \right)+\cos\left( -\dfrac{5\pi}{6} \right) \)
			\item \( \ctg\left( \dfrac{3\pi}{2} \right)+\dfrac{1}{\sqrt{2}}\sin\left( \dfrac{5\pi}{4} \right) \)
			\item \( \sin(-2,5\pi)-(3\cos(-\pi))^2 \)
		\end{enumcols}
		\item Упростить выражение:
		\begin{enumcols}[itemcolumns=1]
			\item \( \ctg\left( \dfrac{3\pi}{2}+x \right)\ctg(\pi-x)-\ctg\left( \dfrac{\pi}{2}+x \right)\tg(2\pi+x) \)
			\item \( \cos\left( \dfrac{3\pi}{2}+x \right)\sin x + \sin^2(3\pi+x)+\tg(5\pi+x)\ctg x \)
			\item \( \dfrac{\sin x}{1+\cos x}+\ctg x \)
			% Галицкий 13.37, 38, 39
		\end{enumcols}
		\item \exercise{1116}
		\item \exercise{2858}
	\end{listofex}
\end{class}
\begin{consultation}
	\begin{listofex}
		\item Найдите область определения функции:
		\begin{enumcols}[itemcolumns=3]
			\item \( y=\dfrac{x-3}{x^2-8x+15} \)
			\item \( y=\sqrt{x^2+6x-27} \)
			\item \( y=\dfrac{1-\sqrt{-x^2-10x+11}}{1+\sqrt{x-4}} \)
		\end{enumcols}
		\item Найдите область значений функции:
		\begin{enumcols}[itemcolumns=2]
			\item \( y=\dfrac{1}{5}x-4 \)
			\item \( y=3x^2+4 \)
			\item \( y=2x^2-6x+4,\; x\in[-3;4] \)
			\item \( y=7-\dfrac{5}{x} \)
			\item \( y=\dfrac{x-4}{x+4} \)
		\end{enumcols}
		\item Найдите промежутки монотонности:
		\begin{enumcols}[itemcolumns=3]
			\item \( y=x^2-11x+28 \)
			\item \( y=(x+5)^2-1 \)
			\item \( y=|x+1|-1 \)
		\end{enumcols}
		\item Пусть функция \( y=f(x) \) определена и возрастает на \( R \). Решите уравнение: \[ f\left( \dfrac{12}{x} \right)=f\left( 3+\dfrac{5-x}{x+1} \right) \]
		\item Найдите область определения функции и исследуйте ее на четность и нечетность: \[ y=\dfrac{x^3}{1+x}-\dfrac{x^3}{1-x} \]
		\item Являются ли функции \( y=f(x) \) и \( y=g(x) \) взаимно обратными, если \( f(x)=5x-1 \) и \( g(x)=\dfrac{1}{5}x+\dfrac{2}{5} \)? Если нет, то какая функция будет обратной для \( f(x) \)?
		\item Найдите функцию, обратную \( y=\dfrac{7}{3x-4} \).
	\end{listofex}
\end{consultation}
%
%===============>>  Домашняя работа 3  <<===============
%
%\begin{homework}[number=3]
%	\begin{listofex}
%
%	\end{listofex}
%\end{homework}
\newpage
%===============>>  Подготовка к проверочной работе  <<===============
\title{Подготовка к проверочной работе}
\begin{listofex}
	\item Перевести радианы в градусы:
	\begin{enumcols}[itemcolumns=4]
		\item \( \dfrac{\pi}{4} \)
		\item \( \dfrac{3\pi}{2} \)
		\item \( \dfrac{25\pi}{6} \)
		\item \( \dfrac{38\pi}{4} \)
	\end{enumcols}
	\item Перевести градусы в радианы:
	\begin{enumcols}[itemcolumns=4]
		\item \( 45\degree \)
		\item \( 22,5\degree \)
		\item \( 15\degree \)
		\item \( 165\degree \)
	\end{enumcols}
	\item Вычислить через формулы суммы/разности:
	\begin{enumcols}[itemcolumns=4]
		\item \( \sin150\degree \)
		\item \( \cos135\degree \)
		\item \( \sin(-225\degree) \)
		\item \( \cos300\degree \)
	\end{enumcols}
	\item Вычислить:
	\begin{enumcols}[itemcolumns=2]
		\item \( \cos240\degree\sin210\degree\cos(-150)\degree\tg30\degree \)
		\item \( \sin\dfrac{3\pi}{2}\cos\dfrac{16\pi}{4}\tg\dfrac{3\pi}{4}\ctg\dfrac{1\pi}{2} \)
	\end{enumcols}
	\item Вычислить:
	\begin{enumcols}[itemcolumns=2]
		\item \exercise{1118}
		\item \exercise{1117}
	\end{enumcols}
	\item Вычислить:
	\begin{enumcols}[itemcolumns=2]
		\item \(\dfrac{13!}{11!}\)
		\item \(\dfrac{1}{2} \cdot \dfrac{22!}{20! \cdot 7}\)
	\end{enumcols}
	\item Вычислить:
	\begin{enumcols}[itemcolumns=2]
		\item \exercise{1609}
		\item \exercise{1608}
	\end{enumcols}
	\item В треугольнике \(ABC\) \(CH\) - высота, \(AD\) - биссектриса, \(O\) - точка пересечения прямых \(CH\) и \(AD\), угол \(BAD = 26\degree\). Найдите угол \(AOC\). Ответ дайте в градусах.
	\item В треугольнике \(ABC\) \( AC = BC \), \(AB=9,6\), \(\sin A =\dfrac{7}{25}\). Найдите AC.
\end{listofex}
%===============>>  Занятие 7  <<===============
%
%\begin{class}[number=7]
%	\begin{listofex}
%	
%	\end{listofex}
%\end{class}
%
%===============>>  Провечная работа  <<===============
%
\newpage
\title{Проверочная работа}
\begin{listofex}
	\item Перевести радианы в градусы:
	\begin{enumcols}[itemcolumns=4]
		\item \( \dfrac{3\pi}{4} \)
		\item \( \dfrac{7\pi}{2} \)
		\item \( \dfrac{11\pi}{6} \)
		\item \( \dfrac{42\pi}{4} \)
	\end{enumcols}
	\item Перевести градусы в радианы:
	\begin{enumcols}[itemcolumns=4]
		\item \( 30\degree \)
		\item \( 11\degree \)
		\item \( 22,5\degree \)
		\item \( 225\degree \)
	\end{enumcols}
	\item Вычислить через формулы суммы/разности:
	\begin{enumcols}[itemcolumns=4]
		\item \( \sin75\degree \)
		\item \( \sin315\degree \)
		\item \( \cos(-585\degree) \)
		\item \( \sin300\degree \)
	\end{enumcols}
	\item Вычислить:
	\begin{enumcols}[itemcolumns=2]
		\item \( \sin150\degree\sin(-120\degree)\ctg(-225\degree) \)
		\item \( \tg\left( -\dfrac{17\pi}{4} \right)\sin\dfrac{14\pi}{3} \cos\left( -\dfrac{25\pi}{4}\right)\ \)
	\end{enumcols}
	\item Вычислить:
	\begin{enumcols}[itemcolumns=1]
		\item \exercise{1117}
		\item \exercise{1119}
	\end{enumcols}
	\item Вычислить:
	\begin{enumcols}[itemcolumns=2]
		\item \(\dfrac{14!}{10!\cdot 42}\)
		\item \(\dfrac{1}{3} \cdot \dfrac{22!}{20! \cdot 7}\)
	\end{enumcols}
	\item Вычислить:
	\begin{enumcols}[itemcolumns=2]
		\item \exercise{1610}
		\item \exercise{1606}
	\end{enumcols}
	\item В треугольнике \(ABC\) \(AС=BC=8\), \(\cos A = 0,5\). Найдите AB.
	\item В треугольнике \(ABC\) проведена биссектриса \(AD\) и \(AB = AD = CD\) Найдите меньший угол треугольника \(ABC\). Ответ дайте в градусах.
\end{listofex}