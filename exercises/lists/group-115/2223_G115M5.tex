%
%===============>>  ГРУППА 11-5 МОДУЛЬ 5  <<=============
%
\setmodule{5}

%BEGIN_FOLD % ====>>_____ Занятие 1 _____<<====
\begin{class}[number=1]
	\begin{listofex}
		\item Найдите значение выражения:
		\begin{tasks}(1)
			\task \( (\log_216)\cdot(\log_636) \)
			\task \( 7\cdot5^{\log_54} \)
			\task \( 36^{\log_65} \)
			\task \( \log_{0,25}2 \)
			\task \( \log_48 \)
			\task \( \log_560-\log_512 \)
			\task \( \log_50,2+\log_{0,5}4 \)
			\task \( \log_{0,3}10-\log_{0,3}3 \)
			\task \( \dfrac{\log_325}{\log_35} \)
			\task \( \dfrac{\log_{7}13}{\log_{49}13} \)
			\task \( \log_59\cdot\log_325 \)
			\task \( \dfrac{9^{\log_550}}{9^{\log_52}} \)
			\task \( 6\log_4\sqrt[3]{7} \)
			\task \( \log_{\sqrt[6]{13}}13 \)
			\task \( \dfrac{\log_318}{2+\log_32} \)
			\task \( \dfrac{\log_35}{\log_37}+\log_70,2 \)
			\task \( \log_{0,8}3\log_31,25 \)
			\task \( 5^{\log_{25}49} \)
			\task \( \log^2_{\sqrt{7}}49 \)
			\task \( 5^{3+\log_52} \)
			\task \( 8^{2+\log_83} \)
			\task \( 64^{\log_8\sqrt{3}} \)
			\task \( \log_4\log_525 \)
			\task \( \dfrac{24}{3^{\log_32}} \)
			\task \( \log_{\dfrac{1}{13}}\sqrt{13} \)
			\task \( \log_38,1+\log_310 \)
			\task \( \dfrac{\log_6\sqrt{13}}{\log_613} \)
			\task \( \left( 3^{\log_23} \right)^{\log_32}\)
			\task \( \log_57\cdot\log_725 \)
			\task \( \dfrac{\log_212,8-\log_20,8}{5^{\log_{25}16}} \)
			\task \( \dfrac{\log_23,2-\log_20,2}{3^{\log_925}} \)
			\task \( 3^{\log_37}+49^{\log_7{\sqrt{13}}} \)
		\end{tasks}	
		\item \mexercise{_78}
		\item Решите уравнение:
		\begin{tasks}(2)
			\task \( \log_2(4-x)=7 \)
			\task \( \log_5(5-x)=\log_53 \)
			\task \( \log_4(x+3)=\log_4(4x-15) \)
			\task \( \log_{\frac{1}{7}}(7-x)=-2 \)
			\task \( \log_5(5-x)=2\log_53 \)
			\task \( \log_5(x^2+2x)=\log_5(x^2+10) \)
			\task \( \log_{x-5}49=2 \)
			\task \( \log_82^{8x-4}=4 \)
			\task \( 2^{\log_8(5x-3)}=4 \)
			\task \( \log_x32=5 \)
		\end{tasks}
	\end{listofex}
\end{class}
%END_FOLD

%BEGIN_FOLD % ====>>_ Домашняя работа 1 _<<====
\begin{homework}[number=1]
		\begin{listofex}
			\item Найти значения выражения:
			\begin{tasks}(2)
				\task \( 2^{\log_26-3} \)
				\task \( 7\cdot5^{\log_54} \)
				\task \( 6^{5\log_63} \)
				\task \( \log_2112-\log_27 \)
				\task \( \log_340,5+\log_36 \)
				\task \( 6^{3\log_62} \)
				\task \( \dfrac{\log_318}{2+\log_32} \)
				\task \( (\log_381)\cdot(\log_6216) \)
			\end{tasks}
			\item В треугольнике \( ABC \) \( AC=BC=20 \), \( AB=32  \). Найдите  \( \sin A \).
			\item В треугольнике \( ABC \) \( AB=BC=24  \) внешний угол при вершине \( C  \) равен \( 150\degree  \). Найдите длину медианы \( BM \).
		\end{listofex}
\end{homework}
%END_FOLD

%BEGIN_FOLD % ====>>_____ Занятие 2 _____<<====
\begin{class}[number=2]
	\begin{listofex}
		\item Вычислить:
			\begin{itasks}[2]
			\task \exercise{1572}
			\task \exercise{1565}
			\task \exercise{1566}
			\task \exercise{1573}
			\task \exercise{1567}
			\task \exercise{1575}
			\task \exercise{1594}
		\end{itasks}
		\item Вычислить:
		\begin{itasks}[1]
			\task \exercise{1569}
			\task \exercise{1570}
			\task \exercise{1571}
			\task \exercise{1574}
		\end{itasks}
		\item \mexercise{_78}
		\item Решите уравнение:
		\begin{tasks}(2)
			\task \( \log_2(4-x)=7 \)
			\task \( \log_5(5-x)=\log_53 \)
			\task \( \log_4(x+3)=\log_4(4x-15) \)
			\task \( \log_{\frac{1}{7}}(7-x)=-2 \)
			\task \( \log_5(5-x)=2\log_53 \)
			\task \( \log_5(x^2+2x)=\log_5(x^2+10) \)
			\task \( \log_{x-5}49=2 \)
			\task \( \log_82^{8x-4}=4 \)
			\task \( 2^{\log_8(5x-3)}=4 \)
			\task \( \log_x32=5 \)
		\end{tasks}
	
	\end{listofex}
\end{class}
%END_FOLD

%BEGIN_FOLD % ====>>_ Домашняя работа 2 _<<====
\begin{homework}[number=2]
	\begin{listofex}
		\item Вычислить:
		\begin{tasks}(2)
			\task \( 7^{-2\log_72} \)
			\task \( \log_60.8+\log_645 \)
			\task \( \log_354-\log_32 \)
			\task \( \log_x4=2 \)
		\end{tasks}
		\item Решите уравнение:
		\begin{tasks}(2)
			\task \( \log_2(11-x)=4\log_25 \)
			\task \( \log_7(x+5)=\log_7(4x-7) \)
			\task \( \log_4(x+2)+\log_43=\log_415 \)
			\task \( \log_{x-1}81=2 \)
		\end{tasks}
		\item Выпускники \( 11 \) "А" покупают букеты цветов для последнего звонка: из \( 3 \) роз каждому учителю и из \( 9 \) роз классному руководителю и директору. Они собираются подарить букеты \( 15 \) учителям (включая директора и классного руководителя), розы покупаются по оптовой цене \( 30 \) рублей за штуку. Сколько рублей стоят все розы?
	\end{listofex}
\end{homework}
%END_FOLD

%BEGIN_FOLD % ====>>_____ Занятие 3 _____<<====
\begin{class}[number=3]
	\begin{listofex}
		\item Занятие 3
	\end{listofex}
\end{class}
%END_FOLD

%BEGIN_FOLD % ====>>_____ Занятие 4 _____<<====
\begin{class}[number=4]
	\begin{listofex}
		\item Пусто
	\end{listofex}
\end{class}
%END_FOLD


%BEGIN_FOLD % ====>>_ Проверочная работа _<<====
\begin{exam}
	\begin{listofex}
		\item Проверочная
	\end{listofex}
\end{exam}
%END_FOLD