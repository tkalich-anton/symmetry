%
%===============>>  ГРУППА 11-6 МОДУЛЬ 8  <<=============
%
\setmodule{Вспомнить всё}

%BEGIN_FOLD % ====>>_____ Занятие 1 _____<<====
\begin{class}[number=1]
	\begin{listofex}
		\item Упростите выражение:
		\begin{tasks}(2)
			\task \( \dfrac{18^{n+3}}{3^{2n+5}\cdot2^{n-2}} \)
			\task \( \dfrac{100^n}{5^{2n-1}\cdot4^{n-2}} \)
		\end{tasks}
		\item Сократите дробь:
			\[\dfrac{x^3+2x^2-9x-18}{(x-3)(x+2)}\]
		\item Решите уравнения:
		\begin{tasks}(2)
			\task \( (x^2-16)+(x^2+x-12)^2=0 \)
			\task \( \dfrac{1}{(x-2)^2} +\dfrac{4}{(x-1)}-12=0\)
			\task \( x(x^2+2x+1)=2(x+1) \)
		\end{tasks}
		\item Решите неравенства:
		\begin{tasks}(2)
			\task \( \dfrac{11x-4}{5}\ge\dfrac{x^2}{2} \)
			\task \( (\sqrt{3}-1,5)(3-2x)>0 \)
			\task \( (x-7)^2<\sqrt{11}(x-7) \)
		\end{tasks}
		\item Решите систему неравенств:
		\[ \begin{cases} x^2+9x+8\le0 \\ -0,3x\ge2,4 \end{cases}\]
		\item Решите неравенство:
		\[|x^2-5x|<6\]
		\item 
		\begin{tasks}(1)
			\task Решите уравнение: \( 	\dfrac{(x^2-x-12)^2}{x+\sqrt{13}}=\dfrac{(2x^2+x-27)^2}{x+\sqrt{13}} \)
			\task Укажите корни этого уравнения, принадлежащие отрезку \\ \( [\sqrt{15}-1; \sqrt{17}-1] \).
		\end{tasks}
%		\item 
%		\begin{tasks}(1)
%			\task Решите уравнение: \( x-3\sqrt{x-1}+1=0 \)
%			\task Укажите корни этого уравнения, принадлежащие отрезку \( [\sqrt{3};\sqrt{20}] \).
%		\end{tasks}
%		\item 
%		\begin{tasks}(1)
%			\task Решите уравнение: \( \sqrt{x^3-4x^2-10x+29}=3-x \)
%			\task Укажите корни этого уравнения, принадлежащие отрезку \( [-\sqrt{3};\sqrt{30}] \).
%		\end{tasks}
	\end{listofex}
\end{class}
%END_FOLD

%BEGIN_FOLD % ====>>_ Домашняя работа 1 _<<====
\begin{homework}[number=1]
		\begin{listofex}
			\item Домашняя работа
		\end{listofex}
\end{homework}
%END_FOLD

%BEGIN_FOLD % ====>>_____ Занятие 2 _____<<====
\begin{class}[number=2]
	\begin{listofex}
		\item У треугольника со сторонами \( 9 \) и \( 6 \) проведены высоты к этим сторонам. Высота, проведенная к первой стороне, равна \( 4 \). Чему равна высота, проведенная ко второй стороне?
		\item Угол при вершине, противолежащей основанию равнобедренного треугольника, равен \( 30\degree \). Боковая сторона треугольника равна \( 10 \). Найдите площадь этого треугольника.
		\item Углы треугольника относятся как \( 2:3:4 \). Найдите меньший из них. Ответ дайте в градусах.
		\item В треугольнике \( ABC \) угол \( C \) равен \( 90\degree \), \( AC=4,8 \),  \( \sin A=\dfrac{7}{25} \).  Найдите \( AB \).
		\item В треугольнике \( ABC \) \( AC=BC=8 \),  \( \cos A=0,5 \). Найдите \( AB \).
		\item В треугольнике \( ABC \) угол \( B \) равен \( 45\degree \), угол \( C \) равен \( 85\degree \), \( AD \) --- биссектриса, \( E \) --- такая точка на \( AB \), что \( AE=AC \). Найдите угол \( BDE \). Ответ дайте в градусах.
		\item Найдите периметр прямоугольника, если его площадь на \( 4 \) меньше периметра, а отношение соседних сторон равно \( 1:4 \).
		\item Сумма двух углов параллелограмма равна \( 100\degree \). Найдите один из оставшихся углов. Ответ дайте в градусах.
		\item Стороны параллелограмма равны \( 9 \) и \( 15 \). Высота, опущенная на первую сторону, равна \( 10 \). Найдите высоту, опущенную на вторую сторону параллелограмма.
		\item Найдите площадь ромба, если его высота равна \( 2 \), а острый угол \( 30\degree \).
		\item Диагональ параллелограмма образует с двумя его сторонами углы \( 26\degree \) и \( 34\degree \). Найдите больший угол параллелограмма. Ответ дайте в градусах.
		\item Основания равнобедренной трапеции равны \( 51 \) и \( 65 \). Боковые стороны равны \( 25 \). Найдите синус острого угла трапеции.
		\item Основания равнобедренной трапеции равны \( 14 \) и \( 26 \), а ее периметр равен \( 60 \). Найдите площадь трапеции.
		\item Найдите площадь прямоугольной трапеции, основания которой равны \( 6 \) и \( 2 \), большая боковая сторона составляет с основанием угол \( 45\degree \).
		\item Найдите вписанный угол, опирающийся на дугу, которая составляет \( \dfrac{1}{5} \) окружности. Ответ дайте в градусах.
		\item Четырёхугольник \( ABCD \) вписан в окружность. Угол \( ABD \) равен \( 61\degree \), угол \( CAD \) равен \( 37\degree \). Найдите угол \( ABC \). Ответ дайте в градусах.
		\item Найдите хорду, на которую опирается угол \( 30\degree \), вписанный в окружность радиуса \( 3 \).
		\item Угол \( ACO \) равен \( 28\degree \), где \( O \) --- центр окружности. Его сторона \( CA \) касается окружности. Найдите величину меньшей дуги \( AB \) окружности, заключенной внутри этого угла. Ответ дайте в градусах.
		\item Угол \( ACO \) равен \( 24\degree \). Его сторона \( CA \) касается окружности с центром в точке \( O \). Сторона \( CO \) пересекает окружность в точках \( B \) и \( D \). Найдите градусную меру дуги \( AD \) окружности, заключенной внутри этого угла. Ответ дайте в градусах.
	\end{listofex}
\end{class}
%END_FOLD

%BEGIN_FOLD % ====>>_ Домашняя работа 2 _<<====
\begin{homework}[number=2]
	\begin{listofex}
		\item Домашняя работа
	\end{listofex}
\end{homework}
%END_FOLD

%BEGIN_FOLD % ====>>_____ Занятие 3 _____<<====
\begin{class}[number=3]
	\begin{listofex}
		\item Занятие 3
	\end{listofex}
\end{class}
%END_FOLD

%BEGIN_FOLD % ====>>_ Домашняя работа 3 _<<====
\begin{homework}[number=3]
	\begin{listofex}
		\item Домашняя работа
	\end{listofex}
\end{homework}
%END_FOLD

%BEGIN_FOLD % ====>>_____ Занятие 4 _____<<====
\begin{class}[number=4]
	\begin{listofex}
		\item Пусто
	\end{listofex}
\end{class}
%END_FOLD