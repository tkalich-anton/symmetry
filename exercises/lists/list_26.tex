%Группа 92 Модуль 8 ЗДомашняя работа №1
\begin{enumcols}[label=\textbf{\arabic*.}]
	\item \exercise{757}
	\item \exercise{758}
	\item \exercise{759}
	\item \exercise{760}
	\item \exercise{761}
	\item Острый угол прямоугольного треугольника равен \( 30\degree \). Докажите, что высота и медиана, проведенные из вершины прямого угла, делят его на три равные части.
	\item Две различные окружности пересекаются в точках \( A \) и  \( B \). Докажите, что прямая, проходящая через центры окружностей, делит отрезок \( AB \) пополам и перпендикулярна ему.
	\item Две окружности пересекаются в точках \( A \) и \( B \), \( AM \) и \( AN \) -- диаметры окружностей. Докажите, что точки \( M \), \( N \), \( B \) лежат на одной прямой.
	\item Точки \( A \) и \( B \) лежат на окружности. Касательные к окружности, проведенные через эти точки, пересекаются в точке \( C \). Найдите углы треугольника \( ABC \), если \( AB=AC \).
	\item На катете \( AC \) прямоугольного треугольника \( ABC \) как на диаметре построена окружность, пересекающая гипотенузу \( AB \) в точке \( K \). Найдите \( CK \), если \( AC=2 \) и \( \angle A = 30\degree \).
	\item Окружность, построенная на стороне треугольника как на диаметре, проходит через середину другой стороны. Докажите, что треугольник равнобедренный.
	\item Окружность, построенная на биссектрисе \( AD \) треугольника \( ABC \) как на диаметре, пересекает стороны \( AB \) и \( AC \) соответственно в точках \( M \) и \( N \), отличных от \( A \). Докажите, что \( AM = AN \).
\end{enumcols}