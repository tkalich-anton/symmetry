%Группа 101-1 Модуль 2
\title{Занятие №1}
\begin{listofex}
	\item \exercise{2350}
	\item \exercise{2095}
	\item \exercise{2390}
	\item \exercise{2380}
	\item \exercise{2385}
	\item \exercise{2381}
	\item \exercise{2393}
	\item \exercise{2559}
	\item \exercise{2040}
	\item \exercise{2424}
\end{listofex}
\newpage
\title{Занятие №2}
\begin{listofex}
	\item Упростить выражение:
	\begin{enumcols}[itemcolumns=2]
		\item \exercise{750}
		\item \exercise{1500}
	\end{enumcols}
	\item Докажите, что если медиана равна половине стороны, к которой она проведена, то треугольник прямоугольный.
	\item \exercise{2415}
	\item Докажите, что если треугольник вписан в окружность и одна из его сторон является диаметром этой окружности, то такой треугольник является прямоугольным.
	\item Докажите обратное, что если треугольник прямоугольный и вписан в окружность, то гипотенуза будет являться диаметром окружности.
	\item \exercise{2455}
	\item \exercise{2418}
	\item \exercise{2422}
\end{listofex}
\newpage
\title{Домашняя работа №1}
	\begin{listofex}
	\item Упростить выражение:
	\begin{enumcols}[itemcolumns=2]
		\item \exercise{748}
		\item \exercise{1500}
	\end{enumcols}
	\item Вычислить:
	\begin{enumcols}[itemcolumns=2]
		\item \exercise{1649}
		\item \exercise{1682}
	\end{enumcols}
	\item Докажите, что в равных треугольниках соответствующие биссектрисы равны.
	\item \exercise{2423}
	\item \exercise{2456}
	\item \exercise{2412}
\end{listofex}
%\newpage
%\title{Занятие №3}
%\begin{listofex}
%\item \exercise{2424}
%\item \exercise{2420}
%\end{listofex}
%\newpage
%\title{Занятие №4}
%\begin{listofex}
%
%\end{listofex}
%\newpage
%\title{Домашняя работа №2}
%\begin{listofex}
%
%\end{listofex}
%\newpage
%\title{Занятие №5}
%\begin{listofex}
%
%\end{listofex}
%\newpage
%\title{Занятие №6}
%\begin{listofex}
%
%\end{listofex}
%\newpage
%\title{Занятие №7}
%\begin{listofex}
%
%\end{listofex}
%\newpage
%\title{Проверочная работа}
%\begin{listofex}
%
%\end{listofex}