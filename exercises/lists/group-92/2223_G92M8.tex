%
%===============>>  ГРУППА 9-2 МОДУЛЬ 8  <<=============
%
\setmodule{8}

%BEGIN_FOLD % ====>>_____ Занятие 1 _____<<====
\begin{class}[number=1]
	\begin{listofex}
		\item Первый сплав содержит \( 5\% \) меди, второй --- \( 11\% \) меди. Масса второго сплава больше массы первого на \( 4 \) кг. Из этих двух сплавов получили третий сплав, содержащий \( 10\% \) меди. Найдите массу третьего сплава.
		\item Смешали некоторое количество \( 19 \)-процентного раствора некоторого вещества с таким же количеством \( 23 \)-процентного раствора этого же вещества. Сколько процентов составляет концентрация получившегося раствора?
		\item Смешали некоторое количество \( 82 \)-процентного раствора некоторого вещества с таким же количеством \( 94 \)-процентного раствора этого же вещества. Сколько процентов составляет концентрация получившегося раствора?
		\item Имеются два сосуда, содержащие \( 24 \) кг и \( 26 \) кг раствора кислоты различной концентрации. Если их слить вместе, то получится раствор, содержащий \( 39\% \) кислоты. Если же слить равные массы этих растворов, то полученный раствор будет содержать \( 40\% \) кислоты. Сколько килограммов кислоты содержится в первом растворе?
		\item Имеются два сосуда, содержащие \( 30 \) кг и \( 42 \) кг раствора кислоты различной концентрации. Если их слить вместе, то получим раствор, содержащий \( 40\% \) кислоты. Если же слить равные массы этих растворов, то полученный раствор будет содержать \( 37\% \) кислоты. Сколько килограммов кислоты содержится во втором растворе?
		\item Свежие фрукты содержат \( 93\% \) воды, а высушенные --- \( 16\% \). Сколько сухих фруктов получится из \( 252 \) кг свежих фруктов?
		\item Свежие фрукты содержат \( 75\% \) воды, а высушенные --- \( 25\% \). Сколько сухих фруктов получится из \( 135 \) кг свежих фруктов?
		\item Имеется два сплава с разным содержанием меди: в первом содержится \( 60\% \), а во втором --- \( 45\% \) меди. В каком отношении надо взять первый и второй сплавы, чтобы получить из них новый сплав, содержащий \( 55\% \) меди?
		\item При смешивании первого раствора кислоты, концентрация которого \( 30\% \), и второго раствора этой же кислоты, концентрация которого \( 50\% \), получили раствор, содержащий \( 45\% \) кислоты. В каком отношении были взяты первый и второй растворы?
	\end{listofex}
\end{class}
%END_FOLD

%BEGIN_FOLD % ====>>_____ Занятие 2 _____<<====
\begin{class}[number=2]
	\begin{listofex}
		\item Занятие 2
	\end{listofex}
\end{class}
%END_FOLD

%BEGIN_FOLD % ====>>_ Домашняя работа 1 _<<====
\begin{homework}[number=1]
	\begin{listofex}
		\item Домашняя работа 1
	\end{listofex}
\end{homework}
%END_FOLD

%BEGIN_FOLD % ====>>_____ Занятие 3 _____<<====
\begin{class}[number=3]
	\begin{listofex}
		\item Занятие 3 
	\end{listofex}
\end{class}
%END_FOLD

%BEGIN_FOLD % ====>>_____ Занятие 4 _____<<====
\begin{class}[number=4]
	\begin{listofex}
		\item Занятие 4
	\end{listofex}
\end{class}
%END_FOLD

%BEGIN_FOLD % ====>>_ Домашняя работа 2 _<<====
\begin{homework}[number=2]
	\begin{listofex}
		\item Домашняя работа 2
	\end{listofex}
\end{homework}
%END_FOLD

%BEGIN_FOLD % ====>>_____ Занятие 5 _____<<====
\begin{class}[number=5]
	\begin{listofex}
		\item Занятие 5
	\end{listofex}
\end{class}
%END_FOLD

%BEGIN_FOLD % ====>>_____ Занятие 6 _____<<====
\begin{class}[number=6]
	\begin{listofex}
		\item Занятие 6
	\end{listofex}
\end{class}
%END_FOLD

%BEGIN_FOLD % ====>>_ Домашняя работа 3 _<<====
\begin{homework}[number=3]
	\begin{listofex}
		\item Домашняя работа 3
	\end{listofex}
\end{homework}
%END_FOLD

%BEGIN_FOLD % ====>>_____ Занятие 7 _____<<====
\begin{class}[number=7]
	\title{Подготовка к проверочной}
	\begin{listofex}
		\item Занятие 7
	\end{listofex}
\end{class}
%END_FOLD

=%BEGIN_FOLD % ====>>_ Проверочная работа _<<====
\begin{exam}
	\begin{listofex}
		\item Проверочная
	\end{listofex}
\end{exam}
%END_FOLD