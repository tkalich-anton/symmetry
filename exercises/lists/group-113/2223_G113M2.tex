%Группа 11-3 Модуль 2
\title{Занятие №1}
\begin{listofex}
	\item \exercise{1417}
	\item \exercise{1419}
	\item Вычислить:
	\begin{enumcols}[itemcolumns=3]
		\item \exercise{562}
		\item \exercise{564}
		\item \exercise{569}
		\item \exercise{571}
		\item \exercise{579}
	\end{enumcols}
	\item Вычислить:
	\begin{enumcols}[itemcolumns=3]
		\item \exercise{1577}
		\item \exercise{1578}
		\item \exercise{1579}
	\end{enumcols}
	\item Вычислить:
	\begin{enumcols}[itemcolumns=3]
		\item \exercise{1572}
		\item \exercise{1565}
		\item \exercise{1566}
		\item \exercise{1573}
		\item \exercise{1567}
		\item \exercise{1575}
		\item \exercise{1594}
	\end{enumcols}
	\item Вычислить:
	\begin{enumcols}[itemcolumns=2]
		\item \exercise{1569}
		\item \exercise{1570}
		\item \exercise{1571}
		\item \exercise{1574}
	\end{enumcols}
	\item Решить уравнение:
	\begin{enumcols}[itemcolumns=2]
		\item \( \log_2(4-x)=7 \)
		\item \( \log_{1/7}(7-2x)=-2 \)
		\item \( \log_4(x+3)=\log_4(4x-15) \)
		\item \( \log_5(7-x)=\log_5(3-x)+1 \)
		\item \( \log_8 2^{8x-4}=4 \)
		\item \( \log_5(x^2+13x)=\log_5(9x+5) \)
	\end{enumcols}
\end{listofex}
\newpage
\title{Занятие №2}
\begin{listofex}
	\item \exercise{1115}
	\item Упростить выражение:
	\begin{enumcols}[itemcolumns=2]
		\item \exercise{750}
		\item \exercise{1500}
	\end{enumcols}
	\item Вычислить:
	\begin{enumcols}[itemcolumns=3]
		\item \exercise{563}
		\item \exercise{569}
		\item \exercise{581}
		\item \exercise{1589}
		\item \exercise{1595}
	\end{enumcols}
	\item Вычислить:
	\begin{enumcols}[itemcolumns=2]
		\item \exercise{582}
		\item \exercise{584}
		\item \exercise{591}
		\item \exercise{594}
		\item \exercise{1596}
	\end{enumcols}
	\item Вычислить:
	\begin{enumcols}[itemcolumns=2]
		\item \exercise{592}
		\item \exercise{1597}
	\end{enumcols}
	\item Вычислить:
	\begin{enumcols}[itemcolumns=3]
		\item \exercise{1578}
		\item \exercise{1580}
		\item \exercise{1581}
	\end{enumcols}
\end{listofex}
\newpage
\title{Домашняя работа №1}
\begin{listofex}
	\item Упростить выражение:
	\begin{enumcols}[itemcolumns=2]
		\item \exercise{748}
		\item \exercise{1500}
	\end{enumcols}
	\item Вычислить:
	\begin{enumcols}[itemcolumns=2]
		\item \exercise{1649}
		\item \exercise{1682}
	\end{enumcols}
	\item Вычислить:
	\begin{enumcols}[itemcolumns=4]
		\item \exercise{588}
		\item \exercise{1590}
		\item \exercise{1293}
		\item \exercise{567}
		\item \exercise{580}
		\item \exercise{570}
		\item \exercise{573}
		\item \exercise{572}
	\end{enumcols}
	\item Вычислить:
	\begin{enumcols}[itemcolumns=2]
		\item \exercise{582}
		\item \exercise{590}
		\item \exercise{595}
		\item \exercise{1587}
	\end{enumcols}
	\item Вычислить:
	\begin{enumcols}[itemcolumns=3]
		\item \exercise{1573}
		\item \exercise{1588}
		\item \exercise{586}
	\end{enumcols}
	\item Вычислить:
	\begin{enumcols}[itemcolumns=3]
		\item \exercise{1583}
		\item \exercise{595}
		\item \exercise{1585}
	\end{enumcols}
	\item Решить уравнение:
	\begin{enumcols}[itemcolumns=2]
		\item \exercise{506}
		\item \exercise{3653}
		\item \exercise{1164}
		\item \exercise{3529}
		\item \exercise{606}
		\item \exercise{613}
		\item \exercise{601}
	\end{enumcols}
\end{listofex}
\newpage
\title{Занятие №3}
\begin{listofex}
	\item \exercise{4156}
	\item Упростить и найти значение выражения: \( \dfrac{\left( c^{\frac{5}{3}} \right)^3\cdot c^{\frac{1}{2}}}{c^{\frac{3}{2}}} \),\quad при \( c=\dfrac{3}{2} \) \answer{ \( 5,0625 \) }
	\item Вычислить:
	\begin{enumcols}[itemcolumns=2]
		\item \exercise{583}
		\item \exercise{589}
		\item \exercise{593}
		\item \exercise{1569}
		\item \exercise{1592}
		\item \exercise{1585}
	\end{enumcols}
	\item Вычислить:
	\begin{enumcols}[itemcolumns=3]
		\item \exercise{1578}
		\item \exercise{1580}
		\item \exercise{1581}
	\end{enumcols}
	\item Решить уравнение:
	\begin{enumcols}[itemcolumns=2]
		\item \exercise{3626}
		\item \exercise{3777}
		\item \( \sqrt{\dfrac{12}{3x-2}}=\dfrac{3}{2} \)
		\item \exercise{489}
	\end{enumcols}
	\item Решить уравнение:
	\begin{enumcols}[itemcolumns=2]
		\item \( \log_2(4-x)=7 \) \answer{ \( -124 \) }
		\item \( \log_{1/7}(7-2x)=-2 \) \answer{ \( -21 \) }
		\item \( \log_4(x+3)=\log_4(4x-15) \) \answer{ \( 6 \) }
		\item \( \log_5(7-x)=\log_5(3-x)+1 \) \answer{ \( 2 \) }
		\item \( \log_8 2^{8x-4}=4 \) \answer{ \( 2 \) }
		\item \( \log_5(x^2+13x)=\log_5(9x+5) \) \answer{ \( 1 \) }
	\end{enumcols}
\end{listofex}
\newpage
\title{Занятие №4}
\begin{listofex}
		\item Вычислить значения синуса и косинуса \( 30\degree \), \( 45\degree \), \( 60\degree \).
	\item Вычислить значения тангенса и котангенса с теми же самыми аргументами.
	\item Доказать следующие факты:\\
	ОТТ: \( \sin^2x+\cos^2x=1 \); \( \tg x = \dfrac{\sin x}{\cos x} \); \( \ctg x = \dfrac{\cos x}{\sin x} \) и \( \tg x \cdot \ctg x = 1 \)\\[1em]
	\fbox{%
		\begin{minipage}[t]{0,9\textwidth}
			\textit{Расширенное понятие синуса и косинуса.}\vspace{1em}
			
			\textbf{Косинус угла \boldmath\( {\alpha} \)} --- абсцисса точки на единичной окружности, соответствующей углу \( \alpha \).
			
			\textbf{Синус угла \boldmath\( \alpha \)} --- ордината точки на единичной окружности, соответствующей углу \( \alpha \).
		\end{minipage}
	}
	\item \exercise{2807}
	\item \fbox{%
		\parbox[t]{0.9\textwidth}{
			\begin{minipage}[t]{0,3\textwidth}
				\( \sin(x+360\degree\cdot n) = \sin x \)
				
				\( \cos(x+360\degree\cdot n) = \cos x \)
			\end{minipage}
			\begin{minipage}[t]{0,3\textwidth}
				
				\( \tg(x+360\degree\cdot n) = \tg x \)
				
				\( \ctg(x+360\degree\cdot n) = \ctg x \)
				
				\vspace{1em}
			\end{minipage}
			\begin{minipage}[t]{0,25\textwidth}
				
				\( \sin(-x) = -\sin x \)
				
				\( \cos(-x) = \cos x \)
			\end{minipage}
			\begin{minipage}[t]{0,25\textwidth}
				\( \sin(180 - x) = \sin x \)
				
				\( \cos(180 - x) = -\cos x \)
			\end{minipage}
			\begin{minipage}[t]{0,25\textwidth}
				
				\( \sin(180+x) = -\sin x \)
				
				\( \cos(180+x) = -\cos x \)
			\end{minipage}
		}
	}
	\item Вычислить:
	\begin{enumcols}[itemcolumns=1]
		\item \exercise{2808}
		%\item \exercise{2809}
		\item \exercise{2810}
		\item \exercise{2811}
	\end{enumcols}
\end{listofex}
%\newpage
%\title{Домашняя работа №2}
%\begin{listofex}
%
%\end{listofex}
%\newpage
%\title{Занятие №5}
%\begin{listofex}
%
%\end{listofex}
\newpage
\title{Занятие №6}
\begin{listofex}
	\item Вычислить значение:
	\begin{enumcols}[itemcolumns=3]
		\item \exercise{1137}
		\item \exercise{2969}
		\item \exercise{1143}
		\item \exercise{1144}
		\item \exercise{1145}
		\item \exercise{2962}
	\end{enumcols}
	\item Вычислить значение:
	\begin{enumcols}[itemcolumns=2]
		\item \exercise{2967}
		\item \exercise{2966}
		\item \exercise{2983}
		\item \exercise{2987}
		\item \exercise{2990}
	\end{enumcols}
	\item Вычислить значение:
	\begin{enumcols}[itemcolumns=2]
		\item \exercise{1117}
		\item \exercise{1118}
		\item \exercise{1841}
	\end{enumcols}
	\item \exercise{1828}
	\item Решить уравнения:
	\begin{enumcols}[itemcolumns=2]
		\item \exercise{3696}
		\item \( \sqrt{12-3x}=4 \)\answer{ \( -\dfrac{4}{3} \) }
		\item \( \sqrt{\dfrac{4}{2x-21}}=\dfrac{1}{5} \)
		\item \exercise{3403}
	\end{enumcols}
\end{listofex}
%\newpage
%\title{Занятие №7}
%\begin{listofex}
%	\item \exercise{1820}
%
%\end{listofex}
\newpage
\title{Проверочная работа}
\begin{listofex}
	\item Вычислить:
	\begin{enumcols}[itemcolumns=3]
		\item \exercise{562}
		\item \exercise{571}
		\item \exercise{1570}
		\item \exercise{1583}
		\item \exercise{595}
		\item \exercise{1594}
	\end{enumcols}
	\item Вычислить:
	\begin{enumcols}[itemcolumns=2]
		\item \exercise{1137}
		\item \exercise{1143}
		\item \exercise{1145}
		\item \exercise{1146}
	\end{enumcols}
	\item Вычислить значение:
	\begin{enumcols}[itemcolumns=2]
		\item \( \dfrac{12\sin13\degree\cdot\cos13\degree}{\sin26\degree} \)
		\item \( \dfrac{13}{\sin^237\degree+\sin^2127\degree} \)
	\end{enumcols}
	\item Вычислить:
	\begin{enumcols}[itemcolumns=3]
		\item \( -4\sqrt{3}\sin\left( -\dfrac{4\pi}{3} \right) \)
		\item \( (2\sqrt{5})^2\tg\left( \dfrac{\vphantom{3}\pi}{4} \right)\cdot\tg\left( \dfrac{3\pi}{4} \right) \)
		\item \( \dfrac{7}{\cos^2\left( \dfrac{\vphantom{5}\pi}{4} \right)+\cos^2\left( \dfrac{3\pi}{4} \right)} \)
	\end{enumcols}
	\item Решить уравнение:
	\begin{enumcols}[itemcolumns=2]
		\item \( \log_{1/7}(5-4x)=-1 \)
		\item \( \log_4(3x+3)=\log_4(2x-11) \)
		\item \( \log_5(x^2+13x)=\log_5(9x+5) \)
		\item \exercise{3653}
		\item \exercise{1164}
	\end{enumcols}
	\item \exercise{1117}
\end{listofex}