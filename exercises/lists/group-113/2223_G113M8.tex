%
%===============>>  ГРУППА 11-3 МОДУЛЬ 8  <<=============
%
\setmodule{8}

%BEGIN_FOLD % ====>>_____ Занятие 1 _____<<====
\begin{class}[number=1]
	\begin{listofex}
		\item Занятие 1
	\end{listofex}
\end{class}
%END_FOLD

%BEGIN_FOLD % ====>>_____ Занятие 2 _____<<====
\begin{class}[number=2]
	\begin{listofex}
		\item Найдите все значения \(a\), при каждом из которых система \[ \begin{cases} yx^2+y^2=2y+63-7x^2, \\ x \ge -3, \\ x+y=a  \end{cases} \] имеет ровно два различных решения.
		\item Найдите все значения \(a\), при каждом из которых система \[ \begin{cases} x^2-2x+|y|-15=0 \\ x^2+(y-a)(y+a)=2 (x-0,5)  \end{cases} \] имеет ровно шесть решений.
		\item Найдите все значения \(a\), при каждом из которых система уравнений \[ \begin{cases} |x-4|+3|y|=2 \\ 9y^2+x^2-8x+4(a+3)=0 \end{cases} \] имеет ровно четыре решения.
	\end{listofex}
\end{class}
%END_FOLD

%BEGIN_FOLD % ====>>_ Домашняя работа 1 _<<====
\begin{homework}[number=1]
	\begin{listofex}
		\item Имеются два сосуда. Первый содержит \( 30 \) кг, а второй --- \( 20 \) кг раствора кислоты различной концентрации. Если эти растворы смешать, то получится раствор, содержащий \( 68\% \) кислоты. Если же смешать равные массы этих растворов, то получится раствор, содержащий \( 70\% \) кислоты. Сколько килограммов кислоты содержится в первом сосуде?
		\item Найдите наибольшее значение функции \( y=x+\dfrac{9}{x} \) на отрезке \( [-4; -1] \)
		\item 
		\begin{tasks}(1)
			\task Решите уравнение: \( \log_4\left( 2^{2x}-\sqrt{3}\cos x-\sin2x \right)=x \).
			\task Найжите все корни этого уравнения, принадлежащие отрезку \( \left[ 2\pi; \dfrac{7\pi}{2} \right]  \).
		\end{tasks}
		\item Решите неравенство \( \dfrac{0,2^{|x^2-4x+2|}-0,04}{3-x}\le0 \).
		\item Найдите все значения параметра \( a \), при каждом из которых система уравнений
		\[ \begin{cases}
			3|x-2|+|y|-3=0,\\
			ax-y+2a+2=0
		\end{cases} \]
		имеет ровно \( 2 \) решения.
		\item Найдите все значения параметра \( a \), при каждом из которых система уравнений
		\[ \begin{cases}
			\dfrac{xy^2-2xy-4y+8}{\sqrt{x+4}}=0,\\
			y=ax
		\end{cases} \]
		имеет ровно \( 2 \) различных решения.
		\item Найдите все значения параметра \( a \), при каждом из которых система уравнений
		\[ \begin{cases}
			x^2+12x+|y|+27=0,\\
			x^2+(y-a)(y+a)=-12(x+3)
		\end{cases} \]
		имеет ровно \( 4 \) решения.
	\end{listofex}
\end{homework}
%END_FOLD

%BEGIN_FOLD % ====>>_____ Занятие 3 _____<<====
\begin{class}[number=3]
	\begin{listofex}
		\item Занятие 3 
	\end{listofex}
\end{class}
%END_FOLD

%BEGIN_FOLD % ====>>_____ Занятие 4 _____<<====
\begin{class}[number=4]
	\begin{listofex}
		\item Занятие 4
	\end{listofex}
\end{class}
%END_FOLD

%BEGIN_FOLD % ====>>_ Домашняя работа 2 _<<====
\begin{homework}[number=2]
	\begin{listofex}
		\item Домашняя работа 2
	\end{listofex}
\end{homework}
%END_FOLD

%BEGIN_FOLD % ====>>_____ Занятие 5 _____<<====
\begin{class}[number=5]
	\begin{listofex}
		\item Занятие 5
	\end{listofex}
\end{class}
%END_FOLD

%BEGIN_FOLD % ====>>_____ Занятие 6 _____<<====
\begin{class}[number=6]
	\begin{listofex}
		\item Занятие 6
	\end{listofex}
\end{class}
%END_FOLD

%BEGIN_FOLD % ====>>_ Домашняя работа 3 _<<====
\begin{homework}[number=3]
	\begin{listofex}
		\item Домашняя работа 3
	\end{listofex}
\end{homework}
%END_FOLD

%BEGIN_FOLD % ====>>_____ Занятие 7 _____<<====
\begin{class}[number=7]
	\title{Подготовка к проверочной}
	\begin{listofex}
		\item Занятие 7
	\end{listofex}
\end{class}
%END_FOLD

=%BEGIN_FOLD % ====>>_ Проверочная работа _<<====
\begin{exam}
	\begin{listofex}
		\item Проверочная
	\end{listofex}
\end{exam}
%END_FOLD

	%BEGIN_FOLD % ====>>_ Консультация _<<====
\begin{consultation}
	\begin{listofex}
		\item Какова вероятность того, что случайно выбранный телефонный номер оканчивается двумя чётными цифрами?
		\item Если шахматист А. играет белыми фигурами, то он выигрывает у шахматиста Б. с вероятностью \( 0,52 \). Если А. играет черными, то А. выигрывает у Б. с вероятностью \( 0,3 \). Шахматисты А. и Б. играют две партии, причём во второй партии меняют цвет фигур. Найдите вероятность того, что А. выиграет оба раза.
		\item Вероятность того, что в случайный момент времени температура тела здорового человека окажется ниже чем \( 36,8\degree \)С, равна \( 0,81 \). Найдите вероятность того, что в случайный момент времени у здорового человека температура окажется \( 36,8\degree \)С или выше.
		\item При изготовлении подшипников диаметром \( 67 \) мм вероятность того, что диаметр будет отличаться от заданного не больше, чем на \( 0,01 \) мм, равна \( 0,965 \). Найдите вероятность того, что случайный подшипник будет иметь диаметр меньше чем \( 66,99 \) мм или больше чем \( 67,01 \) мм.
		\item В магазине три продавца. Каждый из них занят с клиентом с вероятностью \( 0,3 \). Найдите вероятность того, что в случайный момент времени все три продавца заняты одновременно (считайте, что клиенты заходят независимо друг от друга).
		\item Из районного центра в деревню ежедневно ходит автобус. Вероятность того, что в понедельник в автобусе окажется меньше \( 20 \) пассажиров, равна 0,94. Вероятность того, что окажется меньше \( 15 \) пассажиров, равна \( 0,56 \). Найдите вероятность того, что число пассажиров будет от \( 15 \) до \( 19 \).
		\item Стрелок стреляет по мишени один раз. В случае промаха стрелок делает второй выстрел по той же мишени. Вероятность попасть в мишень при одном выстреле равна \( 0,7 \). Найдите вероятность того, что мишень будет поражена (либо первым, либо вторым выстрелом).
		\item Стрелок стреляет по \( 4 \) одинаковым мишеням по одному разу, вероятность промаха \( 0,2 \), найдите вероятность что он попадёт в первую мишень, а в \( 3 \) оставшиеся промахнется.
		\item Телефон передаёт SMS-сообщение. В случае неудачи телефон делает следующую попытку. Вероятность того, что сообщение удастся передать без ошибок в каждой отдельной попытке, равна \( 0,4 \). Найдите вероятность того, что для передачи сообщения потребуется не больше двух попыток.
		\item Пристани \( A \) и \( B \) расположены на озере, расстояние между ними \( 390 \) км. Баржа отправилась с постоянной скоростью из \( A \) в \( B \). На следующий день после прибытия она отправилась обратно со скоростью на \( 3 \) км/ч больше прежней, сделав по пути остановку на \( 9 \) часов. В результате она затратила на обратный путь столько же времени, сколько на путь из \( A \) в \( B \). Найдите скорость баржи на пути из \( A \) в \( B \). Ответ дайте в км/ч.
		\item Путешественник переплыл море на яхте со средней скоростью \( 20 \) км/ч. Обратно он летел на спортивном самолете со скоростью \( 480 \) км/ч. Найдите среднюю скорость путешественника на протяжении всего пути. Ответ дайте в км/ч.
		\item Заказ на \( 110 \) деталей первый рабочий выполняет на \( 1 \) час быстрее, чем второй. Сколько деталей за час изготавливает второй рабочий, если известно, что первый за час изготавливает на \( 1 \) деталь больше?
		\item На изготовление \( 99 \) деталей первый рабочий тратит на \( 2 \) часа меньше, чем второй рабочий на изготовление \( 110 \) таких же деталей. Известно, что первый рабочий за час делает на \( 1 \) деталь больше, чем второй. Сколько деталей в час делает второй рабочий?
		\item Первая труба пропускает на \( 1 \) литр воды в минуту меньше, чем вторая. Сколько литров воды в минуту пропускает первая труба, если резервуар объемом \( 110 \) литров она заполняет на \( 1 \) минуту дольше, чем вторая труба?
		\item Первая труба пропускает на \( 5 \) литров воды в минуту меньше, чем вторая. Сколько литров воды в минуту пропускает вторая труба, если резервуар объемом \( 375 \) литров она заполняет на \( 10 \) минут быстрее, чем первая труба заполняет резервуар объемом \( 500 \) литров?
	\end{listofex}
\end{consultation}
%END_FOLD