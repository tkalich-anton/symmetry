%Группа 71-1 Модуль 2
\title{Занятие №1}
\begin{listofex}
	\item На свой день рождения Алиса купила \( 560 \) кг фруктов (на весь класс). Из них \( 4/7 \) составляют яблоки, а остальное --- апельсины. Сколько килограммов апельсинов купила Алиса. Какую часть от всех фруктов составляют апельсины?
	\item В первый день турист прошел \( 42 \) км, что составляет \( 7/11 \) всего пути. Сколько километров осталось пройти туристу?
	\item Запишите произведение в виде степени, назовите основание и показатель степени:
	\begin{enumcols}[itemcolumns=3]
		\item \( 2\cdot2\cdot2 \)
		\item \( 10\cdot10\cdot10\cdot10\cdot10 \)
		\item \( 3\cdot3\cdot3\cdot3 \)
	\end{enumcols}
	\item Запишите произведение в виде степени числа \( 10 \):
	\begin{enumcols}[itemcolumns=3]
		\item \( 2\cdot5 \)
		\item \( 2\cdot5\cdot2\cdot5 \)
		\item \( 2\cdot2\cdot2\cdot2\cdot5\cdot5\cdot5\cdot5 \)
	\end{enumcols}
	\item Запишите произведение в виде степени:
	\begin{enumcols}[itemcolumns=3]
		\item \( 2^4\cdot2^3 \)
		\item \( 3^6\cdot3^2\cdot3^2 \)
		\item \( 5\cdot5^4\cdot5 \)
	\end{enumcols}
	\item Запишите в виде степени:
	\begin{enumcols}[itemcolumns=3]
		\item \( (11^9)^9 \)
		\item \( (2^{11})^{11} \)
		\item \( 7^5\cdot(7^2)^{10} \)
		\item \( (3^4)^5\cdot(3^5)^4\cdot(3^4)^4\cdot(3^5)^5 \)
	\end{enumcols}
	\item Какие числа называют простыми? Какие числа называют составными?
	\item Представьте число в виде произведения степеней простых чисел:
	\begin{enumcols}[itemcolumns=4]
		\item \( 16 \)
		\item \( 81 \)
		\item \( 1000 \)
		\item \( 196 \)
	\end{enumcols}
	\item Сколько градусов составляет \( 4/15 \) прямого угла? Сколько градусов составляет \( 7/20 \) развернутого угла?
	\item Рабочий за \( 4 \) дня окончил некоторую работу, сделав в первый день \( 3/20 \) всей работы, во второй день \( 7/40 \), а в третий --- \( 3/8 \). Какую часть работы он сделал в четвертый день?
\end{listofex}
\newpage
\title{Занятие №2}
\begin{listofex}
	\item Вася прочитал \( 13/15 \) книги. Сколько страниц прочитал Вася, если в книге \( 195 \) страниц?
	\item Запишите произведение в виде степени, назовите основание и показатель степени:
	\begin{enumcols}[itemcolumns=3]
		\item \( 7\cdot7\cdot7\cdot7 \)
		\item \( 11\cdot11\cdot11 \)
		\item \( 5\cdot5\cdot5\cdot5\cdot5\cdot5\cdot5 \)
	\end{enumcols}
	\item Запишите произведение в виде степени числа \( 6 \):
	\begin{enumcols}[itemcolumns=3]
		\item \( 2\cdot3 \)
		\item \( 2\cdot3\cdot2\cdot3\cdot2\cdot3 \)
		\item \( 2\cdot2\cdot2\cdot2\cdot2\cdot3\cdot3\cdot3\cdot3\cdot3 \)
	\end{enumcols}
	\item Запишите произведение в виде степени:
	\begin{enumcols}[itemcolumns=3]
		\item \( 3^2\cdot3^3 \)
		\item \( 4^9\cdot3^8\cdot3^7 \)
		\item \( 6\cdot6^7\cdot6 \)
	\end{enumcols}
	\item Запишите в виде степени:
	\begin{enumcols}[itemcolumns=2]
		\item \( (7^4)^7 \)
		\item \( (3^{99})^2 \)
		\item \( 2^7\cdot(2^6)^5 \)
		\item \( (11^2)^3\cdot(11^4)^5\cdot(11^6)^7 \)
	\end{enumcols}
	\item Представьте число в виде произведения степеней простых чисел:
	\begin{enumcols}[itemcolumns=4]
		\item \( 32 \)
		\item \( 36 \)
		\item \( 10000 \)
		\item \( 500 \)
	\end{enumcols}
	\item Федя читает книжку, в которой \( 720 \) страниц. За первый день он прочитал \( 5/12 \) всей книжки, а за второй --- \( 7/18 \) всей книжки. Сколько страниц ему осталось прочитать?
	\item Автомобиль проехал \( 575 \) км, что составляет \( 23/25 \) расстояния между двумя городами. Найдите расстояние между городами.
\end{listofex}
\newpage
\title{Домашняя работа №1}
\begin{listofex}
	\item Длина дороги \( 84 \) км. За первый день бригада рабочих отремонтировала \( 5/12 \) дороги, а за второй день --- \( 5/14 \) дороги. Сколько километров осталось отремонтировать?
	\item Заказанная работа была выполнена в \( 3 \) дня. В первый день было сделано \( 4/15 \) всей работы, во второй --- \( 5/12 \) всей работы. Какая часть работы была сделана в третий день?
	\item Вася прочитал \( 195 \) страниц, что составляет \( 13/15 \) всей книги. Сколько страниц в книге?
	\item Запишите произведение в виде степени, назовите основание и показатель степени:
	\begin{enumcols}[itemcolumns=3]
		\item \( 27\cdot27\cdot27 \)
		\item \( 4\cdot4\cdot4\cdot4\cdot4 \)
		\item \( 101\cdot101\cdot101\cdot101\cdot101\cdot101 \)
	\end{enumcols}
	\item Запишите произведение в виде степени числа \( 15 \):
	\begin{enumcols}[itemcolumns=3]
		\item \( 5\cdot3 \)
		\item \( 3\cdot5\cdot3\cdot5\cdot3\cdot5 \)
		\item \( 3\cdot3\cdot3\cdot3\cdot3\cdot5\cdot5\cdot5\cdot5\cdot5 \)
	\end{enumcols}
	\item Запишите произведение в виде степени:
	\begin{enumcols}[itemcolumns=3]
		\item \( 2^5\cdot2^9 \)
		\item \( 3^{11}\cdot3^{11}\cdot3^{11}\cdot3^{11} \)
		\item \( 54\cdot54^2\cdot54^3 \)
	\end{enumcols}
	\item Запишите в виде степени:
	\begin{enumcols}[itemcolumns=4]
		\item \( (2^5)^2 \)
		\item \( (3^4)^5 \)
		\item \( 5^2\cdot(5^3)^4 \)
		\item \( (4^3)^5\cdot(4^11)^2 \)
	\end{enumcols}
	\item Представьте число в виде произведения степеней простых чисел:
	\begin{enumcols}[itemcolumns=4]
		\item \( 64 \)
		\item \( 144 \)
		\item \( 4000 \)
		\item \( 504 \)
	\end{enumcols}
\end{listofex}
%\newpage
%\title{Занятие №3}
%\begin{listofex}
%
%\end{listofex}
%\newpage
%\title{Занятие №4}
%\begin{listofex}
%
%\end{listofex}
%\newpage
%\title{Домашняя работа №2}
%\begin{listofex}
%
%\end{listofex}
%\newpage
%\title{Занятие №5}
%\begin{listofex}
%
%\end{listofex}
%\newpage
%\title{Занятие №6}
%\begin{listofex}
%
%\end{listofex}
%\newpage
%\title{Занятие №7}
%\begin{listofex}
%
%\end{listofex}
%\newpage
%\title{Проверочная работа}
%\begin{listofex}
%
%\end{listofex}