%
%===============>>  ГРУППА 10-2 МОДУЛЬ 7  <<=============
%
\setmodule{7}

%BEGIN_FOLD % ====>>_____ Занятие 1 _____<<====
\begin{class}[number=1]
	\begin{definit}
		Логарифмом положительного числа \(b\) по основанию \(a, \\ a>0 \), и \(a \neq 1\) называют число \( \alpha \), такое, что \(b=a^{c}\). Логарифм обозначают так: \[ c = \log_a b \]
	\end{definit}
	\begin{listofex}
		\item Вычислите:
		\begin{tasks}(3)
			\task \( \log_2 1 \)
			\task \( \log_{0,01} 0,01 \)
			\task \( \log_3 27 \)
			\task \( \log_5 125 \)
			\task \( \log_{10} 0,001 \)
			\task \( \log_4 1 \)
			\task \( \log_5 \dfrac{1}{5} \)
			\task \( \log_{10} 100 \)
			\task \( \log_5 5^3 \)
			\task \( \log_7 7^5 \)
		\end{tasks}
		\item Вычислите
		\begin{tasks}(3)
			\task \( 2^{\log_2 3} \)
			\task \( 3^{\log_3 5} \)
			\task \( 2^{\log_2 3 + \log_2 5} \)
			\task \( ( 3^{\log_3 7} )^2 \)
			\task \( 7^{2\log_7 3} \)
			\task \( 0,1^{2\log_{0,1} 10}  \)
			\task \( 9^{\log_3 12} \)
			\task \( 16^{\log_2 5} \)
			\task \( 49^{\log_7 \frac{1}{3}} \)
		\end{tasks}
	\end{listofex}
	\begin{definit}
		Логарифмом положительного числа \(b\) по основанию \(10\) называют \textbf{десятичым логарифмом числа \(b\)} и обозначают так: \( \lg b \).
	\end{definit}
	\begin{definit}
		Логарифмом положительного числа \(b\) по основанию \(e\) называют \textbf{натуральным логарифмом числа \(b\)} и обозначают так: \( \ln b \).
	\end{definit}
	\begin{listofex}[resume]
		\item Вычислите:
		\begin{tasks}(3)
			\task \( \log_{10} 10 \)
			\task \( \log_{10} 100 \)
			\task \( \lg 1000 \)
			\task \( \log_2 2^3 \)
			\task \( \log_5 5^7 \)
			\task \( \log_9 9^{1999} \)
			\task \( e^{\ln 3} \)
			\task \( e^{2\ln 5} \)
			\task \( e^{-2\ln 3} \)
			\task \( \ln e \)
			\task \( \ln e^3 \)
			\task \( \lg 10^n \)
			\task \( \ln \dfrac{1}{e} \)
			\task \( \lg \sqrt[3]{0,01} \)
			\task \( \lg 10^{-1} \)
		\end{tasks}
	\end{listofex}
	\begin{definit}
		Свойства логарифма:
		\[ \log_a(M \cdot N) = \log_a M + \log_a N \]
		\[ \log_a \dfrac{M}{N} = \log_a M - \log_a N \]
		\[ \log_{a^l} M^k=\dfrac{k}{l}\log_a M \]
	\end{definit}
	\begin{listofex}[resume]
		
		\item Вычислите:
		\begin{tasks}(2)
			\task \( \log_6 2 + \log_6 3 \)
			\task \( \log_{15} 5 + \log_{15} 3 \)
			\task \( \log_4 \dfrac{2}{3} + \log_4 6 \)
			\task \( \log_2 \dfrac{2}{5} + \log_2 10 \)
			\task \( \log_2 6 - \log_2 3 \)
			\task \( \log_5 75 - \log_5 3 \)
			\task \( \log_3 36 - \log_3 4 \)
			\task \( \log_3 0,81 - \log_3 0,03 \)
			\task \( 2\log_6 2 + \log_6 9 \)
			\task \( \log_{11} 484 - 2 \log_{11} 2 \)
		\end{tasks}
		\item Вычислите: 
		\[ \log_327-\log_{\sqrt{3}}27-\log_{1/3}27-\log_{\sqrt{3}/264}\left( \dfrac{64}{27} \right)  \]
	\end{listofex}
\end{class}
%END_FOLD

%BEGIN_FOLD % ====>>_____ Занятие 2 _____<<====
\begin{class}[number=2]
	\begin{definit}
		Для положительных чисел \( a, b, M \) таких, что \( a \neq 1, b \neq 1 \) справедливо следующее равенство: \[ \log_a M = \dfrac{\log_b M}{\log_b a} = \dfrac{1}{\log_M a} \]
	\end{definit}
	\begin{listofex}
		\item Выразите через логарифмы по основанию \(2\) и упростите:
		\begin{tasks}(4)
			\task \( \log_3 5 \)
			\task \( \log_4 9 \)
			\task \( \log_5 9 \)
			\task \( \log_{128} 8 \)
			\task \( \log_5 15 \)
			\task \( \log_3 12 \)
			\task \( \log_{16} 15 \)
			\task \( \log_{0,01} 2 \)
			\task \( \log_{0,25} 7 \)
			\task \( \log_{0,125} 3 \)
			\task \( \log_{\frac{1}{16}} 2 \)
			\task \( \log_{\frac{1}{32}} 5 \)
		\end{tasks}
	\end{listofex}
	\begin{definit}
		Свойства логарифма:
		\begin{tasks}(2)
			\task \( \log_a(M \cdot N) = \log_a M + \log_a N \)
			\task \( \log_a \dfrac{M}{N} = \log_a M - \log_a N \)
			\task \( \log_{a^l} M^k=\dfrac{k}{l}\log_a M \)
			\task \( \log_a b \cdot \log_b a = 1 \)
			\task \( a^{\log_b c}=c^{\log_b a} \)
		\end{tasks}
		
	\end{definit}
	\begin{listofex}[resume]
		\item Вычислите:
		\begin{tasks}(2)
			\task \( \log_{0,5}2 \)
			\task \( \log_{\frac{1}{2}}8^3 \)
			\task \( \log_{0,5}4^2 \)
			\task \( \log_3 \dfrac{1}{3} \)
			\task \( \log_3 \left(  \dfrac{1}{9} \right)^3 \)
			\task \( \log_4 \left(  \dfrac{1}{16} \right)^5 \)
			\task \( \log_2 \sqrt{2} + \log_{\sqrt{2}}2 \)
			\task \( \log_3 \sqrt{3^3} + \log^2_{\sqrt{3}}\sqrt{27} \)
			\task \( \log^2_5 \sqrt{5^5} - \log_{\sqrt{5}}5^3 \)
			\task \( 6^{\log_{36} 25} \)
			\task \( 7^{\log_{49} 36} \)
			\task \( 4^{\tfrac{1}{2\log_{625} 16 }} \)
			\task \( 36^{\log_6 2}:4^{\log_2 3} \)
			\task \( 9^{\log_3 5}+25^{\log_5 9} \)
			%\task \( 49^{\log_7 3} \)
			\task \( 8^{\log_2 36^{\log_6 2}} \)
			\task \( (\sqrt[3]{5})^{\log_5 2} \)
			\task \( \log_2 \sqrt[3]{16} + \log_{\sqrt{\frac{1}{16}}}4^{2} \)
			\task \( \log^2_3 (27\sqrt{3} - \log_{\sqrt{\frac{1}{3}}}9) \)
			\task \( \log_5 \sqrt{5\sqrt{5}} \)
		\end{tasks}
		\newpage
		\item Вычислите:
		\begin{tasks}(1)
			\task \( 3^{\log_{\sqrt[3]{9}}4} +2^{\tfrac{1}{\log_{16} 4}} \)
			\task \( \dfrac{\log_3 135}{\log_{15} 3} - \dfrac{\log_3 5}{\log_{405} 3} \)
			\task \( \dfrac{3+\log_{12}27}{3-\log_{12} 27} \cdot \log_6 16 \)
			%\task \( \log_3 27 - \log_{\sqrt{3}}27 - \log_{\frac{1}{3}} 27 - \log_{\frac{\sqrt{3}}{2}} \left( \dfrac{64}{27} \right)  \)
			\task \( \log_{0,4}(0,2 \cdot \sqrt[3]{50}) + \log_{0,5}\left( \dfrac{\sqrt{15}}{5} \right) + \log_{0,32} \left( \dfrac{2\sqrt{2}}{5} \right)  \)
			\task \( \left( \log_{0,5} \sqrt[3]{0,25} + 6\log_{0,25} 0,5 - 2 \log_{\frac{1}{16}}0,25 \right) : \log_{\sqrt{2}}\sqrt[5]{8} \)
		\end{tasks}
		
		\item Решите уравнения: %По первые 4 с решуегэ простейшие и сложнейшие
		\begin{tasks}(1)
			\task \( \log_2 (4-x)=7 \)
			\task \( \log_5(4+x)=2 \)
			\task \( \log_5(5-x)=\log_5 3 \)
			\task \( \log_2(15+x)=\log_2 3 \)
			\task \( \log_5 (2-x) = \log_{25} x^4 \)
			\task \( \log_2 (x^2-14x)=5 \)
			\task \( 6 \log^2_8 x -5\log_8 x+1=0 \)
			\task \( 1+\log_2(9x^2+5)=\log_{\sqrt{2}} \sqrt{8x^4+14} \)
		\end{tasks}
	\end{listofex}
\end{class}
%END_FOLD

%BEGIN_FOLD % ====>>_ Домашняя работа 1 _<<====
\begin{homework}[number=1]
	\begin{listofex}
		\item Вычислите:
		\begin{tasks}(3)
			\task \( \log_2 4 \)
			\task \( \log_2 16 \)
			\task \( \log_3 3 \)
			\task \( \log_3 27 \)
			\task \( \log_4 4^3 \)
			\task \( \log_7 49^8 \)
		\end{tasks}
		\item Вычислите
		\begin{tasks}(3)
			\task \( 2^{\log_2 5} \)
			\task \( 7^{\log_7 9} \)
			\task \( (3^2)^{\log_3 7} \)
			\task \( 10^{3\lg 5} \)
			\task \( 3^{\log_3 90} \)
			\task \( 5^{\log_5 0,5} \)
			\task \( 10^{\lg 3} \)
			\task \( 10^{2\lg 3} \)
			\task \( 10^{-3\lg 2} \)
			\end{tasks}
		\item Вычислите:
		\begin{tasks}(2)
			\task \( \log_3 0,9 + \log_3 30 \)
			\task \( \log_8 \dfrac{8}{7} + \log_8 \dfrac{7}{8} \)
			\task \( \log_4 48 - \log_4 3 \)
			\task \( \log_7 0,98 - \log_7 0,14 \)
			\task \( 4\log_{12} 2 + 2\log_{12} 3 \)
			\task \( \log_5 100 - 2 \log_5 2 \)
		\end{tasks}
		\item Решите уравнения: %По вторые 4 с решуегэ простейшие и сложнейшие
		\begin{tasks}(2)
			\task \( \log_4 (x+3) = \log_4 (4x-15) \)
			\task \( \log_{\tfrac{1}{7}}(7-x) = -2 \)
			\task \( \log_5 (5-x) = 2 \log_5 3 \)
			\task \( \log_5 (x^2+2x) = \log_5 (x^2+10) \)
			\task \( \log_7 (x+2) = \log_{49} (x^4) \)
			\task \( \log_3 (x^2-24x)=4 \)
			\task \( \log_3 (x^2-2x)=1 \)
			\task \( \log_2^2 (x-4) -6 \log_2 (x-4) = 7 \)
			%\task \( \log_2^2(x^2) - 16\log_2 (2x) +31=0 \)
		\end{tasks}
	\end{listofex}
\end{homework}
%END_FOLD

%BEGIN_FOLD % ====>>_____ Занятие 3 _____<<====
\begin{class}[number=3]
	\begin{listofex}
		\item Решите уравнения:
		\begin{tasks}(2)
			\task \( \log_x 32 =5 \)
			\task \( \log_{11} (x-18) = 2 \)
			\task \( \log_{x-5}49=2 \)
			\task \( \log_8 2^{8x-4}=4 \)
			\task \( 2^{\log_8 (5x-3)}=4 \)
			\task \( \log_5(7-x)=\log_5(3-x)+1 \)
			\task \( \log_3 (x-5)^2 = 2 \)
		\end{tasks}
		
		\item Решите уравнения: %7
		\begin{tasks}(1)
			\task \( \log_7(x+2)=\log_{49}x^4 \)
			\task \( \lg (x^2-3x+1)= \lg (2x-5) \)
			\task \( \log_3 (x^2-12)+ 0,5 \log_{\tfrac{1}{3}} x^2 = 0 \)
			\task \( \log_2 (\log_{0,5} (\log_{625} (x^2+x-1) ) )=1 \)
			\task \( \lg x - x + x \lg x -1 =0 \)
			\task \( \log_4^2x - \log_4 \sqrt{x} - 1,5 = 0 \)
			\task \( \log_x 2 - \log_4 x + \dfrac{7}{6} = 0 \)
			
			\task \( \log_4 (2 \cdot 4^{x-2}-1)=2x-4 \)
		\end{tasks}
		
		
		
		
		
		%
		%\item %14
		%\begin{tasks}
		%	\task Решите уравнение: \[ \log_{-x^2-32x+33}(2x^3+136)= \dfrac{1}{\log_{-33x} ((1-x)(x+33))} \]
		%	\task Найдите все корни этого уравнения, принадлежащие отрезку: \( \left[ -\sqrt{333};-\sqrt{33} \right]  \)
		%\end{tasks}
		%
		%
	\end{listofex}
\end{class}
%END_FOLD

%BEGIN_FOLD % ====>>_____ Занятие 4 _____<<====
\begin{class}[number=4]
	\begin{listofex}
		\item Решите уравнения: %Шестаков 13 стр 109 буквы а
		\begin{tasks}(2)
			
			\task \( \log_{4-x} (2x^2-9x+10)=0 \)
			\task \( \log_{19} y^4 = \log_{19}(19y)^2 \)
			\task \( x^2+\log_5 x + \log_5 \dfrac{25}{x}=11 \)
			\task \( \dfrac{(x+6)(x+7)}{\log_7 (x+8)} = 0 \)
			\task \( (x^2-9)\lg(-x)=0 \)
			\task \( \log_5 (x+1)^4=8 \)
			\task \( \log_2 (x^2-7) = \log_{4-x}(4-x) \)
			\task \( \log_{11} (y^2-35)=\log_{7-y} 1 \)
		\end{tasks}
		
		%\item %16
		%\begin{tasks}
		%	\task Решите уравнение: \( (x^2+2x-1)(\log_2(x^2-3)+\log_{0,5}(\sqrt{3}-x))=0 \)
		%	\task Найдите все корни этого уравнения, принадлежащие отрезку: \( [-2,5;-1,5]  \)
		%\end{tasks}
		
	\end{listofex}
\end{class}
%END_FOLD

%BEGIN_FOLD % ====>>_ Домашняя работа 2 _<<====
\begin{homework}[number=2]
	\begin{listofex}
		\item Решите уравнения:
		\begin{tasks}(2) %Шестаков 13 стр 109 буквы Б 3 1 2 4 5 7
			\task \( \log_{25}x=\dfrac{ 1 }{ \log_{25}x } \)
			\task \( \log_{4-y}(3y^2-7y-5)=0 \)
			\task \( \log_{18}x^4 = \log_{18}(16x)^2 \)
			\task \( \log_{3}(y-35)=\log_{3}2y \)
			\task \( \log_{11}(y^2+9y+18)=\log_{11}(y+6) \)
			\task \( \log_4 (2 \cdot 4^{x-2}-1)=2x-4 \)
			%\task \( \log_{(-x^2-32x+33)} (2x^2+136) = \dfrac{1}{\log_{-33x} ((1-x)(x+33))} \)
		\end{tasks}
		\item Решите уравнения:
		\begin{tasks}
			\task \( y^2+\log_{6}y+\log_{6}\dfrac{ 36 }{ y }=83 \)
			\task \( \log_x 2 - \log_4 x + \dfrac{7}{6} = 0 \)
			\task \( \log_2^2 x - 2\log_2 x -3 =0 \)
		\end{tasks}
		
		%\item 
		%\begin{tasks}
		%	\task Решите уравнения: \(  \)
		%	\task Найдите все корни этого уравнения, принадлежащие отрезку: \( \left[  \right]  \)
		%\end{tasks}
	\end{listofex}
\end{homework}
%END_FOLD

%BEGIN_FOLD % ====>>_____ Занятие 5 _____<<====
\begin{class}[number=5]
	\begin{listofex}
		\item Решите уравнения: %Шестаков 13 стр 109 буквы а
		\begin{tasks}(1)
			\task \( \log_9(z^2-24)=\log_9(-2z) \)
			
			\task \( \log ^2 _x \sqrt{2} = 2 - \dfrac{ \ln \sqrt{2} }{ \ln x } \)
			\task \( \log_6(30-7x)=\log_{216}x^6 \)
			%\task \( \log_5(14-5x)=\log_{125}x^6 \)
			\task \( \log_{\sqrt{10}} \sqrt{x^4+1} = \lg (53x^2-5)-1 \)
			\task \( \dfrac{ 1 }{ \lg (3x-2) } + \dfrac{ 2 }{ \lg (3x-2) + \lg 0,01 } = -1 \)
			%\task \( \dfrac{ 1 }{ \lg (9x-8) } + \dfrac{ 4 }{ \lg (9x-8) + \lg 0,001 } = -1 \)
			\task \( \log_4 (2-y) \cdot \log_7 (2y^2-9y+10) = 0 \)
			\task \( (x+1)\log_{x+2}(x+3)=0 \)
			
		\end{tasks}
		\item %13
		\begin{tasks}
			\task Решите уравнение: \[ \log_2(x^2-5) \cdot \log_3^2(7-x) + 3\log_2(x^2-5)-2\log_3^2(7-x)-6=0 \]
			\task Найдите все корни этого уравнения, принадлежащие отрезку: \( \left[ \log_2 \dfrac{1}{7}; \log_2 9 \right]  \)
		\end{tasks}
	\end{listofex}
\end{class}
%END_FOLD

%BEGIN_FOLD % ====>>_____ Занятие 6 _____<<====
\begin{class}[number=6]
	\begin{listofex}
		\item Решите уравнения: %прошлое
		\begin{tasks}(1)
			\task \( \log_4 (2-y) \cdot \log_7 (2y^2-9y+10) = 0 \)
			\task \( (x+1)\log_{x+2}(x+3)=0 \)
			\task \( \log_8 (3-y) \cdot \log_5 (2y^2-13y+21)=0 \)
			\task \( \dfrac{ 1 }{ \lg (9x-8) } + \dfrac{ 4 }{ \lg (9x-8) + \lg 0,001 } = -1 \)
		\end{tasks}
		\item Решите неравенства: %299 1-8 A 34-37
		\begin{tasks}(2)
			\task \( \log_5 x<2 \)
			\task \( \log_3 x \le 3 \)
			\task \( \log_4 x \ge -0,5 \)
			\task \( \log_{0,123} x \le 0 \)
			\task \( \log_{\tfrac{1}{7}}x>2 \)
			\task \( \log_{0,04}x \ge -1 \)
			\task \( \log_5 (4x+5) < 0 \)
			\task \( \log_{0,1} (3x+25) < -2 \)
			\task \( \log_2(5x+7) < \log_2 (3x+11) \)
			\task \( \log_{0,9}(5x+7) < \log_{0,9} (2x+33) \)
			\task \( \ln(2x+17) \ge \ln(4x-13) \)
			\task \( \lg(25x^2-4) \le \lg(25-4x^2) \)
		\end{tasks}
	\end{listofex}
\end{class}
%END_FOLD

%BEGIN_FOLD % ====>>_ Домашняя работа 3 _<<====
\begin{homework}[number=3]
	\begin{listofex}
		\item Решите уравнения: %c109 14-17 b
		\begin{tasks}(2)
			\task \( (x+8)\log_{x+4}(x+1)=0 \)
			\task \( \dfrac{ (x-11)(x+21) }{ \log_{17}(x+13) }=0 \)
			\task \( (x-8)\log_9(x-8)=0 \)
			\task \( \log_8(x^2-18)=\log_8(3x) \)
		\end{tasks}
		\item Решите неравенства: %c299 33-36 b
		\begin{tasks}(2)
			\task \( \log_{\tfrac{1}{256}}(5x^2-6x-11) \le -0,5 \)
			\task \( \log_3 (5x+6) < \log_3 (2x+9) \)
			\task \( \log_{0,3}(5x-22) > \log_{0,3}(3x+22) \)
			\task \( \ln(3x+19)\ge \ln(5x-17) \)
		\end{tasks}
	\end{listofex}
\end{homework}
%END_FOLD

%BEGIN_FOLD % ====>>_____ Занятие 7 _____<<====
\begin{class}[number=7]
	\title{Подготовка к проверочной}
	\begin{listofex}
		\item Решите уравнения: %тренир с110 1-6
		\begin{tasks}(2)
			\task \( \log_{x^2}13=\log_{4-3x}13 \)
			\task \( (x+2)\log_{x+3}(x+4)=0 \)
			\task \( x^2+\log_7 x + \log_7 \dfrac{ 7 }{ x }=50 \)
			\task \( \log_{15}x^4=\log_{15}(15x)^2 \)
			\task \( \log_5^2(5x-4)=\log_5 (5x-4)^2 \)
			\task \( \dfrac{ (x-16)(x+19) }{ \log_{12}(x+17) }=0 \)
		\end{tasks}
		\item Решите неравенства: %диагн 1 с300 v1 4-8
		\begin{tasks}(2)
			\task \( \log_3 (2x^2+3x) \ge 3 \)
			\task \( \log_{\tfrac{1}{24}} (625-x^2) > -2 \)
			\task \( \log_{\tfrac{1}{6}}(5x-4) \ge -2 \)
			\task \( \log_3 (2x^2+5x-3) < 2 \)
			\task \( \log_{0,7} (2x+11) \le \log_{0,7} (4x-5) \)
			\task \( \log_2  (x^2-7)=\log_{4-x} (4-x)\)
			
		\end{tasks}
	\end{listofex}
\end{class}
%END_FOLD

%BEGIN_FOLD % ====>>_ Проверочная работа _<<====
\begin{exam}
	\begin{listofex}
		\item Вычислите:
		\begin{tasks}(2)
			\task \( \log_26 - \log_2 3 \)
			\task \( \log_7  \dfrac{ 49 }{ 50 } - \log_7 \dfrac{ 7 }{ 50 } \)
			\task \( 2\log_6 2 + \log_6 9 \)
			\task \( \log_{11} 484 - 2 \log_{11} 2  \)
		\end{tasks}
		\item Решите уравнения: %тренир с110 7-12
		\begin{tasks}(1)
			\task \( \log_7 (x^2-12) = \log_7 x \)
			\task \( \log_6 (x+3)^4 = 8 \)
			\task \( \log_6 (4-x) \cdot \log_7 (2x^2-17x+36) = 0 \)
			\task \( \log_{17}(x^2-24)=\log_{6-x}1 \)
			%\task \( \dfrac{ \log_{x+1}^2 (x-1) + \log_5^2(2x-5) }{ \log_{x+1}^2 (x-1) + \log_5^2(x-2) }=1 \)
		\end{tasks}
		\item Решите неравенства: %диагн 2 с301 v2 1-8
		\begin{tasks}(2)
			\task \( \log_9x > 0,5 \)
			\task \( \log_{0,5} x < 2 \)
			\task \( \log_2(11x-12) \le 5 \)
			\task \( \log_{\tfrac{1}{9}} (5x+6) \ge -2 \)
			\task \( \log_5 (2x^2+5x) \ge 2 \)
			\task \( \log_{\tfrac{1}{16}}(400-x^2) > -2 \)
			\task \( \log_4 (5x^2+9x-2) < 2 \)
			\task \( \log_{0,6}(2x+13) \le \log_{0,6} (4x-1) \)
		\end{tasks}
		%\item Решите системы неравенств: %диагн 2 с301 v2 10-11
		%\begin{tasks}(2)
		%	\task \( \begin{cases} \log_{0,2}(2x-7) \le -2 \\ \log_3 (x-5) \le 3 \end{cases} \)
		%	\task \( \begin{cases} \log_3 (25-x^2) \ge 2 \\ \log_5 (x^2+9) \ge 2 \end{cases} \)
		%\end{tasks}
	\end{listofex}
\end{exam}
%END_FOLD

%УРОК НА ПОТОМ
%\item Решите системы неравенств: %Шестаков15 с103 11-15 А
%\begin{tasks}(2)
%	
%	\task \( \begin{cases} -0,7x \le 2,1 \\ 2,1x < 0,7 \end{cases} \)
%	\task \( \begin{cases} 4x+9 \le 9x+4 \\ 1,7x \le 51 \end{cases} \)
%	\task \( \begin{cases} 5(4x +3)-4(5x +3)>3x \\ \dfrac{ 2 }{ 3 }x < \dfrac{ 3 }{ 2 }x+5 \end{cases} \)
%	
%	\task \( \begin{cases} \dfrac{  2x+5}{ 5 } > \dfrac{ 5x+2 }{ 2 } \\[8pt] \dfrac{ x+2 }{ 5 } < \dfrac{ x+5 }{ 2 } \end{cases} \)
%	\task \( \begin{cases} (x+6)^2 < (x+4)^2 \\ 6x+13 > 5x-7 \end{cases} \)
%	\task \( \begin{cases} 7(6x+5)-5(6x+7) \le 12x+21 \\ \dfrac{ x+2 }{ 3 }+\dfrac{ x+3 }{ 2 } > \dfrac{ x+5 }{ 4 } + \dfrac{ x+4 }{ 5 } \end{cases} \) %Пример
%\end{tasks}
%\item Решите системы неравенств: %Шестаков15 с122 14-16 А
%\begin{tasks}
%	\task \( \begin{cases} (x-7)(x^2-49) > 0 \\ (x-9)(x^2-81) \le 0 \end{cases} \)
%	\task \( \begin{cases} 2x^2(4x-3) > 7(3-4x) \\ 3(2x-7) \le 4x^2(7-2x) \end{cases} \)
%	\task \( \begin{cases} (x+4)^3(x+2)^2 \le (x+5)^2(x+2)^3 \\ x^2<15 \end{cases} \)
%\end{tasks}



%BEGIN_FOLD % ====>>_ Консультация _<<====
\begin{consultation}
	\begin{listofex}
		\item Вычислите с помощью формул приведения:
		\begin{tasks}(3)
			\task \( \sin 240\degree \)
			\task \( \tg 300\degree \)
			\task \( \cos 330\degree \)
			\task \( \ctg 315\degree \)
			\task \( \cos \dfrac{5\pi}{3} \)
			\task \( \sin \dfrac{7\pi}{6} \)
			\task \( \sin \left( -\dfrac{11\pi}{6} \right)  \)
			\task \( \sin \left( -\dfrac{7\pi}{3} \right)  \)
		\end{tasks}
		\item Упростите выражения:
		\begin{tasks}(3)
			\task \( \cos(90\degree - x) \)
			\task \( \cos(270\degree + x) \)
			%\task \( \sin(360\degree - x) \)
			\task \( \cos(180\degree + x) \)
			%\task \( \tg(90\degree - x) \)
			\task \( \tg(270\degree + x) \)
			%\task \( \ctg(360\degree + x) \)
			\task \( \ctg(180\degree - x) \)
			\task \( \tg \left( \dfrac{\pi}{2}-t \right)  \)
			%\task \( \cos \left( \dfrac{3\pi}{2}+t \right) \)
			%\task \( \cos(2\pi-t) \)
			\task \( \sin(\pi+t) \)
			%\task \( \sin \left( \dfrac{3\pi}{2}-t \right)  \)
			%\task \( \cos \left( \dfrac{\pi}{2}+t \right) \)
			\task \( \ctg(2\pi+t) \)
			\task \( \sin(\pi-t) \)
		\end{tasks}
		\item Упростите выражения:
		\begin{tasks}(2)
			\task \( \cos (a-b) - \cos a \cos b \)
			\task \( \sin(a+b) + \sin(a-b) \)
			\task \( \sin a \cos b - \sin(a-b) \)
			\task \( \cos(a-b)-\cos(a+b) \)
			\task \( \sin(a+b) - \sin a \cos b \)
			\task \( \sin a \sin b + \cos (a+b) \)
		\end{tasks}
		\item Упростите выражения:
		\begin{tasks}(2)
			
			\task \( \sin \left( \dfrac{\pi}{3}+a \right) -0,5\sin a \)
			
			\task \( \cos \left( \dfrac{\pi}{4}+a \right) + \dfrac{\sqrt{2}}{2}\sin a \)
			\task \( \sin \left( \dfrac{5\pi}{6}-a \right) -0,5\cos a \)
			\task \( \sqrt{3}\cos a - 2 \cos \left( a-\dfrac{\pi}{6} \right)  \)
		\end{tasks}
		\item Вычислите:
		\begin{tasks}(4)
			\task \( \tg 15 \degree \)
			\task \( \tg 75 \degree  \)
			\task \( \tg 105 \degree \)
			\task \( \tg 165 \degree \)
		\end{tasks}
	
		\item Вычислите:
		\begin{tasks}(2)
			
			\task \( \dfrac{12 \sin 11 \degree \cdot \cos 11 \degree}{\sin 22 \degree} \)
			\task \( \dfrac{5 \cos 29 \degree}{\sin 61 \degree} \)
			\task \( 36 \sqrt{6} \tg \dfrac{\pi}{6} \sin \dfrac{\pi}{4} \)
			\task \( 4 \sqrt{2} \cos \dfrac{\pi}{4} \cos \dfrac{7\pi}{3} \)
			
			%\task \( \dfrac{8}{\sin \left( -\dfrac{27\pi}{4} \right) \cos \left( \dfrac{31\pi}{4} \right) }  \)
			\task \( \dfrac{\tg 25 \degree + \tg 20 \degree}{1-\tg 25 \degree \tg 20 \degree} \)
			\task \( \dfrac{1-\tg 70 \degree \tg 65 \degree}{\tg 70 \degree + \tg 65 \degree} \)
			\task \( \dfrac{\tg 9 \degree + \tg 51 \degree}{1-\tg 9 \degree \tg 51 \degree} \)
			\task \( \dfrac{1+\tg 54 \degree \tg 9 \degree}{\tg 54 \degree - \tg 9 \degree} \)
		\end{tasks}
	\end{listofex}
\end{consultation}
%END_FOLD

%BEGIN_FOLD % ====>>_ Консультация _<<====
\begin{consultation}
	\begin{listofex}
		\item Упростите выражения:
		\begin{tasks}(3)
			\task \( \cos(90\degree - x) \)
			\task \( \cos(270\degree + x) \)
			\task \( \sin(360\degree - x) \)
			\task \( \cos(180\degree + x) \)
			\task \( \tg(90\degree - x) \)
			\task \( \tg(270\degree + x) \)
			\task \( \ctg(360\degree + x) \)
			\task \( \ctg(180\degree - x) \)
			\task \( \sin \left( \dfrac{\pi}{2}-t \right)  \)
			\task \( \cos \left( \dfrac{3\pi}{2}+t \right) \)
			\task \( \cos(2\pi-t) \)
			\task \( \sin(\pi+t) \)
			\task \( \sin \left( \dfrac{3\pi}{2}-t \right)  \)
			\task \( \cos \left( \dfrac{\pi}{2}+t \right) \)
			\task \( \cos(2\pi+t) \)
			\task \( \sin(\pi-t) \)
		\end{tasks}
		\item Упростите выражения:
		\begin{tasks}(2)
			\task \( \cos (a-b) - \cos a \cos b \)
			\task \( \sin(a+b) + \sin(a-b) \)
			\task \( \sin a \cos b - \sin(a-b) \)
			\task \( \cos(a-b)-\cos(a+b) \)
			\task \( \dfrac{3\cos(\pi-a)+\sin \left( \dfrac{\pi}{2} + a \right)}{\cos (a + 3 \pi)} \)
			\task \( \dfrac{2 \sin(a-7\pi)+\cos \left( \dfrac{3\pi}{2}+a \right)}{\sin(a+\pi)} \)
		\end{tasks}
		\item Вычислите:
		\begin{tasks}(2)
			
			\task \( \dfrac{12 \sin 11 \degree \cdot \cos 11 \degree}{\sin 22 \degree} \)
			\task \( \dfrac{14 \sin 19 \degree}{\sin 341 \degree} \)
			\task \( \dfrac{4 \cos 146 \degree}{\cos 34 \degree} \)
			\task \( -4 \sqrt{3} \cos (-750 \degree) \)
			\task \( 2 \sqrt{3} \tg(-300\degree) \)
			\task \( -18\sqrt{2} \sin(-135 \degree) \)
			\task \( 24 \sqrt{2} \cos \left( -\dfrac{\pi}{3} \right) \sin \left( -\dfrac{\pi}{4} \right) \)
			\task \( \dfrac{5 \cos 29 \degree}{\sin 61 \degree} \)
			\task \( 36 \sqrt{6} \tg \dfrac{\pi}{6} \sin \dfrac{\pi}{4} \)
			\task \( 4 \sqrt{2} \cos \dfrac{\pi}{4} \cos \dfrac{7\pi}{3} \)
			\task \( \dfrac{8}{\sin \left( -\dfrac{27\pi}{4} \right) \cos \left( \dfrac{31\pi}{4} \right) }  \)
			\task \( \dfrac{\tg 25 \degree + \tg 20 \degree}{1-\tg 25 \degree \tg 20 \degree} \)
			
		\end{tasks}
	\end{listofex}
\end{consultation}
%END_FOLD


%BEGIN_FOLD % ====>>_ Консультация моисей _<<====
\begin{consultation}
	\begin{listofex}
		\item Решите уравнения: %тренировочная до 8
		\begin{tasks}(2)
			%\task \( \log_{x^2}13=\log_{4-3x}13 \)
			\task \( (x+2)\log_{x+3}(x+4)=0 \)
			
			%\task \( \log_{15}x^4=\log_{15}(15x)^2 \)
			%\task \( \log_5 ^2 (5x-4)=\log_5 (5x-4)^2 \)
			
			\task \( \log_7 (x^2-12)=\log_7 x \)
			\task \( \log_6 (x+3)^4=8 \)
			\task \( x^2+\log_7x+\log_7 \dfrac{ 7 }{  x}=50 \)
			
			\task \( \dfrac{ (x-16)(x+19) }{ \log_{12}(x+17) }=0 \)
		\end{tasks}
		\item Решите неравенства: %Диагност с300 1 вар 5-8
		\begin{tasks}(2)
			\task \( \log_3(2x^2+3x) \ge 3 \)
			\task \( \log_{\tfrac{1}{24}}(625-x^2)>-2 \)
			
			\task \( \log_{0,7}(2x+11)\le \log_{0,7}(4x-5) \)
		\end{tasks}
		\item Решите систему неравенств: %Диагност с300 1 вар 10-11 2 вар 10-11
			\[ \begin{cases} \log_{0,2}(2x-7) \le -2 \\ \log_3 (x-5) \le 3 \end{cases} \]
			%\task \( \begin{cases} \log_3 (25-x^2) \ge 2 \\ \log_5 (x^2+9) \ge 2 \end{cases} \)
			%\task \( \begin{cases} \log_7(3x+1) \ge 2 \\ \log_{0,5} (x-2) \ge -4 \end{cases} \)
			%\task \( \begin{cases} \log_4 (25-x^2) \ge 2 \\ \log_5 (x^2+16) \ge 2 \end{cases} \)
	\end{listofex}
\end{consultation}
%END_FOLD
