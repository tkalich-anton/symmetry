%
%===============>>  ГРУППА 10-2 МОДУЛЬ 7  <<=============
%
\setmodule{7}

%BEGIN_FOLD % ====>>_____ Занятие 1 _____<<====
\begin{class}[number=1]
	\begin{definit}
		Логарифмом положительного числа \(b\) по основанию \(a, \\ a>0 \), и \(a \neq 1\) называют число \( \alpha \), такое, что \(b=a^{c}\). Логарифм обозначают так: \[ c = \log_a b \]
	\end{definit}
	\begin{listofex}
		\item Вычислите:
		\begin{tasks}(3)
			\task \( \log_2 1 \)
			\task \( \log_{0,01} 0,01 \)
			\task \( \log_3 27 \)
			\task \( \log_5 125 \)
			\task \( \log_{10} 0,001 \)
			\task \( \log_4 1 \)
			\task \( \log_5 \dfrac{1}{5} \)
			\task \( \log_{10} 100 \)
			\task \( \log_5 5^3 \)
			\task \( \log_7 7^5 \)
		\end{tasks}
		\item Вычислите
		\begin{tasks}(3)
			\task \( 2^{\log_2 3} \)
			\task \( 3^{\log_3 5} \)
			\task \( 2^{\log_2 3 + \log_2 5} \)
			\task \( ( 3^{\log_3 7} )^2 \)
			\task \( 7^{2\log_7 3} \)
			\task \( 0,1^{2\log_{0,1} 10}  \)
			\task \( 9^{\log_3 12} \)
			\task \( 16^{\log_2 5} \)
			\task \( 49^{\log_7 \frac{1}{3}} \)
		\end{tasks}
	\end{listofex}
	\begin{definit}
		Логарифмом положительного числа \(b\) по основанию \(10\) называют \textbf{десятичым логарифмом числа \(b\)} и обозначают так: \( \lg b \).
	\end{definit}
	\begin{definit}
		Логарифмом положительного числа \(b\) по основанию \(e\) называют \textbf{натуральным логарифмом числа \(b\)} и обозначают так: \( \ln b \).
	\end{definit}
	\begin{listofex}[resume]
		\item Вычислите:
		\begin{tasks}(3)
			\task \( \log_{10} 10 \)
			\task \( \log_{10} 100 \)
			\task \( \lg 1000 \)
			\task \( \log_2 2^3 \)
			\task \( \log_5 5^7 \)
			\task \( \log_9 9^{1999} \)
			\task \( e^{\ln 3} \)
			\task \( e^{2\ln 5} \)
			\task \( e^{-2\ln 3} \)
			\task \( \ln e \)
			\task \( \ln e^3 \)
			\task \( \lg 10^n \)
			\task \( \ln \dfrac{1}{e} \)
			\task \( \lg \sqrt[3]{0,01} \)
			\task \( \lg 10^{-1} \)
		\end{tasks}
	\end{listofex}
	\begin{definit}
		Свойства логарифма:
		\[ \log_a(M \cdot N) = \log_a M + \log_a N \]
		\[ \log_a \dfrac{M}{N} = \log_a M - \log_a N \]
		\[ \log_{a^l} M^k=\dfrac{k}{l}\log_a M \]
	\end{definit}
	\begin{listofex}[resume]
		
		\item Вычислите:
		\begin{tasks}(2)
			\task \( \log_6 2 + \log_6 3 \)
			\task \( \log_{15} 5 + \log_{15} 3 \)
			\task \( \log_4 \dfrac{2}{3} + \log_4 6 \)
			\task \( \log_2 \dfrac{2}{5} + \log_2 10 \)
			\task \( \log_2 6 - \log_2 3 \)
			\task \( \log_5 75 - \log_5 3 \)
			\task \( \log_3 36 - \log_3 4 \)
			\task \( \log_3 0,81 - \log_3 0,03 \)
			\task \( 2\log_6 2 + \log_6 9 \)
			\task \( \log_{11} 484 - 2 \log_{11} 2 \)
		\end{tasks}
		\item Вычислите: 
		\[ \log_327-\log_{\sqrt{3}}27-\log_{1/3}27-\log_{\sqrt{3}/264}\left( \dfrac{64}{27} \right)  \]
	\end{listofex}
\end{class}
%END_FOLD

%BEGIN_FOLD % ====>>_____ Занятие 2 _____<<====
\begin{class}[number=2]
	\begin{definit}
		Для положительных чисел \( a, b, M \) таких, что \( a \neq 1, b \neq 1 \) справедливо следующее равенство: \[ \log_a M = \dfrac{\log_b M}{\log_b a} = \dfrac{1}{\log_M a} \]
	\end{definit}
	\begin{listofex}
		\item Выразите через логарифмы по основанию \(2\) и упростите:
		\begin{tasks}(4)
			\task \( \log_3 5 \)
			\task \( \log_4 9 \)
			\task \( \log_5 9 \)
			\task \( \log_{128} 8 \)
			\task \( \log_5 15 \)
			\task \( \log_3 12 \)
			\task \( \log_{16} 15 \)
			\task \( \log_{0,01} 2 \)
			\task \( \log_{0,25} 7 \)
			\task \( \log_{0,125} 3 \)
			\task \( \log_{\frac{1}{16}} 2 \)
			\task \( \log_{\frac{1}{32}} 5 \)
		\end{tasks}
	\end{listofex}
	\begin{definit}
		Свойства логарифма:
		\begin{tasks}(2)
			\task \( \log_a(M \cdot N) = \log_a M + \log_a N \)
			\task \( \log_a \dfrac{M}{N} = \log_a M - \log_a N \)
			\task \( \log_{a^l} M^k=\dfrac{k}{l}\log_a M \)
			\task \( \log_a b \cdot \log_b a = 1 \)
			\task \( a^{\log_b c}=c^{\log_b a} \)
		\end{tasks}
		
	\end{definit}
	\begin{listofex}[resume]
		\item Вычислите:
		\begin{tasks}(2)
			\task \( \log_{0,5}2 \)
			\task \( \log_{\frac{1}{2}}8^3 \)
			\task \( \log_{0,5}4^2 \)
			\task \( \log_3 \dfrac{1}{3} \)
			\task \( \log_3 \left(  \dfrac{1}{9} \right)^3 \)
			\task \( \log_4 \left(  \dfrac{1}{16} \right)^5 \)
			\task \( \log_2 \sqrt{2} + \log_{\sqrt{2}}2 \)
			\task \( \log_3 \sqrt{3^3} + \log^2_{\sqrt{3}}\sqrt{27} \)
			\task \( \log^2_5 \sqrt{5^5} - \log_{\sqrt{5}}5^3 \)
			\task \( 6^{\log_{36} 25} \)
			\task \( 7^{\log_{49} 36} \)
			\task \( 4^{\tfrac{1}{2\log_{625} 16 }} \)
			\task \( 36^{\log_6 2}:4^{\log_2 3} \)
			\task \( 9^{\log_3 5}+25^{\log_5 9} \)
			%\task \( 49^{\log_7 3} \)
			\task \( 8^{\log_2 36^{\log_6 2}} \)
			\task \( (\sqrt[3]{5})^{\log_5 2} \)
			\task \( \log_2 \sqrt[3]{16} + \log_{\sqrt{\frac{1}{16}}}4^{2} \)
			\task \( \log^2_3 (27\sqrt{3} - \log_{\sqrt{\frac{1}{3}}}9) \)
			\task \( \log_5 \sqrt{5\sqrt{5}} \)
		\end{tasks}
		\newpage
		\item Вычислите:
		\begin{tasks}(1)
			\task \( 3^{\log_{\sqrt[3]{9}}4} +2^{\tfrac{1}{\log_{16} 4}} \)
			\task \( \dfrac{\log_3 135}{\log_{15} 3} - \dfrac{\log_3 5}{\log_{405} 3} \)
			\task \( \dfrac{3+\log_{12}27}{3-\log_{12} 27} \cdot \log_6 16 \)
			%\task \( \log_3 27 - \log_{\sqrt{3}}27 - \log_{\frac{1}{3}} 27 - \log_{\frac{\sqrt{3}}{2}} \left( \dfrac{64}{27} \right)  \)
			\task \( \log_{0,4}(0,2 \cdot \sqrt[3]{50}) + \log_{0,5}\left( \dfrac{\sqrt{15}}{5} \right) + \log_{0,32} \left( \dfrac{2\sqrt{2}}{5} \right)  \)
			\task \( \left( \log_{0,5} \sqrt[3]{0,25} + 6\log_{0,25} 0,5 - 2 \log_{\frac{1}{16}}0,25 \right) : \log_{\sqrt{2}}\sqrt[5]{8} \)
		\end{tasks}
		
		\item Решите уравнения: %По первые 4 с решуегэ простейшие и сложнейшие
		\begin{tasks}(1)
			\task \( \log_2 (4-x)=7 \)
			\task \( \log_5(4+x)=2 \)
			\task \( \log_5(5-x)=\log_5 3 \)
			\task \( \log_2(15+x)=\log_2 3 \)
			\task \( \log_5 (2-x) = \log_{25} x^4 \)
			\task \( \log_2 (x^2-14x)=5 \)
			\task \( 6 \log^2_8 x -5\log_8 x+1=0 \)
			\task \( 1+\log_2(9x^2+5)=\log_{\sqrt{2}} \sqrt{8x^4+14} \)
		\end{tasks}
	\end{listofex}
\end{class}
%END_FOLD

%BEGIN_FOLD % ====>>_ Домашняя работа 1 _<<====
\begin{homework}[number=1]
	\begin{listofex}
		\item Вычислите:
		\begin{tasks}(3)
			\task \( \log_2 4 \)
			\task \( \log_2 16 \)
			\task \( \log_3 3 \)
			\task \( \log_3 27 \)
			\task \( \log_4 4^3 \)
			\task \( \log_7 49^8 \)
		\end{tasks}
		\item Вычислите
		\begin{tasks}(3)
			\task \( 2^{\log_2 5} \)
			\task \( 7^{\log_7 9} \)
			\task \( (3^2)^{\log_3 7} \)
			\task \( 10^{3\lg 5} \)
			\task \( 3^{\log_3 90} \)
			\task \( 5^{\log_5 0,5} \)
			\task \( 10^{\lg 3} \)
			\task \( 10^{2\lg 3} \)
			\task \( 10^{-3\lg 2} \)
			\end{tasks}
		\item Вычислите:
		\begin{tasks}(2)
			\task \( \log_3 0,9 + \log_3 30 \)
			\task \( \log_8 \dfrac{8}{7} + \log_8 \dfrac{7}{8} \)
			\task \( \log_4 48 - \log_4 3 \)
			\task \( \log_7 0,98 - \log_7 0,14 \)
			\task \( 4\log_{12} 2 + 2\log_{12} 3 \)
			\task \( \log_5 100 - 2 \log_5 2 \)
		\end{tasks}
		\item Решите уравнения: %По вторые 4 с решуегэ простейшие и сложнейшие
		\begin{tasks}(2)
			\task \( \log_4 (x+3) = \log_4 (4x-15) \)
			\task \( \log_{\tfrac{1}{7}}(7-x) = -2 \)
			\task \( \log_5 (5-x) = 2 \log_5 3 \)
			\task \( \log_5 (x^2+2x) = \log_5 (x^2+10) \)
			\task \( \log_7 (x+2) = \log_{49} (x^4) \)
			\task \( \log_3 (x^2-24x)=4 \)
			\task \( \log_3 (x^2-2x)=1 \)
			\task \( \log_2^2 (x-4) -6 \log_2 (x-4) = 7 \)
			%\task \( \log_2^2(x^2) - 16\log_2 (2x) +31=0 \)
		\end{tasks}
	\end{listofex}
\end{homework}
%END_FOLD

%BEGIN_FOLD % ====>>_____ Занятие 3 _____<<====
\begin{class}[number=3]
	\begin{listofex}
		\item 
	\end{listofex}
\end{class}
%END_FOLD

%BEGIN_FOLD % ====>>_____ Занятие 4 _____<<====
\begin{class}[number=4]
	\begin{listofex}
		\item Занятие 4
	\end{listofex}
\end{class}
%END_FOLD

%BEGIN_FOLD % ====>>_ Домашняя работа 2 _<<====
\begin{homework}[number=2]
	\begin{listofex}
		\item Домашняя работа 2
	\end{listofex}
\end{homework}
%END_FOLD

%BEGIN_FOLD % ====>>_____ Занятие 5 _____<<====
\begin{class}[number=5]
	\begin{listofex}
		\item Занятие 5
	\end{listofex}
\end{class}
%END_FOLD

%BEGIN_FOLD % ====>>_____ Занятие 6 _____<<====
\begin{class}[number=6]
	\begin{listofex}
		\item Занятие 6
	\end{listofex}
\end{class}
%END_FOLD

%BEGIN_FOLD % ====>>_ Домашняя работа 3 _<<====
\begin{homework}[number=3]
	\begin{listofex}
		\item Домашняя работа 3
	\end{listofex}
\end{homework}
%END_FOLD

%BEGIN_FOLD % ====>>_____ Занятие 7 _____<<====
\begin{class}[number=7]
	\title{Подготовка к проверочной}
	\begin{listofex}
		\item Решите неравенства:
	\begin{tasks}(1)
		\task \( \log_3 \dfrac{1}{x} + \log_3 (x^2+3x-9) \le \log_3 \left( x^2+3x+\dfrac{1}{2}-10 \right)  \)
		\task \( 9\log_6 (x^2+x-2) \le 10 + \log_7 \dfrac{(x-1)^9}{x+2} \)
	\end{tasks}
	\end{listofex}
\end{class}
%END_FOLD

=%BEGIN_FOLD % ====>>_ Проверочная работа _<<====
\begin{exam}
	\begin{listofex}
		\item Проверочная
	\end{listofex}
\end{exam}
%END_FOLD