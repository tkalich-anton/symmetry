%
%===============>>  ГРУППА 10-2 МОДУЛЬ 8  <<=============
%
\setmodule{8}

%BEGIN_FOLD % ====>>_____ Занятие 1 _____<<====
\begin{class}[number=1]
	\begin{listofex}
		\item Решите уравнения: %c109 21 24 25 a
		\begin{tasks}
			\task \((x^2-9)\lg(-x)=0  \)
			\task \( \log_2(x^2-7)=\log_{4-x}(4-x) \)
			\task \( \log_{11}(y^2-35)=\log_{7-y}1 \)
			%\task \( \dfrac{ \log_{x+1}^2 (x-1) + \log_5^2(2x-5) }{ \log_{x+1}^2 (x-1) + \log_5^2(x-2) }=1 \) НА ОЧЕНЬ ПОТОМ
			\task \( \log_2^2x-\log_2x^{-7}+12=0 \)
		\end{tasks}
		\item Решите системы неравенств: %диагн 2 с301 v2 10-11
		\begin{tasks}(2)
			\task \( \begin{cases} \log_{0,2}(2x-7) \le -2 \\ \log_3 (x-5) \le 3 \end{cases} \)
			\task \( \begin{cases} \log_3 (25-x^2) \ge 2 \\ \log_5 (x^2+9) \ge 2 \end{cases} \)
		\end{tasks}
		\item Решите неравенства: %c321 a 7 8 9 10
		\begin{tasks}
			\task \( \log_3(x+2)+\log_3(8-x)\le 1 + \log_3(x+4) \)
			\task \( \log_7(4x+11)-\log_7(25-x^2)\ge \sin \left( \dfrac{ 11\pi }{ 2 } \right) \)
			\task \( \log_3(x+5) \ge \log_{9-x}(9-x) \)
			\task \( 1-\dfrac{ 1 }{ \log_{x-4}0,2 } \le \dfrac{ 2 }{ \log_{x+20}25 } \)
		\end{tasks}
		
	\end{listofex}
\end{class}
%END_FOLD

%BEGIN_FOLD % ====>>_____ Занятие 2 _____<<====
\begin{class}[number=2]
	\begin{listofex}
		\item Решите уравнения: %109 б 8 10 11 15 17 trening10
		\begin{tasks}(2)
			\task \( \log_2^2 x - \log_2 x^{-7} + 12 = 0 \)
			\task \( \log_{y^2}17=\log_{4+3y}17 \)
			\task \( \dfrac{ (x-9)(x-7) }{ \log_3(x+8) }=0 \)
			%\task \( \dfrac{ (x-11)(x+21) }{ \log_{17}(x+13) }=0 \)
			\task \( \log_8(x^2-18)=\log_8(3x) \)
			\task \( \log_{17}(x^2-24)=\log_{6-x}1 \)
		\end{tasks}
		\item Решите неравенства: %c321 a 7 8 9 10
		\begin{tasks}
			\task \( \log_3(x+2)+\log_3(8-x)\le 1 + \log_3(x+4) \)
			\task \( \log_7(4x+11)-\log_7(25-x^2)\ge \sin \left( \dfrac{ 11\pi }{ 2 } \right) \)
			\task \( \log_3(x+5) \ge \log_{9-x}(9-x) \)
			\task \( 1-\dfrac{ 1 }{ \log_{x-4}0,2 } \le \dfrac{ 2 }{ \log_{x+20}25 } \)
		\end{tasks}
		\item Решите неравенства: \( \dfrac{ 4^x-2^{x+3}+7 }{ 4^x-5\cdot 2^x+4 } \le \dfrac{ 2^x-9 }{ 2^x-4 }+\dfrac{ 1 }{ 2^x-6 } \).
	\end{listofex}
\end{class}
%END_FOLD

%BEGIN_FOLD % ====>>_ Домашняя работа 1 _<<====
\begin{homework}[number=1]
	\begin{listofex}
		\item Решите уравнения: %109 b 12 16 20 22
		\begin{tasks}
			\task \( (x+5)(x+2)\log_{(-1-x)}(x+9)=0 \)
			\task \( (x-8)\log_9(x-8)=0 \)
			\task \( \log_{2-3x}(x^2-8)=0 \)
			\task \( (x+3)\lg(x-7)=0 \)
		\end{tasks}
		\item Решите неравенства: %321 b 7 8 9 10
		\begin{tasks}
			\task \( \log_3 (x+3)+\log_3 (7-x) \le 1 + \log_3 (x+5) \)
			\task \( \log_2 (3x-2) - \log_7 (25-x^2) \ge \sin \dfrac{ 15\pi }{ 2 } \)
			\task \( \log_4 (x+8) \ge \log_{3-x} (3-x) \)
			\task \( 1-\dfrac{ 1 }{ \log_{x-1}0,1 } \le \dfrac{ 2 }{ \log_{x+17} 100 } \)
		\end{tasks}
	\end{listofex}
\end{homework}
%END_FOLD

%BEGIN_FOLD % ====>>_____ Занятие 3 _____<<====
\begin{class}[number=3]
	\begin{listofex}
		\item При каких значениях параметра \(b\) уравнение \(x^2+bx+4=0\):
		\begin{tasks}
			\task имеет один из корней, равный \( 3 \);
			\task имеет действительные различные корни;
			\task имеет один корень;
			\task не имеет действительных корней?
		\end{tasks}
		\item При каких значениях \(m\) ровно один из корней уравнения равен нулю:
		\begin{tasks}(2)
			\task \( 3x^2+x+2m-3=0 \)
			\task \( 2x^2-mx+2m^2-3m=0 \)
		\end{tasks}
		\item При каких значениях \(m\) оба корня уравнения равны нулю: \[ 3x^2+(m-1)x+1-m^2=0 \]
		\item При каких значениях \(k\) произведение корней квадратного уравнения \[x^2+3x+(k^2-7k+12)=0\] равно нулю?
		\item При каких значениях \(k\) сумма корней квадратного уравнения \[x^2+(k^2+4k-5)x-k=0\] равна нулю?
		\item При каком значении \(a\) уравнение \[ax^2-(a+1)x+2a-1=0\] имеет один корень?
		\item Найдите наименьшее целое значение \(a\), при котором уравнение \[x^2-2(a+2)x+12+a^2=0\] имеет два различных действительных корня.
		\item При каком значении \(a\) уравнение \[(a+2)x^2+2(a+2)x+2=0\] имеет один корень?
	\end{listofex}
\end{class}
%END_FOLD

%BEGIN_FOLD % ====>>_____ Занятие 4 _____<<====
\begin{class}[number=4]
	\begin{listofex}
		\item При каком значении \(a\) уравнение \[(a+2)x^2+2(a+2)x+2=0\] имеет один корень?
	\end{listofex}
\end{class}
%END_FOLD

%BEGIN_FOLD % ====>>_ Домашняя работа 2 _<<====
\begin{homework}[number=2]
	\begin{listofex}
		\item Домашняя работа 2
	\end{listofex}
\end{homework}
%END_FOLD

%BEGIN_FOLD % ====>>_____ Занятие 5 _____<<====
\begin{class}[number=5]
	\begin{listofex}
		\item Занятие 5
	\end{listofex}
\end{class}
%END_FOLD

%BEGIN_FOLD % ====>>_____ Занятие 6 _____<<====
\begin{class}[number=6]
	\begin{listofex}
		\item Занятие 6
	\end{listofex}
\end{class}
%END_FOLD

%BEGIN_FOLD % ====>>_ Домашняя работа 3 _<<====
\begin{homework}[number=3]
	\begin{listofex}
		\item Домашняя работа 3
	\end{listofex}
\end{homework}
%END_FOLD

%BEGIN_FOLD % ====>>_____ Занятие 7 _____<<====
\begin{class}[number=7]
	\title{Подготовка к проверочной}
	\begin{listofex}
		\item Занятие 7
	\end{listofex}
\end{class}
%END_FOLD

=%BEGIN_FOLD % ====>>_ Проверочная работа _<<====
\begin{exam}
	\begin{listofex}
		\item Проверочная
	\end{listofex}
\end{exam}
%END_FOLD