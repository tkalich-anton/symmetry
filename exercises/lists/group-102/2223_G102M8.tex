%
%===============>>  ГРУППА 10-2 МОДУЛЬ 8  <<=============
%
\setmodule{8}

%BEGIN_FOLD % ====>>_____ Занятие 1 _____<<====
\begin{class}[number=1]
	\begin{listofex}
		\item Занятие 1
	\end{listofex}
\end{class}
%END_FOLD

%BEGIN_FOLD % ====>>_____ Занятие 2 _____<<====
\begin{class}[number=2]
	\begin{listofex}
		\item Занятие 2
	\end{listofex}
\end{class}
%END_FOLD

%BEGIN_FOLD % ====>>_ Домашняя работа 1 _<<====
\begin{homework}[number=1]
	\begin{listofex}
		\item Домашняя работа 1
	\end{listofex}
\end{homework}
%END_FOLD

%BEGIN_FOLD % ====>>_____ Занятие 3 _____<<====
\begin{class}[number=3]
	\begin{listofex}
		\item Занятие 3 
	\end{listofex}
\end{class}
%END_FOLD

%BEGIN_FOLD % ====>>_____ Занятие 4 _____<<====
\begin{class}[number=4]
	\begin{listofex}
		\item При каком значении \( a \) уравнение
		\[ (a+2)x^2+2(a+2)x+2=0 \]
		имеет один корень?
		\item При каком значении \( a \) уравнение
		\[ (a+4)x^2+2(a+5)x+2=0 \]
		имеет один корень?
		\item Найдите все значения параметра \( a \), при каждом из которых уравнение \( ax^2+4x+a=3 \) имеет более одного корня.
		\item Найдите все значения параметра \( a \), при каждом из которых неравенство
		\[ (a^2-1)x^2+2(a-1)x+1>0 \]
		выполнено при любом значении \( x \).
		\item Найдите все значения параметра \( a \), при каждом из которых уравнение \( x^2+2(a^2-6a-3)x+16=0 \) имеет два различных отрицательных корня.
		\item Найдите все значения параметра \( a \), при каждом из которых уравнение \( ax^2-(a+1)x+2a^2-5a-3=0 \) имеет два корня разных знаков.
	\end{listofex}
\end{class}
%END_FOLD

%BEGIN_FOLD % ====>>_ Домашняя работа 2 _<<====
\begin{homework}[number=2]
	\begin{listofex}
		\item Домашняя работа 2
	\end{listofex}
\end{homework}
%END_FOLD

%BEGIN_FOLD % ====>>_____ Занятие 5 _____<<====
\begin{class}[number=5]
	\begin{listofex}
		\item При каких значения параметра \( a \) уравнение \( x^2-a=0 \) имеет:\\
		а) ровно \( 1 \) корень;\\
		б) ровно 2 различных корня.
		\item При каких значения параметра \( a \) уравнение \( xa-1=0 \) не имеет корней.
		\item При каких значения параметра \( a \) уравнение \( x^2-x-a=0 \) имеет хотя бы одно решение, удовлетворяющее неравенству \( x>\dfrac{1}{2} \).
		\item Решите уравнение \( |x-3|+|x-7|=9 \)
		\item Решите уравнение \( |x-3|+|x-7|=4 \)
		\item Для каждого значения \( a \) решить уравнение
		\[ |x+a|+|x-a|=2 \]
		\item Для каждого значения \( a \) решить уравнение
		\[ |x^2-1|+|x^2-4|=a \]
	\end{listofex}
\end{class}
%END_FOLD

%BEGIN_FOLD % ====>>_____ Занятие 6 _____<<====
\begin{class}[number=6]
	\begin{listofex}
		\item При каких значения параметра \( a \) уравнение \( x^2+2a-2=0 \) имеет:\\
		а) ровно \( 1 \) корень;\\
		б) ровно \( 2 \) различных корня.
		\item Найдите все значения параметра \( a \), при которых уравнение
		\[ (2-x)(x+1)=a \]
		имеет два различных неотрицательных решения.
		\item При каких значения параметра \( a \) уравнение
		\[ x^2+|x|+a=0 \]
		имеет два различных решения?
		\item При каких значения параметра \( a \) уравнение
		\[ x^2-4x-2|x-a|+2+a=0 \]
		имеет ровно два решения?
		\item При каких значения параметра \( a \) уравнение
		\[ 3(x-2)^2 - |x-2|+a=0 \]
		имеет ровно два различных решения?
%		\item Для каждого значения \( a \) решить уравнение
%		\[ |x^2-1|+|x^2-4|=a \]
	\end{listofex}
\end{class}
%END_FOLD

%BEGIN_FOLD % ====>>_ Домашняя работа 3 _<<====
\begin{homework}[number=3]
	\begin{listofex}
		\item Домашняя работа 3
	\end{listofex}
\end{homework}
%END_FOLD

%BEGIN_FOLD % ====>>_____ Занятие 7 _____<<====
\begin{class}[number=7]
	\title{Подготовка к проверочной}
	\begin{listofex}
		\item Занятие 7
	\end{listofex}
\end{class}
%END_FOLD

=%BEGIN_FOLD % ====>>_ Проверочная работа _<<====
\begin{exam}
	\begin{listofex}
		\item Проверочная
	\end{listofex}
\end{exam}
%END_FOLD