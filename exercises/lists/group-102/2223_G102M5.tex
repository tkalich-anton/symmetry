%
%===============>>  ГРУППА 10-2 МОДУЛЬ 5  <<=============
%
\setmodule{5}

%BEGIN_FOLD % ====>>_____ Занятие 1 _____<<====
\begin{class}[number=1]
	\begin{definit}
		Сечением фигуры плоскостью является многоугольник, стороны которого принадлежат граням многогранника.
	\end{definit}
	\begin{definit}
		Соединяйте точки, которые лежат в одной грани (в плоскости этой грани).
	\end{definit}
	\begin{definit}
		Плоскость сечения пересекает параллельные грани по параллельным прямым.
	\end{definit}
	\begin{definit}
		Если плоскость сечения проходит через прямую,
		параллельную какой-то плоскости, то секущая плоскость
		пересечёт данную плоскость по параллельной прямой.
	\end{definit}
	\begin{listofex}
		\item Постройте сечение треугольной пирамиды \( DABC \) плоскостью,
		проходящей через следующие точки:
		\begin{tasks}(1)
			\task \( B \), \( D \) и середину \( M \) ребра \( AC \);
			\task середину \( K \) ребра \( AD \) и точки \( L \) и \( M \), лежащие на продолжениях
			рёбер \( AB \) и \( AC \) за точки \( B \) и \( C \);
			\task середины \( K \), \( L \) и \( M \) рёбер AD, AB и BC;
		\end{tasks}
		\item Постройте сечение параллелепипеда \( ABCDA_1B_1C_1D_1 \)
		плоскостью, проходящей через следующие точки:
		\begin{tasks}(1)
			\task середины рёбер \( AB \), \( AD \) и \( AA_1 \);
			\task \( A \), \( C \) и середину ребра \( A_1B_1 \);
			\task середины рёбер \( AA_1 \), \( AD \) и центр грани \( BB_1C_1C \);
			\task середину ребра \( CC_1 \) и точки \( K \), \( L \), лежащие на рёбрах \( AB \) и \( A_1B_1 \),
			если \( BK : KA= A_1L : LB_1=1 : 2 \);
		\end{tasks}
		\item Основание пирамиды \( SABCD \) --- параллелограмм \( ABCD \).
		Постройте сечение пирамиды плоскостью, проходящей через следующие
		точки:
		\begin{tasks}(1)
			\task \( A \), \( B \) и середина ребра \( SD \);
			\task середины рёбер \( AB \), \( BC \) и \( SC \);
			\task середины рёбер \( AB \), \( BC \) и \( SD \);
			\task центр основания, середину ребра \( SD \) и точку \( M \) ребра \( SA \),
			если \( AM : MS = 1 : 3 \);
		\end{tasks}
		\newpage
		\item Основание шестиугольной призмы \( ABCDEFA_1B_1C_1D_1E_1F_1 \) ---
		правильный шестиугольник \( ABCDEF \).
		Постройте сечение призмы плоскостью,
		проходящей через следующие точки:
		\begin{tasks}(1)
			\task \( A \), \( B \) и \( F_1 \);
			\task \( A \), \( C \) и \( D_1 \);
			\task \( B \), \( E \) и середину ребра \( FF_1 \);
			\task \( B \), \( C \) и \( E_1 \);
			\task \( B \), \( D \) и середину ребра \( FF_1 \).
	\end{tasks}
	\end{listofex}
\end{class}
%END_FOLD

%BEGIN_FOLD % ====>>_____ Занятие 2 _____<<====
\begin{class}[number=2]
	\begin{listofex}
		\item Занятие 2
	\end{listofex}
\end{class}
%END_FOLD

%BEGIN_FOLD % ====>>_____ Занятие 3 _____<<====
\begin{class}[number=3]
	\begin{listofex}
		\item Занятие 3
	\end{listofex}
\end{class}
%END_FOLD

%BEGIN_FOLD % ====>>_____ Занятие 4 _____<<====
\begin{class}[number=4]
	\begin{listofex}
		\item Занятие 4
	\end{listofex}
\end{class}
%END_FOLD

%BEGIN_FOLD % ====>>_____ Занятие 5 _____<<====
\begin{class}[number=5]
	\begin{listofex}
		\item Занятие 5
	\end{listofex}
\end{class}
%END_FOLD

%BEGIN_FOLD % ====>>_____ Занятие 6 _____<<====
\begin{class}[number=6]
	\begin{listofex}
		\item Занятие 6
	\end{listofex}
\end{class}
%END_FOLD

%BEGIN_FOLD % ====>>_____ Занятие 7 _____<<====
\begin{class}[number=7]
	\begin{listofex}
		\item Занятие 7
	\end{listofex}
\end{class}
%END_FOLD

%BEGIN_FOLD % ====>>_ Домашняя работа 1 _<<====
\begin{homework}[number=1]
	\begin{listofex}
		\item ДЗ 1
	\end{listofex}
\end{homework}
%END_FOLD

%BEGIN_FOLD % ====>>_ Домашняя работа 2 _<<====
\begin{homework}[number=2]
	\begin{listofex}
		\item ДЗ 2
	\end{listofex}
\end{homework}
%END_FOLD

%BEGIN_FOLD % ====>>_ Домашняя работа 3 _<<====
\begin{homework}[number=3]
	\begin{listofex}
		\item ДЗ 3
	\end{listofex}
\end{homework}
%END_FOLD

%BEGIN_FOLD % ====>>_ Проверочная работа _<<====
\begin{exam}
	\begin{listofex}
		\item Проверочная работа
	\end{listofex}
\end{exam}
%END_FOLD

%BEGIN_FOLD ====>>_ Консультация 27.12 _<<====
\begin{consultation}
	\begin{listofex}
		\item Постройте сечение треугольной пирамиды \(SABC\) плоскостью, проходящей через вершину \(B\) и середины ребер \(SA\) и \(SC\).
		\item Постройте сечение треугольной пирамиды \(SABC\) плоскостью, проходящей через вершину \(A\), середину ребра \(BS\) параллельно ребру \(BC\).
		\item  Постройте сечение треугольной пирамиды \(SABC\) плоскостью, проходящей через точку \(M\), лежащую на ребре \(AS\), точку \(N\), лежащую на ребре \(CS\), точку \(K\) лежащую на ребре \(BC\). Если \(AM:MS=1:2\), \(SN:NC=1:2\), \(BK:KC=1:2\)
		\item  Постройте сечение параллелепипеда \(ABCDA_1B_1C_1D_1\) плоскостью, проходящей середины рёбер \(A_1B_1\), \(CC_1\) и вершину \(A\).
		\item Основание пирамиды \(SABCD\) — параллелограмм \(ABCD\). Постройте сечение пирамиды плоскостью, проходящей середину ребра \(SA\) и точки \(M\) и \(N\) рёбер \(SB\) и \(SC\), если \(BM:MS=SN:NC =1:3\).
		\item  Основания шестиугольной призмы \(ABCDEF\) и \(A_1B_1C_1E_1F_1\) --- правильные шестиугольники. Точка \(M\) --- середина ребра \(CC_1\), \(O\) --- центр грани \(ABCDEF\). Постройте сечение призмы плоскостью, проходящей через точки \(M\), \(O\) и \(E_1\).
		\item  Дана правильная четырёхугольная пирамида \(SABCD\) с вершиной \(S\). Точки \(M\) и \(N\) —середины рёбер \(AB\) и \(SC\). Постройте сечение пирамиды плоскостью, проходящей через прямую \(MN\) параллельно \(SA\).
		\newpage
		
		\item Решить уравнение:
		\begin{tasks}(2)
			\task \( \sin x = \dfrac{1}{2} \)
			\task \( \cos x = -1 \)
			\task \( 2\sin x = -\sqrt{2} \)
			\task \( 3\tg x = \sqrt{3} \)
			\task \( \sin^2 x = \dfrac{1}{2} \)
		\end{tasks}
		\item Решить неравенство:
		\begin{tasks}(2)
			\task \( \sin x > \dfrac{1}{2} \)
			\task \( \cos x \le \dfrac{\sqrt{2}}{2} \)
			\task \( \sin x < 1 \)
			\task \( \sin x \ge -\dfrac{3}{2}\)
		\end{tasks}
	\end{listofex}
\end{consultation}
%END_FOLD