%
%===============>>  ГРУППА 10-2 МОДУЛЬ 5  <<=============
%
\setmodule{5}

%BEGIN_FOLD % ====>>_____ Занятие 1 _____<<====
\begin{class}[number=1]
	\begin{listofex}
		\item Занятие 1
	\end{listofex}
\end{class}
%END_FOLD

%BEGIN_FOLD % ====>>_____ Занятие 2 _____<<====
\begin{class}[number=2]
	\begin{listofex}
		%1
		\item Точка \(M\) лежит на ребре \(AB\) треугольной пирамиды \(ABCD\), причём \(AM : MB = 1:2\). а) Постройте сечение пирамиды плоскостью, проходящей через точку \(M\) и середины рёбер \(BC\) и \(AD\). б) В каком отношении плоскость сечения делит ребро \(CD\)?
		%2
		\item Точка \(M\) — середина ребра \(AD\) треугольной пирамиды \(ABCD\). Точки \(K\) и \(L\) лежат на прямых \(AB\) и \(AC\) соответственно, причём \(B\) --- середина отрезка \(AK\), а \(C\) — середина отрезка \(AL\). а) Постройте сечение пирамиды плоскостью, проходящей через точки \(M, K, L\). б) В каком отношении плоскость сечения делит ребро \(BD\)?
		%3
		\item Точки \(M\) и \(N\) — середины рёбер соответственно \(AB\) и \(BC\) параллелепипеда \(ABCDA_1B_1C_1D_1\).
		а) Постройте сечение параллелепипеда плоскостью, проходящей через точки \(M, N, D_1\). б) В каком отношении плоскость сечения делит ребро \(AA_1\)?
		%4
		\item Точка \(M\) — середина ребра \(CD\) параллелепипеда \(ABCDA_1B_1C_1D_1\). а) Постройте сечение параллелепипеда плоскостью, проходящей через точки \(M, A, C\). б) Пусть секущая плоскость пересекает прямую \(DD_1\) в точке \(P\). Найдите отношение \(PD:PDV\)
		%5
		\item Основание пирамиды \(SABCD\) — параллелограмм \(ABCD\) с центром \(O\). Точка \(M\) лежит на отрезке \(SO\), причём \(OM:MS = 1:2\). а) Постройте сечение пирамиды плоскостью, проходящей через прямую AM параллельно прямой \(BD\). б) В каком отношении плоскость сечения делит ребро \(SC\)?
		%6
		\item Основание пирамиды \(SABCD\) — параллелограмм \(ABCD\) с центром \(O\). Точка \(M\) — середина отрезка \(AO\). а) Постройте сечение пирамиды плоскостью, проходящей через точку \(M\) параллельно прямым \(SA\) и \(BD\). б) В каком отношении плоскость сечения делит ребро \(SC\)?
	\end{listofex}
\end{class}
%END_FOLD

%BEGIN_FOLD % ====>>_____ Занятие 3 _____<<====
\begin{class}[number=3]
	\begin{listofex}
		\item Занятие 3
	\end{listofex}
\end{class}
%END_FOLD

%BEGIN_FOLD % ====>>_____ Занятие 4 _____<<====
\begin{class}[number=4]
	\begin{listofex}
		\item Занятие 4
	\end{listofex}
\end{class}
%END_FOLD

%BEGIN_FOLD % ====>>_____ Занятие 5 _____<<====
\begin{class}[number=5]
	\begin{listofex}
		\item Занятие 5
	\end{listofex}
\end{class}
%END_FOLD

%BEGIN_FOLD % ====>>_____ Занятие 6 _____<<====
\begin{class}[number=6]
	\begin{listofex}
		\item Занятие 6
	\end{listofex}
\end{class}
%END_FOLD

%BEGIN_FOLD % ====>>_____ Занятие 7 _____<<====
\begin{class}[number=7]
	\begin{listofex}
		\item Занятие 7
	\end{listofex}
\end{class}
%END_FOLD

%BEGIN_FOLD % ====>>_ Домашняя работа 1 _<<====
\begin{homework}[number=1]
	\begin{listofex}
		\item Дана треугольная пирамида \(ABCD\). Точка \(M\) лежит на ребре \(BC\), причём \(BM:MC = 1:2\). Постройте точку пересечения прямой, проходящей через точку \(M\) и середину ребра \(CD\), с плоскостью \(ABD\).
		\item Основание пирамиды \(SABCDEF\) --- шестиугольник \(ABCDEF\), противоположные стороны \(BC\) и \(EF\) которого параллельны. Точка \(M\) лежит на ребре \(SC\). Постройте точку пересечения прямой \(BM\) с плоскостью \(ESF\).
		\item Основание пирамиды \(SABCD\) --- параллелограмм \(ABCD\). Постройте сечение пирамиды плоскостью, проходящей через следующие точки:
		\begin{itasks}[1]
			\task \(A, B\) и середина ребра \(SD\);
			\task середины рёбер \(AB, BC, SC\);
			\task середины рёбер \(AB, BC, SD\);
			\task середины рёбер \(AB, AD\) параллельно ребру \(SC\);
			\task середины рёбер \(AB, BC, SD\) и точку \(B\);
			\task середины рёбер \(AB, AD, SC\);
			\task центр основания, середину ребра \(SD\) и точку \(M\) ребра \(SA\), если \(AM:MS = 1:3\);
			\task середину ребра \(SA\) и точки \(M\) и \(N\) рёбер \(SB\) и \(SC\), если \(BM : MS = = SN :NC=1:2\)
		\end{itasks}
		\item Точка \(M\) — середина ребра \(AB\) треугольной призмы \(ABCA_1B_1C_1\). а) Постройте сечение призмы плоскостью, проходящей через прямую \(A_1M\) параллельно прямой \(AC\). б) В каком отношении плоскость сечения делит отрезок, соединяющий точку \(B_1\) с серединой ребра \(AC\)?
	\end{listofex}
\end{homework}
%END_FOLD

%BEGIN_FOLD % ====>>_ Домашняя работа 2 _<<====
\begin{homework}[number=2]
	\begin{listofex}
		\item ДЗ 2
	\end{listofex}
\end{homework}
%END_FOLD

%BEGIN_FOLD % ====>>_ Домашняя работа 3 _<<====
\begin{homework}[number=3]
	\begin{listofex}
		\item ДЗ 3
	\end{listofex}
\end{homework}
%END_FOLD

%BEGIN_FOLD % ====>>_ Проверочная работа _<<====
\begin{exam}
	\begin{listofex}
		\item Проверочная работа
	\end{listofex}
\end{exam}
%END_FOLD

%BEGIN_FOLD ====>>_ Консультация 27.12 _<<====
\begin{consultation}
	\begin{listofex}
		\item Постройте сечение треугольной пирамиды \(SABC\) плоскостью, проходящей через вершину \(B\) и середины ребер \(SA\) и \(SC\).
		\item Постройте сечение треугольной пирамиды \(SABC\) плоскостью, проходящей через вершину \(A\), середину ребра \(BS\) параллельно ребру \(BC\).
		\item  Постройте сечение треугольной пирамиды \(SABC\) плоскостью, проходящей через точку \(M\), лежащую на ребре \(AS\), точку \(N\), лежащую на ребре \(CS\), точку \(K\) лежащую на ребре \(BC\). Если \(AM:MS=1:2\), \(SN:NC=1:2\), \(BK:KC=1:2\)
		\item  Постройте сечение параллелепипеда \(ABCDA_1B_1C_1D_1\) плоскостью, проходящей середины рёбер \(A_1B_1\), \(CC_1\) и вершину \(A\).
		\item Основание пирамиды \(SABCD\) — параллелограмм \(ABCD\). Постройте сечение пирамиды плоскостью, проходящей середину ребра \(SA\) и точки \(M\) и \(N\) рёбер \(SB\) и \(SC\), если \(BM:MS=SN:NC =1:3\).
		\item  Основания шестиугольной призмы \(ABCDEF\) и \(A_1B_1C_1E_1F_1\) --- правильные шестиугольники. Точка \(M\) —середина ребра \(CC_1\), \(O\) --- центр грани \(ABCDEF\). Постройте сечение призмы плоскостью, проходящей через точки \(M\), \(O\) и \(E_1\).
		\item  Дана правильная четырёхугольная пирамида \(SABCD\) с вершиной \(S\). Точки \(M\) и \(N\) —середины рёбер \(AB\) и \(SC\). Постройте сечение пирамиды плоскостью, проходящей через прямую \(MN\) параллельно \(SA\).
	\end{listofex}
\end{consultation}
%END_FOLD