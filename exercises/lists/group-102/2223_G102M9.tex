\setmodule{9}

%BEGIN_FOLD % ====>>_____ Занятие 1 _____<<====
\begin{class}[number=1]
	\begin{listofex}
		%G111M6L1 L2
		\item Площадь грани прямоугольного параллелепипеда равна \( 15 \). Ребро, перпендикулярное этой грани, равно \(3\). Найдите объем параллелепипеда.
		\item Три ребра прямоугольного параллелепипеда, выходящие из одной вершины, равны \(4, 6, 9\). Найдите ребро равновеликого ему куба.
		\item Два ребра прямоугольного параллелепипеда, выходящие из одной вершины, равны \(3\) и \(4\). Площадь поверхности этого параллелепипеда равна \(94\). Найдите третье ребро, выходящее из той же вершины.
		\item Объем прямоугольного параллелепипеда равен \(24\). Одно из его ребер равно \(3\). Найдите площадь грани параллелепипеда, перпендикулярной этому ребру.
		\item Прямоугольный параллелепипед описан около сферы радиуса \(1\). Найдите его площадь поверхности.
		\item Диагональ куба равна \( 2\sqrt{3} \). Найдите объем куба и площадь его поверхности.
		\item Объем первого куба в \( 8 \) раз больше объема второго куба. Во сколько раз площадь поверхности первого куба больше площади поверхности второго куба?
		\item Найдите площадь боковой поверхности правильной шестиугольной призмы, сторона основания которой равна \( 5 \), а высота  --- \( 10 \).
		\item Дан куб \( ABCDA_1B_1C_1D_1 \). Площадь четырехугольника \( ABC_1D_1 \) равна \( 4\sqrt{2} \). Найдите площадь поверхности куба.
		\item 
		\begin{minipage}[t]{\bodywidth}
			В правильной треугольной пирамиде \(SABC\) с вершиной \(S\) биссектрисы треугольника \(ABC\) пересекаются в точке \(O\). Площадь треугольника \(ABC\) равна \(2\); объем пирамиды равен \(6\). Найдите длину отрезка \(OS\).
		\end{minipage}
		\hspace{0.02\linewidth}
		\begin{minipage}[t]{\picwidth}
			\includegraphics[align=t, width=\linewidth]{\picpath/G111M6L1-1}
		\end{minipage}
		\item В правильной четырехугольной пирамиде \(SABCD\) точка \(O\) --- центр основания, \(S\) --- вершина, \(SO=15, BD=16\). Найдите боковое ребро \(SA\).
		
		\item 
		\begin{minipage}[t]{\bodywidth}
			В правильной треугольной пирамиде \(SABC\) точка \(M\) --- середина ребра \(AB\), \(S\) --- вершина. Известно, что \(BC = 3\), а площадь боковой поверхности пирамиды равна \(45\). Найдите длину отрезка \(SM\).
		\end{minipage}
		\hspace{0.02\linewidth}
		\begin{minipage}[t]{\picwidth}
			\includegraphics[align=t, width=\linewidth]{\picpath/G111M6L1-2}
		\end{minipage}
		\item Объем параллелепипеда \(ABCDA_1B_1C_1D_1\) равен \(9\). Найдите объем треугольной пирамиды \(ABCA_1\).
		\item Во сколько раз увеличится объем правильного тетраэдра, если все его ребра увеличить в два раза?
		\item В треугольнике \(ABC\) \(AB = 10, AC = BC\), высота \(AH = 8\). Найдите \(\cos{BAC}\).
		\item В треугольнике со сторонами \(9\) и \(6\) проведены высоты к этим сторонам. Высота, проведённая к первой из этих сторон, равна \(4\). Чему равна высота, проведённая ко второй стороне?
		\item Площадь параллелограмма \(ABCD\) равна \(24\). Точка \(M\) --- середина стороны \(BC\). Найдите площадь трапеции \(AMCD\).
		\item 
		\begin{minipage}[t]{\bodywidth}
			На рисунке изображен многогранник, все двугранные углы многогранника прямые. Найдите квадрат расстояния между вершинами \(B_2\) и \(D_3\) .
		\end{minipage}
		\hspace{0.02\linewidth}
		\begin{minipage}[t]{\picwidth}
			\includegraphics[align=t, width=\linewidth]{\picpath/G101M5L6-2}
		\end{minipage}
		\item 
		\begin{minipage}[t]{\bodywidth}
			На рисунке изображен многогранник, все двугранные углы многогранника прямые. Найдите его объём и площадь поверхности.
		\end{minipage}
		\hspace{0.02\linewidth}
		\begin{minipage}[t]{\picwidth}
			\includegraphics[align=t, width=\linewidth]{\picpath/G101M5L7-2}
		\end{minipage}
	\end{listofex}
\end{class}
%END_FOLD

%BEGIN_FOLD % ====>>_____ Занятие 2 _____<<====
\begin{class}[number=2]
	\begin{listofex}
		\item Занятие 2
	\end{listofex}
\end{class}
%END_FOLD

%BEGIN_FOLD % ====>>_ Домашняя работа 1 _<<====
\begin{homework}[number=1]
	\begin{listofex}
		\item Домашняя работа 1
	\end{listofex}
\end{homework}
%END_FOLD

%BEGIN_FOLD % ====>>_____ Занятие 3 _____<<====
\begin{class}[number=3]
	\begin{listofex}
		\item Занятие 3 
	\end{listofex}
\end{class}
%END_FOLD

%BEGIN_FOLD % ====>>_____ Занятие 4 _____<<====
\begin{class}[number=4]
	\begin{listofex}
		\item Занятие 4
	\end{listofex}
\end{class}
%END_FOLD

%BEGIN_FOLD % ====>>_ Домашняя работа 2 _<<====
\begin{homework}[number=2]
	\begin{listofex}
		\item Домашняя работа 2
	\end{listofex}
\end{homework}
%END_FOLD

%BEGIN_FOLD % ====>>_____ Занятие 5 _____<<====
\begin{class}[number=5]
	\begin{listofex}
		\item Занятие 5
	\end{listofex}
\end{class}
%END_FOLD

%BEGIN_FOLD % ====>>_____ Занятие 6 _____<<====
\begin{class}[number=6]
	\begin{listofex}
		\item Занятие 6
	\end{listofex}
\end{class}
%END_FOLD

%BEGIN_FOLD % ====>>_ Домашняя работа 3 _<<====
\begin{homework}[number=3]
	\begin{listofex}
		\item Домашняя работа 3
	\end{listofex}
\end{homework}
%END_FOLD

%BEGIN_FOLD % ====>>_____ Занятие 7 _____<<====
\begin{class}[number=7]
	\title{Подготовка к проверочной}
	\begin{listofex}
		\item Занятие 7
	\end{listofex}
\end{class}
%END_FOLD

%BEGIN_FOLD % ====>>_ Проверочная работа _<<====
\begin{exam}
	\begin{listofex}
		\item Проверочная
	\end{listofex}
\end{exam}
%END_FOLD


%BEGIN_FOLD % ====>>_ Консультация _<<====
\begin{consultation}
	\begin{listofex}
		\item Вероятность того, что в случайный момент времени температура тела здорового человека окажется ниже чем \( 36,8\degree C \), равна \( 0,81 \). Найдите вероятность того, что в случайный момент времени у здорового человека температура окажется \( 36,8\degree C \) или выше.
		\item При изготовлении подшипников диаметром \( 67 \) мм вероятность того, что диаметр будет отличаться от заданного не больше, чем на \( 0,01 \) мм, равна \( 0,965 \). Найдите вероятность того, что случайный подшипник будет иметь диаметр меньше чем \( 66,99 \) мм или больше чем \( 67,01 \) мм.
		\item Вероятность того, что батарейка бракованная, равна \( 0,06 \). Покупатель в магазине выбирает случайную упаковку, в которой две таких батарейки. Найдите вероятность того, что обе батарейки окажутся исправными.
		\item Вероятность того, что новый электрический чайник прослужит больше года, равна \( 0,93 \). Вероятность того, что он прослужит больше двух лет, равна \( 0,87 \). Найдите вероятность того, что он прослужит меньше двух лет, но больше года.
		\item Из районного центра в деревню ежедневно ходит автобус. Вероятность того, что в понедельник в автобусе окажется меньше \( 18 \) пассажиров, равна \( 0,82 \). Вероятность того, что окажется меньше \( 10 \) пассажиров, равна \( 0,51 \). Найдите вероятность того, что число пассажиров будет от \( 10 \) до \( 17 \).
		\item Помещение освещается фонарём с двумя лампами. Вероятность перегорания лампы в течение года равна \( 0,3 \). Найдите вероятность того, что в течение года хотя бы одна лампа не перегорит.
		\item В магазине стоят два платёжных автомата. Каждый из них может быть неисправен с вероятностью \( 0,05 \) независимо от другого автомата. Найдите вероятность того, что хотя бы один автомат исправен.
		\item B июле \( 2022 \) года планируется взять кредит в банке на некоторую сум-
		му. Условия его возврата таковы:\\
		— каждый январь долг увеличивается на \( 20\% \) по сравнению с концом предыдущего года;\\
		— с февраля по июнь каждого года необходимо выплатить одним платежом часть долга.\\
		Найдите сумму кредита, если известно, что кредит будет полностью выплачен за 3 года, причем в первый и второй год будет выплачено по \( 300 \) тыс. руб., а в третий \( 417,6 \) тыс. руб.
		\item B июле \( 2026 \) года планируется взять кредит на пять лет в размере \( 3,3 \) млн руб. Условия его возврата таковы:\\
		— каждый январь долг будет возрастать на \( 20\% \) по сравнению с концом предыдущего года;\\
		— с февраля по июнь каждого года необходимо выплатить часть долга;\\
		— в июле \( 2027 \), \( 2028 \) и \( 2029 \) годах долг остается равен \( 3,3 \) млн руб.;\\
		— платежи в \( 2030 \) и \( 2031 \) годах должны быть равны;\\
		— к июлю \( 2031 \) года долг должен быть выплачен полностью.\\
		Найдите разницу между первым и последним платежами.
		\item a) Решите уравнение \( \sin2x=2\sin x + \sin \left( x+\dfrac{ 3\pi }{ 2 } \right)+1\) \\
		б) Найдите все корни этого уравнения, принадлежащие отрезку \( \left[ - 4\pi; -\dfrac{5\pi}{2} \right]  \)
		\item a) Решите уравнение \( \cos2x+0,5=\cos^2x \). \\
		б) Найдите все корни этого уравнения, принадлежащие отрезку \( \left[ -2\pi;-\dfrac{\pi}{2} \right]  \)
	\end{listofex}
\end{consultation}
%END_FOLD