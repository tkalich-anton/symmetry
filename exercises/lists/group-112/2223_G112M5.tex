%
%===============>>  ГРУППА 9-2 МОДУЛЬ 5  <<=============
%
\setmodule{5}

%BEGIN_FOLD % ====>>_____ Занятие 1 _____<<====
\begin{class}[number=1]
	\begin{listofex}
		\item Занятие 1
	\end{listofex}
\end{class}
%END_FOLD

%BEGIN_FOLD % ====>>_____ Занятие 2 _____<<====
\begin{class}[number=2]
	\begin{listofex}
		\item Занятие 2
	\end{listofex}
\end{class}
%END_FOLD

%BEGIN_FOLD % ====>>_ Домашняя работа 1 _<<====
\begin{homework}[number=1]
	\begin{listofex}
		\item ДЗ 1
	\end{listofex}
\end{homework}
%END_FOLD

%BEGIN_FOLD % ====>>_____ Занятие 3 _____<<====
\begin{class}[number=3]
	\begin{listofex}
		\item В треугольнике \( ABC \) угол \( C \) равен \( 90\degree \), \( AC = 4,8 \),  \( \sin A = \dfrac{7}{25} \).  Найдите \( AB \).
		\item В треугольнике \( ABC \) угол \( C \) равен \( 90\degree \),  \( \tg A = \dfrac{33}{4\sqrt{33}} \),  \( AC =  4 \). Найдите \( AB \).
		\item В треугольнике \( АВС \) угол \( С \) равен \( 90\degree \), \( AB = 13 \),  \( \tg A = \dfrac{1}{5} \).  Найдите высоту \( CH \).
		\item В треугольнике \( АВС \) угол \( С \) равен \( 90\degree \), высота \( CH \) равна \( 7 \), \( BH = 24 \). Найдите  \( \cos A \).
		\item В треугольнике \( ABC \) \( AC=BC=5\),  \( \sin A = \dfrac{7}{25} \).  Найдите \(AB\).
		\item В треугольнике \( ABC \) \( AC = BC = 8 \),  \( \cos A = 0,5 \). Найдите \(AB\).
		\item В треугольнике \( ABC \) \( AC = BC \), \( AH \) --- высота, \( AB = 8 \),  \( \cos BAC = 0,5 \). Найдите \( BH \).
		\item Треугольник \( ABC \) вписан в окружность с центром \( O \). Найдите угол \( BOC \), если угол \( BAC \) равен \( 32\degree \).
		\item Найдите центральный угол \( AOB \), если он на \( 15\degree \) больше вписанного угла \( ACB \), опирающегося на ту же дугу. Ответ дайте в градусах.
		\item Чему равен острый вписанный угол, опирающийся на хорду, равную радиусу окружности? Ответ дайте в градусах.
		\item Найдите вписанный угол, опирающийся на дугу, которая составляет \( \dfrac{1}{5} \) окружности. Ответ дайте в градусах.
		\item Четырёхугольник \( ABCD \) вписан в окружность. Угол \( ABD \) равен \( 61\degree \), угол \( CAD \) равен \( 37\degree \) Найдите угол \( ABC \). Ответ дайте в градусах.
		\item Четырёхугольник \( ABCD \) вписан в окружность. Угол \( ABC \) равен \( 102\degree \), угол \( CAD  \) равен \( 46\degree \). Найдите угол \( ABD \). Ответ дайте в градусах.
		\item В треугольнике \( ABC \) угол \( A \) равен \( 43\degree \), углы \( B \) и \( C \) --- острые, высоты \( BD \) и \( CE \) пересекаются в точке \( O \). Найдите угол \( DOE \). Ответ дайте в градусах.
	\end{listofex}
\end{class}
%END_FOLD

%BEGIN_FOLD % ====>>_____ Занятие 4 _____<<====
\begin{class}[number=4]
	\begin{listofex}
		\item В треугольнике \( ABC \) угол \( C \) равен \( 90\degree \), \( AC  =  24 \), \( BC  =  7 \). Найдите  \( \sin A \).
		\item В треугольнике \( АВС \) угол \( С \) равен \( 90\degree \), \( СН \) --- высота, \( AB = 13 \),  \( \tg A = 5 \). Найдите \( BH \).
		\item Угол \( ABD \) равен \( 53\degree \). Угол \( ВСА  \) равен \( 38\degree \). Найдите вписанный угол \( BCD \). Ответ дайте в градусах.
		\item Угол между двумя соседними сторонами правильного многоугольника, равен \( 160\degree \). Найдите число вершин многоугольника.
		\item В треугольнике \( ABC \) сторона \( AB \) равна \( 3 \sqrt{2} \), угол \( С \) равен \( 135\degree \). Найдите радиус описанной около этого треугольника окружности.
		\item В треугольнике \( ABC \) угол \( A \) равен \( 40\degree \), внешний угол при вершине \( B \) равен \( 102\degree \). Найдите угол \( C \). Ответ дайте в градусах.
		\item Углы треугольника относятся как \( 2 : 3 : 4 \). Найдите меньший из них. Ответ дайте в градусах.
		\item В треугольнике \( ABC \) угол \( A \) равен \( 30\degree \), угол \( В \) --- тупой, \( CH \) --- высота, угол \( BCH \) равен \( 22\degree \). Найдите угол \( ACB \). Ответ дайте в градусах.
		\item В треугольнике \( ABC \) \( AD \) --- биссектриса, угол \( C \) равен \( 30\degree \), угол \( BAD \) равен \( 22\degree \). Найдите угол \( ADB \). Ответ дайте в градусах.
		\item В остроугольном треугольнике \( ABC \) угол \( A \) равен \( 65\degree \). \( BD \) и \( CE \) --- высоты, пересекающиеся в точке \( O \). Найдите угол \( DOE \). Ответ дайте в градусах.
		\item В треугольнике \( ABC \) угол \( C \) равен \( 58\degree \), \( AD \) и \( BE \) --- биссектрисы, пересекающиеся в точке \( O \). Найдите угол \( AOB \). Ответ дайте в градусах.
		\item В треугольнике \( ABC \) угол \( A \) равен \( 44\degree \), угол \( C \) равен \( 62\degree \). На продолжении стороны \( AB \) за точку \( B \) отложен отрезок \( BD \), равный стороне \( BC \). Найдите угол \( D \) треугольника \( BCD \). Ответ дайте в градусах.
		\item В треугольнике \( ABC \) угол \( A \) равен \( 60\degree \), угол \( B \) равен \( 82\degree \). \( AD \), \( BE \) и \( CF \) --- биссектрисы, пересекающиеся в точке \( O \). Найдите угол \( AOF \). Ответ дайте в градусах.
	\end{listofex}
\end{class}
%END_FOLD

%BEGIN_FOLD % ====>>_ Домашняя работа 2 _<<====
\begin{homework}[number=2]
	\begin{listofex}
		\item В треугольнике \( АВС \) угол \( С \) равен \( 90\degree \), \( BC = 5 \),  \( \sin A = \dfrac{7}{25} \).  Найдите высоту \( CH \).
		\item В треугольнике \( ABC \) \( AC = BC = 4\sqrt{15 }\),  \( \sin BAC = 0,25 \). Найдите высоту \( AH \).
		\item Чему равен тупой вписанный угол, опирающийся на хорду, равную радиусу окружности? Ответ дайте в градусах.
		\item Найдите вписанный угол, опирающийся на дугу, которая составляет \( \dfrac{4}{9} \) окружности. Ответ дайте в градусах.
		\item В треугольнике \( ABC \) сторона \( AB \) равна \( 2 \sqrt{3} \), угол \( С \) равен \( 120\degree \). Найдите радиус описанной около этого треугольника окружности.
		\item Углы треугольника относятся как \( 1:2:15 \). Найдите меньший из них. Ответ дайте в градусах.
		\item В треугольнике \( ABC \) \( AD \) --- биссектриса, угол \( C \) равен \( 70\degree \), угол \( BAD \) равен \( 51\degree \). Найдите угол \( ADB \). Ответ дайте в градусах.
		\item В треугольнике \( ABC \) угол \( A \) равен \( 60\degree \), угол \( B \) равен \( 82\degree \). \( AD \), \( BE \) и \( CF \) --- высоты, пересекающиеся в точке \( O \). Найдите угол \( AOF \). Ответ дайте в градусах.
	\end{listofex}
\end{homework}
%END_FOLD

%BEGIN_FOLD % ====>>_____ Занятие 5 _____<<====
\begin{class}[number=5]
	\begin{listofex}
		\item Занятие 5
	\end{listofex}
\end{class}
%END_FOLD

%BEGIN_FOLD % ====>>_____ Занятие 6 _____<<====
\begin{class}[number=6]
	\begin{listofex}
		\item Занятие 6
	\end{listofex}
\end{class}
%END_FOLD

%BEGIN_FOLD % ====>>_ Домашняя работа 3 _<<====
\begin{homework}[number=3]
	\begin{listofex}
		\item ДЗ 3
	\end{listofex}
\end{homework}
%END_FOLD

%BEGIN_FOLD % ====>>_____ Занятие 7 _____<<====
\begin{class}[number=7]
	\begin{listofex}
		\item Занятие 7
	\end{listofex}
\end{class}
%END_FOLD

%BEGIN_FOLD % ====>>_ Проверочная работа _<<====
\begin{exam}
	\begin{listofex}
		\item Проверочная работа
	\end{listofex}
\end{exam}
%END_FOLD
