%
%===============>>  ГРУППА 9-2 МОДУЛЬ 5  <<=============
%
\setmodule{5}

%BEGIN_FOLD % ====>>_____ Занятие 1 _____<<====
\begin{class}[number=1]
	\begin{listofex}
		\item Занятие 1
	\end{listofex}
\end{class}
%END_FOLD

%BEGIN_FOLD % ====>>_____ Занятие 2 _____<<====
\begin{class}[number=2]
	\begin{listofex}
		\item Занятие 2
	\end{listofex}
\end{class}
%END_FOLD

%BEGIN_FOLD % ====>>_ Домашняя работа 1 _<<====
\begin{homework}[number=1]
	\begin{listofex}
		\item ДЗ 1
	\end{listofex}
\end{homework}
%END_FOLD

%BEGIN_FOLD % ====>>_____ Занятие 3 _____<<====
\begin{class}[number=3]
	\begin{listofex}
		\item В треугольнике \( ABC \) угол \( C \) равен \( 90\degree \), \( AC = 4,8 \),  \( \sin A = \dfrac{7}{25} \).  Найдите \( AB \).
		\item В треугольнике \( ABC \) угол \( C \) равен \( 90\degree \),  \( \tg A = \dfrac{33}{4\sqrt{33}} \),  \( AC =  4 \). Найдите \( AB \).
		\item В треугольнике \( АВС \) угол \( С \) равен \( 90\degree \), \( AB = 13 \),  \( \tg A = \dfrac{1}{5} \).  Найдите высоту \( CH \).
		\item В треугольнике \( АВС \) угол \( С \) равен \( 90\degree \), высота \( CH \) равна \( 7 \), \( BH = 24 \). Найдите  \( \cos A \).
		\item В треугольнике \( ABC \) \( AC=BC=5\),  \( \sin A = \dfrac{7}{25} \).  Найдите \(AB\).
		\item В треугольнике \( ABC \) \( AC = BC = 8 \),  \( \cos A = 0,5 \). Найдите \(AB\).
		\item В треугольнике \( ABC \) \( AC = BC \), \( AH \) --- высота, \( AB = 8 \),  \( \cos BAC = 0,5 \). Найдите \( BH \).
		\item Треугольник \( ABC \) вписан в окружность с центром \( O \). Найдите угол \( BOC \), если угол \( BAC \) равен \( 32\degree \).
		\item Найдите центральный угол \( AOB \), если он на \( 15\degree \) больше вписанного угла \( ACB \), опирающегося на ту же дугу. Ответ дайте в градусах.
		\item Чему равен острый вписанный угол, опирающийся на хорду, равную радиусу окружности? Ответ дайте в градусах.
		\item Найдите вписанный угол, опирающийся на дугу, которая составляет \( \dfrac{1}{5} \) окружности. Ответ дайте в градусах.
		\item Четырёхугольник \( ABCD \) вписан в окружность. Угол \( ABD \) равен \( 61\degree \), угол \( CAD \) равен \( 37\degree \) Найдите угол \( ABC \). Ответ дайте в градусах.
		\item Четырёхугольник \( ABCD \) вписан в окружность. Угол \( ABC \) равен \( 102\degree \), угол \( CAD  \) равен \( 46\degree \). Найдите угол \( ABD \). Ответ дайте в градусах.
		\item В треугольнике \( ABC \) угол \( A \) равен \( 43\degree \), углы \( B \) и \( C \) --- острые, высоты \( BD \) и \( CE \) пересекаются в точке \( O \). Найдите угол \( DOE \). Ответ дайте в градусах.
	\end{listofex}
\end{class}
%END_FOLD

%BEGIN_FOLD % ====>>_____ Занятие 4 _____<<====
\begin{class}[number=4]
	\begin{listofex}
		\item В треугольнике \( ABC \) угол \( C \) равен \( 90\degree \), \( AC  =  24 \), \( BC  =  7 \). Найдите  \( \sin A \).
		\item В треугольнике \( АВС \) угол \( С \) равен \( 90\degree \), \( СН \) --- высота, \( AB = 13 \),  \( \tg A = 5 \). Найдите \( BH \).
		\item Угол \( ABD \) равен \( 53\degree \). Угол \( ВСА  \) равен \( 38\degree \). Найдите вписанный угол \( BCD \). Ответ дайте в градусах.
		\item Угол между двумя соседними сторонами правильного многоугольника, равен \( 160\degree \). Найдите число вершин многоугольника.
		\item В треугольнике \( ABC \) сторона \( AB \) равна \( 3 \sqrt{2} \), угол \( С \) равен \( 135\degree \). Найдите радиус описанной около этого треугольника окружности.
		\item В треугольнике \( ABC \) угол \( A \) равен \( 40\degree \), внешний угол при вершине \( B \) равен \( 102\degree \). Найдите угол \( C \). Ответ дайте в градусах.
		\item Углы треугольника относятся как \( 2 : 3 : 4 \). Найдите меньший из них. Ответ дайте в градусах.
		\item В треугольнике \( ABC \) угол \( A \) равен \( 30\degree \), угол \( В \) --- тупой, \( CH \) --- высота, угол \( BCH \) равен \( 22\degree \). Найдите угол \( ACB \). Ответ дайте в градусах.
		\item В треугольнике \( ABC \) \( AD \) --- биссектриса, угол \( C \) равен \( 30\degree \), угол \( BAD \) равен \( 22\degree \). Найдите угол \( ADB \). Ответ дайте в градусах.
		\item В остроугольном треугольнике \( ABC \) угол \( A \) равен \( 65\degree \). \( BD \) и \( CE \) --- высоты, пересекающиеся в точке \( O \). Найдите угол \( DOE \). Ответ дайте в градусах.
		\item В треугольнике \( ABC \) угол \( C \) равен \( 58\degree \), \( AD \) и \( BE \) --- биссектрисы, пересекающиеся в точке \( O \). Найдите угол \( AOB \). Ответ дайте в градусах.
		\item В треугольнике \( ABC \) угол \( A \) равен \( 44\degree \), угол \( C \) равен \( 62\degree \). На продолжении стороны \( AB \) за точку \( B \) отложен отрезок \( BD \), равный стороне \( BC \). Найдите угол \( D \) треугольника \( BCD \). Ответ дайте в градусах.
		\item В треугольнике \( ABC \) угол \( A \) равен \( 60\degree \), угол \( B \) равен \( 82\degree \). \( AD \), \( BE \) и \( CF \) --- биссектрисы, пересекающиеся в точке \( O \). Найдите угол \( AOF \). Ответ дайте в градусах.
	\end{listofex}
\end{class}
%END_FOLD

%BEGIN_FOLD % ====>>_ Домашняя работа 2 _<<====
\begin{homework}[number=2]
	\begin{listofex}
		\item В треугольнике \( АВС \) угол \( С \) равен \( 90\degree \), \( BC = 5 \),  \( \sin A = \dfrac{7}{25} \).  Найдите высоту \( CH \).
		\item В треугольнике \( ABC \) \( AC = BC = 4\sqrt{15 }\),  \( \sin BAC = 0,25 \). Найдите высоту \( AH \).
		\item Чему равен тупой вписанный угол, опирающийся на хорду, равную радиусу окружности? Ответ дайте в градусах.
		\item Найдите вписанный угол, опирающийся на дугу, которая составляет \( \dfrac{4}{9} \) окружности. Ответ дайте в градусах.
		\item В треугольнике \( ABC \) сторона \( AB \) равна \( 2 \sqrt{3} \), угол \( С \) равен \( 120\degree \). Найдите радиус описанной около этого треугольника окружности.
		\item Углы треугольника относятся как \( 1:2:15 \). Найдите меньший из них. Ответ дайте в градусах.
		\item В треугольнике \( ABC \) \( AD \) --- биссектриса, угол \( C \) равен \( 70\degree \), угол \( BAD \) равен \( 51\degree \). Найдите угол \( ADB \). Ответ дайте в градусах.
		\item В треугольнике \( ABC \) угол \( A \) равен \( 60\degree \), угол \( B \) равен \( 82\degree \). \( AD \), \( BE \) и \( CF \) --- высоты, пересекающиеся в точке \( O \). Найдите угол \( AOF \). Ответ дайте в градусах.
	\end{listofex}
\end{homework}
%END_FOLD

%BEGIN_FOLD % ====>>_____ Занятие 5 _____<<====
\begin{class}[number=5]
	\begin{listofex}
		\item Четырёхугольник \( ABCD \) вписан в окружность. Угол \( ABD \) равен \( 61\degree \), угол \( CAD \) равен \( 37\degree \) Найдите угол \( ABC \). Ответ дайте в градусах.
		\item Четырёхугольник \( ABCD \) вписан в окружность. Угол \( ABC \) равен \( 102\degree \), угол \( CAD  \) равен \( 46\degree \). Найдите угол \( ABD \). Ответ дайте в градусах.
		\item В треугольнике \( ABC \) угол \( A \) равен \( 43\degree \), углы \( B \) и \( C \) --- острые, высоты \( BD \) и \( CE \) пересекаются в точке \( O \). Найдите угол \( DOE \). Ответ дайте в градусах.
		\item Найдите хорду, на которую опирается угол \( 30\degree \), вписанный в окружность радиуса \( 3 \).
		\item Найдите хорду, на которую опирается угол \( 120\degree \), вписанный в окружность радиуса \( \sqrt{3} \).
		\item Хорда \( AB \) делит окружность на две части, градусные величины которых относятся как \( 5:7 \). Под каким углом видна эта хорда из точки \( C \), принадлежащей меньшей дуге окружности? Ответ дайте в градусах.
		\item Найдите угол \( ACO \), если его сторона \( CA \) касается окружности, \( O \) --- центр окружности, сторона \( CO \) пересекает окружность в точке \( B \), дуга \( AB \) окружности, заключённая внутри этого угла равна \( 64\degree \). Ответ дайте в градусах.
		\item Угол \( ACO \) равен \( 28\degree \), где \( O \) --- центр окружности. Его сторона \( CA \) касается окружности. Найдите величину меньшей дуги \( AB \) окружности, заключенной внутри этого угла. Ответ дайте в градусах.
		\item Найдите угол \( ACO \), если его сторона \( CA \) касается окружности, \( O \) --- центр окружности, сторона \( CO \) пересекает окружность в точках \( B \) и \( D \), а дуга \( AD \) окружности, заключенная внутри этого угла, равна \( 116\degree \). Ответ дайте в градусах.
		\item Точки \( A \), \( B \), \( C \), расположенные на окружности, делят ее на три дуги, градусные величины которых относятся как \( 1:3:5 \). Найдите больший угол треугольника \( ABC \). Ответ дайте в градусах.
		\item Угол \( A \) четырехугольника \( ABCD \), вписанного в окружность, равен \( 58\degree \). Найдите угол \( C \) этого четырехугольника. Ответ дайте в градусах.
		\item Стороны четырехугольника \( ABCD \) \( AB \), \( BC \), \( CD \) и \( AD \) стягивают дуги описанной окружности, градусные величины которых равны соответственно \( 95\degree \), \( 49\degree \), \( 71\degree \), \( 145\degree \). Найдите угол \( B \) этого четырехугольника. Ответ дайте в градусах.
		\item Точки \( A \), \( B \), \( C \), расположенные на окружности, делят ее на четыре дуги \( AB \), \( BC \), \( CD \) и \( AD \), градусные величины которых относятся соответственно как \( 4:2:3:6 \). Найдите угол \( A \) четырехугольника \( ABCD \). Ответ дайте в градусах.
	\end{listofex}
\end{class}
%END_FOLD

%BEGIN_FOLD % ====>>_____ Занятие 6 _____<<====
\begin{class}[number=6]
	\begin{listofex}
		\item Хорда \( AB \) стягивает дугу окружности в \( 92\degree \). Найдите угол \( ABC \) между этой хордой и касательной к окружности, проведенной через точку \( B \). Ответ дайте в градусах.
		\item Через концы \( A \) и \( B \) дуги окружности с центром \( O \) проведены касательные \( AC \) и \( BC \). Угол \( CAB \) равен \( 32\degree \). Найдите угол \( AOB \). Ответ дайте в градусах.
		\item Через концы \( A \), \( B \) дуги окружности в \( 62\degree \) проведены касательные \( AC \) и \( BC \). Найдите угол \( ACB \). Ответ дайте в градусах.
		\item Сторона правильного треугольника равна \( \sqrt{3} \). Найдите радиус окружности, описанной около этого треугольника.
		\item Высота правильного треугольника равна \( 3 \). Найдите радиус окружности, описанной около этого треугольника.
		\item Гипотенуза прямоугольного треугольника равна \( 12 \). Найдите радиус описанной окружности этого треугольника.
		\item Боковая сторона равнобедренного треугольника равна \( 1 \), угол при вершине, противолежащей основанию, равен \( 120\degree \). Найдите диаметр описанной окружности этого треугольника.
		\item Чему равна сторона правильного шестиугольника, вписанного в окружность, радиус которой равен \( 6 \)?
		\item Периметр правильного шестиугольника равен \( 72 \). Найдите диаметр описанной окружности.
		\item Угол между двумя соседними сторонами правильного многоугольника, вписанного в окружность, равен \( 108\degree \). Найдите число вершин многоугольника.
		\item Сторона \( AB \) треугольника \( ABC \) равна \( 1 \). Противолежащий ей угол \( C \) равен \( 30\degree \). Найдите радиус окружности, описанной около этого треугольника.
		\item Боковые стороны равнобедренного треугольника равны \( 40 \), основание равно \( 48 \). Найдите радиус описанной окружности этого треугольника.
		\item Одна сторона треугольника равна \( \sqrt{2} \), радиус описанной окружности равен \( 1 \). Найдите острый угол треугольника, противолежащий этой стороне. Ответ дайте в градусах.
	\end{listofex}
\end{class}
%END_FOLD

%BEGIN_FOLD % ====>>_ Домашняя работа 3 _<<====
\begin{homework}[number=3]
	\begin{listofex}
		\item Хорда \( AB \) делит окружность на две части, градусные величины которых относятся как \( 1:2 \). Под каким углом видна эта хорда из точки \( C \), принадлежащей меньшей дуге окружности? Ответ дайте в градусах.
		\item Касательные \( CA \) и \( CB \) к окружности образуют угол \( ACB \), равный \( 122\degree \). Найдите величину меньшей дуги \( AB \), стягиваемой точками касания. Ответ дайте в градусах.
		\item Радиус окружности, описанной около правильного треугольника, равен \( \sqrt{3} \). Найдите сторону этого треугольника.
		\item Радиус окружности, описанной около прямоугольного треугольника, равен \( 4 \). Найдите гипотенузу этого треугольника.
		\item Одна сторона треугольника равна радиусу описанной окружности. Найдите острый угол треугольника, противолежащий этой стороне. Ответ дайте в градусах
		\item Два угла вписанного в окружность четырехугольника равны \( 82\degree\) и \( 58\degree \). Найдите больший из оставшихся углов. Ответ дайте в градусах.
	\end{listofex}
\end{homework}
%END_FOLD

%BEGIN_FOLD % ====>>_____ Занятие 7 _____<<====
\begin{class}[number=7]
	\begin{listofex}
		\item Основания равнобедренной трапеции равны \(51\) и \(65\). Боковые стороны равны \(25\). Найдите синус острого угла трапеции.
		\item Основания равнобедренной трапеции равны \(43\) и \(73\). Косинус острого угла трапеции равен \(\dfrac{5}{7}\).  Найдите боковую сторону.
		\item Меньшее основание равнобедренной трапеции равно \(23\). Высота трапеции равна \(39\). Тангенс острого угла равен \(\dfrac{13}{8}\).  Найдите большее основание.
		\item Большее основание равнобедренной трапеции равно \(34\). Боковая сторона равна \(14\). Синус острого угла равен \( \dfrac{2\sqrt{10}}{7} \).  Найдите меньшее основание.
		\item Основания равнобедренной трапеции равны \(14\) и \(26\), а ее периметр равен \(60\). Найдите площадь трапеции.
		\item Основания трапеции равны \(18\) и \(6\), боковая сторона, равная \(7\), образует с одним из оснований трапеции угол \(150 \degree \). Найдите площадь трапеции.
		\item Основания трапеции равны \(27\) и \(9\), боковая сторона равна \(8\). Площадь трапеции равна \(72\). Найдите острый угол трапеции, прилежащий к данной боковой стороне
		\item Чему равен больший угол равнобедренной трапеции, если известно, что разность противолежащих углов равна \(50 \degree \)? 
		\item Средняя линия трапеции равна \(28\), а меньшее основание равно \(18\). Найдите большее основание трапеции.
		
		\item Найдите площадь квадрата, если его диагональ равна \(1\).
		\item В параллелограмме \(ABCD, AD = 3, BC=21,  \sin A= \dfrac{6}{7}\).  Найдите большую высоту параллелограмма.
		\item Стороны параллелограмма равны \(9\) и \(15\). Высота, опущенная на первую сторону, равна \(10\). Найдите высоту, опущенную на вторую сторону параллелограмма.
		\item Площадь параллелограмма равна \(40\), две его стороны равны \(5\) и \(10\). Найдите большую высоту этого параллелограмма.
		\item Найдите площадь ромба, если его высота равна \(2\), а острый угол \(30 \degree \).
		\item Сумма двух углов параллелограмма равна \(100\degree \). Найдите один из оставшихся углов.
		
		\item Периметр треугольника равен \(12\), а радиус вписанной окружности равен \(1\). Найдите площадь этого треугольника.
		\item Найдите радиус окружности, вписанной в правильный треугольник, высота которого равна \(6\).
		\item Сторона правильного треугольника равна  корень из 3. Найдите радиус окружности, вписанной в этот треугольник.
		\item Радиус окружности, вписанной в правильный треугольник, равен \(\dfrac{\sqrt{3}}{6}\).  Найдите сторону этого треугольника.
	\end{listofex}
\end{class}
%END_FOLD

%BEGIN_FOLD % ====>>_ Занятие 8 _<<====
\begin{class}[number=8]
	\begin{listofex}
		\item Найдите площадь треугольника, две стороны которого равны \(8\) и \(12\), а угол между ними равен \(30 \degree \).
		\item В треугольнике \(ABC, AD\)  --- биссектриса, угол \(C\) равен \(30 \degree\), угол \(BAD\) равен \(22 \degree\). Найдите угол \(ADB\). Ответ дайте в градусах.
		\item В треугольнике \(ABC\) угол \(C\) равен \(90 \degree \), \(CH\)  --- высота, \(AH = 12\),  \(\cos A = \dfrac{2}{3} \).  Найдите \(AB\).
		\item В треугольнике \(ABC\) угол \(C\) равен \(90 \degree \), высота \(CH\) равна \(20, BC  =  25\). Найдите  синус \(A\).
		\item Основания равнобедренной трапеции равны \(7\) и \(13\), а ее площадь равна \(40\). Найдите периметр трапеции и её боковую сторону.
		\item Найдите площадь прямоугольной трапеции, основания которой равны \(6\) и \(2\), большая боковая сторона составляет с основанием угол \(45 \degree\).
		\item Основания трапеции равны \(18\) и \(6\), боковая сторона, равная \(7\), образует с одним из оснований трапеции угол \(150 \degree \). Найдите площадь трапеции.
		\item Найдите площадь прямоугольной трапеции, основания которой равны \(6\) и \(2\), большая боковая сторона составляет с основанием угол \(45 \degree \).
		\item Периметр прямоугольника равен \(34\), а площадь равна \(60\). Найдите диагональ этого прямоугольника.
		\item Найдите периметр прямоугольника, если его площадь равна \(18\), а отношение соседних сторон равно \(1:2\).
		\item Периметр прямоугольника равен \(28\), а диагональ равна \(10\). Найдите площадь этого прямоугольника.
		\item Один угол параллелограмма больше другого на \(70 \degree \). Найдите больший угол.
		\item Диагональ параллелограмма образует с двумя его сторонами углы \(26\degree\) и \(34\degree \). Найдите больший угол параллелограмма.
		\item Периметр параллелограмма равен \(46\). Одна сторона параллелограмма на \(3\) больше другой. Найдите меньшую сторону параллелограмма.
		\item Радиус окружности, вписанной в правильный треугольник, равен \(6\). Найдите высоту этого треугольника.
		\item Сторона ромба равна \(1\), острый угол равен \(30\) градусов. Найдите радиус вписанной окружности этого ромба.
		\item Острый угол ромба равен \(30\degree \). Радиус вписанной в этот ромб окружности равен \(2\). Найдите сторону ромба.
		\item Найдите сторону правильного шестиугольника, описанного около окружности, радиус которой равен \(\sqrt{3}\).
		\item Точки \(A, B, C,\) расположенные на окружности, делят ее на три дуги, градусные величины которых относятся как \(1 : 3 : 5\). Найдите больший угол треугольника \(ABC\). Ответ дайте в градусах.
		\item Угол \(A\) четырехугольника \(ABCD\), вписанного в окружность, равен \(58 \degree \). Найдите угол \(C\) этого четырехугольника.
		\item Стороны четырехугольника \(ABCD\) --- \(AB, BC, CD и AD\) стягивают дуги описанной окружности, градусные величины которых равны соответственно \(95\degree \), \(49\degree\), \(71\degree\), \(145\degree\). Найдите разность наибольшего и наименьшего углов четырехугольника.
	\end{listofex}
\end{class}
%END_FOLD
