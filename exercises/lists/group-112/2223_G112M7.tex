%
%===============>>  ГРУППА 11-2 МОДУЛЬ 7  <<=============
%
\setmodule{7}

%BEGIN_FOLD % ====>>_____ Занятие 1 _____<<====
\begin{class}[number=1]
	\begin{listofex}
	\item %484557
	\begin{tasks}(1)
		\task Решите уравнение \( (2\sin x+\sqrt{3}) \cdot \sqrt{\cos x}=0 \)
		\task Найдите все корни этого уравнения, принадлежащие отрезку: \( \left[ \dfrac{3\pi}{2}; \dfrac{7\pi}{2} \right] \)
	\end{tasks}
	\item %501689
	\begin{tasks}(1)
		\task Решите уравнение \( 15^{\cos x}=3^{\cos x} \cdot 5^{\sin x} \)
		\task Найдите все корни этого уравнения, принадлежащие отрезку: \( \left[ 5\pi; \dfrac{13\pi}{2} \right]  \)
	\end{tasks}
	\item %505102
	\begin{tasks}(1)
		\task Решите уравнение \( 9^{\sin{x}}+9^{-\sin x}=\dfrac{10}{3} \)
		\task Найдите все корни этого уравнения, принадлежащие отрезку: \( \left[ -\dfrac{7\pi}{2}; -2\pi \right]  \)
	\end{tasks}
	\item %505236
	\begin{tasks}(1)
		\task Решите уравнение \( \left( \dfrac{2}{5} \right)^{\cos x}+ \left( \dfrac{5}{2} \right)^{\cos x}=2 \)
		\task Найдите все корни этого уравнения, принадлежащие отрезку: \( \left[ -3\pi;-\dfrac{3\pi}{2} \right]  \)
	\end{tasks}
	\item %505565
	\begin{tasks}(1)
		\task Решите уравнение \( \dfrac{3^{\cos x}}{9^{\cos^2x}}=4^{2\cos^2x-\cos x} \)
		\task Найдите все корни этого уравнения, принадлежащие отрезку: \( \left[ -\dfrac{3\pi}{2}; \dfrac{\pi}{6} \right]  \)
	\end{tasks}
	\item %501729
	\begin{tasks}(1)
		\task Решите уравнение \( (27^{\cos x})^{\sin x}=3^{\frac{3\cos x}{2}} \)
		\task Найдите все корни этого уравнения, принадлежащие отрезку: \( \left[ -\pi; \dfrac{\pi}{2} \right]  \)
	\end{tasks}
	\item %500192
	\begin{tasks}(1)
		\task Решите уравнение \( \left( \dfrac{1}{81} \right)^{\cos x}=9^{2\sin{2x}} \)
		\task Найдите все корни этого уравнения, принадлежащие отрезку: \( [-3\pi; -2\pi] \)
	\end{tasks}
	\end{listofex}
\end{class}
%END_FOLD

%BEGIN_FOLD % ====>>_____ Занятие 2 _____<<====
\begin{class}[number=2]
	\begin{listofex}
		\item Занятие 2
	\end{listofex}
\end{class}
%END_FOLD

%BEGIN_FOLD % ====>>_ Домашняя работа 1 _<<====
\begin{homework}[number=1]
	\begin{listofex}
		\item Домашняя работа 1
	\end{listofex}
\end{homework}
%END_FOLD

%BEGIN_FOLD % ====>>_____ Занятие 3 _____<<====
\begin{class}[number=3]
	\begin{listofex}
		\item Занятие 3 
	\end{listofex}
\end{class}
%END_FOLD

%BEGIN_FOLD % ====>>_____ Занятие 4 _____<<====
\begin{class}[number=4]
	\begin{listofex}
		\item Занятие 4
	\end{listofex}
\end{class}
%END_FOLD

%BEGIN_FOLD % ====>>_ Домашняя работа 2 _<<====
\begin{homework}[number=2]
	\begin{listofex}
		\item Домашняя работа 2
	\end{listofex}
\end{homework}
%END_FOLD

%BEGIN_FOLD % ====>>_____ Занятие 5 _____<<====
\begin{class}[number=5]
	\begin{listofex}
		\item Занятие 5
	\end{listofex}
\end{class}
%END_FOLD

%BEGIN_FOLD % ====>>_____ Занятие 6 _____<<====
\begin{class}[number=6]
	\begin{listofex}
		\item Занятие 6
	\end{listofex}
\end{class}
%END_FOLD

%BEGIN_FOLD % ====>>_ Домашняя работа 3 _<<====
\begin{homework}[number=3]
	\begin{listofex}
		\item Домашняя работа 3
	\end{listofex}
\end{homework}
%END_FOLD

%BEGIN_FOLD % ====>>_____ Занятие 7 _____<<====
\begin{class}[number=7]
	\title{Подготовка к проверочной}
	\begin{listofex}
		\item Занятие 7
	\end{listofex}
\end{class}
%END_FOLD

%BEGIN_FOLD % ====>>_ Проверочная работа _<<====
\begin{exam}
	\begin{listofex}
		\item Проверочная
	\end{listofex}
\end{exam}
%END_FOLD

%BEGIN_FOLD % ====>>_ Проверочная работа _<<====
\begin{class}[number=доп. занятие]
	\begin{listofex}
		\item Найдите все значения параметра \( a \), при каждом из которых уравнение
		\[ (a-2)x^2-2(a-1)x+3=0 \]
		имеет единственный корень.
		\item При каких значениях параметра \( a \) уравнение
		\[ (a-2)x^2-4ax+a-1=0 \]
		имеет два корня разных знаков?
		\item При каких значениях параметра \( a \) уравнение
		\[ (a^2-9)x^2-(2a^2+5a-9)x+a+3=0 \]
		имеет два корня разных знаков?
		\item Найдите все значения параметра \( a \), при каждом из которых уравнение
		\[ x^2-2(a^2-4a+1)x+4=0 \]
		имеет два различных отрицательных корня.
		\item При каких значениях параметра \( a \) уравнение
		\[ x^2-(a+1)|x|+a=0 \]
		имеет три решения?
		\item При каких значениях параметра \( a \) уравнение
		\[ x^4-(3a-1)x^2+2a^2-a=0 \]
		имеет два решения?
	\end{listofex}
	\newpage
	\title{Домашняя работа}
	\begin{listofex}
		\item Найдите все значения параметра \( a \), при каждом из которых уравнение
		\[ (a-4)x^2-3(a+1)x+1=0 \]
		\item При каких значениях параметра \( a \) уравнение
		\[ (a+2)x^2-5ax+a-3=0 \]
		имеет два корня разных знаков?
		\item Найдите все значения параметра \( a \), при каждом из которых уравнение
		\[ x^2+2(a^2-6a-3)x+16=0 \]
		имеет два различных отрицательных корня.
	\end{listofex}
\end{class}
%END_FOLD