%Группа 11-2 Модуль 2
\title{Занятие №1}
\begin{listofex}
	\item На изготовление \(667\) деталей первый рабочий тратит на \( 6 \) часов меньше, чем второй рабочий на изготовление \(754\) таких же деталей. Известно, что первый рабочий за час делает на \(3\) детали больше, чем второй. Сколько деталей в час делает первый рабочий?
	\item Двое рабочих, работая вместе, могут выполнить работу за \( 15 \) дней. За сколько дней, работая отдельно, выполнит эту работу первый  рабочий, если он за \(  2 \) дня выполняет такую же часть работы, какую второй – за \( 3 \) дня?
	\item Каждый из двух рабочих одинаковой квалификации может выполнить заказ за \( 15 \) часов. Через \( 3 \) часа после того, как один из них приступил к выполнению заказа, к нему присоединился второй рабочий, и работу над заказом они довели до конца уже вместе. Сколько часов потребовалось на выполнение всего заказа?
	\item Первый насос наполняет бак за \( 18 \) минут, второй – за \( 24 \) минуты, а третий – за \( 36 \) минут. За сколько минут наполнят бак три насоса, работая одновременно?
	\item  Плиточник планирует уложить \( 324 \) м\( ^2\) плитки. Если он будет укладывать на \( 6 \) м\( ^2 \) в день больше, чем запланировал, то закончит работу на \( 9 \) дней раньше. Сколько квадратных метров плитки в день планирует укладывать плиточник?
	\item Дима, Артем, Никита и Денис учредили компанию с уставным капиталом \( 100000 \) рублей. Дима внес \( 20\% \) уставного капитала, Артем –  \( 50000 \) рублей, Никита – \( 0,26 \) уставного капитала, а оставшуюся часть капитала внес Денис. Учредители договорились делить ежегодную прибыль пропорционально внесенному в уставной капитал вкладу. Какая сумма от прибыли \( 700000 \) рублей причитается Денису? Ответ дайте в рублях.
	\item Имеется два сплава. Первый содержит \( 15\% \) никеля, второй – \( 35\% \) никеля. Из этих двух сплавов получили третий сплав массой \( 140 \) кг, содержащий \( 30\%  \) никеля. На сколько килограммов масса первого сплава была меньше массы второго?
	\item Изюм получается в процессе сушки винограда. Сколько килограммов винограда потребуется для получения \( 56 \) килограммов изюма, если виноград содержит \( 90\% \) воды, а изюм содержит \( 5\% \) воды?
	\item По двум параллельным железнодорожным путям друг навстречу другу следуют скорый и пассажирский поезда, скорости которых равны соответственно \( 75  \) км/ч и \( 60  \) км/ч. Длина пассажирского поезда равна \( 400  \) метрам. Найдите длину скорого поезда, если время, за которое он прошел мимо пассажирского поезда, равно \( 16  \) секундам. Ответ дайте в метрах.
\end{listofex}
\newpage
\title{Занятие №2}
\begin{listofex}
	\item Вычислить:
\begin{enumcols}[itemcolumns=3]
	\item \exercise{562}
	\item \exercise{564}
	\item \exercise{569}
	\item \exercise{571}
	\item \exercise{579}
\end{enumcols}
\item Вычислить:
\begin{enumcols}[itemcolumns=3]
	\item \exercise{1577}
	\item \exercise{1578}
	\item \exercise{1579}
\end{enumcols}
\item Вычислить:
\begin{enumcols}[itemcolumns=3]
	\item \exercise{1572}
	\item \exercise{1565}
	\item \exercise{1566}
	\item \exercise{1573}
	\item \exercise{1567}
	\item \exercise{1575}
	\item \exercise{1594}
\end{enumcols}
\item Вычислить:
\begin{enumcols}[itemcolumns=2]
	\item \exercise{1569}
	\item \exercise{1570}
	\item \exercise{1571}
	\item \exercise{1574}
\end{enumcols}
\item Решить уравнение:
\begin{enumcols}[itemcolumns=2]
	\item \( \log_2(4-x)=7 \)
	\item \( \log_{1/7}(7-2x)=-2 \)
	\item \( \log_4(x+3)=\log_4(4x-15) \)
	\item \( \log_5(7-x)=\log_5(3-x)+1 \)
	\item \( \log_8 2^{8x-4}=4 \)
	\item \( \log_5(x^2+13x)=\log_5(9x+5) \)
\end{enumcols}
\end{listofex}
\newpage
\title{Домашняя работа №1}
\begin{listofex}
	\item Из пункта \( A \) в пункт \( B \), расстояние между которыми \(75\), км одновременно выехали автомобилист и велосипедист. Известно, что за час автомобилист проезжает на \(40\) км больше, чем велосипедист. Определите скорость велосипедиста, если известно, что он прибыл в пункт \(B\) на \(6\) часов позже автомобилиста. Ответ дайте в км/ч.
	\item Два мотоциклиста стартуют одновременно в одном направлении из двух диаметрально противоположных точек круговой трассы, длина которой равна \( 14  \) км. Через сколько минут мотоциклисты поравняются в первый раз, если скорость одного из них на \( 21  \) км/ч больше скорости другого?
	\item Первая труба пропускает на \(1\) литр воды в минуту меньше, чем вторая. Сколько литров воды в минуту пропускает первая труба, если резервуар объемом \(420\) литров она заполняет на \( 2 \) минуты дольше, чем вторая заполняет резервуар объемом \(399\) литров?
	\item Первый насос наполняет бак за \( 20 \) минут, второй – за \( 30 \) минут, а третий – за \( 1 \) час. За сколько минут наполнят бак три насоса, работая одновременно?
	\item Вычислить:
	\begin{enumcols}[itemcolumns=4]
		\item \exercise{588}
		\item \exercise{1590}
		\item \exercise{1293}
		\item \exercise{567}
		\item \exercise{580}
		\item \exercise{570}
		\item \exercise{573}
		\item \exercise{572}
	\end{enumcols}
	\item Вычислить:
	\begin{enumcols}[itemcolumns=2]
		\item \exercise{582}
		\item \exercise{590}
		\item \exercise{595}
		\item \exercise{1587}
	\end{enumcols}
	\item Вычислить:
	\begin{enumcols}[itemcolumns=2]
		\item \exercise{1568}
		\item \exercise{1573}
		\item \exercise{1588}
		\item \exercise{586}
	\end{enumcols}
	\item Вычислить:
	\begin{enumcols}[itemcolumns=3]
		\item \exercise{1583}
		\item \exercise{595}
		\item \exercise{1585}
	\end{enumcols}
	\item Решить уравнение:
	\begin{enumcols}[itemcolumns=2]
		\item \exercise{506}
		\item \exercise{3653}
		\item \exercise{1164}
		\item \exercise{3529}
		\item \exercise{606}
		\item \exercise{613}
		\item \exercise{601}
	\end{enumcols}
\end{listofex}
\newpage
\title{Занятие №3}
\begin{listofex}
	\item Решить уравнение:
	\begin{enumcols}[itemcolumns=2]
		\item \( \left( \dfrac{1}{8} \right)^{-3+x}=512 \) \answer{ \( 0 \) }
		\item \exercise{3422}
		\item \exercise{3430}
		\item \exercise{3358}
		\item \( 2^{3+x}=0,4\cdot5^{3+x} \)
		\item \(\log_5(x^2+2x)=\log_5(x^2+10) \)\answer{ \( 5 \) }
		\item \( \log_{x-5}49=2 \) \answer{ \( 12 \) }
	\end{enumcols}
	\item Решить уравнение:
	\begin{enumcols}[itemcolumns=4]
		\item \( \arcsin\dfrac{1}{2} \)
		\item \( \arccos\dfrac{\sqrt{2}}{2} \)
		\item \( \arccos0 \)
		\item \( \arctg\sqrt{3} \)
		\item \( \arccos\left( -\dfrac{\sqrt{3}}{2} \right) \)
		\item \( \arcsin\left( -1 \right) \)
		\item \( \arcsin\left( -\dfrac{2}{2} \right) \)
		\item \( \arctg\left( -\dfrac{\sqrt{3}}{3} \right) \)
		\item \( \arcctg(-1) \)
		\item \( \arcctg\left( -\sqrt{3} \right) \)
	\end{enumcols}
	\item Решить уравнение:
	\begin{enumcols}[itemcolumns=2]
		\item \( \sin x = \dfrac{1}{2} \)
		\item \( \cos x = \dfrac{\sqrt{2}}{2} \)
		\item \( \sin x = -\dfrac{\sqrt{3}}{2} \)
		\item \( \tg x = \dfrac{\sqrt{3}}{3} \)
		\item \( \ctg x = 1 \)
		\item \( \sin x = \dfrac{\sqrt{2}}{2} \)
	\end{enumcols}
	\item Решить уравнение:
	\begin{enumcols}[itemcolumns=3]
		\item \( \sin 2x = \dfrac{\sqrt{2}}{2} \)
		\item \( \sin 5x = -1 \)
		\item \( \cos 4x = 0 \)
		\item \( \sin \dfrac{2}{3}x = -\dfrac{1}{2} \)
		\item \( \tg 0,5x = 1 \)
		\item \( \ctg (-2x) = -1 \)
	\end{enumcols}
	\item Решить уравнение:
	\begin{enumcols}[itemcolumns=2]
		\item \( \sin \left( x+\dfrac{\pi}{2} \right) = \dfrac{\sqrt{3}}{2} \)
		\item \( \cos \left( 2x-\dfrac{3\pi}{2} \right) = 1 \)
		\item \( \sin \left( \dfrac{1}{2}\pi-x \right)=1 \)
		\item \( \tg\left( 3x-\dfrac{5}{4}\pi \right)=-1 \)
	\end{enumcols}
	\item Решить уравнение:
	\begin{enumcols}[itemcolumns=3]
		\item \( \sin x = \dfrac{4}{5} \)
		\item \( \cos x = \dfrac{5}{4}\)
		\item \( \cos 2x = \dfrac{1}{3}\)
	\end{enumcols}
	\item Решить уравнение \( \cos\dfrac{\pi(x-7)}{3}=\dfrac{1}{2} \). В ответ запишите наибольший отрицательный корень.
	\item Решить уравнение \( \tg\dfrac{\pi x}{4}=\dfrac{1}{2} \). В ответ запишите наибольший отрицательный корень.
\end{listofex}
\newpage
\title{Занятие №4}
\begin{listofex}
	\item Решить уравнение:
	\begin{enumcols}[itemcolumns=2]
		\item \exercise{3502}
		\item \exercise{3500}
		\item \exercise{3498}
		\item \exercise{3429}
		\item \exercise{607}
		\item \exercise{777}
	\end{enumcols}
	\item Решить уравнение:
	\begin{enumcols}[itemcolumns=4]
		\item \( \arcsin\dfrac{\sqrt{2}}{2} \)
		\item \( \arccos\dfrac{\sqrt{3}}{2} \)
		\item \( \arcsin0 \)
		\item \( \arcctg\sqrt{3} \)
		\item \( \arcsin\left( -\dfrac{\sqrt{2}}{2} \right) \)
		\item \( \arccos\left( -1 \right) \)
		\item \( \arctg(-1) \)
		\item \( \arcctg\left( -\dfrac{\sqrt{3}}{3} \right) \)
	\end{enumcols}
	\item Решить уравнение:
	\begin{enumcols}[itemcolumns=3]
		\item ☺\( \cos x = \dfrac{1}{2} \)
		\item ☺\( \sin x = \dfrac{\sqrt{2}}{2} \)
		\item ☺\( \sin x = -\dfrac{\sqrt{2}}{2} \)
		\item ☺\( \tg x = \dfrac{-\sqrt{3}}{3} \)
		\item ☺\( \ctg x = -1 \)
		\item ☺\( \cos x = \dfrac{\sqrt{2}}{2} \)
	\end{enumcols}
	\item Решить уравнение:
	\begin{enumcols}[itemcolumns=3]
		\item \( \sin 3x = \dfrac{\sqrt{2}}{2} \)
		\item \( \cos 2x = -1 \)
		\item \( \tg 4x = 0 \)
		\item \( \sin 2,5x = -\dfrac{1}{2} \)
		\item \( \tg \dfrac{1}{5}x = -1 \)
		\item \( \ctg 3x = \sqrt{3} \)
	\end{enumcols}
	\item Решить уравнение:
	\begin{enumcols}[itemcolumns=2]
		\item \( \sin \left( x+\dfrac{\pi}{3} \right) = \dfrac{\sqrt{2}}{2} \)
		\item \( \sin \left( 2x-\dfrac{3\pi}{2} \right) = -1 \)
		\item \( \cos \left( \dfrac{\pi}{4}-x \right)=\dfrac{\sqrt{3}}{2} \)
		\item \( \ctg\left( 2x-\dfrac{3\pi}{4} \right)=-1 \)
	\end{enumcols}
	\item Решить уравнение:
	\begin{enumcols}[itemcolumns=3]
		\item \( \sin x = \dfrac{1}{3} \)
		\item \( \sin x = \dfrac{3}{2}\)
		\item \( \tg 2x = \dfrac{1}{2}\)
	\end{enumcols}
	\item Решить уравнение \( \cos\dfrac{\pi(x-4)}{2}=\dfrac{3}{2} \). В ответ запишите наибольший отрицательный корень.
	\item Решить уравнение \( \sin\dfrac{2\pi x}{3}=\dfrac{1}{2} \). В ответ запишите наибольший отрицательный корень.
\end{listofex}
\newpage
\title{Домашняя работа №2}
\begin{listofex}
	\item Решить уравнение:
	\begin{enumcols}[itemcolumns=2]
		\item \exercise{3497}
		\item \exercise{3424}
		\item \exercise{3421}
		\item \exercise{3427}
		\item \exercise{606}
		\item \exercise{778}
		\item \exercise{784}
	\end{enumcols}
	\item Вычислить:
	\begin{enumcols}[itemcolumns=1]
		\item \( 4\sqrt{3}\cos150\degree\cdot\sin210\degree \) \answer{ \( 3 \) }
		\item \( \dfrac{15\cos395\degree}{\cos35\degree} \)
		\item \( \cos240\degree(\sin45\degree+\sin135\degree)-\sin60\degree(\cos180\degree+\ctg45\degree) \)
	\end{enumcols}
	\item Вычислить:
	\begin{enumcols}[itemcolumns=1]
		\item \( \left( \dfrac{4\tg120\degree\cdot\cos210\degree-\sin270\degree}{2\cos240\degree-3\sqrt{3}\sin210\degree} \right)\cdot\dfrac{5}{3\sqrt{3}+2}-\dfrac{1}{23} \)\answer{ \( 3 \) }
		\item \( \dfrac{\sqrt{8}\sin\left( -\dfrac{\pi}{4} \right)+\sqrt{27}\cos\left( \dfrac{\pi}{3} \right)-4\sin\left( -\dfrac{\pi}{6} \right)}{6\sqrt{3}} \) \answer{ \( 0,25 \) }
		\item \( 4\cos\left( \dfrac{2\pi}{3} \right)-\left( \sqrt{3}+1 \right)\left( \ctg\left( \dfrac{7\pi}{6} \right)-1 \right) \) \answer{ \( -4 \) }
		\item \( \left( 4 - \sin\left( -\dfrac{10\pi}{3} \right) \right)^2+4\tg\left( \dfrac{\pi}{3} \right) \) \answer{ \( 16,75 \) }
	\end{enumcols}
	\item Вычислить:
	\begin{enumcols}[itemcolumns=2]
		\item \( 4\sqrt{2}\tg\dfrac{\pi}{4}\cos\dfrac{7\pi}{3}+11 \)
		\item \( \dfrac{8}{\sin\left( -\dfrac{27\pi}{4}\right)\cos\left( \dfrac{31\pi}{4} \right) } \)
	\end{enumcols}
	\item Вычислить:
	\begin{enumcols}[itemcolumns=2]
		\item \( \dfrac{4\sin22\degree\cos22\degree}{\cos66\degree}+\dfrac{\sin100}{4\sin50\degree\cos50\degree} \)
		\item \( \dfrac{22(\sin^2 16\degree-\cos^2 16\degree)}{\cos32\degree}+5 \)
	\end{enumcols}
	\item Найдите значение выражения \( 5\tg(5\pi-x)-\tg(-x) \), если \( \tg x = 7 \)
	\item Вычислить:
	\begin{enumcols}[itemcolumns=4]
		\item \exercise{1588}
		\item \exercise{1567}
		\item \exercise{1294}
		\item \exercise{590}
		\item \exercise{1569}
	\end{enumcols}
	\item Расстояние между городами \( A \) и \( B \) равно \( 435 \) км. Из города \( A \) в город \( B \) со скоростью \( 60 \) км/ч выехал первый автомобиль, а через час после этого навстречу ему из города B выехал со скоростью \( 65 \) км/ч второй автомобиль. На каком расстоянии от города A автомобили встретятся? Ответ дайте в километрах.
\end{listofex}
\newpage
\title{Занятие №5}
\begin{listofex}
	\item Вычислить:
	\begin{enumcols}[itemcolumns=2]
		\item \exercise{2971}
		\item \exercise{2972}
		\item \exercise{2975}
		\item \exercise{2985}
	\end{enumcols}
	\item Решить уравнения
	\begin{enumcols}[itemcolumns=3]
		\item \( \sin \left( x+\dfrac{\pi}{4} \right) = \dfrac{1}{2} \)
		\item \( \tg \left( 3x-\dfrac{12\pi}{7} \right) = -1 \)
		\item \( 2\cos \left( \dfrac{5\pi}{8}+x \right) = \sqrt{2} \)
	\end{enumcols}
	\item Решить уравнение \( \cos\dfrac{\pi(2x-1)}{3}=\dfrac{1}{2} \). В ответ запишите наименьший положительный корень.
	\item Решить уравнения
	\begin{enumcols}[itemcolumns=2]
		\item \exercise{3631}
		\item \exercise{3678}
		\item \exercise{1175}
		\item \exercise{3403}
		\item \exercise{1185}
		\item \exercise{1186}
	\end{enumcols}
	\item Решить уравнения
	\begin{enumcols}[itemcolumns=2]
		\item \( 2\cos^2x+19\sin x+8=0\) \answer{ \( -\dfrac{\pi}{6}+2\pi n;\;-\dfrac{5\pi}{6}+2\pi n \) }
		\item \( \cos2x +3\sin x-2 =0 \) \answer{ \( \dfrac{\pi}{2}+2\pi n \) }
		\item \( 1-2\cos^2 x=\sin(\pi-x) \) \answer{ \( \pm\dfrac{\pi}{4}+2\pi n \) }
		\item \( \sin x\cdot(2\sin x - 1)+\sqrt{3}\sin x + \sin \dfrac{4\pi}{3}=0 \) \answer{ \( \dfrac{\pi}{6}+2\pi n;\;\dfrac{5\pi}{6}+2\pi n;\; -\dfrac{\pi}{3}+2\pi n;\;-\dfrac{2\pi}{3}+2\pi n \) }
	\end{enumcols}
	
\end{listofex}
\newpage
\title{Занятие №6}
\begin{listofex}
	\item На экзамен вынесено 60 вопросов, Андрей не выучил 3 из них. Найдите вероятность того, что ему попадется выученный вопрос.
	\item На рок-фестивале выступают группы — по одной от каждой из заявленных стран. Порядок выступления определяется жребием. Какова вероятность того, что группа из Дании будет выступать после группы из Швеции и после группы из Норвегии? Результат округлите до сотых.
	\item На борту самолёта 12 кресел расположены рядом с запасными выходами и 18 — за перегородками, разделяющими салоны. Все эти места удобны для пассажира высокого роста. Остальные места неудобны. Пассажир В. высокого роста. Найдите вероятность того, что на регистрации при случайном выборе места пассажиру В. достанется удобное место, если всего в самолёте 300 мест.
	\item Механические часы с двенадцатичасовым циферблатом в какой-то момент сломались и перестали идти. Найдите вероятность того, что часовая стрелка остановилась, достигнув отметки 10, но не дойдя до отметки 1.
	\item За круглый стол на 9 стульев в случайном порядке рассаживаются 7 мальчиков и 2 девочки. Найдите вероятность того, что обе девочки будут сидеть рядом.
	\item За круглый стол на 201 стул в случайном порядке рассаживаются 199 мальчиков и 2 девочки. Найдите вероятность того, что между девочками будет сидеть один мальчик.
	\item В случайном эксперименте симметричную монету бросают дважды. Найдите вероятность того, что орел выпадет ровно один раз.
	\item В случайном эксперименте симметричную монету бросают трижды. Найдите вероятность того, что орел выпадет ровно два раза.
	\item Какова вероятность того, что случайно выбранный телефонный номер оканчивается двумя чётными цифрами?
	\item Если шахматист А. играет белыми фигурами, то он выигрывает у шахматиста Б. с вероятностью 0,52. Если А. играет черными, то А. выигрывает у Б. с вероятностью 0,3. Шахматисты А. и Б. играют две партии, причём во второй партии меняют цвет фигур. Найдите вероятность того, что А. выиграет оба раза.
	\item Из городов \( A  \) и \( B  \) навстречу друг другу выехали мотоциклист и велосипедист. Мотоциклист приехал в \(В\)на \(1\) час раньше, чем велосипедист приехал в \( A \), а встретились они через \(40\) минут после выезда. Сколько часов затратил на путь из \(  B  \) в \( A  \) велосипедист?
	\item Смешали некоторое количество \( 18 \)-процентного раствора некоторого вещества с таким же количеством \( 14 \)-процентного раствора этого вещества. Сколько процентов составляет концентрация получившегося раствора?
	\item Имеется два сплава. Первый сплав содержит \( 5\% \) меди, второй – \( 12\% \) меди. Масса второго сплава больше массы первого на \( 9 \) кг. Из этих двух сплавов получили третий сплав, содержащий \( 10\% \) меди. Найдите массу третьего сплава. Ответ дайте в килограммах.
\end{listofex}
\newpage
\title{Подготовка к проверочной работе}
\begin{listofex}
	\item Вычислить:
\begin{enumcols}[itemcolumns=2]
	\item \( 12\sin150\degree\cdot\cos120\degree \) \answer{ \( -3 \) }
	\item \( \dfrac{12\sin407\degree}{\sin47\degree} \)\answer{ \( 12 \) }
	\item \( \dfrac{5\sin10\degree\cdot\cos10\degree}{\sin20\degree} \)\answer{ \( 2,5 \) }
	\item \( \dfrac{2\sqrt{3}\sin60\degree\cdot\cos60\degree}{\cos^2 30\degree - \sin^2 30\degree} \)\answer{ \( \sqrt{3} \) }
\end{enumcols}
	\item Вычислить:
	\begin{enumcols}[itemcolumns=2]
		\item \( \dfrac{3\cos39\degree}{\sin51\degree}+\dfrac{2\cos31\degree}{\sin59\degree} \) \answer{ \( 5 \) }
		\item \( \dfrac{2\sin388\degree}{\cos242\degree} \) \answer{ \( -2 \) }
		\item \( \dfrac{6\sin33\degree\cos33\degree}{\sin66\degree}+\dfrac{\sin88\degree}{6\sin44\degree\cos44\degree} \) \answer{ \( 3\dfrac{1}{3} \) }
		\item \( \dfrac{10(\sin^2 32\degree-\cos^2 32\degree)}{-4\cos64\degree}+11 \) \answer{ \( 13,5 \) }
	\end{enumcols}
	\item Вычислить:
	\begin{enumcols}[itemcolumns=2]
		\item \( -4\sqrt{3}\sin\left( -\dfrac{7\pi}{3} \right) \) \answer{ \( 6 \) }
		\item \( 2\sqrt{3}\tg\left( -\dfrac{13\pi}{6} \right) \) \answer{ \( -2 \) }
		\item \( (3\sqrt{3})^2\tg\left( \dfrac{\vphantom{7}\pi}{12} \right)\cdot\tg\left( \dfrac{7\pi}{12} \right) \) \answer{ \( -6 \) }
		\item \( \dfrac{7}{\cos^2\left( \dfrac{\vphantom{9}\pi}{16} \right)+\cos^2\left( \dfrac{9\pi}{16} \right)} \) \answer{ \( 7 \) }
		\item \( \sqrt{3}-\sqrt{12}\sin^2\dfrac{10\pi}{12} \) \answer{ \( \dfrac{\sqrt{3}}{2} \) }
		\item \( \dfrac{25}{\sin^2\dfrac{11\pi}{24}+1+\sin^2\dfrac{23\pi}{24}} \)\answer{ \( 12,5 \) }
	\end{enumcols}
	\item Вычислить:
	\begin{enumcols}[itemcolumns=4]
		\item \( \log_4 16 \)
		\item \( \log_{1/5}5\sqrt{5} \)
		\item \( \log_{\sqrt[5]{2}}32 \)
		\item \( \log_{1/7}^2 49 \)
	\end{enumcols}
	\item Вычислить:
	\begin{enumcols}[itemcolumns=1]
		\item \exercise{1095}
		\item \exercise{1093}
	\end{enumcols}
	\item За круглый стол на 17 стульев в случайном порядке рассаживаются 15 мальчиков и 2 девочки. Найдите вероятность того, что девочки будут сидеть рядом.
	\item В случайном эксперименте симметричную монету бросают дважды. Найдите вероятность того, что орел выпадет ровно один раз.
\end{listofex}
\newpage
\title{Проверочная работа}
\begin{listofex}
	\item Вычислить:
	\begin{enumcols}[itemcolumns=3]
		\item \exercise{562}
		\item \exercise{571}
		\item \exercise{1570}
		\item \exercise{1573}
		\item \exercise{1583}
		\item \exercise{595}
		\item \exercise{1594}
	\end{enumcols}
	\item Вычислить:
	\begin{enumcols}[itemcolumns=2]
		\item \exercise{1137}
		\item \exercise{1143}
		\item \exercise{1145}
		\item \exercise{1146}
	\end{enumcols}
	\item Вычислить значение:
	\begin{enumcols}[itemcolumns=3]
		\item \( \dfrac{20\sin13\degree\cdot\cos13\degree}{-\sin26\degree} \)
		\item \( \dfrac{13}{4\sin^237\degree+4\sin^2127\degree} \)
		\item \exercise{2987}
	\end{enumcols}
	\item Вычислить:
	\begin{enumcols}[itemcolumns=2]
		\item \( -4\sqrt{3}\sin\left( -\dfrac{4\pi}{3} \right) \)
		\item \( (2\sqrt{5})^2\tg\left( \dfrac{\vphantom{3}\pi}{4} \right)\cdot\tg\left( \dfrac{3\pi}{4} \right) \)
		\item \( \dfrac{7}{\cos^2\left( \dfrac{\vphantom{9}\pi}{8} \right)+\cos^2\left( \dfrac{5\pi}{8} \right)} \)
		\item \( \sqrt{3}-\sqrt{12}\sin^2\dfrac{7\pi}{12} \)
	\end{enumcols}
	\item Решить уравнение:
	\begin{enumcols}[itemcolumns=2]
		\item \( \log_{1/7}(5-4x)=-1 \)
		\item \( \log_4(3x+3)=\log_4(2x-11) \)
		\item \( \log_5(7-x)=\log_5(3-x)+1 \)
		\item \( \log_4 2^{8x-4}=2 \)
		\item \( \log_5(x^2+13x)=\log_5(9x+5) \)
		\item \( \sin 2x = \dfrac{\sqrt{2}}{2} \)
		\item \( \sin \left( \dfrac{1}{2}\pi-x \right)=1 \)
	\end{enumcols}
	\item \exercise{1117}
	\item Решить уравнение \( \cos\dfrac{\pi(3x+6)}{3}=\dfrac{\sqrt{2}}{2} \). В ответ запишите наименьший положительный корень.
\end{listofex}
