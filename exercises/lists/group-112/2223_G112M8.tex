%
%===============>>  ГРУППА 11-2 МОДУЛЬ 8  <<=============
%
\setmodule{8}

%BEGIN_FOLD % ====>>_____ Занятие 1 _____<<====
\begin{class}[number=1]
	\begin{listofex}
		\item
		\begin{minipage}[t]{\bodywidth}
			На рисунке изображён график функции \[ f(x)=a \cos{x}+b \] Найдите \(a\).
		\end{minipage}
		\hspace{0.02\linewidth}
		\begin{minipage}[t]{\picwidth}
			\includegraphics[align=t, width=\linewidth]{\picpath/MECGERM6H3-1}
		\end{minipage}
		%?
		\item
		\begin{minipage}[t]{\bodywidth}
			На рисунке изображён график функции \[ f(x)=a \tg{x}+b \] Найдите \(a\).
		\end{minipage}
		\hspace{0.02\linewidth}
		\begin{minipage}[t]{\picwidth}
			\includegraphics[align=t, width=\linewidth]{\picpath/MECGERM6H3-2}
		\end{minipage}
		\item %1 параболы
		\begin{minipage}[t]{\bodywidth}
			На рисунке изображены графики функций \(f(x) = 4x^2-25x+41 \) и \( g(x)=ax^2+bx+c \), которые пересекаются в точках \(A\) и \(B\). Найдите абсциссу точки \(B\).
		\end{minipage}
		\hspace{0.02\linewidth}
		\begin{minipage}[t]{\picwidth}
			\includegraphics[align=t, width=\linewidth]{../pics/G112M3C1-10}
		\end{minipage}
		\item %1 LOGARIFM
		\begin{minipage}[t]{\bodywidth}
			На рисунке изображен график функции \(f(x) = b+\log_ax \). Найдите \(f(32)\).
		\end{minipage}
		\hspace{0.02\linewidth}
		\begin{minipage}[t]{\picwidth}
			\includegraphics[align=t, width=\linewidth]{../pics/G111M8L1-1}
		\end{minipage}
		%\item %1 s golovi
		%\begin{minipage}[t]{\bodywidth}
		%	На рисунке изображен график функции \(f(x) = \dfrac{ x^2 }{ a }+bx+c \), где числа \(a, b, c\) --- целые. Найдите значение \(f(3,5)\).
		%\end{minipage}
		%\hspace{0.02\linewidth}
		%\begin{minipage}[t]{\picwidth}
		%	\includegraphics[align=t, width=\linewidth]{../pics/G112M3C1-6}
		%\end{minipage}
		\item Решите неравенства: %c193 23 24 27 28 a
		\begin{tasks}(2)
			\task \( \sqrt[4]{x^2-24x} \le 3 \)
			\task \( \sqrt[28]{8x-x^2-15} < 1 \)
			\task \( \sqrt[]{x^2-2x-15} < 3 \)
			\task \( \sqrt[]{3x^2-14x+51} \ge 6 \)
		\end{tasks}
		\item Решите неравенства:
		\begin{tasks}(1)
			\task \( \sqrt[]{x^4-2x+6} \ge x \)
			\task \( \sqrt[]{5x^4-28x^2+16} \ge x^2+4 \)
		\end{tasks}
		\newpage
		\item Решите неравенства: %с289 1 2 3 5 6 8
		\begin{tasks}(2)
			\task \( 4^{\tfrac{5}{x}} \ge 64 \)
			\task \( \left( \dfrac{ 1 }{ 3 } \right)^{\tfrac{ 3x+2 }{ 1-x }} < 81 \)
			
			\task \( 3^x \cdot \left( \dfrac{ 1 }{ 81 } \right)^{2x+3} < 9 \)
			\task \( 4^{3x-2}+4^{3x-1} \le 80 \)
			\task \( 5^{x-3}+5^{x-2}+5^{x-1} \ge 155 \)
			\task \( (0,2)^{\tfrac{x^2+11x+49}{2x-9}} \ge 5 \)
		\end{tasks}
	\end{listofex}
\end{class}
%END_FOLD

%BEGIN_FOLD % ====>>_____ Занятие 2 _____<<====
\begin{class}[number=2]
	\begin{listofex}
		\item Решить неравенство: 
		\[ 5^x+\dfrac{125}{5^x-126}\ge 0. \]
		\item Решить неравенство:
		\[ \dfrac{2}{5^x+75} \ge \dfrac{1}{5^x-25} \]
		\item Решить неравенство:
		\[ 16^{\tfrac{1}{x}-1}-4^{\tfrac{1}{x}-1}-2 \ge 0 \]
		\item Решить неравенство:
		\[ \dfrac{1}{2^x-1}+\dfrac{4^{x+\tfrac{1}{2}}-2^{x+5}+4}{2^x-16} \ge 2^{x+1} \]
		\item Решить неравенство:
		\[ 27\cdot45^x-27^{x+1}-12\cdot15^x+12\cdot9^x+5^x-3^x \le 0  \]
		\item Решить неравенство:
		\[ 2^{x+1}+\dfrac{9}{x}-\dfrac{3\cdot2^x}{x}\ge 6 \]
		\item Решить неравенство:
		\[ \dfrac{9^x+2 \cdot 3^x-117}{3^x-27} \le 1 \]
		\item Решить неравенство: \[ \dfrac{ 3^{|x^2-2x-1|}-9}{ x } \ge 0 \]
		
		
		\item Решить неравенство: \[ 64^{x^2-3x+20}-0,125^{2x^2-6x-200} \le 0 \]
		\item Решить неравенство: \[ 4^{x^2+x-3}-0,5^{2x^2-6x-2} \le 0 \]
	\end{listofex}
\end{class}
%END_FOLD

%BEGIN_FOLD % ====>>_ Домашняя работа 1 _<<====
\begin{homework}[number=1]
	\begin{listofex}
		\item Решите неравенства: %c192 n 25 26 27 28 b /c289 1 2 4 5 b
		\begin{tasks}(2)
			\task \( \sqrt[3]{9x-x^2} \le 2 \)
			\task \( \sqrt[]{x^2-25} \le 12 \)
			\task \( \sqrt[]{x^2-4x-5} < 4 \)
			\task \( \sqrt{4x^2-29x+61} \ge 3 \)
			\task \( 3^{\tfrac{ 4 }{ x }} \ge 27 \)
			\task \( 0,5^{\tfrac{ 3x-2 }{ 3-x }}<16 \)
			\task \( \left( \dfrac{ 1 }{ 3 } \right)^{\tfrac{ 4x-1 }{ x-5 }} > 81^{\tfrac{ x-2 }{ x+5 }} \)
			\task \( 6^x \cdot \left( \dfrac{ 1 }{ 36 } \right)^{5x+3} < 6 \)
			\task \( \dfrac{ 9^x-3^x-90 }{ 3^x-82 } \le 1 \)
			\task \( 2^{2x-1}-7 \cdot 2^{x-1}+5 \le 0 \)
			\task \( 2^x+80 \cdot 2^{4-x} \le 261 \)
			\task \( 2^{2x+4}-16 \cdot 2^{x+3}-2^{x+1}+16 \le 0 \)
			\item Решить неравенство: \( 6^x-4 \cdot 3^x-2^x+4 \le 0 \)
		\end{tasks}
		%trigon10 8
		\item
		\begin{minipage}[t]{\bodywidth}
			На рисунке изображён график функции вида \[ f(x)= a\cos \left( \dfrac{ \pi x }{ b }+c \right)+d, \] где \(a,b,c, d\) --- целые. Найдите \(f\left( -\dfrac{ 22 }{ 3 } \right)\).
		\end{minipage}
		\hspace{0.02\linewidth}
		\begin{minipage}[t]{\picwidth}
			\includegraphics[align=t, width=\linewidth]{\picpath/G112M8H1-1}
		\end{minipage}
	\end{listofex}
\end{homework}
%END_FOLD

%BEGIN_FOLD % ====>>_____ Занятие 3 _____<<====
\begin{class}[number=3]
	\begin{listofex}
		\item Решите неравенства: %192 25 27 28 a
		\begin{tasks}(2)
			\task \( \sqrt[3]{6x-x^2} \le 2 \)
			\task \( \sqrt[]{x^2-2x-15}<3 \)
			\task \( \sqrt[]{3x^2-14x+51} \ge 6 \)
			\task \( \sqrt[]{x^2-144} \le 5 \)
			\task \( \sqrt[]{x^4-2x+6} \ge x \)
			\task \( \sqrt[]{5x^4-28x^2+16} \ge x^2+4 \)
		\end{tasks}
		\item Решите неравенства: %с289 1 2 3 5 6 8
		\begin{tasks}(2)
			\task \( 4^{\tfrac{5}{x}} \ge 64 \)
			\task \( \left( \dfrac{ 1 }{ 3 } \right)^{\tfrac{ 3x+2 }{ 1-x }} < 81 \)
			
			\task \( 3^x \cdot \left( \dfrac{ 1 }{ 81 } \right)^{2x+3} < 9 \)
			\task \( 4^{3x-2}+4^{3x-1} \le 80 \)
			\task \( 5^{x-3}+5^{x-2}+5^{x-1} \ge 155 \)
			\task \( (0,2)^{\tfrac{x^2+11x+49}{2x-9}} \ge 5 \)
		\end{tasks}
		%trigon10 9
		\item
		\begin{minipage}[t]{\bodywidth}
			На рисунке изображён график функции вида \[ a\cos \left( \dfrac{ \pi x }{ b }+c \right)+d, \] где \(a,b,c, d\) --- целые. Найдите \(f\left( -\dfrac{ 20 }{ 3 } \right)\).
		\end{minipage}
		\hspace{0.02\linewidth}
		\begin{minipage}[t]{\picwidth}
			\includegraphics[align=t, width=\linewidth]{\picpath/G111M8L2-1}
		\end{minipage}
	\end{listofex}
\end{class}
%END_FOLD

%BEGIN_FOLD % ====>>_____ Занятие 4 _____<<====
\begin{class}[number=4]
	\begin{listofex}
		\item Занятие 4
	\end{listofex}
\end{class}
%END_FOLD

%BEGIN_FOLD % ====>>_ Домашняя работа 2 _<<====
\begin{homework}[number=2]
	\begin{listofex}
		\item Домашняя работа 2
	\end{listofex}
\end{homework}
%END_FOLD

%BEGIN_FOLD % ====>>_____ Занятие 5 _____<<====
\begin{class}[number=5]
	\begin{listofex}
		\item Занятие 5
	\end{listofex}
\end{class}
%END_FOLD

%BEGIN_FOLD % ====>>_____ Занятие 6 _____<<====
\begin{class}[number=6]
	\begin{listofex}
		\item Занятие 6
	\end{listofex}
\end{class}
%END_FOLD

%BEGIN_FOLD % ====>>_ Домашняя работа 3 _<<====
\begin{homework}[number=3]
	\begin{listofex}
		\item Домашняя работа 3
	\end{listofex}
\end{homework}
%END_FOLD

%BEGIN_FOLD % ====>>_____ Занятие 7 _____<<====
\begin{class}[number=7]
	\title{Подготовка к проверочной}
	\begin{listofex}
		\item Занятие 7
	\end{listofex}
\end{class}
%END_FOLD

=%BEGIN_FOLD % ====>>_ Проверочная работа _<<====
\begin{exam}
	\begin{listofex}
		\item Проверочная
	\end{listofex}
\end{exam}
%END_FOLD