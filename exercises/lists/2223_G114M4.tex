%
%===============>>  ГРУППА 11-4 МОДУЛЬ 4  <<=============
%
\setmodule{4}
%
%===============>>  Занятие 1  <<===============
%
\begin{class}[number=1]
	\begin{listofex}
		\item В прямоугольном треугольнике угол между высотой и биссектрисой, проведенными из вершины прямого угла, равен \( 21\degree \). Найдите меньший угол данного треугольника. Ответ дайте в градусах.
		\item В треугольнике \( ABC \) \( AC = BC \), \( AB = 10 \), высота \( AH \) равна \( 3 \). Найдите синус угла \( BAC \).
		\item В треугольнике \( ABC \) угол \( C \) равен \( 90\degree \), \( AC=4,8 \) и \( \sin\angle A = \dfrac{7}{25} \). Найдите \( AB \).
		\item В треугольнике \( ABC \) угол \( C \) равен \( 90\degree \), \( CH \) --- высота, \( BC=3 \), а \( \sin A=\dfrac{1}{6} \). Найдите \( AH \).
		\item В треугольнике \( ABC \) угол \( C \) равен \( 90\degree \), высота \( CH=4 \), \( BC=8 \). Найдите \( \cos A \).
		\item Острый угол прямоугольного треугольника равен \( 32\degree \). Найдите острый угол, образованный биссектрисами этого и прямого углов треугольника. Ответ дайте в градусах.
		\item В прямоугольном треугольнике острые углы относятся как \( 1:2 \), а больший
		катет равен \( 4\sqrt{3} \). Найти радиус окружности, описанной около треугольника.
		\item Один из катетов прямоугольного треугольника равен \( 15 \), а проекция второго катета на гипотенузу равна 16. Найдите диаметр окружности, описанной около этого треугольника.
		\item На катете \( AC \) прямоугольного треугольника \( ABC \) как на диаметре построена окружность, пересекающая гипотенузу \( AB \) в точке \( K \). Найдите \( CK \), если \( AC = 2 \) и \( \angle A = 30\degree \).
		\item Докажите, что угол между биссектрисами двух смежных углов прямой.
		\item Окружность касается двух параллельных прямых и их секущей. Докажите, что отрезок секущей, заключенный между параллельными прямыми, виден из центра окружности под
		прямым углом.
		\item В треугольнике \( ABC \) к стороне \( AC \) проведены высота \( BK \) и медиана \( MB \), причем \( AM =BM \). Найдите косинус угла \( KBM \), если \( AB=1 \), \( BC = 2 \).
		\item В прямоугольном треугольнике один из углов равен \( 30\degree \). Докажите, что в этом треугольнике отрезок перпендикуляра, проведенного к гипотенузе через ее середину до
		пересечения с катетом, втрое меньше большего катета.
	\end{listofex}
\end{class}
%
%===============>>  Занятие 2  <<===============
%
\begin{class}[number=2]
	\begin{listofex}
%		\item Добавить 3-4 заадчи на прямогульные треугольники из Решу ЕГЭ.
		\item Острый угол прямоугольного треугольника равен \( 30\degree \), а гипотенуза равна \( 8 \). Найдите отрезки, на которые делит гипотенузу высота, проведенная из вершины прямого угла.
		\item В прямоугольном треугольнике \( ABC \) угол \( C \) прямой, \( CM \) --- медиана треугольника. Найти острые углы треугольника, если угол \( \angle AMC = 42\degree \) равен.
		\item Через точку \( A \), лежащую на окружности, проведены диаметр \( AB \) и хорда \( AC \), причем \( AC = 8 \) и \( \angle BAC = 30\degree \).
		\item Окружность, построенная на катете прямоугольного треугольника как на диаметре, делит гипотенузу в отношении \( 1 : 3 \). Найдите острые углы треугольника.
		Найдите хорду CM, перпендикулярную AB.
		\item На катетах \( AC \) и \( BC \) прямоугольного треугольника \( ABC \) вне его построены квадраты \( ACDE \) и \( CBFK \) \textit{(вершины обоих квадратов перечислены против часовой стрелки)}. Из точек \( E \) и \( F \) на прямую \( AB \) опущены перпендикуляры \( EM \) и \( FN \). Докажите, что \( EM + FN = AB \).
		\item Высота и медиана, проведенные из одной вершины, делят угол треугольника на три равные части. Найдите углы треугольника.
		\item Пусть \( r \) --- радиус окружности, вписанной в прямоугольный треугольник с катетами \( a \), \( b \) и гипотенузой \( c \). Докажите, что \( r=\dfrac{a+b-c}{2} \).
	\end{listofex}
\end{class}
%
%===============>>  Домашняя работа 1  <<===============
%
\begin{homework}[number=1]
	\begin{listofex}
		\item Найдите наименьшее значение функции \( y=9x-\ln(9x)+3 \) на отрезке \( \left[ \dfrac{1}{18};\dfrac{5}{18} \right] \).
		\item Найдите \( \tg x \), если \( \dfrac{6\sin x - 2\cos x}{4\sin x - 3\cos x}=-1 \).
		\item В прямоугольном треугольнике угол между биссектрисойи высотой, проведёнными из вершины прямого угла, равен \( 10\degree \) . Найдите острые углы треугольника.
		\item Катеты прямоугольного треугольника имеют длину 12 и 5. Найдите длину медианы, проведенной к гипотенузе.
		\item В прямоугольном треугольнике один из катетов равен \( 24 \), а косинус угла, противолежащего данному катету, равен \( \dfrac{2\sqrt{10}}{7} \). Найдите гипотенузу и площадь треугольника.
		\item Высоты двух углов треугольника пересекаются под углом \( 130\degree \). Найдите третий угол треугольника.
		\item Угол при вершине B равнобедренного треугольника \( ABC \) равен \( 108\degree \). Перпендикуляр к биссектрисе \( AD \) этого треугольника, проходящий через точку \( D \), пересекает сторону \( AC \) в точке \( E \). Докажите, что \( DE = BD \).
		\item Высота прямоугольного треугольника, опущенная на гипотенузу, равна \( 1 \), один из острых углов равен \( 15\degree \). Найдите гипотенузу.
		\item Точка \( D \) лежит на стороне \( BC \) треугольника \( ABC \).
		В треугольник \( ABD \) и \( ACD \) вписаны окружности с центрами \( O_1 \) и \( O_2 \). Докажите, что отрезок \( O_1O_2 \) виден из точки \( D \) под прямым углом.
	\end{listofex}
\end{homework}
%
%===============>>  Занятие 3  <<===============
%
\begin{class}[number=3]
	\begin{listofex}
		\item Два угла треугольника равны \( 58\degree \) и \( 72\degree \). Найдите тупой угол, который образуют высоты треугольника, выходящие из вершин этих углов. Ответ дайте в градусах.
		\item В треугольнике \( ABC \) угол \( A \) равен \( 30\degree \), угол \( B \) равен \( 86\degree \), \( CD \)  --- биссектриса внешнего угла при вершине \( C \), причем точка \( D \) лежит на прямой \( AB \). На продолжении стороны \( AC \) за точку \( C \) выбрана такая точка \( E \), что \( CE = CB \). Найдите угол \( BDE \). Ответ дайте в градусах.
		\item В параллелограмме \( ABCD \) известно, что \( AD = 21 \), \( AB = 3 \), \( \sin A = \dfrac{6}{7} \). Найдите длину наибольшей высоты параллелограмма.
		\item На сторонах \( AB \) и \( BC \) треугольника \( ABC \) выбраны точки \( M \) и \( N \) соответственно так, что \( MN||AC \). Найдите \( AC \), если \( NM =9 \), \( NC =4 \) и \( NB= AC \).
		\item В трапеции \( ABCD \) (\( AD || BC \)) известно, что \( AB=3 \), \( BC =4 \), \( CD=3\sqrt{10} \) и \( \angle A=90\degree \). Найдите \( AD \).
		\item Стороны прямоугольника \( AB=9 \) и \( BC =24 \). Точка \( M \) --- середина стороны \( DA \). Отрезки \( AC \) и \( MB \) пересекаются в точке \( K \). Найдите \( BK \).
		\item Найдите высоту прямоугольного треугольника, опущенную на
		гипотенузу, если известно, что основание этой высоты делит гипотенузу на отрезки, равные \( 1 \) и \( 4 \).
		\item В равнобедренном треугольнике основание и боковая сторо-
		на равны соответственно \( 5 \) и \( 20 \). Найдите биссектрису треугольника, проведённую из вершины угла при основании.
		\item Медианы прямоугольного треугольника, проведённые к ка-
		тетам, равны \( 2\sqrt{13} \) и\( \sqrt{73} \). Найдите длину медианы, проведённой к гипотенузе.
		\item Докажите, что отличная от \( A \) точка пересечения окружностей, построенных на сторонах \( AB \) и \( AC \) треугольника \( ABC \) как на диаметрах, лежит на прямой \( BC \).
		\item В треугольник \( ABC \) вписана окружность, касающаяся стороны \( AB \) в точке \( M \). Пусть \( AM = x \), \( BC = a \), полупериметр треугольника равен \( p \). Докажите, что \( x = p - a \).
		\item В треугольник со сторонами \( 6 \), \( 10 \) и \( 12 \) вписана окружность. К окружности проведена касательная так, что
		она пересекает две б´ольшие стороны. Найдите периметр отсеченного треугольника.
		\item Прямая, проходящая через точку \( A \) и центр \( O \) вписанной окружности треугольника \( ABC \), вторично пересекает
		описанную окружность этого треугольника в точке \( M \). Докажите, что треугольники \( BOM \) и \( COM \) равнобедренные.
	\end{listofex}
\end{class}
%
%===============>>  Занятие 4  <<===============
% смещение на одно занятие с прошлого месяца
%\begin{class}[number=4]
%	\begin{listofex}
%
%	\end{listofex}
%\end{class}
%
%===============>>  Домашняя работа 2  <<===============
%
%\begin{homework}[number=2]
%	\begin{listofex}
%		\item Основания равнобедренной трапеции равны 51 и 65, а боковые стороны равны 25. Найдите синус острого угла трапеции. Диагональ прямоугольника в полтора раза длиннее одной из его сторон. Другая сторона прямоугольника равна \( 3\sqrt{5} \). Какова длина диагонали?
%		\item Стороны треугольника равны 10, 17 и 21. Найдите высоту треугольника, проведённую из вершины наибольшего угла.
%		\item Продолжения биссектрис остроугольного треуголь-
%			ника ABC пересекают описанную окружность этого треуголь-
%			ника в точках A1, B1, C1. Докажите, что высоты треугольни-
%			ка A1B1C1 лежат на прямых AA1, BB1, CC1.
%	\end{listofex}
%\end{homework}
%
%===============>>  Занятие 5  <<===============
% смещение на одно занятие с прошлого месяца
%\begin{class}[number=5]
%	\begin{listofex}
%		\item Пусто
%	\end{listofex}
%\end{class}
%
%===============>>  Домашняя работа 3  <<===============
%
%\begin{homework}[number=2]
%	\begin{listofex}
%
%	\end{listofex}
%\end{homework}
%\newpage
%\title{Подготовка к проверочной работе}
%\begin{listofex}
%	
%\end{listofex}
%
%===============>>  Занятие 7  <<===============
%
%\begin{class}[number=7]
%	\begin{listofex}
%	
%	\end{listofex}
%\end{class}
%
%===============>>  Провечная работа  <<===============
%
%\begin{exam}
%	\begin{listofex}
%	
%	\end{listofex}
%\end{exam}