\setmodule{9}

%BEGIN_FOLD % ====>>_____ Занятие 1 _____<<====
\begin{class}[number=1]
	\begin{listofex}
		%11.2,3,4 по 3 первых
		\item Найдите точку максимума функции \(y=x^3-48x+17\).
		\item Найдите наименьшее значение функции \(y=x^3-27x\) на отрезке \([0;4]\).
		\item Найдите наибольшее значение функции \(y=x^3-3x+4\) на отрезке \([-2;0]\).
		\item Найдите точку максимума функции \(y=-\dfrac{ x^2+289 }{ x }\).
		\item Найдите точку минимума функции \(y=-\dfrac{ x^2+1 }{ x }\).
		\item Найдите наименьшее значение функции \(y=\dfrac{ x^2+25 }{ x }\) на отрезке \([1;10]\).
		\item Найдите наименьшее значение функции \(y=(x-8)e^{x-7}\) на отрезке \([6;8]\).
		\item Найдите точку минимума функции \(y=(x+16)e^{x-16}\).
		\item Найдите точку максимума функции \(y=(9-x)e^{x+9}\).
		\item Изюм получается в процессе сушки винограда. Сколько килограммов винограда потребуется для получения \(20\) килограммов изюма, если виноград содержит \(90\%\) воды, а изюм содержит \(5\%\) воды?
		%\item Изюм получается в процессе сушки винограда. Сколько килограммов винограда потребуется для получения \(82\) килограммов изюма, если виноград содержит \(90\%\) воды, а изюм содержит \(5\%\) воды?
		%\item Изюм получается в процессе сушки винограда. Сколько килограммов винограда потребуется для получения \(16\) килограммов изюма, если виноград содержит \(90\%\) воды, а изюм содержит \(5\%\) воды?
		%BANK https://math-ege.sdamgia.ru/test?likes=99575 // https://math-ege.sdamgia.ru/test?likes=99576 // https://math-ege.sdamgia.ru/test?likes=99576 // https://math-ege.sdamgia.ru/test?likes=99578
		%по 2
		\item Имеется два сплава. Первый сплав содержит \(5\%\) меди, второй --- \(12\%\) меди. Масса второго сплава больше массы первого на \(9\) кг. Из этих двух сплавов получили третий сплав, содержащий \(10\%\) меди. Найдите массу третьего сплава. Ответ дайте в килограммах.
		%\item Имеется два сплава. Первый сплав содержит \(5\%\) меди, второй --- \(13\%\) меди. Масса второго сплава больше массы первого на \(9\) кг. Из этих двух сплавов получили третий сплав, содержащий \(10\%\) меди. Найдите массу третьего сплава. Ответ дайте в килограммах.
		
		\item Смешав \(11\)-процентный и \(72\)-процентный растворы кислоты и добавив \(10\) кг чистой воды, получили \(31\)-процентный раствор кислоты. Если бы вместо \(10\) кг воды добавили \(10\) кг \(50\)-процентного раствора той же кислоты, то получили бы \(51\)-процентный раствор кислоты. Сколько килограммов \(11\)-процентного раствора использовали для получения смеси?
		%\item Смешав \(55\)-процентный и \(97\)-процентный растворы кислоты и добавив \(10\) кг чистой воды, получили \(65\)-процентный раствор кислоты. Если бы вместо \(10\) кг воды добавили \(10\) кг \(50\)-процентного раствора той же кислоты, то получили бы \(75\)-процентный раствор кислоты. Сколько килограммов \(55\)-процентного раствора использовали для получения смеси?
		
		%\item Имеется два сплава. Первый содержит \(10\%\) никеля, второй --- \(35\%\) никеля. Из этих двух сплавов получили третий сплав массой \(150\) кг, содержащий \(30\%\) никеля. На сколько килограммов масса первого сплава была меньше массы второго?
		\item Имеется два сплава. Первый содержит \(15\%\) никеля, второй --- \(35\%\) никеля. Из этих двух сплавов получили третий сплав массой \(140\) кг, содержащий \(30\%\) никеля. На сколько килограммов масса первого сплава была меньше массы второго?
		
		
		\item Имеются два сосуда. Первый содержит \(30\) кг, а второй --- \(15\) кг раствора кислоты различной концентрации. Если эти растворы смешать, то получится раствор, содержащий \(34\%\) кислоты. Если же смешать равные массы этих растворов, то получится раствор, содержащий \(46\%\) кислоты. Сколько килограммов кислоты содержится в первом сосуде?
		%\item Имеются два сосуда. Первый содержит \(100\) кг, а второй --- \(20\) кг раствора кислоты различной концентрации. Если эти растворы смешать, то получится раствор, содержащий \(67\%\) кислоты. Если же смешать равные массы этих растворов, то получится раствор, содержащий \(77\%\) кислоты. Сколько килограммов кислоты содержится в первом сосуде?
	\end{listofex}
\end{class}
%END_FOLD

%BEGIN_FOLD % ====>>_____ Занятие 2 _____<<====
\begin{class}[number=2]
	\begin{listofex}
		%11.2,3 по 4-5 И 11.4 3 первых
		\item Найдите точку максимума функции \(y=x^3-3x^2+2\).
		\item Найдите точку минимума функции \(y=2x^3-5x^2+1\).
		\item Найдите наибольшее значение функции \(y=\dfrac{ x^2+25 }{ x }\) на отрезке \([-10;-1]\).
		\item Найдите точку максимума функции \(y=\dfrac{ 16 }{x  }+x+3\).
		\item Найдите наименьшее значение функции \(y=(x-8)e^{x-7}\) на отрезке \([6;8]\).
		\item Найдите точку минимума функции \(y=(x+16)e^{x-16}\).
		\item Найдите точку максимума функции \(y=(9-x)e^{x+9}\).
		%по 2
		\item Имеется два сплава. Первый сплав содержит \(5\%\) меди, второй --- \(12\%\) меди. Масса второго сплава больше массы первого на \(9\) кг. Из этих двух сплавов получили третий сплав, содержащий \(10\%\) меди. Найдите массу третьего сплава. Ответ дайте в килограммах.
		%\item Имеется два сплава. Первый сплав содержит \(5\%\) меди, второй --- \(13\%\) меди. Масса второго сплава больше массы первого на \(9\) кг. Из этих двух сплавов получили третий сплав, содержащий \(10\%\) меди. Найдите массу третьего сплава. Ответ дайте в килограммах.
		
		\item Смешав \(11\)-процентный и \(72\)-процентный растворы кислоты и добавив \(10\) кг чистой воды, получили \(31\)-процентный раствор кислоты. Если бы вместо \(10\) кг воды добавили \(10\) кг \(50\)-процентного раствора той же кислоты, то получили бы \(51\)-процентный раствор кислоты. Сколько килограммов \(11\)-процентного раствора использовали для получения смеси?
		%\item Смешав \(55\)-процентный и \(97\)-процентный растворы кислоты и добавив \(10\) кг чистой воды, получили \(65\)-процентный раствор кислоты. Если бы вместо \(10\) кг воды добавили \(10\) кг \(50\)-процентного раствора той же кислоты, то получили бы \(75\)-процентный раствор кислоты. Сколько килограммов \(55\)-процентного раствора использовали для получения смеси?
		
		%\item Имеется два сплава. Первый содержит \(10\%\) никеля, второй --- \(35\%\) никеля. Из этих двух сплавов получили третий сплав массой \(150\) кг, содержащий \(30\%\) никеля. На сколько килограммов масса первого сплава была меньше массы второго?
		\item Имеется два сплава. Первый содержит \(15\%\) никеля, второй --- \(35\%\) никеля. Из этих двух сплавов получили третий сплав массой \(140\) кг, содержащий \(30\%\) никеля. На сколько килограммов масса первого сплава была меньше массы второго?
		
		
		\item Имеются два сосуда. Первый содержит \(30\) кг, а второй --- \(15\) кг раствора кислоты различной концентрации. Если эти растворы смешать, то получится раствор, содержащий \(34\%\) кислоты. Если же смешать равные массы этих растворов, то получится раствор, содержащий \(46\%\) кислоты. Сколько килограммов кислоты содержится в первом сосуде?
		%\item Имеются два сосуда. Первый содержит \(100\) кг, а второй --- \(20\) кг раствора кислоты различной концентрации. Если эти растворы смешать, то получится раствор, содержащий \(67\%\) кислоты. Если же смешать равные массы этих растворов, то получится раствор, содержащий \(77\%\) кислоты. Сколько килограммов кислоты содержится в первом сосуде?
	\end{listofex}
\end{class}
%END_FOLD

%BEGIN_FOLD % ====>>_ Домашняя работа 1 _<<====
\begin{homework}[number=1]
	\begin{listofex}
		\item Домашняя работа 1
	\end{listofex}
\end{homework}
%END_FOLD

%BEGIN_FOLD % ====>>_____ Занятие 3 _____<<====
\begin{class}[number=3]
	\begin{listofex}
		\item Занятие 3 
	\end{listofex}
\end{class}
%END_FOLD

%BEGIN_FOLD % ====>>_____ Занятие 4 _____<<====
\begin{class}[number=4]
	\begin{listofex}
		\item Занятие 4
	\end{listofex}
\end{class}
%END_FOLD

%BEGIN_FOLD % ====>>_ Домашняя работа 2 _<<====
\begin{homework}[number=2]
	\begin{listofex}
		\item Домашняя работа 2
	\end{listofex}
\end{homework}
%END_FOLD

%BEGIN_FOLD % ====>>_____ Занятие 5 _____<<====
\begin{class}[number=5]
	\begin{listofex}
		\item Занятие 5
	\end{listofex}
\end{class}
%END_FOLD

%BEGIN_FOLD % ====>>_____ Занятие 6 _____<<====
\begin{class}[number=6]
	\begin{listofex}
		\item Занятие 6
	\end{listofex}
\end{class}
%END_FOLD

%BEGIN_FOLD % ====>>_ Домашняя работа 3 _<<====
\begin{homework}[number=3]
	\begin{listofex}
		\item Домашняя работа 3
	\end{listofex}
\end{homework}
%END_FOLD

%BEGIN_FOLD % ====>>_____ Занятие 7 _____<<====
\begin{class}[number=7]
	\title{Подготовка к проверочной}
	\begin{listofex}
		\item Занятие 7
	\end{listofex}
\end{class}
%END_FOLD

%BEGIN_FOLD % ====>>_ Проверочная работа _<<====
\begin{exam}
	\begin{listofex}
		\item Проверочная
	\end{listofex}
\end{exam}
%END_FOLD