%11 класс Проифль Модуль 1 Занятие №4
\begin{listofex}
	\item Найти значение выражения:
	\begin{enumcols}[itemcolumns=3]
		\item \exercise{2829}
		\item \exercise{2828}
		\item \exercise{1215}
		\item \exercise{2826}
		\item \exercise{2825}
		\item \exercise{2824}
		\item \exercise{2837}
	\end{enumcols}
	\item Вычислить:
	\begin{enumcols}[itemcolumns=3]
		\item \exercise{1744}
		\item \exercise{1738}
		\item \exercise{2827}
	\end{enumcols}
	\item Найти значение выражения:
	\begin{enumcols}[itemcolumns=2]
		\item \exercise{2838}
		\item \exercise{2841}
		\item \exercise{1743}
	\end{enumcols}
	\item Найти значение выражения:
	\begin{enumcols}[itemcolumns=1]
		\item \exercise{1338}
		\item \exercise{1339}
	\end{enumcols}
	\item \exercise{1326}
	\item Найти значение выражения:
	\begin{enumcols}[itemcolumns=1]
		\item \exercise{1328}
		\item \exercise{1334}
	\end{enumcols}
	\item Решить уравнения:
	\begin{enumcols}[itemcolumns=2]
		\item \( \sqrt{15-2x}=3 \)
		\item \( \sqrt{\dfrac{6}{4x-54}}=\dfrac{1}{7} \)
		\item \( \sqrt{-72-17x}=-x \)
		\item \( \sqrt[3]{x-4}=3 \)
	\end{enumcols}
\end{listofex}
\newpage
\title{Занятие №4}
\begin{listofex}
	\item Вычислить:
	\begin{enumcols}[itemcolumns=2]
		\item \exercise{1690}
		\item \exercise{1694}
		\item \exercise{1783}
		\item \exercise{1720}
		\item \exercise{1723}
		\item \exercise{1785}
	\end{enumcols}
	\item Вычислить:
	\begin{enumcols}[itemcolumns=2]
		\item \exercise{1733}
		\item \exercise{1737}
		\item \exercise{1773}
		\item \exercise{1756}
		\item \( (\sqrt{2}+1)^2+(\sqrt{2}-1)^2 \) \answer{\( 6 \)}
		\item \( (\sqrt{7}-2)^2+4\sqrt{7} \) \answer{\( 11 \)}
	\end{enumcols}
	\item Упростить выражение:
	\begin{enumcols}[itemcolumns=2]
		\item \( \sqrt{2}+3\sqrt{32}+\dfrac{1}{2}\sqrt{128}-6\sqrt{18} \) \answer{\( -\sqrt{2} \)}
		\item \exercise{1763}
		\item \exercise{1757}
		\item \exercise{1758}
		\item \exercise{1689}
	\end{enumcols}
	%\item Освободитесь от иррациональности в знаменателе:
	%\begin{enumcols}[itemcolumns=5]
	%	\item \( \dfrac{3\sqrt{2}+2\sqrt{2}}{\sqrt{200}} \)
	%	\item \( \dfrac{\sqrt{5}+5}{\sqrt{5}} \)
	%	\item \( \dfrac{1}{\sqrt{2}-1} \)
	%	\item \( \dfrac{2}{\sqrt{3}-1} \)
	%	\item \( \dfrac{\sqrt{5}-\sqrt{3}}{\sqrt{5}+\sqrt{3}} \)
	%\end{enumcols}
	\item Упростить выражение:
	\begin{enumcols}[itemcolumns=3]
		%\item \( \dfrac{5\sqrt{x}-2}{\sqrt{x}}-\dfrac{\sqrt{x}}{x} \)
		\item \( \dfrac{\sqrt{x}}{\sqrt{x}-1}-\dfrac{\sqrt{x}}{x-1} \) \answer{\( \dfrac{x}{x-1} \)}
		%\item \( \dfrac{\sqrt{x}}{\sqrt{x}-6}-\dfrac{3}{\sqrt{x}+6}+\dfrac{x}{36-x} \)
		%\item \( \left( \dfrac{\sqrt{x}}{\sqrt{x}+1}+1 \right):\left( \dfrac{\sqrt{x}+1}{\sqrt{x}-1} \right) \)
		\item \( \dfrac{x-1}{x-2\sqrt{x}+1}-\dfrac{\sqrt{x}+1}{\sqrt{x}-1} \) \answer{\( 0 \)}
		%\item \( \dfrac{x\sqrt{x}-1}{x-4\sqrt{x}+3}-\dfrac{\sqrt{x}+10}{\sqrt{x}-3} \)
	\end{enumcols}
	%\item Упростить выражение:
	%\[ \left( \dfrac{2x\sqrt{y}}{2\sqrt{x}-\sqrt{y}}-\dfrac{y\sqrt{x}}{2\sqrt{x}+\sqrt{y}} \right)\cdot\dfrac{2\sqrt{x}-\sqrt{y}}{4\sqrt{x^3y}+\sqrt{xy^3}} \]
	\item Найти значение выражения \( x-\sqrt{(10-x)^2} \),\quad если \( x>10 \) \answer{\( 10 \)}
	%\begin{enumcols}[itemcolumns=1]
	%	%\item \( 2x-\sqrt{(2x-3)^2} \),\quad если \( x<1,5\)
	%	%\item \( \dfrac{x-16}{\sqrt{x}-4}-\dfrac{x-36}{\sqrt{x}+6} \),\quad если \( x>16 \)
	%	%\item \( \sqrt{x-3}-|\sqrt{x-3}+1| \),\quad если \( x=\pi\)
	%	%\item \( |\sqrt{x+5}-3|+\sqrt{x+5} \),\quad если \( -5\le x< -\pi\)
	%	%\item \( \sqrt{(x+4)^2}-\sqrt{x^2-6x+9} \),\quad если \( -4\le x \le 3\)
	%	%\item \( 4x+\sqrt{9-x^2}+|\sqrt{9-x^2}-3| \),\quad если \( x=2,5\)
	%\end{enumcols}
	\item Вычислить:
	\begin{enumcols}[itemcolumns=2]
		%\item \( \sqrt{11-4\sqrt{7}}-\sqrt{7} \)
		%\item \( \sqrt{17-6\sqrt{8}}+\sqrt{8} \)
		%\item \( \dfrac{\sqrt{2+\sqrt{3}}-\sqrt{2-\sqrt{3}}}{\sqrt{2}} \)
		\item \exercise{1646}
		\item \exercise{1635}
	\end{enumcols}
\end{listofex}
\newpage
\title{Домашняя работа №2}
\begin{listofex}
	\item Вычислить:
	\begin{enumcols}[itemcolumns=2]
		\item \exercise{1757}
		\item \exercise{1661}
		\item \exercise{1775}
		\item \exercise{1638}
	\end{enumcols}
	\item Вычислить:
	\begin{enumcols}[itemcolumns=2]
		\item \exercise{1763}
		\item \exercise{1741}
	\end{enumcols}
	\item Найти значение выражения:
	\begin{enumcols}[itemcolumns=2]
		\item \exercise{1663}
		\item \exercise{1650}
	\end{enumcols}
	\item \exercise{17}
	\item Найти значение выражения:
	\begin{enumcols}[itemcolumns=2]
		\item \exercise{1668}
		\item \exercise{1634}
	\end{enumcols}
	\item \exercise{1522}
	\item Найти значение выражения \( 2x-\sqrt{(2x-3)^2} \),\quad если \( x<1,5\)
	\item Решить уравнения:
	\begin{enumcols}[itemcolumns=2]
		\item \exercise{3696}
		\item \( \sqrt{12-3x}=4 \)
		\item \( \sqrt{\dfrac{4}{2x-21}}=\dfrac{1}{5} \)
		\item \exercise{3403}
	\end{enumcols}
\end{listofex}
\newpage
\title{Занятие №5}
\begin{listofex}
	\item Вычислить:
	\begin{enumcols}[itemcolumns=3]
		\item \exercise{1109}
		\item \exercise{1111}
		\item \exercise{1110}
		\item \exercise{1341}
		\item \exercise{1342}
		\item \exercise{1421}
	\end{enumcols}
	\item Вычислить:
	\begin{enumcols}[itemcolumns=2]
		\item \exercise{1404}
		\item \exercise{1542}
	\end{enumcols}
	\item Найдите значение выражения:
	\begin{enumcols}[itemcolumns=2]
		\item \exercise{1103}
		\item \exercise{1340}
		\item \exercise{1104}
	\end{enumcols}
	\item Найдите значение выражения:
	\begin{enumcols}[itemcolumns=1]
		\item \exercise{1106}
		\item \exercise{1105}
		\item \exercise{1291}
	\end{enumcols}
	\item \exercise{1344}
	\item Найти значение выражения:\\
	\textit{Пример:} \( \sqrt{11-4\sqrt{7}}=\sqrt{\displaystyle2\mathstrut^2+\sqrt{7}\mathstrut^2-2\sqrt{2\cdot7}}=\sqrt{\displaystyle(2-\sqrt{7})^2}=\left|2-\sqrt{7}\right|=\sqrt{7} - 2 \)
	\begin{enumcols}[itemcolumns=2]
		\item \exercise{1332}
		\item \exercise{1331}
	\end{enumcols}
	%\item \exercise{1345}
\end{listofex}
\newpage
\title{Занятие №6}
\begin{listofex}
	\item Вычислить:
	\begin{enumcols}[itemcolumns=2]
		\item \exercise{1329}
		\item \exercise{1850}
		\item \exercise{1398}
		\item \exercise{1421}
		\item \exercise{1378}
		\item \exercise{1845}
	\end{enumcols}
	\item \exercise{1846}
	\item Упростить и вычислить:
	\begin{enumcols}[itemcolumns=2]
		\item \exercise{1528}
		\item \exercise{1495}
	\end{enumcols}
	\item \exercise{1538}
	\item \exercise{1544}
	\item \exercise{1291}
	%\item \exercise{1345}
\end{listofex}
\newpage
\title{Домашняя работа №3}
\begin{listofex}
	\item Вычислить:
	\begin{enumcols}[itemcolumns=2]
		\item \exercise{1849}
		\item \exercise{1551}
		\item \exercise{1847}
		\item \exercise{1112}
	\end{enumcols}
	\item Вычислить:
	\begin{enumcols}[itemcolumns=2]
		\item \exercise{1853}
		\item \exercise{1469}
	\end{enumcols}
	\item Упростить и вычислить:
	\begin{enumcols}[itemcolumns=2]
		\item \exercise{1506}
		\item \exercise{1290}
		\item \exercise{1231}
	\end{enumcols}
	\item Упростить:
	\begin{enumcols}[itemcolumns=2]
		\item \exercise{1754}
		\item \exercise{1538}
	\end{enumcols}
	\item \exercise{1543}
	\item \exercise{1229}
\end{listofex}
\newpage
\title{Занятие №7}
\begin{listofex}
	\item Вычислить:
	\begin{enumcols}[itemcolumns=2]
		\item \( \dfrac{6n^{\tfrac{1}{3}}}{n^{\tfrac{1}{12}}\cdot n^{\tfrac{1}{4}}} \) \answer{\( 6 \)}
		\item \( \dfrac{(\sqrt[3]{7a^2})^6}{a^4} \) \answer{\( 49 \)}
		\item \( \dfrac{b^{3\sqrt{2}+2}}{(b^{\sqrt{2}})^3} \), если \( b=13 \) \answer{\( 169 \)}
		\item \( \dfrac{f(x-6)}{f(x-8)} \), если \( f(x)=13\cdot4^x \) \answer{\( 16 \)}
	\end{enumcols}
	\item Решить уравнение:
	\begin{enumcols}[itemcolumns=2]
		\item \( \dfrac{13x}{2x^2-7}=1 \) \answer{\( -0,5;\;7 \)}
		\item \( \dfrac{x-6}{7x+3}=\dfrac{x-6}{5x-1} \) \answer{\( -2;\;6 \)}
		\item \exercise{3678}
		\item \exercise{3627}
	\end{enumcols}
	\item Решить уравнение:
	\begin{enumcols}[itemcolumns=2]
		\item \( \sqrt{\dfrac{2x+5}{3}}=5 \) \answer{\( 35 \)}
		\item \( \sqrt[3]{x-4}=3 \) \answer{\( 31 \)}
		\item \( \sqrt{6+5x}=x \) \answer{\( 6 \)}
		\item \( \sqrt{34-3x}=x-2 \) \answer{\( 6 \)}
		\item \exercise{1171}
		\item \exercise{513}
	\end{enumcols}
	\item Решить уравнение:
	\begin{enumcols}[itemcolumns=2]
		\item \exercise{1165}
		\item \exercise{1184}
		\item \exercise{3421}
		\item \exercise{3424}
		\item \exercise{3499}
		\item \exercise{3497}
	\end{enumcols}
\end{listofex}
\newpage
\title{Проверочная работа}
\begin{listofex}
	\item \exercise{4092}
	\item Вычислить:
	\begin{enumcols}[itemcolumns=2]
		\item \exercise{1609}
		\item \exercise{1601}
	\end{enumcols}
	\item \exercise{1302}
	\item Решить уравнение:
	\begin{enumcols}[itemcolumns=3]
		\item \( \dfrac{1}{4x-1}=5 \)
		\item \( \dfrac{1}{3x-4}=\dfrac{1}{4x-11} \)
		\item \exercise{3665}
	\end{enumcols}
	\item Вычислить:
	\begin{enumcols}[itemcolumns=3]
		\item \exercise{1757}
		\item \exercise{1775}
		\item \exercise{1741}
	\end{enumcols}
	\item Решить уравнение:
	\begin{enumcols}[itemcolumns=2]
		\item \( \sqrt{12-3x}=4 \)
		\item \exercise{3403}
	\end{enumcols}
	\item Найдите значение выражения:
	\begin{enumcols}[itemcolumns=1]
		\item \exercise{1105}
		\item \( f(x-4) \),\quad если \( f(x)=3^{x+6} \)
	\end{enumcols}
	\item Вычислить:
	\begin{enumcols}[itemcolumns=2]
		\item \exercise{1329}
		\item \exercise{1378}
	\end{enumcols}
	\item \exercise{1528}
\end{listofex}
\newpage
\title{Домашняя работа №4}
\begin{listofex}
	\item Вычислить:
	\begin{enumcols}[itemcolumns=2]
		\item \( \dfrac{(9b)^{1,5}\cdot b^{2,7}}{b^{4,2}} \)
		\item \( \dfrac{(4a)^{2,5}}{a^2\cdot\sqrt{a}} \)
		\item \( 6x\cdot(3x^{12})^3:(3x^9)^4 \) при \( x=75 \)
		\item \( \dfrac{f(x-1)}{f(x-3)} \), если \( f(x)=5^x \)
	\end{enumcols}
	\item Решить уравнение:
	\begin{enumcols}[itemcolumns=2]
		\item \( \dfrac{9}{x^2-16}=1 \)
		\item \( \dfrac{x+89}{x-7}=\dfrac{-5}{x-7} \)
		\item \exercise{3656}
	\end{enumcols}
\end{listofex}
%\newpage
%\title{Занятие №7}
%\begin{listofex}
%	\item Вычислить:
%	\begin{enumcols}[itemcolumns=3]
%		\item \exercise{562}
%		\item \exercise{564}
%		\item \exercise{569}
%		\item \exercise{571}
%		\item \exercise{579}
%	\end{enumcols}
%	\item Вычислить:
%	\begin{enumcols}[itemcolumns=3]
%		\item \exercise{1577}
%		\item \exercise{1578}
%		\item \exercise{1579}
%	\end{enumcols}
%	\item Вычислить:
%	\begin{enumcols}[itemcolumns=3]
%		\item \exercise{1572}
%		\item \exercise{1565}
%		\item \exercise{1566}
%		\item \exercise{1573}
%		\item \exercise{1567}
%		\item \exercise{1575}
%		\item \exercise{1594}
%	\end{enumcols}
%	\item Вычислить:
%	\begin{enumcols}[itemcolumns=2]
%		\item \exercise{1569}
%		\item \exercise{1570}
%		\item \exercise{1571}
%		\item \exercise{1574}
%	\end{enumcols}
%	\item Вычислить:
%	\begin{enumcols}[itemcolumns=3]
%		\item \exercise{1580}
%		\item \exercise{1581}
%		\item \exercise{1582}
%	\end{enumcols}
%	\item Вычислить:
%	\begin{enumcols}[itemcolumns=2]
%		\item \exercise{1583}
%		\item \exercise{1584}
%		\item \exercise{1585}
%		\item \exercise{1586}
%		\item \exercise{595}
%	\end{enumcols}
%\end{listofex}
%\newpage
%\title{Занятие №8}
%\begin{listofex}
%	\item Вычислить:
%	\begin{enumcols}[itemcolumns=5]
%		\item \exercise{563}
%		\item \exercise{569}
%		\item \exercise{581}
%		\item \exercise{1589}
%		\item \exercise{1595}
%	\end{enumcols}
%	\item Вычислить:
%	\begin{enumcols}[itemcolumns=2]
%		\item \exercise{582}
%		\item \exercise{584}
%		\item \exercise{591}
%		\item \exercise{594}
%		\item \exercise{1596}
%	\end{enumcols}
%	\item Вычислить:
%	\begin{enumcols}[itemcolumns=2]
%		\item \exercise{592}
%		\item \exercise{1597}
%	\end{enumcols}
%	\item Вычислить:
%	\begin{enumcols}[itemcolumns=3]
%		\item \exercise{1578}
%		\item \exercise{1580}
%		\item \exercise{1581}
%	\end{enumcols}
%	\item Вычислить:
%	\begin{enumcols}[itemcolumns=2]
%		\item \exercise{1580}
%		\item \exercise{1582}
%	\end{enumcols}
%\end{listofex}
%\newpage
%\title{Домашняя работа №4}
%\begin{listofex}
%	\item Вычислить:
%	\begin{enumcols}[itemcolumns=5]
%		\item \exercise{588}
%		\item \exercise{1590}
%		\item \exercise{1293}
%		\item \exercise{567}
%		\item \exercise{580}
%		\item \exercise{570}
%		\item \exercise{573}
%		\item \exercise{572}
%	\end{enumcols}
%	\item Вычислить:
%	\begin{enumcols}[itemcolumns=2]
%		\item \exercise{582}
%		\item \exercise{590}
%		\item \exercise{595}
%		\item \exercise{1587}
%	\end{enumcols}
%	\item Вычислить:
%	\begin{enumcols}[itemcolumns=2]
%		\item \exercise{1568}
%		\item \exercise{1576}
%		\item \exercise{1573}
%		\item \exercise{1588}
%	\end{enumcols}
%	\item Вычислить:
%	\begin{enumcols}[itemcolumns=3]
%		\item \exercise{586}
%		\item \exercise{1585}
%	\end{enumcols}
%	\item Вычислить:
%	\begin{enumcols}[itemcolumns=2]
%		\item \exercise{1583}
%		\item \exercise{595}
%		\item \exercise{1585}
%	\end{enumcols}
%\end{listofex}