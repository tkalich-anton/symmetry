%
%===============>>  ГРУППА 11-1 МОДУЛЬ 8  <<=============
%
\setmodule{8}

%BEGIN_FOLD % ====>>_____ Занятие 1 _____<<====
\begin{class}[number=1]
	\begin{listofex} %Взяты 222 a 2 4 5 6 c193 25 26 a
		%\item Решите уравнения: %c51 1 2 4 5 a
		%\begin{tasks}(2)
		%	\task \( \dfrac{ x^2-9 }{ \sqrt[]{-5x} }=0 \)
		%	\task \( (x^2-16)\sqrt[]{3-x}=0 \)
		%	\task \( \dfrac{ x^2-10x+9 }{ \sqrt[]{x}-3 }=0 \)
		%	\task \( \sqrt[]{2x^2+7}=x-4 \)
		%\end{tasks}
		%\item Решите неравенства: %c193 23 24 27 28 a
		%\begin{tasks}(2)
		%	\task \( \sqrt[4]{x^2-24x} \le 3 \)
		%	\task \( \sqrt[28]{8x-x^2-15} < 1 \)
		%	\task \( \sqrt[]{x^2-2x-15} < 3 \)
		%	\task \( \sqrt[]{3x^2-14x+51} \ge 6 \)
		%\end{tasks}
		%\item Решите неравенства:
		%\begin{tasks}(1)
		%	\task \( \sqrt[]{x^4-2x+6} \ge x \)
		%	\task \( \sqrt[]{5x^4-28x^2+16} \ge x^2+4 \)
		%	\task \( \sqrt[]{x^2-17x-29} \ge 3|x+2| \)
		%\end{tasks}
		\item Вычислить: %114m3
		\begin{tasks}(2)
			\task \( \dfrac{-13\sin126\degree}{\sin54\degree} \)
			\task \( \cos^2(-46\degree)+\sin^2(-46\degree) \)
			\task \( \sin^223\degree+9+\cos^223 \)
			\task \( \dfrac{2\sin^221\degree+2\cos^221\degree}{4} \)
		\end{tasks}
		%\item Вычислить: %114m3
		%\begin{tasks}(1)
		%	%\task \( \left( \dfrac{4\tg120\degree\cdot\cos210\degree-\sin270\degree}{2\cos240\degree-3\sqrt{3}\sin210\degree} \right)\cdot\dfrac{5}{3\sqrt{3}+2}-\dfrac{1}{23} \)\answer{ \( 3 \) }
		%	\task \( \dfrac{\sqrt{8}\sin\left( -\dfrac{\pi}{4} \right)+\sqrt{27}\cos\left( \dfrac{\pi}{3} \right)-4\sin\left( -\dfrac{\pi}{6} \right)}{6\sqrt{3}} \) \answer{ \( 0,25 \) }
		%	\task \( 4\cos\left( \dfrac{2\pi}{3} \right)-\left( \sqrt{3}+1 \right)\left( \ctg\left( \dfrac{7\pi}{6} \right)-1 \right) \) \answer{ \( -4 \) }
		%	\task \( \left( 4 - \sin\left( -\dfrac{10\pi}{3} \right) \right)^2+4\tg\left( \dfrac{\pi}{3} \right) \) \answer{ \( 16,75 \) }
		%\end{tasks}
		\item Найдите значение выражения: \( \sin\left( \dfrac{\pi}{3}+\alpha \right) \), если \( \cos\alpha=-\dfrac{8}{17} \) и \\ \( \pi<\alpha<\dfrac{3\pi}{2} \).
		%\item Найдите значение выражения: \( \dfrac{2\cos^2x-7\sin^2x}{3\cos^2x+4\sin x\cdot\cos x} \), если \( \ctg x = -2 \).
		
		%\item %1 параболы
		%\begin{minipage}[t]{\bodywidth}
		%	На рисунке изображены графики функций \(f(x) = 4x^2-25x+41 \) и \( g(x)=ax^2+bx+c \), которые пересекаются в точках \(A\) и \(B\). Найдите абсциссу точки \(B\).
		%\end{minipage}
		%\hspace{0.02\linewidth}
		%\begin{minipage}[t]{\picwidth}
		%	\includegraphics[align=t, width=\linewidth]{../pics/G112M3C1-10}
		%\end{minipage}
		%\item %1 LOGARIFM
		%\begin{minipage}[t]{\bodywidth}
		%	На рисунке изображен график функции \(f(x) = b+\log_ax \). Найдите \(f(32)\).
		%\end{minipage}
		%\hspace{0.02\linewidth}
		%\begin{minipage}[t]{\picwidth}
		%	\includegraphics[align=t, width=\linewidth]{../pics/G111M8L1-1}
		%\end{minipage}
		%\item %1 s golovi
		%\begin{minipage}[t]{\bodywidth}
		%	На рисунке изображен график функции \(f(x) = \dfrac{ x^2 }{ a }+bx+c \), где числа \(a, b, c\) --- целые. Найдите значение \(f(3,5)\).
		%\end{minipage}
		%\hspace{0.02\linewidth}
		%\begin{minipage}[t]{\picwidth}
		%	\includegraphics[align=t, width=\linewidth]{../pics/G112M3C1-6}
		%\end{minipage}
		
		
		
		%EGE-1(90g) - 1
		\item В треугольнике \(ABC\) угол \(C\) равен \(90\degree\), \(AC=4,8\), \( \sin A = \dfrac{ 7 }{ 25 } \). Найдите \(AB\).
		%EGE-1(90g) - 4
		\item В треугольнике \(ABC\) угол \(C\) равен \(90\degree\), \(AC=4\), \( \tg A = \dfrac{ 33 }{ 4\sqrt{33} } \). Найдите \(AB\). 
		%n5 cos1
		\item Найдите корни уравнения: \( \cos \dfrac{ \pi(x-7) }{ 3 } = 0,5 \).  В ответ запишите наибольший отрицательный корень.
		%n5 cos3
		\item Найдите корни уравнения: \( \cos \dfrac{ \pi(2x+9) }{ 3 } = \dfrac{ \sqrt{2} }{ 2 } \).  В ответ запишите наибольший отрицательный корень.
		%n5 tg1
		\item Найдите корни уравнения: \( \tg \dfrac{ \pi x }{ 4 } = -1 \).  В ответ запишите наибольший отрицательный корень.
		%n5 tg2
		\item Найдите корни уравнения: \( \tg \dfrac{ \pi (x+2) }{ 3 } = -\sqrt{3} \).  В ответ запишите наибольший отрицательный корень.
		%n5 sin1
		\item Найдите корни уравнения: \( \sin \dfrac{ \pi x }{ 3 } = 0,5 \).  В ответ запишите наименьший положительный корень.
		
		%???chelnokovam5
		\item Вычислить: 
		\begin{tasks}(2)
			\task \( \dfrac{5\cos29\degree}{\sin61\degree} \)
			\task \( -4\sqrt{3}\cos(-750\degree) \)
			\task \( \dfrac{4\cos146\degree}{\cos34\degree} \)
			\task \( 7\tg13\degree\cdot\tg77\degree \)
			\task \( \dfrac{12}{\sin^227\degree+\cos^2207\degree} \)
			\task \( \dfrac{5\sin98\degree}{\sin49\degree\cdot\sin41\degree} \)
			\task \( -50\tg9\degree\cdot\tg81\degree+31 \)
		\end{tasks}
		%?
		\item
		\begin{minipage}[t]{\bodywidth}
			На рисунке изображён график функции \[ f(x)=a \cos{x}+b \] Найдите \(a\).
		\end{minipage}
		\hspace{0.02\linewidth}
		\begin{minipage}[t]{\picwidth}
			\includegraphics[align=t, width=\linewidth]{\picpath/MECGERM6H3-1}
		\end{minipage}
		%?
		\item
		\begin{minipage}[t]{\bodywidth}
			На рисунке изображён график функции \[ f(x)=a \tg{x}+b \] Найдите \(a\).
		\end{minipage}
		\hspace{0.02\linewidth}
		\begin{minipage}[t]{\picwidth}
			\includegraphics[align=t, width=\linewidth]{\picpath/MECGERM6H3-2}
		\end{minipage}
		%EGE-11 Trigon 1-3
		\item Найдите наибольшее значение функции \[ y=12\cos x + 6\sqrt{3}x - 2 \sqrt{3}x - 2 \sqrt{3} \pi + 6\] на отрезке \( \left[ 0; \dfrac{ \pi }{ 2 } \right]  \).
		\item Найдите наименьшее значение функции \( y=3+\dfrac{ 5\pi }{ 4 }-5x-5\sqrt{2} \cos x\) на отрезке \( \left[ 0; \dfrac{ \pi }{ 2 } \right]  \).
		\item Найдите наибольшее значение функции \( y=5 \cos x - 6x + 4 \) на отрезке \( \left[ -\dfrac{ 3\pi }{ 2 }; 0 \right]  \).
	\end{listofex}
\end{class}
%END_FOLD

%BEGIN_FOLD % ====>>_____ Занятие 2 _____<<====
\begin{class}[number=2]
	\begin{listofex}
		\item Вычислить с помощью метода приведения: %из 101L1
		\[ \cos 225 \degree;\;\sin 420 \degree;\;\sin 270 \degree;\;\sin (-300) ;\;\cos 210 \degree \]
		%;\;\sin\dfrac{13\pi}{4};\;\sin\left( -\dfrac{7\pi}{6}  \right);\;\cos\dfrac{21\pi}{4} 
		\item Вычислите: %из 101L1
		\begin{tasks}(2)
			\task \( \dfrac{\sqrt{3}}{\sin60\degree}+\dfrac{3}{\sin30\degree} \)
			\task \( \dfrac{17\sin155\degree}{\sin25\degree} \)
			\task \( \dfrac{-2\sin105\degree}{\cos15\degree} \)
			\task \( \dfrac{-13\sin126\degree}{\sin54\degree} \)
			\task \( \dfrac{ 8 }{ \sin \left( \tfrac{ -27\pi }{ 4 } \right) \cos \left( \tfrac{ 31\pi }{ 4 } \right) } \)
			\task \( \dfrac{ 60 }{ \sin \left( -\tfrac{ -32\pi }{ 3 } \right) \cos \left( \tfrac{ 25\pi }{ 6 } \right)} \)
			\task \( 4 \sqrt{2} \cos \dfrac{ \pi }{ 4 } \cos \dfrac{ 7\pi }{3  } \)
			\task \( \dfrac{ 5 \cos 29 \degree }{ \sin 61 \degree } \)
			\task \( 36 \sqrt{6} \tg \dfrac{ \pi }{ 6 } \sin \dfrac{ \pi }{ 4 } \)
			\task \( -4 \sqrt{3} \cos (-750 \degree) \)
			
			\task \( \sin^215\degree-1+\cos^215 \degree \)
			\task \( \cos^2\left( -\dfrac{ 3\pi }{ 8 } \right) +\sin^2\left( -\dfrac{ 3\pi }{ 8 } \right) \)
			\task \( \sin^2 (-88\degree) + \cos^2 (-88\degree) -5 \)
			
			\task \( \dfrac{2\sin^221\degree+2\cos^221\degree}{4} \)
			\task \( \dfrac{ 12\sin 11 \degree \cdot \cos 11 \degree }{ \sin 22 \degree } \)
			\task \( \dfrac{ 24 (\sin^2 17 \degree-\cos^217 \degree) }{ \cos 34 \degree } \)
			
			
			
			%\task \( -\sqrt{27}\cos30\degree-\sqrt{2}\sin45\degree\ctg60\degree\tg60\degree\)
		\end{tasks}
		\item Найдите: %101L1
		\begin{tasks}
			
			\task \( 5\sin\alpha \), если \( \cos\alpha=\dfrac{2\sqrt{6}}{5} \) и \( \alpha\in\left( \dfrac{3\pi}{2}; 2\pi \right) \);
			\task \( 3\cos\alpha \), если \( \sin\alpha=-\dfrac{2\sqrt{2}}{3} \) и \( \alpha\in\left( \dfrac{3\pi}{2}; 2\pi \right) \);
			\task \( 24\cos\alpha \), если \( \sin\alpha=-0,2 \);
			\task \( \sin\left( \dfrac{7\pi}{2}-\alpha \right) \), если \( \sin\alpha=0,8 \) и \( \alpha\in\left( \dfrac{\pi}{2}; \pi \right) \).
		\end{tasks}
		%priklad trigon 1
		\item При нормальном падении света с длиной волны \( \lambda=400 \) нм на дифракционную решeтку с периодом \(d\) нм наблюдают серию дифракционных максимумов. При этом угол \(\varphi\)  (отсчитываемый от перпендикуляра к решeтке), под которым наблюдается максимум, и номер максимума \(k\) связаны соотношением \(d \sin \varphi= k\lambda\). Под каким минимальным углом \(\varphi\) (в градусах) можно наблюдать второй максимум на решeтке с периодом, не превосходящим \(1600\) нм?
		%priklad trigon 2
		\item Два тела массой \(m=2\) кг каждое, движутся с одинаковой скоростью  \(v =10\) м/с под углом \(2\alpha\) друг к другу. Энергия (в джоулях), выделяющаяся при их абсолютно неупругом соударении определяется выражением \(Q= m v^2 \sin^2 \alpha \). Под каким наименьшим углом \(2\alpha\) (в градусах) должны двигаться тела, чтобы в результате соударения выделилось не менее \(50\) джоулей?
		\item Найдите площадь треугольника, две стороны которого равны \(8\) и \(12\), а угол между ними равен \(30 \degree\).
		\item Большее основание равнобедренной трапеции равно \(34\). Боковая сторона равна \(14\). Синус острого угла равен \( \dfrac{ 2\sqrt{10}}{ 7 } \). Найдите меньшее основание.
		\item Основания равнобедренной трапеции равны \(7\) и \(51\). Тангенс острого угла равен \( \dfrac{ 5 }{ 11 } \).  Найдите высоту трапеции.
		
	\end{listofex}
\end{class}
%END_FOLD

%BEGIN_FOLD % ====>>_ Домашняя работа 1 _<<====
\begin{homework}[number=1]
	\begin{listofex}
		%\item Решите уравнения: %(1) 1 2 // (2) 1 2
		%\begin{tasks}(2)
		%	\task \( (\tg^2x-1)\sqrt{13\cos x} = 0 \)
		%	\task \( (2\cos^2 x + \sin x - 2)\sqrt{5\tg x}=0 \)
		%	\task \( 2\cos \left( \dfrac{ \pi }{ 2 }-x \right)=\tg x \)
		%	\task \( \cos 2x + \sin^2x=0,75 \)
		%\end{tasks}
		\item Вычислите: %n5 10 11 12 13
		\begin{tasks}(2)
			\task \( 24 \sqrt{2} \cos \left( -\dfrac{ \pi }{ 3 } \right) \sin \left( -\dfrac{ \pi }{ 4 } \right)  \)
			\task \( -4 \sqrt{3} \cos \left( -\dfrac{ 3\pi }{ 2 } \right) \cos \left( \dfrac{ \pi}{4  } \right) \)
			\task \( \dfrac{ 14 \sin 19 \degree }{ \sin 341 \degree } \)
			\task \( \dfrac{ 4 \cos 146 \degree }{ \cos 34 \degree } \)
			\task \( \dfrac{ 5 \tg 163 \degree }{ \tg 17 \degree } \)
			\task \( \dfrac{14 \sin 409 \degree  }{ \sin 49 \degree } \)
		\end{tasks}
		\item Найдите: %6 1-6 9
		\begin{tasks}
			\task \( \tg \alpha \), если \( \cos \alpha = \dfrac{ \sqrt{10} }{ 10 } \) и \( \alpha\in\left( \dfrac{ 3\pi }{ 2 };2\pi \right) \);
			\task \( 26 \cos \left( \dfrac{ 3\pi }{ 2 }+\alpha \right) \), если \( \cos \alpha = \dfrac{ 12 }{ 13 } \) и \( \alpha\in\left( \dfrac{ 3\pi }{ 2 };2\pi \right) \);
			\task \( \dfrac{ 10\sin 6 \alpha }{ 3 \cos 3 \alpha } \), если \( \sin 3 \alpha = 0,6 \).
		\end{tasks}
		%priklad trigon 3
		\item Катер должен пересечь реку шириной \(L = 100\) м и со скоростью течения \(u =0,5\) м/с так, чтобы причалить точно напротив места отправления. Он может двигаться с разными скоростями, при этом время в пути, измеряемое в секундах, определяется выражением \(t = \dfrac{ L }{ u } \ctg \alpha \),  где \(\alpha\) --- острый угол, задающий направление его движения (отсчитывается от берега). Под каким минимальным углом \(\alpha\) (в градусах) нужно плыть, чтобы время в пути было не больше \(200\) с?
		\item В треугольнике \(ABC\): \(AC=BC, AB=8, \cos A = 0,5\). Найдите \(AC\).
		\item В треугольнике \(ABC\): \(AC=BC=7, \tg A = \dfrac{ 33 }{ 4\sqrt{3} }\). Найдите \(AB\).
	\end{listofex}
\end{homework}
%END_FOLD

%BEGIN_FOLD % ====>>_____ Занятие 3 _____<<====
\begin{class}[number=3]
	\begin{listofex}
		\item Занятие 3 
	\end{listofex}
\end{class}
%END_FOLD

%BEGIN_FOLD % ====>>_____ Занятие 4 _____<<====
\begin{class}[number=4]
	\begin{listofex}
		\item Занятие 4
	\end{listofex}
\end{class}
%END_FOLD

%BEGIN_FOLD % ====>>_ Домашняя работа 2 _<<====
\begin{homework}[number=2]
	\begin{listofex}
		\item Домашняя работа 2
	\end{listofex}
\end{homework}
%END_FOLD

%BEGIN_FOLD % ====>>_____ Занятие 5 _____<<====
\begin{class}[number=5]
	\begin{listofex}
		\item Занятие 5
	\end{listofex}
\end{class}
%END_FOLD

%BEGIN_FOLD % ====>>_____ Занятие 6 _____<<====
\begin{class}[number=6]
	\begin{listofex}
		\item Занятие 6
	\end{listofex}
\end{class}
%END_FOLD

%BEGIN_FOLD % ====>>_ Домашняя работа 3 _<<====
\begin{homework}[number=3]
	\begin{listofex}
		\item Домашняя работа 3
	\end{listofex}
\end{homework}
%END_FOLD

%BEGIN_FOLD % ====>>_____ Занятие 7 _____<<====
\begin{class}[number=7]
	\title{Подготовка к проверочной}
	\begin{listofex}
		\item Занятие 7
	\end{listofex}
\end{class}
%END_FOLD

=%BEGIN_FOLD % ====>>_ Проверочная работа _<<====
\begin{exam}
	\begin{listofex}
		\item Проверочная
	\end{listofex}
\end{exam}
%END_FOLD