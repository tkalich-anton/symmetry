%
%===============>>  ГРУППА 11-1 МОДУЛЬ 8  <<=============
%
\setmodule{8}

%BEGIN_FOLD % ====>>_____ Занятие 1 _____<<====
\begin{class}[number=1]
	\begin{listofex} %Взяты 222 a 2 4 5 6 c193 25 26 a
		%\item Решите уравнения: %c51 1 2 4 5 a
		%\begin{tasks}(2)
		%	\task \( \dfrac{ x^2-9 }{ \sqrt[]{-5x} }=0 \)
		%	\task \( (x^2-16)\sqrt[]{3-x}=0 \)
		%	\task \( \dfrac{ x^2-10x+9 }{ \sqrt[]{x}-3 }=0 \)
		%	\task \( \sqrt[]{2x^2+7}=x-4 \)
		%\end{tasks}
		%\item Решите неравенства: %c193 23 24 27 28 a
		%\begin{tasks}(2)
		%	\task \( \sqrt[4]{x^2-24x} \le 3 \)
		%	\task \( \sqrt[28]{8x-x^2-15} < 1 \)
		%	\task \( \sqrt[]{x^2-2x-15} < 3 \)
		%	\task \( \sqrt[]{3x^2-14x+51} \ge 6 \)
		%\end{tasks}
		%\item Решите неравенства:
		%\begin{tasks}(1)
		%	\task \( \sqrt[]{x^4-2x+6} \ge x \)
		%	\task \( \sqrt[]{5x^4-28x^2+16} \ge x^2+4 \)
		%	\task \( \sqrt[]{x^2-17x-29} \ge 3|x+2| \)
		%\end{tasks}
		\item Вычислить: %114m3
		\begin{tasks}(2)
			\task \( \dfrac{-13\sin126\degree}{\sin54\degree} \)
			\task \( \cos^2(-46\degree)+\sin^2(-46\degree) \)
			\task \( \sin^223\degree+9+\cos^223 \)
			\task \( \dfrac{2\sin^221\degree+2\cos^221\degree}{4} \)
		\end{tasks}
		%\item Вычислить: %114m3
		%\begin{tasks}(1)
		%	%\task \( \left( \dfrac{4\tg120\degree\cdot\cos210\degree-\sin270\degree}{2\cos240\degree-3\sqrt{3}\sin210\degree} \right)\cdot\dfrac{5}{3\sqrt{3}+2}-\dfrac{1}{23} \)\answer{ \( 3 \) }
		%	\task \( \dfrac{\sqrt{8}\sin\left( -\dfrac{\pi}{4} \right)+\sqrt{27}\cos\left( \dfrac{\pi}{3} \right)-4\sin\left( -\dfrac{\pi}{6} \right)}{6\sqrt{3}} \) \answer{ \( 0,25 \) }
		%	\task \( 4\cos\left( \dfrac{2\pi}{3} \right)-\left( \sqrt{3}+1 \right)\left( \ctg\left( \dfrac{7\pi}{6} \right)-1 \right) \) \answer{ \( -4 \) }
		%	\task \( \left( 4 - \sin\left( -\dfrac{10\pi}{3} \right) \right)^2+4\tg\left( \dfrac{\pi}{3} \right) \) \answer{ \( 16,75 \) }
		%\end{tasks}
		\item Найдите значение выражения: \( \sin\left( \dfrac{\pi}{3}+\alpha \right) \), если \( \cos\alpha=-\dfrac{8}{17} \) и \\ \( \pi<\alpha<\dfrac{3\pi}{2} \).
		%\item Найдите значение выражения: \( \dfrac{2\cos^2x-7\sin^2x}{3\cos^2x+4\sin x\cdot\cos x} \), если \( \ctg x = -2 \).
		
		%\item %1 параболы
		%\begin{minipage}[t]{\bodywidth}
		%	На рисунке изображены графики функций \(f(x) = 4x^2-25x+41 \) и \( g(x)=ax^2+bx+c \), которые пересекаются в точках \(A\) и \(B\). Найдите абсциссу точки \(B\).
		%\end{minipage}
		%\hspace{0.02\linewidth}
		%\begin{minipage}[t]{\picwidth}
		%	\includegraphics[align=t, width=\linewidth]{../pics/G112M3C1-10}
		%\end{minipage}
		%\item %1 LOGARIFM
		%\begin{minipage}[t]{\bodywidth}
		%	На рисунке изображен график функции \(f(x) = b+\log_ax \). Найдите \(f(32)\).
		%\end{minipage}
		%\hspace{0.02\linewidth}
		%\begin{minipage}[t]{\picwidth}
		%	\includegraphics[align=t, width=\linewidth]{../pics/G111M8L1-1}
		%\end{minipage}
		%\item %1 s golovi
		%\begin{minipage}[t]{\bodywidth}
		%	На рисунке изображен график функции \(f(x) = \dfrac{ x^2 }{ a }+bx+c \), где числа \(a, b, c\) --- целые. Найдите значение \(f(3,5)\).
		%\end{minipage}
		%\hspace{0.02\linewidth}
		%\begin{minipage}[t]{\picwidth}
		%	\includegraphics[align=t, width=\linewidth]{../pics/G112M3C1-6}
		%\end{minipage}
		
		
		
		%EGE-1(90g) - 1
		\item В треугольнике \(ABC\) угол \(C\) равен \(90\degree\), \(AC=4,8\), \( \sin A = \dfrac{ 7 }{ 25 } \). Найдите \(AB\).
		%EGE-1(90g) - 4
		\item В треугольнике \(ABC\) угол \(C\) равен \(90\degree\), \(AC=4\), \( \tg A = \dfrac{ 33 }{ 4\sqrt{33} } \). Найдите \(AB\). 
		%n5 cos1
		\item Найдите корни уравнения: \( \cos \dfrac{ \pi(x-7) }{ 3 } = 0,5 \).  В ответ запишите наибольший отрицательный корень.
		%n5 cos3
		\item Найдите корни уравнения: \( \cos \dfrac{ \pi(2x+9) }{ 3 } = \dfrac{ \sqrt{2} }{ 2 } \).  В ответ запишите наибольший отрицательный корень.
		%n5 tg1
		\item Найдите корни уравнения: \( \tg \dfrac{ \pi x }{ 4 } = -1 \).  В ответ запишите наибольший отрицательный корень.
		%n5 tg2
		\item Найдите корни уравнения: \( \tg \dfrac{ \pi (x+2) }{ 3 } = -\sqrt{3} \).  В ответ запишите наибольший отрицательный корень.
		%n5 sin1
		\item Найдите корни уравнения: \( \sin \dfrac{ \pi x }{ 3 } = 0,5 \).  В ответ запишите наименьший положительный корень.
		
		%???chelnokovam5
		\item Вычислить: 
		\begin{tasks}(2)
			\task \( \dfrac{5\cos29\degree}{\sin61\degree} \)
			\task \( -4\sqrt{3}\cos(-750\degree) \)
			\task \( \dfrac{4\cos146\degree}{\cos34\degree} \)
			\task \( 7\tg13\degree\cdot\tg77\degree \)
			\task \( \dfrac{12}{\sin^227\degree+\cos^2207\degree} \)
			\task \( \dfrac{5\sin98\degree}{\sin49\degree\cdot\sin41\degree} \)
			\task \( -50\tg9\degree\cdot\tg81\degree+31 \)
		\end{tasks}
		%?
		\item
		\begin{minipage}[t]{\bodywidth}
			На рисунке изображён график функции \[ f(x)=a \cos{x}+b \] Найдите \(a\).
		\end{minipage}
		\hspace{0.02\linewidth}
		\begin{minipage}[t]{\picwidth}
			\includegraphics[align=t, width=\linewidth]{\picpath/MECGERM6H3-1}
		\end{minipage}
		%?
		\item
		\begin{minipage}[t]{\bodywidth}
			На рисунке изображён график функции \[ f(x)=a \tg{x}+b \] Найдите \(a\).
		\end{minipage}
		\hspace{0.02\linewidth}
		\begin{minipage}[t]{\picwidth}
			\includegraphics[align=t, width=\linewidth]{\picpath/MECGERM6H3-2}
		\end{minipage}
		%EGE-11 Trigon 1-3
		\item Найдите наибольшее значение функции \[ y=12\cos x + 6\sqrt{3}x - 2 \sqrt{3}x - 2 \sqrt{3} \pi + 6\] на отрезке \( \left[ 0; \dfrac{ \pi }{ 2 } \right]  \).
		\item Найдите наименьшее значение функции \( y=3+\dfrac{ 5\pi }{ 4 }-5x-5\sqrt{2} \cos x\) на отрезке \( \left[ 0; \dfrac{ \pi }{ 2 } \right]  \).
		\item Найдите наибольшее значение функции \( y=5 \cos x - 6x + 4 \) на отрезке \( \left[ -\dfrac{ 3\pi }{ 2 }; 0 \right]  \).
	\end{listofex}
\end{class}
%END_FOLD

%BEGIN_FOLD % ====>>_____ Занятие 2 _____<<====
\begin{class}[number=2]
	\begin{listofex}
		\item Вычислить с помощью метода приведения: %из 101L1
		\[ \cos 225 \degree;\;\sin 420 \degree;\;\sin 270 \degree;\;\sin (-300) ;\;\cos 210 \degree \]
		%;\;\sin\dfrac{13\pi}{4};\;\sin\left( -\dfrac{7\pi}{6}  \right);\;\cos\dfrac{21\pi}{4} 
		\item Вычислите: %из 101L1
		\begin{tasks}(2)
			\task \( \dfrac{\sqrt{3}}{\sin60\degree}+\dfrac{3}{\sin30\degree} \)
			\task \( \dfrac{17\sin155\degree}{\sin25\degree} \)
			\task \( \dfrac{-2\sin105\degree}{\cos15\degree} \)
			\task \( \dfrac{-13\sin126\degree}{\sin54\degree} \)
			\task \( \dfrac{ 8 }{ \sin \left( \tfrac{ -27\pi }{ 4 } \right) \cos \left( \tfrac{ 31\pi }{ 4 } \right) } \)
			\task \( \dfrac{ 60 }{ \sin \left( -\tfrac{ -32\pi }{ 3 } \right) \cos \left( \tfrac{ 25\pi }{ 6 } \right)} \)
			\task \( 4 \sqrt{2} \cos \dfrac{ \pi }{ 4 } \cos \dfrac{ 7\pi }{3  } \)
			\task \( \dfrac{ 5 \cos 29 \degree }{ \sin 61 \degree } \)
			\task \( 36 \sqrt{6} \tg \dfrac{ \pi }{ 6 } \sin \dfrac{ \pi }{ 4 } \)
			\task \( -4 \sqrt{3} \cos (-750 \degree) \)
			
			\task \( \sin^215\degree-1+\cos^215 \degree \)
			\task \( \cos^2\left( -\dfrac{ 3\pi }{ 8 } \right) +\sin^2\left( -\dfrac{ 3\pi }{ 8 } \right) \)
			\task \( \sin^2 (-88\degree) + \cos^2 (-88\degree) -5 \)
			
			\task \( \dfrac{2\sin^221\degree+2\cos^221\degree}{4} \)
			\task \( \dfrac{ 12\sin 11 \degree \cdot \cos 11 \degree }{ \sin 22 \degree } \)
			\task \( \dfrac{ 24 (\sin^2 17 \degree-\cos^217 \degree) }{ \cos 34 \degree } \)
			
			
			
			%\task \( -\sqrt{27}\cos30\degree-\sqrt{2}\sin45\degree\ctg60\degree\tg60\degree\)
		\end{tasks}
		\item Найдите: %101L1
		\begin{tasks}
			
			\task \( 5\sin\alpha \), если \( \cos\alpha=\dfrac{2\sqrt{6}}{5} \) и \( \alpha\in\left( \dfrac{3\pi}{2}; 2\pi \right) \);
			\task \( 3\cos\alpha \), если \( \sin\alpha=-\dfrac{2\sqrt{2}}{3} \) и \( \alpha\in\left( \dfrac{3\pi}{2}; 2\pi \right) \);
			\task \( 24\cos\alpha \), если \( \sin\alpha=-0,2 \);
			\task \( \sin\left( \dfrac{7\pi}{2}-\alpha \right) \), если \( \sin\alpha=0,8 \) и \( \alpha\in\left( \dfrac{\pi}{2}; \pi \right) \).
		\end{tasks}
		%priklad trigon 1
		\item При нормальном падении света с длиной волны \( \lambda=400 \) нм на дифракционную решeтку с периодом \(d\) нм наблюдают серию дифракционных максимумов. При этом угол \(\varphi\)  (отсчитываемый от перпендикуляра к решeтке), под которым наблюдается максимум, и номер максимума \(k\) связаны соотношением \(d \sin \varphi= k\lambda\). Под каким минимальным углом \(\varphi\) (в градусах) можно наблюдать второй максимум на решeтке с периодом, не превосходящим \(1600\) нм?
		%priklad trigon 2
		\item Два тела массой \(m=2\) кг каждое, движутся с одинаковой скоростью  \(v =10\) м/с под углом \(2\alpha\) друг к другу. Энергия (в джоулях), выделяющаяся при их абсолютно неупругом соударении определяется выражением \(Q= m v^2 \sin^2 \alpha \). Под каким наименьшим углом \(2\alpha\) (в градусах) должны двигаться тела, чтобы в результате соударения выделилось не менее \(50\) джоулей?
		\item Найдите площадь треугольника, две стороны которого равны \(8\) и \(12\), а угол между ними равен \(30 \degree\).
		\item Большее основание равнобедренной трапеции равно \(34\). Боковая сторона равна \(14\). Синус острого угла равен \( \dfrac{ 2\sqrt{10}}{ 7 } \). Найдите меньшее основание.
		\item Основания равнобедренной трапеции равны \(7\) и \(51\). Тангенс острого угла равен \( \dfrac{ 5 }{ 11 } \).  Найдите высоту трапеции.
		
	\end{listofex}
\end{class}
%END_FOLD

%BEGIN_FOLD % ====>>_ Домашняя работа 1 _<<====
\begin{homework}[number=1]
	\begin{listofex}
		%\item Решите уравнения: %(1) 1 2 // (2) 1 2
		%\begin{tasks}(2)
		%	\task \( (\tg^2x-1)\sqrt{13\cos x} = 0 \)
		%	\task \( (2\cos^2 x + \sin x - 2)\sqrt{5\tg x}=0 \)
		%	\task \( 2\cos \left( \dfrac{ \pi }{ 2 }-x \right)=\tg x \)
		%	\task \( \cos 2x + \sin^2x=0,75 \)
		%\end{tasks}
		\item Вычислите: %n6 10 11 12 13 14
		\begin{tasks}(2)
			\task \( 24 \sqrt{2} \cos \left( -\dfrac{ \pi }{ 3 } \right) \sin \left( -\dfrac{ \pi }{ 4 } \right)  \)
			\task \( -4 \sqrt{3} \cos \left( -\dfrac{ 3\pi }{ 2 } \right) \cos \left( \dfrac{ \pi}{4  } \right) \)
			\task \( \dfrac{ 14 \sin 19 \degree }{ \sin 341 \degree } \)
			\task \( \dfrac{ 4 \cos 146 \degree }{ \cos 34 \degree } \)
			\task \( \dfrac{ 5 \tg 163 \degree }{ \tg 17 \degree } \)
			\task \( \dfrac{14 \sin 409 \degree  }{ \sin 49 \degree } \)
		\end{tasks}
		\item Найдите корень уравнения:
		\begin{tasks}(2)
			\task \( \sqrt{15-2x}=3 \)
			\task \( \sqrt{-72-17x}=-x \)
			\task \( \sqrt{\dfrac{ 6 }{ 4x-54 }}=\dfrac{ 1 }{ 7 } \)
			\task \( \sqrt{\dfrac{ 2x+5 }{ 3 }}=5 \)
		\end{tasks}
		
		
		\item В треугольнике \(ABC\): \(AC=BC, AB=8, \cos A = 0,5\). Найдите \(AC\).
		
	\end{listofex}
\end{homework}
%END_FOLD

%BEGIN_FOLD % ====>>_____ Занятие 3 _____<<====
\begin{class}[number=3]
	\begin{listofex}
		\item Вычислите:
		\begin{tasks}(2)
			\task \( \dfrac{ 5\sin 98 \degree }{ \sin 49\degree \cdot \sin 41\degree } \)
			\task \( \dfrac{ 5\sin 74 \degree }{ \cos 37 \degree \cdot \cos 53 \degree } \)
			\task \( \dfrac{ 23 }{ \sin^2 56 \degree + 1 + \sin^2 146 \degree } \)
			\task \( -\dfrac{ 4 }{ \sin^2 27 \degree + \sin^2 117 \degree } \)
			\task \( \dfrac{ 50 \sin 19 \degree \cos 19 \degree }{ \sin 38 \degree } \)
			\task \( -\dfrac{ 7 }{ \sin^2 \left( \dfrac{ \pi }{ 15 } \right) + \cos^2 \left( \dfrac{ \pi }{ 15 } \right) } \)
			\task \( 36 \sqrt{6} \tg \dfrac{ \pi }{ 6 } \sin \dfrac{ \pi }{ 4 } \)
			\task \( -4 \sqrt{3} \cos (-750 \degree) \)
			
			\task \( \sin^215\degree-1+\cos^215 \degree \)
			\task \( \cos^2\left( -\dfrac{ 3\pi }{ 8 } \right) +\sin^2\left( -\dfrac{ 3\pi }{ 8 } \right) \)
			\task \( \sin^2 (-88\degree) + \cos^2 (-88\degree) -5 \)
			
			\task \( \dfrac{2\sin^221\degree+2\cos^221\degree}{4} \)
			\task \( \dfrac{ 12\sin 11 \degree \cdot \cos 11 \degree }{ \sin 22 \degree } \)
			\task \( \dfrac{ 24 (\sin^2 17 \degree-\cos^217 \degree) }{ \cos 34 \degree } \)
			
		\end{tasks}
		\item Найдите: %101L1
		\begin{tasks}
			
			\task \( 5\sin\alpha \), если \( \cos\alpha=\dfrac{2\sqrt{6}}{5} \) и \( \alpha\in\left( \dfrac{3\pi}{2}; 2\pi \right) \);
			\task \( 3\cos\alpha \), если \( \sin\alpha=-\dfrac{2\sqrt{2}}{3} \) и \( \alpha\in\left( \dfrac{3\pi}{2}; 2\pi \right) \);
			\task \( 24\cos\alpha \), если \( \sin\alpha=-0,2 \);
			\task \( \sin\left( \dfrac{7\pi}{2}-\alpha \right) \), если \( \sin\alpha=0,8 \) и \( \alpha\in\left( \dfrac{\pi}{2}; \pi \right) \);
			\task \( \tg \alpha \), если \( \cos \alpha = \dfrac{ \sqrt{10} }{ 10 } \) и \( \alpha\in\left( \dfrac{ 3\pi }{ 2 };2\pi \right) \);
			\task \( 26 \cos \left( \dfrac{ 3\pi }{ 2 }+\alpha \right) \), если \( \cos \alpha = \dfrac{ 12 }{ 13 } \) и \( \alpha\in\left( \dfrac{ 3\pi }{ 2 };2\pi \right) \);
			\task \( \dfrac{ 10\sin 6 \alpha }{ 3 \cos 3 \alpha } \), если \( \sin 3 \alpha = 0,6 \).
		\end{tasks}
		%priklad trigon 1
		\item При нормальном падении света с длиной волны \( \lambda=400 \) нм на дифракционную решeтку с периодом \(d\) нм наблюдают серию дифракционных максимумов. При этом угол \(\varphi\)  (отсчитываемый от перпендикуляра к решeтке), под которым наблюдается максимум, и номер максимума \(k\) связаны соотношением \(d \sin \varphi= k\lambda\). Под каким минимальным углом \(\varphi\) (в градусах) можно наблюдать второй максимум на решeтке с периодом, не превосходящим \(1600\) нм?
		%priklad trigon 2
		\item Два тела массой \(m=2\) кг каждое, движутся с одинаковой скоростью  \(v =10\) м/с под углом \(2\alpha\) друг к другу. Энергия (в джоулях), выделяющаяся при их абсолютно неупругом соударении определяется выражением \(Q= m v^2 \sin^2 \alpha \). Под каким наименьшим углом \(2\alpha\) (в градусах) должны двигаться тела, чтобы в результате соударения выделилось не менее \(50\) джоулей?
		%priklad trigon 3
		\item Катер должен пересечь реку шириной \(L = 100\) м и со скоростью течения \(u =0,5\) м/с так, чтобы причалить точно напротив места отправления. Он может двигаться с разными скоростями, при этом время в пути, измеряемое в секундах, определяется выражением \(t = \dfrac{ L }{ u } \ctg \alpha \),  где \(\alpha\) --- острый угол, задающий направление его движения (отсчитывается от берега). Под каким минимальным углом \(\alpha\) (в градусах) нужно плыть, чтобы время в пути было не больше \(200\) с?
		\item Найдите площадь треугольника, две стороны которого равны \(8\) и \(12\), а угол между ними равен \(30 \degree\).
		\item Большее основание равнобедренной трапеции равно \(34\). Боковая сторона равна \(14\). Синус острого угла равен \( \dfrac{ 2\sqrt{10}}{ 7 } \). Найдите меньшее основание.
		\item Основания равнобедренной трапеции равны \(7\) и \(51\). Тангенс острого угла равен \( \dfrac{ 5 }{ 11 } \).  Найдите высоту трапеции.
		\item В треугольнике \(ABC\): \(AC=BC=7, \tg A = \dfrac{ 33 }{ 4\sqrt{3} }\). Найдите \(AB\).
	\end{listofex}
\end{class}
%END_FOLD

%BEGIN_FOLD % ====>>_____ Занятие 4 _____<<====
\begin{class}[number=4]
	\begin{listofex}
		\item Найдите:
		\begin{tasks}(2)
			\task \( \dfrac{ \cos 100 \degree }{ 5\sin 50 \degree \cos 50 \degree } \)
			\task \( -\dfrac{ 4\cos 22,5 \degree \sin 22,5 \degree }{ 8 } \)
			\task \( \dfrac{ 50 \sin 19 \degree \cos 19 \degree }{ \sin 38 \degree } \)
			\task \( \dfrac{ \sin \dfrac{ 3\pi }{8 } \cos \dfrac{ 3\pi }{ 8 } }{ 20 } \)
			\task \( \sqrt{50} \cos^2 \dfrac{ 9\pi }{ 8 } - \sqrt{50} \sin^2 \dfrac{ 9\pi }{ 8 } \)
			\task \( 4\sqrt{2} \cos^2 \dfrac{ 15\pi }{ 8 } -2\sqrt{2} \)
		\end{tasks}
		\item Найдите: %последние 5
		\begin{tasks}(1)
			
			\task \(\sin 2 \alpha \), если \( \cos \alpha = 0,6 \) и \( \pi < \alpha < 2 \pi \);
			\task \( \cos \alpha \), если \( \sin \alpha = \dfrac{ 2 \sqrt{6} }{ 5 } \) и \( \alpha \in \left( \dfrac{ \pi }{ 2 }; \pi \right) \);
			\task \( 2\cos 2 \alpha \), если \( \sin \alpha=-0,7 \);
			
			
		\end{tasks}
		
		%priklad trigon 4
		\item Скейтбордист прыгает на стоящую на рельсах платформу, со скоростью  \(v = 3\) м/с под острым углом \(\alpha\)  к рельсам. От толчка платформа начинает ехать со скоростью \(u = \dfrac{ m }{ m+M }v \cos \alpha\) (м/с), где \(m = 80\) кг --- масса скейтбордиста со скейтом, а \(M = 400\) кг --- масса платформы. Под каким максимальным углом \(\alpha\) (в градусах) нужно прыгать, чтобы разогнать платформу не менее чем до \(0,25\) м/с?
		%priklad trigon 5
		\item Груз массой \(0,08\) кг колеблется на пружине. Его скорость υ меняется по закону \( v = v_0 \sin \dfrac{ 2\pi t }{ T } \), где \(t\) --- время с момента начала колебаний, \(T  =  12\) с --- период колебаний, \(v _0=0,5\) м/с. Кинетическая энергия\(E\) (в джоулях) груза вычисляется по формуле \( E = \dfrac{ m v^2 }{ 2 } \),  где \(m\) --- масса груза в килограммах, \(v\) --- скорость груза в м/с. Найдите кинетическую энергию груза через \(1\) секунду после начала колебаний. Ответ дайте в джоулях.
		\item В параллелограмме \(ABCD\) \(AB  = 6, AD  =  42\), \(\sin A= \dfrac{ 6 }{ 7 } \).  Найдите большую высоту параллелограмма.
		\item Найдите площадь ромба, если его высота равна \(2\), а острый угол \(30 \degree \).
		\item Основания равнобедренной трапеции равны \(17\) и \(87\). Высота трапеции равна \(14\). Найдите тангенс острого угла.
		\item Найдите площадь прямоугольной трапеции, основания которой равны \(6\) и \(2\), большая боковая сторона составляет с основанием угол \(45\degree \).
		\item Основания прямоугольной трапеции равны \(12\) и \(4\). Ее площадь равна \(64\). Найдите острый угол этой трапеции. Ответ дайте в градусах.
		\item В треугольнике \(ABC\) \( \angle C = 90 \degree, \tg A = \dfrac{  1}{ 5 } \).  Найдите \(AH\).
		\item В треугольнике \(ABC\) \(\angle C = 90 \degree\). \(CH\) --- высота, \(AB=13, \tg A = 5\). Найдите \(BH\).
	\end{listofex}
\end{class}
%END_FOLD

%BEGIN_FOLD % ====>>_ Домашняя работа 2 _<<====
\begin{homework}[number=2]
	\begin{listofex}
		\item Вычислите:
		\begin{tasks}(2)
			\task \( \dfrac{ 11\cos 150 \degree }{ 3\cos 75 \degree \sin 75 \degree } \)
			\task \( - \cos \left( \dfrac{ 3\pi }{ 8 }  \right) \sin \left( \dfrac{ 3\pi }{ 8 } \right) \)
			\task \( -\sqrt{2} \sin^2 \dfrac{ 2\pi }{ 7 } - \sqrt{2} \cos^2 \dfrac{ 2\pi }{ 7 }  \)
			\task \( 23 \cdot(-5 \sin 54 \degree - 5 \cos 54 \degree) \)
		\end{tasks}
		\item Найдите: %8-10
		\begin{tasks}
			\task \( \sin \left( \dfrac{ 7\pi }{2  }-\alpha \right) \), если \( \sin \alpha = 0,8 \) и \( \alpha \in \left( \dfrac{ \pi }{ 2 }; \pi \right) \);
			\task \( 26 \cos \left( \dfrac{ 3\pi }{ 2 }+\alpha \right) \), если \( \cos \alpha = \dfrac{ 12 }{ 13 } \) и \( \alpha \in \left( \dfrac{ 3\pi }{ 2 }; 2 \pi \right) \).
			%\task \( \tg \left( \alpha+\dfrac{ 5\pi }{ 2 } \right) \), если \( \tg \alpha = 0,4 \).
		\end{tasks}
		%priklad trigon 7
		\item Скорость колеблющегося на пружине груза меняется по закону \(v(5) = 5 \sin \pi t\) (см/с), где \(t\) --- время в секундах. Какую долю времени из первой секунды скорость движения была не менее \(2,5\) см/с? Ответ выразите десятичной дробью, если нужно, округлите до сотых.

	\end{listofex}
\end{homework}
%END_FOLD

%BEGIN_FOLD % ====>>_____ Занятие 5 _____<<====
\begin{class}[number=5]
	\begin{listofex}
		%\item
		%\begin{minipage}[t]{\bodywidth}
		%	На рисунке изображен график функции \( f(x)=kx+b \). Найдите \( f(-9) \).
		%\end{minipage}
		%\begin{minipage}[t]{\picwidth}
		%	\includegraphics[align=t, width=0.8\textwidth]{\picpath/G112M3C2-1}
		%\end{minipage}
		\item 
		\begin{minipage}[t]{\bodywidth}
				На рисунке изображён график функции \(f(x)=kx+b\). Найдите \(f(-5)\).
			\end{minipage}
		\begin{minipage}[t]{\picwidth}
				\includegraphics[align=t, width=0.8\textwidth]{\picpath/G101M4H2-1.jpg}
			\end{minipage}
		\item
		\begin{minipage}[t]{\bodywidth}
			На рисунке изображены графики двух линейных функций. Найдите абсциссу точки пересечения графиков.
		\end{minipage}
		\begin{minipage}[t]{\picwidth}
			\includegraphics[align=t, width=0.8\textwidth]{\picpath/G112M3C2-2}
		\end{minipage}
		\item
	\begin{minipage}[t]{\bodywidth}
			На рисунке изображён график функции вида \(f(x)=ax^2+bx+c\), где числа \(a, b, c\) --- целые. Найдите значение \(f(-3)\).
		\end{minipage}
	\begin{minipage}[t]{\picwidth}
			\includegraphics[align=t, width=\textwidth]{\picpath/G101M4C4-3.jpg}
		\end{minipage}
		
		\item
		\begin{minipage}[t]{\bodywidth}
			На рисунке изображен график функции \( f(x)=\dfrac{x^2}{a}+bx+c \). Найдите \( f(3,5) \).
		\end{minipage}
		\begin{minipage}[t]{\picwidth}
			\includegraphics[align=t, width=\textwidth]{\picpath/G112M3C2-3}
		\end{minipage}
		
		\item
	\begin{minipage}[t]{\bodywidth}
			На рисунке изображён график функции вида \(f(x)=ax^2+bx+c\), где числа \(a, b, c\) --- целые. Найдите значение дискриминанта уравнения \(f(x)=0\).
		\end{minipage}
	\begin{minipage}[t]{\picwidth}
			\includegraphics[align=t, width=\textwidth]{\picpath/G101M4C4-5.jpg}
		\end{minipage}
		
		\item
		\begin{minipage}[t]{\bodywidth}
			На рисунке изображены графики функций \( f(x)=2x^2+11x+11 \) и \( y=ax^2+bx+c \), которые пересекаются в точках \( A \) и \( B \). Найдите абсциссу точки \( B \).
		\end{minipage}
		\begin{minipage}[t]{\picwidth}
			\includegraphics[align=t, width=\textwidth]{\picpath/G112M3C2-6}
		\end{minipage}
			\item
		\begin{minipage}[t]{\bodywidth}
			На рисунке изображен график функции \( f(x)=\dfrac{k}{x}+a \). Найдите, при каком значении \( x \) значение функции будет равно \( 0,8 \).
		\end{minipage}
		\begin{minipage}[t]{\picwidth}
			\includegraphics[align=t, width=\textwidth]{\picpath/G112M3C2-5}
		\end{minipage}
		\item
		\begin{minipage}[t]{\bodywidth}
			На рисунке изображён график функции вида \[ f(x)=\dfrac{a}{x+b}+c, \] где числа \(a, b, c\) --- целые. Найдите значение \(x\), при котором \(f(x)=3\).
		\end{minipage}
		\hspace{0.02\linewidth}
		\begin{minipage}[t]{\picwidth}
			\includegraphics[align=t, width=\linewidth]{\picpath/G101M4C6-2}
		\end{minipage}
		\newpage
		%analog 26784 1-2 V 62785 1-2
		\item Найдите:
		\begin{tasks}
			\task \( 8 \sin \left( \dfrac{ \pi }{ 2 }- \alpha \right) \), если \( \sin \alpha = -0,6 \) и \( \alpha \in (1,5\pi;2\pi) \);
			\task \( -2 \sin \left( \dfrac{ \pi }{ 2 }+\alpha \right) \), если \( \sin \alpha = -0,96 \) и \( \alpha \in (\pi;1,5\pi) \);
			\task \( -20 \cos \left( \dfrac{ 5\pi }{ 2 }+ \alpha \right) \), если \( \cos \alpha = \dfrac{ 7 }{ 25 } \) и \( \alpha \in (1,5\pi; 2\pi) \);
			\task \( 39 \cos \left( \dfrac{ 7\pi }{ 2 } + \alpha \right) \), если \( \cos \alpha = -\dfrac{ 7\pi }{ 2 } + \alpha \) и \( \alpha \in (0,5\pi;\pi) \).
		\end{tasks}
		\item Вычислите:
		\begin{tasks}(2)
			\task \( \dfrac{ 36 \sin 102 \degree \cdot \cos 102 \degree }{ \sin 204 \degree } \)
			\task \( \dfrac{ 50 \sin 38 \degree }{ \sin 19 \degree \cdot \cos 19 \degree } \)
			\task \( \dfrac{ -10 \sin 97 \degree \cdot \cos 97 \degree }{ \sin 194 \degree } \)
			\task \( \dfrac{ -16 \sin 112 \degree \cdot \cos 112 \degree }{ 5 \sin 224 \degree } \)
			\task \( \dfrac{22 (\sin^2 72 \degree - \cos^2 72 \degree)}{ \cos 144 \degree } \)
			\task \( \dfrac{ 22 \cos 18 \degree }{ \sin^2 9 \degree - \cos^2 9 \degree  } \)
			\task \( \dfrac{ 18(-\sin^2 24 \degree + \cos^2 24 \degree) }{ \cos 48 \degree } \)
			\task \( \dfrac{ 7(\sin^2 11 \degree - \cos^2 11 \degree) }{ -\cos 22 \degree } \)
		\end{tasks}
	\end{listofex}
\end{class}
%END_FOLD

%BEGIN_FOLD % ====>>_____ Занятие 6 _____<<====
\begin{class}[number=6]
	\begin{listofex}
		\item
		\begin{minipage}[t]{\bodywidth}
			На рисунке изображен график функции \( f(x)=\dfrac{x^2}{a}+bx+c \). Найдите \( f(4) \).
		\end{minipage}
		\begin{minipage}[t]{\picwidth}
			\includegraphics[align=t, width=\textwidth]{\picpath/G112M3C2-4}
		\end{minipage}
		\item
		\begin{minipage}[t]{\bodywidth}
			На рисунке изображён график функции вида \(f(x)= \log_ax\). Найдите значение \(f(8)\).
		\end{minipage}
		\begin{minipage}[t]{\picwidth}
			\includegraphics[align=t, width=\textwidth]{\picpath/G111M8L6-1}
		\end{minipage}
		\item
		\begin{minipage}[t]{\bodywidth}
			На рисунке изображён график функции вида \(f(x)= \log_a x\). Найдите значение \(f(8)\).
		\end{minipage}
		\begin{minipage}[t]{\picwidth}
			\includegraphics[align=t, width=\textwidth]{\picpath/G111M8L6-4}
		\end{minipage}
		\item
		\begin{minipage}[t]{\bodywidth}
			На рисунке изображён график функции вида \(f(x)= \log_ax\). Найдите значение \(f(2)\).
		\end{minipage}
		\begin{minipage}[t]{\picwidth}
			\includegraphics[align=t, width=\textwidth]{\picpath/G111M8L6-8}
		\end{minipage}
		\item
		\begin{minipage}[t]{\bodywidth}
			На рисунке изображён график функции вида \(f(x)= a^x\). Найдите значение \(f(3)\).
		\end{minipage}
		\begin{minipage}[t]{\picwidth}
			\includegraphics[align=t, width=\textwidth]{\picpath/G111M8L6-2}
		\end{minipage}
		\item
		\begin{minipage}[t]{\bodywidth}
			На рисунке изображён график функции вида \(f(x)= a^x\). Найдите значение \(f(-3)\).
		\end{minipage}
		\begin{minipage}[t]{\picwidth}
			\includegraphics[align=t, width=\textwidth]{\picpath/G111M8L6-3}
		\end{minipage}
		
		\item
		\begin{minipage}[t]{\bodywidth}
			На рисунке изображён график функции вида \(f(x)= a^x\). Найдите значение \(f(4)\).
		\end{minipage}
		\begin{minipage}[t]{\picwidth}
			\includegraphics[align=t, width=\textwidth]{\picpath/G111M8L6-5}
		\end{minipage}
		
		\item
		\begin{minipage}[t]{\bodywidth}
			На рисунке изображён график функции вида \(f(x)= a^x\). Найдите значение \(f(2)\).
		\end{minipage}
		\begin{minipage}[t]{\picwidth}
			\includegraphics[align=t, width=\textwidth]{\picpath/G111M8L6-7}
		\end{minipage}
		
		\item
		\begin{minipage}[t]{\bodywidth}
			На рисунке изображены графики функций видов \(f(x)= a\sqrt{x} \) и  \(g(x)=kx\), пересекающиеся в точках \(A\) и \(B\). Найдите абсциссу точки \(B\).
		\end{minipage}
		\begin{minipage}[t]{\picwidth}
			\includegraphics[align=t, width=\textwidth]{\picpath/G111M8L6-6}
		\end{minipage}
		%priklad trigon 1
		\item При адиабатическом процессе для идеального газа выполняется закон \(pV^k=10^5\)Па\(\cdot\)м\(^5\), где \(p\) --- давление газа в паскалях, \(V\) --- объeм газа в кубических метрах, \(k=\dfrac{  5}{ 3 }\).  Найдите, какой объём \(V\) (в куб. м) будет занимать газ при давлении \(p\), равном \(3,2\cdot10^6\) Па.
		%priklad trigon 2
		\item В ходе распада радиоактивного изотопа его масса уменьшается по закону \( m(t)=m_02^{-\tfrac{ t }{ T }} \), где \(m_0\) --- начальная масса изотопа, \(t\) --- время, прошедшее от начального момента, \(T\) --- период полураспада. В начальный момент времени масса изотопа \(40\) мг. Период его полураспада составляет \(10\) мин. Найдите, через сколько минут масса изотопа будет равна \(5\) мг.
	\end{listofex}
\end{class}
%END_FOLD

%BEGIN_FOLD % ====>>_ Домашняя работа 3 _<<====
\begin{homework}[number=3]
	\begin{listofex}
		\item Вычислите: %первые 4
		\begin{tasks}(2)
			\task \( 6^{0,36}\cdot 36^{0,32} \)
			\task \( 7^{\tfrac{ 4 }{ 9 }} \cdot 49^{\tfrac{ 5 }{ 18 }} \)
			\task \( \dfrac{ 5^{6,5} }{ 25^{2,25} } \)
			\task \( \dfrac{  2^{3,5}\cdot 3^{5,5}}{ 6^{4,5} } \)
		\end{tasks}
		\item Найдите значение выражений: %первые 4
		\begin{tasks}(2)
			\task \( \dfrac{ 7(m^5)^6+11(m^3)^{10} }{ (3m^{15})^2 } \)
			\task \( \dfrac{ (3x)^3\cdot x^{-9} }{ x^{-10}\cdot2x^4 } \)
			\task \( \dfrac{ a^2b^{-6} }{ (4a)^3b^{-2} }\cdot \dfrac{ 16 }{ a^{-1} b^{-4}} \)
			\task \( ((2x^3)^4-(x^2)^6):(3x^{12}) \)
		\end{tasks}
		\item
		\begin{minipage}[t]{\bodywidth}
			На рисунке изображены графики функций \(f(x)=a\sqrt{x} \) и \(g(x)=kx+b\), которые пересекаются в точке \(A\). Найдите абсциссу точки \(A\).
		\end{minipage}
		\begin{minipage}[t]{\picwidth}
			\includegraphics[align=t, width=\textwidth]{\picpath/G111M8H3-3}
		\end{minipage}
		\item
		\begin{minipage}[t]{\bodywidth}
			На рисунке изображены графики функций \(f(x)=-3x+13\) и \(g(x)=ax^2+bx+c\), которые пересекаются в точках \(A\) и \(B\). Найдите абсциссу точки \(B\).
		\end{minipage}
		\begin{minipage}[t]{\picwidth}
			\includegraphics[align=t, width=\textwidth]{\picpath/G111M8H3-2}
		\end{minipage}
		\item
		\begin{minipage}[t]{\bodywidth}
			На рисунке изображён график функции \(f(x)=\log_{a}(x+b)\). Найдите \(f(29)\).
		\end{minipage}
		\begin{minipage}[t]{\picwidth}
			\includegraphics[align=t, width=\textwidth]{\picpath/G111M8H3-4}
		\end{minipage}
		\item
		\begin{minipage}[t]{\bodywidth}
			На рисунке изображён график функции \(f(x)=\dfrac{ kx+a }{ x+b }\). Найдите \(a\).
		\end{minipage}
		\begin{minipage}[t]{\picwidth}
			\includegraphics[align=t, width=\textwidth]{\picpath/G111M8H3-1}
		\end{minipage}
	\end{listofex}
\end{homework}
%END_FOLD

%BEGIN_FOLD % ====>>_____ Занятие 7 _____<<====
\begin{class}[number=7]
	\title{Подготовка к проверочной}
	\begin{listofex}
		\item Занятие 7
	\end{listofex}
\end{class}
%END_FOLD

=%BEGIN_FOLD % ====>>_ Проверочная работа _<<====
\begin{exam}
	\begin{listofex}
		\item Проверочная
	\end{listofex}
\end{exam}
%END_FOLD