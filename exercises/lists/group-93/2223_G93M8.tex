%
%===============>>  ГРУППА 9-2 МОДУЛЬ 8  <<=============
%
\setmodule{8}

%BEGIN_FOLD % ====>>_____ Занятие 1 _____<<====
\begin{class}[number=1]
	\begin{listofex}
		\item Переведите в десятичную дробь:
		\begin{tasks}(4)
			\task \( \dfrac{1}{4} \)
			\task \( \dfrac{2}{5} \)
			\task \( \dfrac{3}{8} \)
			\task \( \dfrac{8}{25} \)
			\task \( \dfrac{15}{10} \)
			\task \( \dfrac{1}{100} \)
			\task \( \dfrac{3}{200} \)
			\task \( \dfrac{7}{500} \)
		\end{tasks}
		\item Вычислите:
		\begin{tasks}(4)
			\task \( \dfrac{4}{25}+\dfrac{15}{4} \)
			\task \( \dfrac{9}{4}+\dfrac{8}{5} \)
			\task \( \dfrac{19}{2}-\dfrac{7}{25} \)
			\task \( \dfrac{3}{2}-\dfrac{9}{5} \)
			\task \( \dfrac{9}{5}\cdot\dfrac{2}{3} \)
			\task \( \dfrac{21}{5}\cdot\dfrac{3}{7} \)
			\task \( \dfrac{14}{5}:\dfrac{7}{2} \)
			\task \( \dfrac{3}{5}:\dfrac{4}{35} \)
		\end{tasks}
		\item Вычислите:
		\begin{tasks}(2)
			\task \( \left( \dfrac{19}{8}+\dfrac{11}{12} \right):\dfrac{5}{48} \)
			\task \( \left( \dfrac{14}{11}+\dfrac{17}{10} \right)\cdot\dfrac{11}{15} \)
			\task \( \left( \mfrac{2}{3}{4}+\mfrac{2}{1}{5} \right)\cdot16 \)
			\task \( \mfrac{1}{8}{17}:\left( \dfrac{12}{17}+\mfrac{2}{7}{11} \right) \)
		\end{tasks}
		\item Какое из данных ниже чисел принадлежит промежутку \( [3;4] \)?
		\begin{tasks}(4)
			\task \( \dfrac{45}{19} \)
			\task \( \dfrac{52}{19} \)
			\task \( \dfrac{68}{19} \)
			\task \( \dfrac{77}{19} \)
		\end{tasks}
		\item Какому из данных промежутков принадлежит число \( \dfrac{5}{9} \)?
		\begin{tasks}(4)
			\task \( [0,5; \, 0,6] \)
			\task \( [0,6 \, 0,7] \)
			\task \( [0,7 \, 0,8] \)
			\task \( [0,8 \, 0,9] \)
		\end{tasks}
		\item Вычислите:
		\begin{tasks}(4)
			\task \( \dfrac{1}{4}+0,7 \)
			\task \( \dfrac{1}{2}+0,07 \)
			\task \( \dfrac{24}{3,2\cdot2} \)
			\task \( \dfrac{27}{3\cdot4,5} \)
			\task \( \dfrac{11}{4,4\cdot2,5} \)
			\task \( \dfrac{7}{12,5\cdot1,4} \)
			\task \( \dfrac{0,9}{1+\frac{1}{8}} \)
			\task \( \dfrac{0,3}{1+\frac{1}{9}} \)
		\end{tasks}
	\end{listofex}
\end{class}
%END_FOLD

%BEGIN_FOLD % ====>>_____ Занятие 2 _____<<====
\begin{class}[number=2]
	\begin{listofex}
		\item Вычислите:
		\begin{tasks}(3)
			\task \( \dfrac{6,9-1,5}{2,4} \)
			\task \( \dfrac{2,4}{2,9-1,4} \)
			\task \( \dfrac{9,4}{4,1+5,3} \)
			\task \( \dfrac{6,9+4,1}{0,2} \)
			\task \( \dfrac{4,8\cdot0,4}{0,6} \)
			\task \( \dfrac{21}{0,3\cdot2,8} \)
			\task \( 0,6\cdot(-10)^3+50 \)
			\task \( -0,2\cdot(-10)^2+55 \)
			\task \( 30-0,8\cdot(-10)^2 \)
			\task \( 5,4\cdot0,8+0,08 \)
			\task \( 0,03\cdot0,3\cdot30000 \)
			\task \( 0,007\cdot7\cdot700 \)
		\end{tasks}
		\item Какому из промежутков принадлежит число \( \dfrac{7}{11} \)?
		\begin{tasks}(4)
			\task \( [0,4; \, 0,5] \)
			\task \( [0,5; \, 0,6] \)
			\task \( [0,6; \, 0,7] \)
			\task \( [0,7; \, 0,8] \)
		\end{tasks}
		\item Известно, что \( 0<a<1 \). Выберите наименьшее число
		\begin{tasks}(4)
			\task \( a^2 \)
			\task \( a^3 \)
			\task \( -a \)
			\task \( \dfrac{1}{a} \)
		\end{tasks}
		\item Известно, что \( a<b<0 \). Выберите наименьшее из чисел.
		\begin{tasks}(4)
			\task \( a-1 \)
			\task \( b-1 \)
			\task \( ab \)
			\task \( -b \)
		\end{tasks}
		\item Известно, что \( a \) и \( b \) --- положительные числа и \( a>b \). Сравните \( \dfrac{1}{a} \) и \( \dfrac{1}{b} \).
		\begin{tasks}(4)
			\task \( \dfrac{1}{a}>\dfrac{1}{b} \)
			\task \( \dfrac{1}{a}<\dfrac{1}{b} \)
			\task \( \dfrac{1}{a}=\dfrac{1}{b} \)
			\task Сравнить невозможно
		\end{tasks}
		\item Известно, что \( a>b>c \). Какое из следующих чисел отрицательно?
		\begin{tasks}(4)
			\task \( a-b \)
			\task \( a-c \)
			\task \( b-c \)
			\task \( c-b \)
		\end{tasks}
		\item Какое из следующих чисел заключено между числами \( \dfrac{1}{6} \) и \( \dfrac{1}{4} \)?
		\begin{tasks}(4)
			\task \( 0,1 \)
			\task \( 0,2 \)
			\task \( 0,3 \)
			\task \( 0,4 \)
		\end{tasks}
		\item Какое из приведенных ниже неравенств является верным при любых значениях \( a \) и \( b \), удовлетворяющих условию \( a>b \)?
		\begin{tasks}(4)
			\task \( b-a<-2 \)
			\task \( a-b>-1 \)
			\task \( a-b<3 \)
			\task \( b-a>-3 \)
		\end{tasks}
	\newpage
		\item Значение какого из данных выражений положительно, если известно, что \( x>0 \), \( y<0 \)?
		\begin{tasks}(4)
			\task \( xy \)
			\task \( (x-y)y \)
			\task \( (y-x)y \)
			\task \( (y-x)x \)
		\end{tasks}
		\item Вычислите:
		\begin{tasks}(2)
			\task \( \left( \dfrac{17}{8}-\dfrac{11}{20} \right):\dfrac{5}{46} \)
			\task \( \mfrac{4}{3}{4}:\left( \mfrac{1}{1}{15}+\dfrac{3}{5} \right) \)
			\task \( \mfrac{1}{1}{12}:\left( \mfrac{1}{13}{18}-\mfrac{2}{5}{9} \right) \)
			\task \( \left( \mfrac{1}{11}{16}-\mfrac{3}{7}{8} \right)\cdot4 \)
			\task \( \left( \dfrac{5}{6}+\mfrac{1}{1}{10} \right)\cdot24 \)
			\task \( \dfrac{1}{\frac{1}{22}+\frac{1}{18}} \)
			\task \( \dfrac{1}{\frac{1}{36}-\frac{1}{44}} \)
			\task \( \dfrac{1}{\frac{1}{33}+\frac{1}{12}} \)
		\end{tasks}
	\end{listofex}
\end{class}
%END_FOLD

%BEGIN_FOLD % ====>>_ Домашняя работа 1 _<<====
\begin{homework}[number=1]
	\begin{listofex}
		\item Вычислите:
		\begin{tasks}(3)
			\task \( \left( \dfrac{10}{13}+\dfrac{15}{4} \right)\cdot\dfrac{26}{5} \)
			\task \( \mfrac{2}{2}{5}:\left( \dfrac{9}{10}-\mfrac{1}{5}{14} \right) \)
			\task \( \dfrac{1}{\frac{1}{72}-\frac{1}{99}} \)
			\task \( \dfrac{11}{4,4\cdot2,5} \)
			\task \( \dfrac{2,7}{1,7+0,1} \)
			\task \( \dfrac{0,3\cdot4,4}{0,8} \)
		\end{tasks}
		\item Вычислите:
		\begin{tasks}(2)
			\task \( -2,54+6,6\cdot4,1 \)
			\task \( 80+0,9\cdot(-10)^3 \)
			\task \( 0,7\cdot(-10)^3-20 \)
			\task \( 45+0,6\cdot(-10)^2 \)
			\task \( -0,7\cdot(-10)^2+90 \)
			\task \( 400\cdot0,004\cdot40 \)
		\end{tasks}
		\item О числах \( a \), \( b \), \( c \) и \( d \) известно, что \( a>b \), \( b<c \), \( d=c \). Сравните числа \( d \) и \( a \).
		\begin{tasks}(4)
			\task \( d=a \)
			\task \( d>a \)
			\task \( d<a \)
			\task Сравнить невозможно
		\end{tasks}
		\item Какому из данных промежутков принадлежит число \( \dfrac{5}{11} \)?
		\begin{tasks}(1)
			\task \( [0,2; \, 0,3] \)
			\task \( [0,3; \, 0,4] \)
			\task \( [0,4; \, 0,5] \)
			\task \( [0,5; \, 0,6] \)
		\end{tasks}
		\item Какое из чисел принадлежит отрезку \( [8; \, 9] \)?
		\begin{tasks}(4)
			\task \( \dfrac{46}{7} \)
			\task \( \dfrac{53}{7} \)
			\task \( \dfrac{55}{7} \)
			\task \( \dfrac{61}{7} \)
		\end{tasks}
	\end{listofex}
\end{homework}
%END_FOLD

%BEGIN_FOLD % ====>>_____ Занятие 3 _____<<====
\begin{class}[number=3]
	\begin{listofex}
		\item Переведите в десятичную дробь:
		\begin{tasks}(4)
			\task \( \dfrac{5}{10} \)
			\task \( \dfrac{41}{100} \)
			\task \( \dfrac{89}{1000} \)
			\task \( \dfrac{555}{100} \)
			\task \( \dfrac{1}{20} \)
			\task \( \dfrac{3}{25} \)
			\task \( \dfrac{9}{8} \)
			\task \( \dfrac{6}{125} \)
		\end{tasks}
		\item Выполните умножение:
		\begin{tasks}(4)
			\task \( 0,6\cdot0,48 \)
			\task \( 1,5\cdot8,99 \)
			\task \( 9,3\cdot7,1 \)
			\task \( 7,01\cdot150,02 \)
			\task \( 100,23\cdot8,96 \)
			\task \( 6,12\cdot7,36 \)
			\task \( 2,39\cdot7,12 \)
			\task \( 19,03\cdot0,002 \)
		\end{tasks}
		\item Вычислите:
		\begin{tasks}(4)
			\task \( \dfrac{9,4}{4,1+5,3} \)
			\task \( \dfrac{5,6}{1,9-7,5} \)
			\task \( \dfrac{7,5+3,5}{2,5} \)
			\task \( \dfrac{9,5+8,9}{2,3} \)
			\task \( \dfrac{2,7}{1,4+0,1} \)
			\task \( \dfrac{1,8+1,9}{3,7} \)
			\task \( \dfrac{4,7-1,4}{7,5} \)
			\task \( \dfrac{2,6-2,6}{7,8} \)
		\end{tasks}
		\item Какому из промежутков принадлежит число \( \dfrac{7}{11} \)?
		\begin{tasks}(4)
			\task \( [0,4; \, 0,5] \)
			\task \( [0,5; \, 0,6] \)
			\task \( [0,6; \, 0,7] \)
			\task \( [0,7; \, 0,8] \)
		\end{tasks}
		\item Известно, что \( 0<a<1 \). Выберите наименьшее число
		\begin{tasks}(4)
			\task \( a^2 \)
			\task \( a^3 \)
			\task \( -a \)
			\task \( \dfrac{1}{a} \)
		\end{tasks}
		\item Известно, что \( a<b<0 \). Выберите наименьшее из чисел.
		\begin{tasks}(4)
			\task \( a-1 \)
			\task \( b-1 \)
			\task \( ab \)
			\task \( -b \)
		\end{tasks}
		\item Известно, что \( a \) и \( b \) --- положительные числа и \( a>b \). Сравните \( \dfrac{1}{a} \) и \( \dfrac{1}{b} \).
		\begin{tasks}(4)
			\task \( \dfrac{1}{a}>\dfrac{1}{b} \)
			\task \( \dfrac{1}{a}<\dfrac{1}{b} \)
			\task \( \dfrac{1}{a}=\dfrac{1}{b} \)
			\task Сравнить невозможно
		\end{tasks}
		\item Известно, что \( a>b>c \). Какое из следующих чисел отрицательно?
		\begin{tasks}(4)
			\task \( a-b \)
			\task \( a-c \)
			\task \( b-c \)
			\task \( c-b \)
		\end{tasks}
		\item Какое из следующих чисел заключено между числами \( \dfrac{1}{6} \) и \( \dfrac{1}{4} \)?
		\begin{tasks}(4)
			\task \( 0,1 \)
			\task \( 0,2 \)
			\task \( 0,3 \)
			\task \( 0,4 \)
		\end{tasks}
		\item Какое из приведенных ниже неравенств является верным при любых значениях \( a \) и \( b \), удовлетворяющих условию \( a>b \)?
		\begin{tasks}(4)
			\task \( b-a<-2 \)
			\task \( a-b>-1 \)
			\task \( a-b<3 \)
			\task \( b-a>-3 \)
		\end{tasks}
		\newpage
		\item Значение какого из данных выражений положительно, если известно, что \( x>0 \), \( y<0 \)?
		\begin{tasks}(4)
			\task \( xy \)
			\task \( (x-y)y \)
			\task \( (y-x)y \)
			\task \( (y-x)x \)
		\end{tasks}
	\end{listofex}
\end{class}
%END_FOLD

%BEGIN_FOLD % ====>>_____ Занятие 4 _____<<====
\begin{class}[number=4]
	\begin{listofex}
		\item Занятие 4
	\end{listofex}
\end{class}
%END_FOLD

%BEGIN_FOLD % ====>>_ Домашняя работа 2 _<<====
\begin{homework}[number=2]
	\begin{listofex}
		\item Домашняя работа 2
	\end{listofex}
\end{homework}
%END_FOLD

%BEGIN_FOLD % ====>>_____ Занятие 5 _____<<====
\begin{class}[number=5]
	\begin{listofex}
		\item Занятие 5
	\end{listofex}
\end{class}
%END_FOLD

%BEGIN_FOLD % ====>>_____ Занятие 6 _____<<====
\begin{class}[number=6]
	\begin{listofex}
		\item Занятие 6
	\end{listofex}
\end{class}
%END_FOLD

%BEGIN_FOLD % ====>>_ Домашняя работа 3 _<<====
\begin{homework}[number=3]
	\begin{listofex}
		\item Домашняя работа 3
	\end{listofex}
\end{homework}
%END_FOLD

%BEGIN_FOLD % ====>>_____ Занятие 7 _____<<====
\begin{class}[number=7]
	\title{Подготовка к проверочной}
	\begin{listofex}
		\item Занятие 7
	\end{listofex}
\end{class}
%END_FOLD

=%BEGIN_FOLD % ====>>_ Проверочная работа _<<====
\begin{exam}
	\begin{listofex}
		\item Проверочная
	\end{listofex}
\end{exam}
%END_FOLD