%
%===============>>  ГРУППА 9-2 МОДУЛЬ 8  <<=============
%
\setmodule{8}

%BEGIN_FOLD % ====>>_____ Занятие 1 _____<<====
\begin{class}[number=1]
	\begin{listofex}
		\item Переведите в десятичную дробь:
		\begin{tasks}(4)
			\task \( \dfrac{1}{4} \)
			\task \( \dfrac{2}{5} \)
			\task \( \dfrac{3}{8} \)
			\task \( \dfrac{8}{25} \)
			\task \( \dfrac{15}{10} \)
			\task \( \dfrac{1}{100} \)
			\task \( \dfrac{3}{200} \)
			\task \( \dfrac{7}{500} \)
		\end{tasks}
		\item Вычислите:
		\begin{tasks}(4)
			\task \( \dfrac{4}{25}+\dfrac{15}{4} \)
			\task \( \dfrac{9}{4}+\dfrac{8}{5} \)
			\task \( \dfrac{19}{2}-\dfrac{7}{25} \)
			\task \( \dfrac{3}{2}-\dfrac{9}{5} \)
			\task \( \dfrac{9}{5}\cdot\dfrac{2}{3} \)
			\task \( \dfrac{21}{5}\cdot\dfrac{3}{7} \)
			\task \( \dfrac{14}{5}:\dfrac{7}{2} \)
			\task \( \dfrac{3}{5}:\dfrac{4}{35} \)
		\end{tasks}
		\item Вычислите:
		\begin{tasks}(2)
			\task \( \left( \dfrac{19}{8}+\dfrac{11}{12} \right):\dfrac{5}{48} \)
			\task \( \left( \dfrac{14}{11}+\dfrac{17}{10} \right)\cdot\dfrac{11}{15} \)
			\task \( \left( \mfrac{2}{3}{4}+\mfrac{2}{1}{5} \right)\cdot16 \)
			\task \( \mfrac{1}{8}{17}:\left( \dfrac{12}{17}+\mfrac{2}{7}{11} \right) \)
		\end{tasks}
		\item Какое из данных ниже чисел принадлежит промежутку \( [3;4] \)?
		\begin{tasks}(4)
			\task \( \dfrac{45}{19} \)
			\task \( \dfrac{52}{19} \)
			\task \( \dfrac{68}{19} \)
			\task \( \dfrac{77}{19} \)
		\end{tasks}
		\item Какому из данных промежутков принадлежит число \( \dfrac{5}{9} \)?
		\begin{tasks}(4)
			\task \( [0,5; \, 0,6] \)
			\task \( [0,6 \, 0,7] \)
			\task \( [0,7 \, 0,8] \)
			\task \( [0,8 \, 0,9] \)
		\end{tasks}
		\item Вычислите:
		\begin{tasks}(4)
			\task \( \dfrac{1}{4}+0,7 \)
			\task \( \dfrac{1}{2}+0,07 \)
			\task \( \dfrac{24}{3,2\cdot2} \)
			\task \( \dfrac{27}{3\cdot4,5} \)
			\task \( \dfrac{11}{4,4\cdot2,5} \)
			\task \( \dfrac{7}{12,5\cdot1,4} \)
			\task \( \dfrac{0,9}{1+\frac{1}{8}} \)
			\task \( \dfrac{0,3}{1+\frac{1}{9}} \)
		\end{tasks}
	\end{listofex}
\end{class}
%END_FOLD

%BEGIN_FOLD % ====>>_____ Занятие 28 _____<<====
\begin{class}[number=2]
	\begin{listofex}
		\item Занятие 2
	\end{listofex}
\end{class}
%END_FOLD

%BEGIN_FOLD % ====>>_ Домашняя работа 1 _<<====
\begin{homework}[number=1]
	\begin{listofex}
		\item Домашняя работа 1
	\end{listofex}
\end{homework}
%END_FOLD

%BEGIN_FOLD % ====>>_____ Занятие 3 _____<<====
\begin{class}[number=3]
	\begin{listofex}
		\item Занятие 3 
	\end{listofex}
\end{class}
%END_FOLD

%BEGIN_FOLD % ====>>_____ Занятие 4 _____<<====
\begin{class}[number=4]
	\begin{listofex}
		\item Занятие 4
	\end{listofex}
\end{class}
%END_FOLD

%BEGIN_FOLD % ====>>_ Домашняя работа 2 _<<====
\begin{homework}[number=2]
	\begin{listofex}
		\item Домашняя работа 2
	\end{listofex}
\end{homework}
%END_FOLD

%BEGIN_FOLD % ====>>_____ Занятие 5 _____<<====
\begin{class}[number=5]
	\begin{listofex}
		\item Занятие 5
	\end{listofex}
\end{class}
%END_FOLD

%BEGIN_FOLD % ====>>_____ Занятие 6 _____<<====
\begin{class}[number=6]
	\begin{listofex}
		\item Занятие 6
	\end{listofex}
\end{class}
%END_FOLD

%BEGIN_FOLD % ====>>_ Домашняя работа 3 _<<====
\begin{homework}[number=3]
	\begin{listofex}
		\item Домашняя работа 3
	\end{listofex}
\end{homework}
%END_FOLD

%BEGIN_FOLD % ====>>_____ Занятие 7 _____<<====
\begin{class}[number=7]
	\title{Подготовка к проверочной}
	\begin{listofex}
		\item Занятие 7
	\end{listofex}
\end{class}
%END_FOLD

=%BEGIN_FOLD % ====>>_ Проверочная работа _<<====
\begin{exam}
	\begin{listofex}
		\item Проверочная
	\end{listofex}
\end{exam}
%END_FOLD