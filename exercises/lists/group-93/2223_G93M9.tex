%
%===============>>  ГРУППА 9-3 МОДУЛЬ 9  <<=============
%
\setmodule{9}

%BEGIN_FOLD % ====>>_____ Занятие 1 _____<<====
\begin{class}[number=1]
	\begin{listofex}
		\item Вычислите:
		\begin{tasks}(2)
			\task \( \left( -\dfrac{ 5 }{ 10 } \right)+4 \cdot (-5) \)
			\task \( (-12 \cdot 0,4) \cdot 0,5 - 11 \)
			\task \( (-25 \cdot \dfrac{ 1 }{ 5 } - 30) \cdot \dfrac{ 1 }{ 7 } \)
			\task \( -\mfrac{3}{5}{21}+\mfrac{3}{17}{42}+\left(-\mfrac{18}{45}{66}\right) \)
			\task \( -15,2 + \left(-\mfrac{2}{17}{25}\right) + 12 \)
			\task \( -\mfrac{4}{2}{3} + \left(-\dfrac{22}{24}\right) - (-6) \)
		\end{tasks}
		\item Найдите значение выражений:
		\begin{tasks}(4)
			\task \( \dfrac{3^7}{81} \)
			\task \( \dfrac{27^5}{9^6} \)
			\task \( \dfrac{81^5}{27^6} \)
			\task \( \dfrac{125^6}{25^8} \)
			\task \( \dfrac{10^6}{2^5\cdot5^4} \)
			\task \( \dfrac{24^4}{3^2\cdot8^3} \)
			\task \( \dfrac{(3\cdot10)^8}{3^6\cdot10^7} \)
			\task \( \dfrac{(2\cdot3)^5}{2^4\cdot3^3} \)
		\end{tasks}
		\item Чему равно значение выражений?
		\begin{tasks}(3)
			\task \( (3\sqrt{2})^2 \)
			\task \( (5\sqrt{3})^2 \)
			\task \( (2\sqrt{7})^2 \)
		\end{tasks}
		\item Найдите значение выражения \( \dfrac{a^{23}\cdot(b^5)^4}{(a\cdot b)^{20}} \) при \( a=2 \) и \( b=\sqrt{2} \).
		\item Найдите значение выражения \( \dfrac{a^{17}\cdot(b^5)^3}{(a\cdot b)^{15}} \) при \( a=7 \) и \( b=\sqrt{7} \).
		\item Найдите значение выражений:
		\begin{tasks}(3)
			\task \( 5\sqrt{11}\cdot2\sqrt{2}\cdot\sqrt{22} \)
			\task \( \sqrt{4\cdot12}\cdot\sqrt{21} \)
			\task \( \sqrt{2\cdot45}\cdot\sqrt{10} \)
			\task \( 3\sqrt{19}\cdot4\sqrt{2}\cdot\sqrt{38} \)
			\task \( 3\sqrt{11}\cdot4\sqrt{2}\cdot\sqrt{22} \)
			\task \( \sqrt{45}\cdot\sqrt{605} \)
		\end{tasks}
	\end{listofex}
\end{class}
%END_FOLD

%BEGIN_FOLD % ====>>_____ Занятие 2 _____<<====
\begin{class}[number=2]
	\begin{listofex}
		\item Решите уравнения:
		\begin{tasks}(2)
			\task \( 7+9(4x+5)=-2 \)
			\task \( 9+2(3-4x)=3x-3 \)
			\task \( x+7-\dfrac{x}{3}=3 \)
			\task \( 1+\dfrac{x}{7}=x+7 \)
		\end{tasks}
		\item Найдите значение выражения \( a^{12}\cdot(a^{-4})^4 \) при \( a=-\dfrac{1}{2} \).
		\item Найдите значение выражения:
		\begin{tasks}(4)
			\task \( \dfrac{(2\cdot5)^6}{2^4\cdot5^5} \)
			\task \( \dfrac{(5\cdot7)^6}{5^4\cdot7^6} \)
			\task \( \dfrac{15^8}{3^6\cdot5^7} \)
			\task \( \dfrac{10^6}{2^5\cdot5^4} \)
		\end{tasks}
		\item Длину окружности \( l \) можно вычислить по формуле \( l=2\pi R \), где \( R \) --- радиус окружности (в метрах). Пользуясь этой формулой, найдите радиус окружности, если её длина равна \( 78 \) м. (Считать \( \pi=3 \)).
		\item Площадь ромба \( S \) (в м\( ^2 \))  можно вычислить по формуле \( S=\dfrac{1}{2}d_1d_2 \),  где \( d_1 \), \( d_2 \) --- диагонали ромба (в метрах). Пользуясь этой формулой, найдите диагональ \( d_1 \), если диагональ \( d_2 \) равна \( 30 \) м, а площадь ромба \( 120 \) м\( ^2 \).
		\item Площадь треугольника \( S \) (в м\( ^2 \)) можно вычислить по формуле \( S=\dfrac{1}{2}ah \),  где a\( a \) --- сторона треугольника, \( h \) --- высота, проведенная к этой стороне (в метрах). Пользуясь этой формулой, найдите сторону \( a \), если площадь треугольника равна \( 28 \) м\( ^2 \), а высота \( h \) равна \( 14 \) м.
		\item Площадь трапеции \( S \) (в м\( ^2 \)) можно вычислить по формуле \( S=\dfrac{a+b}{2}\cdot h \),  где \( a \), \( b \) --- основания трапеции, \( h \) --- высота (в метрах). Пользуясь этой формулой, найдите высоту \( h \), если основания трапеции равны \( 5 \) м и \( 7 \) м, а её площадь \( 24 \) м\( ^2 \).
		\item Объём пирамиды вычисляют по формуле \(V=\dfrac{ 1 }{ 3 }Sh\),  где \(S\) --- площадь основания пирамиды, \(h\) --- её высота. Объём пирамиды равен \(40\), площадь основания \(15\). Чему равна высота пирамиды?
		\item Найдите значение выражений:
		\begin{tasks}(3)
			\task \( 5\sqrt{11}\cdot2\sqrt{2}\cdot\sqrt{22} \)
			\task \( \sqrt{4\cdot12}\cdot\sqrt{21} \)
			\task \( \sqrt{2\cdot45}\cdot\sqrt{10} \)
			\task \( 3\sqrt{19}\cdot4\sqrt{2}\cdot\sqrt{38} \)
			\task \( 3\sqrt{11}\cdot4\sqrt{2}\cdot\sqrt{22} \)
			\task \( \sqrt{45}\cdot\sqrt{605} \)
		\end{tasks}
	\end{listofex}
\end{class}
%END_FOLD

%BEGIN_FOLD % ====>>_ Домашняя работа 1 _<<====
\begin{homework}[number=1]
	\begin{listofex}
		\item .
	\end{listofex}
\end{homework}
%END_FOLD

%BEGIN_FOLD % ====>>_____ Занятие 3 _____<<====
\begin{class}[number=3]
	\begin{listofex}
		\item.
	\end{listofex}
\end{class}
%END_FOLD

%BEGIN_FOLD % ====>>_____ Занятие 4 _____<<====
\begin{class}[number=4]
	\begin{listofex}
		\item Занятие 4
	\end{listofex}
\end{class}
%END_FOLD

%BEGIN_FOLD % ====>>_ Домашняя работа 2 _<<====
\begin{homework}[number=2]
	\begin{listofex}
		\item .
	\end{listofex}
\end{homework}
%END_FOLD

%BEGIN_FOLD % ====>>_____ Занятие 5 _____<<====
\begin{class}[number=5]
	\begin{listofex}
		\item .
	\end{listofex}
\end{class}
%END_FOLD

%BEGIN_FOLD % ====>>_____ Занятие 6 _____<<====
\begin{class}[number=6]
	\begin{listofex}
		\item .
	\end{listofex}
\end{class}
%END_FOLD

%BEGIN_FOLD % ====>>_ Домашняя работа 3 _<<====
\begin{homework}[number=3]
	\begin{listofex}
		\item .
	\end{listofex}
\end{homework}
%END_FOLD

%BEGIN_FOLD % ====>>_____ Занятие 7 _____<<====
\begin{class}[number=7]
	\begin{listofex}
		\item .
	\end{listofex}
\end{class}
%END_FOLD

%BEGIN_FOLD % ====>>_ Проверочная работа _<<====
\begin{class}[number=8]
	\begin{listofex}
		\item .
	\end{listofex}
\end{class}
%END_FOLD