%
%===============>>  ГРУППА 9-3 МОДУЛЬ 9  <<=============
%
\setmodule{9}

%BEGIN_FOLD % ====>>_____ Занятие 1 _____<<====
\begin{class}[number=1]
	\begin{listofex}
		\item Вычислите:
		\begin{tasks}(2)
			\task \( \left( -\dfrac{ 5 }{ 10 } \right)+4 \cdot (-5) \)
			\task \( (-12 \cdot 0,4) \cdot 0,5 - 11 \)
			\task \( (-25 \cdot \dfrac{ 1 }{ 5 } - 30) \cdot \dfrac{ 1 }{ 7 } \)
			\task \( -\mfrac{3}{5}{21}+\mfrac{3}{17}{42}+\left(-\mfrac{18}{45}{66}\right) \)
			\task \( -15,2 + \left(-\mfrac{2}{17}{25}\right) + 12 \)
			\task \( -\mfrac{4}{2}{3} + \left(-\dfrac{22}{24}\right) - (-6) \)
		\end{tasks}
		\item Найдите значение выражений:
		\begin{tasks}(4)
			\task \( \dfrac{3^7}{81} \)
			\task \( \dfrac{27^5}{9^6} \)
			\task \( \dfrac{81^5}{27^6} \)
			\task \( \dfrac{125^6}{25^8} \)
			\task \( \dfrac{10^6}{2^5\cdot5^4} \)
			\task \( \dfrac{24^4}{3^2\cdot8^3} \)
			\task \( \dfrac{(3\cdot10)^8}{3^6\cdot10^7} \)
			\task \( \dfrac{(2\cdot3)^5}{2^4\cdot3^3} \)
		\end{tasks}
		\item Чему равно значение выражений?
		\begin{tasks}(3)
			\task \( (3\sqrt{2})^2 \)
			\task \( (5\sqrt{3})^2 \)
			\task \( (2\sqrt{7})^2 \)
		\end{tasks}
		\item Найдите значение выражения \( \dfrac{a^{23}\cdot(b^5)^4}{(a\cdot b)^{20}} \) при \( a=2 \) и \( b=\sqrt{2} \).
		\item Найдите значение выражения \( \dfrac{a^{17}\cdot(b^5)^3}{(a\cdot b)^{15}} \) при \( a=7 \) и \( b=\sqrt{7} \).
		\item Найдите значение выражений:
		\begin{tasks}(3)
			\task \( 5\sqrt{11}\cdot2\sqrt{2}\cdot\sqrt{22} \)
			\task \( \sqrt{4\cdot12}\cdot\sqrt{21} \)
			\task \( \sqrt{2\cdot45}\cdot\sqrt{10} \)
			\task \( 3\sqrt{19}\cdot4\sqrt{2}\cdot\sqrt{38} \)
			\task \( 3\sqrt{11}\cdot4\sqrt{2}\cdot\sqrt{22} \)
			\task \( \sqrt{45}\cdot\sqrt{605} \)
		\end{tasks}
	\end{listofex}
\end{class}
%END_FOLD

%BEGIN_FOLD % ====>>_____ Занятие 2 _____<<====
\begin{class}[number=2]
	\begin{listofex}
		\item Решите уравнения:
		\begin{tasks}(2)
			\task \( 7+9(4x+5)=-2 \)
			\task \( 9+2(3-4x)=3x-3 \)
			\task \( x+7-\dfrac{x}{3}=3 \)
			\task \( 1+\dfrac{x}{7}=x+7 \)
		\end{tasks}
		\item Найдите значение выражения \( a^{12}\cdot(a^{-4})^4 \) при \( a=-\dfrac{1}{2} \).
		\item Найдите значение выражения:
		\begin{tasks}(4)
			\task \( \dfrac{(2\cdot5)^6}{2^4\cdot5^5} \)
			\task \( \dfrac{(5\cdot7)^6}{5^4\cdot7^6} \)
			\task \( \dfrac{15^8}{3^6\cdot5^7} \)
			\task \( \dfrac{10^6}{2^5\cdot5^4} \)
		\end{tasks}
		\item Длину окружности \( l \) можно вычислить по формуле \( l=2\pi R \), где \( R \) --- радиус окружности (в метрах). Пользуясь этой формулой, найдите радиус окружности, если её длина равна \( 78 \) м. (Считать \( \pi=3 \)).
		\item Площадь ромба \( S \) (в м\( ^2 \))  можно вычислить по формуле \( S=\dfrac{1}{2}d_1d_2 \),  где \( d_1 \), \( d_2 \) --- диагонали ромба (в метрах). Пользуясь этой формулой, найдите диагональ \( d_1 \), если диагональ \( d_2 \) равна \( 30 \) м, а площадь ромба \( 120 \) м\( ^2 \).
		\item Площадь треугольника \( S \) (в м\( ^2 \)) можно вычислить по формуле \( S=\dfrac{1}{2}ah \),  где a\( a \) --- сторона треугольника, \( h \) --- высота, проведенная к этой стороне (в метрах). Пользуясь этой формулой, найдите сторону \( a \), если площадь треугольника равна \( 28 \) м\( ^2 \), а высота \( h \) равна \( 14 \) м.
		\item Площадь трапеции \( S \) (в м\( ^2 \)) можно вычислить по формуле \( S=\dfrac{a+b}{2}\cdot h \),  где \( a \), \( b \) --- основания трапеции, \( h \) --- высота (в метрах). Пользуясь этой формулой, найдите высоту \( h \), если основания трапеции равны \( 5 \) м и \( 7 \) м, а её площадь \( 24 \) м\( ^2 \).
		\item Объём пирамиды вычисляют по формуле \(V=\dfrac{ 1 }{ 3 }Sh\),  где \(S\) --- площадь основания пирамиды, \(h\) --- её высота. Объём пирамиды равен \(40\), площадь основания \(15\). Чему равна высота пирамиды?
		\item Найдите значение выражений:
		\begin{tasks}(3)
			\task \( 5\sqrt{11}\cdot2\sqrt{2}\cdot\sqrt{22} \)
			\task \( \sqrt{4\cdot12}\cdot\sqrt{21} \)
			\task \( \sqrt{2\cdot45}\cdot\sqrt{10} \)
			\task \( 3\sqrt{19}\cdot4\sqrt{2}\cdot\sqrt{38} \)
			\task \( 3\sqrt{11}\cdot4\sqrt{2}\cdot\sqrt{22} \)
			\task \( \sqrt{45}\cdot\sqrt{605} \)
		\end{tasks}
	\end{listofex}
\end{class}
%END_FOLD

%BEGIN_FOLD % ====>>_ Домашняя работа 1 _<<====
\begin{homework}[number=1]	
	\begin{listofex}
		\item Вычислите:
		\begin{tasks}(2)
			\task \( \dfrac{7,2-6,1}{2,2} \)
			\task \( \dfrac{11}{4,4\cdot2,5} \)
			\task \( \left( \dfrac{7}{25}+\dfrac{7}{33} \right):\dfrac{14}{33} \)
			\task \( \left( \dfrac{17}{25}-\dfrac{1}{17} \right)\cdot\dfrac{17}{4} \)
		\end{tasks}
		\item Вычислите:
		\begin{tasks}(4)
			\task \( 3^{-7}\cdot(3^5)^2 \)
			\task \( \dfrac{(2\cdot6)^7}{2^5\cdot6^6} \)
			\task \( \dfrac{32^5}{8^8} \)
			\task \( \dfrac{6^7}{2^6\cdot3^5} \)
		\end{tasks}
		\item Вычислите:
		\begin{tasks}(2)
			\task \( \sqrt{40\cdot60\cdot75} \)
			\task \( 2\sqrt{11}\cdot2\sqrt{3}\cdot\sqrt{33} \)
			\task \( 4\sqrt{30}\cdot2\sqrt{2}\cdot\sqrt{60} \)
			\task \( 4\sqrt{15}\cdot2\sqrt{5}\cdot\sqrt{75} \)
		\end{tasks}
		\item Решите уравнения:
		\begin{tasks}(1)
			\task \( -3x+1-36(x+3)=-2(1-x)+2 \)
			\task \( -5x-2+4(x+1)=4(-3-x)-1 \)
		\end{tasks}
		\item Зная длину своего шага, человек может приближённо подсчитать пройденное им расстояние \( s \) по формуле \( s=nl \), где \( n \) --- число шагов, \( l \) --- длина шага. Какое расстояние прошёл человек, если \( l=80 \) см, \( n=1600 \)? Ответ выразите в километрах.
	\end{listofex}
\end{homework}
%END_FOLD

%BEGIN_FOLD % ====>>_____ Занятие 3 _____<<====
\begin{class}[number=3]
	\begin{listofex}
		\item В фирме «Родник» стоимость (в рублях) колодца из железобетонных колец рассчитывается по формуле \( C=6000+4100\cdot n \), где \( n \) --- число колец, установленных при рытье колодца. Пользуясь этой формулой, рассчитайте стоимость колодца из \( 5 \) колец.
		\item В фирме «Эх, прокачу!» стоимость поездки на такси (в рублях) рассчитывается по формуле \( C=150+11\cdot(t-5) \), где \( t \) --- длительность поездки, выраженная в минутах (\( t>5 \)). Пользуясь этой формулой, рассчитайте стоимость \( 8 \)-минутной поездки.
		\item Радиус вписанной в прямоугольный треугольник окружности можно найти по формуле \( r=\dfrac{a+b-c}{2} \),  где \( a \) и \( b \) --- катеты, а \( c \) --- гипотенуза треугольника. Пользуясь этой формулой, найдите \( b \), если \( r=1,2 \); \( c=6,8 \) и \( a=6 \).
		\item Площадь любого выпуклого четырехугольника можно вычислять по формуле \( S=\dfrac{1}{2}d_1d_2\sin\alpha \),  где \( d_1 \), \( d_2 \) --- длины его диагоналей, а \( \alpha \) угол между ними. Вычислите \( \sin\alpha \), если \( S=21 \), \( d_1=7 \), \( d_2=15 \).
		\item Мощность постоянного тока (в ваттах) вычисляется по формуле \( P=I^2R \), где \( I \) --- сила тока (в амперах), \( R \) --- сопротивление (в омах). Пользуясь этой формулой, найдите сопротивление \( R \) (в омах), если мощность составляет \( 150 \) ватт, а сила тока равна \( 5 \) амперам.
		\item На экзамене \( 25 \) билетов, Сергей не выучил \( 3 \) из них. Найдите вероятность того, что ему попадётся выученный билет.
		\item Телевизор у Маши сломался и показывает только один случайный канал. Маша включает телевизор. В это время по трем каналам из двадцати показывают кинокомедии. Найдите вероятность того, что Маша попадет на канал, где комедия не идет.
		\item На тарелке \( 12 \) пирожков: \( 5 \) с мясом, \( 4 \) с капустой и \( 3 \) с вишней. Наташа наугад выбирает один пирожок. Найдите вероятность того, что он окажется с вишней.
		\item В фирме такси в данный момент свободно \( 20 \) машин: \( 9 \) черных, \( 4 \) желтых и \( 7 \) зеленых. По вызову выехала одна из машин, случайно оказавшаяся ближе всего к заказчику. Найдите вероятность того, что к нему приедет желтое такси.
		\item В каждой десятой банке кофе согласно условиям акции есть приз. Призы распределены по банкам случайно. Варя покупает банку кофе в надежде выиграть приз. Найдите вероятность того, что Варя не найдет приз в своей банке.
		\item Миша с папой решили покататься на колесе обозрения. Всего на колесе двадцать четыре кабинки, из них \( 5 \) --- синие, \( 7 \) --- зеленые, остальные  — красные. Кабинки по очереди подходят к платформе для посадки. Найдите вероятность того, что Миша прокатится в красной кабинке.
		\item У бабушки \( 20 \) чашек: \( 5 \) с красными цветами, остальные с синими. Бабушка наливает чай в случайно выбранную чашку. Найдите вероятность того, что это будет чашка с синими цветами.
		\item Родительский комитет закупил \( 25 \) пазлов для подарков детям на окончание года, из них \( 15 \) с машинами и \( 10 \) с видами городов. Подарки распределяются случайным образом. Найдите вероятность того, что Толе достанется пазл с машиной.
		\item В среднем из каждых \( 80 \) поступивших в продажу аккумуляторов \( 76 \) аккумуляторов заряжены. Найдите вероятность того, что купленный аккумулятор не заряжен.
		\item Для экзамена подготовили билеты с номерами от \( 1 \) до \( 50 \). Какова вероятность того, что наугад взятый учеником билет имеет однозначный номер?
		\item Из \( 900 \) новых флеш-карт в среднем \( 54 \) не пригодны для записи. Какова вероятность того, что случайно выбранная флеш-карта пригодна для записи?
		\item В коробке \( 14 \) пакетиков с чёрным чаем и \( 6 \) пакетиков с зелёным чаем. Павел наугад вынимает один пакетик. Какова вероятность того, что это пакетик с зелёным чаем?
		\item Стас, Денис, Костя, Маша, Дима бросили жребий --- кому начинать игру. Найдите вероятность того, что начинать игру должна будет девочка.
		\item Вычислите:
		\begin{tasks}(2)
			\task \( \sqrt{66\cdot110\cdot15} \)
			\task \( 3\sqrt{19}\cdot3\sqrt{2}\cdot\sqrt{38} \)
			\task \( 2\sqrt{22}\cdot2\sqrt{3}\cdot\sqrt{66} \)
			\task \( 2\sqrt{10}\cdot5\sqrt{6}\cdot\sqrt{60} \)
			\task \( 3\sqrt{13}\cdot3\sqrt{2}\cdot\sqrt{26} \)
			\task \( 2\sqrt{41}\cdot2\sqrt{3}\cdot\sqrt{123} \)
			\task \( 2\sqrt{10}\cdot3\sqrt{3}\cdot\sqrt{30} \)
			\task \( 4\sqrt{11}\cdot4\sqrt{3}\cdot\sqrt{33} \)
			\task \( 2\sqrt{11}\cdot8\sqrt{3}\cdot\sqrt{33} \)
		\end{tasks}
	\end{listofex}
\end{class}
%END_FOLD

%BEGIN_FOLD % ====>>_____ Занятие 4 _____<<====
\begin{class}[number=4]
	\begin{listofex}
		\item Занятие 4
	\end{listofex}
\end{class}
%END_FOLD

%BEGIN_FOLD % ====>>_ Домашняя работа 2 _<<====
\begin{homework}[number=2]
	\begin{listofex}
		\item Вычислите:
		\begin{tasks}(2)
			\task \( \sqrt{66\cdot110\cdot15} \)
			\task \( 3\sqrt{19}\cdot3\sqrt{2}\cdot\sqrt{38} \)
			\task \( 2\sqrt{22}\cdot2\sqrt{3}\cdot\sqrt{66} \)
			\task \( 2\sqrt{10}\cdot5\sqrt{6}\cdot\sqrt{60} \)
			\task \( 3\sqrt{13}\cdot3\sqrt{2}\cdot\sqrt{26} \)
			\task \( 2\sqrt{41}\cdot2\sqrt{3}\cdot\sqrt{123} \)
		\end{tasks}
		\item Решите уравнения:
		\begin{tasks}(2)
			\task \( \dfrac{6x+8}{2}+5=\dfrac{5x}{3} \)
			\task \( 6+\dfrac{x}{2}=\dfrac{x+3}{5} \)
		\end{tasks}
		\item Закон Джоуля-Ленца можно записать в виде \( Q=I^2Rt \), где Q --- количество теплоты (в джоулях), \( I \) --- сила тока (в амперах), \( R \) --- сопротивление цепи (в омах), а \( t \) ---  время (в секундах). Пользуясь этой формулой, найдите время \( t \) (в секундах), если \( Q=1521 \)  Дж, \( I=6,5  \) A, \( R=9 \)  Ом.
		\item Закон Менделеева-Клапейрона можно записать в виде \( PV=vRT \), где \( P \) --- давление (в паскалях), \( V \) --- объём (в м\( ^3 \)), \( v \) --- количество вещества (в молях), \( T \) --- температура (в градусах Кельвина), а \( R \) --- универсальная газовая постоянная, равная \( 8,31 \) Дж/(К\( \cdot \)моль). Пользуясь этой формулой, найдите количество вещества \( v \) (в молях), если \( T=700 \) К, \( P=20 941,2 \) Па, \( V=9,5 \) м\( ^3 \).
		\item На экзамене \( 20 \) билетов, Сергей не выучил \( 3 \) из них. Найдите вероятность того, что ему попадётся выученный билет.
		\item Максим с папой решили покататься на колесе обозрения. Всего на колесе двадцать кабинок, из них \( 4 \) --- синие, \( 10 \) --- зеленые, остальные --- красные. Кабинки по очереди подходят к платформе для посадки. Найдите вероятность того, что Максим прокатится в красной кабинке.
		\item Из \( 900 \) новых флеш-карт в среднем \( 54 \) не пригодны для записи. Какова вероятность того, что случайно выбранная флеш-карта пригодна для записи?
		\item В лыжных гонках участвуют \( 13 \) спортсменов из России, \( 2 \) спортсмена из Норвегии и \( 5 \) спортсменов из Швеции. Порядок, в котором спортсмены стартуют, определяется жребием. Найдите вероятность того, что первым будет стартовать спортсмен не из России.
	\end{listofex}
\end{homework}
%END_FOLD

%BEGIN_FOLD % ====>>_____ Занятие 5 _____<<====
\begin{class}[number=5]
	\begin{listofex}
		\item .
	\end{listofex}
\end{class}
%END_FOLD

%BEGIN_FOLD % ====>>_____ Занятие 6 _____<<====
\begin{class}[number=6]
	\begin{listofex}
		\item .
	\end{listofex}
\end{class}
%END_FOLD

%BEGIN_FOLD % ====>>_ Домашняя работа 3 _<<====
\begin{homework}[number=3]
	\begin{listofex}
		\item .
	\end{listofex}
\end{homework}
%END_FOLD

%BEGIN_FOLD % ====>>_____ Занятие 7 _____<<====
\begin{class}[number=7]
	\begin{listofex}
		\item .
	\end{listofex}
\end{class}
%END_FOLD

%BEGIN_FOLD % ====>>_ Проверочная работа _<<====
\begin{class}[number=8]
	\begin{listofex}
		\item .
	\end{listofex}
\end{class}
%END_FOLD