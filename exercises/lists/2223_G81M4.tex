%
%===============>>  ГРУППА 8-1 МОДУЛЬ 4  <<=============
%
\setmodule{4}
%
%===============>>  Занятие 1  <<===============
%
\begin{class}[number=1]
	\begin{listofex}
		\item Вычислить рациональным способом:
		\begin{enumcols}[itemcolumns=3]
			\item \( \sqrt{16+4\cdot4\cdot24} \)
			\item \( \sqrt{83^3\cdot2^2-83^2\cdot2^3} \)
			\item \( \sqrt{50^2-4\cdot7\cdot7} \)
		\end{enumcols}
		\item Решить уравнение:
		\begin{enumcols}[itemcolumns=4]
			\item \( 81x^2-16=0 \)
			\item \( 5x^2-25=0 \)
			\item \( -7x^2=-1 \)
			\item \( 50-2x^2=0 \)
		\end{enumcols}
		\item Решить уравнение:
		\begin{enumcols}[itemcolumns=3]
			\item \( 7x^2=-x \)
			\item \( x=x^2 \)
			\item \( 11x-5x^2 \)
		\end{enumcols}
		\item Решить уравнение:
		\begin{enumcols}[itemcolumns=2]
			\item \( (15-x)(x-2)=(x-6)(x+5) \)
			\item \( (x-1)(x-2)+(x+4)(x-4)+3x=0 \)
		\end{enumcols}
		\item Решить уравнение:
		\begin{enumcols}[itemcolumns=3]
			\item \( (x+1)^2=9 \)
			\item \( (2x-1)^2=64 \)
			\item \( (7x+4)^2=225 \)
			\item \( (10x+10)^2=10000 \)
			\item \( (2x-3)^2=121 \)
			\item \( \left( \dfrac{1}{2}x+\dfrac{2}{3} \right)^2=\dfrac{4}{9} \)
			\item \( (x+0,4)^2=2,89 \)
			\item \( (2,5x-10)^2=0,25 \)
			\item \( (0,2x+4,1)^2=1,21 \)
		\end{enumcols}
		\item Решить уравнение:
		\begin{enumcols}[itemcolumns=2]
			\item \( (x-7)^2=3 \)
			\item \( (x+5)^2=5 \)
		\end{enumcols}
		\item Решить уравнение:
		\begin{enumcols}[itemcolumns=3]
			\item \( x^2-10x+25=0 \)
			\item \( 4x^2+20x+25=0 \)
			\item \( 16x^2-24x+9=0 \)
		\end{enumcols}
		\item Решить уравнение:
		\begin{enumcols}[itemcolumns=3]
			\item \( x^2-12x+36=4 \)
			\item \( 4x^2+40x+100=81 \)
			\item \( 9x^2-60x+100=25 \)
		\end{enumcols}
		\item Решить уравнение:
		\begin{enumcols}[itemcolumns=3]
			\item \( x^2-18x+77=0 \)
			\item \( x^2-10x-39=0 \)
			\item \( x^2-22x+72=0 \)
		\end{enumcols}
		\item Решить уравнение: \( (2x-3)^2-(x-5)(x+5)=2(2x+7) \)
	\end{listofex}
\end{class}
%
%===============>>  Занятие 2  <<===============
%
\begin{class}[number=2]
	\begin{listofex}
		\item Вычислить рациональным способом:
		\begin{enumcols}[itemcolumns=3]
			\item \( \sqrt{2^2+4\cdot15} \)
			\item \( \sqrt{90^2-4\cdot25\cdot81} \)
			\item \( \sqrt{4^2+4\cdot5\cdot12} \)
		\end{enumcols}
	\item Решить уравнение:
	\begin{enumcols}[itemcolumns=3]
		\item \exercise{392}
		\item \exercise{397}
		\item \exercise{391}
		\item \exercise{400}
		\item \exercise{393}
		\item \exercise{396}
	\end{enumcols}
	\item Решить уравнение:
	\begin{enumcols}[itemcolumns=3]
		\item \exercise{404}
		\item \exercise{416}
		\item \exercise{418}
		\item \exercise{421}
		\item \exercise{424}
		\item \exercise{425}
	\end{enumcols}
	\item Решить уравнение:
	\begin{enumcols}[itemcolumns=3]
		\item \( 9x^2+24x+16=0 \)
		\item \( 36x^2-60x+25=0 \)
		\item \( 100y^2-1y+\dfrac{1}{4}=0 \)
		\end{enumcols}
	\item Решить уравнение:
	\begin{enumcols}[itemcolumns=3]
		\item \( x^2-12x+36=4 \)
		\item \( 4x^2+40x+100=81 \)
		\item \( 9x^2-60x+100=25 \)
	\end{enumcols}
	\item Решить уравнение:
	\begin{enumcols}[itemcolumns=3]
		\item \( x^2-18x+77=0 \)
		\item \( x^2-10x-39=0 \)
		\item \( x^2-22x+72=0 \)
	\end{enumcols}
	\item Решить уравнение: \( (2x-3)^2-(x-5)(x+5)=2(2x+7) \)
	\end{listofex}
\end{class}
%
%===============>>  Домашняя работа 1  <<===============
%
\begin{homework}[number=1]
	\begin{listofex}
		\item Решить уравнение:
		\begin{enumcols}[itemcolumns=3]
			\item \exercise{388}
			\item \exercise{400}
			\item \exercise{395}
			\item \exercise{404}
			\item \exercise{415}
			\item \exercise{423}
		\end{enumcols}
		\item \exercise{4139}
		\item Решить уравнение:
			\begin{enumcols}[itemcolumns=3]
			\item \( 4y^2-12y+9=0 \)
			\item \( 9z^2-60z+100=0 \)
			\item \( 64x^2+48x+9=0 \)
		\end{enumcols}
		\item Решить уравнение:
			\begin{enumcols}[itemcolumns=3]
			\item \( (2x+1)^2=0 \)
			\item \( \left( \dfrac{1}{3}-x^2 \right)=9 \)
			\item \( (5x-x^2)=36 \)
		\end{enumcols}
	 \item От листа жести, имеющего форму квадрата, отрезали полосу шириной \( 3 \) см. Площадь его оставшейся части равна \( 10 \) см\( ^2 \) . Найдите первоначальные размеры листа жести.
	 \item Решить уравнение: \( (2x-3)(x+1)+(x-6)(x+6)+x=0 \)
	\end{listofex}
\end{homework}
%
%===============>>  Занятие 3  <<===============
%
%\begin{class}[number=3]
%	\begin{listofex}
%		\item Пусто
%	\end{listofex}
%\end{class}
%
%===============>>  Занятие 4  <<===============
% смещение на одно занятие с прошлого месяца
%\begin{class}[number=4]
%	\begin{listofex}
%		\item Пусто
%	\end{listofex}
%\end{class}
%
%===============>>  Домашняя работа 2  <<===============
%
%\begin{homework}[number=2]
%	\begin{listofex}
%
%	\end{listofex}
%\end{homework}
%
%===============>>  Занятие 5  <<===============
% смещение на одно занятие с прошлого месяца
%\begin{class}[number=5]
%	\begin{listofex}
%		\item Пусто
%	\end{listofex}
%\end{class}
%
%===============>>  Домашняя работа 3  <<===============
%
%\begin{homework}[number=2]
%	\begin{listofex}
%
%	\end{listofex}
%\end{homework}
%\newpage
%\title{Подготовка к проверочной работе}
%\begin{listofex}
%	
%\end{listofex}
%
%===============>>  Занятие 7  <<===============
%
%\begin{class}[number=7]
%	\begin{listofex}
%	
%	\end{listofex}
%\end{class}
%
%===============>>  Провечная работа  <<===============
%
%\begin{exam}
%	\begin{listofex}
%	
%	\end{listofex}
%\end{exam}