%
%===============>>  ГРУППА 8-1 МОДУЛЬ 7  <<=============
%
\setmodule{7}

%BEGIN_FOLD % ====>>_____ Занятие 1 _____<<====
\begin{class}[number=1]
	\begin{listofex}
		\item Занятие 1
	\end{listofex}
\end{class}
%END_FOLD

%BEGIN_FOLD % ====>>_____ Занятие 2 _____<<====
\begin{class}[number=2]
	\begin{listofex}
		\item В прямоугольном треугольнике два катета равны \( 3 \) и \(  4 \). Найдите гипотенузу.
		\item Катеты прямоугольного треугольника равны \( 5 \) и \( 12 \). Чему равна гипотенуза?
		\item В прямоугольном треугольнике гипотенуза равна \( 20 \), а один из катетов равен \( 12 \). Чему равен другой катет?
		\item В прямоугольном треугольнике один из катетов равен \( 8 \), гипотенуза этого треугольника равна \( 17 \). Чему равен второй катет?
		\item Катет прямоугольного треугольника равен \( 7 \), а гипотенуза равна \( 25 \). Найдите длину второго катета.
		\item В равнобедренном прямоугольном треугольнике гипотенуза равна  \( 14\sqrt{2} \). Найдите длину катетов.
		\item Основание равнобедренного треугольника равно \( 16 \), боковая сторона равна \( 10 \). Чему равна высота проведенная к основанию этого треугольника?
		\item Боковая сторона равнобедренного треугольника равна \( 13 \), а длина основания равна \( 10 \). Найдите площадь этого треугольника.
		\item Найдите высоты равностороннего треугольника, если известно, что его сторона равна \( 8\sqrt{2} \).
		\item Дан параллелограмм \( ABCD \), где \( AB=15 \), \( AD=9 \). Также известно, что высота из точки \( B \) к основанию \( AD \) падает на точку \( D \). Чему равна площадь параллелограмма?
		\item Диагонали ромба равны \( 12 \) и \( 16 \) см. Чему равен периметр? Чему равна площадь? 
		\item Найдите все углы параллелограмма, если сумма двух из них равна \( 100^{\circ}\).
		\item В параллелограмме биссектриса угла \( А \) делит сторону \( ВС  \) на отрезки, равные \( 3 \) и \( 5  \) см. Найдите его периметр.
		\item Найдите углы параллелограмма, зная, что один из них больше другого на \( 50^{\circ} \).
	\end{listofex}
\end{class}
%END_FOLD

%BEGIN_FOLD % ====>>_____ Занятие 2 _____<<====
\begin{class}[number=2]
	\begin{listofex}
		\item Миша прошел от дома по направлению на восток \( 800 \) м. Затем повернул на север и прошел \( 600 \) м. На каком расстоянии (в метрах) от дома оказался Миша?
		\item Полина прошла от дома по направлению на запад \( 500 \) м. Затем повернула на север и прошла \( 300 \) м. После этого она повернула на восток и прошла еще \( 100 \) м. На каком расстоянии (в метрах) от дома оказалась Полина?
		\item Точка крепления троса, удерживающего флагшток в вертикальном положении, находится на высоте \( 15 \) м от земли. Расстояние от основания флагштока до места крепления троса на земле равно \( 8 \) м. Найдите длину троса.
		\item Пожарную лестницу длиной \( 13 \) м приставили к окну пятого этажа дома. Нижний конец лестницы отстоит от стены на \( 5 \) м. На какой высоте расположено окно? Ответ дайте в метрах.
		\item Высота равнобедренного треугольника равна \( 20 \) см, а его основание --- \( 10 \) см. Найдите его боковую сторону.
		\item Найдите площадь и периметр прямоугольника, сторона которого равна \( 9 \) см, а диагональ --- \( 15 \) см.
		\item Дан параллелограмм \( ABCD \), где \( AB=15 \), \( AD=9 \). Также известно, что высота из точки \( B \) к основанию \( AD \) падает на точку \( D \). Чему равна площадь параллелограмма?
		\item Диагонали ромба равны \( 12 \) и \( 16 \) см. Чему равен периметр? Чему равна площадь? 
		\item Найдите все углы параллелограмма, если сумма двух из них равна \( 100^{\circ}\).
		\item В параллелограмме биссектриса угла \( А \) делит сторону \( ВС  \) на отрезки, равные \( 3 \) и \( 5  \) см. Найдите его периметр.
		\item Найдите углы параллелограмма, зная, что один из них больше другого на \( 50^{\circ} \).
	\end{listofex}
\end{class}
%END_FOLD

%BEGIN_FOLD % ====>>_ Домашняя работа 1 _<<====
\begin{homework}[number=1]
	\begin{listofex}
		\item Стороны прямоугольника имеют длину \( 8 \) и \( 15 \) см. Найдите длину его диагонали.
		\item Пожарную лестницу приставили к окну, расположенному на высоте \( 12 \) м от земли. Нижний конец лестницы отстоит от стены на \( 5 \) м. Какова длина лестницы? Ответ дайте в метрах.
		\item Точка крепления троса, удерживающего флагшток в вертикальном положении, находится на высоте \( 6,3 \) м от земли. Расстояние от основания флагштока до места крепления троса на земле равно \( 1,6 \) м. Найдите длину троса в метрах.
		\item Один из острых углов прямоугольного треугольника составляет \( 30\degree \), а его гипотенуза равна \( 10 \). Найдите оба катета.
		\item Решите квадратное уравнение:
		\[x(x-5)=1-4x\]
	\end{listofex}
\end{homework}
%END_FOLD

%BEGIN_FOLD % ====>>_____ Занятие 3 _____<<====
\begin{class}[number=3]
	\begin{definit}
		\textbf{Синусом} острого угла прямоугольного треугольника называется отношение противолежащего катета к гипотенузе.
	\end{definit}
	\begin{definit}
		\textbf{Косинусом} острого угла прямоугольного треугольника называется отношение прилежащего катета к гипотенузе.
	\end{definit}
	\begin{listofex}
		\item В треугольнике \( ABC \) угол \( C \) равен \( 90 \) градусов, \( AC=6 \), \( AB=20 \). Найдите \( \sin B \).
		\item  В треугольнике \( ABC \) угол \( C \) равен \( 90 \) градусов, \( BC=9 \), \( AB=20 \). Найдите \( \cos B \).
		\item Найдите синус, косинус углов \( A \) и \( B \) треугольника \( ABC \) с прямым углом \( C \), если:
		\begin{tasks}(2)
			\task \( BC=8 \), \( AB=17 \)
			\task \( BC=21 \), \( AC=20 \)
			\task \( BC=1 \), \( AC=2 \)
		\end{tasks}
		
		\item В треугольнике \( ABC \) угол \( C \) прямой, \( BC=8 \), \( \sin A=0,4 \). Найдите \( AB \).
		\item В треугольнике \( ABC \) угол \( C \) прямой, \( AC=15 \), \( \cos A=\dfrac{5}{7} \). Найдите \( AB \).
		\item В треугольнике \( ABC \) угол \( C \) равен \( 90\degree \), \( BC=12 \), \( \sin A=\dfrac{4}{11} \). Найдите \( AB \).
		\item Катеты прямоугольного треугольника равны \( \sqrt{15} \) и \( 1 \). Найдите синус наименьшего угла этого треугольника.
		\item В треугольнике \( ABC \) угол \( C \) равен \( 90\degree \), \( \sin A=\dfrac{4}{5} \), \( AC=9 \). Найдите \( AB \).
		\item  Есть прямоугольный треугольник \( ABC \), где \( \angle C=90^{\circ} \), и \( AC=7 \), \( AB=25 \). Найдите длину \( BC \).
		\item Сумма двух углов параллелограмма равна \( 62^{\circ} \). Найдите один из оставшихся углов. Ответ дайте в градусах.
		\item Две стороны параллелограмма относятся как \( 9:11 \), а периметр его равен \( 40 \). Найдите большую сторону параллелограмма.
	\end{listofex}
\end{class}
%END_FOLD

%BEGIN_FOLD % ====>>_____ Занятие 4 _____<<====
\begin{class}[number=4]
	\begin{listofex}
		\item 
		\begin{minipage}[t]{\bodywidth}
		Найдите синус и косинус угла AOB, изображенного на рисунке.
	\end{minipage}
	\hspace{0.02\linewidth}
	\begin{minipage}[t]{\picwidth}
		\includegraphics[align=t, width=\linewidth]{\picpath/G81M7L4-1}
	\end{minipage}
	\item Найдите синус, косинус углов \( A \) и \( B \) треугольника \( ABC \) с прямым углом \( C \), если:
	\begin{tasks}(2)
		\task \( BC=4 \), \( AB=12 \)
		\task \( BC=3 \), \( AC=5 \)
		\task \( BC=13 \), \( AC=24 \)
	\end{tasks}
	\end{listofex}
		\begin{definit}
			\textbf{Основное тригонометрическое тождество }
			\begin{tasks}(3)
				\task[]
				\task[] \( \sin ^{2} x+\cos ^{2} x=1  \)
				\task[]
			\end{tasks} 
		\end{definit}
		\begin{listofex}[resume]
		\item Найдите:
		\begin{tasks}(2)
			\task \( \sin\alpha \), если \( \cos\alpha=\dfrac{1}{2} \)
			\task \( \cos\alpha \), если \( \sin\alpha=\dfrac{\sqrt{3}}{2} \)
		\end{tasks}
		\item Вычислить \( \sin \alpha \), если \( \cos \alpha =\dfrac{3}{5}^{\circ} \)	и \( 0 < \angle < \dfrac{\pi}{2} \).
		\item В треугольнике \( ABC  \) угол \( C \) равен \( 90^{\circ} \), \( \sin B = \dfrac{4}{15} \) , \( AB=45 \). Найдите \( AC \).
		\item В треугольнике \( ABC \) угол \( C \) прямой, \( AC=18 \), \( \cos A=\dfrac{6}{13} \). Найдите \( AB \).
		\item В треугольнике \( ABC \) угол \( C \) равен \( 90^{\circ} \), \( AC  =  4 \),  \(  \cos A = 0,5 \). Найдите \( AB \).  
		\item Катеты прямоугольного треугольника равны \( 4\sqrt{6} \) и 2. Найдите синус наименьшего угла этого треугольника.
		\item Найдите синус и косинус  большего острого угла прямоугольного треугольника с катетами \( 7 \) см и \( 24  \) см.
		\item Периметр равнобедренного треугольника равен \( 64 \) см, косинус угла при основании равен \( 0,6 \). Найдите площадь треугольника.
		\item В равнобедренном треугольнике \( ABC \) с основанием \( AB \) боковая сторона равна \( 16\sqrt{15} \),  \( \sin\angle BAC=0,25 \). Найдите длину высоты \( AH \).
		\item Площадь параллелограмма равна \( 40 \), а две его стороны равны \( 5  \) и \( 10 \). Найдите его высоты.
	\end{listofex}
\end{class}
%END_FOLD

%BEGIN_FOLD % ====>>_ Домашняя работа 2 _<<====
\begin{homework}[number=2]
	\begin{listofex}
		\item В треугольнике \( ABC \) угол \( C \) прямой, \( BC=12 \), \( \sin A=\dfrac{6}{13} \). Найдите \( AB \).
		\item В треугольнике \( ABC \) угол \( C \) прямой, \( BC=15 \), \( \cos B=\dfrac{18}{30} \). Найдите \( AB \).
<<<<<<< HEAD
<<<<<<< HEAD
<<<<<<< HEAD
=======
>>>>>>> 5b82ef55a567f609102f68cabde235f7e87c6459
		\item  В параллелограмме \( ABCD \) \( CH \) – высота, \( AB = 4 \), \( \cos C = 0,8 \). Найдите \( CH \).
		\item  В треугольнике \( ABC \) \( AB = BC = 10 \), \( \sin A = 0,2 \). Найдите \( AC \).
		\item Вычислить \( \cos \alpha \), если \( \sin \alpha =\dfrac{12}{13}\).Угол \( \alpha \) острый.
		\item  Высота треугольника равна \( 10 \) и образует с прилежащими сторонами углы \( 45\degree \) и \( 60\degree \). Найдите стороны треугольника.
<<<<<<< HEAD
=======
=======
>>>>>>> b462c8b7b84e7b37a6423b335d847ef2fc3b38cf
=======
>>>>>>> 5b82ef55a567f609102f68cabde235f7e87c6459
		\item  Высота треугольника равна \( 10 \) и образует с прилежащими сторонами углы \( 45\degree \) и \( 60\degree \). Найдите стороны треугольника.
		\item  В параллелограмме \( ABCD \) \( CH \) --- высота, \( AB = 4 \), \( \cos C = 0,8 \). Найдите \( CH \).
		\item  В треугольнике \( ABC \) \( AB = BC = 10 \), \( \sin A = 0,2 \). Найдите \( AC \).
		\item Вычислите \( \cos \alpha \), если \( \sin \alpha =\dfrac{12}{13} \). Угол \( \alpha \) острый.
		\item Катеты прямоугольного треугольника равны \( \sqrt{19} \) и \( 9 \). Найдите синус наименьшего угла этого треугольника.
		\item  В параллелограмме \( ABCD \) \( CH \) – высота, \( AB = 4 \), \( \cos C = 0,8 \). Найдите \( CH \).
		\item  В треугольнике \( ABC \) \( AB = BC = 10 \), \( \sin A = 0,2 \). Найдите \( AC \).
		\item Вычислить \( \cos \alpha \), если \( \sin \alpha =\dfrac{12}{13}\).Угол \( \alpha \) острый.
		\item  Высота треугольника равна \( 10 \) и образует с прилежащими сторонами углы \( 45\degree \) и \( 60\degree \). Найдите стороны треугольника.
	\end{listofex}
\end{homework}
%END_FOLD

%BEGIN_FOLD % ====>>_____ Занятие 5 _____<<====
\begin{class}[number=5]
	\begin{listofex}
		\item Решите уравнение 
		\begin{tasks}(2)
			\task \( 10 (x-9) =7 \)
			\task \( 2-3(2x+2)=5-4x \)
			\task \( 3x + 5 +(x+5) = (1-x) + 4 \)
			\task \( \dfrac{x-4}{x-6} =2 \)  
			\task \( \dfrac{ 10}{x+6} =1.  \) 
			\task \( \dfrac{6}{x+8} = - \dfrac{3}{4}\) 
			\task \( \dfrac{ x-14}{x-8} = \dfrac{7}{10} \)
			\task \( \dfrac{3}{x-19} =  \dfrac{19}{x-3}  \)
		\end{tasks} 
		\item В треугольнике \( ABC\) \( \angle C = 90\degree \), \(  AB = 1 \), \( BC = 5 \). Найдите \( \sin B \), \( \cos B \).
		\item  Треугольник \( ABC \) прямоугольный, \( AC = 3 \),  \( \sin B = 0,6 \).  Найдите \(AB, BC \).
		\item Треугольник \( ABC \)  \( AB = 15 \), \( \cos B = 0,6 \). Найдите \( AC, BC \).
		\item В треугольнике \( ABC \) угол \( C \) равен \( 90° \),  \( \cos A =  \dfrac{7}{25} \).  Найдите  \( \cos B \).
		\item Дан прямоугольный треугольник \( MNK \) с гипотенузой \( MN \). Найдите \( \cos⁡\angle N \), если \( \sin⁡\angle M=0,8 \).
		\item В треугольнике \( ABC \) \( \angle C=90\degree \) \( \sin\angle BAC=\dfrac{2}{3} \). Найдите \( AC \), если \( AB=6\sqrt{5} \).
		\item Дан прямоугольный треугольник \( ABC \), причем \( \angle C=90\degree \). Известно, что \( \cos⁡\angle B=13 \), \( AB=9 \). Найдите \( BC \).
		\item В параллелограмме \( ABCD \) диагональ \( AC \) в \( 2 \) раза больше стороны \( AB \) и \( \angle ACD=169\degree \). Найдите меньший угол между диагоналями параллелограмма. Ответ дайте в градусах.
		\item В треугольнике \( ABC \) известно, что \( AC=BC=27 \), \( AH  \) – высота, \( \cos\angle BAC=\dfrac{2}{3} \). Найдите \( BH \).
		\item Разность углов, прилежащих к одной стороне параллелограмма, равна \( 40\degree \). Найдите меньший угол параллелограмма. Ответ дайте в градусах.
		\item Высота \( BH \) ромба \( ABCD \) делит его сторону \( AD \) на отрезки \( AH  =  5 \) и \( HD  =  8 \). Найдите площадь и периметр ромба.
		\item Сторона ромба равна \( 5 \), а диагональ равна \( 6 \). Найдите площадь ромба.
	\end{listofex}
\end{class}
%END_FOLD

%BEGIN_FOLD % ====>>_____ Занятие 6 _____<<====
\begin{class}[number=6]
	\begin{listofex}
		\item Решите уравнение:
		\begin{tasks}(2)
			\task \( \dfrac{42}{x^{2}+5x}=\dfrac{7}{x} \)
			\task \( 7x + 8 +(2x+3) = (2-x) + 8 \)
			\task \( \dfrac{x+4}{x-2}=5 \)
			\task \( \dfrac{x-1}{x}=-\dfrac{1}{3} \)
		\end{tasks}
		\item В треугольнике \(ABC \angle C = 90\degree \), \( AC = 6 \), \( \cos A = 0,3 \). Найдите \( AB \), \( BC \).
		\item Найдите синус наименьшего угла треугольника если один катет равен \( 4\sqrt{2} \), а другой \( 2 \) .
		\item 
		\begin{minipage}[t]{\bodywidth}
			Найдите гипотенузу \( AB \) катет \( BC \) по данным рисунков.
		\end{minipage}
		\hspace{0.02\linewidth}
		\begin{minipage}[t]{\picwidth}
			\includegraphics[align=t, width=\linewidth]{\picpath/G81M7L6-2}
		\end{minipage}
		\item \begin{minipage}[t]{\bodywidth}
			Найдите синусы и косинусы углов \( B \) и \( C \).
		\end{minipage}
		\hspace{0.02\linewidth}
		\begin{minipage}[t]{\picwidth}
			\includegraphics[align=t, width=0.8\linewidth]{\picpath/G81M7L6-4}
		\end{minipage}
		\item В треугольнике \( ABC \) угол \( C \) прямой,
		\( AC=6 \), \( \cos A=0,6 \). Найдите \( AB \).
		\item В треугольнике \( ABC \) угол \( C \) равен \( 90\degree \), \( \sin A=\dfrac{4}{5} \), \( AC=9 \). Найдите \( AB \).
		\item В прямоугольном треугольнике \( ABC \) \( \angle C=90\degree \) \( \cos B=\dfrac{2}{7} \), \( AC=21 \). Найдите \( AB \).
		\item В прямоугольном треугольнике \( MNK \) \( \angle N = 90\degree \), \( \sin M = 0,6 \), \( MN=9 \). Найдите длину \( NK \).
		\item 
		\begin{minipage}[t]{\bodywidth}
			Найдите площадь треугольника, изображённого на
			рисунке.
		\end{minipage}
		\hspace{0.02\linewidth}
		\begin{minipage}[t]{\picwidth}
			\includegraphics[align=t, width=\linewidth]{\picpath/G81M7L6-3}
		\end{minipage}
	\end{listofex}
\end{class}
%END_FOLD

%BEGIN_FOLD % ====>>_ Домашняя работа 3 _<<====
\begin{homework}[number=3]
	\begin{listofex}
		\item \begin{tasks}(2)
			\task \(\dfrac{3x}{1,7}=\dfrac{0,21}{6,8}\)
			\task \( 2x(1+x)=3(x+5)	 \)
			\task \(\dfrac{x^2}{2}-3x-8=0\)
			\task \( -4x^{2} + 10x — 20 = 29-18x.\)
		\end{tasks}
		\item Вычислите синус и косинус углов, изображенных на рисунке:
		\begin{figure}[h!]
			\centering
			\includegraphics[width=0.8\linewidth]{../../../../../exercises/lists/pics/G81M7H3-1}
		\end{figure}
		\item В треугольнике \( ABC \) \( \angle C = 90\degree \), \( CH \) – высота. \( AC = 6 \), \( BH = 9 \). Найдите \( AB \), \( AH \), \( BC \), \( CH \).
		\item В треугольнике \( MNK \) угол \( K \) прямой,
		\( NK=6 \), \( \cos M=0,6 \). Найдите \( MK \).
		\item  В треугольнике \( ABC \) \(AB = BC = 12 \), \( \cos A = \dfrac{\sqrt{7}}{4} \) . Найдите \(AC \).
	\end{listofex}
\end{homework}
%END_FOLD

%BEGIN_FOLD % ====>>_____ Занятие 7 _____<<====
\begin{class}[number=7]
	\title{Подготовка к проверочной}
	\begin{listofex}
		\item Найдите гипотенузу прямоугольного треугольника, если его катеты равны \( 25 \) см и \( 60 \) см.
		\item Сторона ромба равна \( 10 \) см, а одна из его диагоналей – \( 16 \) см. Найдите вторую диагональ.
		\item  В треугольнике  \( ABC \) \( \angle B = 45° \), высота \( AN \) делит сторону \(BC \) на отрезки \( BN = 8 \) см и \( NC = 6 \) см. Найдите площадь треугольника \( ABC \) и сторону \( AC \).
		\item Диагонали ромба равны \( 12 \) см и \( 16 \) см. Найдите площадь и периметр ромба.
		\item В равнобедренном треугольнике боковая сторона равна \( 13 \) см, а высота, проведенная к основанию \( 5 \) см. Найдите площадь этого треугольника.
		\item  В треугольнике \( ABC \) \( \angle C = 90\degree \), \( AB = 13 \) см, \( AC = 5 \) см. Найдите \( \sin B \).
		\item Найдите гипотенузу прямоугольного треугольника \( ABC \), если угол \( C \) прямой, \( BC = 6 \) см, \( \cos B = \dfrac{3}{7} \).
		\item Найдите значение выражения \( \sin^{2} 37\degree + \cos^{2 } 37\degree – \sin^{2 } 45\degree \)
	\end{listofex}
\end{class}
%END_FOLD

%BEGIN_FOLD % ====>>_ Проверочная работа _<<====
\begin{exam}
	\begin{listofex}
		\item Найдите второй катет прямоугольного треугольника, если его гипотенуза \( 17 \) см, а другой катет \( 15 \) см.
		\item  Диагонали ромба равны \( 14 \) см и \( 48 \) см. Найдите сторону ромба.
		\item В треугольнике \( ABC \) \( \angle A = 30\degree \), \( \angle B = 75\degree \), высота \( BD \) равна \( 6 \) см. Найдите площадь треугольника \( ABC \).
		\item Диагональ прямоугольника равна \( 13 \) см, а одна из сторон – \( 5 \) см. Найдите площадь и периметр прямоугольника.
		\item В треугольнике \( ABC \) \( \angle B = 90\degree \), \( AC = 17 \) см, \( BC = 8 \) см. Найдите \( \cos C \).
		\item  Найдите гипотенузу прямоугольного треугольника \( MNK  \), если  \( \angle N = 90\degree \), \( MN = 10 \) см, \( \sin K = \dfrac{5}{9} \).		
		\item  Найдите значение выражения \( \cos^{2} 45\degree + \sin^{2} 74\degree + \cos^{2} 74\degree \).
	\end{listofex}
\end{exam}
%END_FOLD

%BEGIN_FOLD % ====>>_ Консультация _<<====
\begin{consultation}
	\begin{listofex}
		\item Постройте график \( y=|x| \) и определите: 
		\begin{tasks}(1)
			\task Промежутки возрастания функции;
			\task Промежутки убывания функции;
			\task Область определения функции;
			\task Область значений функции.
		\end{tasks}
		\item Постройте график \( y=-|x| \) и определите: 
		\begin{tasks}(1)
			\task Промежутки возрастания функции;
			\task Промежутки убывания функции;
			\task Область определения функции;
			\task Область значений функции.
		\end{tasks}
		\item Постройте график \( y=|x|-3 \)
	\end{listofex}
	\newpage
	\title{Домашняя работа}
	\begin{listofex}
		\item Постройте график \( y=|x|+1 \)
	\end{listofex}
\end{consultation}
%END_FOLD
%BEGIN_FOLD % ====>>_ Консультация _<<====
\begin{consultation}
	\begin{listofex}
		\item  На одном из рисунков изображена гипербола. Укажите этот рисунок.
		\begin{tasks}(2)
			\task\begin{minipage}[t]{\picwidth}
				\includegraphics[align=t, width=1.5\linewidth]{\picpath/MoisenkoConsultation17.03-1}
			\end{minipage}
			\task\begin{minipage}[t]{\picwidth}
				\includegraphics[align=t, width=1.5\linewidth]{\picpath/MoisenkoConsultation17.03-2}
			\end{minipage}
			\task\begin{minipage}[t]{\picwidth}
				\includegraphics[align=t, width=1.5\linewidth]{\picpath/MoisenkoConsultation17.03-3}
			\end{minipage}
			\task\begin{minipage}[t]{\picwidth}
				\includegraphics[align=t, width=1.5\linewidth]{\picpath/MoisenkoConsultation17.03-4}
			\end{minipage}
		\end{tasks}
		\item Постройте график функции \( y=-\dfrac{4}{x} \).
		\item Найдите точку пересечения графиков \( y=\dfrac{4}{x} \) и \( y=2 \) 
		\item Постройте график функции \( y = \dfrac{11}{x}+2 \)
		\item Постройте график функции \( y = \dfrac{7+2x}{x}+5 \).
		\item  Какая из указанных точек принадлежит графику функции \( y = -\dfrac{8}{x} \)?\begin{tasks}(3)
			\task \( A (1;8) \) 
			\task \(  B (-1;-8) \)
			\task \( C (1 ; -8) \) 
		\end{tasks} 
		\item Постройте график функции  \(y=\dfrac{ 2x+1}{2x^{2}+x} \).
	\end{listofex}
\end{consultation}
%END_FOLD