%
%===============>>  ГРУППА 8-1 МОДУЛЬ 7  <<=============
%
\setmodule{7}

%BEGIN_FOLD % ====>>_____ Занятие 1 _____<<====
\begin{class}[number=1]
	\begin{listofex}
		\item В прямоугольном треугольнике два катета равны \( 3 \) и \(  4 \). Найдите гипотенузу.
		\item Катеты прямоугольного треугольника равны \( 5 \) и \( 12 \). Чему равна гипотенуза?
		\item В прямоугольном треугольнике гипотенуза равна \( 20 \), а один из катетов равен \( 12 \). Чему равен другой катет?
		\item В прямоугольном треугольнике один из катетов равен \( 8 \), гипотенуза этого треугольника равна \( 17 \). Чему равен второй катет?
		\item Катет прямоугольного треугольника равен \( 7 \), а гипотенуза равна \( 25 \). Найдите длину второго катета.
		\item В равнобедренном прямоугольном треугольнике гипотенуза равна  \( 14\sqrt{2} \). Найдите длину катетов.
		\item Основание равнобедренного треугольника равно \( 16 \), боковая сторона равна \( 10 \). Чему равна высота проведенная к основанию этого треугольника?
		\item Боковая сторона равнобедренного треугольника равна \( 13 \), а длина основания равна \( 10 \). Найдите площадь этого треугольника.
		\item Найдите высоты равностороннего треугольника, если известно, что его сторона равна \( 8\sqrt{2} \).
		\item Дан параллелограмм \( ABCD \), где \( AB=15 \), \( AD=9 \). Также известно, что высота из точки \( B \) к основанию \( AD \) падает на точку \( D \). Чему равна площадь параллелограмма?
		\item Диагонали ромба равны \( 12 \) и \( 16 \) см. Чему равен периметр? Чему равна площадь? 
		\item Найдите все углы параллелограмма, если сумма двух из них равна \( 100^{\circ}\).
		\item В параллелограмме биссектриса угла \( А \) делит сторону \( ВС  \) на отрезки, равные \( 3 \) и \( 5  \) см. Найдите его периметр.
		\item Найдите углы параллелограмма, зная, что один из них больше другого на \( 50^{\circ} \).
	\end{listofex}
\end{class}
%END_FOLD

%BEGIN_FOLD % ====>>_____ Занятие 2 _____<<====
\begin{class}[number=2]
	\begin{listofex}
		\item Миша прошел от дома по направлению на восток \( 800 \) м. Затем повернул на север и прошел \( 600 \) м. На каком расстоянии (в метрах) от дома оказался Миша?
		\item Полина прошла от дома по направлению на запад \( 500 \) м. Затем повернула на север и прошла \( 300 \) м. После этого она повернула на восток и прошла еще \( 100 \) м. На каком расстоянии (в метрах) от дома оказалась Полина?
		\item Точка крепления троса, удерживающего флагшток в вертикальном положении, находится на высоте \( 15 \) м от земли. Расстояние от основания флагштока до места крепления троса на земле равно \( 8 \) м. Найдите длину троса.
		\item Пожарную лестницу длиной \( 13 \) м приставили к окну пятого этажа дома. Нижний конец лестницы отстоит от стены на \( 5 \) м. На какой высоте расположено окно? Ответ дайте в метрах.
		\item Высота равнобедренного треугольника равна \( 20 \) см, а его основание --- \( 10 \) см. Найдите его боковую сторону.
		\item Найдите площадь и периметр прямоугольника, сторона которого равна \( 9 \) см, а диагональ --- \( 15 \) см.
		\item Дан параллелограмм \( ABCD \), где \( AB=15 \), \( AD=9 \). Также известно, что высота из точки \( B \) к основанию \( AD \) падает на точку \( D \). Чему равна площадь параллелограмма?
		\item Диагонали ромба равны \( 12 \) и \( 16 \) см. Чему равен периметр? Чему равна площадь? 
		\item Найдите все углы параллелограмма, если сумма двух из них равна \( 100^{\circ}\).
		\item В параллелограмме биссектриса угла \( А \) делит сторону \( ВС  \) на отрезки, равные \( 3 \) и \( 5  \) см. Найдите его периметр.
		\item Найдите углы параллелограмма, зная, что один из них больше другого на \( 50^{\circ} \).
	\end{listofex}
\end{class}
%END_FOLD

%BEGIN_FOLD % ====>>_ Домашняя работа 1 _<<====
\begin{homework}[number=1]
	\begin{listofex}
		\item Стороны прямоугольника имеют длину \( 8 \) и \( 15 \) см. Найдите длину его диагонали.
		\item Пожарную лестницу приставили к окну, расположенному на высоте \( 12 \) м от земли. Нижний конец лестницы отстоит от стены на \( 5 \) м. Какова длина лестницы? Ответ дайте в метрах.
		\item Точка крепления троса, удерживающего флагшток в вертикальном положении, находится на высоте \( 6,3 \) м от земли. Расстояние от основания флагштока до места крепления троса на земле равно \( 1,6 \) м. Найдите длину троса в метрах.
		\item Один из острых углов прямоугольного треугольника составляет \( 30\degree \), а его гипотенуза равна \( 10 \). Найдите оба катета.
		\item Решите квадратное уравнение:
		\[x(x-5)=1-4x\]
	\end{listofex}
\end{homework}
%END_FOLD

%BEGIN_FOLD % ====>>_____ Занятие 3 _____<<====
\begin{class}[number=3]
	\begin{listofex}
		\item Занятие 3 
	\end{listofex}
\end{class}
%END_FOLD

%BEGIN_FOLD % ====>>_____ Занятие 4 _____<<====
\begin{class}[number=4]
	\begin{listofex}
		\item Занятие 4
	\end{listofex}
\end{class}
%END_FOLD

%BEGIN_FOLD % ====>>_ Домашняя работа 2 _<<====
\begin{homework}[number=2]
	\begin{listofex}
		\item Домашняя работа 2
	\end{listofex}
\end{homework}
%END_FOLD

%BEGIN_FOLD % ====>>_____ Занятие 5 _____<<====
\begin{class}[number=5]
	\begin{listofex}
		\item Занятие 5
	\end{listofex}
\end{class}
%END_FOLD

%BEGIN_FOLD % ====>>_____ Занятие 6 _____<<====
\begin{class}[number=6]
	\begin{listofex}
		\item Занятие 6
	\end{listofex}
\end{class}
%END_FOLD

%BEGIN_FOLD % ====>>_ Домашняя работа 3 _<<====
\begin{homework}[number=3]
	\begin{listofex}
		\item Домашняя работа 3
	\end{listofex}
\end{homework}
%END_FOLD

%BEGIN_FOLD % ====>>_____ Занятие 7 _____<<====
\begin{class}[number=7]
	\title{Подготовка к проверочной}
	\begin{listofex}
		\item Занятие 7
	\end{listofex}
\end{class}
%END_FOLD

=%BEGIN_FOLD % ====>>_ Проверочная работа _<<====
\begin{exam}
	\begin{listofex}
		\item Проверочная
	\end{listofex}
\end{exam}
%END_FOLD

%BEGIN_FOLD % ====>>_ Консультация _<<====
\begin{consultation}
	\begin{listofex}
		\item Постройте график \( y=|x| \) и определите: 
		\begin{tasks}(1)
			\task Промежутки возрастания функции;
			\task Промежутки убывания функции;
			\task Область определения функции;
			\task Область значений функции.
		\end{tasks}
		\item Постройте график \( y=-|x| \) и определите: 
		\begin{tasks}(1)
			\task Промежутки возрастания функции;
			\task Промежутки убывания функции;
			\task Область определения функции;
			\task Область значений функции.
		\end{tasks}
		\item Постройте график \( y=|x|-3 \)
	\end{listofex}
	\newpage
	\title{Домашняя работа}
	\begin{listofex}
		\item Постройте график \( y=|x|+1 \)
	\end{listofex}
\end{consultation}
%END_FOLD