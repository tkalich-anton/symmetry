%
%===============>>  ГРУППА 8-1 МОДУЛЬ 5  <<=============
%
\setmodule{5}
%BEGIN_FOLD% ====>>_____ Занятие 1 _____<<====
\begin{class}[number=1]
	\begin{listofex}
		\item Известно, что периметр ромба \( ABCD \) равен \( 120 \). Чему равна сторона ромба?
		\item В прямоугольнике одна сторона больше другой на \( 3 \). Чему равны стороны прямоугольника, если периметр равен \( 50 \)?
		\item В параллелограмме \( ABCD \). Биссектриса, проведенная из вершины \( A \), пересекает сторону \( BC \) в точке \( M \). Найдите периметр параллелограмма \( ABCD \), если \( BM=4 \) и \( MC=3 \).
		\item В прямоугольнике \( ABCD \) \( AB=7 \). Биссектриса, проведенная из вершины \( A \), проходит через вершину \( C \). Найдите площадь прямоугольника \( ABCD \).
		\item В параллелограмме \( ABCD \) проведены биссектрисы из вершин \( A \) и \( B \), которые пересекаются в точке \( O \). Чему равен угол \( \angle AOB \), если \( \angle B= 40 \)?
		\item Биссектриса внутреннего угла при вершине A и
		биссектриса внешнего угла при вершине \( C \) треугольника \( ABC \)
		пересекаются в точке \( M \). Найдите \( \angle BMC \), если \( \angle BAC = 40\degree \) и \( \angle ABC = 60\degree \).
		\item Решить уравнение:
		\begin{tasks}(2)
			\task \( 2x-(0,5x+3)=x-5 \)
			\task \( x^2-7x+10=0 \)
		\end{tasks}
	\end{listofex}
\end{class}
%END_FOLD

%BEGIN_FOLD % ====>>_____ Занятие 2 _____<<====
\begin{class}[number=2]
	\begin{listofex}
			\item Решить уравнение:
		\begin{tasks}(2)
			\task \( 2x-(0,5x+3)=x-5 \)
			\task \( x^2-7x+10=0 \)
		\end{tasks}
		\item Биссектриса внутреннего угла при вершине A и биссектриса внешнего угла при вершине \( C \) треугольника \( ABC \)
		пересекаются в точке \( M \). Найдите \( \angle BMC \), если \( \angle BAC = 40\degree \) и \( \angle ABC = 60\degree \).
		\item Докажите, что если в параллелограмме две смежные стороны равны, то такой параллелограмм --- ромб.
		\item Докажите, что если в четырехугольнике все стороны равны, то такой четырехугольник --- ромб.
		\item Угол при вершине \( A \) ромба \( ABCD \) равен \( 20\degree \). Точки \( M \) и \( N \) --- основания перпендикуляров, опущенных из вершины \( B \) на стороны \( AD \) и \( CD \). Найдите углы треугольника \( BMN \).
		\item Докажите, что если в параллелограмме один из углов равен \( 90\degree \), то такой параллелограмм- прямоугольник.
	\end{listofex}
\end{class}
%END_FOLD

%BEGIN_FOLD % ====>>_ Домашняя работа 1 _<<====
\begin{homework}[number=1]
	\begin{listofex}
		\item Докажите, что если в параллелограмме диагонали перпендикулярны, то такой параллелограмм --- ромб.
		\item Найдите расстояние от центра ромба до его стороны, если острый его угол равен \( 30\degree \), а сторона равна \( 4 \).
		\item Угол при вершине \( A \) ромба \( ABCD \) равен \( 60\degree \). На сторонах \( AB \) и \( BC \) взяты соответственно точки \( M \) и \( N \), причем \( AMBN \). Докажите, что треугольник DMN равносторонний.
		\item Докажите, что если в четырехугольнике три угла равны \( 90\degree \), то такой четырехугольник --- прямоугольник.
		\item \exercise{947}
	\end{listofex}
\end{homework}
%END_FOLD

%BEGIN_FOLD % ====>>_____ Занятие 3 _____<<====
\begin{class}[number=3]
	\begin{listofex}
		\item Диагонали прямоугольника равны \( 8 \) и пересекаются под углом \( 60\degree \). Найдите меньшую сторону прямоугольника.
		\item Найдите углы ромба, если его диагонали составляют с его стороной углы, один из которых равен \( 30\degree \), а второй на \( 30\degree \) больше первого.
		\item В ромбе \( ABCD \) биссектриса угла \( BAC \) пересекает сторону \( BC \) и диагональ \( BD \) соответственно в точках \( M \) и \( N \). Найдите угол \( ANB \), если \( \angle AMC=120\degree \)
		\item Периметр  прямоугольника равен \( 32 \), а площадь \( 28 \). Найдите большую сторону прямоугольника.
		\item В прямоугольнике диагональ делит угол в отношении \( 1:2 \), меньшая его сторона равна \( 41 \). Найдите диагональ данного прямоугольника.
		\item В ромбе \( ABCD \)  угол \( DAB \)  равен \( 36\degree \). Найдите угол \( BDC \). Ответ дайте в градусах.
	\end{listofex}
\end{class}
%END_FOLD

%BEGIN_FOLD % ====>>_____ Занятие 4 _____<<====
\begin{class}[number=4]
	\begin{listofex}
		\item В треугольнике \( ABC \) высота \( CD=15 \), \( AB=22 \). Найдите площадь треугольника \( ABC \).
		\item Периметр треугольника \( ABC \) равен \( 14 \), \( AB=6 \), \( BC=5 \). Высота \( BH=4 \). Найдите площадь треугольника.
		\item В прямоугольном треугольнике \( ABC \quad \angle C=90\degree\), \( AC=9 \), \( BC=4 \). Найдите площадь треугольника \( ABC \).
		\item В равнобедренном треугольнике \( ABC \quad \angle B=60\degree \), \( AB=AC=8 \). Найдите площадь треугольника \( ABC \).
		\item В треугольнике \( ABC \) \( CD \) --- высота, \( AD=12 \), \( DB=16 \), \( \angle ACD=45\degree \). Найдите площадь треугольника \( ABC \).
		\item В прямоугольном треугольнике \( ABC  \angle C=90\degree\), медиана \( CD=10 \), \( CB=16 \). Найдите площадь треугольника \( ABC \).
		\item В треугольнике \( ABC \) проведена биссектриса \( CK=9 \), \( AK=KB=7 \), \( CB=14 \). Найдите площадь треугольника \( ABC \).
		\item Основание параллелограмма равно \( 12 \), а высота --- \( 7 \). Найдите его площадь.
		\item Площадь параллелограмма равна \( 36 \) дм\( ^2 \), высота --- \( 4 \) дм. Найдите сторону параллелограмма, к которой проведена высота.
		\item В параллелограмме \( ABCD \) \( AB=12 \), высоты \( DH \) и \( DE \) равны \( 5 \) и \( 10 \) соответственно. Найдите \( BC \).
		\item Высота \( BE \) делит сторону \( AD \) параллелограмма \( ABCD \) на отрезки, равные \( 6 \) и \( 3 \). \( AB=8 \), \( \angle A=30\degree \). Найдите площадь параллелограмма.
	\end{listofex}
\end{class}
%END_FOLD

%BEGIN_FOLD % ====>>_ Домашняя работа 2 _<<====
\begin{homework}[number=2]
	\begin{listofex}
		\item В ромбе \( ABCD \)  угол \( DAB \)  равен \( 36\degree \). Найдите угол \( BDC \). Ответ дайте в градусах.
		\item В прямоугольном треугольнике один из катетов равен \( 10 \), а угол, лежащий напротив него, равен \( 45\degree \). Найдите площадь треугольника.
		\item Периметр равнобедренного треугольника равен \( 16 \), а боковая сторона --- \( 5 \). Найдите площадь треугольника.
		\item Одна из сторон параллелограмма равна \( 12 \), а опущенная на нее высота равна \( 10 \). Найдите площадь параллелограмма.
		\item Периметр параллелограмма равен \( 40 \), меньшая сторона равна \( 4 \). Найдите площадь параллелограмма, если его острый угол равен \( 60\degree \).
	\end{listofex}
\end{homework}
%END_FOLD

%BEGIN_FOLD % ====>>_____ Занятие 5 _____<<====
\begin{class}[number=5]
	\begin{listofex}
		\item Решить уравнения:
		\begin{tasks}(2)
			\task \( (x+2)(-x+6)=0 \)
			\task \( (-5x+3)(-x-6)=0 \)
			\task \( 2x^2-10x=0 \)
			\task \( 4x^2+x=0 \)
			\task \( x^2-16=0 \)
			\task \( x^2-144=0 \)
		\end{tasks}
		\item Решить уравнения:
		\begin{tasks}(2)
			\task \( x^2+7x-18=0 \)
			\task \( x^2+4=5x \)
			\task \( \dfrac{5}{4}x^2+7x+9=0 \)
			\task \( (x+10)^2=(5-x)^2 \)
		\end{tasks}
	
		\item В ромбе \( ABCD \) \( \angle A=140\degree \), а диагонали пересекаются в точке \( O \). Чему равны углы треугольника \( AOB \)?
		\item Найдите углы ромба, если его сторона образует с диагоналями углы, разность которых равна \( 18\degree \).
		\item Периметр параллелограмма \( ABCD \) равен \( 50 \) см, \( \angle C = 30\degree \), а перпендикуляр \( BH \) к прямой \( CD \) равен \( 6,5 \) см. Найдите стороны параллелограмма.
		\item Площадь параллелограмма равна \( 40 \), а две его стороны равны \( 5 \) и \( 10 \). Найдите его высоты. В ответе укажите большую высоту.
	\end{listofex}
\end{class}
%END_FOLD

%BEGIN_FOLD % ====>>_____ Занятие 6 _____<<====
\begin{class}[number=6]
	\begin{listofex}
		\item Решить уравнения:
		\begin{tasks}(1)
		\task \( -3x^2+4x-7=-x^2+5x-(-1+2x^2) \)
		\task \( (x+2)^2+(x-3)^2=2x^2 \)
		\end{tasks}
		\item Найдите  площадь ромба  \( ABCD \), если \( AO = 4  \) см, \( BO = 2,5 \) см, где  \( O \) --- точка пересечения диагоналей ромба.
		\item Найти диагональ \( PF \) ромба \( KPRF \), если \( KR =4 \) см, \( S_{KPRF}=28 \) cм\( ^2 \).
		\item \exercise{1508}
		\item Примените ФСУ:
		\begin{tasks}(2)
			\task \( (2a+4)^2 \)
			\task \( (3b-12c)^2 \)
			\task \( (8a^3+6b^4)^2 \)
			\task \( (\dfrac{3}{4}x^5-\dfrac{1}{2})^2 \)
			\task \( (0,2z^{16}+0,09y^{12})^2 \)
			\task \( (\mfrac{1}{3}{5}p^6-\mfrac{3}{8}{9}q^4) \)
		\end{tasks}
	\end{listofex}
\end{class}
%END_FOLD


%BEGIN_FOLD % ====>>_ Домашняя работа 3 _<<====
\begin{homework}[number=3]
	\begin{listofex}
		\item Решить уравнения:
		\begin{tasks}(1)
			\task \( x^2-x-6=0 \)
			\task \( x^2+2x-24=0 \)
			\task \( x^2=-12x-32 \)
			\task \( x^2=4x+45 \)
		\end{tasks}
		\item Сколько понадобится краски, чтобы покрасить панно, которое имеет форму ромба с диагоналями \( 5 \) дм и \( 3,4 \) дм, если на \( 1 \) дм\( ^2 \) расходуется \( 2 \) г краски?
		\item Сократить дробь:
		\begin{itasks}[2]
			\task \exercise{73}
			\task \exercise{76}
		\end{itasks}
	\end{listofex}
\end{homework}
%END_FOLD

%BEGIN_FOLD % ====>>_____ Занятие 7 _____<<====
\begin{class}[number=7]
	\begin{listofex}
		\item Одна из сторон параллелограмма равна \( 12 \), а высота, опущенная на неё --- \( 6 \). Найдите площадь параллелограмма.
		\item Одна из сторон параллелограмма в \( 3 \) раза меньше другой, а его периметр равен \( 72 \) см. Найдите стороны параллелограмма.
		\item Биссектриса угла \( D \) параллелограмма \( ABCD \) пересекает сторону \( BC \) в точке \( M \), \( BM:MC=4:3 \). Найдите периметр параллелограмма, если \( BC=28 \) см.
		\item Один из углов ромба равен \( 64\degree \). Найдите углы, которые образует сторона ромба с его диагоналями.
		\item Диагонали прямоугольника \( ABCD \) пересекаются в точке \( O \), \( CD=15 \) см, \( AC=20 \) см. Найдите периметр треугольника \( AOB \).
		\item  Площадь параллелограмма равна \( 32 \), а две его стороны равны \( 8 \) и \( 16 \). Найдите его высоты. В ответе укажите большую высоту.
		\item Решите уравнения:
		\begin{tasks}(2)
			\task \( 6x^2+x-2=0 \)
			\task \( (3x+1)^2=(x+2)^2 \)
		\end{tasks}
		\item Упростите выражение: \quad \( \left( \dfrac{y^2-xy}{x^2+xy}-xy+y^2 \right)\cdot\dfrac{x}{x-y}+\dfrac{y}{x+y} \)
	\end{listofex}
\end{class}
%END_FOLD

%BEGIN_FOLD % ====>>_ Проверочная работа _<<====
\begin{exam}
	\begin{listofex}
		\item Найдите площадь параллелограмма, если одна из его сторон равна \( 30 \), а высота, опущенная на неё, --- \( 16 \).
		\item Одна из сторон параллелограмма на \( 7 \) см меньше другой, а его периметр равен \( 54 \) см. Найдите стороны параллелограмма.
		\item Биссектриса угла \( B \) параллелограмма \( ABCD \) пересекает сторону \( AD \) в точке \( K \), \( AK:KD=3:2 \). Найдите периметр параллелограмма, если \( AB=12 \) см.
		\item Сторона ромба образует с одной из его диагоналей угол \( 18\degree \). Найдите углы ромба.
		\item Диагонали прямоугольника \( KLMN \) пересекаются в точке \( O \), \( KN=18 \) см, \( LN=22 \) см. Найдите периметр треугольника \( LOM \).
		\item Стороны параллелограмма равны \( 32 \) и \( 64 \). Высота, опущенная на первую сторону, равна \( 48 \). Найдите высоту, опущенную на вторую сторону параллелограмма.
		\item Решите уравнения:
		\begin{tasks}(2)
			\task \( 9x^2+3x-2=0 \)
			\task \( (4x+3)^2=(2x-1)^2 \)
		\end{tasks}
		\item Упростите выражение и приведите подобные: \\ \[ (a+1)(a-1)(a^2-1)-(9+a^2)^2 \]
	\end{listofex}
\end{exam}
%END_FOLD

%BEGIN_FOLD % ====>>_ Консультация _<<====
\begin{consultation}
	\begin{listofex}
		\item Периметр параллелограмма \( ABCD \) равен \( 50 \) см, \( \angle C = 30\degree \), а перпендикуляр \( BH \) к прямой \( CD \) равен \( 6,5 \) см. Найдите стороны параллелограмма.
		\item Площадь параллелограмма равна \( 40 \), а две его стороны равны \( 5 \) и \( 10 \). Найдите его высоты. В ответе укажите большую высоту.
		\item \( ABCD \) --- квадрат, \( AC \) и \( BD \) --- диагонали. \( AC = 4  \) см, \( BC \) --- диагональ квадрата \( OBKC \).  Найдите \( BK \).
		\item Треугольник \( ABC \) --- равнобедренный, \( \angle A=90\degree \). Внутри него находится квадрат \( AB_1DC_1 \); \( AB = 2 \) см. Найдите периметр квадрата \( AB_1DC_1  \).
	\end{listofex}
\end{consultation}
%END_FOLD

%BEGIN_FOLD % ====>>_ Консультация _<<====
\begin{consultation}
	\begin{listofex}
		\item Найдите периметр четырехугольника с вершинами в серединах сторон прямоугольника с диагональю, равной \( 8 \).
		\item Найдите все углы параллелограмма, если сумма двух из них равна \( 110\degree \).
		\item Угол при вершине \( A \) ромба \( ABCD \) равен \( 20\degree \). Точки \( M \) и \( N \) --- основания перпендикуляров, опущенных из вершины \( B \) на стороны \( AD \) и \( CD \). Найдите углы треугольника \( BMN \).
		\item Докажите, что концы двух различных диаметров окружности являются вершинами прямоугольника.
		\item \exercise{1313}
	\end{listofex}
\end{consultation}
%END_FOLD