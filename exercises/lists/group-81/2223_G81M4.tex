%
%===============>>  ГРУППА 8-1 МОДУЛЬ 4  <<=============
%
\setmodule{4}
%
%===============>>  Занятие 1  <<===============
%
\begin{class}[number=1]
	\begin{listofex}
		\item Вычислить рациональным способом:
		\begin{enumcols}[itemcolumns=3]
			\item \( \sqrt{16+4\cdot4\cdot24} \)
			\item \( \sqrt{83^3\cdot2^2-83^2\cdot2^3} \)
			\item \( \sqrt{50^2-4\cdot7\cdot7} \)
		\end{enumcols}
		\item Решить уравнение:
		\begin{enumcols}[itemcolumns=4]
			\item \( 81x^2-16=0 \)
			\item \( 5x^2-25=0 \)
			\item \( -7x^2=-1 \)
			\item \( 50-2x^2=0 \)
		\end{enumcols}
		\item Решить уравнение:
		\begin{enumcols}[itemcolumns=3]
			\item \( 7x^2=-x \)
			\item \( x=x^2 \)
			\item \( 11x-5x^2 \)
		\end{enumcols}
		\item Решить уравнение:
		\begin{enumcols}[itemcolumns=2]
			\item \( (15-x)(x-2)=(x-6)(x+5) \)
			\item \( (x-1)(x-2)+(x+4)(x-4)+3x=0 \)
		\end{enumcols}
		\item Решить уравнение:
		\begin{enumcols}[itemcolumns=3]
			\item \( (x+1)^2=9 \)
			\item \( (2x-1)^2=64 \)
			\item \( (7x+4)^2=225 \)
			\item \( (10x+10)^2=10000 \)
			\item \( (2x-3)^2=121 \)
			\item \( \left( \dfrac{1}{2}x+\dfrac{2}{3} \right)^2=\dfrac{4}{9} \)
			\item \( (x+0,4)^2=2,89 \)
			\item \( (2,5x-10)^2=0,25 \)
			\item \( (0,2x+4,1)^2=1,21 \)
		\end{enumcols}
		\item Решить уравнение:
		\begin{enumcols}[itemcolumns=2]
			\item \( (x-7)^2=3 \)
			\item \( (x+5)^2=5 \)
		\end{enumcols}
		\item Решить уравнение:
		\begin{enumcols}[itemcolumns=3]
			\item \( x^2-10x+25=0 \)
			\item \( 4x^2+20x+25=0 \)
			\item \( 16x^2-24x+9=0 \)
		\end{enumcols}
		\item Решить уравнение:
		\begin{enumcols}[itemcolumns=3]
			\item \( x^2-12x+36=4 \)
			\item \( 4x^2+40x+100=81 \)
			\item \( 9x^2-60x+100=25 \)
		\end{enumcols}
		\item Решить уравнение:
		\begin{enumcols}[itemcolumns=3]
			\item \( x^2-18x+77=0 \)
			\item \( x^2-10x-39=0 \)
			\item \( x^2-22x+72=0 \)
		\end{enumcols}
		\item Решить уравнение: \( (2x-3)^2-(x-5)(x+5)=2(2x+7) \)
	\end{listofex}
\end{class}
%
%===============>>  Занятие 2  <<===============
%
\begin{class}[number=2]
	\begin{listofex}
		\item Вычислить рациональным способом:
		\begin{enumcols}[itemcolumns=3]
			\item \( \sqrt{2^2+4\cdot15} \)
			\item \( \sqrt{90^2-4\cdot25\cdot81} \)
			\item \( \sqrt{4^2+4\cdot5\cdot12} \)
		\end{enumcols}
	\item Решить уравнение:
	\begin{enumcols}[itemcolumns=3]
		\item \exercise{392}
		\item \exercise{397}
		\item \exercise{391}
		\item \exercise{400}
		\item \exercise{393}
		\item \exercise{396}
	\end{enumcols}
	\item Решить уравнение:
	\begin{enumcols}[itemcolumns=3]
		\item \exercise{404}
		\item \exercise{416}
		\item \exercise{418}
		\item \exercise{421}
		\item \exercise{424}
		\item \exercise{425}
	\end{enumcols}
	\item Решить уравнение:
	\begin{enumcols}[itemcolumns=3]
		\item \( 9x^2+24x+16=0 \)
		\item \( 36x^2-60x+25=0 \)
		\item \( 100y^2-1y+\dfrac{1}{4}=0 \)
		\end{enumcols}
	\item Решить уравнение:
	\begin{enumcols}[itemcolumns=3]
		\item \( x^2-12x+36=4 \)
		\item \( 4x^2+40x+100=81 \)
		\item \( 9x^2-60x+100=25 \)
	\end{enumcols}
	\item Решить уравнение:
	\begin{enumcols}[itemcolumns=3]
		\item \( x^2-18x+77=0 \)
		\item \( x^2-10x-39=0 \)
		\item \( x^2-22x+72=0 \)
	\end{enumcols}
	\item Решить уравнение: \( (2x-3)^2-(x-5)(x+5)=2(2x+7) \)
	\end{listofex}
\end{class}
%
%===============>>  Домашняя работа 1  <<===============
%
\begin{homework}[number=1]
	\begin{listofex}
		\item Решить уравнение:
		\begin{enumcols}[itemcolumns=3]
			\item \exercise{388}
			\item \exercise{400}
			\item \exercise{395}
			\item \exercise{404}
			\item \exercise{415}
			\item \exercise{423}
		\end{enumcols}
		\item \exercise{4139}
		\item Решить уравнение:
			\begin{enumcols}[itemcolumns=3]
			\item \( 4y^2-12y+9=0 \)
			\item \( 9z^2-60z+100=0 \)
			\item \( 64x^2+48x+9=0 \)
		\end{enumcols}
		\item Решить уравнение:
			\begin{enumcols}[itemcolumns=3]
			\item \( (2x+1)^2=0 \)
			\item \( \left( \dfrac{1}{3}-x^2 \right)=9 \)
			\item \( (5x-x^2)=36 \)
		\end{enumcols}
	 \item От листа жести, имеющего форму квадрата, отрезали полосу шириной \( 3 \) см. Площадь его оставшейся части равна \( 10 \) см\( ^2 \) . Найдите первоначальные размеры листа жести.
	 \item Решить уравнение: \( (2x-3)(x+1)+(x-6)(x+6)+x=0 \)
	\end{listofex}
\end{homework}
%
%===============>>  Занятие 3  <<===============
%
\begin{class}[number=3]
	\begin{listofex}
		\item Решить уравнение:
		\begin{enumcols}[itemcolumns=2]
			\item \( 2x^2-5x-3=0 \)
			\item \( 5x^2+9x+4=0 \)
			\item \( 36x^2-12x+1=0 \)
			\item \( 3x^2-3x+1=0 \)
			\item \( 25=26x-x^2 \)
			\item \( x^2=4x+96 \)
			\item \( -x^2=54x-14 \)
			\item \( 6x+9=x^2 \)
			\item \( 3x^2+3=10x \)
		\end{enumcols}
		\item Решить уравнение:
		\begin{enumcols}[itemcolumns=2]
			\item \( (3x-1)(x+3)=x(1+6x) \)
			\item \( -x(x+7)=(x-2)(x+2) \)
		\end{enumcols}
		\item Решить уравнение:
		\begin{enumcols}[itemcolumns=2]
			\item \( (x+4)^2=3x+40 \)
			\item \( 3(x+4)^2=10x+32 \)
			\item \( (x+1)^2=(2x-1)^2 \)
			\item \( (x-2)^2+48=(2-3x)^2 \)
		\end{enumcols}
	\end{listofex}
\end{class}
%
%===============>>  Занятие 4  <<===============
% смещение на одно занятие с прошлого месяца
\begin{class}[number=4]
	\begin{listofex}
		\item Решить уравнение:
		\begin{enumcols}[itemcolumns=2]
			\item \( 3x^2-4x-7=0 \)
			\item \( 6x^2+3x+1=0 \)
			\item \( 11x^2-9x-1=0 \)
			\item \( x^2+12x+36=0 \)
		\end{enumcols}
		\item Решить уравнение:
		\begin{enumcols}[itemcolumns=2]
			\item \( -x(x+7)=(x-2)(x+2) \)
			\item \( (3x-1)(x+3)=x(1+6x) \)
		\end{enumcols}
		\item Решить уравнение:
		\begin{tasks}(2)
			\task \( (x+4)^2=3x+40 \)
			\task \( 3(x+4)^2=10x+32 \)
			\task \( (x+1)^2=(2x-1)^2 \)
			\task \( (x-2)^2+48=(2-3x)^2 \)
		\end{tasks}
		\item Найти два последовательных натуральных числа, произведение которых равно \( 210 \).
	\end{listofex}
\end{class}
%
%===============>>  Домашняя работа 2  <<===============
%
\begin{homework}[number=2]
	\begin{listofex}
		\item Решите уравнение:
		\begin{itasks}[2]
			\task \exercise{459}
			\task \exercise{460}
			\task \exercise{457}
			\task \exercise{483}
		\end{itasks}
		\item Решите уравнение:
		
		\prompt{Перемножить скобки и перенести все слагаемые в левую сторону. Далее привести подобные слагаемые и получить квадратное уравнение.}
		\begin{enumcols}[itemcolumns=2]
			\item \exercise{493}
			\item \exercise{3669}
		\end{enumcols}
	\item Докажите, что если в четырехугольнике противолежащие углы попарно равны, то такой четырехугольник – параллелограмм.
	\item Сторона \( BC \) параллелограмма \( ABCD \) вдвое больше стороны \( AB \). Биссектрисы углов \( A \) и \( B \) пересекают прямую \( CD \) в точках \( M \) и \( N \), причем \( MN=12 \). Найдите стороны параллелограмма.
	\item Через центр параллелограмма \( ABCD \) проведены две прямые. Одна из них пересекает стороны \( АВ \) и \( CD \) соответственно в точках \( М \) и \( К \), вторая -- стороны \( ВС \) и \( AD \) соответственно в точках \( N \) и \( L \). Докажите, что четырехугольник \( MNKL \) -- параллелограмм.
	\end{listofex}
\end{homework}
%
%===============>>  Занятие 5  <<===============
\begin{class}[number=5]
	\begin{center}
		СВОЙСТВА ПАРАЛЛЕЛОГРАММА 
	\end{center}
	\begin{tasks}(1)
		\task Противоположные стороны параллелограмма параллельны и равны;
		\task Противоположные углы параллелограмма равны;
		\task Диагонали параллелограмма делятся точкой пересечения пополам;
		\task Сумма углов, прилежащих к одной стороне параллелограмма, равна \( 180\degree \).
	\end{tasks}
	\begin{listofex}
		\item Докажите, что если в четырехугольнике диагонали пересекаются и делятся точкой пересечения пополам, то такой четырехугольник --- параллелограмм.
		\item Докажите, что если в четырехугольнике противолежащие стороны попарно равны, то такой четырехугольник --- параллелограмм.
%		\item Докажите, что если в четырехугольнике противоположные углы равны, то такой четырехугольник --- параллелограмм.
		\item Сторона параллелограмма втрое больше другой стороны, а его периметр равен \( 24 \).
		Найдите его стороны.
		\item Биссектриса угла параллелограмма делит его сторону на отрезки \( 3 \) и \( 4 \).
		Найдите стороны параллелограмма.
		\item Высота параллелограмма, проведенная из вершины тупого угла, равного \( 150\degree \), равна двум и делит сторону параллелограмма пополам. Найдите диагональ, проведенную из вершины тупого угла, и углы, которые она образует со сторонами.
		\item Решите уравнение:
		\begin{enumcols}[itemcolumns=2]
			\item \exercise{414}
			\item \exercise{413}
			\item \exercise{434}
			\item \exercise{435}
		\end{enumcols}
	\end{listofex}
\end{class}
%
%===============>>  Занятие 6  <<===============
%
\begin{class}[number=6]
	\begin{listofex}
		\item Докажите, что если в треугольнике один угол равен сумме двух других углов, то такой треугольник будет прямоугольным.
		\item Докажите, что если в четырехугольнике противоположные углы равны, то два соседних угла в сумме будут составлять \( 180\degree \).
		\item Докажите, что если в четырехугольнике суммы углов, прилежащих к одной стороне четырехугольника равны по \( 180\degree \), то такой четырехугольник --- параллелограмм.
		\item В параллелограмме \( ABCD \) два противоположных угла в сумме равны \( 140\degree \). Найдите два других угла.
		\item В параллелограмме \( ABCD \) одна сторона в два раза больше соседней стороны. Найдите длины сторон параллелограмма \( ABCD \), если его периметр равен \( 36 \).
		\item Известно, что в параллелограмме \( ABCD \) все стороны равны. Угол \( BAC \) равен \( 60\degree \). Найдите периметр параллелограмма \( ABCD \), если периметр треугольника \( ABD \) равен \( 18 \).
		\item В треугольнике \( АВС \) медиана \( АМ \) продолжена за точку \( М \) до точки \( D \) на расстояние, равное \( AM \) (так что \( AM=MD\)). Докажите, что \( ABCD \) --- параллелограмм.
	\end{listofex}
\end{class}
%===============>>  Домашняя работа 3  <<===============
%
%\begin{homework}[number=2]
%	\begin{listofex}
%
%	\end{listofex}
%\end{homework}
%\newpage
%\title{Подготовка к проверочной работе}
%\begin{listofex}
%	
%\end{listofex}
%
%===============>>  Занятие 7  <<===============
%
\begin{class}[number=7]
	\begin{listofex}
		\item Решить уравнение:
		\begin{enumcols}[itemcolumns=2]
			\item \exercise{414}
			\item \exercise{413}
			\item \exercise{434}
			\item \exercise{435}
		\end{enumcols}
		\item Решить уравнение:
		\begin{enumcols}[itemcolumns=2]
			\item \exercise{458}
			\item \exercise{460}
			\item \exercise{461}
			\item \exercise{468}
		\end{enumcols}
		\item Докажите, что если в четырехугольнике противоположные углы равны, то такой четырехугольник --- параллелограмм.
		\item Сторона параллелограмма втрое больше другой стороны, а его периметр равен \( 24 \).
		Найдите его стороны.
		\item Биссектриса угла параллелограмма делит его сторону на отрезки \( 3 \) и \( 4 \).
		Найдите стороны параллелограмма.
		\item Высота параллелограмма, проведенная из вершины тупого угла, равного \( 150\degree \), равна двум и делит сторону параллелограмма пополам. Найдите диагональ, проведенную из вершины тупого угла, и углы, которые она образует со сторонами.
	\end{listofex}
\end{class}
%
%===============>>  Провечная работа  <<===============
%
\begin{exam}
	\begin{listofex}
		\item Решить уравнение:
		\begin{tasks}(2)
			\task \( x^2-14x+33=0 \)
			\task \( x^2-11x=42 \)
		\end{tasks}
		\item Стороны прямоугольника равны \( 12 \) и \( 20 \). На сколько площадь прямоугольника больше его периметра?
		\item Периметр параллелограмма равен \( 50 \). Одна сторона на \( 5 \) больше другой. Чему равны стороны параллелограмма?
		\item Один из углов параллелограмма равен \( 20\degree \). Найдите остальные углы параллелограмма.
		\item В параллелограмме \( ABCD \) угол \( A \) острый. Из вершины \( B \) опущен перпендикуляр \( BK \) к прямой \( AD \). Оказалось, что \( AK=BK \). Найдите углы \( C \) и \( D \).
		\item В параллелограмме \( ABCD \) диагональ \( AC \) образует со сторонами \( AB \) и \( BC \) углы, равные \( 45\degree \) и \( 25\degree \). Чему равен угол \( C \)?
		\item Решить уравнение:\quad\( (3x-1)(3x+1)-(x-1)(x+2)=8 \)
\end{listofex}
\end{exam}
%
%===============>>  Консультация  <<===============
%
\begin{consultation}
	\begin{listofex}
		\item Вычислить: \( \left( \mfrac{8}{1}{2}-\mfrac{7}{3}{8} \right)\cdot\mfrac{5}{2}{3}-\mfrac{1}{4}{5}\cdot\left( \mfrac{3}{1}{3}-\mfrac{2}{7}{9} \right) \)
		\item Преобразовать в многочлен стандартного вида:
		\begin{enumcols}[itemcolumns=2]
			\item \( (2x-3)^2-(x+2)(3x-1)-(4x+3)(3-4x) \)
			\item \( (7-2x)(x+2)-(6-x)(x+6)-(-2x-3)^2 \)
			\item \( (x-4)^3-x(x^2-6)+12(x+1) \)
			\item \( 3(x+1)^2-2(x-1)^3+2x^3 \)
		\end{enumcols}
		\item Решить уравнение:
		\begin{enumcols}[itemcolumns=2]
			\item \( (5x-1)(x-8)-(x+3)(5x+2)=0 \)
			\item \( (x-3)^2+x(8-x)=12 \)
			\item \( (2x-3)^2+(3-4x)(x-4)=0 \)
			\item \( (3x+4)^2=(3x-2)(2+3x) \)
		\end{enumcols}
		\item Биссектрисы углов \( N \) и \( M \) треугольника \( MNP \) пересекаются в точке \( A \). Найдите \( \angle NAM \), если \( \angle N = 84\degree \), а \( \angle M = 42\degree \).
		\item Диагональ прямоугольника образует угол \( 51\degree \) с одной из его сторон. Найдите острый угол между диагоналями этого прямоугольника. Ответ дайте в градусах.
	\end{listofex}
\end{consultation}
\newpage
\begin{consultation}
	\begin{listofex}
		\item Преобразуйте трехчлен в квадрат двучлена:
		\begin{tasks}(2)
			\task \( 81a^2-18ab+b^2 \)
			\task \( 4x^2+12x+9 \)
			\task \( 1+y^2-2y \)
			\task \( 100x^2+y^2+20xy \)
		\end{tasks}
		\item Преобразуйте трехчлен в квадрат двучлена:
		\begin{tasks}(3)
			\task \( x^4-8x^2y^2+16y^4 \)
			\task \( \dfrac{1}{4}x^2+2xy^2+4y^4 \)
			\task \( a^2x^2-2abx+b^2 \)
		\end{tasks}
		\item Представьте в виде многочлена произведение:
		\begin{tasks}(3)
			\task \( (x^2-5)(x^2+5) \)
			\task \( (a^3-b^2)(a^3+b^2) \)
			\task \( \left( \dfrac{3}{7}m^3+\dfrac{1}{4}n^3 \right)\left( \dfrac{3}{7}m^3-\dfrac{1}{4}n^3 \right) \)
			\task \( 5x(x-2)(x+2) \)
		\end{tasks}
		\item Упростить выражение: \( \dfrac{x-y}{6}:\dfrac{x^2-y^2}{y}\cdot\dfrac{x^2+2xy+y^2}{x} \)
		\item Докажите, что если диагонали четырёхугольника делят друг друга пополам, то противоположные стороны четырёхугольника --- равны.
		\item Сторона параллелограмма втрое больше другой стороны, а его периметр равен \( 24 \). Найдите его стороны.
		\item Биссектриса угла параллелограмма делит его сторону на отрезки \( 3 \) и \( 4 \). Найдите стороны параллелограмма.
		
	\end{listofex}
\end{consultation}