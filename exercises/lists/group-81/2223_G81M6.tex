%
%===============>>  ГРУППА 8-1 МОДУЛЬ 6  <<=============
%
\setmodule{6}

%BEGIN_FOLD % ====>>_____ Занятие 1 _____<<====
\begin{class}[number=1]
	\begin{listofex}
		\item Упростить выражения:
		\begin{itasks}[2]
			\task \exercise{4223}
			\task \exercise{913}
			\task \exercise{921}
			\task \exercise{928}
			\task \exercise{930}
			\task \exercise{933}
			\task \exercise{936}
			\task \exercise{938}
			\task \exercise{942}
		\end{itasks}
		\item Упростите выражения:
		\begin{tasks}(2)
			\task \( \dfrac{25x^2+10x+4}{5x^2+2x} \)
			\task \( \dfrac{(2a^2)^3\cdot(3b)^2}{(6a^3b)^2} \)
			\task \( \dfrac{x^3+2x^2-9x-18}{(x-3)(x+2)} \)
			\task \( \dfrac{ab-2b-6+3a}{a^2-4} \)
			\task \( (a^3-16a)\cdot\left( \dfrac{1}{a+4}-\dfrac{1}{a-4} \right) \)
			\task \( \dfrac{8a}{9c}-\dfrac{64a^2+81c^2}{72ac}+\dfrac{9c-64a}{8a} \)
		\end{tasks}
		\item \exercise{1510}
	\end{listofex}
\end{class}
%END_FOLD

%BEGIN_FOLD % ====>>_____ Занятие 2 _____<<====
\begin{class}[number=2]
	\begin{listofex}
		\item Сократите дробь:
		\begin{tasks}(2)
			\task \( \dfrac{7x+7y}{x^2+2xy+y^2} \)
			\task \( \dfrac{az+at-bz-bt}{az-at-bz+bt} \)
			\task \( \dfrac{a^3-a^2-a+1}{a^2-2a^2+1} \)
			\task \( \dfrac{-15x^2+4yz-10xz+6xy}{15x^2+2yz-5xz-6xy} \)
			\task \( \dfrac{9x^3-2bc^2-xc^2+18x^2b}{3x^2-2bc-xc+6xb} \)
		\end{tasks}
		\item Упростите выражения:
		\begin{tasks}(2)
			\task \( \dfrac{b-6}{4-b^2}+\dfrac{2}{2b-b^2} \)
			\task \( \dfrac{b}{ab-5a^2}-\dfrac{15b-25a}{b^2-25a^2} \)
			\task \( \dfrac{x-12a}{x^2-16a^2}-\dfrac{4a}{4ax-x^2} \)
			\task \( \dfrac{a-30y}{a^2-100y^2}-\dfrac{10y}{10ay-a^2} \)
		\end{tasks}
		\item Упростите выражения:
		\begin{tasks}(1)
			\task \( \left( a+\dfrac{2+a^2}{1-a} \right)\cdot\dfrac{1-2a+a^2}{a+2} \)
			\task \( \dfrac{7-5m}{m-4}+\dfrac{4m}{m+4}\cdot\dfrac{m^2-16}{4m}+\dfrac{9m-23}{m-4} \)
			\task \( \left( \dfrac{16b}{16-b^2}+\dfrac{4-b}{4+b} \right):\dfrac{3a+16}{a^2-36}+\dfrac{6(a-6)}{a+6} \)
		\end{tasks}
	\end{listofex}
\end{class}
%END_FOLD

%BEGIN_FOLD % ====>>_ Домашняя работа 1 _<<====
\begin{homework}[number=1]
	\begin{listofex}
		\item Упростить выражения:
		\begin{tasks}(2)
			\task \( \dfrac{x-3}{4}-1-\dfrac{x-4}{3} \)
			\task \( \dfrac{(x+y)^2}{y}-2x \)
			\task \( \dfrac{a^2}{5(a-b)}-\dfrac{b^3}{4(a-b)} \)
			\task \( \dfrac{15}{10x^3y-15x^2y^2}-\dfrac{6y}{9xy^3-6x^2y^2} \)
		\end{tasks}
		\item Выполните умножение:
		\begin{tasks}(1)
			\task \( \dfrac{4x^2-6xy+9y^2}{2x-3y}\cdot\dfrac{9y^2-4x^2}{8x^3+27y^3} \)
			\task \( \dfrac{3-6x}{2x^2+4x+8}\cdot\dfrac{2x+1}{x^2+4-4x}\cdot\dfrac{8-x^3}{4x^2-1} \)
		\end{tasks}
		\item Площадь квадрата равна \( 100 \). Найдите площадь ромба с такими же сторонами, если его тупой угол равен \( 150\degree \).
	\end{listofex}
\end{homework}
%END_FOLD

%BEGIN_FOLD % ====>>_____ Занятие 3 _____<<====
\begin{class}[number=3]
	\begin{listofex}
		\item Постройте график функции \( y=x^2 \) и определите:
		\begin{tasks}(1)
			\task Промежутки возрастания функции;
			\task Промежутки убывания функции;
			\task Область определения функции;
			\task Область значений функции.
		\end{tasks}
		\item Постройте график \( y=-\dfrac{1}{5}x^2 \) и определите:
		\begin{tasks}(1)
			\task Промежутки возрастания функции;
			\task Промежутки убывания функции;
			\task Область определения функции;
			\task Область значений функции.
		\end{tasks}
		\item Постройте график функции \( y=-2x^2 \) и найдите координаты точек пересечения с прямой \( y=-3x+1 \) графическим, а затем аналитическим способом.
		\item Найдите координаты вершины параболы и постройте их:
		\begin{tasks}(2)
			\task \( y=6x^2 \)
			\task \( y=x^2+4 \)
			\task \( y=(x+3)^2-3 \)
			\task \( y=2x^2-16x+37 \)
		\end{tasks}
		\item Найдите аналитическим, а затем графическим способом точки пересечения графиков функций \(f(x)=-3x^2+2x+1\)	и \( g(x)=x^2-2x+1 \).
		\item Цена на товар в течение месяца упала сначала на \( 40\% \), а потом увеличилась на \( 50\% \) и составила \( 5130  \) рублей. Найдите первоначальную цену товара.
	\end{listofex}
\end{class}
%END_FOLD

%BEGIN_FOLD % ====>>_____ Занятие 4 _____<<====
\begin{class}[number=4]
	\begin{listofex}
		\item Постройте график функции \( y=-2x^2 \) и найдите координаты точек пересечения с прямой \( y=-3x+1 \) графическим, а затем аналитическим способом.
		\item Найдите координаты вершины параболы и постройте их:
		\begin{tasks}(2)
			\task \( y=6x^2 \)
			\task \( y=x^2+4 \)
			\task \( y=(x+3)^2-3 \)
			\task \( y=2x^2-16x+37 \)
		\end{tasks}
		\item Найдите аналитическим, а затем графическим способом точки пересечения графиков функций \(f(x)=-3x^2+2x+1\)	и \( g(x)=x^2-2x+1 \).
		\item Цена на товар в течение месяца упала сначала на \( 40\% \), а потом увеличилась на \( 50\% \) и составила \( 5130  \) рублей. Найдите первоначальную цену товара.
	\end{listofex}
\end{class}
%END_FOLD

%BEGIN_FOLD % ====>>_ Домашняя работа 2 _<<====
\begin{homework}[number=2]
	\begin{listofex}
		\item Постройте график функции \( y=3x^2 \):
		\begin{tasks}(1)
			\task Определите промежутки возрастания и убывания функции;
			\task Найдите область определения и область значений функции;
			\task Найдите точки пересечения графика данной функции с графиком функции \( y=7x-2 \).
		\end{tasks}
		\item \exercise{1422}
	\end{listofex}
\end{homework}
%END_FOLD

%BEGIN_FOLD % ====>>_____ Занятие 5 _____<<====
\begin{class}[number=5]
	\begin{listofex}
		\item Занятие 5
	\end{listofex}
\end{class}
%END_FOLD

%BEGIN_FOLD % ====>>_____ Занятие 6 _____<<====
\begin{class}[number=6]
	\begin{listofex}
		\item Занятие 6
	\end{listofex}
\end{class}
%END_FOLD

%BEGIN_FOLD % ====>>_ Домашняя работа 3 _<<====
\begin{homework}[number=3]
	\begin{listofex}
		\item Домашняя работа 3
	\end{listofex}
\end{homework}
%END_FOLD

%BEGIN_FOLD % ====>>_____ Занятие 7 _____<<====
\begin{class}[number=7]
	\title{Подготовка к проверочной}
	\begin{listofex}
		\item Занятие 7
	\end{listofex}
\end{class}
%END_FOLD

%BEGIN_FOLD % ====>>_ Проверочная работа _<<====
\begin{exam}
	\begin{listofex}
		\item Проверочная
	\end{listofex}
\end{exam}
%END_FOLD

%BEGIN_FOLD % ====>>_ Консультация _<<====
\begin{consultation}
	\begin{listofex}
		\item Решите уравнения:
		\begin{tasks}(2)
			\task \( \dfrac{x+6}{11}=0 \)
			\task \( \dfrac{2x+3}{5}=5x \)
			\task \( \dfrac{x+5}{4}=\dfrac{x-9}{6} \)
			\task \( \dfrac{x^2-4}{5}=\dfrac{x+2}{7} \)
		\end{tasks}
		\item Решите уравнения:
			\begin{tasks}(2)
				\task \( \dfrac{x^2+2x}{x-2}=0 \)
				\task \( \dfrac{(x-7)(1,5+x)}{x^2-3x+4}=0 \)
				\task \( \dfrac{x^2-8x+7}{x-3}=0 \)
				\task \( \dfrac{4x^2-12x+27}{x^2-3x-10}=0 \)
			\end{tasks}
		\item Решите биквадратные уравнения:
		\begin{tasks}(2)
			\task \( x^4-5x^2+4=0 \)
			\task \( x^2-20x^2+64=0 \)
			\task \( k^2=12k^2+64 \)
			\task \( n^4-2n^2+1=0 \)
		\end{tasks}
	\end{listofex}
	\newpage
	\title{Домашняя работа}
	\begin{listofex}
		\item Решите уравнения:
		\begin{tasks}(2)
			\task \( \dfrac{x-12}{3}=0 \)
			\task \( \dfrac{x-7}{5}=15x \)
			\task \( \dfrac{x-3}{5}=\dfrac{x+1}{9} \)
			\task \( \dfrac{x^2-9}{2}=\dfrac{x+10}{4} \)
		\end{tasks}
		\item Решите уравнения:
		\begin{tasks}(2)
			\task \( \dfrac{3x^2-7x}{x^2+1}=0 \)
			\task \( \dfrac{(-2-x)(x-8,5)}{(x-3)(x+4)}=0 \)
			\task \( \dfrac{4x^2-4x-3}{x+2}=0 \)
			\task \( \dfrac{4x^2+4x-35}{x^2-7x+12}=0 \)
		\end{tasks}
		\item Решите биквадратные уравнения:
		\begin{tasks}(2)
			\task \( x^4-10x^2+9=0 \)
			\task \( x^2-26x^2+25=0 \)
			\task \( y^4+9y^2=400 \)
			\task \( m^4=21m^2+100 \)
		\end{tasks}
	\end{listofex}
\end{consultation}
%END_FOLD

%BEGIN_FOLD % ====>>_ Консультация _<<====
\begin{consultation}
	\begin{listofex}
		\item В прямоугольном треугольнике \( \angle C=90\degree \), \( AB=8 \), \( \angle A=30\degree \). Найдите угол \( B \). Чему будет равен катет \( CB \)?
		\item В прямоугольном треугольнике \( ABC \) катет \( AС \) равен половине гипотенузы \( AB \). Найдите градусную меру всех углов треугольника.
		\item Известно, что в прямоугольном треугольнике (\( \angle C=90\degree \)) угол \( B=60\degree \). Найдите прилежащий к этому углу катет, если гипотенуза равна \( 50 \).
		\item Найдите площадь квадрата, если его сторона равна \( 6 \).
		\item Найдите площадь параллелограмма, если одна из его сторон равна \( 8 \), а опущенная на неё высота равна \( 4 \).
		\item Найдите площадь ромба, если одна из его сторон равна \( 3 \), а высота равна \( 2 \).
		\item Стороны квадрата и ромба равны. Известно, что площадь квадрата равна \( 64 \), а острый угол ромба равен \( 30\degree \). Найдите площадь ромба.
	\end{listofex}
	\newpage
	\title{Домашняя работа}
	\begin{listofex}
		\item В прямоугольном треугольнике \( ABC \) \( \angle C=90\degree \), \( BC=4 \), \( \angle A=30\degree \). Найдите гипотенузу.
		\item В прямоугольном треугольнике один угол равен \( 60\degree \). Чем будет равен прилежащий к этому углу катет, если гипотенуза равна \( 12 \)?
		\item Найдите площадь параллелограмма, если одна из его сторон равна \( 30 \), а высота, опущенная на неё, равна \( 20 \).
	\end{listofex}
\end{consultation}
%END_FOLD