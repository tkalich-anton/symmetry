%
%===============>>  ГРУППА 8-1 МОДУЛЬ 8  <<=============
%
\setmodule{8}

%BEGIN_FOLD % ====>>_____ Занятие 1 _____<<====
\begin{class}[number=1]
	\begin{listofex}
		\item Решите уравнения: \begin{tasks}(1))
			\task \( (x-2)(x-3)(x-4)= (x-3)(x-4)(x-5) \)
			\task \( x(x^{2}+4x+4)=3(x+2) \)
			\task \( (x+5)^{3}=25(x+5)\)
		\end{tasks} 
		\item Решите уравнения: 
		\begin{tasks}(2)
			\task \( x^{2}-18x+77=0 \)
			\task \( x^{2}-22x+72=0 \)
			\task \( 21x^{2}-2x-3=0 \)
			\task \( x^{2}+10x+25=0 \)
			\task \( 8x^{2}-14x+5=0 \)
			\task \( 5x^{2}-18x+16=0 \)
		\end{tasks}
		\item Решите уравнения: 
		\begin{tasks}(1)
			\task \( x^{4}-18x^{2}+32=0 \)
			\task \( x^{4}-6x^{2}-27=0 \)
			\task \( (3x-2)^{4}-7(3x-2)^{2}-18=0 \)
			\task \( (10x-5)^{4}-29(10x-5)^{2}+100=0 \)
		\end{tasks}
		
	\item Площадь прямоугольника равна \( 14 \) м\( ^{2} \). Если его длину уменьшить на \( 1 \) м, а ширину увеличить на \( 3 \) м, то площадь полученного прямоугольника будет равна \( 30 \) м\( ^{2} \) . Найдите ширину первого прямоугольника.
	\end{listofex}
\end{class}
%END_FOLD

%BEGIN_FOLD % ====>>_____ Занятие 2 _____<<====
\begin{class}[number=2]
	\begin{listofex}
		\item \begin{tasks}(2)
			\task \( 3x^{2}=18x \)
			\task \( 100x^{2}-16=0 \)
			\task \( 16x^{2}=49 \)
			\task \( 7x^{2}-28=0 \)
			\task \( 3x^{2}=12x \)
			\task \( 4x^{2}=144 \)
			
		\end{tasks}
	\end{listofex}
		\begin{definit}
			\textbf{Теорема Виетта} Если дано \( x^{2} + bx + c = 0 \), где \( x_{1} \) и \( x_{2} \) являются корнями, то справедливы два равенства: 
			\begin{tasks}(2)
				\task[] \( x_{1}+x_{2}=-b \)
				\task[] \( x_{1}\cdot x_{2}=c \)
			\end{tasks}
		\end{definit}
	\begin{listofex}[resume]
		\item Один из корней уравнения \( x^{2}+px+36=0 \) равен \( 12 \). Найдите другой корень и коэффициент \( p \).
		\item В уравнении \( x^{2}+11x+q=0 \)  один из его корнеей равен \( -3 \). Найдите другой корень и коэффициент \( q \).
		\item \begin{tasks}(1)
			\task \( x^{4}+2x^{2}-3=0 \)
			\task \( x^{4}-x^{2}-12=0 \)
			\task \(  х^{4}  — 18х^{2} + 81=0 \)
			\task \( 256х^{4}  — 32х^{2} +1=0 \)
			\task \( (x+2)^{4}-13(x+2)^{2}+36=0 \)
			\task \( (5x+6)^{4}-21(5x+6)^{2}-100=0 \)
			\task \( (x^{2}+8)^{2}+4(x^{2}-8)-5=0 \)
		\end{tasks}
	\end{listofex}
\end{class}
%END_FOLD

%BEGIN_FOLD % ====>>_ Домашняя работа 1 _<<====
\begin{homework}[number=1]
	\begin{listofex}
		\item Решите уравнения: 
		\begin{tasks}(2)
			\task \( 12 - x^{2} =11 \)
			\task \( x^{2} - 10x =0 \)
		\end{tasks}
		\item Решите уравнения: 
		\begin{tasks}(2)
			\task \(  x^{2} - 5x+4=0 \)
			\task \(  x^{2} - 5 =(x+5)(2x – 1)\)
		\end{tasks}
		\item Решите уравнения: 
		\begin{tasks}(2)
			\task \(  3x^{4}-18x^{2}+27=0 \)
			\task \(  -5y^{4}+30y^{2}-40=0\)
		\end{tasks}
		\item Прямоугольный газон обнесен изгородью длиной 30 м. Площадь газона 56 м\( ^{2} \). Найдите длины сторон. 
		\item При каких значениях \( k \) уравнение \( x^{2} + 2x + k = 0 \)  имеет один  корень?
		\item Число \( -6 \) является корнем уравнения \( 2x^{2} + bx - 6 = 0 \). Найдите второй корень уравнения и значение \( b \).
	\end{listofex}
\end{homework}
%END_FOLD

%BEGIN_FOLD % ====>>_____ Занятие 3 _____<<====
\begin{class}[number=3]
	\begin{listofex}
		\item Решите уравнения:
		\begin{tasks}(2)
			\task \( x^{4}-24x^{2}-25=0 \)
			\task \( x^{4}-12x^{2}-64=0 \)
			\task \( x^{4}-8x^{2}-9=0 \)
			\task \( x^{4}-2x^{2}-8=0 \)
		\end{tasks}
		\item Решите уравнения:
		\begin{tasks}(1)
			\task \( (x^{2}-2)^{2}+16(x^{2}-2)-161=0 \)
			\task \( (x^{2}-9)^{2}+8(x^{2}-9)-105=0 \)
			\task \( (x^{2}+3)^{2}-32(x^{2}+3)-273=0 \)
		\end{tasks}
		\item Решите уравнения: \begin{tasks}(2)
			\task \( (x^{2}-9)^{2}-4(x^{2}-9)+3=0 \)
			\task \( (x^{2}+x+1)(x^{2}+x+2)=12 \)
			\task \( (x^{4}-5x^{2})^{2}-2(x^{4}-5x^{2})=24 \)
			\task \( (x^{2}-6x)^{2}-2(x-3)^{2}=81 \)
			\task \( (x^{2}+3x)^{2}-4(x^{2}+3x+8)=0 \)
			\task \( (x^{2}+8x)(x^{2}+8x-6)=280 \)
		\end{tasks}
		
	\end{listofex}
\end{class}
%END_FOLD

%BEGIN_FOLD % ====>>_____ Занятие 4 _____<<====
\begin{class}[number=4]
	\begin{listofex}
		\item Найдите корень уравнения \begin{tasks}(2)
			\task \( 1-7(4+2x)= -9-4x \)
			\task \( 8-5(2x-3)=13-6x \)
		\end{tasks}
		\item Решите системы уравнений:
		\begin{tasks}(2)
			\task \( \begin{cases}
				x-2y=-9\\\ y=3x+2
			\end{cases} \)
			\task \( \begin{cases}
				2y+x=-9\\\ 5x-4y=16
			\end{cases} \)
			\task \( \begin{cases}
				4-x=y+5\\\ y-4x=14
			\end{cases} \)
			\task \( \begin{cases}
				3x+y=14\\\ 5x=3y
			\end{cases} \)
			\task \( \begin{cases}
				7x-2y=28 \\\ x+y=-5
			\end{cases} \)
			\task \( \begin{cases}
				4y=x+46 \\\ 3x+2y=7
			\end{cases} \)
		\end{tasks}
		\item \begin{tasks}(2)
			\task \( \begin{cases}
				2x=9 \\\
				4x-y=8
			\end{cases} \)
			\task \( \begin{cases}
				2x=7\\\
				6x-y=10
			\end{cases} \)
			\task \( \begin{cases}
				3x=2\\\
				9x-y=7
			\end{cases} \)
			\task \( \begin{cases}
				4x-y=11\\\
				2x+5y=11
			\end{cases} \)
			\task \( \begin{cases}
				2x-y=2\\\
				3x+7y=20
			\end{cases} \)
			\task \( \begin{cases}
				2x-y=2\\\
				3x+7y=20
			\end{cases} \)
			
		\end{tasks}
		
	\end{listofex}
\end{class}
%END_FOLD

%BEGIN_FOLD % ====>>_ Домашняя работа 2 _<<====
\begin{homework}[number=2]
	\begin{listofex}
		\item Решите уравнения:
		\begin{tasks}(2)
			\task \( x^{4}-1x^{2}-2=0 \)
			\task \( 4x^{4}-32x^{2}+64=0 \)
		\end{tasks}
		\item Решите уравнения:
		\begin{tasks}(2)
			\task \( (x^{2}-1)^{4}-4(x^{2}-1)^{2}+\dfrac{9}{4}=0 \)
			\task \( (x^{2}-4)^{2}-6(x^{2}-4)+8=0 \)
		\end{tasks}
		\item Решите системы уравнений:
		\begin{tasks}(2)
			\task \( \begin{cases}
				2x-5y=13\\\ 3x-5y=-13
			\end{cases} \)
			\task \( \begin{cases}
				7x-3y=11\\\ 2x+3y=7
			\end{cases} \)
			\task \( \begin{cases}
				4x-y=2\\\ 2x-1=-8y
			\end{cases} \)
			\task \( \begin{cases}
				6x=5+4y\\\ y=5-x
			\end{cases} \)
		\end{tasks}
	\end{listofex}
\end{homework}
%END_FOLD

%BEGIN_FOLD % ====>>_____ Занятие 5 _____<<====
\begin{class}[number=5]
	\begin{listofex}
		\item Решите системы уравнений:
		\begin{tasks}(3)
			\task \( \begin{cases}
				y=x-1\\\ x+3y=9
			\end{cases} \)
			\task \( \begin{cases}
				2x-3y=9\\\ x+2y=1
			\end{cases} \)
			\task \( \begin{cases}
				y=-2x\\\ x-2y=0
			\end{cases} \)
			\task \( \begin{cases}
				7x-2y=15\\\ 2x+y=9
			\end{cases} \)
			\task \( \begin{cases}
				y=3x\\\ x+2y=7
			\end{cases} \)
			\task \( \begin{cases}
				-4x=2-y\\\ 3y=11x
			\end{cases} \)
		\end{tasks}
		\item Решите системы уравнений:
		\begin{tasks}(2)
			\task \( \begin{cases}
				\dfrac{3x+2y}{5}+\dfrac{x-3y}{6}=3\\\ 2x+7y+43=0
			\end{cases} \)
			\task \( \begin{cases}
				\dfrac{1}{2}x-\dfrac{1}{3}y=1\\\ 6x-5y=3
			\end{cases} \)
			\task \( \begin{cases}
				\dfrac{1}{5}(x+y)=2\\\ \dfrac{1}{2}(x-y)=1
			\end{cases} \)
			\task \( \begin{cases}
				\dfrac{1}{3}(x-y)=4\\\ \dfrac{1}{4}(x+y)=2
			\end{cases} \)
			\task \( \begin{cases}
				4(x+1)-2(y+5)=4\\\ 5(7-x)+4(y-2)=10
			\end{cases} \)
			\task \( \begin{cases}
				2(x-3)-4(y+3)=2\\\ 3(2-x)+7(y-1)=3
			\end{cases} \)
		\end{tasks}
		\item  За 2 кг конфет и 3 кг печенья заплатили 480 р. Сколько стоит 1 кг печенья  и 1 кг конфет, если 1,5 кг конфет дешевле 4 кг печенья на 15 р.?
		\item  В кассе было 136 монет пятирублёвого и  двухрублёвого достоинства на сумму 428 р. Сколько монет каждого достоинства было в кассе?
		\item Семь альбомов и две тетради стоят вместе 111 руб, а пять альбомов и три тетради стоят 84 руб. Сколько стоит один альбом и сколько стоит одна тетрадь?
	\end{listofex}
\end{class}
%END_FOLD

%BEGIN_FOLD % ====>>_____ Занятие 6 _____<<====
\begin{class}[number=6]
	\begin{listofex}
		\item Решите системы уравнений:
		\begin{tasks}(2)
			\task \( \begin{cases}
				x+2y=0\\\
				5x+y=-18
			\end{cases} \)
			\task \( \begin{cases}
				2x-5y=10\\\
				4x-y=2
			\end{cases} \)
			\task \( \begin{cases}
				x-2y=1\\\
				y-x=-2
			\end{cases} \)
			\task \( \begin{cases}
				x+y=-3\\\
				x-y=-1
			\end{cases} \)
			\task \( \begin{cases}
				2x-7y=6\\\
				8x-28y=24
			\end{cases} \)
			\task \( \begin{cases}
				x+2y=0,5\\\
				2x+4y=2
			\end{cases} \)
		\end{tasks}
		\item 
		\begin{tasks}(2)
			\task \( \begin{cases}
				4(x+1)-2(y+5)=4\\\ 5(7-x)+4(y-2)=10
			\end{cases} \)
			\task \( \begin{cases}
				2(x-3)-4(y+3)=2\\\ 3(2-x)+7(y-1)=3
			\end{cases} \)
		\end{tasks}
		\item Три пирожка и две булки стоят 40 рублей, а два пирожка и три булки стоят 45 рублей. Сколько стоят один пирожок и одна булка?
		\item Три марки и пять конвертов стоят 39 рублей, а четыре марки и два конверта стоят 24 рубля. Сколько стоят один конверт и одна марка?
		\item  В кассе было 136 монет пятирублёвого и  двухрублёвого достоинства на сумму 428 р. Сколько монет каждого достоинства было в кассе?
		\item Семь альбомов и две тетради стоят вместе 111 руб, а пять альбомов и три тетради стоят 84 руб. Сколько стоит один альбом и сколько стоит одна тетрадь?
		
		
	\end{listofex}
\end{class}
%END_FOLD

%BEGIN_FOLD % ====>>_ Домашняя работа 3 _<<====
\begin{homework}[number=3]
	\begin{listofex}
		\item Домашняя работа 3
	\end{listofex}
\end{homework}
%END_FOLD

%BEGIN_FOLD % ====>>_____ Занятие 7 _____<<====
\begin{class}[number=7]
	\title{Подготовка к проверочной}
	\begin{listofex}
		\item  Решите системы уравнений:
		\begin{tasks}(2)
			\task \( \begin{cases}
				2x+3=7 \\\ 3x^{2}-12=0
			\end{cases} \)
			\task \( \begin{cases}
				(x-3)(2x+1)=0 \\\ x^{2}-14x+33=0
			\end{cases} \)
			\task \( \begin{cases}
				x^{2}-8x=-16 \\\ (2x-1)(x+2)=42
			\end{cases} \)
			\task \( \begin{cases}
				x^{3}-x^{2}-30x=0\\\ 12x-2x^{2}=0
			\end{cases} \)
			\task \( \begin{cases}
				5(x-3)+1=2x-5\\\ x^{3}-3x^{2}+2x -6=0
			\end{cases} \)
			\task \( \begin{cases}
				4x^{4}-12x^{2}+36=0\\\ x^{5}-6x^{3}=0
			\end{cases} \)
		\end{tasks}
	\end{listofex}
\end{class}
%END_FOLD


%BEGIN_FOLD % ====>>_____ Занятие 7 _____<<====
\begin{consultation}[number=7]
	\begin{listofex}
		\item Постройте график функции \( y=|x| \).
		\item Постройте график функции \( y=-|x|+2 \).
		\item Постройте график функции \( y=|2x+3| \).
		\item Постройте график функции \( y=|3x|-1 \).
		\item Постройте график функции \( y=|x^{2}-4| \).
		\item Через вершину прямого угла прямоугольного треугольника с катетами 6 и 8 см проведен перпендикуляр к гипотенузе. Вычислите площади образовавшихся треугольников.
		\item Прямая, параллельная основанию треугольника, делит его на треугольник и трапецию, площади которых относятся как 4:5. Периметр образовавшегося треугольника равен 20 см. Найдите периметр данного треугольника.
		\item Через точки \( M \) и \( N \), принадлежащие сторонам \( AB \) и \( BC \) треугольника \( ABC \) соответственно, проведена прямая \( MN \), параллельная стороне \( AC \). Найдите длину \( CN \), если \( BC = 6 \), \( MN = 4 \), \( AC = 9 \).
	\end{listofex}
\end{consultation}
%END_FOLD


%BEGIN_FOLD % ====>>_____ Консультация 1 _____<<====
\begin{consultation}[number=1]
	\begin{listofex}
		\item В треугольнике \( ABC \) \( \angle C=90\degree \), \( CH \) --- высота, опущенная из прямого угла. \( AH=9 \), \( HB=4 \). Найдите \( CH \).
		\item В треугольнике \( ABC \) \( \angle C=90\degree \), \( CH \) --- высота, опущенная из прямого угла. \( AH=9 \), \( HB=7 \). Найдите \( AC \).
		\item В треугольнике \( ABC \) \( \angle C=90\degree \), \( CH \) --- высота, опущенная из прямого угла. \( AH=21 \), \( HB=4 \). Найдите \( BC \).
		\item В треугольнике \( ABC \) \( \angle B=90\degree \), \( BH \) --- высота, опущенная из прямого угла. \( AB=20 \), \( HC=30 \). Найдите \( BH \).
		\item В треугольнике, стороны которого равны \( 15 \), \( 20 \) и \( 25 \), проведена высота к большей стороне. Найдите отрезки, на которые высота делит эту сторону.
		\item В треугольнике \( ABC \) точки \( M \), \( N \) и \( K \) --- середины сторон \( AB \), \( BC \) и \( AC \) соответственно. Найдите периметр треугольника \( ABC \), если \( MN=12 \), \( MK=10 \), \( KN=8 \).
		\item Стороны треугольника равны \( 4 \) и \( 6 \). Через середину третьей стороны проведены прямые, параллельные двум другим сторонам. Найдите периметр полученного четырехугольника.
		\item Дан четырехугольник, сумма диагоналей которого равна \( 18 \) Найдите периметр четырехугольника с вершинами в серединах сторон данного.
		\item Найдите стороны и углы четырехугольника с вершинами в серединах сторон ромба, диагонали которого равны \( 6 \) и \( 10 \).
	\end{listofex}
\end{consultation}
%END_FOLD
>>>>>>> 5b82ef55a567f609102f68cabde235f7e87c6459
