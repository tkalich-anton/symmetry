%
%===============>>  ГРУППА 8-1 МОДУЛЬ 8  <<=============
%
\setmodule{8}

%BEGIN_FOLD % ====>>_____ Занятие 1 _____<<====
\begin{class}[number=1]
	\begin{listofex}
		\item Решите уравнения: \begin{tasks}(1))
			\task \( (x-2)(x-3)(x-4)= (x-3)(x-4)(x-5) \)
			\task \( x(x^{2}+4x+4)=3(x+2) \)
			\task \( (x+5)^{3}=25(x+5)\)
		\end{tasks} 
		\item Решите уравнения: 
		\begin{tasks}(2)
			\task \( x^{2}-18x+77=0 \)
			\task \( x^{2}-22x+72=0 \)
			\task \( 21x^{2}-2x-3=0 \)
			\task \( x^{2}+10x+25=0 \)
			\task \( 8x^{2}-14x+5=0 \)
			\task \( 5x^{2}-18x+16=0 \)
		\end{tasks}
		\item Решите уравнения: 
		\begin{tasks}(1)
			\task \( x^{4}-18x^{2}+32=0 \)
			\task \( x^{4}-6x^{2}-27=0 \)
			\task \( (3x-2)^{4}-7(3x-2)^{2}-18=0 \)
			\task \( (10x-5)^{4}-29(10x-5)^{2}+100=0 \)
		\end{tasks}
		
	\item Площадь прямоугольника равна \( 14 \) м\( ^{2} \). Если его длину уменьшить на \( 1 \) м, а ширину увеличить на \( 3 \) м, то площадь полученного прямоугольника будет равна \( 30 \) м\( ^{2} \) . Найдите ширину первого прямоугольника.
	\end{listofex}
\end{class}
%END_FOLD

%BEGIN_FOLD % ====>>_____ Занятие 2 _____<<====
\begin{class}[number=2]
	\begin{listofex}
		\item \begin{tasks}(2)
			\task \( 3x^{2}=18x \)
			\task \( 100x^{2}-16=0 \)
			\task \( 16x^{2}=49 \)
			\task \( 7x^{2}-28=0 \)
			\task \( 3x^{2}=12x \)
			\task \( 4x^{2}=144 \)
			
		\end{tasks}
	\end{listofex}
		\begin{definit}
			\textbf{Теорема Виетта} Если дано \( x^{2} + bx + c = 0 \), где \( x_{1} \) и \( x_{2} \) являются корнями, то справедливы два равенства: 
			\begin{tasks}(2)
				\task[] \( x_{1}+x_{2}=-b \)
				\task[] \( x_{1}\cdot x_{2}=c \)
			\end{tasks}
		\end{definit}
	\begin{listofex}[resume]
		\item Один из корней уравнения \( x^{2}+px+36=0 \) равен \( 12 \). Найдите другой корень и коэффициент \( p \).
		\item В уравнении \( x^{2}+11x+q=0 \)  один из его корнеей равен \( -3 \). Найдите другой корень и коэффициент \( q \).
		\item \begin{tasks}(1)
			\task \( x^{4}+2x^{2}-3=0 \)
			\task \( x^{4}-x^{2}-12=0 \)
			\task \(  х^{4}  — 18х^{2} + 81=0 \)
			\task \( 256х^{4}  — 32х^{2} +1=0 \)
			\task \( (x+2)^{4}-13(x+2)^{2}+36=0 \)
			\task \( (5x+6)^{4}-21(5x+6)^{2}-100=0 \)
			\task \( (x^{2}+8)^{2}+4(x^{2}-8)-5=0 \)
		\end{tasks}
	\end{listofex}
\end{class}
%END_FOLD

%BEGIN_FOLD % ====>>_ Домашняя работа 1 _<<====
\begin{homework}[number=1]
	\begin{listofex}
		\item Решите уравнения: 
		\begin{tasks}(2)
			\task \( 12 - x^{2} =11 \)
			\task \( x^{2} - 10x =0 \)
		\end{tasks}
		\item Решите уравнения: 
		\begin{tasks}(2)
			\task \(  x^{2} - 5x+4=0 \)
			\task \(  x^{2} - 5 =(x+5)(2x – 1)\)
		\end{tasks}
		\item Решите уравнения: 
		\begin{tasks}(2)
			\task \(  3x^{4}-18x^{2}+27=0 \)
			\task \(  -5y^{4}+30y^{2}-40=0\)
		\end{tasks}
		\item Прямоугольный газон обнесен изгородью длиной 30 м. Площадь газона 56 м\( ^{2} \). Найдите длины сторон. 
		\item При каких значениях \( k \) уравнение \( x^{2} + 2x + k = 0 \)  имеет один  корень?
		\item Число \( -6 \) является корнем уравнения \( 2x^{2} + bx - 6 = 0 \). Найдите второй корень уравнения и значение \( b \).
	\end{listofex}
\end{homework}
%END_FOLD

%BEGIN_FOLD % ====>>_____ Занятие 3 _____<<====
\begin{class}[number=3]
	\begin{listofex}
		\item Занятие 3 
	\end{listofex}
\end{class}
%END_FOLD

%BEGIN_FOLD % ====>>_____ Занятие 4 _____<<====
\begin{class}[number=4]
	\begin{listofex}
		\item Занятие 4
	\end{listofex}
\end{class}
%END_FOLD

%BEGIN_FOLD % ====>>_ Домашняя работа 2 _<<====
\begin{homework}[number=2]
	\begin{listofex}
		\item Домашняя работа 2
	\end{listofex}
\end{homework}
%END_FOLD

%BEGIN_FOLD % ====>>_____ Занятие 5 _____<<====
\begin{class}[number=5]
	\begin{listofex}
		\item Занятие 5
	\end{listofex}
\end{class}
%END_FOLD

%BEGIN_FOLD % ====>>_____ Занятие 6 _____<<====
\begin{class}[number=6]
	\begin{listofex}
		\item Занятие 6
	\end{listofex}
\end{class}
%END_FOLD

%BEGIN_FOLD % ====>>_ Домашняя работа 3 _<<====
\begin{homework}[number=3]
	\begin{listofex}
		\item Домашняя работа 3
	\end{listofex}
\end{homework}
%END_FOLD

%BEGIN_FOLD % ====>>_____ Занятие 7 _____<<====
\begin{class}[number=7]
	\title{Подготовка к проверочной}
	\begin{listofex}
		\item Занятие 7
	\end{listofex}
\end{class}
%END_FOLD

=%BEGIN_FOLD % ====>>_ Проверочная работа _<<====
\begin{exam}
	\begin{listofex}
		\item Проверочная
	\end{listofex}
\end{exam}
%END_FOLD

%BEGIN_FOLD % ====>>_____ Консультация 1 _____<<====
\begin{consultation}[number=1]
	\begin{listofex}
		\item В треугольнике \( ABC \) \( \angle C=90\degree \), \( CH \) --- высота, опущенная из прямого угла. \( AH=9 \), \( HB=4 \). Найдите \( CH \).
		\item В треугольнике \( ABC \) \( \angle C=90\degree \), \( CH \) --- высота, опущенная из прямого угла. \( AH=9 \), \( HB=7 \). Найдите \( AC \).
		\item В треугольнике \( ABC \) \( \angle C=90\degree \), \( CH \) --- высота, опущенная из прямого угла. \( AH=21 \), \( HB=4 \). Найдите \( BC \).
		\item В треугольнике \( ABC \) \( \angle B=90\degree \), \( BH \) --- высота, опущенная из прямого угла. \( AB=20 \), \( HC=30 \). Найдите \( BH \).
		\item В треугольнике, стороны которого равны \( 15 \), \( 20 \) и \( 25 \), проведена высота к большей стороне. Найдите отрезки, на которые высота делит эту сторону.
		\item В треугольнике \( ABC \) точки \( M \), \( N \) и \( K \) --- середины сторон \( AB \), \( BC \) и \( AC \) соответственно. Найдите периметр треугольника \( ABC \), если \( MN=12 \), \( MK=10 \), \( KN=8 \).
		\item Стороны треугольника равны \( 4 \) и \( 6 \). Через середину третьей стороны проведены прямые, параллельные двум другим сторонам. Найдите периметр полученного четырехугольника.
		\item Дан четырехугольник, сумма диагоналей которого равна \( 18 \) Найдите периметр четырехугольника с вершинами в серединах сторон данного.
		\item Найдите стороны и углы четырехугольника с вершинами в серединах сторон ромба, диагонали которого равны \( 6 \) и \( 10 \).
	\end{listofex}
\end{consultation}
%END_FOLD