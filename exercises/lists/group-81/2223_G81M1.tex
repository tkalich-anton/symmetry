%Группа 8-1 Модуль 1
\title{Занятие №1}
\begin{listofex}
	\item Вычислить:
	\begin{enumcols}[itemcolumns=2]
		\item \exercise{4089}
		\item \exercise{4090}
	\end{enumcols}
	\item Вычислить:
	\begin{enumcols}[itemcolumns=2]
		\item \exercise{4091}
		\item \exercise{4092}
	\end{enumcols}
	\item Вычислить:
	\begin{enumcols}[itemcolumns=1]
		\item \exercise{4093}
		\item \exercise{4094}
		\item \exercise{4095}
	\end{enumcols}
	\item Вычислить рациональным образом:
	\begin{enumcols}[itemcolumns=2]
		\item \exercise{4096}
		\item \exercise{4097}
	\end{enumcols}
	\item \exercise{4107}
\end{listofex}
\newpage
\title{Занятие №2}
\begin{listofex}
	\item Вычислить:
	\begin{enumcols}[itemcolumns=2]
		\item \exercise{4098}
		\item \exercise{4099}
	\end{enumcols}
	\item Вычислить:
	\begin{enumcols}[itemcolumns=2]
		\item \exercise{4100}
		\item \exercise{4101}
	\end{enumcols}
	\item Вычислить:
	\begin{enumcols}[itemcolumns=1]
		\item \exercise{4102}
		\item \exercise{4103}
		\item \exercise{4104}
	\end{enumcols}
	\item Вычислить рациональным образом:
	\begin{enumcols}[itemcolumns=2]
		\item \exercise{4105}
		\item \exercise{4106}
	\end{enumcols}
	\item \exercise{4108}
\end{listofex}
\newpage
\title{Домашняя работа №1}
\begin{listofex}
	\item Вычислить:
	\begin{enumcols}[itemcolumns=2]
		\item \exercise{4110}
		\item \exercise{4111}
	\end{enumcols}
	\item Вычислить:
	\begin{enumcols}[itemcolumns=2]
		\item \exercise{4112}
		\item \exercise{4113}
	\end{enumcols}
	\item Вычислить:
	\begin{enumcols}[itemcolumns=1]
		\item \exercise{4114}
		\item \exercise{4115}
		\item \exercise{4116}
	\end{enumcols}
	\item Вычислить рациональным образом:
	\begin{enumcols}[itemcolumns=2]
		\item \exercise{4117}
		\item \exercise{4118}
	\end{enumcols}
	\item \exercise{4119}
\end{listofex}
\newpage
\title{Занятие №3}
\begin{listofex}
	\item Вычислить: \( -0,24\cdot(-1,625):(38,1:7,5-4,3)+11,7:(-1,5) \)
	\item Вычислить рациональным образом:
	\begin{enumcols}[itemcolumns=3]
		\item \( \dfrac{7}{9}:9+\dfrac{5}{9}\cdot\dfrac{1}{9}-\dfrac{1}{3}\cdot\dfrac{1}{9} \)
		\item \( 47^2-47\cdot46 \)
		\item \( \dfrac{87\cdot35-81\cdot35}{37\cdot28-28^2} \)
	\end{enumcols}
	\item Вычислить:
	\begin{enumcols}[itemcolumns=2]
		\item \( \dfrac{5^{10}\cdot(5^3)^4}{5^{18}} \)
		\item \( \dfrac{3^{10}\cdot3^{34}}{3^{17}\cdot{(3^5)^2}} \)
	\end{enumcols}
	\item Упростить выражение:
	\begin{enumcols}[itemcolumns=2]
		\item \( -2(7x-2y-3a)+3(3y-2a+x) \)
		\item \( 2(a-7b)+5(11b-12a)-8(7a-5b) \)
	\end{enumcols}
	\item Упростить выражение:
	\begin{enumcols}[itemcolumns=2]
		\item \( (5a b^{2}+4b^{3})\,(3a b^{3}-\,4a^{2})-\,18a^{2}b^{3} \)
		\item \( (7x^{3}y^{2}-x y)(-2x^{2}y^{2}+5x y^{3})+12x^{5}y^{4} \)
		\item \( (x^{3}+x^{2}y+x y^{2}+y^{3})\,(x-y)-x^{2}y\,(x-y) \)
		\item \( (x+2)(x^{2}-2x+4)-x(x-3)(x+3) \)
		\item \( \left( \dfrac{1}{2}a-2b \right)\left( \dfrac{1}{4}a^2+ab+4b^2 \right)-\left( \dfrac{1}{8}a^3-8b^3 \right) \)
		\item \( 15x^3y^2-(5xy-2)(3x^2y+x) \)
	\end{enumcols}
	\item Докажите тождество: \( (4a^2+4a+1)(4a^2-4a+1)-a^2(2a^2-8)=1 \)
	%\item Решить уравнение:
	%\begin{enumcols}[itemcolumns=2]
	%	\item \( 8(x-3)-5(2x-4)=6x-7 \)
	%	\item \( \dfrac{-0,2(6x+1)}{3,6}=\dfrac{0,5x}{-9} \)
	%	\item \( \dfrac{2x-5,6}{3}=\dfrac{1-x}{1,5} \)
	%	\item \( \dfrac{1}{2}\left( 4-3\dfrac{1}{2}x \right)=1\dfrac{1}{4}x-\dfrac{1}{2} \)
	%\end{enumcols}
	%\item Решить уравнение:
	%\begin{enumcols}[itemcolumns=2]
	%	\item \( (2x-3)(3x-2)=0 \)
	%	\item \( \left( 1\dfrac{2}{3}x+8 \right)\left( 4x-4\dfrac{16}{57} \right)=0 \)
	%\end{enumcols}
\end{listofex}
\newpage
\title{Занятие №4}
\begin{listofex}
	\item Вычислить: \( \left( -0,25-\dfrac{3}{4}-\dfrac{1}{2} \right)\cdot(-0,2)+3,9 \)
	\item Вычислить рациональным образом:
	\begin{enumcols}[itemcolumns=2]
		\item \( 3\dfrac{4}{5}\cdot3\dfrac{2}{19}+3\dfrac{4}{5}\cdot1\dfrac{17}{19} \)
		\item \( 34^2-34\cdot32 \)
		\item \( \dfrac{15^2+15\cdot13}{71\cdot49-11\cdot49} \)
	\end{enumcols}
	\item Вычислить:
	\begin{enumcols}[itemcolumns=3]
		\item \( \dfrac{7^{11}}{7^2\cdot(7^4)^3} \)
		\item \exercise{1609}
		\item \exercise{1610}
	\end{enumcols}
	\item Раскрыть скобки и привести подобные слагаемые:
	\begin{enumcols}[itemcolumns=2]
		\item \( -(m+n)-(x+y)-(x-y-m-n) \)
		\item \( 2(3x+7t-11)-3(2x-3t-11) \)
	\end{enumcols}
	\item Упростить выражение:
	\begin{enumcols}[itemcolumns=1]
		\item \( 2-(-4x+1)(x-1)+2(6x-4)(x+3) \)
		\item \( 6(x+1)(x+1)+2(x-1)(x^2+x+1)-2(x+1) \)
		\item \( (a+2b)(a+c)-(a-2b)(a-c) \)
		\item \( (x^2+y^2+x+y)(x+y+xy) \)
	\end{enumcols}
	\item Докажите тождество \( (a-1)(a+1)(a^2+1)(a^4+1)-a^8=-1 \)
\end{listofex}
\newpage
\title{Домашняя работа №2}
\begin{listofex}
	\item Вычислить: \( \left( 5\dfrac{4}{17}+3\dfrac{7}{8}-7\dfrac{4}{17} \right)\cdot\left( -5\dfrac{1}{3} \right):(-6,25) \)
	\item Вычислить рациональным образом:
	\begin{enumcols}[itemcolumns=3]
		\item \( 2\dfrac{1}{2}\cdot1\dfrac{1}{7}+2\dfrac{1}{2}\cdot\dfrac{6}{7} \)
		\item \( 124^2-124\cdot120 \)
		\item \( \dfrac{16^2-16\cdot14}{8^2+8\cdot32} \)
	\end{enumcols}
	\item Вычислить:
	\begin{enumcols}[itemcolumns=2]
		\item \( \dfrac{5^{12}}{5^3\cdot(5^2)^4} \)
		\item \( \dfrac{8^3\cdot3^4}{6^3\cdot2^4} \)
	\end{enumcols}
	\item Упростить выражение:
	\begin{enumcols}[itemcolumns=1]
		\item \( a^2(a^2-b^2)-(a^3-a^2b+ab^2-b^3)(a+b) \)
		\item \( (x^2+2)(x^2+2)-(x-2)(x+2)(x^2+4) \)
		\item \( \dfrac{1}{2}(a+b+c)(a+b-c)-ab \)
		\item \( (0,1p^3-2p^2q-0,5pq^2+1,2p^3)(8p^2-0,2pq+5q^2) \)
	\end{enumcols}
	\item Докажите тождество \( (a-2)(a+2)(a^2+4)(a^4+16)-a^8=-256 \)
	
\end{listofex}
\newpage
\title{Занятие №5}
\begin{listofex}
	\item Преобразуйте в многочлен:
	\begin{enumcols}[itemcolumns=3]
		\item \( (x+3)^2 \)
		\item \( (0,1x-2,5)^2 \)
		\item \( \left( \dfrac{1}{3}x^2-3y^3 \right)^2 \)
		\item \( (-3c-a^2)^2 \)
		\item \( \left( 1\dfrac{1}{3}ab^2-3a^2b \right) \)
	\end{enumcols}
	\item Представьте в виде квадрата суммы или разности:
	\begin{enumcols}[itemcolumns=3]
		\item \( 9m^2+6mn+n^2 \)
		\item \( 14+p^2+8p \)
		\item \( 16p^2+40pq+25q^2 \)
		\item \( 16p^2+49q^2-56pq \)
		\item \( a^6+2a^3b^3+b^6 \)
		\item \( x^4-6x^2y+9y^2 \)
	\end{enumcols}
	\item Вычислите применив формулу квадрата суммы или разности:
	\begin{enumcols}[itemcolumns=4]
		\item \( 31^2 \)
		\item \( 101^2 \)
		\item \( 199^2 \)
		\item \( 999^2 \)
	\end{enumcols}
	\item Преобразуйте в многочлен стандартного вида:
	\begin{enumcols}[itemcolumns=2]
		\item \( 2(a+1)^2 +3(a+2)^2\)
		\item \( (a+3)^2+(x+1)^2\)
		\item \( (2a+3b)^2-(3a+2b)^2 \)
		\item \( (m+n)^2-(m-n)^2 \)
		\item \( 4(m-2n)^2-3(3m+n)^2 \)
	\end{enumcols}
	\item Разложите на множители:
	\begin{enumcols}[itemcolumns=3]
		\item \( 3x^2-9x \)
		\item \( 125x^2y^3-75xy^4 \)
		\item \( 12ab^2-6a^3+9ab \)
		\item \( 16-p^4 \)
		\item \( 4a^2-1 \)
		\item \( 9x^4-4 \)
		\item \( (3x+2)^2-x^2 \)
		\item \( (4x+3)^2-(x+1)^2 \)
		\item \( (2x^2-y)^2-x^4 \)
	\end{enumcols}
	\item Преобразуйте в многочлен стандартного вида:
	\begin{enumcols}[itemcolumns=1]
		\item \( 2(p+3q)(p+2q)-(p+2q)^2-(3q+p)^2 \)
		\item \( 5(n-5m)^2-6(2n-3m)^2-(3m-n)(7m-n) \)
		\item \( -(2+m)^2+2(1+m)^2-2(1-m)(m+1) \)
	\end{enumcols}
	\item Представить в виде многочлена:
	\begin{enumcols}[itemcolumns=2]
		\item \( (m+n)(m^2-mn+n^2) \)
		\item \( (25-5m+m^2)(5+m) \)
		\item \( (a^4b^2-2a^2b+4)(2+a^2b) \)
		\item \( (a^4+1)(a^8-a^4+1) \)
	\end{enumcols}
	\item Представить в виде суммы или разности кубов:
	\begin{enumcols}[itemcolumns=4]
		\item \( x^3+8 \)
		\item \( 8m^6+n^9 \)
		\item \( 64p^9+q^{12} \)
		\item \( c^6+125d^3 \)
	\end{enumcols}
\end{listofex}
%\newpage
%\title{Занятие №6}
%\begin{listofex}
%	\item 1
%	
%\end{listofex}
\newpage
\title{Домашняя работа №3}
\begin{listofex}
	\item Представьте в виде многочлена:
	\begin{enumcols}[itemcolumns=1]
		\item \( (7x^3-2x+12)-(7x^3+2x+11)+5x \)
		\item \( (1-x+4x^2-8x^3)+(2x^3+x^2-6x-3)-(5x^3+8x^2) \)
	\end{enumcols}
	\item Представьте в виде многочлена:
	\begin{enumcols}[itemcolumns=1]
		\item \( 5(4x^2-2x+1)-2(10x^2-6x-1) \)
		\item \( a(3b-1)-b(a-3)-2(ab-a+b) \)
	\end{enumcols}
	\item Докажите, что значение выражения не зависит от значения переменной:
	\[ (a^2-3)^2-(a-2)(a^2+4)(a+2)-6(5-a^2) \]
\end{listofex}
\newpage
\title{Занятие №7}
\begin{listofex}
	\item Представьте в виде многочлена:
	\begin{enumcols}[itemcolumns=1]
		\item \( (-2x^2+x+1)-(x^2-x+7)-(4x^2+2x+8) \)
		\item \( (3a^2-a+2)+(-3a^2+3a-1)-(a^2-1) \)
	\end{enumcols}
	\item Преобразуйте в многочлен стандартного вида:
	\begin{enumcols}[itemcolumns=2]
		\item \( 7(2y^2-5y-3)-4(3y^2-9y-5) \)
		\item \( x^2(4-y^2)+y^2(x^2-7)-4x(x-3) \)
		\item \( 2(p+3q)(p+2q)-(p+2q)^2-(3q+p)^2 \)
		\item \( 5(n-5m)^2-6(2n-3m)^2-(3m-n)(7m-n) \)
		\item \( -(2+m)^2+2(1+m)^2-2(1-m)(m+1) \)
	\end{enumcols}
	\item Представить в виде многочлена:
	\begin{enumcols}[itemcolumns=2]
		\item \( (m+n)(m^2-mn+n^2) \)
		\item \( (25-5m+m^2)(5+m) \)
		\item \( (a^4b^2-2a^2b+4)(2+a^2b) \)
		\item \( (a^4+1)(a^8-a^4+1) \)
	\end{enumcols}
	\item Докажите тождество:
	\begin{enumcols}[itemcolumns=2]
		\item \( (y^4+y^3)(y^2-y)=y^3(y^2+1)(y-1) \)
		\item \( (c^4-c^2+1)(v^4+c^2+1)=c^8+c^4+1 \)
	\end{enumcols}
	\item Докажите, что значение выражения не зависит от значения переменной:
	\[ (a-1)(a^2+1)(a+1)-(a^2-1)^2-2(a^2-3) \]
\end{listofex}
\newpage
\title{Проверочная работа}
\begin{listofex}
	\item \exercise{4110}
	\item Преобразуйте в многочлен:
	\begin{enumcols}[itemcolumns=4]
		\item \( (3x+5)^2 \)
		\item \( \left( \dfrac{1}{3}x+y^2 \right)^2 \)
		\item \( (0,2x-3,6)^2 \)
		\item \( \left( ab^2-\dfrac{3}{4}b^3 \right)^2 \)
	\end{enumcols}
	\item Преобразуйте в многочлен стандартного вида:
	\begin{enumcols}[itemcolumns=2]
		\item \( 2(a+1)^2 +3(a+2)^2\)
		\item \( 4(m-2n)^2-3(3m+n)^2 \)
	\end{enumcols}
	\item Представить в виде многочлена:
	\begin{enumcols}[itemcolumns=1]
		\item \( 2(p+2q)(p+2q)-(2q+p)^2-(3q+p)^2 \)
		\item \( 3(n-2m)^2-2(4n-3m)^2-(2m-n)(5m-n) \)
	\end{enumcols}
	\item Представить в виде многочлена:
	\begin{enumcols}[itemcolumns=3]
		\item \( (2m+n)(m^2-mn+n^3) \)
		\item \( \left( 25-\dfrac{1}{5}m+m^2 \right)(25+m) \)
		\item \( (a^4+1)(a^8-a^4+1) \)
	\end{enumcols}
	\item Вычислите применив формулу квадрата суммы или разности:
	\begin{enumcols}[itemcolumns=3]
		\item \( 21^2 \)
		\item \( 201^2 \)
		\item \( 299^2 \)
	\end{enumcols}
	\item Докажите, что значение выражения не зависит от значения переменной:
	\[ (a-1)(a^2+1)(a+1)-(a^2-1)^2-2(a^2-3) \]
\end{listofex}
%\newpage
%\title{Домашняя работа №4}
%\begin{listofex}
%	\item 1
%	
%\end{listofex}