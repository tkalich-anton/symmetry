%Группа 8-1 Модуль 2
\title{Занятие №1}
\begin{listofex}
	\item \exercise{2350}
	\item \exercise{2390}
	\item \exercise{2380}
	\item Докажите, что внешний угол треугольника равен сумме двух внутренних углов не смежных с ним.
	\item \exercise{2385}
	\item \exercise{2381}
	\item \exercise{2393}
	\item \exercise{2387}
	\item \exercise{2407}
	\item \exercise{2400}
\end{listofex}
\newpage
\title{Занятие №2}
\begin{listofex}
	\item Дан треугольник \( ABC \), причем \( AB = AC \) и \( \angle A = 110\degree \). Внутри треугольника взята точка \( M \) такая, что \( \angle MBC = 30\degree \), а \( \angle MCB = 25\degree \). Найдите \( \angle AMC \).
	\item Докажите, что если медиана равна половине стороны, к которой она проведена, то треугольник прямоугольный.
	\item \exercise{2415}
	\item Докажите, что если треугольник вписан в окружность и одна из его сторон является диаметром этой окружности, то такой треугольник является прямоугольным.
	\item Докажите обратное, что если треугольник прямоугольный и вписан в окружность, то гипотенуза будет являться диаметром окружности.
	\item \exercise{2455}
	\item \exercise{2418}
	\item \exercise{2424}
\end{listofex}
\newpage
\title{Домашняя работа №1}
\begin{listofex}
	\item Вычислить:
	\begin{enumcols}[itemcolumns=2]
		\item \( 3^7\cdot3^9:3^{14} \)
		\item \( \dfrac{10^8}{2^9\cdot2^8} \)
	\end{enumcols}
	\item Упростить выражение:
	\begin{enumcols}[itemcolumns=2]
		\item \( (3x-y)^2-3x(3x+2y^2) \)
		\item \( (2x+1)^3-(2x-1)^3 \)
	\end{enumcols}
	\item Докажите, что в равных треугольниках соответствующие биссектрисы равны.
	\item В равностороннем треугольнике \( ABC \) биссектрисы \( CN \) и \( AM \) пересекаются в точке \( P \). Найдите \( \angle MPN \).
	\item \exercise{2347}
	\item \exercise{2423}
	\item \exercise{2456}
\end{listofex}
\newpage
\title{Занятие №3}
\begin{listofex}
	\item Докажите следующие свойства окружности:
	\begin{enumcols}[itemcolumns=1]
		\item диаметр, перпендикулярный хорде, делит ее пополам \textit{(теорема о диаметре, перпендикулярном хорде)};
		\item диаметр, проходящий через середину хорды, не являющейся диаметром, перпендикулярен этой хорде \textit{(теорема о диаметре, проходящем через середину хорды)};
		\item хорды, удаленные от центра окружности на равные расстояния, равны.
	\end{enumcols}
	\item \exercise{2437}
	\item \exercise{2439}
	\item \exercise{2442}
	\item \exercise{2444}
	\item \exercise{2445}
	\item \exercise{2422}
	\item[\loeitem{8*}] \exercise{2420}
\end{listofex}
\newpage
\title{Занятие №4}
\begin{listofex}
	\item Внутренние углы треугольника \( ABC \) относятся как \( 10:5:3 \). Найдите внутренние и внешние углы треугольника \( ABC \) и вычислите разницу самого наибольшего и наименьшего внешних углов. \answer{ внутренние:\( 100;\;50;\;30 \), внешние: \( 80;\;130;\;100; \), разница: \( 50 \) }
	\item В треугольнике \( ABC \) углы \( B \) и \( C \) равны \( 30 \) и \( 40 \) соответственно. Сторону \( AB \) продлили за вершину \( A \) и из это вершины провели высоту и биссектрису внешнего угла. Найдите угол между высотой и биссектрисой внешнего угла. \answer{ \( 85 \) }
	\item Две параллельные прямые пересечены третьей. Найдите угол между биссектрисами внутренних односторонних углов. \answer{ \( 90 \) }
	\item Через точку \( M \), лежащую внутри угла с вершиной \( A \), проведены прямые, параллельные сторонам угла и пересекающие эти стороны в точках \( B \) и \( C \). Известно, что \( \angle ACB = 50\degree \), а угол, смежный с углом \( \angle ACM \), равен \( 40\degree \). Найдите углы треугольников \( BCM \) и \( ABC \).
	\item \exercise{2438}
	\item В равнобедренном треугольнике \( ABC \), с основанием \( AB \), угол \(  ABC = 70\degree \). Найдите величину внешнего угла при вершине \( C \). \answer{ \( 140 \) }
	\item[\loeitem{7*}] \exercise{2441}
\end{listofex}
\newpage
\title{Домашняя работа №2}
\begin{listofex}
	\item \exercise{2436}
	\item \exercise{2440}
	\item \exercise{2457}
	\item \exercise{2412}
	\item В треугольнике \( ABC \) угол \( \angle B = 80 \). Найдите угол между высотами проведенными из двух других углов.
\end{listofex}
\newpage
\title{Занятие №5}
\begin{listofex}
	\item Прямая пересекает параллельные прямые \( a \) и \( b \) в точках \( A \) и \( B \) соответственно. Биссектриса одного из образовавшихся углов с вершиной \( B \) пересекает прямую a в точке \( C \). Найдите \( AC \), если \( AB = 1 \).
	\item \exercise{2472}
	\item \exercise{2473}
	\item \exercise{2474}
	\item \exercise{2476}
	\item \exercise{2483}
	\item \exercise{2493}
	\item \exercise{1608}
\end{listofex}
\newpage
\title{Занятие №6}
\begin{listofex}
	\item \exercise{2484}
	\item \exercise{2477}
	\item \exercise{2478}
	\item \exercise{2459}
	\item В прямой угол \( O \) вписана окружность радиуса \( 12 \), касающаяся сторон угла в точках \( A  \) и \( B \). Через некоторую точку \( K \) на меньшей дуге \( AB  \) окружности проведена касательная, пересекающая \( OA \) в точке \( M \) и \( OB \) в точке \( N \).
	\begin{enumcols}[itemcolumns=1]
		\item Доказать, что \( AM=MK \) и \( BN=NK \);
		\item Найти периметр треугольника \( OMN \).
	\end{enumcols} \answer{ \( 40 \) }
	\item \exercise{2486}
	\item \exercise{2479}
%	\item \exercise{2481}
%	\item \exercise{2500}
%	\item \exercise{2502}
%	\item \exercise{1115}
\end{listofex}
\newpage
\title{Подготовка к проверочной работе}
\begin{listofex}
	\item Чему равен угол между биссектрисами двух смежных углов?
	\item Чему равен угол между биссектрисами двух внутренних односторонних углов при параллельных прямых? Докажите это.
	\item Сформулируйте и докажите теорему о внешнем угле треугольника.
	\item Докажите, что биссектриса внешнего угла при вершине равнобедренного треугольника, параллельна основанию.
	\item Докажите, что если в треугольнике один угол равен сумме двух других, то такое треугольник прямоугольный.
	\item Докажите, что если медиана равна половине стороны, к которой она проведена, то такой треугольник прямоугольный.
	\item Докажите, что если треугольник вписан в окружность и одна из его сторон является диаметром этой окружности, то такой треугольник прямоугольный.
	\item Сформулируйте теорему об угле в \( 30\degree \) в прямоугольном треугольнике. Сформулируйте обратную теорему.
	\item Сформулируйте теорему о диаметре, перпендикулярном хорде.
	\item Сформулируйте теорему о диаметре, проходящем через середину хорды.
	\item Сформулируйте теорему о двух касательных, проведенных из одной точки к окружности.
	\item \exercise{2472}
	\item В треугольнике \( ABC \) стороны \( AB \) и \( BC \) равны. Чему равен угол \( ACB \), если угол \( \angle ABC = 40\degree \)?
	\item В треугольнике \( ABC \) обе стороны \( AB \) и \( BC \) равны \( 15 \). Чему равна сторона \( AC \), если \( \angle BAC = 60 \degree \)?
	\item В треугольнике \( ABC \) известно, что \( \angle A = 50 \) и \( \angle B = 80 \). Найдите сторону \( BC \), если \( AC = 10 \) и \( P_{ABC}=40 \).
	\item В треугольнике \( ABC \) из вершин \( A \) и \( B \) проведены биссектрисы, который пересекаются в точке \( O \). Угол \( \angle AOB \) равен \( 120\degree \). Чему равен третий угол \( ACB \)?
	\item Угол треугольника равен \( 50\degree \). Найдите угол между высотами, проведенными из двух других углов.
	\item \exercise{2456}
	\item \exercise{2437}
	\item \exercise{2459}
	\item \exercise{2475}
	\item \exercise{2478}
\end{listofex}
\newpage
\title{Консультация}
\begin{listofex}
	\item Постройте следующие точки в декартовой системе координат:
	\begin{enumcols}[itemcolumns=3]
		\item \( A(3;1) \)
		\item \( B(-2;4) \)
		\item \( C(7;-7) \)
		\item \( D(-2;-5) \)
		\item \( E(0;4) \)
		\item \( F(-5;0) \)
	\end{enumcols}
	Какие из этих точек лежат на оси абсцисс? Какие на оси ординат? Определите для остальных точек четверть, в которой они лежат.
	\item Подберите вторую координату так, чтобы точка:
	\begin{enumcols}[itemcolumns=1]
		\item \( A(*;4) \) лежала в 1 четверти;
		\item \( B(-1;*) \) лежала в 3 четверти;
		\item \( A(10;*) \) лежала в 4 четверти;
		\item \( A(*;5) \) лежала в 2 четверти.
	\end{enumcols}
	\item Найдите координаты точки, которая симметрична точке \( A(2;4) \) относительно оси \( OX \).
	\item Найдите координаты точки, которая симметрична точке \( A(-5;-5) \) относительно оси \( OY \).
	\item Даны точки \( A(2;1) \) и \( B(-5;1) \). Найдите координаты таких двух точек \( C \) и \( D \), чтобы соединив их получился квадрат.
	\item Постройте графики линейных функций:
	\begin{enumcols}[itemcolumns=2]
		\item \( y=2x-1 \)
		\item \( y=\dfrac{1}{2}x+4 \)
		\item \( y=0,25x-3 \)
		\item \( y=0,5x+0,5 \)
	\end{enumcols}
	\item Найдите уравнение прямой, которая проходит через начало координат и точку \( A(4;5) \)
\end{listofex}
\newpage
\newpage
\title{Проверочная работа}
\title{Вариант 1}
\begin{listofex}
	\item
	\begin{enumcols}[itemcolumns=1]
		\item Чему равен угол между биссектрисами двух смежных углов?
		\item Сформулируйте и докажите теорему о внешнем угле треугольника.
		\item Докажите, что если медиана равна половине стороны, к которой она проведена, то такой треугольник прямоугольный.
		\item Докажите, что если треугольник вписан в окружность и одна из его сторон является диаметром этой окружности, то такой треугольник прямоугольный.
		\item Сформулируйте теорему о диаметре, проходящем через середину хорды.
	\end{enumcols}
	\item В треугольнике \( ABC \) обе стороны \( AB \) и \( BC \) равны \( 15 \). Чему равна сторона \( AC \), если \( \angle BAC = 60 \degree \)?
	\item В треугольнике \( ABC \) известно, что \( \angle A = 50 \) и \( \angle B = 80 \). Найдите сторону \( BC \), если \( AC = 16 \) и \( P_{ABC}=40 \).
	\item Угол треугольника равен \( 80\degree \). Найдите угол между высотами, проведенными из двух других углов.
	\item \exercise{2456}
	\item \exercise{2478}
\end{listofex}
\newpage
\title{Проверочная работа}
\title{Вариант 2}
\begin{listofex}
	\item
	\begin{enumcols}[itemcolumns=1]
		\item Чему равен угол между биссектрисами двух внутренних односторонних углов при параллельных прямых?
		\item Сформулируйте и докажите теорему о внешнем угле треугольника.
		\item Докажите, что если в треугольнике один угол равен сумме двух других, то такое треугольник прямоугольный.
		\item Сформулируйте теорему о диаметре, перпендикулярном хорде.
		\item Сформулируйте теорему о двух касательных, проведенных из одной точки к окружности.
	\end{enumcols}
	\item В треугольнике \( ABC \) обе стороны \( AB \) и \( BC \) равны \( 30 \). Чему равна сторона \( AC \), если \( \angle BAC = 60 \degree \)?
	\item В треугольнике \( ABC \) известно, что \( \angle A = 50 \) и \( \angle B = 80 \). Найдите сторону \( BC \), если \( AC = 20 \) и \( P_{ABC}=50 \).
	\item Угол треугольника равен \( 80\degree \). Найдите угол между высотами, проведенными из двух других углов.
	\item \exercise{2456}
	\item \exercise{2478}
\end{listofex}
\newpage
\title{Консультация}
\begin{listofex}
	\item Постройте график функции:
	\begin{enumcols}[itemcolumns=2]
		\item \( y=3x+2 \)
		\item \( y=-\dfrac{1}{2}x-1 \)
		\item \( y=-4 \)
		\item \( y=-0,25x+3 \)
	\end{enumcols}
	\item Постройте график функции:
	\begin{enumcols}[itemcolumns=2]
		\item \( y=x^2 \)
		\item \( y=\dfrac{1}{x} \)
	\end{enumcols}
	\item Найдите область определения:
	\begin{enumcols}[itemcolumns=2]
		\item \( y=1+\dfrac{1}{x+2} \)
		\item \( y=4-\dfrac{3}{2x-6} \)
	\end{enumcols}
	\item Найдите уравнение прямой, которая проходит через начало координат и точку \( A(7;1) \).
	\item Проходит ли график функции \( y=x^2+2x-3 \) через точку с координатами \( 5;32 \).
	\item Не выполняя построений, найдите координаты точек пересечения с осями координат графика функции \( y=-2,4x+9,6 \).
	\item Найдите координаты точки пересечения прямых, заданных уравнениями \( y=3x-7 \) и \( y=2 \).
	\item Найдите координаты точки пересечения прямых, заданных уравнениями \( y=2x+5 \) и\\ \( y=-\dfrac{1}{2}x-1 \).
\end{listofex}