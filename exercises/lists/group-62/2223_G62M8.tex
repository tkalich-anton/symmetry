%
%===============>>  ГРУППА 6-2 МОДУЛЬ 8  <<=============
%
\setmodule{8}

%BEGIN_FOLD % ====>>_____ Занятие 1-2 _____<<====
\begin{class}[number=1-2]
	\begin{definit}
		Раскрытие скобок --- это умножение общего множителя на каждое слагаемое в скобках. \[ a(b+c) = a \cdot b + a \cdot c \]
	\end{definit}
	\begin{definit}
		Если перед скобкой стоит знак минус, то при раскрытии скобок необходимо поменять знак каждому слагаемому, находящемуся в скобках, на противоположный. \[ -(a+b) = -a-b \]
	\end{definit}
	\begin{listofex}
		\item Вынесите за скобки общей множитель: %5.1 а-е
		\begin{tasks}(2)
			\task \( 7 \cdot 51 + 7 \cdot 29 \)
			\task \( 4,27 \cdot 35 - 20 \cdot 4,27 \)
			\task \( 3,75 \cdot 2,2 + 6,8 \cdot 2,2 \)
			\task \( 4,2 \cdot 6,7 - 13,4 \cdot 2 \)
			\task \( 16 \cdot 71 + 8 \cdot 4 \)
			\task \( 1,05 \cdot 38 - 2,1 \cdot 50 \)
		\end{tasks}
		\item Раскройте скобки:
		\begin{tasks}(2)
			\task \( 6 \cdot (11,5 - 4) \)
			\task \( 5 \cdot (3,3+11) \)
			\task \( -4 \cdot (3-16) \)
			\task \( 2 \cdot (-5,4-21) \)
			\task \( 26 \cdot \left( \mfrac{ 1 }{ 7 }{13}+\mfrac{ 7 }{5  }{39} \right) \)
			\task \( -3,5 \cdot \left( 7,7 + \mfrac{ 4 }{ 2 }{3} \right) \)
			\task \( \left( \dfrac{ 5 }{ 3 }+4,75 - \mfrac{ 5 }{ 7 }{12} \right) \cdot 12 \)
			\task \( -6 \cdot \left( \dfrac{ -4 }{ 2 }-\dfrac{ -5 }{ -2 } \right) \)
		\end{tasks}
	\end{listofex}
	\begin{definit}
		Если отрицательное число возводится в четную степень, то перед ним знак меняется на плюс: \[(-a)^{2n}=a^{2n}\] \\
		Если отрицательное число возводится в \textbf{не}четную степень, то перед ним остаётся знак минус: \[(-a)^{2n+1}=-a^{2n+1}\]
	\end{definit}
	\begin{listofex}[resume]
		\item Вычислите:
		\begin{tasks}(3)
			\task \( (-1)^2 \)
			\task \( (-1)^5 \)
			\task \( (-3)^4 \)
			\task \( (-2)^5 \)
			\task \( (-6)^2 \)
			\task \( (-5)^3 \)
		\end{tasks}
		\newpage
		\item Вычислите:
		\begin{tasks}(2)
			\task \( -11 + 21 \)
			\task \( -17+25 \)
			\task \( -37+65 \)
			\task \( -51-14 \)
			\task \( -26-37 \)
			\task \( -83-22 \)
			\task \( -\dfrac{1}{3}+\dfrac{1}{2} \)
			\task \( -\mfrac{2}{1}{6}+11 \)
			\task \( -1,5-\dfrac{3}{7} \)
			\task \( -\mfrac{3}{5}{21}+\mfrac{3}{17}{42}+\left(-\mfrac{18}{45}{66}\right) \)
			\task \( -15,2 + \left(-\mfrac{2}{17}{25}\right) + 12 \)
			\task \( -\mfrac{4}{2}{3} + \left(-\dfrac{22}{24}\right) - (-6) \)
		\end{tasks}
		%c46 1.187-1.191
		\item Автомобиль выехал из пункта \(A\) со скоростью \(60\) км/ч. Через \(2\) ч вслед за ним выехал второй автомобиль со скоростью \(90\) км/ч. Через какое время и на каком расстоянии от \(A\) второй автомобиль догонит первый?
		\item Собственная скорость катера \(25,5\) км/ч, скорость течения \(2,5\) км/ч. Какой путь пройдёт катер за полтора часа по течению и против течения?
		\item Собственная скорость лодки \(8,5\) км/ч, а скорость течения \(3,5\) км/ч. Расстояние между пристанями \(15\) км. Сколько времени затратит лодка на путь между пристанями туда и обратно?
		\item Плот и лодка движутся навстречу друг другу по реке. Они находятся на расстоянии \(20\) км друг другу по реке. Они находятся на расстоянии \(20\) км друг от друга. Через какое время они встретятся, если собственная скорость лодки \(8\) км/ч, а скорость течения реки \(2\) км/ч?
		\item Два велосипедиста одновременно выехали из лагеря в противоположных направлениях со скоростями \(10\) км/ч и \(12\) км/ч. Какое расстояние будет между ними через \(2\) ч? Через \(3\) ч \(6\) минут? Через какое время расстояние между ними будет равно \(33\) км?
	\end{listofex}
\end{class}
%END_FOLD

%BEGIN_FOLD % ====>>_ Домашняя работа 1 _<<====
\begin{homework}[number=1]
	\begin{listofex}
		\item Найдите и вынесите за скобки общие множители: %5.3 м н о п р с
		\begin{tasks}
			\task \( s+2s+4s \)
			\task \( 3a+7a+11a \)
			\task \( a+3b+2a+5b \)
			\task \( a-3c-4c+7a \)
			\task \( -x+3a-6a-7x \)
			\task \( x+3a+6b-2x-3a-b \)
		\end{tasks}
		\item Вычислите:
		\begin{tasks}(4)
			\task \( (-1)^{10} \)
			\task \( (-1)^7 \)
			\task \( (-2)^3 \)
			\task \( (-2)^6 \)
		\end{tasks}
		\item Вынесите за скобки общей множитель: %5.1 ё-и
		\begin{tasks}
			\task \( \mfrac{ 6 }{4  }{ 9 } \cdot \mfrac{ 5 }{ 6 }{ 29 }-\mfrac{ 6 }{4  }{ 9 } \cdot \mfrac{ 6 }{ 3 }{ 28 } \)
			\task \(37 \cdot 9+111\cdot27 \)
			\task \( 45 \cdot 19 + 75 \cdot9 \)
			\task \( 68 \cdot 95 + 35 \cdot 105 + 205 \cdot 34 \)
			
		\end{tasks}
		\item Вычислите:
		\begin{tasks}(2)
			\task \( -12 + 31 \)
			\task \( -15+11 \)
			\task \( -6+8,8 \)
			%\task \( -7,3-14 \)
			\task \( -15-\dfrac{2}{4} \)
			%\task \( -|11,1|+\mfrac{2}{3}{4} \)
			\task \( -\dfrac{1}{3}-\dfrac{1}{2} \)
			\task \( -\mfrac{2}{1}{6}-\left|-\mfrac{7}{5}{6}\right| \)
			%\task \( -1,5-\dfrac{3}{7} \)
			%\task \( -\mfrac{4}{8}{11}+\left|-\mfrac{2}{17}{21}\right|+\left(-\mfrac{3}{5}{11}\right) \)
			%\task \( |-1,2| + \left(-\mfrac{2}{17}{25}\right) - |-12| \)
			%\task \( -\mfrac{5}{1}{4} + \left(-\dfrac{3}{8}\right) - \left|-\mfrac{1}{5}{7}-\dfrac{11}{14}\right| \)
		\end{tasks}
	\end{listofex}
\end{homework}
%END_FOLD

%BEGIN_FOLD % ====>>_____ Занятие 3-4 _____<<====
\begin{class}[number=3-4]
	\begin{listofex}
		\item Раскройте скобки:
		\begin{tasks}(2)
			\task \( -(25+11x-45y) \)
			\task \( -(11x-3y)+14x \)
			\task \( -4(15y+2xa-4a) \)
			\task \( 4y+2x-0,8(-2x+6y) \)
			
			\task! \( 8 \cdot (0,4y+0,5x)+3 \cdot \left( -\dfrac{ 1 }{ 3 }x + y \right) \)
			\task! \( -0,25(4y+2x+8z)+(2y-4x)\cdot 2 \)
		\end{tasks}
		\item Вычислите:
		\begin{tasks}(3)
			\task \( (-1)^{250} \)
			\task \( (-1)^{10001} \)
			\task \( (-2)^{4} \)
			\task \( (-3)^3 \)
			\task \( (-2)^8 \)
			\task \( (-10)^2 \)
		\end{tasks}
		%c46 1.187-1.191
		\item Автомобиль выехал из пункта \(A\) со скоростью \(60\) км/ч. Через \(2\) ч вслед за ним выехал второй автомобиль со скоростью \(90\) км/ч. Через какое время и на каком расстоянии от \(A\) второй автомобиль догонит первый?
		\item Собственная скорость катера \(25,5\) км/ч, скорость течения \(2,5\) км/ч. Какой путь пройдёт катер за полтора часа по течению и против течения?
		\item Собственная скорость лодки \(8,5\) км/ч, а скорость течения \(3,5\) км/ч. Расстояние между пристанями \(15\) км. Сколько времени затратит лодка на путь между пристанями туда и обратно?
		\item Плот и лодка движутся навстречу друг другу по реке. Они находятся на расстоянии \(20\) км друг другу по реке. Они находятся на расстоянии \(20\) км друг от друга. Через какое время они встретятся, если собственная скорость лодки \(8\) км/ч, а скорость течения реки \(2\) км/ч?
		\item Два велосипедиста одновременно выехали из лагеря в противоположных направлениях со скоростями \(10\) км/ч и \(12\) км/ч. Какое расстояние будет между ними через \(2\) ч? Через \(3\) ч \(6\) минут? Через какое время расстояние между ними будет равно \(33\) км?
	\end{listofex}
\end{class}
%END_FOLD

%BEGIN_FOLD % ====>>_____ Занятие 4 _____<<====
\begin{class}[number=4]
	\begin{listofex}
		\item Занятие 4
	\end{listofex}
\end{class}
%END_FOLD

%BEGIN_FOLD % ====>>_ Домашняя работа 2 _<<====
\begin{homework}[number=2]
	\begin{listofex}
		\item Домашняя работа 2
	\end{listofex}
\end{homework}
%END_FOLD

%BEGIN_FOLD % ====>>_____ Занятие 5 _____<<====
\begin{class}[number=5]
	\begin{listofex}
		\item Занятие 5
	\end{listofex}
\end{class}
%END_FOLD

%BEGIN_FOLD % ====>>_____ Занятие 6 _____<<====
\begin{class}[number=6]
	\begin{listofex}
		\item Занятие 6
	\end{listofex}
\end{class}
%END_FOLD

%BEGIN_FOLD % ====>>_ Домашняя работа 3 _<<====
\begin{homework}[number=3]
	\begin{listofex}
		\item Домашняя работа 3
	\end{listofex}
\end{homework}
%END_FOLD

%BEGIN_FOLD % ====>>_____ Занятие 7 _____<<====
\begin{class}[number=7]
	\title{Подготовка к проверочной}
	\begin{listofex}
		\item Занятие 7
	\end{listofex}
\end{class}
%END_FOLD

=%BEGIN_FOLD % ====>>_ Проверочная работа _<<====
\begin{exam}
	\begin{listofex}
		\item Проверочная
	\end{listofex}
\end{exam}
%END_FOLD