%
%===============>>  ГРУППА 6-2 МОДУЛЬ 8  <<=============
%
\setmodule{8}

%BEGIN_FOLD % ====>>_____ Занятие 1-2 _____<<====
\begin{class}[number=1-2]
	\begin{definit}
		Раскрытие скобок --- это умножение общего множителя на каждое слагаемое в скобках. \[ a(b+c) = a \cdot b + a \cdot c \]
	\end{definit}
	\begin{definit}
		Если перед скобкой стоит знак минус, то при раскрытии скобок необходимо поменять знак каждому слагаемому, находящемуся в скобках, на противоположный. \[ -(a+b) = -a-b \]
	\end{definit}
	\begin{listofex}
		\item Вынесите за скобки общей множитель: %5.1 а-е
		\begin{tasks}(2)
			\task \( 7 \cdot 51 + 7 \cdot 29 \)
			\task \( 4,27 \cdot 35 - 20 \cdot 4,27 \)
			\task \( 3,75 \cdot 2,2 + 6,8 \cdot 2,2 \)
			\task \( 4,2 \cdot 6,7 - 13,4 \cdot 2 \)
			\task \( 16 \cdot 71 + 8 \cdot 4 \)
			\task \( 1,05 \cdot 38 - 2,1 \cdot 50 \)
		\end{tasks}
		\item Раскройте скобки:
		\begin{tasks}(2)
			\task \( 6 \cdot (11,5 - 4) \)
			\task \( 5 \cdot (3,3+11) \)
			\task \( -4 \cdot (3-16) \)
			\task \( 2 \cdot (-5,4-21) \)
			\task \( 26 \cdot \left( \mfrac{ 1 }{ 7 }{13}+\mfrac{ 7 }{5  }{39} \right) \)
			\task \( -3,5 \cdot \left( 7,7 + \mfrac{ 4 }{ 2 }{3} \right) \)
			\task \( \left( \dfrac{ 5 }{ 3 }+4,75 - \mfrac{ 5 }{ 7 }{12} \right) \cdot 12 \)
			\task \( -6 \cdot \left( \dfrac{ -4 }{ 2 }-\dfrac{ -5 }{ -2 } \right) \)
		\end{tasks}
	\end{listofex}
	\begin{definit}
		Если отрицательное число возводится в четную степень, то перед ним знак меняется на плюс: \[(-a)^{2n}=a^{2n}\] \\
		Если отрицательное число возводится в \textbf{не}четную степень, то перед ним остаётся знак минус: \[(-a)^{2n+1}=-a^{2n+1}\]
	\end{definit}
	\begin{listofex}[resume]
		\item Вычислите:
		\begin{tasks}(3)
			\task \( (-1)^2 \)
			\task \( (-1)^5 \)
			\task \( (-3)^4 \)
			\task \( (-2)^5 \)
			\task \( (-6)^2 \)
			\task \( (-5)^3 \)
		\end{tasks}
		\newpage
		\item Вычислите:
		\begin{tasks}(2)
			\task \( -11 + 21 \)
			\task \( -17+25 \)
			\task \( -37+65 \)
			\task \( -51-14 \)
			\task \( -26-37 \)
			\task \( -83-22 \)
			\task \( -\dfrac{1}{3}+\dfrac{1}{2} \)
			\task \( -\mfrac{2}{1}{6}+11 \)
			\task \( -1,5-\dfrac{3}{7} \)
			\task \( -\mfrac{3}{5}{21}+\mfrac{3}{17}{42}+\left(-\mfrac{18}{45}{66}\right) \)
			\task \( -15,2 + \left(-\mfrac{2}{17}{25}\right) + 12 \)
			\task \( -\mfrac{4}{2}{3} + \left(-\dfrac{22}{24}\right) - (-6) \)
		\end{tasks}
		%c46 1.187-1.191
		\item Автомобиль выехал из пункта \(A\) со скоростью \(60\) км/ч. Через \(2\) ч вслед за ним выехал второй автомобиль со скоростью \(90\) км/ч. Через какое время и на каком расстоянии от \(A\) второй автомобиль догонит первый?
		\item Собственная скорость катера \(25,5\) км/ч, скорость течения \(2,5\) км/ч. Какой путь пройдёт катер за полтора часа по течению и против течения?
		\item Собственная скорость лодки \(8,5\) км/ч, а скорость течения \(3,5\) км/ч. Расстояние между пристанями \(15\) км. Сколько времени затратит лодка на путь между пристанями туда и обратно?
		\item Плот и лодка движутся навстречу друг другу по реке. Они находятся на расстоянии \(20\) км друг другу по реке. Они находятся на расстоянии \(20\) км друг от друга. Через какое время они встретятся, если собственная скорость лодки \(8\) км/ч, а скорость течения реки \(2\) км/ч?
		\item Два велосипедиста одновременно выехали из лагеря в противоположных направлениях со скоростями \(10\) км/ч и \(12\) км/ч. Какое расстояние будет между ними через \(2\) ч? Через \(3\) ч \(6\) минут? Через какое время расстояние между ними будет равно \(33\) км?
	\end{listofex}
\end{class}
%END_FOLD

%BEGIN_FOLD % ====>>_ Домашняя работа 1 _<<====
\begin{homework}[number=1]
	\begin{listofex}
		\item Найдите и вынесите за скобки общие множители: %5.3 м н о п р с
		\begin{tasks}
			\task \( s+2s+4s \)
			\task \( 3a+7a+11a \)
			\task \( a+3b+2a+5b \)
			\task \( a-3c-4c+7a \)
			\task \( -x+3a-6a-7x \)
			\task \( x+3a+6b-2x-3a-b \)
		\end{tasks}
		\item Вычислите:
		\begin{tasks}(4)
			\task \( (-1)^{10} \)
			\task \( (-1)^7 \)
			\task \( (-2)^3 \)
			\task \( (-2)^6 \)
		\end{tasks}
		\item Вынесите за скобки общей множитель: %5.1 ё-и
		\begin{tasks}
			\task \( \mfrac{ 6 }{4  }{ 9 } \cdot \mfrac{ 5 }{ 6 }{ 29 }-\mfrac{ 6 }{4  }{ 9 } \cdot \mfrac{ 6 }{ 3 }{ 28 } \)
			\task \(37 \cdot 9+111\cdot27 \)
			\task \( 45 \cdot 19 + 75 \cdot9 \)
			\task \( 68 \cdot 95 + 35 \cdot 105 + 205 \cdot 34 \)
			
		\end{tasks}
		\item Вычислите:
		\begin{tasks}(2)
			\task \( -12 + 31 \)
			\task \( -15+11 \)
			\task \( -6+8,8 \)
			%\task \( -7,3-14 \)
			\task \( -15-\dfrac{2}{4} \)
			%\task \( -|11,1|+\mfrac{2}{3}{4} \)
			\task \( -\dfrac{1}{3}-\dfrac{1}{2} \)
			\task \( -\mfrac{2}{1}{6}-\left|-\mfrac{7}{5}{6}\right| \)
			%\task \( -1,5-\dfrac{3}{7} \)
			%\task \( -\mfrac{4}{8}{11}+\left|-\mfrac{2}{17}{21}\right|+\left(-\mfrac{3}{5}{11}\right) \)
			%\task \( |-1,2| + \left(-\mfrac{2}{17}{25}\right) - |-12| \)
			%\task \( -\mfrac{5}{1}{4} + \left(-\dfrac{3}{8}\right) - \left|-\mfrac{1}{5}{7}-\dfrac{11}{14}\right| \)
		\end{tasks}
	\end{listofex}
\end{homework}
%END_FOLD

%BEGIN_FOLD % ====>>_____ Занятие 3-4 _____<<====
\begin{class}[number=3-4]
	\title{Урок 3}
	\begin{listofex}
		\item Вычислите \[ \mfrac{4}{ 5 }{ 8 } - \mfrac{ 6 }{ 2 }{9} + \left( -\mfrac{ 5 }{ 1 }{6} - 4,8 \right)+11,2 \]
		\item Раскройте скобки:
		\begin{tasks}(2)
			\task \( -(25+11x-45y) \)
			\task \( -(11x-3y)+14x \)
			\task \( -4(15y+2xa-4a) \)
			\task \( 4y+2x-0,8(-2x+6y) \)
			\task! \( 8 \cdot (0,4y+0,5x)+3 \cdot \left( -\dfrac{ 1 }{ 3 }x + y \right) \)
			\task! \( -0,25(4y+2x+8z)+(2y-4x)\cdot 2 \)
		\end{tasks}
		
		\item Вычислите:
		\begin{tasks}(3)
			\task \( (-1)^{250} \)
			\task \( (-1)^{10001} \)
			\task \( (-2)^{4} \)
			\task \( (-3)^3 \)
			\task \( (-2)^8 \)
			\task \( (-10)^2 \)
		\end{tasks}
		\item Решите уравнения:
		\begin{tasks}(2)
			\task \( \dfrac{ -20x+18,5 }{ 34 }=\dfrac{ -5,5-4x }{ 8,5 } \)
			\task \( \dfrac{ -3,2x+\dfrac{ 1 }{ 3 } }{ 15 }=\dfrac{ 0,5x-22 }{ -2,5 } \)
		\end{tasks}
		
	\end{listofex}
	\newpage
	\title{Урок 4}
	\begin{listofex}
		%c46 1.187-1.191
		\item Автомобиль выехал из пункта \(A\) со скоростью \(60\) км/ч. Через \(2\) ч вслед за ним выехал второй автомобиль со скоростью \(90\) км/ч. Через какое время и на каком расстоянии от \(A\) второй автомобиль догонит первый?
		\item Собственная скорость катера \(25,5\) км/ч, скорость течения \(2,5\) км/ч. Какой путь пройдёт катер за полтора часа по течению и против течения?
		\item Собственная скорость лодки \(8,5\) км/ч, а скорость течения \(3,5\) км/ч. Расстояние между пристанями \(15\) км. Сколько времени затратит лодка на путь между пристанями туда и обратно?
		\item Плот и лодка движутся навстречу друг другу по реке. Они находятся на расстоянии \(20\) км друг другу по реке. Через какое время они встретятся, если собственная скорость лодки \(8\) км/ч, а скорость течения реки \(2\) км/ч?
		\item Два велосипедиста одновременно выехали из лагеря в противоположных направлениях со скоростями \(10\) км/ч и \(12\) км/ч. Какое расстояние будет между ними через \(2\) ч? Через \(3\) ч \(6\) минут? Через какое время расстояние между ними будет равно \(33\) км?
	\end{listofex}
\end{class}
%END_FOLD

%BEGIN_FOLD % ====>>_ Домашняя работа 2 _<<====
\begin{homework}[number=2]
	\begin{listofex}
		\item Раскройте скобки:
		\begin{tasks}(2)
			\task \( -(15x+2y-4) \)
			\task \( -4 \cdot (0,14x-0,25+1,3) \)
			\task \( -2 \cdot(6,6x-11,5 -|-5y|) \)
			\task \( -0,5 \cdot \left( \dfrac{ 2x }{ 3 }-13,2 \right) \)
		\end{tasks}
		\item Вычислите:
		\begin{tasks}(4)
			\task \( (-1)^7 \)
			\task \( (-3)^{4} \)
			\task \( (-4)^3 \)
			\task \( (-10)^{4} \)
		\end{tasks}
		\item Решите уравнение: \( \dfrac{ 0,2-14x }{ 0,4 }=\dfrac{ -1,5x+4,4 }{ -1,6 } \)
		%1.197
		\item Расстояние между станциями \(A\) и \(B\) равно \(165\) км. От этих станций одновременно навстречу друг другу выходят два поезда и встречаются через \(1,5\) ч на разъезде, который находится в \(90\) км от станции \(A\). С какой скоростью идут поезда?
	\end{listofex}
\end{homework}
%END_FOLD

%BEGIN_FOLD % ====>>_____ Занятие 5 _____<<====
\begin{class}[number=5]
	\title{Урок 5}
	\begin{listofex}
		\item Раскройте скобки:
		\begin{tasks}(2)
			\task \( -2(3x+8y-11,3) \)
			\task \( (4,6-25x-0,6y) \cdot \dfrac{ -1 }{ 4 } \)
			\task \( (-9,5y+4x-2) \cdot (-3x) \)
			\task \( -0,5(52x-16y-\dfrac{ 4 }{ 7 }) \)
			\task \( -5y-4,2x+0,75(4x-0,8y) \)
			\task \( \mfrac{1 }{ 2}{3 } \cdot \left( -6x - 1,2 \right) \)
			\task \( 7 \cdot (2x-1,4) - (5y+22,7) \)
			\task \( 4 \cdot \left( \mfrac{3 }{7 }{12 }x - \dfrac{ y }{ 8 } \right) \)
		\end{tasks}
		\item Решите уравнения:
		\begin{tasks}(2)
			\task \( \dfrac{ 4x }{ -5 }=\dfrac{ 1,5x }{ 10 } \)
			\task \( \dfrac{ 7,8x-5 }{ 16 }=\dfrac{ -3-14x }{ 8 } \)
			\task \( 2x-14=8,17 \)
			\task \( -2,5x-4=\mfrac{3 }{6 }{25 } \)
			\task \( 5x-8,4=-14,75+8,25x \)
			\task \( -4,8x-3=11,7x-12,56 \)
		\end{tasks}
		\item Вычислите:
		\begin{tasks}(3)
			\task \( (-4)^{4} \)
			\task \( (-2)^{8} \)
			\task \( -2^3 \)
			\task \( -1^{15}+(-1)^{28} \)
			\task \( -2^{4}+(-2)^{4} \)
			\task \( -3^{2}+(-3)^{3} \)
		\end{tasks}
	\end{listofex}
	\newpage
	\title{Урок 6}
		\begin{listofex}
			%c46 1.187-1.191
			\item Автомобиль выехал из пункта \(A\) со скоростью \(60\) км/ч. Через \(2\) ч вслед за ним выехал второй автомобиль со скоростью \(90\) км/ч. Через какое время и на каком расстоянии от \(A\) второй автомобиль догонит первый?
			\item Собственная скорость катера \(25,5\) км/ч, скорость течения \(2,5\) км/ч. Какой путь пройдёт катер за полтора часа по течению и против течения?
			\item Плот и лодка движутся навстречу друг другу по реке. Они находятся на расстоянии \(20\) км друг другу по реке. Через какое время они встретятся, если собственная скорость лодки \(8\) км/ч, а скорость течения реки \(2\) км/ч?
			\item Два велосипедиста одновременно выехали из лагеря в противоположных направлениях со скоростями \(10\) км/ч и \(12\) км/ч. Какое расстояние будет между ними через \(2\) ч? Через \(3\) ч \(6\) минут? Через какое время расстояние между ними будет равно \(33\) км?
	\end{listofex}
\end{class}
%END_FOLD

%BEGIN_FOLD % ====>>_ Домашняя работа 3 _<<====
\begin{homework}[number=3]
	\begin{listofex}
		\item Раскройте скобки:
		\begin{tasks}(2)
			\task \( -5 \cdot (11x-0,8y+4) \)
			\task \( \dfrac{ 1 }{ 3 } \cdot (5,4-3y-0,9x) \)
			\task \( -\dfrac{ 2 }{ 4 } \left( 0,2y-11x + 3 \right) \)
			\task \( -1,5 \cdot (0,4y-6,25x-12,4) \)
		\end{tasks}
		\item Решите уравнения:
		\begin{tasks}(2)
			\task \( 3x-5=-11 \)
			\task \( -8y-4,4=-12,2 \)
			\task \( \dfrac{ 3 }{ 5 }x-2,8=-\dfrac{ 9 }{ 15 } \)
			\task \( -0,7x-\mfrac{3 }{2 }{5 }=11,4 \)
		\end{tasks}
		%1.192
		\item Два велосипедиста выехали одновременно из двух сёл навстречу друг другу и встретились через \(1,6\) ч. Скорость первого \(10\) км/ч, а второго --- \(12\) км/ч. Найдите расстояние между сёлами.
	\end{listofex}
\end{homework}
%END_FOLD

%BEGIN_FOLD % ====>>_____ Занятие 7 _____<<====
\begin{class}[number=7-8]
	\title{Урок 7}
	\begin{listofex}
		\item Два поезда одновременно вышли с одной станции в одном направлении. Их скорости \(60\) км/ч и \(70\) км/ч. Какое расстояние будет между ними через \(1,5\) часа? Через \(2\) часа \(25\) мин? Через сколько часов расстояние между ними будет равно \(35\) км?
		\item Города \(A\) и \(B\) расположены на реке, причём \(B\) ниже по течению. Расстояние между ними равно \(30\) км. Моторная лодка проходит путь от \(A\) до \(B\) за \(2\) ч, а обратно за \(3\) ч. За какое время проплывёт от \(A\) до \(B\) плот?
		\item Пассажир метро, стоя на ступеньке эскалатора, поднимается наверх за \(3\) мин. За сколько минут он поднимется вверх по движущемуся эскалатору, если будет идти со скоростью \(25\) м/мин? Длина эскалатора \(150\) м.
		\item Расстояние между станциями \(350\) км. От этих станций одновременно навстречу друг другу вышли два поезда. Они встретились через \(2,5\) часа. Определите скорость первого поезда, если скорость второго равна \(65\) км.
	\end{listofex}
	\title{Урок 8}
	\begin{listofex}[resume]
		\item Расстояние между станциями \(A\) и \(B\) равно \(165\) км. От этих станций одновременно навстречу друг другу выходят два поезда и встречаются через \(1,5\) ч на разъезде, который находится в \(90\) км от станции \(A\). С какой скоростью идут поезда?
		\item Из двух городов, расстояние между которыми \(45\) км, одновременно в одном направлении вышли поезда со скоростями \(70\) км/ч и \(60\) км/ч, причём первый поезд догоняет второй. Через сколько времени расстояние между поездами будет равно \(10\) км? Сколько решений имеет задача?
		\item Два поезда выехали одновременно из пунктов \(A\) и \(B\) навстречу друг другу. Расстояние между пунктами \(A\) и \(B\) равно \(350\) км. Скорость первого \(65\) км/ч, второго --- \(75\) км/ч. Через сколько часов расстояние между поездами составит \(70\) км? Сколько решений имеет задача?
		\item Города \(A\) и \(B\) расположены на одном шоссе. Из этих городов одновременно в одном направлении выехали два автобуса. Первый автобус двигался со скоростью \(54\) км/ч, что составляет \(0,6\) скорости второго автобуса. Второй автобус догнал первый через \(1\) ч \(30\) мин после выезда. Каково расстояние между городами \(A\) и \(B\)? На каком расстоянии друг от друга были автобусы через \(24\) мин после выезда? Через \(2\) ч после выезда?
		%до 1.201
		\item Решите уравнения:
		\begin{tasks}(2)
			\task \( |x-5|=11 \)
			\task \( |2x+14|=3,1 \)
			\task \( |0,8-10x|=2,5 \)
			\task \( \left| \dfrac{ x }{ 5 }-8 \right|=\dfrac{ 17 }{ 25 } \)
		\end{tasks}
	\end{listofex}
\end{class}
%END_FOLD

%BEGIN_FOLD % ====>>_ ДЗ 4 _<<====
\begin{homework}[number=4]
	\begin{listofex}
		% 1.208-1.211
		\item Два пешехода вышли одновременно навстречу друг другу из пунктов \(A\) и \(B\). При встрече оказалось, что первый пешеход прошел \(\dfrac{ 1 }{ 4 }\) всего пути и еще \(3,2\) км, а второй --- в \(2\) раза больше первого. Чему равно расстояние от \(A\) до \(B\)?
		\item Из города в \(8\) часов утра выехал велосипедист со скоростью \(20\) км/ч. Через \(4\) часа велосипедист сделал часовой привал, а в этот момент вслед за ним из города выехал мотоциклист со скоростью \(50\) км/ч. В какое время мотоциклист догонит велосипедиста? На каком расстоянии от города это произойдёт? Какое расстояние будет между велосипедистом и мотоциклистом через в \(6\) часов вечера?
		\item Из города \(A\) в город \(B\), расстояние между которыми \(620\) км выехала легковая машина со скоростью \(60\) км/ч. Через два часа из города \(B\) в город \(A\) выехал грузовик со скоростью \(40\) км/ч.
		\item Решите уравнения:
		\begin{tasks}(2)
			\task \( -50,3+|2x|=11,7 \)
			\task \( \left| -x+\dfrac{ 5 }{ 3 } \right|=|-4| \)
			\task \( |x|=-|0,5-2|+\mfrac{3 }{4 }{5 } \)
			\task \( \left|\dfrac{ -x }{ 3 } \right|=\dfrac{ 8 }{ 9 } \)
		\end{tasks}
	\end{listofex}
\end{homework}
%END_FOLD