%
%===============>>  ГРУППА 6-2 МОДУЛЬ 6  <<=============
%
\setmodule{6}

%BEGIN_FOLD % ====>>_____ Занятие 1 _____<<====
\begin{class}[number=1-2]
	\title{Занятие 1}
	\begin{listofex}
		\item Вычислить:
		\begin{tasks}(2)
			\task \( 3,12+\dfrac{1}{4} \)
			\task \( 5,404+\dfrac{13}{25} \)
			\task \( \mfrac{3}{4}{50}-1,57\)
			\task \( \mfrac{2}{1}{10}+2,2 \)
			\task \( \mfrac{4}{1}{7}+1,66 \)
			\task \( \mfrac{12}{13}{15}-10,6 \)
			\task \( \mfrac{2}{5}{18}-1,1\)
			\task \( \mfrac{1}{1}{3}+35,7 \)
		\end{tasks}
		\item Вычислить:
		\begin{tasks}(4)
			\task \( 1,5 : \dfrac{1}{3} \)
			\task \( \dfrac{3}{8} : 0,25 \)
			\task \( 3,2 \cdot \dfrac{5}{8} \)
			\task \( 3,6 \cdot \mfrac{2}{7}{9} \)
			\task \( \mfrac{5}{2}{5} \cdot  \mfrac{2}{7}{9} \)
			\task \( \mfrac{2}{11}{12} \cdot 0,1 \)
			\task \( 1,45 : \mfrac{2}{1}{3} \)
			\task \( 20 : \mfrac{33}{1}{3} \)
		\end{tasks}
		\item Вычислить:
		\begin{tasks}(1)
			\task \( \left(  \mfrac{2}{3}{4} + \mfrac{4}{1}{8} \right) \cdot \mfrac{1}{5}{11} \)
			\task \( 16 \cdot \left( \mfrac{5}{8}{21} - \mfrac{3}{79}{84} \right) + 15 \cdot \left( \dfrac{1}{2} + \mfrac{2}{1}{3} \cdot \mfrac{1}{4}{5} \right) \)
			\task \( \left( \mfrac{1}{2}{13} \cdot 0,42 + 0,78 \cdot \mfrac{1}{2}{13} \right) \cdot \mfrac{1}{4}{9} : 0,6 - 0,5 \cdot \mfrac{5}{2}{3} \)
		\end{tasks}
	\newpage
	\title{Занятие 2}
	\end{listofex}
	\begin{definit}
		Сумма внутренних углов в треугольнике равна \( 180\degree \).
	\end{definit}
	\begin{definit}
		\textbf{Внешний угол} --- угол между стороной треугольника и продолжением другой стороны. Внешний угол является смежным с одним из внутренних.
	\end{definit}
	\begin{listofex}
		\item В треугольнике \( ABC \) два угла равны \( 50\) и \( 70 \) градусов. Найдите третий угол.
		\item Один угол треугольника равен \( 26\degree \), а второй в три раза больше. Найдите третий угол.
		\item Один внутренний угол треугольника в два, а второй в три раза больше третьего, найдите все углы треугольника.
		\item Один внешний угол равен \( 40\degree \), а второй --- \( 100\degree \). Чему равны внутренние углы треугольника?
		\item Угол треугольника равен \( 30\degree \), второй угол в \( 3 \) раза больше первого. Чему равны внешние углы при каждой вершине? Чему равна сумма внешних углов?
		\item В прямоугольном треугольнике один угол равен \( 40 \) градусов. Найдите сумму наибольшего и наименьшего угла.
		\item В прямоугольном треугольнике один острый угол на \( 17 \) градусов больше другого. Найдите углы треугольника.
		\item В прямоугольном треугольнике два острых угла равны. Какая у них градусная мера?
	\end{listofex}
\end{class}
%END_FOLD

%%BEGIN_FOLD % ====>>_____ Занятие 2 _____<<====
%\begin{class}[number=2]
%	\begin{listofex}
%		\item Занятие 2
%	\end{listofex}
%\end{class}
%%END_FOLD

%BEGIN_FOLD % ====>>_ Домашняя работа 1 _<<====
\begin{homework}[number=1]
	\begin{listofex}
		\item В треугольнике \( ABC \) два угла равны \( 130\) и \( 15 \) градусов. Найдите третий угол.
		\item Один угол треугольника равен \( 31\degree \), а второй в четыре раза больше. Найдите третий угол треугольника.
		\item Один внешний угол равен \( 15\degree \), а второй --- \( 120\degree \). Чему равны внутренние углы треугольника?
		\item В прямоугольном треугольнике один острый угол равен \( 32 \) градуса. Найдите сумму наибольшего и наименьшего угла.
		\item В прямоугольном треугольнике один угол в два раза меньше другого. Какими могут быть эти углы?
		\item Вычислить:
		\begin{tasks}(2)
			\task \( 3,12+\dfrac{1}{4} \)
			\task \( 5,404-\mfrac{4}{21}{25} \)
			\task \( \mfrac{1}{5}{12}-1,3\)
			\task \( 4 : \dfrac{10}{4} \)
			\task \( \dfrac{1}{8} \cdot 4,25 \)
			\task \( 1,5 \cdot \dfrac{2}{3} \)
			\task \( \left(  \mfrac{1}{5}{6} + \mfrac{2}{11}{12} \right) \cdot 12 \)
		\end{tasks}
		\item \( (*) \) \(4\) кузнеца должны подковать \(5\) лошадей. Каждый кузнец тратит на одну подкову \(5\) минут. Какое наименьшее время они должны потратить на работу? Учтите, что лошадь не может стоять на двух ногах.
	\end{listofex}
\end{homework}
%END_FOLD

%BEGIN_FOLD % ====>>_____ Занятие 3 _____<<====
\begin{class}[number=3]
	\begin{listofex}
		\item Занятие 3 
	\end{listofex}
\end{class}
%END_FOLD

%BEGIN_FOLD % ====>>_____ Занятие 4 _____<<====
\begin{class}[number=4]
	\begin{listofex}
		\item Занятие 4
	\end{listofex}
\end{class}
%END_FOLD

%BEGIN_FOLD % ====>>_ Домашняя работа 2 _<<====
\begin{homework}[number=2]
	\begin{listofex}
		\item Домашняя работа 2
	\end{listofex}
\end{homework}
%END_FOLD

%BEGIN_FOLD % ====>>_____ Занятие 5 _____<<====
\begin{class}[number=5]
	\begin{listofex}
		\item Занятие 5
	\end{listofex}
\end{class}
%END_FOLD

%BEGIN_FOLD % ====>>_____ Занятие 6 _____<<====
\begin{class}[number=6]
	\begin{listofex}
		\item Занятие 6
	\end{listofex}
\end{class}
%END_FOLD

%BEGIN_FOLD % ====>>_ Домашняя работа 3 _<<====
\begin{homework}[number=3]
	\begin{listofex}
		\item Домашняя работа 3
	\end{listofex}
\end{homework}
%END_FOLD

%BEGIN_FOLD % ====>>_____ Занятие 7 _____<<====
\begin{class}[number=7]
	\title{Подготовка к проверочной}
	\begin{listofex}
		\item Занятие 7
	\end{listofex}
\end{class}
%END_FOLD

%BEGIN_FOLD % ====>>_ Проверочная работа _<<====
\begin{exam}
	\begin{listofex}
		\item Проверочная
	\end{listofex}
\end{exam}
%END_FOLD