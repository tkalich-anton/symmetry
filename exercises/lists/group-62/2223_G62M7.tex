%
%===============>>  ГРУППА 6-2 МОДУЛЬ 7  <<=============
%
\setmodule{7}

%BEGIN_FOLD % ====>>_____ Занятие 1-2 _____<<====
\begin{class}[number=1-2]
	\begin{listofex}
		\item Найдите сумму чисел, противоположных данным:
		\begin{tasks}(2)
			\task \( 13  \) и \( 14 \)
			\task \( -8,2  \) и \( 1,5 \)
			\task \( \dfrac{16}{5}  \) и \( \mfrac{3}{4}{15} \)
			\task \( ((-0,6) \cdot (-9))  \) и \( \left(-\mfrac{5}{4}{5}+4\right) \)
		\end{tasks}
		
	\end{listofex}
	\begin{definit}
		Уравнение – это равенство, содержащее неизвестную величину. Корнем уравнения называется число, при подстановке которого в уравнение вместо неизвестной величины получается верное числовое равенство. Решить уравнение – значит найти все его корни или доказать, что других корней нет.
	\end{definit}
	\begin{listofex}[resume]
		\item Проверьте, какие из чисел являются корнем уравнения:
		\begin{tasks}(1)
			\task \( 2x^2+5x=7, x=0, x=1, x=2 \)
			\task \( 2x^2+2=5, x=0, x=0,5, x=1, x=2, x=2,5 \)
			\task \( x^3 +3x-5x+1=0, x=0, x=1, x=2 \)
			\task \( (x-4)(x-5)=0, x=0, x=1, x=2, x=3, x=4, x=5, x=6 \)
		\end{tasks}
		\item Решите уравнения: %1.306 6-57 1.308 1.310
		\begin{tasks}(3)
			\task \( x+39=61 \)
			\task \( y+53=73 \)
			\task \( k+87=87 \)
			\task \( x+0=99 \)
			\task \( x+3,4=6,15 \)
			\task \( y+\dfrac{5}{14}=\dfrac{11}{21} \)
			\task \( \dfrac{3}{4}+n=\dfrac{5}{6} \)
			\task \( x+0,7=\dfrac{13}{15} \)
			\task \( y+7,1=\mfrac{9}{1}{15} \)
			\task \( x-54=178 \)
			\task \( y-86=125,6 \)
			\task \( x-\dfrac{5}{12}=\dfrac{7}{8} \)
			
			
			\task \( 12x=144 \)
			\task \( 13y=169 \)
			\task \( y \cdot 46=782 \)
			\task \( 59y=118472 \)
			
			
			\task \( 48 : x =4 \)
			\task \( 41112 : x =58 \)
			\task \( 28014 : x = 46 \)
			\task \( 13 : x =1 \)
			\task \( 12 : x = 0,018 \)
			\task \( 6:x = 0,015 \)
			\task \( 0:x = 0 \)
			\task \( \dfrac{8}{x}=1,5 \)
		\end{tasks}
		\item Решите уравнения:
		\begin{tasks}(3) %1.314
			\task \( 7(x-9)=14 \)
			\task \( 8(x-11)=64 \)
			\task \( 3(x-1)=3 \)
			\task \( 13(x-2)=26 \)
			\task \( 0,3(x-1)=12 \)
			\task \( 0,7(x-7)=43 \)
			\task \( \dfrac{10}{11}(x-8)=0,1 \)
			\task \( \dfrac{5}{7}(x-5)=1,4 \)
		\end{tasks}
		\item Решите уравнение: \[ \left( \mfrac{3}{1}{3}x-\mfrac{2}{5}{7} \right):2,5 = \dfrac{86}{105}  \]
		%\begin{tasks}
		%	\task \(  \)
		%	\task \(  \)
		%	\task \(  \)
		%	\task \(  \)
		%\end{tasks}
	\end{listofex}
\end{class}
%END_FOLD

%BEGIN_FOLD % ====>>_ Домашняя работа 1 _<<====
\begin{homework}[number=1]
	\begin{listofex}
		\item Вычислите:
		\begin{tasks}(2)
			\task \( -15 + -\dfrac{1}{8} \)
			\task \( -8,3-(-16,4) \)
			\task \( -(-37+65)+(-3 \cdot |-6|) \)
			\task \( -(-21-24) \cdot |-5 \cdot -3| \)
			\task \( \dfrac{|-18|}{-7+14} \)
			
			\task \( \left| \dfrac{5}{-6}-\dfrac{2}{3} \right| \cdot (-|-21|) \)
			\task \(  -\left|\dfrac{-5}{8}\right|-|\dfrac{-3}{4}| \)
			\task \(  -\left| \dfrac{-18}{9} \right| + |-7,5-|-1,2|| \)
			
		\end{tasks}
		\item Сравните рациональные числа:
		\begin{tasks}(3)
			\task \( |-6,5| \) и \( -|0,5| \)
			\task \( -\dfrac{6}{21} \) и \( -\dfrac{19}{63} \)
			\task \( 5,5 \) и \( |-5,51| \)
			\task \( -|-14,4| \) и \( -14,45 \)
			\task \( -\dfrac{5}{-11} \) и \( \left| -\dfrac{6}{33} \right| \)
			\task \( |11,5| \) и \( |-11,22| \)
		\end{tasks}
		\item Решите уравнения:
		\begin{tasks}(2)
			\task \( x-4=-10 \)
			\task \( x+3=-5 \)
			\task \( -15+x=-8 \)
			\task \( -16-x=-4 \)
		\end{tasks}
	\end{listofex}
\end{homework}
%END_FOLD

%BEGIN_FOLD % ====>>_____ Занятие 3 _____<<====
\begin{class}[number=3-4]
	\begin{definit}
		Длина окружности --- длина замкнутой кривой, ограничивающей круг. Вычисляется по формуле: \( C = 2 \pi R \), где \(R\) --- радиус окружности.
	\end{definit}
	\begin{definit}
		Площадь окружности: \( S = \pi R^2 \), где \(R\) --- радиус окружности
	\end{definit}
	\begin{listofex}
		\item Найдите площадь и длину окружности, если диаметр окружности равен \( \dfrac{8}{\pi} \).
		\item Найдите площадь окружности, если её радиус равен \( 2 \) см.
		\item Найдите радиус окружности, если её длина равна \( 11\pi \) см.
		\item Диаметр окружности равен \(5\) см, найдите площадь окружности.
		\item Две окружности, имеющие общий центр, образуют кольцо. Радиус внешней окружности равен \(10\) см, а внутренней \(8\) см. Найти площадь этого кольца.
		\item Найдите площадь окружности, если длина окружности равна \(6\) см.
		\item Вычислите:
		\begin{tasks}(3)
			\task \( -17 + 21 \)
			\task \( -8,5 - 0,6 \)
			\task \( \dfrac{1}{6} - 13,5 \)
			\task \( -4 + \mfrac{7}{4}{5} \)
			\task \( -17 + \dfrac{5}{-3} \)
			\task \( -10,33 - (-11,72) \)
			\task \( 15 \cdot (-4) \)
			\task \( -5 \cdot \dfrac{4}{-5} \)
			\task \( -11 \cdot (-0,5) \)
			\task \( -\dfrac{1}{3} \cdot \mfrac{1}{1}{2} \)
			\task \( -\dfrac{5}{3}\cdot \dfrac{-1}{-18} \)
			\task \( 14 : (-0,7) \)
			\task \( 0,16 : (-0,4) \)
			\task \( 42 : (-3,5) \)
			\task \( -\dfrac{8}{25} : (-0,125) \)
			\task \( 7,11 : (-711) \)
		\end{tasks}
	\end{listofex}
\end{class}
%END_FOLD

%BEGIN_FOLD % ====>>_ Домашняя работа 2 _<<====
\begin{homework}[number=2]
	\begin{listofex}
		\item Найдите площадь и длину окружности, если радиус окружности равен \( \dfrac{16}{\pi} \) см.
		\item Найдите радиус окружности, если её длина равна \( 6\pi \) см.
		\item Диаметр окружности равен \(10\pi\) см, найдите площадь окружности.
		\item Две окружности имеют общий центр. Диаметр внешней окружности равен \(4\) см, а внутренней \(2,5\) см. Найти площадь кольца, которое образуется между окружностями.
		\item Вычислите:
		\begin{tasks}(3)
			
			\task \( 14,6-23,5 \)
			\task \( -78,15 + 16,22 \)
			\task \( |-8,5| - |-15,16| \)
			
			\task \( -8 \cdot (-0,25)\)
			
			\task \( -28 : (-7) \)
			
			\task \( \dfrac{|-4|}{-|2|} \cdot 0,25 \)
			
		\end{tasks}
		\item Решите уравнения:
		\begin{tasks}(2)
			\task \( 85-x=25 \)
			
			\task \( -\left( -\dfrac{8}{15} + x \right) = -4 \)
			
			\task \( |-15|:(-3)=x+2 \)
			
			\task \( \dfrac{1}{x} = -|6-25,3| \)
			
			%\task \( \dfrac{18}{x} = \dfrac{3}{9} \)
			\task \( \dfrac{x}{4} = -\dfrac{24}{11} \)
			%\task \( \dfrac{17}{x} = \dfrac{34}{-28} \)
			\task \( \dfrac{-x}{15} = \left| -\dfrac{5}{6} \right| \)
		\end{tasks}
	\end{listofex}
\end{homework}
%END_FOLD

%BEGIN_FOLD % ====>>_____ Занятие 5-6 _____<<====
\begin{class}[number=5-6]
	\begin{listofex}
		\item Вычислите: %старые
		\begin{tasks}(2)
			\task \( 46 : (-|23|) \)
			\task \( -(-12 + 4) \cdot |-5,5| \)
			\task \( 6,36 \cdot \dfrac{-1}{6} \)
			\task \( \dfrac{-5}{16} \cdot (-0,8) \)
			\task \( -\dfrac{15}{-4} + |4,34| \)
			\task \( \dfrac{|-11,5| + 13,5}{|-5 - 2,5|} + (-17)\cdot |-1+4| \)
			\task \( -\mfrac{3}{5}{21}+\mfrac{3}{17}{42}+\left(-\mfrac{18}{45}{66} \right) \)
		\end{tasks}
		\item Решите уравнения:
		\begin{tasks}(2)
			\task \( 93x=0 \)
			\task \( 7x=15 \)
			\task \( \dfrac{2}{7}x=\dfrac{6}{49} \)
			\task \( \dfrac{3}{7}y=\dfrac{9}{35} \)
			\task \( m-\dfrac{19}{22}=3,2 \)
			\task \( 11,6-x=35,11 \)
			\task \( -\dfrac{4}{10}+x=-0,27 \)
			\task \( \mfrac{11}{5}{6}-p=\mfrac{4}{7}{8} \)
			\task \( 8,3-x=\mfrac{7}{17}{25} \)
			\task \( 6,7-x=\mfrac{5}{19}{20} \)
			\task \( \dfrac{-4}{x}=116 \)
			\task \( -\dfrac{-y}{-18}=\dfrac{1}{3} \)
			\task \( -15,6:y=\dfrac{3}{-11} \)
			\task \( \dfrac{x}{-0,72}=1,2 \cdot 0,4 \)
			\task \( 15x=x-28 \)
			\task \( 3x+5=-14-5x \)
			
			\task \( 7-\dfrac{3}{5}x=|-18,5|-6x \)
			\task \( 0,25x-|-3,8| \cdot 2 = 2,6 - 0,8x \)
			\task \( |-9 + 14,76| = -x+28 \)
			\task \( 0,5 \cdot x-16,3=- |-11-2| \)
		\end{tasks}
		\item Разделите число:
		\begin{tasks}(2)
			\task \( 156 \) в отношении \( 2:1 \)
			\task \( 270 \) в отношении \( 5:3:2 \)
			\task \( 2160 \) в отношении \( 4:5:6 \)
			\task \( 1210 \) в отношении \( 2,5:4:5,5 \)
		\end{tasks}
		\item Изобразите числа на числовой прямой:
		\begin{tasks}(4)
			\task \( -5 \)
			\task \( -3,5 \)
			\task \( 2 \)
			\task \( 0,6 \)
			\task \( -0,1 \)
			\task \( -1,5 \)
			\task \( 0,3 \)
			\task \( -3,8 \)
			\task \( 0,25 \)
			\task \( 0,33 \)
			\task \( -0,15 \)
			\task \( -1,56 \)
		\end{tasks}
		\item Диаметр окружности равен \(12\) см, найдите длину и площадь окружности.
		\item Найдите длину окружности, если длина окружности равна \(\dfrac{ 5 }{ \pi }\) см.
		\item Найдите радиус окружности, если её площадь равна \( 11\pi \) см.
		\item Две окружности с одним центром с радиусами соответственно \( 10\) и \(12\) см образуют кольцо. Найдите площадь этого кольца.
	\end{listofex}
\end{class}
%END_FOLD


%BEGIN_FOLD % ====>>_ Домашняя работа 3 _<<====
\begin{homework}[number=3]
	\begin{listofex}
		\item Вычислите:
		\begin{tasks}(2)
			\task \( |-15| \cdot (-2) \)
			\task \( \left| \dfrac{ -6 }{ 3 } \right|-14,5  \)
			\task \( -11,76 + |-3,2| \)
			\task \( -1,5 : (-|-15|) \)
			\task \( -6 \cdot |-3 \cdot (-3,2)| \)
			\task \( \dfrac{ -5 }{ 25 } \cdot |-0,4| \)
		\end{tasks}
		\item Решите уравнения:
		\begin{tasks}(2)
			\task \( -\dfrac{-x}{-5}=\dfrac{3}{|-5|} \)
			\task \( -14,1:x=\dfrac{-3}{5} \)
			\task \( \dfrac{x}{-0,24}=2,4 \cdot (-11) \)
			\task \( 11,5-\dfrac{x}{3}=-0,5x-1,8 \)
		\end{tasks}
		\item Изобразите числа на числовой прямой:
		\begin{tasks}(4)
			\task \( 0,2 \)
			\task \( 1,01 \)
			\task \( 0,6 \)
			\task \( 0,75 \)
			\task \( -1 \)
			\task \( -0,7 \)
			\task \( -0,55 \)
			\task \( -0,5 \)
		\end{tasks}
		\item Найдите длину и площадь окружности, если радиус окружности равен \( \dfrac{11}{\pi} \) см.
		\item Найдите радиус окружности, если её длина равна \( 2\pi \) см.
		\item Длина окружности равна \(12\pi\) см, найдите диаметр окружности.
	\end{listofex}
\end{homework}
%END_FOLD

%BEGIN_FOLD % ====>>_____ Занятие 7 _____<<====
\begin{class}[number=7-8]
	%\title{Подготовка к проверочной}
	\begin{listofex}
		\item Вычислите:
		\begin{tasks}(2)
			\task \( -2:0,03 - \mfrac{ 11 }{2  }{3}:(-1) \)
			\task \( \left( 4,5-\mfrac{ 5 }{ 1 }{6} \right) \cdot (4,5:0,1) \)
			\task \( 0,125 \cdot (-0,32)+\dfrac{ 5 }{ 9 } \cdot (-2,7) \)
			\task \( 4,8:\mfrac{ 1 }{ 5 }{12}-1,75-(-1) \)
		\end{tasks}
		\item Решите уравнения: %5.30 5.32(о ф ь)
		\begin{tasks}(2)
			\task \( 4x-x=24 \)
			\task \( 8x-8=20-6x \)
			\task \( 9-4x=3x-40 \)
			\task \( 0,6x-5,4=-0,8+5,8 \)
			\task \( -\dfrac{4}{10}+x=-0,27 \)
			\task \( \mfrac{11}{5}{6}-p=\mfrac{4}{7}{8} \)
			\task \( 8,3-x=\mfrac{7}{17}{25} \)
			\task \( 6,7-x=\mfrac{5}{19}{20} \)
			\task \( \dfrac{-4}{x}=116 \)
			\task \( -\dfrac{-y}{-18}=\dfrac{1}{3} \)
			\task \( -15,6:y=\dfrac{3}{-11} \)
			\task \( \dfrac{x}{-0,72}=1,2 \cdot 0,4 \)
			\task \( 15x=x-28 \)
			\task \( 3x+5=-14-5x \)
			
			\task \( 7-\dfrac{3}{5}x=|-18,5|-6x \)
			\task \( 0,25x-|-3,8| \cdot 2 = 2,6 - 0,8x \)
			\task \( |-9 + 14,76| = -x+28 \)
			\task \( 0,5 \cdot x-16,3=- |-11-2| \)
			%\task \( 4,7-1,1x=0,5x-3,3 \)
			%\task \( \dfrac{5x}{6}+16=\dfrac{4}{9}x+9 \)
			%\task \( \dfrac{x}{2}+\dfrac{x}{3}+\mfrac{1}{5}{6}x=2 \)
			%\task \( \dfrac{5x}{7}+\dfrac{x}{35}+x=-1 \)
			%\task \( \dfrac{x}{2}+\dfrac{x}{3}+\dfrac{x}{4}=12 \)
			%\task \( 0,18x-3,54=0,19x-2,89 \)
			%\task \( \mfrac{2}{2}{5}x+ \mfrac{3}{2}{15}=\mfrac{3}{1}{5}x + \mfrac{2}{1}{3} \)
			%\task \( \dfrac{1}{4} - \dfrac{1}{3}m=\mfrac{4}{1}{4}-3m \)
			
		\end{tasks}
		\item Диаметр окружности равен \(12\) см, найдите длину и площадь окружности.
		\item Найдите длину окружности, если длина окружности равна \(\dfrac{ 5 }{ \pi }\) см.
		\item Найдите радиус окружности, если её площадь равна \( 11\pi \) см.
		\item Две окружности с одним центром с радиусами соответственно \( 10\) и \(12\) см образуют кольцо. Найдите площадь этого кольца.
	\end{listofex}
\end{class}
%END_FOLD

%BEGIN_FOLD % ====>>_ Домашняя работа 4 _<<====
\begin{homework}[number=4]
	\begin{listofex}
		\item Решите уравнения:
		\begin{tasks}(2)
			\task \( \dfrac{-0,2(6x+1)}{3,6}=\dfrac{0,5x}{-9} \)
			\task \( \dfrac{3x-2,4}{0,02}=\dfrac{8-x}{0,1} \)
			\task \( \dfrac{3,6}{0,2(6y+1)}=\dfrac{9}{0,5y} \)
			\task \( \dfrac{1,4x-3,5}{0,25}=\dfrac{4,6x-18}{-1,5} \)
			\task \( 4(1,2x+3,7)=0,2(2,6x-14) \)
			\task \( 0,3(5x-7)=3(0,2x+3,2) \)
			\task \( \dfrac{x}{5}-4=-0,1x+2 \)
			\task \( 4,37+6,7x=7,75+9,3x \)
		\end{tasks}
		\item Диаметр окружности равен \(15\) см, найдите длину и площадь окружности.
		\item Найти радиус окружности, длина которой равна \(7,85\) см.
		\item Найти радиус круга, если его площадь равна \(12,56\) м\(^2\).
	\end{listofex}
\end{homework}
%END_FOLD