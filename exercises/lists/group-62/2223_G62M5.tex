%
%===============>>  ГРУППА 6-2 МОДУЛЬ 5  <<=============
%
\setmodule{5}

%BEGIN_FOLD % ====>>_____ Занятие 1 _____<<====
\begin{class}[number=1-2]
	\title{Занятие 1}
	\begin{listofex}
		\item Вычислить:
		\begin{tasks}(2)
			\task \( 3,12+\dfrac{1}{4} \)
			\task \( 5,404+\dfrac{13}{25} \)
			\task \( \mfrac{3}{4}{50}-1,57\)
			\task \( \mfrac{2}{1}{10}+2,2 \)
		\end{tasks}
		\item Вычислить:
		\begin{tasks}(2)
			\task \( \mfrac{4}{1}{7}+1,66 \)
			\task \( \mfrac{12}{13}{15}-10,6 \)
			\task \( \mfrac{2}{5}{18}-1,1\)
			\task \( \mfrac{1}{1}{3}+35,7 \)
		\end{tasks}
		\item Вычислить:
		\begin{tasks}(2)
			\task \( 4,25\cdot2,12 \)
			\task \( 1,02\cdot5,4 \)
			\task \( 0,45\cdot55,3 \)
			\task \( 0,02\cdot0,007 \)
		\end{tasks}
		\item В магазине за день продали \( 56,4 \) кг картофеля по \( 31 \) рублей за килограмм и \( 22,5 \) кг морковки по \( 25 \) рублей за килограмм. Какую выручку получили с продажи картофеля и морковки?
		\item Найти \( 0,25 \) от \( 54 \)
		\item Найти \( 0,18 \) от \( 35,5 \)
		\item Найти \( 1,3 \) от \( 0,57 \)
	\end{listofex}
	\title{Занятие 2}
	\begin{definit}
		Чтобы поделить десятичную дробь на число, будем производить следующие действия:
		\begin{enumcols}[itemcolumns=1]
			\item Выполнить деление, не обращая внимания на запятую.
			\item Поставить запятую в ответе в том момент, когда мы в делимом
			начинаем использовать разряды, стоящие после запятой.
			\item При необходимости дописать в делимом справа столько нулей,
			сколько необходимо.
		\end{enumcols}
	\end{definit}
	\newpage
	\begin{listofex}
		\item Вычислить:
		\begin{tasks}(4)
			\task \( 20,7:9 \)
			\task \( 243,2:8 \)
			\task \( 72,57:59 \)
			\task \( 7,368:24 \)
			\task \( 25:125 \)
			\task \( 1:8 \)
			\task \( 72,57:59 \)
			\task \( 0,7:25 \)
			\task \( 6,78:26 \)
		\end{tasks}
		\item Вычислить:
		\begin{tasks}(4)
			\task \( 45,5:10 \)
			\task \( 45,5:1000 \)
			\task \( 45,5:10000 \)
			\task \( 89:10 \)
			\task \( 89:100 \)
			\task \( 32,2:10 \)
			\task \( 7,98:10 \)
			\task \( 47,7:1000 \)
			\task \( 0,911:1000 \)
		\end{tasks}
	\end{listofex}
	\begin{definit}
		Чтобы поделить число на десятичную дробь, будем придерживаться следующего алгоритма:
		\begin{enumcols}[itemcolumns=1]
			\item Делить на десятичную дробь нельзя!
			\item \textbf{Для деления на десятичную дробь нужно перенести в делимом и в
				делителе запятую на столько цифр вправо, сколько их после запятой в
				делителе.}
			\item \textbf{Если в десятичной части делимого (справа от запятой) знаков меньше, чем в делителе, то во время переноса запятой добавляем в делимом нули для каждого недостающего разряда.}
			\item \textbf{Выполнить деление десятичной дроби на натуральное число.}
			\item Еще раз подчеркнем, что избавляться от запятой нужно только в
			делителе, в то время как в делимом запятая может остаться!
		\end{enumcols}
	\end{definit}
	\begin{listofex}[resume]
		\item Вычислить:
		\begin{tasks}(4)
			\task \( 2:0,4 \)
			\task \( 70:1,75 \)
			\task \( 24:0,2 \)
			\task \( 2:0,5 \)
			\task \( 45:0,05 \)
			\task \( 125:2,5 \)
			\task \( 484:0,004 \)
		\end{tasks}
		\item Вычислить:
		\begin{tasks}(3)
			\task \( 5,1:0,17 \)
			\task \( 25,2:0,4 \)
			\task \( 200,1:0,69 \)
		\end{tasks}
	\end{listofex}
\end{class}
%END_FOLD

%BEGIN_FOLD % ====>>_____ Занятие 2 _____<<====
\begin{class}[number=2]
	\begin{listofex}
		\item Вася сначала истратил \(0,7\) своих денег, а потом --- \(\dfrac{2}{3} \) остатка, после чего у него осталось \(54,3\) р. Сколько денег Вася истратил во второй раз?
		\item В связи с поступлением новой коллекции одежды цена на старую коллекцию снизилась сначала на \(10\%\), а потом ещё на \(30\%\). На сколько процентов снизилась цена по сравнению с первоначальной? Сколько теперь стоит шляпа, которая раньше стоила \(800\) крон?
		\item В двух коробках \(7,8\) кг конфет. Когда из одной коробки взяли \(1,25\) кг конфет, то в обеих коробках конфет стало поровну. Сколько конфет было в каждой коробке?
		\item Представьте число \(7\) в виде суммы трех слагаемых так, чтобы первое слагаемое было вдвое меньше второго и на \(\dfrac{1}{6}\) больше третьего.
		\item Двое рабочих должны были сделать некоторое количество деталей. По окончании работы выяснилось, что первый рабочий сделал \(\dfrac{4}{5}\) всего задания и еще \(40\) деталей, а второй рабочий --- \(0,15\) того, что выполнил первый. Сколько деталей сделали двое рабочих?
		\item Вычислить:
			\begin{itasks}[1]
				\task \( \left(  1,5 : \dfrac{1}{3} - \dfrac{3}{8} : 0,25 \right) \cdot 3,2 - 3,2 \cdot \dfrac{5}{8} \)
				\task \( \left(  3,6 \cdot \mfrac{2}{7}{9} + 1,125 + \mfrac{5}{2}{5} \cdot  \mfrac{2}{7}{9} - \mfrac{1}{1}{8} \right) : 2,5 \)
				\task \( \dfrac{7}{40} : \mfrac{2}{11}{12} - 0,1 \cdot \left( 1,45 : \mfrac{2}{1}{3} - \dfrac{1}{20} : \mfrac{2}{1}{3} \right) \)
				\task \( 20 : \mfrac{33}{1}{3} - \left( \mfrac{4}{7}{25} - 1,28 \right) : \left( 0,75 + \mfrac{3}{1}{4} \right) \cdot 0,2 \)
				\task \( \left( \mfrac{5}{3}{11} + \mfrac{3}{7}{22} - 8,25 \right) \cdot 1,76 : 1,875 + \left( \mfrac{1}{5}{6} + \mfrac{2}{4}{9} \right) \cdot \mfrac{2}{5}{11} : 5,25 \)
			\end{itasks}
	\end{listofex}
\end{class}
%END_FOLD

%BEGIN_FOLD % ====>>_____ Занятие 3 _____<<====
\begin{class}[number=3-4]
	\title{Занятие 3}
	\begin{listofex}
		\item Вычислить:
		\begin{tasks}(3)
			\task \( 0,94: 0,01 \)
			\task \( 0,5168:0,085 \)
			\task \( 4319,856:91,6 \)
			\task \( 8,93:0,004 \)
			\task \( 0,0392:0,00125 \)
			\task \( 0,1218:0,058 \)
		\end{tasks}
		\item Вычислить:
		\begin{tasks}(2)
			\task \( 25,96:11 \)
			\task \( 41,625:37 \)
			\task \( 0,21012:17 \)
			\task \( 1240,8308:31 \)
		\end{tasks}
		\item Вычислить:\begin{tasks}(1)
			\task \( 4,5+0,5\cdot(2,4\cdot1,375-1,64:0,8):\mfrac{2}{1}{12}-\mfrac{1}{2}{7}\cdot1,4 \)
			\task \( \left( \mfrac{2}{3}{4}+0,15-\mfrac{1}{8}{25} \right):\left( \mfrac{2}{1}{2}-\mfrac{1}{3}{4}+0,04 \right) \)
		\end{tasks}
	\end{listofex}
		\title{Занятие 4}
		\begin{listofex}
		\item Вычислить:
		\begin{tasks}(1)
			\task \( 1,456:\dfrac{7}{25}+\dfrac{5}{16}:0,125+\mfrac{4}{1}{2}\cdot0,8 \)
		\end{tasks}
		\item Бревно укоротили сначала на \(0,3\) его длины, а потом на \(\dfrac{2}{5}\) остатка, после чего длина оставшейся части стала равна \(2,1\) м. Сколько метров отпилили от бревна второй раз?
		\item Вася сначала истратил \(0,7\) своих денег, а потом --- \(\dfrac{2}{3} \) остатка, после чего у него осталось \(54,3\) р. Сколько денег Вася истратил во второй раз?
	\end{listofex}
\end{class}
%END_FOLD

%BEGIN_FOLD % ====>>_____ Занятие 4 _____<<====
\begin{class}[number=4]
	\begin{listofex}
		\item Занятие 4
	\end{listofex}
\end{class}
%END_FOLD

%BEGIN_FOLD % ====>>_____ Занятие 5 _____<<====
\begin{class}[number=5]
	\begin{listofex}
		\item Занятие 5
	\end{listofex}
\end{class}
%END_FOLD

%BEGIN_FOLD % ====>>_____ Занятие 6 _____<<====
\begin{class}[number=6]
	\begin{listofex}
		\item Занятие 6
	\end{listofex}
\end{class}
%END_FOLD

%BEGIN_FOLD % ====>>_____ Занятие 7 _____<<====
\begin{class}[number=7-8]
	\title{Занятие 7}
	\begin{listofex}
		\item Вычислить:
		\begin{tasks}(2)
			\task \( 0,25: 0,1 \)
			\task \( 0,513:0,03 \)
			\task \( 434,4:72,4 \)
			\task \( 144,45:96,3 \)
		\end{tasks}
		\item Вычислить:
		\begin{tasks}(2)
			\task \( 2,15\cdot3,15 \)
			\task \( 75\cdot8,2 \)
			\task \( 3,55\cdot1,4 \)
			\task \( 0,04\cdot0,062 \)
		\end{tasks}
		\item В двух коробках \(7,8\) кг конфет. Когда из одной коробки взяли \(1,25\) кг конфет, то в обеих коробках конфет стало поровну. Сколько конфет было в каждой коробке?
		\item В первый день продали --- \(\dfrac{1}{3}\), а во второй день --- \(\dfrac{1}{2}\) поступившего в магазин винограда. Какую часть винограда продали за два дня, а какая часть осталась? Во сколько раз проданного винограда больше чем оставшегося?
		\item Кладовщик выдал первому рабочему \(0,5\) всей имевшейся проволоки, а второму --- \(4\) метра, после чего у него осталось еще \(30\) м проволоки. Сколько проволоки было первоначально?
		\item Чашка, которая стоила \(90\) рублей, продаётся с \(10\)-процентной скидкой. Покупатель отдал кассиру \(1000\) рублей. Сколько рублей сдачи он должен получить, если он хочет купить максимально возможное количество чашек на эту сумму?
	\end{listofex}
	\newpage
	\title{Занятие 8}
	\begin{listofex}
		\item Вычислить:
		\begin{tasks}(2)
			\task \( 0,25: 0,1 \)
			\task \( 0,513:0,03 \)
			\task \( 434,4:72,4 \)
			\task \( 144,45:96,3 \)
		\end{tasks}
		\item Вычислить:
		\begin{tasks}(2)
			\task \( 2,15\cdot3,15 \)
			\task \( 75\cdot8,2 \)
			\task \( 3,55\cdot1,4 \)
			\task \( 0,04\cdot0,062 \)
		\end{tasks}
		\item Вычислить: \( \left( \mfrac{6}{1}{6} + \mfrac{5}{2}{3} \cdot \mfrac{3}{2}{17} \right) - \left( \mfrac{1}{2}{7} - \mfrac{5}{11}{12} \right) \)
			%\task \( \left(  \mfrac{2}{3}{4} + \mfrac{4}{1}{8} \right) \cdot \mfrac{1}{5}{11} \)
		\item Вычислить удобным способом:
		\begin{tasks}(2)
			\task \( 39 \cdot \dfrac{119}{123} - 37 \cdot \dfrac{119}{123} \)
			\task \( \left( \dfrac{17}{18} + \dfrac{19}{27} \right) - \dfrac{17}{18} \)
			\task \( \mfrac{4}{7}{22} - \left( \mfrac{4}{7}{22} - \dfrac{26}{11} \right) \)
		\end{tasks}
			%\task \( 16 \cdot \left( \mfrac{5}{8}{21} - \mfrac{3}{79}{84} \right) + 15 \cdot \left( \dfrac{1}{2} + \mfrac{2}{1}{3} \cdot \mfrac{1}{4}{5} \right) \)
		\item Представьте в минутах:
		\begin{tasks}(2)
			\task \( \dfrac{7}{12} \) часа
			\task \( \mfrac{2}{9}{15} \) часа
			\task \( \mfrac{4}{5}{8} \) часа
			\task \( 0,55 \) часа
			\task \( 1,7 \) часа
			\task \( 3,4 \) часа
		\end{tasks}
	\end{listofex}
\end{class}
%END_FOLD

%BEGIN_FOLD % ====>>_ Домашняя работа 1 _<<====
\begin{homework}[number=1]
	\begin{listofex}
		\item Цена на товар снизилась на \(\dfrac{3}{7}\) и составила \(931\) рубля. Найдите первоначальную цену товара.
		\item Перед тем как брокер продал \( \dfrac{1}{6} \) акций своего клиента, у него было \(1800\) акций. Сколько акций у брокера теперь?
		\item В четырех домах \(3672\) жителя. В одном доме \(\dfrac{1}{3}\) всех жителей, во втором --- в \(2\) раза меньше, чем в первом, а остальные живут поровну в третьем и четвёртом домах. По скольку жителей живёт в третьем и четвёртом домах?
		\item Сплав состоит из \(10 \%\) олова, \(35\%\) меди и \(55\%\) свинца. Сколько каждого металла содержится в \(2\) кг сплава. Ответ дайте в граммах.
		\item С помощью десятичных дробей выразите, какую часть метра составляют:
		\begin{itasks}[4]
			\task \(73\) см
			\task \(3\) м \(15\) см
			\task \(33\) мм
			\task \(15\) дм
		\end{itasks}
		\item Вычислите:
		\begin{itasks}[3]
			\task \(\mfrac{2}{3}{4}-0,6\) 
			\task \( 0,36+\dfrac{1}{2} \)
			\task \(\mfrac{7}{3}{8}-3,059\)
			\task \(121,2:\dfrac{6}{25}\)
			\task \( 5,18 \cdot \dfrac{5}{7} \) 
			\task \( \mfrac{2}{3}{11} \cdot 0,22 \)
		\end{itasks}
	\end{listofex}
\end{homework}
%END_FOLD

%BEGIN_FOLD % ====>>_ Домашняя работа 3 _<<====
\begin{homework}[number=3]
	\begin{listofex}
		%\item Кладовщик выдал первому рабочему \(0,4\) всей имевшейся проволоки, а второму --- \(0,75\) остатка, после чего у него осталось еще \(28,5\) м проволоки. Сколько проволоки было первоначально?
		\item В школу привезли \(900\) новых учебников, из них учебники по математике составляли \(\dfrac{8}{25}\) всех книг, учебники по русскому языку \(\dfrac{33}{100}\) всех книг, а остальные книги были по литературе. Сколько привезли книг по литературе?
		%\item В первый день продали \(\dfrac{1}{3}\), а во второй день \(\dfrac{1}{2}\) поступившего в магазин винограда. Какую часть винограда продали за два дня?
		\item Мастер и его ученик должны были сделать некоторое количество деталей. По окончании работы выяснилось, что мастер сделал \(\dfrac{2}{3}\) всего задания и еще \(8\) деталей, а его ученик --- \(0,25\) того, что выполнил мастер. Сколько деталей сделали ученик и мастер?
		\item Вычислить: \[ 0,5 \cdot (4,4  - 1,64 : 0,8) : \mfrac{2}{1}{12} \]
		%\begin{tasks}(1)
			%\task \( \left(  8,96 : 0,8 + \mfrac{1}{1}{8} \cdot 0,8 \right) : 1,1 - \left( - \mfrac{2}{17}{36} + \mfrac{5}{7}{12} \right) \cdot 0,9 - \mfrac{4}{1}{3} : 2,6 \cdot 0,6 \)
			%\task \( 0,198 \cdot \mfrac{9}{1}{11} - \left(  2,56 + \dfrac{3}{4} - 2,56 - 0,125 \right) \cdot \mfrac{2}{2}{3} - \dfrac{1}{15} \)
			%\task \( \left( \mfrac{1}{2}{13} \cdot 0,42 + 0,78 \cdot \mfrac{1}{2}{13} \right) \cdot \mfrac{1}{4}{9} : 0,6 - 0,5 \cdot \mfrac{5}{2}{3} \)
		%\end{tasks}
		%\item Вычислить:
		%\begin{tasks}(2)
		%	\task \( 5\cdot5,7 \)
		%	\task \( 12\cdot4,2 \)
		%	\task \( 3,8\cdot2,6 \)
		%	\task \( 0,05\cdot0,11 \)
		%\end{tasks}
		\item Вычислить удобным способом: \[ \dfrac{5}{12} - \left( \dfrac{5}{12} - \mfrac{2}{12}{20} \right) - 2,55 \]
		\item \(\left( * \right)\) Имеются два ведра: одно ёмкостью \(4\) л, другое --- \(9\) л. Можно ли набрать из реки ровно \(6\) л воды, если мы не можем измерять ни массу, не объём воды?
	\end{listofex}
\end{homework}
%END_FOLD

%BEGIN_FOLD % ====>>_ Домашняя работа 2 _<<====
\begin{homework}[number=2]
	\begin{listofex}
		\item Вычислите:
	\begin{tasks}(4)
		\task \( 55:10 \) 
		%\task \( 99:100 \)
		\task \( 0,1:10 \)
		%\task \( 105:1000 \)
		\task \( 25,2:100 \)
		\task \( 11,65:10 \)
		%\task \( 125,667:1000 \)
	\end{tasks}
		\item Вычислите:
		\begin{tasks}(4)
			%\task \( 16:64 \) 
			\task \( 27:81 \)
			\task \( 20,1:3 \)
			%\task \( 117,6:8 \)
			%\task \( 1,16:16 \)
			\task \( 15,6:33 \)
			\task \( 6,95:20 \)
		\end{tasks}
		\item Вычислите:
		\begin{tasks}(4)
			\task \( 5:0,5 \) 
			%\task \( 64:0,8 \)
			%\task \( 4:0,05 \)
			\task \( 11:0,055 \)
			%\task \( 140:1,75\)
			\task \( 125:2,5 \)
			\task \( 160:0,8 \)
		\end{tasks}
		\item Вычислите:
		\begin{tasks}(4)
			\task \( 4,9:0,7 \) 
			\task \( 6,9:2,3 \)
			\task \( 3,14:0,04 \)
			\task \( 1,04:0,08 \)
		\end{tasks}
		\item \( \left( {\LARGE *} \right) \) В стране Цифра есть \( 9 \) городов с названиями \( 1 \), \( 2 \), \( 3 \), \( 4 \), \( 5 \), \( 6 \), \( 7 \), \( 8 \), \( 9 \). Путешественник обнаружил, что два города соединены авиалинией в том и только в том случае, если двузначное число, составленное из цифр --- названий этих городов, делится на \( 3 \). Можно ли добраться из города \( 1 \) в город \( 9 \)?
		
	\end{listofex}
\end{homework}
%END_FOLD

%BEGIN_FOLD % ====>>_ Проверочная работа _<<====
\begin{exam}
	\begin{listofex}
		\item Цена на акцию сначала снизилась на \(10\%\), потом снизилась ещё на \(10\%\), а потом увеличилась на \(20\%\). На сколько процентов изменилась цена акции по сравнению с первоначальной? Сколько стоит теперь акция, если первоначально она стоила \(5000\) рублей?
		\item Цена на товар была повышена на \( 60\% \). На сколько процентов надо теперь её понизить, чтобы получить первоначальную цену?
		\item Кладовщик выдал первому рабочему \(0,4\) всей имевшейся проволоки, а второму --- \(0,75\) остатка, после чего у него осталось еще \(28,5\) м проволоки. Сколько проволоки было первоначально?
		\item Автомобиль проехал \(575\) км, что составляет \(\dfrac{23}{25}\) расстояния между двумя городами. Найдите расстояние между городами.
		\item Вычислить:
		\begin{tasks}(1)
			\task \( \mfrac{1}{2}{5} \cdot \left( \mfrac{1}{1}{14} - \dfrac{5}{7} \right)  \)
			\task \( \mfrac{5}{3}{17} \cdot \dfrac{5}{8} - \dfrac{3}{8} \cdot \mfrac{4}{3}{17} \)
			\task \( 57 \cdot \left( \mfrac{3}{1}{3} \cdot \mfrac{4}{2}{7} - \mfrac{11}{20}{21} \right) - \left( \mfrac{2}{3}{11} \cdot \mfrac{4}{2}{5} - \mfrac{5}{11}{12} \right) \cdot \mfrac{3}{1}{7} \)
		\end{tasks}
		\item Вычислить:
		\begin{tasks}(2)
			\task \( 0,45: 0,3 \)
			\task \( 0,16:0,368 \)
			\task \( 204,8:25,6 \)
			\task \( 134:6,7 \)
		\end{tasks}
		\item Вычислить:
		\begin{tasks}(2)
			\task \( 4\cdot5,25 \)
			\task \( 18,4\cdot2,2 \)
			\task \( 1,45\cdot0,05 \)
			\task \( 55,5\cdot0,015 \)
		\end{tasks}
	\end{listofex}
\end{exam}
%END_FOLD
