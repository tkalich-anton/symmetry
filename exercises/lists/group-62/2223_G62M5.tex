%
%===============>>  ГРУППА 6-2 МОДУЛЬ 5  <<=============
%
\setmodule{5}

%BEGIN_FOLD % ====>>_____ Занятие 1 _____<<====
\begin{class}[number=1]
	\begin{listofex}
		\item Занятие 1
	\end{listofex}
\end{class}
%END_FOLD

%BEGIN_FOLD % ====>>_____ Занятие 2 _____<<====
\begin{class}[number=2]
	\begin{listofex}
		\item Занятие 2
	\end{listofex}
\end{class}
%END_FOLD

%BEGIN_FOLD % ====>>_____ Занятие 3 _____<<====
\begin{class}[number=3]
	\begin{listofex}
		\item Занятие 3
	\end{listofex}
\end{class}
%END_FOLD

%BEGIN_FOLD % ====>>_____ Занятие 4 _____<<====
\begin{class}[number=4]
	\begin{listofex}
		\item Занятие 4
	\end{listofex}
\end{class}
%END_FOLD

%BEGIN_FOLD % ====>>_____ Занятие 5 _____<<====
\begin{class}[number=5]
	\begin{listofex}
		\item Занятие 5
	\end{listofex}
\end{class}
%END_FOLD

%BEGIN_FOLD % ====>>_____ Занятие 6 _____<<====
\begin{class}[number=6]
	\begin{listofex}
		\item Занятие 6
	\end{listofex}
\end{class}
%END_FOLD

%BEGIN_FOLD % ====>>_____ Занятие 7 _____<<====
\begin{class}[number=7]
	\begin{listofex}
		\item Занятие 7
	\end{listofex}
\end{class}
%END_FOLD

%BEGIN_FOLD % ====>>_ Домашняя работа 1 _<<====
\begin{homework}[number=1]
	\begin{listofex}
		\item Цена на товар снизилась на \(\dfrac{3}{7}\) и составила \(931\) рубля. Найдите первоначальную цену товара.
		\item Перед тем как брокер продал \( \dfrac{1}{6} \) акций своего клиента, у него было \(1800\) акций. Сколько акций у брокера теперь?
		\item В четырех домах \(3672\) жителя. В одном доме \(\dfrac{1}{3}\) всех жителей, во втором --- в \(2\) раза меньше, чем в первом, а остальные живут поровну в третьем и четвёртом домах. По скольку жителей живёт в третьем и четвёртом домах?
		\item Сплав состоит из \(10 \%\) олова, \(35\%\) меди и \(55\%\) свинца. Сколько каждого металла содержится в \(2\) кг сплава. Ответ дайте в граммах.
		\item С помощью десятичных дробей выразите, какую часть метра составляют:
		\begin{itasks}[4]
			\task \(73\) см
			\task \(3\) м \(15\) см
			\task \(33\) мм
			\task \(15\) дм
		\end{itasks}
		\item Вычислите:
		\begin{itasks}[3]
			\task \(\mfrac{2}{3}{4}-0,6\) 
			\task \( 0,36+\dfrac{1}{2} \)
			\task \(\mfrac{7}{3}{8}-3,059\)
			\task \(121,2:\dfrac{6}{25}\)
			\task \( 5,18 \cdot \dfrac{5}{7} \) 
			\task \( \mfrac{2}{3}{11} \cdot 0,22 \)
		\end{itasks}
	\end{listofex}
\end{homework}
%END_FOLD

%BEGIN_FOLD % ====>>_ Домашняя работа 2 _<<====
\begin{homework}[number=2]
	\begin{listofex}
		\item ДЗ 2
	\end{listofex}
\end{homework}
%END_FOLD

%BEGIN_FOLD % ====>>_ Домашняя работа 3 _<<====
\begin{homework}[number=3]
	\begin{listofex}
		\item ДЗ 3
	\end{listofex}
\end{homework}
%END_FOLD

%BEGIN_FOLD % ====>>_ Проверочная работа _<<====
\begin{exam}
	\begin{listofex}
		\item Проверочная работа
	\end{listofex}
\end{exam}
%END_FOLD
