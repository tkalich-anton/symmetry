%
%===============>>  ГРУППА 6-2 МОДУЛЬ 5  <<=============
%
\setmodule{5}

%BEGIN_FOLD % ====>>_____ Занятие 1 _____<<====
\begin{class}[number=1]
	\begin{listofex}
		\item Занятие 1
	\end{listofex}
\end{class}
%END_FOLD

%BEGIN_FOLD % ====>>_____ Занятие 2 _____<<====
\begin{class}[number=2]
	\begin{listofex}
		\item Вася сначала истратил \(0,7\) своих денег, а потом --- \(\dfrac{2}{3} \) остатка, после чего у него осталось \(54,3\) р. Сколько денег Вася истратил во второй раз?
		\item В связи с поступлением новой коллекции одежды цена на старую коллекцию снизилась сначала на \(10\%\), а потом ещё на \(30\%\). На сколько процентов снизилась цена по сравнению с первоначальной? Сколько теперь стоит шляпа, которая раньше стоила \(800\) крон?
		\item В двух коробках \(7,8\) кг конфет. Когда из одной коробки взяли \(1,25\) кг конфет, то в обеих коробках конфет стало поровну. Сколько конфет было в каждой коробке?
		\item Представьте число \(7\) в виде суммы трех слагаемых так, чтобы первое слагаемое было вдвое меньше второго и на \(\dfrac{1}{6}\) больше третьего.
		\item Двое рабочих должны были сделать некоторое количество деталей. По окончании работы выяснилось, что первый рабочий сделал \(\dfrac{4}{5}\) всего задания и еще \(40\) деталей, а второй рабочий --- \(0,15\) того, что выполнил первый. Сколько деталей сделали двое рабочих?
		\item Вычислить:
			\begin{itasks}[1]
				\task \( \left(  1,5 : \dfrac{1}{3} - \dfrac{3}{8} : 0,25 \right) \cdot 3,2 - 3,2 \cdot \dfrac{5}{8} \)
				\task \( \left(  3,6 \cdot \mfrac{2}{7}{9} + 1,125 + \mfrac{5}{2}{5} \cdot  \mfrac{2}{7}{9} - \mfrac{1}{1}{8} \right) : 2,5 \)
				\task \( \dfrac{7}{40} : \mfrac{2}{11}{12} - 0,1 \cdot \left( 1,45 : \mfrac{2}{1}{3} - \dfrac{1}{20} : \mfrac{2}{1}{3} \right) \)
				\task \( 20 : \mfrac{33}{1}{3} - \left( \mfrac{4}{7}{25} - 1,28 \right) : \left( 0,75 + \mfrac{3}{1}{4} \right) \cdot 0,2 \)
				\task \( \left( \mfrac{5}{3}{11} + \mfrac{3}{7}{22} - 8,25 \right) \cdot 1,76 : 1,875 + \left( \mfrac{1}{5}{6} + \mfrac{2}{4}{9} \right) \cdot \mfrac{2}{5}{11} : 5,25 \)
			\end{itasks}
	\end{listofex}
\end{class}
%END_FOLD

%BEGIN_FOLD % ====>>_____ Занятие 3 _____<<====
\begin{class}[number=3]
	\begin{listofex}
		\item Занятие 3
	\end{listofex}
\end{class}
%END_FOLD

%BEGIN_FOLD % ====>>_____ Занятие 4 _____<<====
\begin{class}[number=4]
	\begin{listofex}
		\item Занятие 4
	\end{listofex}
\end{class}
%END_FOLD

%BEGIN_FOLD % ====>>_____ Занятие 5 _____<<====
\begin{class}[number=5]
	\begin{listofex}
		\item Занятие 5
	\end{listofex}
\end{class}
%END_FOLD

%BEGIN_FOLD % ====>>_____ Занятие 6 _____<<====
\begin{class}[number=6]
	\begin{listofex}
		\item Занятие 6
	\end{listofex}
\end{class}
%END_FOLD

%BEGIN_FOLD % ====>>_____ Занятие 7 _____<<====
\begin{class}[number=7]
	\begin{listofex}
		\item Занятие 7
	\end{listofex}
\end{class}
%END_FOLD

%BEGIN_FOLD % ====>>_ Домашняя работа 1 _<<====
\begin{homework}[number=1]
	\begin{listofex}
		\item Цена на товар снизилась на \(\dfrac{3}{7}\) и составила \(931\) рубля. Найдите первоначальную цену товара.
		\item Перед тем как брокер продал \( \dfrac{1}{6} \) акций своего клиента, у него было \(1800\) акций. Сколько акций у брокера теперь?
		\item В четырех домах \(3672\) жителя. В одном доме \(\dfrac{1}{3}\) всех жителей, во втором --- в \(2\) раза меньше, чем в первом, а остальные живут поровну в третьем и четвёртом домах. По скольку жителей живёт в третьем и четвёртом домах?
		\item Сплав состоит из \(10 \%\) олова, \(35\%\) меди и \(55\%\) свинца. Сколько каждого металла содержится в \(2\) кг сплава. Ответ дайте в граммах.
		\item С помощью десятичных дробей выразите, какую часть метра составляют:
		\begin{itasks}[4]
			\task \(73\) см
			\task \(3\) м \(15\) см
			\task \(33\) мм
			\task \(15\) дм
		\end{itasks}
		\item Вычислите:
		\begin{itasks}[3]
			\task \(\mfrac{2}{3}{4}-0,6\) 
			\task \( 0,36+\dfrac{1}{2} \)
			\task \(\mfrac{7}{3}{8}-3,059\)
			\task \(121,2:\dfrac{6}{25}\)
			\task \( 5,18 \cdot \dfrac{5}{7} \) 
			\task \( \mfrac{2}{3}{11} \cdot 0,22 \)
		\end{itasks}
	\end{listofex}
\end{homework}
%END_FOLD

%BEGIN_FOLD % ====>>_ Домашняя работа 2 _<<====
\begin{homework}[number=2]
	\begin{listofex}
		\item Кладовщик выдал первому рабочему \(0,4\) всей имевшейся проволоки, а второму --- \(0,75\) остатка, после чего у него осталось еще \(28,5\) м проволоки. Сколько проволоки было первоначально?
		\item В школу привезли \(900\) новых учебников, из них учебники по математике составляли \(\dfrac{8}{25}\) всех книг, учебники по русскому языку \(\dfrac{33}{100}\) всех книг, а остальные книги были по литературе. Сколько привезли книг по литературе?
		\item В первый день продали \(\dfrac{1}{3}\), а во второй день \(\dfrac{1}{2}\) поступившего в магазин винограда. Какую часть винограда продали за два дня?
		\item Мастер и его ученик должны были сделать некоторое количество деталей. По окончании работы выяснилось, что мастер сделал \(\dfrac{2}{3}\) всего задания и еще \(8\) деталей, а его ученик --- \(0,25\) того, что выполнил мастер. Сколько деталей сделали ученик и мастер?
		\item Вычислить:
		\begin{itasks}[1]
			\task \( 4,5 + 0,5 \cdot (2,4 \cdot 1,375 - 1,64 : 0,8) : \mfrac{2}{1}{12} - \mfrac{1}{2}{7} \cdot 1,4 \)
			\task \( \left(  8,96 : 0,8 + \mfrac{1}{1}{8} \cdot 0,8 \right) : 1,1 - \left( - \mfrac{2}{17}{36} + \mfrac{5}{7}{12} \right) \cdot 0,9 - \mfrac{4}{1}{3} : 2,6 \cdot 0,6 \)
			\task \( 0,198 \cdot \mfrac{9}{1}{11} - \left(  2,56 + \dfrac{3}{4} - 2,56 - 0,125 \right) \cdot \mfrac{2}{2}{3} - \dfrac{1}{15} \)
			\task \( \left( \mfrac{1}{2}{13} \cdot 0,42 + 0,78 \cdot \mfrac{1}{2}{13} \right) \cdot \mfrac{1}{4}{9} : 0,6 - 0,5 \cdot \mfrac{5}{2}{3} \)
		\end{itasks}
	\end{listofex}
\end{homework}
%END_FOLD

%BEGIN_FOLD % ====>>_ Домашняя работа 3 _<<====
\begin{homework}[number=3]
	\begin{listofex}
		\item Вычислите:
	\begin{tasks}(4)
		\task \( 55:10 \) 
		%\task \( 99:100 \)
		\task \( 0,1:10 \)
		%\task \( 105:1000 \)
		\task \( 25,2:100 \)
		\task \( 11,65:10 \)
		%\task \( 125,667:1000 \)
	\end{tasks}
		\item Вычислите:
		\begin{tasks}(4)
			%\task \( 16:64 \) 
			\task \( 27:81 \)
			\task \( 20,1:3 \)
			%\task \( 117,6:8 \)
			%\task \( 1,16:16 \)
			\task \( 15,6:33 \)
			\task \( 6,95:20 \)
		\end{tasks}
		\item Вычислите:
		\begin{tasks}(4)
			\task \( 5:0,5 \) 
			%\task \( 64:0,8 \)
			%\task \( 4:0,05 \)
			\task \( 11:0,055 \)
			%\task \( 140:1,75\)
			\task \( 125:2,5 \)
			\task \( 160:0,8 \)
		\end{tasks}
		\item Вычислите:
		\begin{tasks}(4)
			\task \( 4,9:0,7 \) 
			\task \( 6,9:2,3 \)
			\task \( 3,14:0,04 \)
			\task \( 1,04:0,08 \)
		\end{tasks}
		\item Вычислите (задача повышенной сложности):
		\begin{tasks}(2)
			\task \( 3,18:0,06+53,25\cdot4-21\cdot0,4 \) 
			\task \( 55:1,1-6,9:2,3+53,24\cdot0,04 \)
		\end{tasks}
	\end{listofex}
\end{homework}
%END_FOLD

%BEGIN_FOLD % ====>>_ Проверочная работа _<<====
\begin{exam}
	\begin{listofex}
		\item Проверочная работа
	\end{listofex}
\end{exam}
%END_FOLD
