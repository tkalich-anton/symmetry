%
%===============>>  ГРУППА 7-1 МОДУЛЬ 4  <<=============
%
\setmodule{4}
%
%===============>>  Занятие 1  <<===============
%
\begin{class}[number=1]
	\begin{listofex}
		\item Разделить число:
		\begin{enumcols}[itemcolumns=3]
			\item \( 15 \) в отношении \( 2:3 \)
			\item \( 120 \) в отношении \( 5:7 \)
			\item \( 54 \) в отношении \( 3:2:4 \)
		\end{enumcols}
		\item Точка \( K \) лежит на отрезке \( AB \) , \( AB=6 \) см, причём \( AK \) больше
		\( BK \) на \( 4,6 \) см. Найдите длины отрезков \( AK \) и \( BK \).
		\item Отрезок \( AB=14 \) см. Точка \( M \) делит этот отрезок в отношении \( AM:MB=3:4 \). Чему равны длины отрезков \( AM \) и \( MB \)?
		\item Точка \( C \) делит отрезок \( AB \) в отношении \( 4:5 \). Найдите длину \( AB \), если \( BC=10 \).
		\item Отрезок \( AB=14 \), а \( BC \) в \( \mfrac{1}{2}{7} \) раза больше отрезка \( AB \). Найдите \( AB+BC \).
		\item На прямой отложили отрезок \( AB=6 \) см. За точку \( B \) на прямой отметили точку \( C \) так, что \( AC \) в \( 4 \) раза больше, чем \( AB \). Найдите длину \( BC \).
		\item Точка \( B \) делит отрезок \( AC \) в отношении \( AB:BC=2:1 \). Точка \( D \) делит отрезок \( AB \) в отношении \( AD:DB=3:2 \). В каком отношении точка \( D \) делит отрезок \( AC \)?
		\item Решить уравнение:
		\begin{enumcols}[itemcolumns=3]
			\item \( 3x-12=4(x-1) \)
			\item \( 0,25(x-2)=15-0,75x \)
			\item \( 2x(x-15)+24=2x^2-6x \)
		\end{enumcols}
	\end{listofex}
\end{class}
%
%===============>>  Занятие 2  <<===============
%
\begin{class}[number=2]
	\begin{listofex}
		\item Разделить число:
		\begin{enumcols}[itemcolumns=3]
			\item \( 22 \) в отношении \( 5:6 \)
			\item \( 200 \) в отношении \( 13:7 \)
			\item \( 36 \) в отношении \( 7:3:2 \)
		\end{enumcols}
		\item Точки \( O \),\( K \),\( M \) лежат на одной прямой. Найти расстояние между
		точками \( O \) и \( M \) , если \( OK = 8,2 \) см, \( KM = 7,3 \) см. Указать все
		возможные решения.
		\item Боковая сторона равнобедренного треугольника равна \( 12 \), а периметр треугольника равен \( 44 \). Найдите длину основания.
		\item Точка \( K \) делит отрезок \( AB \) в отношении \( AK:BK=3:7 \). Найдите длину \( AB \), если \( BK=21 \).
		\item Отрезок \( AB=11 \), а \( BC \) в \( 1,7 \) раза больше отрезка \( AB \). Найдите \( BC-AB \).
		\item Сумма двух сторон равнобедренного треугольника равна \( 26 \) см, а
		периметр равен \( 36 \) см, какими могут быть стороны этого
		треугольника?
		\item Решить пропорцию:
		\begin{enumcols}[itemcolumns=2]
			\item \( 3:7,5=x:\mfrac{6}{1}{4}\)
			\item \( \dfrac{x}{1,8}=\dfrac{2,7}{0,09} \)
		\end{enumcols}
	\end{listofex}
\end{class}
%
%===============>>  Домашняя работа 1  <<===============
%
\begin{homework}[number=1]
	\begin{listofex}
		\item Разделить число:
		\begin{enumcols}[itemcolumns=3]
			\item \( 30 \) в отношении \( 4:2 \)
			\item \( 70 \) в отношении \( 3:4 \)
			\item \( 150 \) в отношении \( 9:4:2 \)
		\end{enumcols}
		\item Вычислить:
		\begin{enumcols}[itemcolumns=3]
			\item \( \dfrac{15^5}{3^4\cdot5^6} \)
			\item \( 3,5\cdot(8,68+1,136)-135,531:33,3 \)
		\end{enumcols}
		\item Решить пропорцию:
		\begin{enumcols}[itemcolumns=2]
			\item \( x:51,6=11,2:34,4 \)
			\item \( \dfrac{12,3}{6}=\dfrac{x}{4,2} \)
		\end{enumcols}
		\item Боковая сторона равнобедренного треугольника равна \( 13 \), а периметр треугольника равен \( 50 \). Найдите длину основания.
		\item Решить уравнение:
		\begin{enumcols}[itemcolumns=2]
			\item \( 17x+2=4(4x-23) \)
			\item \( 0,2(x-2,5)=7,5-1,8x \)
		\end{enumcols}
	\end{listofex}
\end{homework}
%
%===============>>  Занятие 3  <<===============
%
\begin{class}[number=3]
	\begin{listofex}
		\item Прямой угол поделили в отношении \( 7:3 \). Найдите величины получившихся частей.
		\item Развернутый угол поделили в отношении \( 1:2 \) и в каждой части провели биссектрису. Найдите угол между биссектрисами. Изменится ли результат, если отношение \( 1:2 \) заменить на \( 2:3 \)? Проверить и пояснить результат.
	\end{listofex}
	\begin{definit}
		Сумма углов в треугольнике равна \( 180\degree \).
	\end{definit}
	\begin{listofex}
		\item В треугольнике \( ABC \) угол \( \angle A = 80\degree \) и \( \angle B = 30\degree \). Найдите величину угла \( \angle C \).
		\item В треугольнике \( ABC \) угол \( \angle A \) в две раза меньше угла \( \angle B \) и в три раза меньше угла \( \angle C \). Найдите все углы треугольника \( ABC \).
		\item Два угла в треугольнике \( ABC \) в сумме составляют \( 150\degree \) и относятся друг к другу как \( 7:8 \). Найдите эти углы, а также третий угол треугольника \( ABC \).
		\item Все три угла в треугольнике \( ABC \) относятся друг к другу как \( 1:2:15 \). Найдите углы треугольника \( ABC \).
	\end{listofex}
	\begin{definit}
		В равнобедренном треугольнике прилегающие к основанию углы равны.
	\end{definit}
	\begin{listofex}[resume]
		\item Угол \( B \) при основании \( AB \) равнобедренного треугольника \( ABC \) равен \( 34\degree \). Найдите, чему равен угол при вершине треугольника \( ABC \).
		\item Угол при вершине равнобедренного треугольника в два раза меньше, чем угол при основании. Найдите углы треугольника.
		\item Решить уравнение:
		\begin{enumcols}[itemcolumns=2]
			\item \( 26x+2(x-1)=3(7x-10) \)
			\item \( 0,01(x-3,2)=1,034-0,12x \)
		\end{enumcols}
	\end{listofex}
\end{class}
%
%===============>>  Занятие 4  <<===============
% смещение на одно занятие с прошлого месяца
%\begin{class}[number=4]
%	\begin{listofex}
%		\item Пусто
%	\end{listofex}
%\end{class}
%
%===============>>  Домашняя работа 2  <<===============
%
%\begin{homework}[number=2]
%	\begin{listofex}
%
%	\end{listofex}
%\end{homework}
%
%===============>>  Занятие 5  <<===============
% смещение на одно занятие с прошлого месяца
%\begin{class}[number=5]
%	\begin{listofex}
%		\item Пусто
%	\end{listofex}
%\end{class}
%
%===============>>  Домашняя работа 3  <<===============
%
%\begin{homework}[number=2]
%	\begin{listofex}
%
%	\end{listofex}
%\end{homework}
%\newpage
%\title{Подготовка к проверочной работе}
%\begin{listofex}
%	
%\end{listofex}
%
%===============>>  Занятие 7  <<===============
%
%\begin{class}[number=7]
%	\begin{listofex}
%	
%	\end{listofex}
%\end{class}
%
%===============>>  Провечная работа  <<===============
%
%\begin{exam}
%	\begin{listofex}
%	
%	\end{listofex}
%\end{exam}