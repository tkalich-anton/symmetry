%
%===============>>  Киселев Модуль 4 <<=============
%
\setmodule{4}
%
%===============>>  Занятие 1  <<===============
%
\begin{class}[number=1]
	\begin{listofex}
		\item Решите уравнение:
		\begin{enumcols}[itemcolumns=3]
			\item \( \dfrac{4}{7}x=\mfrac{7}{3}{7} \)
			\item \( -\dfrac{2}{9}x=\mfrac{1}{1}{9} \)
			\item \( (x-10)^2=(x+4)^2 \)
			\item \( (2x+7)^2=(2x-1)^2 \)
			\item \( (x-1)^3=-8 \)
		\end{enumcols}
		\item Решите уравнение:
		\begin{enumcols}[itemcolumns=4]
			\item \( \dfrac{13x}{2x^2-7}=1 \)
			\item \( \dfrac{1}{9x-7}=\dfrac{1}{2} \)
			\item \( \dfrac{1}{10x+6}=1 \)
			\item \( \dfrac{x+89}{x-7}=\dfrac{-5}{x-7} \)
		\end{enumcols}
		\item Решите уравнение:
		\begin{enumcols}[itemcolumns=2]
			\item \( \sqrt{15-2x}=3 \)
			\item \( \sqrt{\dfrac{1}{5-2x}}=\dfrac{1}{3} \)
			\item \( \sqrt[3]{x+2}=-2 \)
			\item \( \sqrt{34-3x}=x-2 \)
		\end{enumcols}
		\item Решите уравнение:
		\begin{enumcols}[itemcolumns=4]
			\item \( 2^{4-2x}=64 \)
			\item \( 5^{x-7}=\dfrac{1}{125} \)
			\item \( \left(  \dfrac{1}{3}   \right)^{x-8}=\dfrac{1}{9}\)
			\item \( 16^{x-9}=\dfrac{1}{2} \)
		\end{enumcols}
		\item Решите уравнение:
		\begin{enumcols}[itemcolumns=2]
			\item \( \log_2(4-x)=7 \)
			\item \( \log_5(5-x)=\log_53 \)
			\item \( \log_4(x+3)=\log_4(4x-15) \)
			\item \( \log_{\frac{1}{7}}(7-x)=-2 \)
			\item \( \log_5(5-x)=2\log_53 \)
			\item \( \log_5(x^2+2x)=\log_5(x^2+10) \)
			\item \( \log_{x-5}49=2 \)
			\item \( \log_82^{8x-4}=4 \)
			\item \( 2^{\log_8(5x-3)}=4 \)
			\item \( \log_x32=5 \)
		\end{enumcols}
		\item Некоторая компания продает свою продукцию по цене p=400 руб. за единицу, переменные затраты на производство одной единицы продукции составляют \( v=200 \) руб., постоянные расходы предприятия \( f= 600000 \) руб. в месяц. Месячная операционная прибыль предприятия (в рублях) вычисляется по формуле \( \pi(q)=q(p-v)-f \). Определите месячный объeм производства \( q \) (единиц продукции), при котором месячная операционная прибыль предприятия будет равна \( 900 000 \) руб.
		\item Если достаточно быстро вращать ведeрко с водой на верeвке в вертикальной плоскости, то вода не будет выливаться. При вращении ведeрка сила давления воды на дно не остаeтся постоянной: она максимальна в нижней точке и минимальна в верхней. Вода не будет выливаться, если сила еe давления на дно будет положительной во всех точках траектории кроме верхней, где она может быть равной нулю. В верхней точке сила давления, выраженная в ньютонах, равна \( P=m \left( \dfrac{v^2}{L}-g \right) \), где \( m \) -- масса воды в килограммах, \( v \) скорость движения ведерка в м/с, \( L \) -- длина вервеки в метрах, \( g \) -- ускорение свободного падения (считайте \( g=10 \) м/с\( ^2 \)). С какой наименьшей скоростью надо вращать ведeрко, чтобы вода не выливалась, если длина верeвки равна \( 40 \) см? Ответ выразите в м/с.
	\end{listofex}
\end{class}
%
%===============>>  Занятие 2  <<===============
%
\begin{class}[number=2]
	\begin{listofex}
		\item Вычислить:
		\begin{enumcols}[itemcolumns=3]
			\item \exercise{562}
			\item \exercise{564}
			\item \exercise{569}
			\item \exercise{571}
			\item \exercise{579}
		\end{enumcols}
		\item Вычислить:
		\begin{enumcols}[itemcolumns=3]
			\item \exercise{1577}
			\item \exercise{1578}
			\item \exercise{1579}
		\end{enumcols}
		\item Вычислить:
		\begin{enumcols}[itemcolumns=3]
			\item \exercise{1572}
			\item \exercise{1565}
			\item \exercise{1566}
			\item \exercise{1573}
			\item \exercise{1567}
			\item \exercise{1575}
			\item \exercise{1594}
		\end{enumcols}
		\item Вычислить:
		\begin{enumcols}[itemcolumns=2]
			\item \exercise{1569}
			\item \exercise{1570}
			\item \exercise{1571}
			\item \exercise{1574}
		\end{enumcols}
		\item Решить уравнение:
		\begin{enumcols}[itemcolumns=2]
			\item \( \log_2(4-x)=7 \)
			\item \( \log_{1/7}(7-2x)=-2 \)
			\item \( \log_4(x+3)=\log_4(4x-15) \)
			\item \( \log_5(7-x)=\log_5(3-x)+1 \)
			\item \( \log_8 2^{8x-4}=4 \)
			\item \( \log_5(x^2+13x)=\log_5(9x+5) \)
		\end{enumcols}
	\end{listofex}
\end{class}
%
%===============>>  Домашняя работа 1  <<===============
%
\begin{homework}[number=1]
	\begin{listofex}
		\item Вычислить:
		\begin{enumcols}[itemcolumns=2]
			\item \( \sqrt{6x+4}=2 \)
			\item \( 9^{x-10}=\dfrac{1}{3} \)
			\item \( \sqrt{\dfrac{1}{15-4x}}=0,2 \)
			\item \( \dfrac{4}{x^2-12}=1 \)
			\item \( -\dfrac{5}{6}x=\mfrac{12}{1}{2} \)
			\item \( 9^{6+x}=81^{2x} \)
			\item \( \dfrac{3x}{x^2-4}=1 \)
			\item \( \sqrt{\dfrac{5}{5-x}=1} \)
			\item \( 5^{x-7}=\dfrac{1}{125} \)
			\item \( \log_5(x^3+2x)=\log_5(x^2+10) \)
			\item \( \left( \dfrac{1}{49} \right)^{x-8}=7 \)
			\item \( \log_8(x^2+x)=\log_8(x^2-4) \)
			\item \( \log_5(5-x)=\log_53 \)
			\item \( \log_5(8+3x)=\log_5(7-3x)+1 \)
		\end{enumcols}
	\item Ёмкость высоковольтного конденсатора в телевизоре \( C=2\cdot10^{-6} \) Ф. Параллельно с конденсатором подключeн резистор с сопротивлением \( R=5\cdot10^6 \) Ом. Во время работы телевизора напряжение на конденсаторе \(  U_0 = 16 \) кВ. После выключения телевизора напряжение на конденсаторе убывает до значения \( U \) (кВ) за время, определяемое выражением \( t=\alpha RC\log_2\dfrac{U}{U_0} \) (с), где \( \alpha=0,7 \) -- постоянная. Определите напряжение на конденсаторе, если после выключения телевизора прошло \( 21 \) с. Ответ дайте в киловольтах.
	\end{listofex}
\end{homework}
%
%===============>>  Занятие 3  <<===============
%
%\begin{class}[number=3]
%	\begin{listofex}
%		\item Пусто
%	\end{listofex}
%\end{class}
%
%===============>>  Занятие 4  <<===============
%
%\begin{class}[number=4]
%	\begin{listofex}
%		\item Пусто
%	\end{listofex}
%\end{class}
%
%===============>>  Домашняя работа 2  <<===============
%
%\begin{homework}[number=2]
%	\begin{listofex}
%		\item Пусто
%	\end{listofex}
%\end{homework}
%
%===============>>  Занятие 5  <<===============
%
%\begin{class}[number=6]
%	\begin{listofex}
%		\item Пусто
%	\end{listofex}
%\end{class}
%
%===============>>  Занятие 6  <<===============
%
%\begin{class}[number=6]
%	\begin{listofex}
%		\item Пусто
%	\end{listofex}
%\end{class}
%
%===============>>  Домашняя работа 3  <<===============
%
%\begin{homework}[number=3]
%	\begin{listofex}
%		\item Пусто
%	\end{listofex}
%\end{homework}
%
%===============>>  Занятие 7  <<===============
%
%\begin{class}[number=7]
%	\begin{listofex}
%		\item Пусто
%	\end{listofex}
%\end{class}
%
%===============>>  Проверочная работа  <<===============
%
%\begin{exam}
%	\begin{listofex}
%		\item Пусто
%	\end{listofex}
%\end{exam}