%6 класс Предподготовка Занятие №1
\begin{listofex}
	\item Заполнить пробелы:
	\begin{center}
		\begin{tikzpicture}[-latex ,auto, on grid ,
			semithick, state/.style ={ rectangle, draw , minimum width =1.5 cm, minimum height =1.2 cm}, pil/.style={ ->, thick, shorten <=2pt, shorten >=2pt,}]
			\node[state] (A) at (0,0) {$1000$};
			\node[state] (B) at (3,0) {$\phantom{280}$};
			\node[state] (C) at (6,0) {$840$};
			\node[state] (D) at (9,0) {$\phantom{42}$};
			\node[state] (E) at (12,0) {$\phantom{2142}$};
			\node[state] (F) at (3,-3) {$\phantom{11500}$};
			\node[state] (G) at (6,-3) {$\phantom{115}$};
			\node[state] (H) at (9,-3) {$23$};
			\node[state] (I) at (12,-3) {$\phantom{2323}$};
			\path (A) edge [bend left =25] node[above] {$-720$} (B);
			\path (B) edge [bend left =25] node[above] {$\times\:?$} (C);
			\path (C) edge [bend left =25] node[above] {$:20$} (D);
			\path (D) edge [bend left =25] node[above] {$\times\:?$} (E);
			\path (B) edge node[left] {$+11\:220$} (F);
			\path (F) edge [bend left =25] node[above] {$:100$} (G);
			\path (G) edge [bend left =25] node[above] {$:\:?$} (H);
			\path (H) edge node[right] {$+\:?$} (D);
			\path (H) edge [bend left =25] node[above] {$\times\:101$} (I);
			\path (I) edge node[right] {\( -181 \)} (E);
		\end{tikzpicture}
	\end{center}
	\item Представьте:
	\begin{enumcols}[itemcolumns=2]
		\item \( 6 \) км \( 215 \) м \( 15 \) см в сантиметрах
		\item \( 3 \) т \( 3 \) кг 3 г в граммах
		\item \( 6 \) км/ч в м/ч
		\item \( 72 \) км/ч в м/с
	\end{enumcols}
	\item Масса угля в железнодорожном вагоне \( 60 \) тонн. Самосвал может взять третью часть этого груза. Сколько рейсов надо сделать на самосвале, чтобы разгрузить \( 6 \) таких вагонов?
	\item В питомнике вырастили саженцы деревьев: елей было \( 360 \), а на каждые \( 8 \) елей приходилось \( 18 \) клёнов и \( 16 \) лип. Сколько всего елей, клёнов и лип вырастили в питомнике?
	\item В семье муж Андрей и жена Алена блогеры. Андрей делает обзоры на новые машины и у него \( 270 \) тысяч подписчиков, а Алена --- бьюти-блогер, у нее \( 320 \) тысяч подписчиков. Андрей зарабатывает на \( 100 \) тысяч руб. меньше, чем Алена. Сколько тысяч рублей зарабатывает семья блогеров, если с каждая тысяча подписчиков приносит одну и ту же сумму?
	\item Учитель математики поставил \( 27 \) отметок, причем четверок было в \( 3 \) раза больше, чем
	пятерок, а пятерок --- в \( 2 \) раза больше троек. Сколько четверок поставил учитель математики?
	\item Сможет ли учитель математики поставить \( 165 \) отметок так, чтобы троек было бы вдвое
	больше двоек, четверок --- в \( 3 \) раза больше троек, а пятерок – вдвое больше четверок?
\end{listofex}