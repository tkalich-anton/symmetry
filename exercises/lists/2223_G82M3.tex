%
%===============>>  ГРУППА 8-2 МОДУЛЬ 3  <<=============
%
\setmodule{3}
%
%===============>>  Занятие 2  <<===============
%
\begin{class}[number=2]
	\begin{definit}
		Арифметическим квадратным корнем из неотрицательного числа \( A \) называют такое неотрицательное число \( B \), квадрат которого равен \( A \).
		\[ \sqrt{A}=B \Rightarrow B\cdot B = A \]
	\end{definit}
	\begin{listofex}
		\item Вычислить:
		\begin{enumcols}[itemcolumns=6]
			\item \( \sqrt{4} \)
			\item \( \sqrt{9} \)
			\item \( \sqrt{25} \)
			\item \( \sqrt{100} \)
			\item \( \sqrt{121} \)
			\item \( \sqrt{400} \)
			\item \( \sqrt{144} \)
			\item \( \sqrt{1600} \)
			\item \( \sqrt{0,04} \)
			\item \( \sqrt{1,21} \)
			\item \( \sqrt{3,24} \)
			\item \( \sqrt{0,0625} \)
			\item \( \sqrt{\dfrac{1}{9}} \)
			\item \( \sqrt{\dfrac{1}{1600}} \)
			\item \( \sqrt{\dfrac{36}{25}} \)
			\item \( \sqrt{\dfrac{81}{100}} \)
			%		===============ДЗ===============
			%		\item \( \sqrt{36} \)
			%		\item \( \sqrt{64} \)
			%		\item \( \sqrt{81} \)
			%		\item \( \sqrt{121} \)
		\end{enumcols}
		\item Вычислить:
		\begin{enumcols}[itemcolumns=2]
			\item \( 2+\sqrt{1}(\sqrt{9}+\sqrt{25}) \)
			\item \( 15-(2\cdot\sqrt{81}-\sqrt{36}) \)
			\item \( \sqrt{16}\cdot\sqrt{9}+\sqrt{0,16}\cdot\sqrt{0,25} \)
			\item \( \sqrt{\dfrac{1}{9}}\cdot\sqrt{81}-\sqrt{4}\cdot\sqrt{0,25} \)
			\item \( 0,1\sqrt{900}-\dfrac{1}{4}\sqrt{400}+\sqrt{49}:\sqrt{0,01} \)
			%		===============ДЗ===============
			%		Источники: Никольский 8кл №132; Макарычев 8кл №304-305; Учебник 8-57 №3.5
		\end{enumcols}
		\item Вычислить:
		\begin{enumcols}[itemcolumns=3]
			\item \( \sqrt{2\dfrac{1}{4}}+\sqrt{1\dfrac{7}{9}} \)
			\item \( -\sqrt{1\dfrac{9}{16}}+\dfrac{3}{2}\cdot\sqrt{5\dfrac{4}{9}} \)
			\item \( -\sqrt{11\dfrac{1}{9}}-\sqrt{1\dfrac{40}{81}} \)
			%		===============ДЗ===============
			%		Придумать похожие самостоятельно
		\end{enumcols}
	\end{listofex}
	\begin{definit}
		Арифметические квадратные корни из равных чисел равны.
	\end{definit}
	\begin{definit}
		Больше тот из арифметических корней, чье подкоренное значение больше.
	\end{definit}
	\begin{listofex}[resume]
		\item Сравните числа:
		\begin{enumcols}[itemcolumns=4]
			\item \( \vphantom{\sqrt{\dfrac{1}{5}}}\sqrt{100} \) и \( \sqrt{81} \)
			\item \( \sqrt{0,2} \) и \( \sqrt{\dfrac{1}{5}} \)
			\item \( \sqrt{0,09} \) и \( \sqrt{\dfrac{4}{25}} \)
			\item \( \sqrt{0,068} \) и \( \sqrt{\dfrac{17}{25}} \)
	%		===============ДЗ===============
	%		Источники: Никольский 8 класс №140; Учебник 8-57 №3.16
		\end{enumcols}
		\item Между какими двумя последовательными натуральными числами находится число:
		\begin{enumcols}[itemcolumns=5]
			\item \( \sqrt{31} \)
			\item \( \sqrt{50} \)
			\item \( \sqrt{71} \)
			\item \( \sqrt{119} \)
			\item \( \sqrt{333} \)
	%		===============ДЗ===============
	%		\item \( \sqrt{38} \)
	%		\item \( \sqrt{73} \)
	%		\item \( \sqrt{114} \)
		\end{enumcols}
	\end{listofex}
	\begin{definit}
		Для любого \textbf{неотрицательного} числа \( A \) справедливо равенство: \( \left( \sqrt{A} \right)^2=A \)
	\end{definit}
	\begin{listofex}[resume]
		\item Вычислить:
		\begin{enumcols}[itemcolumns=3]
			\item \( (\sqrt{2})^2 \)
			\item \( (\sqrt{17})^2 \)
			\item \( (\sqrt{110})^2 \)
			\item \( (\sqrt{29})^2+(\sqrt{29})^2 \)
			\item \( (\sqrt{13})^2-(\sqrt{12})^2 \)
			\item \( (\sqrt{12}-\sqrt{11})(\sqrt{12}+\sqrt{11}) \)
	%		===============ДЗ===============
	%		\item \( (\sqrt{13})^2 \)
	%		\item \( (\sqrt{71})^2 \)
	%		\item \( (\sqrt{112})^2 \)
		\end{enumcols}
		\item Вычислить:
		\begin{enumcols}[itemcolumns=3]
			\item \( (-2\sqrt{11})^2-\sqrt{1,44} \)
			\item \( \dfrac{3}{11}\sqrt{1,21}-\dfrac{1}{5}(\sqrt{7})^2 \)
			\item \( (4\sqrt{3})^2-(3\sqrt{5})^2 \)
			\item \( \sqrt{529}-\left( \dfrac{1}{2}\sqrt{84} \right)^2 \)
			\item \( \sqrt{7\dfrac{1}{9}}+\sqrt{3\dfrac{1}{16}}-\dfrac{\sqrt{25}}{12} \)
			\item \( 32\cdot\left( -\dfrac{1}{2}\sqrt{11} \right):2 \)
	%		===============ДЗ===============
	%		Источники: Учебник 8-57 №3.15; Придумать самостоятельно
		\end{enumcols}
		\item Укажите все целые числа, расположенные на координатной прямой между числами:
		\begin{enumcols}[itemcolumns=4]
			\item \( 7 \) и \( \sqrt{102} \)
			\item \( \sqrt{17} \) и \( \sqrt{123} \)
			\item \( -\sqrt{62} \) и \( 6,2 \)
			\item \( -\sqrt{29} \) и \( -4,2 \)
		\end{enumcols}
		\item Расположите в порядке возрастания: \[ 4;\;3,8;\;\sqrt{15};\;\sqrt{19};\;4,3 \]
		\item Найдите значение выражения \( \left( \dfrac{12}{7}\sqrt{4-2a} \right)^2 \) при \( a=-22,5 \)
	\end{listofex}
\end{class}
%
%===============>>  Занятие 3  <<===============
%
\begin{class}[number=3]
	\begin{definit}
		Корень из произведения неотрицательных множителей равен произведению корней из этих множителей. То есть если \( a\ge0 \) и \( b\ge0 \), то: \[ \sqrt{a \cdot b}=\sqrt{a}\cdot\sqrt{b} \]
	\end{definit}
	\begin{definit}
		Корень из дроби,  числитель которой неотрицателен, а знаменатель положителен, равен корню числителя, деленному на корень из знаменателя. То есть если \( a\ge0 \) и \( b>0 \), то: \[ \sqrt{\dfrac{a}{b}}=\dfrac{\sqrt{a}}{\sqrt{b}} \]
	\end{definit}
	\begin{listofex}
		\item Вычислить:
		\begin{enumcols}[itemcolumns=3]
			\item \( \sqrt{100\cdot49} \)
			\item \( \sqrt{81\cdot400} \)
			\item \( \sqrt{0,01\cdot169} \)
			\item \( \sqrt{81\cdot0,0049} \)
			\item \( \sqrt{25\cdot0,0529} \)
			\item \( \sqrt{2,25\cdot0,04} \)
			\item \( \sqrt{9\cdot64\cdot0,25} \)
			\item \( \sqrt{1,21\cdot0,09\cdot0,0001} \)
			
			
			% ====== ДЗ =======
			%\item \exercise{1718}
			%\item \exercise{1722}
			%\item \exercise{1724}
		\end{enumcols}
		\item Вычислить:
		\begin{enumcols}[itemcolumns=3]
			\item \exercise{1719}
			\item \exercise{1722}
			\item \exercise{1727}
		\end{enumcols}
		\item Вычислить:
		\begin{enumcols}[itemcolumns=5]
			\item \( \sqrt{2}\cdot\sqrt{32} \)
			\item \( \sqrt{45}\cdot\sqrt{5} \)
			\item \( \sqrt{1,3}\cdot\sqrt{5,2} \)
			\item \( \sqrt{50}\cdot\sqrt{4,5} \)
			\item \( \sqrt{16,9}\cdot\sqrt{0,4} \)
		\end{enumcols}
		\item Вычислить:
		\begin{enumcols}[itemcolumns=4]
			\item \( \sqrt{21}\cdot\sqrt{\mfrac{3}{6}{7}} \)
			\item \( \sqrt{15}\cdot\sqrt{\mfrac{6}{2}{3}} \)
			\item \exercise{1710}
			\item \exercise{1728}
		\end{enumcols}
		\item Вычислить:
		\begin{enumcols}[itemcolumns=4]
			\item \( \sqrt{\dfrac{9}{64}} \)
			\item \( \sqrt{\dfrac{36}{25}} \)
			\item \( \sqrt{\mfrac{1}{9}{16}} \)
			\item \( \sqrt{\mfrac{5}{1}{16}} \)
			\item \exercise{1701}
			\item \exercise{1703}
			\item \exercise{1699}
			\item \exercise{2824}
		\end{enumcols}
		\item Вынести множитель из под знака корня:
		\begin{enumcols}[itemcolumns=8]
			\item \( \sqrt{8} \)
			\item \( \sqrt{18} \)
			\item \( \sqrt{32} \)
			\item \( \sqrt{75} \)
			\item \( \sqrt{12} \)
			\item \( \sqrt{98} \)
			\item \( \sqrt{250} \)
			\item \( \sqrt{200} \)
		\end{enumcols}
		\item Упростить:
		\begin{enumcols}[itemcolumns=2]
			\item \( 3\sqrt{5}+4\sqrt{5}-2\sqrt{5} \)
			\item \( 3,2\sqrt{13}-\dfrac{1}{8}\sqrt{13}+0,25\sqrt{13} \)
			\item \( \sqrt{12}+5\sqrt{3} \)
			\item \( \sqrt{27}-\sqrt{3} \)
			\item \( \sqrt{125}+\sqrt{50} \)
			\item \( \dfrac{1}{4}\sqrt{72}+1,5\sqrt{2} \)
			\item \( 9\sqrt{7}-2\sqrt{98} \)
			\item \( 0,5\sqrt{32}-1,2\sqrt{128} \)
			
			% ====== ДЗ =======
			%\item \( 13\sqrt{7}-2\sqrt{7}+5\sqrt{7} \)
			%\item \( 4,6\sqrt{5}-2,5\sqrt{5}+0,2\sqrt{45} \)
			%\item \exercise{1630}
		\end{enumcols}
		\item Вычислить:
		\begin{enumcols}[itemcolumns=3]
			\item \exercise{1755}
			\item \exercise{1687}
			\item \exercise{1763}
			\item \exercise{1216}
			\item \exercise{1761}
			\item \exercise{2827}
		\end{enumcols}
		\item Вычислить:
		\begin{enumcols}[itemcolumns=3]
			\item \exercise{1620}
			\item \exercise{1627}
			\item \exercise{1773}
			\item \exercise{1779}
			\item \exercise{1638}
			\item \exercise{1756}
			\item \exercise{1218}
		\end{enumcols}
		\item Между какими двумя целыми числами стоит число:
		\begin{enumcols}[itemcolumns=3]
			\item \( \sqrt{223} \)
			\item \( \sqrt{1512} \)
			\item \( -\sqrt{215} \)
		\end{enumcols}
	\end{listofex}
\end{class}
%
%===============>>  Домашняя работа 1  <<===============
%
\begin{homework}[number=1]
	\begin{listofex}
		\item Пусто
	\end{listofex}
\end{homework}
%
%===============>>  Занятие 4  <<===============
% смещение на одно занятие с прошлого месяца
\begin{class}[number=4]
	\begin{listofex}
		\item Вычислить:
		\begin{enumcols}[itemcolumns=3]
			\item \( \sqrt{400\cdot81} \)
			\item \( \sqrt{64\cdot900} \)
			\item \( \sqrt{0,001\cdot144} \)
			\item \( \sqrt{9\cdot0,0121} \)
			\item \( \sqrt{225\cdot0,16} \)
			\item \( \sqrt{2,56\cdot0,01} \)
			\item \( \sqrt{25\cdot81\cdot0,49} \)
			\item \( \sqrt{0,0064\cdot2500\cdot36} \)
			\item \( \sqrt{1,21\cdot121\cdot0,0121} \)
		\end{enumcols}
		\item Вычислить:
		\begin{enumcols}[itemcolumns=2]
			\item \exercise{4123}
			\item \exercise{4124}
		\end{enumcols}
		\item Вынести множитель из под знака корня:
		\begin{enumcols}[itemcolumns=8]
			\item \( \sqrt{12} \)
			\item \( \sqrt{20} \)
			\item \( \sqrt{40} \)
			\item \( \sqrt{125} \)
			\item \( \sqrt{72} \)
			\item \( \sqrt{288} \)
			\item \( \sqrt{360} \)
			\item \( \sqrt{500} \)
		\end{enumcols}
		\item Упростить:
		\begin{enumcols}[itemcolumns=3]
			\item \( 2\sqrt{6}+3\sqrt{6}-\sqrt{6} \)
			\item \( 2,5\sqrt{11}-\dfrac{1}{4}\sqrt{11}+0,36\sqrt{11} \)
			\item \( \sqrt{20}+6\sqrt{5}-0,5\sqrt{5} \)
			\item \( 5\sqrt{27}-10\sqrt{3} \)
			\item \( 2\sqrt{125}-5\sqrt{50} \)
			\item \( \dfrac{1}{2}\sqrt{98}+\dfrac{4}{3}\sqrt{20} \)
			\item \( 9\sqrt{50}-2\sqrt{8}+12\sqrt{18} \)
			\item \( 0,25\sqrt{108}-1,25\sqrt{75} \)
		\end{enumcols}
		\item Вычислить:
		\begin{enumcols}[itemcolumns=2]
			\item \exercise{1757}
			\item \exercise{2829}
			\item \exercise{1760}
			\item \exercise{1762}
			\item \exercise{1763}
			\item \exercise{2837}
		\end{enumcols}
		\item Вычислить:
		\begin{enumcols}[itemcolumns=2]
			\item \exercise{2826}
			\item \exercise{1650}
			\item \exercise{1659}
			\item \exercise{1667}
		\end{enumcols}
		\item Вычислить:
		\begin{enumcols}[itemcolumns=2]
			\item \exercise{1665}
			\item \exercise{4127}
		\end{enumcols}
		\item Вычислить:
		\begin{enumcols}[itemcolumns=2]
			\item \exercise{1787}
			\item \exercise{1648}
		\end{enumcols}
	\end{listofex}
\end{class}
%\newpage
%\title{Занятие №3}
%\begin{listofex}
%	
%\end{listofex}
%\newpage
%\title{Занятие №4}
%\begin{listofex}
%	
%\end{listofex}
%\newpage
%\title{Домашняя работа №2}
%\begin{listofex}
%	
%\end{listofex}
%\newpage
%\title{Занятие №5}
%\begin{listofex}
%	
%\end{listofex}
%\newpage
%\title{Занятие №6}
%\begin{listofex}
%	
%\end{listofex}
%\newpage
%\title{Домашняя работа №3}
%\begin{listofex}
%	
%\end{listofex}
%\newpage
%\title{Подготовка к проверочной работе}
%\begin{listofex}
%	
%\end{listofex}
%\newpage
%\title{Проверочная работа}
%\title{Вариант 1}
%\begin{listofex}
%	
%\end{listofex}
%\newpage
%\title{Проверочная работа}
%\begin{listofex}
%	
%\end{listofex}