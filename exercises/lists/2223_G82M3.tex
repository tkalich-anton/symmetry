%
%===============>>  ГРУППА 8-2 МОДУЛЬ 3  <<=============
%
\setmodule{3}
%
%===============>>  Занятие 2  <<===============
%
\begin{class}[number=2]
	\begin{definit}
		Арифметическим квадратным корнем из неотрицательного числа \( A \) называют такое неотрицательное число \( B \), квадрат которого равен \( A \).
		\[ \sqrt{A}=B \Rightarrow B\cdot B = A \]
	\end{definit}
	\begin{listofex}
		\item Вычислить:
		\begin{enumcols}[itemcolumns=6]
			\item \( \sqrt{4} \)
			\item \( \sqrt{9} \)
			\item \( \sqrt{25} \)
			\item \( \sqrt{100} \)
			\item \( \sqrt{121} \)
			\item \( \sqrt{400} \)
			\item \( \sqrt{144} \)
			\item \( \sqrt{1600} \)
			\item \( \sqrt{0,04} \)
			\item \( \sqrt{1,21} \)
			\item \( \sqrt{3,24} \)
			\item \( \sqrt{0,0625} \)
			\item \( \sqrt{\dfrac{1}{9}} \)
			\item \( \sqrt{\dfrac{1}{1600}} \)
			\item \( \sqrt{\dfrac{36}{25}} \)
			\item \( \sqrt{\dfrac{81}{100}} \)
		\end{enumcols}
		\item Вычислить:
		\begin{enumcols}[itemcolumns=2]
			\item \( 2+\sqrt{1}(\sqrt{9}+\sqrt{25}) \)
			\item \( 15-(2\cdot\sqrt{81}-\sqrt{36}) \)
			\item \( \sqrt{16}\cdot\sqrt{9}+\sqrt{0,16}\cdot\sqrt{0,25} \)
			\item \( \sqrt{\dfrac{1}{9}}\cdot\sqrt{81}-\sqrt{4}\cdot\sqrt{0,25} \)
			\item \( 0,1\sqrt{900}-\dfrac{1}{4}\sqrt{400}+\sqrt{49}:\sqrt{0,01} \)
			%		===============ДЗ===============
			%		Источники: Никольский 8кл №132; Макарычев 8кл №304-305; Учебник 8-57 №3.5
		\end{enumcols}
		\item Вычислить:
		\begin{enumcols}[itemcolumns=3]
			\item \( \sqrt{2\dfrac{1}{4}}+\sqrt{1\dfrac{7}{9}} \)
			\item \( -\sqrt{1\dfrac{9}{16}}+\dfrac{3}{2}\cdot\sqrt{5\dfrac{4}{9}} \)
			\item \( -\sqrt{11\dfrac{1}{9}}-\sqrt{1\dfrac{40}{81}} \)
			%		===============ДЗ===============
			%		Придумать похожие самостоятельно
		\end{enumcols}
	\end{listofex}
	\begin{definit}
		Арифметические квадратные корни из равных чисел равны.
	\end{definit}
	\begin{definit}
		Больше тот из арифметических корней, чье подкоренное значение больше.
	\end{definit}
	\begin{listofex}[resume]
		\item Сравните числа:
		\begin{enumcols}[itemcolumns=4]
			\item \( \vphantom{\sqrt{\dfrac{1}{5}}}\sqrt{100} \) и \( \sqrt{81} \)
			\item \( \sqrt{0,2} \) и \( \sqrt{\dfrac{1}{5}} \)
			\item \( \sqrt{0,09} \) и \( \sqrt{\dfrac{4}{25}} \)
			\item \( \sqrt{0,068} \) и \( \sqrt{\dfrac{17}{25}} \)
	%		===============ДЗ===============
	%		Источники: Никольский 8 класс №140; Учебник 8-57 №3.16
		\end{enumcols}
		\item Между какими двумя последовательными натуральными числами находится число:
		\begin{enumcols}[itemcolumns=5]
			\item \( \sqrt{31} \)
			\item \( \sqrt{50} \)
			\item \( \sqrt{71} \)
			\item \( \sqrt{119} \)
			\item \( \sqrt{333} \)
	%		===============ДЗ===============
	%		\item \( \sqrt{38} \)
	%		\item \( \sqrt{73} \)
	%		\item \( \sqrt{114} \)
		\end{enumcols}
	\end{listofex}
	\begin{definit}
		Для любого \textbf{неотрицательного} числа \( A \) справедливо равенство: \( \left( \sqrt{A} \right)^2=A \)
	\end{definit}
	\begin{listofex}[resume]
		\item Вычислить:
		\begin{enumcols}[itemcolumns=3]
			\item \( (\sqrt{2})^2 \)
			\item \( (\sqrt{17})^2 \)
			\item \( (\sqrt{110})^2 \)
			\item \( (\sqrt{29})^2+(\sqrt{29})^2 \)
			\item \( (\sqrt{13})^2-(\sqrt{12})^2 \)
			\item \( (\sqrt{12}-\sqrt{11})(\sqrt{12}+\sqrt{11}) \)
	%		===============ДЗ===============
	%		\item \( (\sqrt{13})^2 \)
	%		\item \( (\sqrt{71})^2 \)
	%		\item \( (\sqrt{112})^2 \)
		\end{enumcols}
		\item Вычислить:
		\begin{enumcols}[itemcolumns=3]
			\item \( (-2\sqrt{11})^2-\sqrt{1,44} \)
			\item \( \dfrac{3}{11}\sqrt{1,21}-\dfrac{1}{5}(\sqrt{7})^2 \)
			\item \( (4\sqrt{3})^2-(3\sqrt{5})^2 \)
			\item \( \sqrt{529}-\left( \dfrac{1}{2}\sqrt{84} \right)^2 \)
			\item \( \sqrt{7\dfrac{1}{9}}+\sqrt{3\dfrac{1}{16}}-\dfrac{\sqrt{25}}{12} \)
			\item \( 32\cdot\left( -\dfrac{1}{2}\sqrt{11} \right):2 \)
	%		===============ДЗ===============
	%		Источники: Учебник 8-57 №3.15; Придумать самостоятельно
		\end{enumcols}
		\item Укажите все целые числа, расположенные на координатной прямой между числами:
		\begin{enumcols}[itemcolumns=4]
			\item \( 7 \) и \( \sqrt{102} \)
			\item \( \sqrt{17} \) и \( \sqrt{123} \)
			\item \( -\sqrt{62} \) и \( 6,2 \)
			\item \( -\sqrt{29} \) и \( -4,2 \)
		\end{enumcols}
		\item Расположите в порядке возрастания: \[ 4;\;3,8;\;\sqrt{15};\;\sqrt{19};\;4,3 \]
		\item Найдите значение выражения \( \left( \dfrac{12}{7}\sqrt{4-2a} \right)^2 \) при \( a=-22,5 \)
	\end{listofex}
\end{class}
%%
%%===============>>  Занятие 3  <<===============
%%
\begin{class}[number=3]
	\begin{definit}
		Корень из произведения неотрицательных множителей равен произведению корней из этих множителей. То есть если \( a\ge0 \) и \( b\ge0 \), то: \[ \sqrt{a \cdot b}=\sqrt{a}\cdot\sqrt{b} \]
	\end{definit}
	\begin{definit}
		Корень из дроби,  числитель которой неотрицателен, а знаменатель положителен, равен корню числителя, деленному на корень из знаменателя. То есть если \( a\ge0 \) и \( b>0 \), то: \[ \sqrt{\dfrac{a}{b}}=\dfrac{\sqrt{a}}{\sqrt{b}} \]
	\end{definit}
	\begin{listofex}
		\item Вычислить:
		\begin{enumcols}[itemcolumns=3]
			\item \( \sqrt{100\cdot49} \)
			\item \( \sqrt{81\cdot400} \)
			\item \( \sqrt{0,01\cdot169} \)
			\item \( \sqrt{81\cdot0,0049} \)
			\item \( \sqrt{25\cdot0,0529} \)
			\item \( \sqrt{2,25\cdot0,04} \)
			\item \( \sqrt{9\cdot64\cdot0,25} \)
			\item \( \sqrt{1,21\cdot0,09\cdot0,0001} \)
			
%			
			% ====== ДЗ =======
			%\item \exercise{1718}
			%\item \exercise{1722}
			%\item \exercise{1724}
		\end{enumcols}
		\item Вычислить:
		\begin{enumcols}[itemcolumns=3]
			\item \exercise{1719}
			\item \exercise{1722}
			\item \exercise{1727}
		\end{enumcols}
		\item Вычислить:
		\begin{enumcols}[itemcolumns=5]
			\item \( \sqrt{2}\cdot\sqrt{32} \)
			\item \( \sqrt{45}\cdot\sqrt{5} \)
			\item \( \sqrt{1,3}\cdot\sqrt{5,2} \)
			\item \( \sqrt{50}\cdot\sqrt{4,5} \)
			\item \( \sqrt{16,9}\cdot\sqrt{0,4} \)
		\end{enumcols}
		\item Вычислить:
		\begin{enumcols}[itemcolumns=4]
			\item \( \sqrt{21}\cdot\sqrt{\mfrac{3}{6}{7}} \)
			\item \( \sqrt{15}\cdot\sqrt{\mfrac{6}{2}{3}} \)
			\item \exercise{1710}
			\item \exercise{1728}
		\end{enumcols}
		\item Вычислить:
		\begin{enumcols}[itemcolumns=4]
			\item \( \sqrt{\dfrac{9}{64}} \)
			\item \( \sqrt{\dfrac{36}{25}} \)
			\item \( \sqrt{\mfrac{1}{9}{16}} \)
			\item \( \sqrt{\mfrac{5}{1}{16}} \)
			\item \exercise{1701}
			\item \exercise{1703}
			\item \exercise{1699}
			\item \exercise{2824}
		\end{enumcols}
		\item Вынести множитель из под знака корня:
		\begin{enumcols}[itemcolumns=8]
			\item \( \sqrt{8} \)
			\item \( \sqrt{18} \)
			\item \( \sqrt{32} \)
			\item \( \sqrt{75} \)
			\item \( \sqrt{12} \)
			\item \( \sqrt{98} \)
			\item \( \sqrt{250} \)
			\item \( \sqrt{200} \)
		\end{enumcols}
		\item Упростить:
		\begin{enumcols}[itemcolumns=2]
			\item \( 3\sqrt{5}+4\sqrt{5}-2\sqrt{5} \)
			\item \( 3,2\sqrt{13}-\dfrac{1}{8}\sqrt{13}+0,25\sqrt{13} \)
			\item \( \sqrt{12}+5\sqrt{3} \)
			\item \( \sqrt{27}-\sqrt{3} \)
			\item \( \sqrt{125}+\sqrt{50} \)
			\item \( \dfrac{1}{4}\sqrt{72}+1,5\sqrt{2} \)
			\item \( 9\sqrt{7}-2\sqrt{98} \)
			\item \( 0,5\sqrt{32}-1,2\sqrt{128} \)
%			
%			% ====== ДЗ =======
%			%\item \( 13\sqrt{7}-2\sqrt{7}+5\sqrt{7} \)
%			%\item \( 4,6\sqrt{5}-2,5\sqrt{5}+0,2\sqrt{45} \)
%			%\item \exercise{1630}
		\end{enumcols}
		\item Вычислить:
		\begin{enumcols}[itemcolumns=3]
			\item \exercise{1755}
			\item \exercise{1687}
			\item \exercise{1763}
			\item \exercise{1216}
			\item \exercise{1761}
			\item \exercise{2827}
		\end{enumcols}
		\item Вычислить:
		\begin{enumcols}[itemcolumns=3]
			\item \exercise{1620}
			\item \exercise{1627}
			\item \exercise{1773}
			\item \exercise{1779}
			\item \exercise{1638}
			\item \exercise{1756}
			\item \exercise{1218}
		\end{enumcols}
		\item Между какими двумя целыми числами стоит число:
		\begin{enumcols}[itemcolumns=3]
			\item \( \sqrt{223} \)
			\item \( \sqrt{1512} \)
			\item \( -\sqrt{215} \)
		\end{enumcols}
	\end{listofex}
\end{class}
%
%===============>>  Домашняя работа 1  <<===============
%
\begin{homework}[number=1]
	\begin{listofex}
		\item Вычислить:
		\begin{enumcols}[itemcolumns=4]
				\item \( \sqrt{36} \)
				\item \( \sqrt{64} \)
				\item \( \sqrt{81} \)
				\item \( \sqrt{121} \)
				\item \( \mfrac{2}{1}{4} \)
				\item \( \mfrac{1}{7}{9} \)
				\item \( \mfrac{1}{9}{16} \)
				\item \( \mfrac{5}{4}{9} \)
		\end{enumcols}
	\end{listofex}
\end{homework}
%
%===============>>  Занятие 4  <<===============
% смещение на одно занятие с прошлого месяца
\begin{class}[number=4]
	\begin{listofex}
		\item Вычислить:
		\begin{enumcols}[itemcolumns=3]
			\item \( \sqrt{400\cdot81} \)
			\item \( \sqrt{64\cdot900} \)
			\item \( \sqrt{0,001\cdot144} \)
			\item \( \sqrt{9\cdot0,0121} \)
			\item \( \sqrt{225\cdot0,16} \)
			\item \( \sqrt{2,56\cdot0,01} \)
			\item \( \sqrt{25\cdot81\cdot0,49} \)
			\item \( \sqrt{0,0064\cdot2500\cdot36} \)
			\item \( \sqrt{1,21\cdot121\cdot0,0121} \)
		\end{enumcols}
		\item Вычислить:
		\begin{enumcols}[itemcolumns=2]
			\item \exercise{4123}
			\item \exercise{4124}
		\end{enumcols}
		\item Вынести множитель из под знака корня:
		\begin{enumcols}[itemcolumns=8]
			\item \( \sqrt{12} \)
			\item \( \sqrt{20} \)
			\item \( \sqrt{40} \)
			\item \( \sqrt{125} \)
			\item \( \sqrt{72} \)
			\item \( \sqrt{288} \)
			\item \( \sqrt{360} \)
			\item \( \sqrt{500} \)
		\end{enumcols}
		\item Упростить:
		\begin{enumcols}[itemcolumns=3]
			\item \( 2\sqrt{6}+3\sqrt{6}-\sqrt{6} \)
			\item \( 2,5\sqrt{11}-\dfrac{1}{4}\sqrt{11}+0,36\sqrt{11} \)
			\item \( \sqrt{20}+6\sqrt{5}-0,5\sqrt{5} \)
			\item \( 5\sqrt{27}-10\sqrt{3} \)
			\item \( 2\sqrt{125}-5\sqrt{50} \)
			\item \( \dfrac{1}{2}\sqrt{98}+\dfrac{4}{3}\sqrt{20} \)
			\item \( 9\sqrt{50}-2\sqrt{8}+12\sqrt{18} \)
			\item \( 0,25\sqrt{108}-1,25\sqrt{75} \)
		\end{enumcols}
		\item Вычислить:
		\begin{enumcols}[itemcolumns=2]
			\item \exercise{1757}
			\item \exercise{2829}
			\item \exercise{1760}
			\item \exercise{1762}
			\item \exercise{1763}
			\item \exercise{2837}
		\end{enumcols}
		\item Вычислить:
		\begin{enumcols}[itemcolumns=2]
			\item \exercise{2826}
			\item \exercise{1650}
			\item \exercise{1659}
			\item \exercise{1667}
		\end{enumcols}
		\item Вычислить:
		\begin{enumcols}[itemcolumns=2]
			\item \exercise{1665}
			\item \exercise{4127}
		\end{enumcols}
		\item Вычислить:
		\begin{enumcols}[itemcolumns=2]
			\item \exercise{1787}
			\item \exercise{1648}
		\end{enumcols}
	\end{listofex}
\end{class}
%
%===============>>  Домашняя работа 2  <<===============
%
\begin{homework}[number=2]
	\begin{listofex}
		\item Упростите выражение:
		 \begin{enumcols}[itemcolumns=2]
		 	\item \( \sqrt{\dfrac{2}{5}}-0,5\sqrt{160}+3\sqrt{\mfrac{1}{1}{9} }\)
		 	\item \( \sqrt{63}-3\sqrt{1,75}-0,5\sqrt{343}+\sqrt{112} \)
		 	\item \( 2\sqrt{9,5}-\sqrt{152}+9\sqrt{\mfrac{4}{2}{9}} \)
		 \end{enumcols}
	 \item Найдите значение выражения \( 10ab-(a+5b)^2 \) при \( a=\sqrt{8}, b=\sqrt{14} \)
	 \item Найдите значение выражения:
	  \begin{enumcols}[itemcolumns=2]
	 	\item \( \dfrac{12}{12-5\sqrt{6}}-\dfrac{12}{12+5\sqrt{6}} \)
	 	\item \( \dfrac{\sqrt{7}+\sqrt{2}}{\sqrt{7}-\sqrt{2}}+\dfrac{\sqrt{7}-\sqrt{2}}{\sqrt{7}+\sqrt{2}} \)
	 \end{enumcols}
 	\item Решите уравнения:
 	\begin{enumcols}[itemcolumns=3]
 		\item \( 5x^2+10x=0 \)
 		\item \( 4x^2-16x=0 \)
 		\item \( x^2=x \)
 		\item \( -25x^2=1 \)
 		\item \( 22x^2=88 \)
 		\item \( 8x=-x^2 \)
 	\end{enumcols}
 	\item Вычислите:
 	\begin{enumcols}[itemcolumns=2]
 		\item \( \sqrt{77}\cdot\sqrt{24}\cdot\sqrt{33}\cdot\sqrt{14} \)
 		\item \( \sqrt{\dfrac{165^2-124^2}{164}} \)
 		\item \( \sqrt{\dfrac{145,5^2-96,5^2}{193,5^2-31,5^2}} \)
 		\item \( \sqrt{10}\cdot\sqrt{20}\cdot\sqrt{48}\cdot\sqrt{36}\cdot\sqrt{75}\cdot\sqrt{98} \)
 	\end{enumcols}
	\end{listofex}
\end{homework}
%
%===============>>  Занятие 5  <<===============
%
\begin{class}[number=5]
	\begin{listofex}
		\item Вычислить:
		\begin{enumcols}[itemcolumns=3]
			\item \( \dfrac{21^3}{7^{15}\cdot3^{12}} \)
			\item \( \vphantom{\dfrac{1^1}{1^1}}(\sqrt{11})^2-\sqrt{1,44} \)
			\item \( \sqrt{784}-\left( \dfrac{1}{7}\sqrt{343} \right)^2 \)
		\end{enumcols}
	\end{listofex}
	\begin{definit}
		Квадратное уравнение --- уравнение вида \( ax^2+bx+c=0 \), где \( a,\, b,\, c \) --- числа. Если \( b \) или \( c \) будут равны 0, то такое квадратное уравнение называют \textbf{неполным квадратным уравнением.}
	\end{definit}
	\begin{definit}
		Неполное квадратное уравнение вида \( ax^2=c \) решается следующим образом:
		\[ x^2=\dfrac{c}{a};\quad x=\pm\sqrt{\dfrac{c}{a}} \]
	\end{definit}
	\begin{listofex}[resume]
		\item Решить уравнение:
		\begin{enumcols}[itemcolumns=4]
			\item \exercise{386}
			\item \exercise{387}
			\item \exercise{388}
			\item \exercise{390}
			\item \( 3x^2=108 \)
			\item \( 4x^2-49=0 \)
			\item \( 9x^2=25 \)
			\item \( 0,04x^2=0,01 \)
			\item \( 5x^2=45 \)
			\item \( \dfrac{2}{5}x^2=40 \)
			\item \( 0,01x^2=4 \)
		\end{enumcols}
	\end{listofex}
	\begin{definit}
	Распадающиеся уравнения --- уравнения, где левая часть состоит из множителей, а во второй части --- \( 0 \). Чтобы решить такое уравнение, приравнивают каждый множитель отдельно к нулю и решают получившиеся уравнения.
	\begin{center}
		\( (x-7)(x+3)=0 \)\\
		\( x-7=0 \) или \( x+3=0 \)\\
		\( x=7 \) или \( x=-3 \)
	\end{center}
	\end{definit}
	\begin{listofex}[resume]
		\item Решить уравнение:
		\begin{enumcols}[itemcolumns=2]
			\item \( (x+1)(x-3)=0 \)
			\item \( (2x-11)(3x-4)=0 \)
			\item \( (3x-3)(2x+24)(5x-12)=0 \)
			\item \( \left( \dfrac{2}{3}x+\dfrac{1}{2} \right)\left( \dfrac{15}{2}-\dfrac{3}{5}x \right)=0 \)
			\item \( (2x-10)(0,4x-2)=0 \)
			\item \( (0,01x-5,42)(0,2+5x)=0 \)
		\end{enumcols}
	\end{listofex}
	\begin{definit}
		Неполное квадратное уравнение вида \( ax^2+bx=0 \) решается следующим образом:
		\begin{center}
			\( ax^2+bx=0 \)\\
		\( x(ax+b)=0 \)\\
			\( x=0 \) или \( ax+b=0 \)
		\end{center}
	\end{definit}
	\begin{listofex}[resume]
		\item Решить уравнение:
		\begin{enumcols}[itemcolumns=3]
			\item \exercise{401}
			\item \exercise{402}
			\item \exercise{404}
			\item \exercise{415}
			\item \exercise{417}
			\item \exercise{420}
			\item \exercise{424}
			\item \exercise{425}
		\end{enumcols}
		\newpage
		\item Решить уравнение:
		\begin{enumcols}[itemcolumns=2]
			\item \exercise{430}
			\item \exercise{431}
			\item \exercise{432}
			\item \exercise{435}
		\end{enumcols}
		\item Решить уравнение:
		\begin{enumcols}[itemcolumns=1]
			\item \exercise{437}
			\item \exercise{438}
			\item \exercise{440}
			\item \exercise{442}
		\end{enumcols}
		\item Решить уравнение:
		\begin{enumcols}[itemcolumns=2]
			\item \exercise{443}
			\item \exercise{445}
		\end{enumcols}
		\item В двух школах поселка было \( 1500 \) учащихся. Через год число учащихся первой
		школы увеличилось на \( 10\% \), а второй --- на \( 20\% \), в результате чего общее число учащихся стало равным \( 1720 \). Сколько учащихся было в каждой школе первоначально?
	\end{listofex}
\end{class}

%===============>>  Занятие 6  <<===============
%
\begin{class}[number=6]
	\begin{listofex}
		\item Решить уравнение:
		\begin{enumcols}[itemcolumns=2]
			\item \( (2x-1)(x-13)=0 \)\answer{\( 0,5;\;13 \)}
			\item \( (4x-5)(2x+58)=0 \)\answer{\( -29;\;1,25 \)}
			\item \( (11x-121)(225x+15)6x=0 \)\answer{\( -\dfrac{1}{15};\;0;\;11 \)}
			\item \( \left( \dfrac{1}{12}x+\dfrac{4}{5} \right)(0,5x-12)=0 \)\answer{\( -9,6;\;24 \)}
		\end{enumcols}
		\item Решить уравнение:
		\begin{enumcols}[itemcolumns=3]
			\item \( 2x^2-3x=0 \)\answer{\( 0;\;1,5 \)}
			\item \( 15x^2+15x=0 \)\answer{\( -1;\;0 \)}
			\item \( 3x^3+2x^2=0 \)\answer{\( -\dfrac{2}{3};\;0 \)}
		\end{enumcols}
		\item Решить уравнение:
		\begin{enumcols}[itemcolumns=2]
			\item \exercise{435}
			\item \( 7x^2-12x=2x^2-14x \)\answer{\( -0,4;\;0 \)}
			\item \( 6x^2-15=3x^2+12 \)\answer{\( -3;\;3 \)}
			\item \( 3x^2+7x-1=2x^2+7x+80 \)\answer{\( -9;\;9 \)}
		\end{enumcols}
		\item Решить уравнение:
		\begin{enumcols}[itemcolumns=1]
			\item \exercise{441}
			\item \exercise{439}
		\end{enumcols}
		\item Решить уравнение: \( \dfrac{3x-4x^2}{5}-\dfrac{5x^2-x}{4}=\dfrac{3x^2}{2} \)\answer{\( 0;\;\dfrac{17}{71} \)}
	\end{listofex}
\end{class}
%
%===============>>  Домашняя работа 3  <<===============
%
%\begin{homework}[number=2]
%	\begin{listofex}
%
%	\end{listofex}
%\end{homework}
%\newpage
%\title{Подготовка к проверочной работе}
%\begin{listofex}
%	
%\end{listofex}
%
%===============>>  Занятие 7  <<===============
%
\begin{class}[number=7]
	\begin{listofex}
		\item Вычислить:
		\begin{enumcols}[itemcolumns=3]
			\item \( 1,5^7:3^6 \)
			\item \( \dfrac{10^4\cdot9^3}{6^4\cdot5^3} \)
			\item \exercise{1529}
		\end{enumcols}
		\item Вычислить:
		\begin{enumcols}[itemcolumns=3]
			\item \( \dfrac{\sqrt{9}}{\sqrt{64}}-2 \)
			\item \( \dfrac{4}{\sqrt{256}}-\dfrac{1}{\sqrt{225}} \)
			\item \( 2,01-2\sqrt{0,0441} \)
			\item \( \dfrac{\sqrt{444}}{\sqrt{111}} \)
			\item \( \dfrac{\sqrt{2,8}\cdot\sqrt{4,2}}{\sqrt{0,24}} \)
			\item \( (0,2\sqrt{10})^2+(0,5\sqrt{3})^3 \)
		\end{enumcols}
		\item Вычислить:
		\begin{enumcols}[itemcolumns=3]
			\item \( \sqrt{6,8^2-3,2^2} \)
			\item \( \sqrt{7+4\sqrt{3}}\cdot\sqrt{7-4\sqrt{3}} \)
			\item \( (2\sqrt{12}-3\sqrt{3})^2 \)
		\end{enumcols}
		\item Вычислить:
		\begin{enumcols}[itemcolumns=2]
			\item \exercise{1747}
			\item \exercise{1661}
			\item \exercise{1638}
		\end{enumcols}
		\item \exercise{1384}
		\item \exercise{4140}
		\item \exercise{4141}
		\item \exercise{1750}
		\item Решить уравнение:
		\begin{enumcols}[itemcolumns=2]
			\item \( x^2=121 \)
			\item \( \dfrac{2}{7}x^2=3,5 \)
			\item \( \dfrac{1}{3}x^2-\dfrac{5}{12}x=0 \)
			\item \( 0,2x^2+3x=(0,5x)^2+2x \)
		\end{enumcols}
		\item Каждый из двух рабочих одинаковой квалификации может выполнить заказ за \( 15 \) ч. Через \( 5 \) ч после того, как один из них приступил к выполнению заказа, к нему присоединился второй рабочий, и работу над заказом они довели до конца вместе. За какое время был выполнен заказ?
	\end{listofex}
\end{class}
%
%===============>>  Провечная работа  <<===============
%
\begin{exam}
	\vphantom{a}
	\begin{listofex}
		\item Какое из чисел отмечено на координатной прямой точкой $A$?
		\begin{center}
			\includegraphics{pics/G82M3E-1}
		\end{center}
		
		\selectanswer
		\begin{enumcols}[itemcolumns=4]
			\item $\sqrt{4}$
			\item $\sqrt{1}$
			\item $\sqrt{2}$
			\item $\sqrt{5}$
		\end{enumcols}
		\item Вычислить:
		\begin{enumcols}[itemcolumns=3]
			\item \( \dfrac{5^{10}\cdot(5^3)^4}{5^{18}} \)
			\item \( 2,5^{43}\cdot\left( \dfrac{2}{5} \right)^{41} \)
			\item \( \dfrac{100^8}{2^{15}\cdot5^{14}} \)
		\end{enumcols}
		\item Вычислить:
		\begin{enumcols}[itemcolumns=3]
			\item \( \sqrt{225\cdot900} \)
			\item \( \sqrt{144\cdot0,0081} \)
			\item \exercise{1719}
		\end{enumcols}
		\item Вычислить:
		\begin{enumcols}[itemcolumns=3]
			\item \( \sqrt{\mfrac{6}{4}{16}} \)
			\item \( \dfrac{\sqrt{15}}{\sqrt{735}} \)
			\item \( \dfrac{\sqrt{1,25}\cdot\sqrt{17,4}}{\sqrt{0,87}} \)
		\end{enumcols}
		\item Вычислить:
		\begin{enumcols}[itemcolumns=3]
			\item \( \sqrt{9,05^2-0,95^2} \)
			\item \( \sqrt{5+2\sqrt{6}}\cdot\sqrt{5-2\sqrt{6}} \)
			\item \( (3\sqrt{20}+2\sqrt{5})^2 \)
		\end{enumcols}
		\item Отметьте числа на числовой прямой и в ответе расположите в порядке убывания:
		\[ 3,\;2\sqrt{2},\;-\sqrt{15},\;-2,\;\sqrt{5},\;4,\;-3\sqrt{3} \]
		\item Вычислить:
		\begin{enumcols}[itemcolumns=2]
			\item \exercise{1649}
			\item \exercise{1766}
		\end{enumcols}
		\item \exercise{4127}
		\item Решить уравнение:
		\begin{enumcols}[itemcolumns=3]
			\item \( x^2=0,25 \)
			\item \( \dfrac{2}{15}x^2=7,5 \)
			\item \( 3x^2-12x=-10x-(4x)^2 \)
		\end{enumcols}
	\end{listofex}
\end{exam}