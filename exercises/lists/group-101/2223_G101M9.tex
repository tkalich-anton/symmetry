\setmodule{7}

%BEGIN_FOLD % ====>>_____ Занятие 1 _____<<====
\begin{class}[number=1]
	\begin{listofex}
		\item Два ребра прямоугольного параллелепипеда, выходящие из одной вершины, равны \(1, 2\). Объем параллелепипеда равен \(6\). Найдите площадь его поверхности.
		\item Два ребра прямоугольного параллелепипеда равны \(7\) и \(4\), а объём параллелепипеда равен \(140\). Найдите площадь поверхности этого параллелепипеда.
		\item 
		\begin{minipage}[t]{\bodywidth}
			В прямоугольном параллелепипеде \(ABCDA_1B_1C_1D_1\) рёбра \(AB, BC\) и диагональ боковой грани \(BC_1\) равны соответственно \(7, \) \( 3\) и \(3\sqrt{5}\). Найдите объём параллелепипеда \(ABCDA_1B_1C_1D_1\).
		\end{minipage}
		\hspace{0.02\linewidth}
		\begin{minipage}[t]{\picwidth}
			\includegraphics[align=t, width=\linewidth]{\picpath/G101M9L1-1}
		\end{minipage}
		\item 
		\begin{minipage}[t]{\bodywidth}
			В прямоугольном параллелепипеде \(ABCDA_1B_1C_1D_1\) рёбра \(CD, CB\) и диагональ \(CD_1\) боковой грани равны соответственно \(2, 4\) и \(2\sqrt{10}\). Найдите площадь поверхности параллелепипеда \(ABCDA_1B_1C_1D_1\).
		\end{minipage}
		\hspace{0.02\linewidth}
		\begin{minipage}[t]{\picwidth}
			\includegraphics[align=t, width=\linewidth]{\picpath/G101M9L1-1}
		\end{minipage}
		\item Основанием прямой треугольной призмы служит прямоугольный треугольник с катетами \(6\) и \(8\), боковое ребро равно \(5\). Найдите объем призмы.
		\item В основании прямой призмы лежит прямоугольный треугольник, один из катетов которого равен \(2\), а гипотенуза равна \(\sqrt{53}\). Найдите объём призмы, если её высота равна \(3\).
		\item В основании прямой призмы лежит прямоугольный треугольник, катеты которого равны \(11\) и \(5\). Найдите объём призмы, если её высота равна \(4\).
		\item 
		\begin{minipage}[t]{\bodywidth}
			Стороны основания правильной шестиугольной пирамиды равны \(10\), боковые ребра равны \(13\). Найдите площадь боковой поверхности этой пирамиды.
		\end{minipage}
		\hspace{0.02\linewidth}
		\begin{minipage}[t]{\picwidth}
			\includegraphics[align=t, width=\linewidth]{\picpath/G101M9L1-2}
		\end{minipage}
		\item 
		\begin{minipage}[t]{\bodywidth}
			Основанием пирамиды является прямоугольник со сторонами \(3\) и \(4\). Ее объем равен \(16\). Найдите высоту этой пирамиды.
		\end{minipage}
		\hspace{0.02\linewidth}
		\begin{minipage}[t]{\picwidth}
			\includegraphics[align=t, width=\linewidth]{\picpath/G101M9L1-3}
		\end{minipage}
		\item 
		\begin{minipage}[t]{\bodywidth}
			Основанием пирамиды является прямоугольник со сторонами \(4\) и \(5\). Ее объем равен \(80\). Найдите высоту этой пирамиды.
		\end{minipage}
		\hspace{0.02\linewidth}
		\begin{minipage}[t]{\picwidth}
			\includegraphics[align=t, width=\linewidth]{\picpath/G101M9L1-3}
		\end{minipage}
		\item 
		\begin{minipage}[t]{\bodywidth}
			Найдите объём правильной четырёхугольной пирамиды, сторона основания которой равна \(4\), а боковое ребро равно \(\sqrt{17}\).
		\end{minipage}
		\hspace{0.02\linewidth}
		\begin{minipage}[t]{\picwidth}
			\includegraphics[align=t, width=\linewidth]{\picpath/G101M9L1-3}
		\end{minipage}
		\item 
		\begin{minipage}[t]{\bodywidth}
			В основании пирамиды \(SABC\) лежит правильный треугольник \(ABC\) со стороной \(10\), а боковое ребро \(SA\) перпендикулярно основанию и равно Найдите объём пирамиды \(SABC\).
		\end{minipage}
		\hspace{0.02\linewidth}
		\begin{minipage}[t]{\picwidth}
			\includegraphics[align=t, width=\linewidth]{\picpath/G101M9L1-4}
		\end{minipage}
		\item 
		\begin{minipage}[t]{\bodywidth}
			В треугольной пирамиде \(ABCD\) рёбра \(AB, AC\) и \(AD\) взаимно перпендикулярны. Найдите объём этой пирамиды, если \(AB = 6, AC = 18\) и \(AD = 8\).
		\end{minipage}
		\hspace{0.02\linewidth}
		\begin{minipage}[t]{\picwidth}
			\includegraphics[align=t, width=\linewidth]{\picpath/G101M9L1-5}
		\end{minipage}
		\newpage
		\item Стороны основания правильной треугольной пирамиды равны \(16\), а боковые рёбра равны \(10\). Найдите площадь боковой поверхности пирамиды.
		\item Даны два цилиндра. Радиус основания и высота первого равны соответственно \(4\) и \(18\), а второго --- \(2\) и \(3\). Во сколько раз площадь боковой поверхности первого цилиндра больше площади боковой поверхности второго?
		
		
		%\item 
		%\begin{minipage}[t]{\bodywidth}
		%	
		%\end{minipage}
		%\hspace{0.02\linewidth}
		%\begin{minipage}[t]{\picwidth}
		%	\includegraphics[align=t, width=\linewidth]{\picpath/G101M9L1-1}
		%\end{minipage}
	\end{listofex}
\end{class}
%END_FOLD

%BEGIN_FOLD % ====>>_____ Занятие 2 _____<<====
\begin{class}[number=2]
	\begin{listofex}
		\item Занятие 2
	\end{listofex}
\end{class}
%END_FOLD

%BEGIN_FOLD % ====>>_ Домашняя работа 1 _<<====
\begin{homework}[number=1]
	\begin{listofex}
		
		\item Основанием прямой треугольной призмы служит прямоугольный треугольник с катетами \(12\) и \(5\), боковое ребро равно \(6\). Найдите объем призмы.
		\item В основании прямой призмы лежит прямоугольный треугольник, катеты которого равны по \(12\). Найдите объём призмы, если её высота равна \(2\).
		\item Два ребра прямоугольного параллелепипеда, выходящие из одной вершины, равны \(4, 6\). Объем параллелепипеда равен \(48\). Найдите площадь его поверхности.
		\item 
		\begin{minipage}[t]{\bodywidth}
			В основании пирамиды \(SABC\) лежит правильный треугольник \(ABC\) со стороной \(4\), а боковое ребро \(SA\) перпендикулярно основанию и равно \(7\) Найдите объём пирамиды \(SABC\).
		\end{minipage}
		\hspace{0.02\linewidth}
		\begin{minipage}[t]{\picwidth}
			\includegraphics[align=t, width=\linewidth]{\picpath/G101M9L1-4}
		\end{minipage}
		\item Объём конуса равен \(50\pi \), а его высота равна \(6\). Найдите радиус основания конуса.
		\item Даны два конуса. Радиус основания и образующая первого конуса равны, соответственно, \(2\) и \(4\), а второго --- \(6\) и \(8\). Во сколько раз площадь боковой поверхности второго конуса больше площади боковой поверхности первого?
		\item Даны два конуса. Радиус основания и высота первого конуса равны соответственно \(9\) и \(2\), а второго --- \(3\) и \(3\). Во сколько раз объём первого конуса больше объёма второго?
		\item Даны два шара с радиусами \(5\) и \(1\). Во сколько раз площадь поверхности первого шара больше площади поверхности второго?
		\item Даны два шара с радиусами \(4\) и \(1\). Во сколько раз объём большего шара больше объёма другого?
		\item Даны два шара с радиусами \(9\) и \(3\). Во сколько раз площадь поверхности большего шара больше площади поверхности меньшего?
	\end{listofex}
\end{homework}
%END_FOLD

%BEGIN_FOLD % ====>>_____ Занятие 3 _____<<====
\begin{class}[number=3]
	\begin{listofex}
		\item Даны два конуса. Радиус основания и высота первого конуса равны соответственно \(5\) и \(10\), а второго --- \(3\) и \(2\). Во сколько раз объём первого конуса больше объёма второго?
		\item Даны два шара с радиусами \(8\) и \(11\). Во сколько раз площадь поверхности первого шара больше площади поверхности второго?
		\item Даны два шара с радиусами \(2\) и \(8\). Во сколько раз объём большего шара больше объёма другого?
		\item Даны два шара с радиусами \(6\) и \(14\). Во сколько раз площадь поверхности большего шара больше площади поверхности меньшего?
		
	\end{listofex}
\end{class}
%END_FOLD

%BEGIN_FOLD % ====>>_____ Занятие 4 _____<<====
\begin{class}[number=4]
	\begin{listofex}
		\item Занятие 4
	\end{listofex}
\end{class}
%END_FOLD

%BEGIN_FOLD % ====>>_ Домашняя работа 2 _<<====
\begin{homework}[number=2]
	\begin{listofex}
		\item Домашняя работа 2
	\end{listofex}
\end{homework}
%END_FOLD

%BEGIN_FOLD % ====>>_____ Занятие 5 _____<<====
\begin{class}[number=5]
	\begin{listofex}
		\item Занятие 5
	\end{listofex}
\end{class}
%END_FOLD

%BEGIN_FOLD % ====>>_____ Занятие 6 _____<<====
\begin{class}[number=6]
	\begin{listofex}
		\item Занятие 6
	\end{listofex}
\end{class}
%END_FOLD

%BEGIN_FOLD % ====>>_ Домашняя работа 3 _<<====
\begin{homework}[number=3]
	\begin{listofex}
		\item Домашняя работа 3
	\end{listofex}
\end{homework}
%END_FOLD

%BEGIN_FOLD % ====>>_____ Занятие 7 _____<<====
\begin{class}[number=7]
	\title{Подготовка к проверочной}
	\begin{listofex}
		\item Занятие 7
	\end{listofex}
\end{class}
%END_FOLD

%BEGIN_FOLD % ====>>_ Проверочная работа _<<====
\begin{exam}
	\begin{listofex}
		\item Проверочная
	\end{listofex}
\end{exam}
%END_FOLD