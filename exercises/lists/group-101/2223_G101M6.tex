%
%===============>>  ГРУППА 10-1 МОДУЛЬ 6  <<=============
%
\setmodule{6}

%BEGIN_FOLD % ====>>_____ Занятие 1 _____<<====
\begin{class}[number=1]
	\begin{listofex}
		\item Вычислите:
		\begin{tasks}(4)
			\task \( \sqrt[4]{16} \)
			\task \( \sqrt[5]{243} \)
			\task \( \sqrt[3]{64} \)
			\task \( \sqrt[3]{-27} \)
			\task \( \sqrt[4]{64} \)
			\task \( \sqrt[5]{-64} \)
			\task \( \sqrt[3]{0,125} \)
			\task \( \sqrt[7]{-128} \)
		\end{tasks}
		\item Найдите значение выражения:
		\begin{tasks}(4)
			\task \( \sqrt[3]{a^6b^9} \)
			\task \( \sqrt[3]{64x^3z^6} \)
			\task \( \sqrt[3]{a^{12}b^3} \)
			\task \( \sqrt[3]{1000a^6b^6} \)
		\end{tasks}
		\item Найдите значение выражения:
		\begin{tasks}(2)
			\task \( \sqrt[3]{ab^2} \cdot \sqrt[3]{a^2b} \)
			\task \( \sqrt[3]{a^2b^4} \cdot \sqrt[3]{ab^2} \)
		\end{tasks}
		\item Вычислите:
		\begin{tasks}(3)
			\task \( \sqrt[3]{27 \cdot 0,001 \cdot 125} \)
			\task \( \sqrt[4]{625 \cdot 0,0016 \cdot 256} \)
			\task \( \sqrt[5]{243 \cdot 0,00001 \cdot 0,00032} \)
		\end{tasks}
		\item Вынесите множитель из под знака корня:
		\begin{tasks}(4)
			\task \( \sqrt[3]{40} \)
			\task \( \sqrt[5]{-64} \)
			\task \( \sqrt[5]{-96} \)
			\task \( \sqrt[3]{54} \)
		\end{tasks}
		\item Вычислите:
		\begin{tasks}(4)
			\task \( \sqrt[3]{125 \cdot 27} \)
			\task \( \sqrt[4]{16 \cdot 625} \)
			\task \( \sqrt[4]{81 \cdot 16} \)
			\task \( \sqrt[4]{5 \cdot 125} \)
		\end{tasks}
		\item Решите уравнения:
		\begin{tasks}(2)
			\task \( \sqrt[3]{15-2x} = 3 \)
			\task \( \sqrt[4]{2x-8}=3 \)
			\task \( \sqrt[3]{4x^2-5x-1}=2 \)
			\task \( \sqrt[]{-72-17x}=-x \)
		\end{tasks}
		\item Из пункта \(A\) в пункт \(B\) одновременно выехали два автомобиля. Первый проехал с постоянной скоростью весь путь. Второй проехал первую половину пути со скоростью, меньшей скорости первого на \(13\) км/ч, а вторую половину пути  --- со скоростью \(78\) км/ч, в результате чего прибыл в пункт \(B\) одновременно с первым автомобилем. Найдите скорость первого автомобиля, если известно, что она больше \(48\) км/ч. Ответ дайте в км/ч.
	\end{listofex}
\end{class}
%END_FOLD

%BEGIN_FOLD % ====>>_____ Занятие 2 _____<<====
\begin{class}[number=2]
	\begin{listofex}
		\item Вычислите:
		\begin{tasks}(2)
			\task \( \sqrt[3]{512} \)
			\task \( \sqrt[7]{128} \)
			\task \( \sqrt[5]{1024} \)
			\task \( \sqrt[4]{81} \)
			\task \( \sqrt[5]{8} \cdot \sqrt[5]{4} \)
			\task \( \sqrt[3]{4} \cdot \sqrt[3]{8} \cdot \sqrt[3]{-2} \)
			\task \( \sqrt[3]{4} \cdot \sqrt[3]{16} \)
			\task \( \sqrt[5]{-7} \cdot \sqrt[5]{49} \cdot \sqrt[5]{49} \)
		\end{tasks}
		\item Вычислите:
		\begin{tasks}(3)
			\task \( \sqrt[3]{1000 \cdot 512} \)
			\task \( \sqrt[3]{343 \cdot 0,125} \)
			\task \( \sqrt[4]{1296 \cdot 0,0001 \cdot 1024} \)
		\end{tasks}
		\item Извлеките корень:
		\begin{tasks}(3)
			\task \( \sqrt[]{ 81a^2b^8 } \)
			\task \( \sqrt[]{ 25a^6b^4 } \)
			\task \( \sqrt[3]{ 8a^3b^6 } \)
			\task \( \sqrt[3]{ 0,001a^6b^{12} } \)
			\task \( \sqrt[4]{ 1000a^4b^{16} } \)
			\task \( \sqrt[4]{ 81a^8b^{12}} \)
		\end{tasks}
		\item Извлеките корень из дроби:
		\begin{tasks}(3)
			\task \( \sqrt[3]{\dfrac{1}{64}} \)
			\task \( \sqrt[4]{\dfrac{81}{16}} \)
			\task \( \sqrt[5]{\dfrac{32}{243}} \)
		\end{tasks}
		\item Вынесите множитель из под знака корня:
		\begin{tasks}(4)
			\task \( \sqrt[3]{\dfrac{3}{8}} \)
			\task \( \sqrt[3]{\dfrac{27}{4}} \)
			\task \( \sqrt[3]{-\dfrac{250}{16}} \)
			\task \( \sqrt[3]{-\dfrac{64}{7}} \)
		\end{tasks}
		\item Вычислите:
		\begin{tasks}(4)
			\task \( \sqrt[]{(-2)^2} \)
			\task \( \sqrt[]{(-5)^4} \)
			\task \( \sqrt[]{(\sqrt{2}-1)^2} \)
			\task \( \sqrt[]{(\sqrt{2}-2)^2} \)
		\end{tasks}
		\item Упростите выражение:
		\begin{tasks}(2)
			\task \( \sqrt[4]{81 \cdot (4-\sqrt{17})^4} \)
			\task \( \sqrt[3]{0,001} - \sqrt[6]{0,000064} \)
		\end{tasks}
		\item Решите уравнения:
		\begin{tasks}(2)
			\task \( \sqrt[3]{5x-10} = 3x \)
			\task \( \sqrt[3]{x^2+1}=2 \)
			\task \( \sqrt[3]{x^2-2x+5}=2 \)
			\task \( \sqrt[]{5x+6}=2x \)
		\end{tasks}
		\item Из пункта \(A\) в пункт \(B\), расстояние между которыми \(75\) км, одновременно выехали автомобилист и велосипедист. Известно, что за час автомобилист проезжает на \(40\) км больше, чем велосипедист. Определите скорость велосипедиста, если известно, что он прибыл в пункт \(B\) на \(6\) часов позже автомобилиста. Ответ дайте в км/ч.
	\end{listofex}
\end{class}
%END_FOLD

%BEGIN_FOLD % ====>>_ Домашняя работа 1 _<<====
\begin{homework}[number=1]
	\begin{listofex}
		\item Вычислите:
		\begin{tasks}(2)
			\task \( \sqrt[3]{343} \)
			\task \( \sqrt[4]{625} \)
			\task \( \sqrt[5]{32} \)
			\task \( \sqrt[4]{1296} \)
		\end{tasks}
		\item Извлеките корень из дроби:
		\begin{tasks}(3)
			\task \( \sqrt[3]{\dfrac{125}{216}} \)
			\task \( \sqrt[4]{\dfrac{625}{16}} \)
			\task \( \sqrt[3]{\dfrac{512}{27}} \)
		\end{tasks}
		\item Извлеките корень:
		\begin{tasks}(3)
			\task \( \sqrt[]{ 25a^4b^6 } \)
			\task \( \sqrt[3]{ 27a^6b^9 } \)
			\task \( \sqrt[5]{ 1024a^{15}b^{20} } \)
		\end{tasks}
		\item Вычислите:
		\begin{tasks}(2)
			\task \( \sqrt[]{(\sqrt{2}-\sqrt{3})^2} \)
			\task \( \dfrac{(\sqrt{7}-\sqrt{6})^3(\sqrt{7}+\sqrt{6})^3}{0,125} \)
		\end{tasks}
		\item Вычислите:
		\begin{tasks}(2)
			\task \( (\sqrt[3]{-2})^3 + (\sqrt[5]{8})^5 \)
			\task \( \sqrt[5]{-1} + \sqrt[3]{-8} \)
			\task \( \sqrt[5]{9} \cdot \sqrt[5]{4} \)
			\task \( \sqrt[3]{-4} \cdot \sqrt[3]{8} \cdot \sqrt[3]{-2} \)
		\end{tasks}
		\item Решите уравнения:
		\begin{tasks}(2)
			\task \( \sqrt[]{4x+4} = x-2 \)
			\task \( \sqrt[3]{x^2+4x-4}=2 \)
		\end{tasks}
		\item Из пункта \(A\) в пункт \(B\) одновременно выехали два автомобиля. Первый проехал с постоянной скоростью весь путь. Второй проехал первую половину пути со скоростью \(24\) км/ч, а вторую половину пути --- со скоростью, на \(16\) км/ч большей скорости первого, в результате чего прибыл в пункт \(B\) одновременно с первым автомобилем. Найдите скорость первого автомобиля. Ответ дайте в км/ч.
	\end{listofex}
\end{homework}
%END_FOLD

%BEGIN_FOLD % ====>>_____ Занятие 3 _____<<====
\begin{class}[number=3]
	\begin{definit}
		Число \(a\) в степени \(\dfrac{p}{q}\) является арифметическим корнем степени \(q\) из числа \(a\) в степени \(p\): \[ a^{\frac{p}{q}}=\sqrt[q]{a^p} \].
	\end{definit}
	\begin{listofex}
		\item Запишите в виде корней:
		\begin{tasks}(3)
			\task \( a^{\tfrac{1}{2}} \)
			%\task \( a^{\dfrac{1}{3}} \)
			\task \( b^{\tfrac{1}{4}} \)
			\task \( (d-c)^{\tfrac{1}{2}} \)
			\task \( (e+4)^{\tfrac{3}{2}} \)
			\task \( a^{1\tfrac{4}{3}} \)
			\task \( b^{1,4} \)
		\end{tasks}
		\item Вычислите:
		\begin{tasks}(4)
			\task \( 25^{\tfrac{1}{2}} \)
			\task \( 49^{\tfrac{1}{2}} \)
			\task \( 16^{0,25} \)
			\task \( 64^{0,25} \)
			\task \( 100^{0,5} \)
			\task \( 16^{\tfrac{3}{4}} \)
			\task \( 27^{\tfrac{2}{3}} \)
			\task \( 25^{2,5} \)
		\end{tasks}
		\item Запишите в виде корня \(n\)-ой степени:
		\begin{tasks}(3)
			\task \( a^{-\tfrac{1}{2}} \)
			\task \( b^{-\tfrac{2}{3}} \)
			\task \( c^{-\tfrac{11}{5}} \)
			\task \( d^{-0,5} \)
			\task \( a^{-2,5} \)
			\task \( b^{-0,25} \)
		\end{tasks}
		\item Вычислите:
		\begin{tasks}(3)
			\task \( 8^{-\tfrac{2}{3}} \)
			\task \( 16^{-\tfrac{3}{2}} \)
			\task \( 64^{-\tfrac{5}{6}} \)
			\task \( 32^{-0,4} \)
			\task \( \left( \dfrac{2}{7}\right)^{-2} \cdot \left( \dfrac{49}{25} \right)^{\tfrac{1}{2}} \)
			\task \( 0,01^{-\tfrac{1}{2}} \cdot (6,25)^{-0,5} \)
		\end{tasks}
		\item Найдите значение выражения:
		\begin{tasks}(2)
			\task \( 5^{0,36} \cdot 25^{0,32} \)
			\task \( 7^{\tfrac{4}{9}} \cdot 49^{\tfrac{5}{18}} \)
			\task \( \dfrac{3^{6,5}}{9^{2,25}} \)
			\task \( \dfrac{2^{3,5}\cdot3^{5,5}}{6^{4,55}} \)
		\end{tasks}
		\item Решите уравнения:
		\begin{tasks}(2)
			\task \( 4^x=64 \)
			\task \( \sqrt[3]{128}=4^{2x} \)
			\task \( \left( \dfrac{1}{64^2} \right)^{-x}=\sqrt{\dfrac{1}{8}} \)
			\task \( 0,5^{2x}\cdot 4^{x+1}=64^{-1} \)
		\end{tasks}
		\item Упростите выражение: \[   \left( \dfrac{4}{a^{1,5}-8} - \dfrac{a^{0,5}-2}{a+2a^{0,5}+4} \right) \cdot \dfrac{a^2-8a^{0,5}}{a-16} - \dfrac{4a^{0,5}}{a^{0,5}+4} \]
			

		\item Моторная лодка прошла против течения реки \(112\) км и вернулась в пункт отправления, затратив на обратный путь на \(6\) часов меньше. Найдите скорость течения, если скорость лодки в неподвижной воде равна \(11\) км/ч. Ответ дайте в км/ч.
	\end{listofex}
\end{class}
%END_FOLD

%BEGIN_FOLD % ====>>_____ Занятие 4 _____<<====
\begin{class}[number=4]
	\begin{listofex}
		\item Упростите выражение:
		\begin{tasks}(2)
			\task \( (a^{\tfrac{1}{2}})^3 \)
			\task \( (x^{\tfrac{2}{3}})^6 \)
			\task \( (b^{\tfrac{2}{3}})^{\tfrac{5}{6}} \)
			\task \( (y^{\tfrac{4}{7}})^{\tfrac{21}{20}} \)
			\task \( (ab^{\tfrac{1}{2}})^{-2} \)
			\task \( (x^{\tfrac{1}{3}}y)^{-1} \)
			\task \( (3a^{0,5}b^{\tfrac{2}{3}})^{0,5} \)
			\task \( (2x^{0,8}y^{\tfrac{1}{6}})^{0,75} \)
		\end{tasks}
		\item Найдите значение выражения:
		\begin{tasks}(2)
			\task \( 35^{-4,7} \cdot 7^{5,7} : 5^{-3,7} \)
			\task \( \left( \dfrac{2^{\tfrac{1}{3}}\cdot2^{\tfrac{1}{4}}}{\sqrt[12]{2}} \right)^2 \)
			\task \( \dfrac{(2^\tfrac{3}{5}\cdot5^{\tfrac{2}{3}})^{15}}{10^9} \)
			\task \( 0,8^{\tfrac{1}{7}}\cdot5^{\tfrac{2}{7}}\cdot20^{\tfrac{6}{7}} \)
			\task \( \dfrac{49^{5,2}}{7^{8,4}} \)
			\task \( 4^8\cdot11^{10}:44^8 \)
		\end{tasks}
		\item Вычислите:
		\begin{tasks}(1)
			\task \( ( 9^{-\tfrac{1}{4}}+(2\sqrt{2})^{-\tfrac{2}{3}} ) \cdot ( \sqrt[4]{9^{-1}} - (2\sqrt{2})^{-\tfrac{2}{3}} ) \)
			\task \( ( (5\sqrt{5})^{-\tfrac{2}{3}} + \sqrt[4]{81^{-1}} ) \cdot ( (5\sqrt{5})^{-\tfrac{2}{3}} - 81^{-\tfrac{1}{4}} ) \)
		\end{tasks}
		\item Решите уравнения:
		\begin{tasks}(2)
			\task \( x^2 = 25 \)
			\task \( (4x)^{0,5} = 16 \)
			\task \( x^{\tfrac{1}{3}} = 17 \)
			\task \( 0,25x^{\tfrac{1}{3}} = \sqrt[3]{\dfrac{1}{8}} \)
			\task \( \left( \dfrac{1}{9} \right)^x=3 \)
			\task \( \left( \dfrac{2}{3} \right)^x=1,5 \)
			\task \( 7 \cdot 5^x = 5 \cdot 7^x \)
			\task \( \dfrac{5^{x+25}}{19}=\dfrac{5}{19^{x+25}} \)
			\task \( \dfrac{3^{x^2}-3}{x-1}=0 \)
		\end{tasks}
		\item Найдите значение выражения: \( \dfrac{6n^{\tfrac{1}{3}}}{n^{\tfrac{1}{12}} \cdot n^ {\tfrac{1}{4}}} \) при \( n>0 \).
		\item Найдите значение выражения: \( \dfrac{(\sqrt{3}a)^2 \cdot \sqrt[5]{a^3}}{a^{2,6}} \) при \( a>0 \).
		\item Найдите значение выражения: \( \dfrac{\sqrt[9]{a} \cdot \sqrt[18]{a}}{a\sqrt[6]{a}} \) при \( a=1,25 \).
	\end{listofex}
\end{class}
%END_FOLD

%BEGIN_FOLD % ====>>_ Домашняя работа 2 _<<====
\begin{homework}[number=2]
	\begin{listofex}
		\item Упростите выражение:
		\begin{tasks}(3)
			\task \( \left(a^{ \tfrac{3}{5}} \right)^5 \)
			\task \( \left(b^{\tfrac{1}{3}} \right)^3 \)
			\task \( \left(b^{\tfrac{2}{5}}\right)^{2,5} \)
			\task \( \left(y^{ \tfrac{2}{7}}\right)^{\tfrac{21}{5}} \)
			\task \( ((xy)^{-0,5})^{-2} \)
			\task \( \left(x^{ \tfrac{1}{3}}y^{-1,5}\right)^{-1} \)
			\task \( \left(a^{2}b^{\tfrac{5}{7}}\right)^{-3,5} \)
			\task \( \left(5x^{0,8}y^{\tfrac{1}{4}}\right)^{2} \)
		\end{tasks}
		\item Решите уравнения:
		\begin{tasks}(2)
			\task \( (2x-1)^{1,4}=9^{2,8} \)
			\task \( \dfrac{1}{8}x^{0,5} = \sqrt{0,5} \)
			\task \( \left( \dfrac{25}{36} \right)^x=\mfrac{1}{1}{5} \)
			\task \( \left( \dfrac{1}{5} \right)^x=0,04 \)
			\task \( 3 \cdot 4^x = 4 \cdot 3^x \)
		\end{tasks}
		\item Найдите значение выражения:
		\begin{tasks}(2)
			\task \( (5^{12})^3 : 5^{37} \)
			\task \( 3^{\sqrt{5}+10}\cdot3^{-5-\sqrt{5}} \)
			\task \( \dfrac{\sqrt[15]{5}\cdot5\cdot\sqrt[10]{5}}{\sqrt[6]{5}} \)
		\end{tasks}
	\end{listofex}
\end{homework}
%END_FOLD

%BEGIN_FOLD % ====>>_____ Занятие 5 _____<<====
\begin{class}[number=5]
	\begin{listofex}
		\item Решите уравнения:
		\begin{tasks}(3)
			\task \( 3^{4x-1}=\dfrac{1}{9} \)
			\task \( 5^{2x-4}=25^{2-x} \)
			\task \( \left(\dfrac{1}{64}\right)^{-x}=\sqrt{\dfrac{1}{8}} \)
			\task \( 0,2^{x-2}=5^{2-x} \)
			\task \( 0,5^{x^2}\cdot 4^{x+1}=\dfrac{1}{64} \)
			\task \( 4^x-2^{2x+1}-8=0 \)
		\end{tasks}
		\item Решите неравенства: %БУКВЫ А ИЗ ШЕСТАКОВ 15 СТР 273
		\begin{tasks}(3)
			\task \( 2^x \le 4 \)
			\task \( 9^x < 27 \)
			\task \( \left( \dfrac{1}{36} \right)^x < 6  \)
			\task \( (0,1)^x \le 100 \)
			\task \( 49^{x^2}>7^{x+1} \)
			\task \( 15^x<25\cdot 3^x \)
			\task \( \sqrt[6]{7^{x^2-9}} \le 7^{x-3} \)
			\task \( \sqrt[4]{5^{x^2-4}} \le 5^{x-2} \)
		\end{tasks}
		\item Решите системы неравенств:
		\begin{tasks}(3)
			\task \( \begin{cases} 3^{x-2}<81 \\ \dfrac{1}{49} \le 7^{x+2} \end{cases} \)
			\task \( \begin{cases} 2^{x-3}<16 \\ \dfrac{1}{36} \le 6^{x+3} \end{cases} \)
			\task \( \begin{cases} 15^{x-7}>3 \cdot 5^{x-7} \\ 15^{x-17}<5 \cdot 3^{x-17} \end{cases} \)
		\end{tasks}
		%\item Упростите выражение: \( \dfrac{2b^{0,5}}{b^{0,5}-3^{1,5}} - \left( \dfrac{b^{\tfrac{1}{3}}+3}{b^{0,5}-3^{\tfrac{3}{2}}} \right)^2 : \left( \dfrac{1}{b^{\tfrac{1}{3}-3}} + \dfrac{3b^{\tfrac{1}{3}}}{b-27} \right) \)
	\end{listofex}
	\end{class}
%END_FOLD
	
%BEGIN_FOLD % ====>>_____ Занятие 6 _____<<====
\begin{class}[number=6]
	\begin{listofex}
		\item Решите уравнения:
		\begin{tasks}(3)
			\task \( 0,2^x=\dfrac{1}{5} \)
			\task \( 5^x-5^{x-1}=100 \)
			\task \( 3^{2x+1}-9^x=18 \)
			\task \( 4^{x+1}+4^{x+2}=40 \)
			\task \( 9^{x+1}+3^{2x+4}=30 \)
			\task \( 9\cdot 5^x-25 \cdot 3^x=0 \)
		\end{tasks}
		\item Решите неравенства: %БУКВЫ Б ИЗ ШЕСТАКОВ 15 СТР 273
		\begin{tasks}(3)
			\task \( 6^x<9 \cdot 2^x \)
			\task \( 0,04 \le 5^x \)
			\task \( 6^{8-5x} \le 5^{5x-8} \)
			\task \( 7^{4-25x^2} > 3^{25x^2-4} \)
			\task \( 4^{3x-2} + 4^{3x-1} \le 80 \)
			\task \( 16 \cdot 5^{x-8} \le 25 \cdot 4^{x-8} \)
			\task \( 5^{25-4x^2} > 2^{4x^2-25} \)
			\task \( 3^{4x-7} \le 4^{7-4x}  \)
			\task \( 4 \cdot 9^{x-8} \le 9 \cdot 4^{x-8} \)
		\end{tasks}
		\item Решите системы неравенств:
		\begin{tasks}(2)
			\task \( \begin{cases} 13^x<169 \\ 169^x > 13 \end{cases} \)
			\task \( \begin{cases} 15^{x-5}>11^{x-5} \\ 6^{x-15}<7^{15-x} \end{cases} \)
			\task \( \begin{cases} 2 \cdot 7^{x-5} \le 7 \cdot 2^{x-5} \\ 8 \cdot 5^{x-5} \le 5 \cdot 8^{x-5} \end{cases} \)
			\task \( \begin{cases} 3^x \cdot 8^x > 24 \\ 6^x \le 27 \cdot 2^x \end{cases} \)
		\end{tasks}
		%\item Упростите выражение: \( \dfrac{2b^{0,5}}{b^{0,5}-3^{1,5}} - \left( \dfrac{b^{\tfrac{1}{3}}+3}{b^{0,5}-3^{\tfrac{3}{2}}} \right)^2 : \left( \dfrac{1}{b^{\tfrac{1}{3}-3}} + \dfrac{3b^{\tfrac{1}{3}}}{b-27} \right) \)
	\end{listofex}
\end{class}
%END_FOLD
	
%BEGIN_FOLD % ====>>_ Домашняя работа 3 _<<====
\begin{homework}[number=3]
	\begin{listofex}
		\item Решите уравнения:
		\begin{tasks}(2)
			\task \( 4^{x+1}-2^{2x-2}=60 \)
			\task \( 3^{x-1}-3^{x-2}=18 \)
			\task \( 27 \cdot 5^x-125 \cdot 3^x=0 \)
			\task \( 27 \cdot 4^x - 8 \cdot 9^x=0 \)
		\end{tasks}
		\item Решите неравенства: %БУКВЫ Б ИЗ ШЕСТАКОВ 15 СТР 273
		\begin{tasks}(2)
			\task \( 9^x \cdot 3^{x^2} > 27 \)
			\task \( 5^x < 2 \cdot 25^x \)
			\task \( 14^{x-16} \le 15^{x-16} \)
			\task \( 64 \le \dfrac{1}{4^{2x+9}} \)
		\end{tasks}
		\item Решите системы неравенств:
		\begin{tasks}(2)
			\task \( \begin{cases} 2^x \cdot 4 ^{x^2} \le 8 \\ 5^{x^2-3} > 0,04 \end{cases} \)
			\task \( \begin{cases} 3 \cdot 26^x \le 2 \cdot 39^x \\ 11 \cdot 15^x > 5 \cdot 33^x \end{cases} \)
		\end{tasks}
	\end{listofex}
\end{homework}
%END_FOLD
	
%BEGIN_FOLD % ====>>_____ Занятие 7 _____<<====
\begin{class}[number=7]
	\title{Подготовка к проверочной}
	\begin{listofex}
		\item Вычислите:
		\begin{tasks}(2)
			\task \( \sqrt[3]{343} \)
			\task \( \sqrt[4]{128} \)
			\task \( \sqrt[5]{1024} \)
			\task \( \sqrt[4]{81} \)
			\task \( \sqrt[4]{16} \cdot \sqrt[4]{256} \)
			\task \( \sqrt[3]{4} \cdot \sqrt[3]{16} \)
			\task \( \sqrt[5]{-5} \cdot \sqrt[5]{5} \cdot \sqrt[5]{125} \)
		\end{tasks}
		\item Вычислите:
		\begin{tasks}(3)
			\task \( \sqrt[4]{10000 \cdot 4096} \)
			\task \( \sqrt[3]{729 \cdot 0,125} \)
			\task \( \sqrt[4]{625 \cdot 0,0001 \cdot 16} \)
		\end{tasks}
		\item Найдите значение выражения:
		\begin{tasks}(4)
			\task \( \sqrt[3]{a^3b^6} \)
			\task \( \sqrt[4]{16x^8z^{12}} \)
			\task \( \sqrt[3]{-8a^{15}b^3} \)
			\task \( \sqrt[4]{1296a^8b^8} \)
		\end{tasks}
		\item Найдите значение выражения:
		\begin{tasks}(2)
			\task \( (49^6)^3:(7^7)^5 \)
			\task \( \dfrac{0,5^{\sqrt{10}-1}}{2^{-\sqrt{10}}} \)
			\task \( 3^{0,36} \cdot 9^{0,32} \)
			\task \( 15^{\tfrac{4}{9}} \cdot 225^{\tfrac{5}{18}} \)
			\task \( \dfrac{4^{2,5}}{16^{2,25}} \)
			\task \( \dfrac{2^{3,5}\cdot3^{5,5}}{6^{4,55}} \)
			\task \( 5^{3\sqrt{7}-1}\cdot 5^{1-\sqrt{7}}:5^{2\sqrt{7}-1} \)
		\end{tasks}
		\item Решите уравнения:
		\begin{tasks}(2)
			\task \( \left( \dfrac{1}{9} \right)^x=3  \)
			\task \( \left( \dfrac{1}{8} \right)^x=16 \)
			\task \( \left( \dfrac{1}{2} \right)^x=-8 \)
			\task \( 5^{x}-5^{x-1}=00 \)
			\task \( 3^{2x+1}-9^x=18 \)
			\task \( 4^{x+1}+4^{x+2}=40 \)
		\end{tasks}
		\item Решите неравества:
		\begin{tasks}(2)
			\task \( 17^{x^2-144}<7^{x^2-144} \)
			\task \( 8^x \cdot 2^{x^2} \le 16 \)
			\task \( 6^x \cdot 5^x > 900 \)
			\task \( (2,5)^x > 0,16 \)
			\task \( \left( \dfrac{1}{11} \right)^{6+5x-3x^2} > 121 \)
			\task \( \sqrt[6]{7^{x^2-9}} \le 7^{x-3} \)
		\end{tasks}
		\item Решите системы неравеств:
		\begin{tasks}(2)
			\task \( \begin{cases} 4^{x-47} \le 64 \\ 7^{x+47} > \dfrac{1}{49} \end{cases} \)
			\task \( \begin{cases} \sqrt[14]{4^x}>14 \\ \sqrt[16]{16^x}<16 \end{cases} \)
		\end{tasks}
		
		\item Моторная лодка прошла против течения реки \(255\) км и вернулась в пункт отправления, затратив на обратный путь на \(2\) часа меньше. Найдите скорость лодки в неподвижной воде, если скорость течения равна \(1\) км/ч. Ответ дайте в км/ч.
	\end{listofex}
\end{class}
%END_FOLD
	
%BEGIN_FOLD % ====>>_ Проверочная работа _<<====
\begin{exam}
	\begin{listofex}
		\item Вычислите:
		\begin{tasks}(2)
			\task \( \sqrt[3]{512} \)
			\task \( \sqrt[5]{-32} \)
			\task \( \sqrt[3]{36} \cdot \sqrt[6]{36} \)
			\task \( -\sqrt[3]{-27} \cdot \sqrt[3]{8} \cdot \sqrt[3]{-125} \)
			\task \( \sqrt[3]{-512 \cdot \dfrac{125}{64}} \)
			\task \( \sqrt[4]{10000 \cdot 25^2} \)
			\task \( \dfrac{6^{\sqrt{3}}\cdot7^{\sqrt{3}}}{42^{\sqrt{3}-1}} \)
			\task \( 3^{-0,7}\cdot 3^{1,3} \cdot 9^{0,7}\)
		\end{tasks}
			\item Найдите значение выражения:
		\begin{tasks}(4)
			\task \( \sqrt[3]{a^6b^9} \)
			\task \( \sqrt[4]{81x^{24}z^{16}} \)
			\task \( \sqrt[3]{27a^{-15} \cdot 8b^3} \)
			\task \( \sqrt[4]{1296a^8 \cdot 16 b^8} \)
		\end{tasks}
		\item Решите уравнения:
		\begin{tasks}(2)
			\task \( \left( \dfrac{1}{64} \right)^x=2  \)
			\task \( \left( \dfrac{2}{3} \right)^x=1,5 \)
			\task \( 0,04^x=0,2 \)
			\task \( 9^{x+1}+3^{2x+4}=30 \)
			\task \( 4^{x+1}-2^{2x-2}=60 \)
			%\task \( 3^{x-1}-3^{x-2}=18 \)
		\end{tasks}
		\item Решите неравества:
		\begin{tasks}(2)
			\task \( 1,25^x > 0,8 \)
			\task \( 2^{2x^2-3x} \le 0,5 \)
			\task \( 5^x \cdot 4^x > 400 \)
			\task \( \sqrt[4]{5^{x^2-4}}\le 5^{x-2} \)
		\end{tasks}
		\item Решите системы неравеств:
		\begin{tasks}(2)
			\task \( \begin{cases} 15^{x-7}> 3 \cdot 5^{x-7} \\ 15^{x-17} < 5 \cdot 3^{x-17} \end{cases} \)
			\task \( \begin{cases} 14^{x-6} > 2 \cdot 7^{x-6} \\ 14^{x-16} < 7 \cdot 2^{x-16} \end{cases} \)
		\end{tasks}
		
		%\item Моторная лодка в \(10:00\) вышла из пункта \(A\) в пункт \(B\), расположенный в \(30\) км от \(A\). Пробыв в пункте В \(2\) часа \(30\) минут, лодка отправилась назад и вернулась в пункт А в \(18:00\) того же дня. Определите (в км/ч) собственную скорость лодки, если известно, что скорость течения реки \(1\) км/ч.
	\end{listofex}
\end{exam}
%END_FOLD