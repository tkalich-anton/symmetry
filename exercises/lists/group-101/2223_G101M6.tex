%
%===============>>  ГРУППА 10-1 МОДУЛЬ 6  <<=============
%
\setmodule{6}

%BEGIN_FOLD % ====>>_____ Занятие 1 _____<<====
\begin{class}[number=1]
	\begin{listofex}
		\item Вычислите:
		\begin{tasks}(4)
			\task \( \sqrt[4]{16} \)
			\task \( \sqrt[5]{243} \)
			\task \( \sqrt[3]{64} \)
			\task \( \sqrt[3]{-27} \)
			\task \( \sqrt[4]{64} \)
			\task \( \sqrt[5]{-64} \)
			\task \( \sqrt[3]{0,125} \)
			\task \( \sqrt[7]{-128} \)
		\end{tasks}
		\item Найдите значение выражения:
		\begin{tasks}(4)
			\task \( \sqrt[3]{a^6b^9} \)
			\task \( \sqrt[3]{64x^3z^6} \)
			\task \( \sqrt[3]{a^{12}b^3} \)
			\task \( \sqrt[3]{1000a^6b^6} \)
		\end{tasks}
		\item Найдите значение выражения:
		\begin{tasks}(2)
			\task \( \sqrt[3]{ab^2} \cdot \sqrt[3]{a^2b} \)
			\task \( \sqrt[3]{a^2b^4} \cdot \sqrt[3]{ab^2} \)
		\end{tasks}
		\item Вычислите:
		\begin{tasks}(3)
			\task \( \sqrt[3]{27 \cdot 0,001 \cdot 125} \)
			\task \( \sqrt[4]{625 \cdot 0,0016 \cdot 256} \)
			\task \( \sqrt[5]{243 \cdot 0,00001 \cdot 0,00032} \)
		\end{tasks}
		\item Вынесите множитель из под знака корня:
		\begin{tasks}(4)
			\task \( \sqrt[3]{40} \)
			\task \( \sqrt[5]{-64} \)
			\task \( \sqrt[5]{-96} \)
			\task \( \sqrt[3]{54} \)
		\end{tasks}
		\item Вычислите:
		\begin{tasks}(4)
			\task \( \sqrt[3]{125 \cdot 27} \)
			\task \( \sqrt[4]{16 \cdot 625} \)
			\task \( \sqrt[4]{81 \cdot 16} \)
			\task \( \sqrt[4]{5 \cdot 125} \)
		\end{tasks}
		\item Решите уравнения:
		\begin{tasks}(2)
			\task \( \sqrt[3]{15-2x} = 3 \)
			\task \( \sqrt[4]{2x-8}=3 \)
			\task \( \sqrt[3]{4x^2-5x-1}=2 \)
			\task \( \sqrt[]{-72-17x}=-x \)
		\end{tasks}
		\item Из пункта \(A\) в пункт \(B\) одновременно выехали два автомобиля. Первый проехал с постоянной скоростью весь путь. Второй проехал первую половину пути со скоростью, меньшей скорости первого на \(13\) км/ч, а вторую половину пути  --- со скоростью \(78\) км/ч, в результате чего прибыл в пункт \(B\) одновременно с первым автомобилем. Найдите скорость первого автомобиля, если известно, что она больше \(48\) км/ч. Ответ дайте в км/ч.
	\end{listofex}
\end{class}
%END_FOLD

%BEGIN_FOLD % ====>>_____ Занятие 2 _____<<====
\begin{class}[number=2]
	\begin{listofex}
		\item Вычислите:
		\begin{tasks}(4)
			\task \( \sqrt[3]{512} \)
			\task \( \sqrt[7]{128} \)
			\task \( \sqrt[5]{1024} \)
			\task \( \sqrt[4]{81} \)
		\end{tasks}
		\item Вычислите:
		\begin{tasks}(3)
			\task \( \sqrt[3]{1000 \cdot 512} \)
			\task \( \sqrt[3]{343 \cdot 0,125} \)
			\task \( \sqrt[4]{1296 \cdot 0,0001 \cdot 1024} \)
		\end{tasks}
		\item Извлеките корень из дроби:
		\begin{tasks}(3)
			\task \( \sqrt[3]{\dfrac{1}{64}} \)
			\task \( \sqrt[4]{\dfrac{81}{16}} \)
			\task \( \sqrt[5]{\dfrac{32}{243}} \)
		\end{tasks}
		\item Вынесите множитель из под знака корня:
		\begin{tasks}(4)
			\task \( \sqrt[3]{\dfrac{3}{8}} \)
			\task \( \sqrt[3]{\dfrac{27}{4}} \)
			\task \( \sqrt[3]{-\dfrac{250}{16}} \)
			\task \( \sqrt[3]{-\dfrac{64}{7}} \)
		\end{tasks}
		\item Вычислите:
		\begin{tasks}(4)
			\task \( \sqrt[]{(-2)^2} \)
			\task \( \sqrt[]{(-5)^4} \)
			\task \( \sqrt[]{(\sqrt{2}-1)^2} \)
			\task \( \sqrt[]{(\sqrt{2}-2)^2} \)
		\end{tasks}
		\item Упростите выражение:
		\begin{tasks}(2)
			\task \( \sqrt[4]{81 \cdot (4-\sqrt{17})^4} \)
			\task \( \sqrt[3]{0,001} - \sqrt[6]{0,000064} \)
		\end{tasks}
	\end{listofex}
\end{class}
%END_FOLD

%BEGIN_FOLD % ====>>_ Домашняя работа 1 _<<====
\begin{homework}[number=1]
	\begin{listofex}
		\item Вычислите:
		\begin{tasks}(4)
			\task \( \sqrt[3]{343} \)
			\task \( \sqrt[4]{625} \)
			\task \( \sqrt[5]{32} \)
			\task \( \sqrt[4]{1296} \)
		\end{tasks}
		\item Извлеките корень из дроби:
		\begin{tasks}(3)
			\task \( \sqrt[3]{\dfrac{125}{216}} \)
			\task \( \sqrt[4]{\dfrac{625}{16}} \)
			\task \( \sqrt[3]{\dfrac{512}{27}} \)
		\end{tasks}
		\item Вычислите:
		\begin{tasks}(2)
			\task \( \sqrt[]{(\sqrt{2}-\sqrt{3})^2} \)
			\task \( \sqrt[]{(\sqrt{5}-\sqrt{7})^2} \)
		\end{tasks}
		\item Вычислите:
		\begin{tasks}(2)
			\task \( (\sqrt[3]{-2})^3 + (\sqrt[5]{8})^5 \)
			\task \( \sqrt[5]{-1} + \sqrt[3]{-8} \)
			\task \( \sqrt[5]{8} \cdot \sqrt[5]{4} \)
			\task \( \sqrt[3]{4} \cdot \sqrt[3]{8} \cdot \sqrt[3]{-2} \)
		\end{tasks}
	\end{listofex}
\end{homework}
%END_FOLD

%BEGIN_FOLD % ====>>_____ Занятие 3 _____<<====
\begin{class}[number=3]
	\begin{listofex}
		\item Занятие 3 
	\end{listofex}
\end{class}
%END_FOLD

%BEGIN_FOLD % ====>>_____ Занятие 4 _____<<====
\begin{class}[number=4]
	\begin{listofex}
		\item Занятие 4
	\end{listofex}
\end{class}
%END_FOLD

%BEGIN_FOLD % ====>>_ Домашняя работа 2 _<<====
\begin{homework}[number=2]
	\begin{listofex}
		\item Домашняя работа 2
	\end{listofex}
\end{homework}
%END_FOLD

%BEGIN_FOLD % ====>>_____ Занятие 5 _____<<====
\begin{class}[number=5]
	\begin{listofex}
		\item Занятие 5
	\end{listofex}
	\end{class}
%END_FOLD
	
%BEGIN_FOLD % ====>>_____ Занятие 6 _____<<====
\begin{class}[number=6]
	\begin{listofex}
		\item Занятие 6
	\end{listofex}
\end{class}
%END_FOLD
	
%BEGIN_FOLD % ====>>_ Домашняя работа 3 _<<====
\begin{homework}[number=3]
	\begin{listofex}
		\item Домашняя работа 3
	\end{listofex}
\end{homework}
%END_FOLD
	
%BEGIN_FOLD % ====>>_____ Занятие 7 _____<<====
\begin{class}[number=7]
	\title{Подготовка к проверочной}
	\begin{listofex}
		\item Занятие 7
	\end{listofex}
\end{class}
%END_FOLD
	
%BEGIN_FOLD % ====>>_ Проверочная работа _<<====
\begin{exam}
	\begin{listofex}
		\item Проверочная
	\end{listofex}
\end{exam}
%END_FOLD