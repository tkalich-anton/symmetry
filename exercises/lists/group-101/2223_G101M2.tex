%Группа 101-1 Модуль 2
\title{Занятие №1}
\begin{listofex}
	\item \exercise{2350}
	\item \exercise{2095}
	\item \exercise{2390}
	\item \exercise{2380}
	\item \exercise{2385}
	\item \exercise{2381}
	\item \exercise{2393}
	\item \exercise{2559}
	\item \exercise{2040}
	\item \exercise{2424}
\end{listofex}
\newpage
\title{Занятие №2}
\begin{listofex}
	\item Упростить выражение:
	\begin{enumcols}[itemcolumns=2]
		\item \exercise{750}
		\item \exercise{1500}
	\end{enumcols}
	\item Докажите, что если медиана равна половине стороны, к которой она проведена, то треугольник прямоугольный.
	\item \exercise{2415}
	\item Докажите, что если треугольник вписан в окружность и одна из его сторон является диаметром этой окружности, то такой треугольник является прямоугольным.
	\item Докажите обратное, что если треугольник прямоугольный и вписан в окружность, то гипотенуза будет являться диаметром окружности.
	\item \exercise{2455}
	\item \exercise{2418}
	\item \exercise{2422}
\end{listofex}
\newpage
\title{Домашняя работа №1}
	\begin{listofex}
	\item Упростить выражение:
	\begin{enumcols}[itemcolumns=2]
		\item \exercise{748}
		\item \exercise{1500}
	\end{enumcols}
	\item Вычислить:
	\begin{enumcols}[itemcolumns=2]
		\item \exercise{1649}
		\item \exercise{1682}
	\end{enumcols}
	\item Докажите, что в равных треугольниках соответствующие биссектрисы равны.
	\item \exercise{2423}
	\item \exercise{2456}
	\item \exercise{2412}
\end{listofex}
\newpage
\title{Занятие №3}
\begin{listofex}
	\item Докажите следующие свойства окружности:
	\begin{enumcols}[itemcolumns=1]
		\item диаметр, перпендикулярный хорде, делит ее пополам;
		\item диаметр, проходящий через середину хорды, не являющейся диаметром, перпендикулярен этой хорде;
		\item хорды, удаленные от центра окружности на равные расстояния, равны.
	\end{enumcols}
	\item \exercise{2437}
	\item \exercise{2439}
	\item \exercise{2442}
	\item \exercise{2444}
	\item \exercise{2445}
	\item \exercise{2422}
	\item \exercise{2420}
	\item \exercise{2424}
\end{listofex}
\newpage
\title{Занятие №4}
	\begin{listofex}
		\item Внутренние углы треугольника \( ABC \) относятся как \( 10:5:3 \). Найдите внутренние и внешние углы треугольника \( ABC \) и вычислите разницу самого наибольшего и наименьшего внешних углов. \answer{ внутренние:\( 100;\;50;\;30 \), внешние: \( 80;\;130;\;100; \), разница: \( 50 \) }
	\item В треугольнике \( ABC \) углы \( B \) и \( C \) равны \( 30 \) и \( 40 \) соответственно. Сторону \( AB \) продлили за вершину \( A \) и из это вершины провели высоту и биссектрису внешнего угла. Найдите угол между высотой и биссектрисой. \answer{ \( 85 \) }
	\item Две параллельные прямые пересечены третьей. Найдите угол между биссектрисами внутренних односторонних углов.
	\item \exercise{2438}
	\item \exercise{2441}
	\item \exercise{2354}
	\item \exercise{2514}
	\item Решить уравнение:
	\begin{enumcols}[itemcolumns=2]
		\item \exercise{1015}
		\item \exercise{1036}
	\end{enumcols}
\end{listofex}
\newpage
\title{Домашняя работа №2}
\begin{listofex}
	\item \exercise{2436}
	\item \exercise{2440}
	\item \exercise{2454}
	\item \exercise{2457}
	\item \exercise{2459}
	\item \exercise{2412}
	\item В треугольнике \( ABC \) угол \( \angle B = 80 \). Найдите угол между высотами проведенными из двух других углов. \answer{ \( 100 \) }
	\item Решить уравнение:
	\begin{enumcols}[itemcolumns=2]
		\item \exercise{3391}
		\item \exercise{1034}
	\end{enumcols}
\end{listofex}
\newpage
\title{Занятие №5}
\begin{listofex}
	\item \exercise{2472}
	\item \exercise{2473}
	\item \exercise{2474}
	\item \exercise{2476}
	\item \exercise{2483}
	\item \exercise{2493}
	\item \exercise{1608}
	\item \exercise{3664}
\end{listofex}
\newpage
\title{Занятие №6}
\begin{listofex}
	\item \exercise{2477}
	\item \exercise{2481}
	\item \exercise{2484}
	\item \exercise{2486}
	\item \exercise{2479}
	\item \exercise{2500}
	\item Найти значение выражения \( 47a-7b+38 \),\quad если \( \dfrac{2a-7b+5}{7a-2b+5}=7 \)
\end{listofex}
\newpage
\title{Домашняя работа №3}
\begin{listofex}
	\item \exercise{2475}
	\item \exercise{2478}
	\item \exercise{2485}
	\item \exercise{2480}
\end{listofex}
\newpage
\title{Подготовка к проверочной работе}
\begin{listofex}
	\item Чему равен угол между биссектрисами двух смежных углов?
	\item Чему равен угол между биссектрисами двух внутренних односторонних углов при параллельных прямых? Докажите это.
	\item Сформулируйте и докажите теорему о внешнем угле треугольника.
	\item Докажите, что биссектриса внешнего угла при вершине равнобедренного треугольника, параллельна основанию.
	\item Докажите, что если в треугольнике один угол равен сумме двух других, то такое треугольник прямоугольный.
	\item Докажите, что если медиана равна половине стороны, к которой она проведена, то такой треугольник прямоугольный.
	\item Докажите, что если треугольник вписан в окружность и одна из его сторон является диаметром этой окружности, то такой треугольник прямоугольный.
	\item Сформулируйте теорему об угле в \( 30\degree \) в прямоугольном треугольнике. Сформулируйте обратную теорему.
	\item Сформулируйте теорему о диаметре, перпендикулярном хорде.
	\item Сформулируйте теорему о диаметре, проходящем через середину хорды.
	\item Где лежит центр вписанной в треугольник окружности?
	\item Сформулируйте теорему о двух касательных, проведенных из одной точки к окружности.
	\item Чему равна сумма всех внешних углов треугольника?
	\item \exercise{2472}
	\item В треугольнике \( ABC \) обе стороны \( AB \) и \( BC \) равны \( 15 \). Чему равна сторона \( AC \), если \( \angle BAC = 60 \degree \)?
	\item В треугольнике \( ABC \) известно, что \( \angle A = 50 \) и \( \angle B = 80 \). Найдите сторону \( BC \), если \( AC = 10 \) и \( P_{ABC}=40 \).
	\item Угол между биссектрисами двух углов треугольника равен \( 120\degree \). Чему равен третий угол треугольника?
	\item Угол треугольника равен \( 50\degree \). Найдите угол между высотами, проведенными из двух других углов.
	\item В треугольнике \( ABC \) угол \( \angle B=60\degree \). Найдите угол между биссектрисами двух других внешних углов.
	\item \exercise{2456}
	\item \exercise{2459}
	\item \exercise{2475}
	\item \exercise{2478}
	\item \exercise{2485}
	\item \exercise{2480}
	\item В треугольнике \( ABC \) медиана \( AM \) продолжена за точку \( M \) на расстояние, равное \( AM \). Найдите расстояние от полученной точки до вершин \( B  \) и \( C\), если \( AB = 5\), \( AC = 12\).
	\item \exercise{2441}
	\item \exercise{2514}
\end{listofex}
\newpage
\title{Проверочная работа}
\title{Вариант 1}
\begin{listofex}
	\item
	\begin{enumcols}[itemcolumns=1]
		\item Чему равен угол между биссектрисами двух смежных углов?
		\item Сформулируйте и докажите теорему о внешнем угле треугольника.
		\item Докажите, что биссектриса внешнего угла при вершине равнобедренного треугольника, параллельна основанию.
		\item Докажите, что если медиана равна половине стороны, к которой она проведена, то такой треугольник прямоугольный.
		\item Докажите, что если треугольник вписан в окружность и одна из его сторон является диаметром этой окружности, то такой треугольник прямоугольный.
		\item Сформулируйте теорему об угле в \( 30\degree \) в прямоугольном треугольнике. Сформулируйте обратную теорему.
		\item Сформулируйте теорему о диаметре, проходящем через середину хорды.
		\item Где лежит центр вписанной в треугольник окружности?
	\end{enumcols}
	\item В треугольнике \( ABC \) обе стороны \( AB \) и \( BC \) равны \( 15 \). Чему равна сторона \( AC \), если \( \angle BAC = 60 \degree \)?
	\item В треугольнике \( ABC \) известно, что \( \angle A = 50 \) и \( \angle B = 80 \). Найдите сторону \( BC \), если \( AC = 16 \) и \( P_{ABC}=40 \).
	\item Угол между биссектрисами двух углов треугольника равен \( 100\degree \). Чему равен третий угол треугольника?
	\item \exercise{2456}
	\item \exercise{2478}
	\item В треугольнике \( ABC \) медиана \( AM \) продолжена за точку \( M \) на расстояние, равное \( AM \). Найдите расстояние от полученной точки до вершин \( B  \) и \( C\), если \( AB = 5\), \( AC = 12\).
	\item \exercise{2485}
\end{listofex}
\newpage
\title{Проверочная работа}
\title{Вариант 2}
\begin{listofex}
	\item
	\begin{enumcols}[itemcolumns=1]
		\item Чему равен угол между биссектрисами двух внутренних односторонних углов при параллельных прямых?
		\item Сформулируйте и докажите теорему о внешнем угле треугольника.
		\item Докажите, что если в треугольнике один угол равен сумме двух других, то такое треугольник прямоугольный.
		\item Докажите, что если треугольник вписан в окружность и одна из его сторон является диаметром этой окружности, то такой треугольник прямоугольный.
		\item Сформулируйте теорему об угле в \( 30\degree \) в прямоугольном треугольнике. Сформулируйте обратную теорему.
		\item Сформулируйте теорему о диаметре, перпендикулярном хорде.
		\item Сформулируйте теорему о двух касательных, проведенных из одной точки к окружности.
	\end{enumcols}
	\item В треугольнике \( ABC \) обе стороны \( AB \) и \( BC \) равны \( 30 \). Чему равна сторона \( AC \), если \( \angle BAC = 60 \degree \)?
	\item В треугольнике \( ABC \) известно, что \( \angle A = 50 \) и \( \angle B = 80 \). Найдите сторону \( BC \), если \( AC = 20 \) и \( P_{ABC}=50 \).
	\item Угол треугольника равен \( 80\degree \). Найдите угол между высотами, проведенными из двух других углов.
	\item \exercise{2456}
	\item \exercise{2478}
	\item В треугольнике \( ABC \) медиана \( AM \) продолжена за точку \( M \) на расстояние, равное \( AM \). Найдите расстояние от полученной точки до вершин \( B  \) и \( C\), если \( AB = 5\), \( AC = 12\).
	\item \exercise{2514}
\end{listofex}