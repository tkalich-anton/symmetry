%
%===============>>  ГРУППА 10-1 МОДУЛЬ 8  <<=============
%
\setmodule{8}

%BEGIN_FOLD % ====>>_____ Занятие 1 _____<<====
\begin{class}[number=1]
	\begin{listofex}
		\item Вычислить с помощью метода приведения:
		\[ \sin135\degree;\;\cos240\degree;\;\tg150\degree;\;\ctg220\degree;\;\sin(-220\degree);\;\tg840\degree;\;\cos(-240\degree);\;\sin315\degree \]
		\item Вычислить:
		\begin{tasks}(1)
			\task \( \dfrac{\sqrt{3}}{\sin60\degree}+\dfrac{3}{\sin30\degree} \)
			\task \( \dfrac{17\sin155\degree}{\sin25\degree} \)
			\task \( \dfrac{-2\sin105\degree}{\cos15\degree} \)
			\task \( \sin^215\degree-1+\cos^215 \)
			\task \( -\sqrt{27}\cos30\degree-\sqrt{2}\sin45\degree\ctg60\degree\tg60\degree\)
		\end{tasks}
		\item Вычислить с помощью метода приведения:
		\[ \cos\dfrac{5\pi}{4};\;\sin\dfrac{7\pi}{3};\;\sin\dfrac{3\pi}{2};\;\sin\left( -\dfrac{5\pi}{3} \right);\;\cos\dfrac{7\pi}{6};\;\sin\dfrac{13\pi}{4};\;\sin\left( -\dfrac{7\pi}{6}  \right);\;\cos\dfrac{21\pi}{4} \]
		\item Вычислить:
		\begin{tasks}(2)
			\task \( \dfrac{5\cos29\degree}{\sin61\degree} \)
			\task \( -4\sqrt{3}\cos(-750\degree) \)
			\task \( \dfrac{4\cos146\degree}{\cos34\degree} \)
			\task \( 7\tg13\degree\cdot\tg77\degree \)
			\task \( \dfrac{12}{\sin^227\degree+\cos^2207\degree} \)
			\task \( \dfrac{5\sin98\degree}{\sin49\degree\cdot\sin41\degree} \)
			\task \( -50\tg9\degree\cdot\tg81\degree+31 \)
		\end{tasks}
		\item Найти:
		\begin{tasks}(1)
			\task \( 5\sin\alpha \), \quad если \( \cos\alpha=\dfrac{2\sqrt{6}}{5} \) и \( \alpha\in\left( \dfrac{3\pi}{2}; 2\pi \right) \);
			\task \( 3\cos\alpha \), \quad если \( \sin\alpha=-\dfrac{2\sqrt{2}}{3} \) и \( \alpha\in\left( \dfrac{3\pi}{2}; 2\pi \right) \);
			\task \( 24\cos\alpha \), \quad если \( \sin\alpha=-0,2 \);
			\task \( \sin\left( \dfrac{7\pi}{2}-\alpha \right) \), \quad если \( \sin\alpha=0,8 \) и \( \alpha\in\left( \dfrac{\pi}{2}; \pi \right) \);
		\end{tasks}
	\end{listofex}
\end{class}
%END_FOLD

%BEGIN_FOLD % ====>>_____ Занятие 2 _____<<====
\begin{class}[number=2]
	\begin{listofex}
		\item Найти:
		\begin{tasks}(2)
			\task \( \cos^2(-46\degree)+\sin^2(-46\degree) \)
			\task \( \sin^223\degree+9+\cos^223\degree \)
			\task \( \dfrac{-13\sin126\degree}{\sin54\degree} \)
			\task \( \dfrac{2\sin^221\degree+2\cos^221\degree}{4} \)
			\task \( \dfrac{ 12\sin 11 \degree \cdot \cos 11 \degree }{ \sin 22 \degree } \)
			\task \( \dfrac{ 24 (\sin^2 17 \degree-\cos^217 \degree) }{ \cos 34 \degree } \)
			\task \( \dfrac{ 5 \cos 29 \degree }{ \sin 61 \degree } \)
			\task \( 36 \sqrt{6} \tg \dfrac{ \pi }{ 4 } \sin \dfrac{ \pi }{ 4 } \)
			\task \( 4 \sqrt{2} \cos \dfrac{ \pi }{ 4 } \cos \dfrac{ 7\pi }{3  } \)
			\task \( \dfrac{ 8 }{ \sin \left( \dfrac{ -27\pi }{ 4 } \right) \cos \left( \dfrac{ 31\pi }{ 4 } \right) } \)
		\end{tasks}
		\item Вычислите:
		\begin{tasks}
			%\task \( \tg\alpha \), \quad если \( \dfrac{7\sin\alpha+13\cos\alpha}{5\sin\alpha-17\cos\alpha}=3 \).
			\task \( 5\sin\alpha \), если \( \cos\alpha=\dfrac{2\sqrt{6}}{5} \) и \( \alpha\in\left( \dfrac{3\pi}{2}; 2\pi \right) \);
			\task \( 3\cos\alpha \), если \( \sin\alpha=-\dfrac{2\sqrt{2}}{3} \) и \( \alpha\in\left( \dfrac{3\pi}{2}; 2\pi \right) \);
			\task \( 24\cos\alpha \), если \( \sin\alpha=-0,2 \);
			\task \( \sin\left( \dfrac{7\pi}{2}-\alpha \right) \), если \( \sin\alpha=0,8 \) и \( \alpha\in\left( \dfrac{\pi}{2}; \pi \right) \);
			%\task \( \dfrac{3\cos\alpha-4\sin\alpha}{2\sin\alpha-5\cos\alpha} \), если \( \tg\alpha=3 \).
		\end{tasks}
		%n5 cos1
		\item Найдите корни уравнения: \( \cos \dfrac{ \pi(x-7) }{ 3 } = 0,5 \).  В ответ запишите наибольший отрицательный корень.
		%n5 cos3
		\item Найдите корни уравнения: \( \cos \dfrac{ \pi(2x+9) }{ 3 } = \dfrac{ \sqrt{2} }{ 2 } \).  В ответ запишите наибольший отрицательный корень.
		%n5 tg1
		\item Найдите корни уравнения: \( \tg \dfrac{ \pi x }{ 4 } = -1 \).  В ответ запишите наибольший отрицательный корень.
		%n5 tg2
		\item Найдите корни уравнения: \( \tg \dfrac{ \pi (x+2) }{ 3 } = -\sqrt{3} \).  В ответ запишите наибольший отрицательный корень.
		%n5 sin1
		\item Найдите корни уравнения: \( \sin \dfrac{ \pi x }{ 3 } = 0,5 \).  В ответ запишите наименьший положительный корень.
	\end{listofex}
\end{class}
%END_FOLD

%BEGIN_FOLD % ====>>_ Домашняя работа 1 _<<====
\begin{homework}[number=1]
	\begin{listofex}
		\item Вычислить: %ПРОВЕРИТЬ НА ДУБЛИКАТЫ 111L1
		\begin{tasks}(2)
			\task \( \dfrac{5\cos29\degree}{\sin61\degree} \)
			\task \( -4\sqrt{3}\cos(-750\degree) \)
			\task \( \dfrac{4\cos146\degree}{\cos34\degree} \)
			\task \( 7\tg13\degree\cdot\tg77\degree \)
			\task \( \dfrac{12}{\sin^227\degree+\cos^2207\degree} \)
			\task \( \dfrac{5\sin98\degree}{\sin49\degree\cdot\sin41\degree} \)
			\task \( -50\tg9\degree\cdot\tg81\degree+31 \)
		\end{tasks}
		%n5 cos2
		\item Найдите корни уравнения: \( \cos \dfrac{ \pi(x-1) }{ 3 }=\dfrac{  1}{ 2 } \).  В ответ запишите наибольший отрицательный корень.
		%n5 sin2
		\item Найдите корни уравнения: \( \sin \dfrac{ \pi(4x-3) }{ 4 }=1 \).  В ответ запишите наибольший отрицательный корень.
		%n5 tg3
		%\item Найдите корни уравнения: \( \tg \dfrac{ \pi(x-3) }{ 6 }=\dfrac{ 1 }{ \sqrt{3} } \).  В ответ запишите наибольший отрицательный корень.
		\item Найдите: %6 1-6 9
		\begin{tasks}
			\task \( \tg \alpha \), если \( \cos \alpha = \dfrac{ \sqrt{10} }{ 10 } \) и \( \alpha\in\left( \dfrac{ 3\pi }{ 2 };2\pi \right) \);
			\task \( 26 \cos \left( \dfrac{ 3\pi }{ 2 }+\alpha \right) \), если \( \cos \alpha = \dfrac{ 12 }{ 13 } \) и \( \alpha\in\left( \dfrac{ 3\pi }{ 2 };2\pi \right) \);
			%\task \( \dfrac{ 10\sin 6 \alpha }{ 3 \cos 3 \alpha } \), если \( \sin 3 \alpha = 0,6 \).
		\end{tasks}
	\end{listofex}
\end{homework}
%END_FOLD

%BEGIN_FOLD % ====>>_____ Занятие 3 _____<<====
\begin{class}[number=3]
	\begin{listofex}
			\item Вычислите: %111
		\begin{tasks}(2)
			\task \( \dfrac{ 5\sin 74 \degree }{ \cos 37 \degree \cdot \cos 53 \degree } \)
			\task \( \dfrac{ 23 }{ \sin^2 56 \degree + 1 + \sin^2 146 \degree } \)
			\task \( -\dfrac{ 4 }{ \sin^2 27 \degree + \sin^2 117 \degree } \)
			\task \( -\dfrac{ 4\cos 22,5 \degree \sin 22,5 \degree }{ 8 } \)
			\task \( \dfrac{ 50 \sin 19 \degree \cos 19 \degree }{ \sin 38 \degree } \)
			\task \( \dfrac{ \sin \dfrac{ 3\pi }{8 } \cos \dfrac{ 3\pi }{ 8 } }{ 20 } \)
			\task \( -\dfrac{ 7 }{ \sin^2 \left( \dfrac{ \pi }{ 15 } \right) + \cos^2 \left( \dfrac{ \pi }{ 15 } \right) } \)
			\task \( \cos^2\left( -\dfrac{ 3\pi }{ 8 } \right) +\sin^2\left( -\dfrac{ 3\pi }{ 8 } \right) \)
			\task \( \sin^2 (-88\degree) + \cos^2 (-88\degree) -5 \)
			\task \( \dfrac{2\sin^2 21\degree+2\cos^221\degree}{4} \)
		\end{tasks}
		%\item Найдите: %111
		%\begin{tasks}
		%	\task \( \dfrac{ 10\sin 6 \alpha }{ 3 \cos 3 \alpha } \), если \( \sin 3 \alpha = 0,6 \).
		%\end{tasks}
		%n5 cos1
		\item Найдите корни уравнения: \( \cos \dfrac{ \pi(x-7) }{ 3 } = 0,5 \).  В ответ запишите наибольший отрицательный корень.
		%n5 cos3
		\item Найдите корни уравнения: \( \cos \dfrac{ \pi(2x+9) }{ 3 } = \dfrac{ \sqrt{2} }{ 2 } \).  В ответ запишите наибольший отрицательный корень.
		%n5 tg1
		\item Найдите корни уравнения: \( \tg \dfrac{ \pi x }{ 4 } = -1 \).  В ответ запишите наибольший отрицательный корень.
		%n5 tg2
		\item Найдите корни уравнения: \( \tg \dfrac{ \pi (x+2) }{ 3 } = -\sqrt{3} \).  В ответ запишите наибольший отрицательный корень.
		%n5 sin1
		\item Найдите корни уравнения: \( \sin \dfrac{ \pi x }{ 3 } = 0,5 \).  В ответ запишите наименьший положительный корень.
	\end{listofex}
\end{class}
%END_FOLD

%BEGIN_FOLD % ====>>_____ Занятие 4 _____<<====
\begin{class}[number=4]
	\begin{listofex}
		%n5 cos2
		\item Найдите корень уравнения: \( \cos \dfrac{ \pi(x-1) }{ 3 }=\dfrac{ 1 }{ 2 } \). В ответе запишите наибольший отрицательный корень.
		%n5 cos4
		\item Найдите корень уравнения: \( \cos \dfrac{ \pi(x+1) }{ 4 }=\dfrac{ \sqrt{2} }{ 2 } \). В ответе запишите наименьший положительный корень
		%n5 tg3
		\item Решите уравнение \(\tg \dfrac{ \pi(x-3) }{ 6 }= \dfrac{ 1 }{ \sqrt{3} }\). В ответе напишите наибольший отрицательный корень.
		%n5 tg4
		\item Решите уравнение \(\tg \dfrac{ \pi(4x-5) }{4 }= -1\). В ответе напишите наименьший положительный корень.
		%n5 sin2
		\item Решите уравнение \( \sin \dfrac{ \pi(4x-3) }{ 4 }=1 \). В ответе напишите наибольший отрицательный корень.
		%n5 sin3
		\item Решите уравнение \( \sin \dfrac{ \pi(8x+3) }{ 6 }=0,5 \). В ответе напишите наименьший положительный корень.
		%priklad trigon 1
		\item При нормальном падении света с длиной волны \( \lambda=400 \) нм на дифракционную решeтку с периодом \(d\) нм наблюдают серию дифракционных максимумов. При этом угол \(\varphi\)  (отсчитываемый от перпендикуляра к решeтке), под которым наблюдается максимум, и номер максимума \(k\) связаны соотношением \(d \sin \varphi= k\lambda\). Под каким минимальным углом \(\varphi\) (в градусах) можно наблюдать второй максимум на решeтке с периодом, не превосходящим \(1600\) нм?
		%priklad trigon 2
		\item Два тела массой \(m=2\) кг каждое, движутся с одинаковой скоростью  \(v =10\) м/с под углом \(2\alpha\) друг к другу. Энергия (в джоулях), выделяющаяся при их абсолютно неупругом соударении определяется выражением \(Q= m v^2 \sin^2 \alpha \). Под каким наименьшим углом \(2\alpha\) (в градусах) должны двигаться тела, чтобы в результате соударения выделилось не менее \(50\) джоулей?
		\item Найдите площадь треугольника, две стороны которого равны \(8\) и \(12\), а угол между ними равен \(30 \degree\).
		\item Большее основание равнобедренной трапеции равно \(34\). Боковая сторона равна \(14\). Синус острого угла равен \( \dfrac{ 2\sqrt{10}}{ 7 } \). Найдите меньшее основание.
		\item Основания равнобедренной трапеции равны \(7\) и \(51\). Тангенс острого угла равен \( \dfrac{ 5 }{ 11 } \).  Найдите высоту трапеции.
	\end{listofex}
\end{class}
%END_FOLD

%BEGIN_FOLD % ====>>_ Домашняя работа 2 _<<====
\begin{homework}[number=2]
	\begin{listofex}
		\item Домашняя работа 2
	\end{listofex}
\end{homework}
%END_FOLD

%BEGIN_FOLD % ====>>_____ Занятие 5 _____<<====
\begin{class}[number=5]
	\begin{listofex}
		\item Занятие 5
	\end{listofex}
\end{class}
%END_FOLD

%BEGIN_FOLD % ====>>_____ Занятие 6 _____<<====
\begin{class}[number=6]
	\begin{listofex}
		\item Занятие 6
	\end{listofex}
\end{class}
%END_FOLD

%BEGIN_FOLD % ====>>_ Домашняя работа 3 _<<====
\begin{homework}[number=3]
	\begin{listofex}
		\item Домашняя работа 3
	\end{listofex}
\end{homework}
%END_FOLD

%BEGIN_FOLD % ====>>_____ Занятие 7 _____<<====
\begin{class}[number=7]
	\title{Подготовка к проверочной}
	\begin{listofex}
		\item Занятие 7
	\end{listofex}
\end{class}
%END_FOLD

=%BEGIN_FOLD % ====>>_ Проверочная работа _<<====
\begin{exam}
	\begin{listofex}
		\item Проверочная
	\end{listofex}
\end{exam}
%END_FOLD