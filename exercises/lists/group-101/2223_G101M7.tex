%
%===============>>  ГРУППА 10-1 МОДУЛЬ 7  <<=============
%
\setmodule{7}

%BEGIN_FOLD % ====>>_____ Занятие 1 _____<<====
\begin{class}[number=1]
	\begin{definit}
		Степень с натуральным показателем: \[ a^{\frac{p}{q}}=\sqrt[q]{a^p} \]
	\end{definit}
	\begin{listofex}
		\item Вычислите: % Никольский с125 н 4.7 аб
		\begin{tasks}(5)
			\task \( 25^{\frac{1}{2}} \)
			\task \( 49^{\frac{1}{2}} \)
			\task \( 49^{\frac{1}{3}} \)
			\task \( 16^{0,25} \)
			\task \( 100^{0,5} \)
			\task \( 16^{\frac{3}{4}} \)
			\task \( 27^{\frac{2}{3}} \)
			\task \( 25^{2,5} \)
			\task \( 8^{\frac{5}{3}} \)
			\task \( 27^{\frac{4}{3}} \)
		\end{tasks}
	\end{listofex}
	\begin{definit}
		\( а \)\\Степень произведения с рациональным показателем: \( (ab)^{r}=a^r \cdot b^r \) \\
		Степень частного с рациональным показателем: \( \left(\dfrac{a}{b}\right)^r=\dfrac{a^r}{b^r} \)
	\end{definit}
	\begin{listofex}
		\item Сравните \( a^r \) и \( a^q \), если \( a>1 \) и рациональные числа \(r\) и \(q\) таковы, что \(r>q\).
		\item Пусть числа \(a\) и \(r\) таковы, что \( 0<a<1 \), \(r\) --- рациональное число. Докажите, что если:
		\begin{tasks}(2)
			\task \( r<0 \), то \( a^r>1 \)
			\task \( r>0 \), то \( 0<a^r<1 \)
		\end{tasks}
		\item Упростите выражение: %4.17
		\begin{tasks}(4)
			\task \( x^{0,5} \cdot x^{0,25} \)
			\task \( a^{\frac{2}{3}} \cdot \left( a^{\frac{1}{6}} \right)^2 \)
			\task \( ax^{\frac{5}{8}} \cdot x^{-0,5} \)
			\task \( b \cdot b^{-\frac{5}{3}} \)
			\task \( a^{\frac{3}{5}} \cdot a \)
			\task \( a^{\frac{3}{8}} \cdot a^{-\frac{3}{8}} \)
			\task \( y^{\frac{1}{2}} \cdot y^{\frac{1}{4}} \cdot y^{\frac{1}{3}} \)
			\task \( b^{\frac{2}{3}} \cdot b^{\frac{3}{4}} \cdot b^{\frac{5}{6}} \)
		\end{tasks}
		\item Вычислите: % 4.18
		\begin{tasks}(2)
			\task \( 0,125^{-\frac{2}{3}}\cdot 8^{-\frac{2}{3}} \)
			\task \( 4,4^{\frac{1}{3}}:0,55^{\frac{1}{3}} \)
			\task \( 125^{1,5}\cdot 25^{-\frac{3}{4}} \)
			\task \( 2^{1,25}\cdot 16^{\frac{1}{16}} \)
		\end{tasks}
		\item Найдите значение выражения:
		\begin{tasks}(2)
			\task \( 4^8 \cdot 11^{10}:44*8 \)
			\task \( \dfrac{2^{-7}\cdot2^{-8}}{2^{-9}} \)
			\task \( (4a)^{2,5} \) при \( a>0 \)
			\task \( \dfrac{\sqrt[9]{a} \cdot \sqrt[18]{a} }{a \sqrt[6]{a}} \) при \(a=1,25\)
		\end{tasks}
		\item Упростите выражение:
		\begin{tasks}(2)
			\task \( \dfrac{5^{n+1}-5^{n-1}}{2\cdot 5^n} \)
			\task \( \dfrac{18^{n+3}}{3^{2n+5} \cdot 2^{n-2}} \)
		\end{tasks}
		\item Решите уравнения: % РЕШУЕГЭ
		\begin{tasks}(2)
			\task \( 2^{4-2x}=64 \) %26650
			\task \( 5^{x-7}=\dfrac{1}{125} \) %26652
			\task \( \left( \dfrac{1}{3} \right)^{x-8} = \dfrac{1}{9}  \) %26651
			\task \( \left( \dfrac{1}{2} \right)^{6-2x} = 4  \) %26653
		\end{tasks}
		\item Вычислите: \[ \left( 9^{-\frac{1}{4}}+(2\sqrt{2})^{-\frac{2}{3}} \right) \cdot \left( \sqrt[4]{9^{-1}} - (2\sqrt{2})^{-\frac{2}{3}} \right)  \]
	\end{listofex}
\end{class}
%END_FOLD

%BEGIN_FOLD % ====>>_____ Занятие 2 _____<<====
\begin{class}[number=2]
	\begin{listofex}
		\item Упростите выражение:
		\begin{tasks}(3)
			\task \( y^{\tfrac{3}{8}} \cdot y^{-0,4} \)
			\task \( x^2 \cdot x^{-\tfrac{2}{5}} \)
			\task \( a^{0,5} \cdot a^{1,5} \)
			\task \( b^{\tfrac{3}{4}} \cdot b^{-\tfrac{3}{4}} \)
			\task \( x^{t\frac{1}{5}} \cdot x^{t\frac{1}{3}} \cdot x^{\frac{4}{12}} \)
			\task \( y^{\tfrac{1}{4}} \cdot y^{0,75} \cdot y^{4} \)
		\end{tasks}
		\item Вычислите: 
		\begin{tasks}(4)
			\task \( 36^{\tfrac{1}{2}} \)
			\task \( 125^{\tfrac{1}{3}} \)
			\task \( 81^{\tfrac{1}{4}} \)
			\task \( 1^{0,25} \)
			\task \( (-9)^{\tfrac{1}{3}} \)
			\task \( 16^{\tfrac{3}{4}} \)
			\task \( 216^{\tfrac{2}{3}} \)
			\task \( (-512)^{\tfrac{1}{3}} \)
		\end{tasks}
		%\item Может ли значение выражения: НА ПОТОМ
		%\begin{tasks}
		%	\task \( \dfrac{x^{\tfrac{4}{3}}-x^{\tfrac{1}{3}}}{x^{\tfrac{1}{3}}-x^{\tfrac{2}{3}}}+0,25^{-1,5}-9(x-2)^0 \) равняться 1?
		%	\task \( \dfrac{x^{\tfrac{8}{5}}-2x^{\tfrac{3}{5}}}{x^{0,6}-2x^{-0,4}} -0,04^{-0,5}+2(x+1)^0 \) равняться -4?
		%\end{tasks}
		
		%\item Упростите выражение: %4.22 абв МИМО НАВСЕГДА
		%\begin{tasks}(1)
		%	\task \( \dfrac{2b^{0,5}}{b^{0,5}\cdot 3^{1,5}} - \left( \dfrac{b^{\frac{1}{3}}+3}{b^{0,5}-3^{1,5}} \right) : \left( \dfrac{1}{b^{\frac{1}{3}}} + \dfrac{3b^{\frac{1}{3}}}{b-27} \right)   \)
		%	\task \( \left( \dfrac{4}{a^{1,5-8}} - \dfrac{a^{0,5}-2}{a+2a^{0,5}+4} \right) \cdot \dfrac{a^2-8a^{0,5}}{a-16} - \dfrac{4a^{0,5}}{a^{0,5}+4}  \)
		%	\task \( \left( \dfrac{a^{0,75}(a+1)^{-\frac{1}{3}}}{a^{0,5}-1} \cdot \dfrac{a^{0,25}(a-1)^{\frac{1}{3}}}{a^{0,5}+1} \right)^{-\frac{1}{3}} : \dfrac{(a+1)^{-\frac{8}{9}}}{(a-1)^{\frac{7}{9}}\cdot a^{\frac{4}{3}}}  \)
		%\end{tasks}
		\item Найдите значение выражения: % первые 4 (1-4) с решуегэ и до 9 (5-8)
		\begin{tasks}(2)
			\task \(5^{0,36} \cdot 25^{0,32}  \)
			\task \( \dfrac{3^{6,5}}{9^{2,25}} \)
			\task \( 7^{\tfrac{4}{9}} \cdot 49^{\tfrac{5}{18}} \)
			\task \( \dfrac{2^{3,5} \cdot 3^{5,5}}{6^{4,5}} \)
			\task \( \dfrac{\sqrt{2,8} \cdot \sqrt{4,2}}{\sqrt{0,24}} \)
			\task \( \dfrac{\sqrt[9]{7} \cdot \sqrt[18]{7}}{\sqrt[6]{7}} \)
			\task \( \dfrac{\sqrt[5]{10} \cdot \sqrt[5]{16}}{\sqrt[5]{5}} \)
			\task \( 5 \cdot \sqrt[3]{9} \cdot \sqrt[6]{9} \)
		\end{tasks}
		
		\item Найдите значение выражения:
		\begin{tasks}(2)
			\task \( 16^{x-9}=0,5 \) %26654
			\task \( \left( \dfrac{1}{9} \right)^{x-13}=3  \) %26655
			\task \( 9^{-5+x}=729 \) %26666
			\task \( \left( \dfrac{1}{8} \right)^{-3+x}=512 \) %26670
			\task \( 125^x - \left( \dfrac{1}{25} \right)^x = 0  \) % Свой
		\end{tasks}
		\item Решите неравенства:
		\begin{tasks}(2)
			\task \( 2^{x^2} \le 4 \cdot 2^x \) %508297
			\task \( 2^x + 6 \cdot 2^{-x} \le 7 \) %508359
			\task \( 6^x+\left( \dfrac{1}{6} \right)^x > 2  \) %508296
			\task \( 25^x + 5^{x+1}+5^{1-x} +  \dfrac{1}{25^x} \le 12 \) %508319
		\end{tasks}
	\end{listofex}
\end{class}
%END_FOLD

%BEGIN_FOLD % ====>>_ Домашняя работа 1 _<<====
\begin{homework}[number=1]
	\begin{listofex}
		\item Домашняя работа 1 %4.22 г
		
		\item Найдите значение выражения:
		\begin{tasks}
			\task \(  \)
			\task \(  \)
		\end{tasks}
	\end{listofex}
\end{homework}
%END_FOLD

%BEGIN_FOLD % ====>>_____ Занятие 3 _____<<====
\begin{class}[number=3]
	\begin{listofex}
		\item Занятие 3 
	\end{listofex}
\end{class}
%END_FOLD

%BEGIN_FOLD % ====>>_____ Занятие 4 _____<<====
\begin{class}[number=4]
	\begin{listofex}
		\item Занятие 4
	\end{listofex}
\end{class}
%END_FOLD

%BEGIN_FOLD % ====>>_ Домашняя работа 2 _<<====
\begin{homework}[number=2]
	\begin{listofex}
		\item Домашняя работа 2
	\end{listofex}
\end{homework}
%END_FOLD

%BEGIN_FOLD % ====>>_____ Занятие 5 _____<<====
\begin{class}[number=5]
	\begin{listofex}
		\item Занятие 5
	\end{listofex}
\end{class}
%END_FOLD

%BEGIN_FOLD % ====>>_____ Занятие 6 _____<<====
\begin{class}[number=6]
	\begin{listofex}
		\item Занятие 6
	\end{listofex}
\end{class}
%END_FOLD

%BEGIN_FOLD % ====>>_ Домашняя работа 3 _<<====
\begin{homework}[number=3]
	\begin{listofex}
		\item Домашняя работа 3
	\end{listofex}
\end{homework}
%END_FOLD

%BEGIN_FOLD % ====>>_____ Занятие 7 _____<<====
\begin{class}[number=7]
	\title{Подготовка к проверочной}
	\begin{listofex}
		\item Занятие 7
	\end{listofex}
\end{class}
%END_FOLD

=%BEGIN_FOLD % ====>>_ Проверочная работа _<<====
\begin{exam}
	\begin{listofex}
		\item Проверочная
	\end{listofex}
\end{exam}
%END_FOLD