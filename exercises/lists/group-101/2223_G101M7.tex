%
%===============>>  ГРУППА 10-1 МОДУЛЬ 7  <<=============
%
\setmodule{7}

%BEGIN_FOLD % ====>>_____ Занятие 1 _____<<====
\begin{class}[number=1]
	\begin{definit}
		Степень с натуральным показателем: \[ a^{\frac{p}{q}}=\sqrt[q]{a^p} \]
	\end{definit}
	\begin{listofex}
		\item Вычислите: % Никольский с125 н 4.7 аб
		\begin{tasks}(5)
			\task \( 25^{\frac{1}{2}} \)
			\task \( 49^{\frac{1}{2}} \)
			\task \( 49^{\frac{1}{3}} \)
			\task \( 16^{0,25} \)
			\task \( 100^{0,5} \)
			\task \( 16^{\frac{3}{4}} \)
			\task \( 27^{\frac{2}{3}} \)
			\task \( 25^{2,5} \)
			\task \( 8^{\frac{5}{3}} \)
			\task \( 27^{\frac{4}{3}} \)
		\end{tasks}
	\end{listofex}
	\begin{definit}
		\( а \)\\Степень произведения с рациональным показателем: \( (ab)^{r}=a^r \cdot b^r \) \\
		Степень частного с рациональным показателем: \( \left(\dfrac{a}{b}\right)^r=\dfrac{a^r}{b^r} \)
	\end{definit}
	\begin{listofex}
		\item Сравните \( a^r \) и \( a^q \), если \( a>1 \) и рациональные числа \(r\) и \(q\) таковы, что \(r>q\).
		\item Пусть числа \(a\) и \(r\) таковы, что \( 0<a<1 \), \(r\) --- рациональное число. Докажите, что если:
		\begin{tasks}(2)
			\task \( r<0 \), то \( a^r>1 \)
			\task \( r>0 \), то \( 0<a^r<1 \)
		\end{tasks}
		\item Упростите выражение: %4.17
		\begin{tasks}(4)
			\task \( x^{0,5} \cdot x^{0,25} \)
			\task \( a^{\frac{2}{3}} \cdot \left( a^{\frac{1}{6}} \right)^2 \)
			\task \( ax^{\frac{5}{8}} \cdot x^{-0,5} \)
			\task \( b \cdot b^{-\frac{5}{3}} \)
			\task \( a^{\frac{3}{5}} \cdot a \)
			\task \( a^{\frac{3}{8}} \cdot a^{-\frac{3}{8}} \)
			\task \( y^{\frac{1}{2}} \cdot y^{\frac{1}{4}} \cdot y^{\frac{1}{3}} \)
			\task \( b^{\frac{2}{3}} \cdot b^{\frac{3}{4}} \cdot b^{\frac{5}{6}} \)
		\end{tasks}
		\item Вычислите: % 4.18
		\begin{tasks}(2)
			\task \( 0,125^{-\frac{2}{3}}\cdot 8^{-\frac{2}{3}} \)
			\task \( 4,4^{\frac{1}{3}}:0,55^{\frac{1}{3}} \)
			\task \( 125^{1,5}\cdot 25^{-\frac{3}{4}} \)
			\task \( 2^{1,25}\cdot 16^{\frac{1}{16}} \)
		\end{tasks}
		\item Найдите значение выражения:
		\begin{tasks}(2)
			\task \( 4^8 \cdot 11^{10}:44*8 \)
			\task \( \dfrac{2^{-7}\cdot2^{-8}}{2^{-9}} \)
			\task \( (4a)^{2,5} \) при \( a>0 \)
			\task \( \dfrac{\sqrt[9]{a} \cdot \sqrt[18]{a} }{a \sqrt[6]{a}} \) при \(a=1,25\)
		\end{tasks}
		\item Упростите выражение:
		\begin{tasks}(2)
			\task \( \dfrac{5^{n+1}-5^{n-1}}{2\cdot 5^n} \)
			\task \( \dfrac{18^{n+3}}{3^{2n+5} \cdot 2^{n-2}} \)
		\end{tasks}
		\item Решите уравнения: % РЕШУЕГЭ
		\begin{tasks}(2)
			\task \( 2^{4-2x}=64 \) %26650
			\task \( 5^{x-7}=\dfrac{1}{125} \) %26652
			\task \( \left( \dfrac{1}{3} \right)^{x-8} = \dfrac{1}{9}  \) %26651
			\task \( \left( \dfrac{1}{2} \right)^{6-2x} = 4  \) %26653
		\end{tasks}
		\item Вычислите: \[ \left( 9^{-\frac{1}{4}}+(2\sqrt{2})^{-\frac{2}{3}} \right) \cdot \left( \sqrt[4]{9^{-1}} - (2\sqrt{2})^{-\frac{2}{3}} \right)  \]
	\end{listofex}
\end{class}
%END_FOLD

%BEGIN_FOLD % ====>>_____ Занятие 2 _____<<====
\begin{class}[number=2]
	\begin{listofex}
		\item Упростите выражение:
		\begin{tasks}(3)
			\task \( y^{\tfrac{3}{8}} \cdot y^{-0,4} \)
			\task \( x^2 \cdot x^{-\tfrac{2}{5}} \)
			\task \( a^{0,5} \cdot a^{1,5} \)
			\task \( b^{\tfrac{3}{4}} \cdot b^{-\tfrac{3}{4}} \)
			\task \( x^{t\frac{1}{5}} \cdot x^{t\frac{1}{3}} \cdot x^{\frac{4}{12}} \)
			\task \( y^{\tfrac{1}{4}} \cdot y^{0,75} \cdot y^{4} \)
		\end{tasks}
		\item Вычислите: 
		\begin{tasks}(4)
			\task \( 36^{\tfrac{1}{2}} \)
			\task \( 125^{\tfrac{1}{3}} \)
			\task \( 81^{\tfrac{1}{4}} \)
			\task \( 1^{0,25} \)
			\task \( (-9)^{\tfrac{1}{3}} \)
			\task \( 16^{\tfrac{3}{4}} \)
			\task \( 216^{\tfrac{2}{3}} \)
			\task \( (-512)^{\tfrac{1}{3}} \)
		\end{tasks}
		
		%\item Упростите выражение: %4.22 абв МИМО НАВСЕГДА
		%\begin{tasks}(1)
		%	\task \( \dfrac{2b^{0,5}}{b^{0,5}\cdot 3^{1,5}} - \left( \dfrac{b^{\frac{1}{3}}+3}{b^{0,5}-3^{1,5}} \right) : \left( \dfrac{1}{b^{\frac{1}{3}}} + \dfrac{3b^{\frac{1}{3}}}{b-27} \right)   \)
		%	\task \( \left( \dfrac{4}{a^{1,5-8}} - \dfrac{a^{0,5}-2}{a+2a^{0,5}+4} \right) \cdot \dfrac{a^2-8a^{0,5}}{a-16} - \dfrac{4a^{0,5}}{a^{0,5}+4}  \)
		%	\task \( \left( \dfrac{a^{0,75}(a+1)^{-\frac{1}{3}}}{a^{0,5}-1} \cdot \dfrac{a^{0,25}(a-1)^{\frac{1}{3}}}{a^{0,5}+1} \right)^{-\frac{1}{3}} : \dfrac{(a+1)^{-\frac{8}{9}}}{(a-1)^{\frac{7}{9}}\cdot a^{\frac{4}{3}}}  \)
		%\end{tasks}
		\item Найдите значение выражения: % первые 4 (1-4) с решуегэ и до 9 (5-8)
		\begin{tasks}(2)
			\task \(5^{0,36} \cdot 25^{0,32}  \)
			\task \( \dfrac{3^{6,5}}{9^{2,25}} \)
			\task \( 7^{\tfrac{4}{9}} \cdot 49^{\tfrac{5}{18}} \)
			\task \( \dfrac{2^{3,5} \cdot 3^{5,5}}{6^{4,5}} \)
			\task \( \dfrac{\sqrt{2,8} \cdot \sqrt{4,2}}{\sqrt{0,24}} \)
			\task \( \dfrac{\sqrt[9]{7} \cdot \sqrt[18]{7}}{\sqrt[6]{7}} \)
			\task \( \dfrac{\sqrt[5]{10} \cdot \sqrt[5]{16}}{\sqrt[5]{5}} \)
			\task \( 5 \cdot \sqrt[3]{9} \cdot \sqrt[6]{9} \)
		\end{tasks}
		
		\item Найдите значение выражения:
		\begin{tasks}(2)
			\task \( 16^{x-9}=0,5 \) %26654
			\task \( \left( \dfrac{1}{9} \right)^{x-13}=3  \) %26655
			\task \( 9^{-5+x}=729 \) %26666
			\task \( \left( \dfrac{1}{8} \right)^{-3+x}=512 \) %26670
			\task \( 125^x - \left( \dfrac{1}{25} \right)^x = 0  \) % Свой
		\end{tasks}
		\item Решите неравенства:
		\begin{tasks}(2)
			\task \( 2^{x^2} \le 4 \cdot 2^x \) %508297
			\task \( 2^x + 6 \cdot 2^{-x} \le 7 \) %508359
			\task \( 6^x+\left( \dfrac{1}{6} \right)^x > 2  \) %508296
			\task \( 25^x + 5^{x+1}+5^{1-x} +  \dfrac{1}{25^x} \le 12 \) %508319
		\end{tasks}
	\end{listofex}
\end{class}
%END_FOLD

%BEGIN_FOLD % ====>>_ Домашняя работа 1 _<<====
\begin{homework}[number=1]
	\begin{listofex}
		\item Вычислите: %Галицкий 11.67 и 11.68(а)
		\begin{tasks}(1)
			\task \( (0,5)^{-4} + 16^{0,5} - (0,0625)^{-0,75} \cdot \left( \dfrac{4}{9} \right)^{-0,5} \)
			\task \( 81^{0,75} \cdot 32^{-0,4} - 8^{-\tfrac{2}{3}}\cdot 27^{\tfrac{1}{3}}+256^{0,5} \)
			
		\end{tasks}
		\item Найдите значение выражения: %Решуегэ до 12
		\begin{tasks}(2)
			\task \( \left( \dfrac{1}{2} \right)^{x-8}=2^x \)
			\task \( 8^{9-x}=64^x \)
			\task \( 2^{3+x}=0,4 \cdot 5^{3+x} \)
			\task \( 7^{18,5x+0,7}=\dfrac{1}{343} \)
		\end{tasks}
		\item Найдите значение выражения: % Решуегэ 5-9 в комменте, остальное с головы
		\begin{tasks}(2)
			\task \( 0,36^{x-3} \le 6 \)
			\task \( 7^{x} \cdot 9 < 441 \)
			\task \( 0,125^{0,3x+8} > 64 \)
			\task \( \left( \dfrac{8}{27} \right)^{\tfrac{1}{x}} > 1,5 \)
		\end{tasks}
	\end{listofex}
\end{homework}
%END_FOLD

%BEGIN_FOLD % ====>>_____ Занятие 3 _____<<====
\begin{class}[number=3]
	\begin{listofex}
		\item Найдите значение выражения: % степенные и до 13
		\begin{tasks}(2)
			\task \( \dfrac{49^{5,2}}{7^{8,4}} \)
			\task \( 3^{\sqrt{5}+10} \cdot 3^{-5-\sqrt{5}} \)
			\task \( (5^{12})^3:5^37 \)
			\task \( (49^6)^3 : (7^7)^5 \)
			\task \( 35^{-4,7} \cdot 7^{5,7} : 5^{-3,7} \)
			\task \( 0,8^{\tfrac{1}{7}} \cdot 5^{\tfrac{2}{7}} \cdot 20^{\tfrac{6}{7}} \)
			\task \( \left( \dfrac{2^{\tfrac{1}{3}} \cdot 2^{0,25}} {\sqrt[12]{2}} \right)^2 \)
			\task \( \dfrac{\left( 2^{0,6} \cdot 5^{\tfrac{2}{3}} \right)^{15}}{10^9} \)
		\end{tasks}
		\item Решите уравнения:
		\begin{tasks}(2)
			\task \( 16^{x-11} = \dfrac{1}{4} \)
			\task \( 7^{x+5}=\dfrac{343}{49} \)
			\task \( \left( \dfrac{3}{27} \right)^{-3+x}=3 \) 
			\task \( 125^x - \left( \dfrac{1}{25} \right)^x = 0  \)
			\task \( \left( \dfrac{4}{81} \right)^{-13-x}= 91,125 \)
			\task \( \left( \dfrac{2048}{256} \right)^{x+1,5}=0,25^x \)
		\end{tasks}
		\item Решите неравенства:
		\begin{tasks}(2)
			\task \( 2^{x^2} \le 4 \cdot 2^x \) %508297
			\task \( 2^x + 6 \cdot 2^{-x} \le 7 \) %508359
			\task \( 6^x+\left( \dfrac{1}{6} \right)^x > 2  \) %508296
			\task \( 25^x + 5^{x+1}+5^{1-x} +  \dfrac{1}{25^x} \le 12 \) %508319
		\end{tasks}
	\end{listofex}
\end{class}
%END_FOLD

%BEGIN_FOLD % ====>>_____ Занятие 4 _____<<====
\begin{class}[number=4]
	\begin{listofex}
		\item Вычислите: %Галицкий 8-9 №11.67 11.68
		\begin{tasks}(1)
			\task \( 16^{-0,75} \cdot 25^{-0,5} + 64^{-\tfrac{4}{3}}\cdot 9^{1,5} - 100^{-0,5} \)
			\task \( 81^{0,75} \cdot 32^{-0,4} - 8^{-\tfrac{2}{3}} \cdot 27^{\tfrac{1}{3}} + 256^{0,5} \)
			\task \( 0,5^{-0,4} + 16^{0,5}-0,0625^{-0,75} \cdot \left( \dfrac{4}{9} \right)^{-0,5}  \)
		\end{tasks}
		%\item Может ли значение выражения:
		%\begin{tasks}
		%	\task \( \dfrac{x^{\tfrac{4}{3}}-x^{\tfrac{1}{3}}}{x^{\tfrac{1}{3}}-x^{\tfrac{2}{3}}}+0,25^{-1,5}-9(x-2)^0 \) равняться 1?
		%	\task \( \dfrac{x^{\tfrac{8}{5}}-2x^{\tfrac{3}{5}}}{x^{0,6}-2x^{-0,4}} -0,04^{-0,5}+2(x+1)^0 \) равняться -4?
		%\end{tasks}
		\item Решите уравнения: %Простейшие 11-16
		\begin{tasks}(2)
			\task \( \left( \dfrac{1}{25} \right)^{x+2}=5^{x+5} \)
			\task \( \left( \dfrac{1}{6} \right)^{6-2x}=36 \)
			\task \( 0,5^{10-3x}=32 \)
			\task \( 7^{18,5x+0,7}=\dfrac{1}{343} \)
			\task \( 6^{12,5x+2}=\dfrac{1}{216} \)
			\task \( 2^{3+x}=0,4 \cdot 5^{3+x} \)
		\end{tasks}
		\item Решите неравенства: %РешуЕГЭ до 17
		\begin{tasks}(2)
			\task \( 3^x+10 \cdot 3^{-x} \le 11 \)
			\task \( 2^{2x} - 6 \cdot 2^{x} + 8 \le 0 \)
			\task \( 4^{x+1} - 5 \cdot 2^{x} + 1 \le 0 \)
			\task \( 9^x - 31 \cdot 3^x + 108 > 0 \)
			\task \( 4^x - 29 \cdot 2^x + 168 \le 0 \)
		\end{tasks}
		\item Решите неравенство: \[ 3^{-2x+4}-81 \cdot 3^{-x+3}-3^{-x+1}+81 \le 0 \]
	\end{listofex}
\end{class}
%END_FOLD

%BEGIN_FOLD % ====>>_ Домашняя работа 2 _<<====
\begin{homework}[number=2]
	\begin{listofex}
		\item Вычислите: %Галицкий 8-9 №11.67 11.68
		\[ 16^{-0,75} \cdot 25^{-0,5}+64^{-\tfrac{4}{3}} \cdot 9^{1,5} - 100^{-0,5} \]
			%\task \( 0,008^{-\tfrac{1}{3}} \cdot 125^{\tfrac{2}{3}}- \left( \mfrac{2}{10}{27} \right)^{-\tfrac{2}{3}}:(2,5)^{-2} \cdot (0,75)^{-1}  \)
		\item Решите уравнения: %Шестаков13 с99 н 1 2 5 7 8 Б
		\begin{tasks}(2)
			\task \( 2 \cdot 7^y = 7 \cdot 2^y \)
			\task \( \left( \dfrac{ 4 }{ 7 } \right)^{2y-5} = \left( \dfrac{  5}{8  } \right)^{2y-5} \)
			\task \( 15^y - 16 \cdot 15^y + 1 = 0 \)
		\end{tasks}
		\item Решите неравенства: %Шестаков15 с289 н 1 2 5 8 Б
		\begin{tasks}(2)
			\task \( 3^{\tfrac{4}{x}} > 27 \)
			\task \( \left( \dfrac{ 1 }{ 2 } \right)^{\tfrac{ 3x-2 }{ 3-x }} < 16 \)
			\task \( 6^x \cdot \left( \dfrac{ 1 }{36  } \right)^{5x-3} < 6 \)
			\task \( 6^{x-2} + 6^{x-1}+6^x > 258 \)
		\end{tasks}
		
		
	\end{listofex}
\end{homework}
%END_FOLD

%BEGIN_FOLD % ====>>_____ Занятие 5 _____<<====
\begin{class}[number=5]
	\begin{definit}
		Логарифмом положительного числа \(b\) по основанию \(a, \\ a>0 \), и \(a \neq 1\) называют число \( \alpha \), такое, что \(b=a^{c}\). Логарифм обозначают так: \[ c = \log_a b \]
	\end{definit}
	\begin{listofex}
		\item Вычислите:
		\begin{tasks}(4)
			\task \( \log_2 1 \)
			\task \( \log_{0,01} 0,01 \)
			\task \( \log_3 27 \)
			\task \( \log_5 125 \)
			\task \( \log_{10} 0,001 \)
			\task \( \log_4 1 \)
			\task \( \log_5 \dfrac{1}{5} \)
			\task \( \log_{10} 100 \)
			%\task \( \log_5 5^3 \)
			%\task \( \log_7 7^5 \)
		\end{tasks}
		\item Вычислите
		\begin{tasks}(3)
			\task \( 2^{\log_2 3} \)
			\task \( 3^{\log_3 5} \)
			\task \( 2^{\log_2 3 + \log_2 5} \)
			\task \( ( 3^{\log_3 7} )^2 \)
			\task \( 7^{2\log_7 3} \)
			\task \( 0,1^{2\log_{10} 10}  \)
			\task \( 9^{\log_3 12} \)
			\task \( 16^{\log_2 5} \)
			\task \( 49^{\log_7 \frac{1}{3}} \)
		\end{tasks}
	\end{listofex}
	\begin{definit}
		Логарифмом положительного числа \(b\) по основанию \(10\) называют \textbf{десятичым логарифмом числа \(b\)} и обозначают так: \( \lg b \).
	\end{definit}
	\begin{definit}
		Логарифмом положительного числа \(b\) по основанию \(e\) называют \textbf{натуральным логарифмом числа \(b\)} и обозначают так: \( \ln b \).
	\end{definit}
	\begin{listofex}[resume]
		\item Вычислите:
		\begin{tasks}(4)
			\task \( \log_{10} 10 \)
			\task \( \log_{10} 100 \)
			\task \( \lg 1000 \)
			\task \( \log_2 2^3 \)
			\task \( \log_5 5^7 \)
			%\task \( \log_9 9^{1999} \)
			\task \( e^{\ln 3} \)
			\task \( e^{2\ln 5} \)
			\task \( e^{-2\ln 3} \)
			\task \( \ln e \)
			\task \( \ln e^3 \)
			%\task \( \lg 10^n \)
			\task \( \ln \dfrac{1}{e} \)
			%\task \( \lg \sqrt[3]{0,01} \)
			\task \( \lg 10^{-1} \)
		\end{tasks}
	\end{listofex}
	\begin{definit}
		Свойства логарифма:
		\[ \log_a(M \cdot N) = \log_a M + \log_a N \]
		\[ \log_a \dfrac{M}{N} = \log_a M - \log_a N \]
		\[ \log_{a^l} M^k=\dfrac{k}{l}\log_a M \]
	\end{definit}
	\begin{listofex}[resume]
		
		\item Вычислите:
		\begin{tasks}(2)
			\task \( \log_6 2 + \log_6 3 \)
			\task \( \log_{15} 5 + \log_{15} 3 \)
			\task \( \log_4 \dfrac{2}{3} + \log_4 6 \)
			\task \( \log_2 \dfrac{2}{5} + \log_2 10 \)
			\task \( \log_2 6 - \log_2 3 \)
			\task \( \log_5 75 - \log_5 3 \)
			\task \( \log_3 36 - \log_3 4 \)
			%\task \( \log_3 0,81 - \log_3 0,03 \)
			\task \( 2\log_6 2 + \log_6 9 \)
			%\task \( \log_{11} 484 - 2 \log_{11} 2 \)
		\end{tasks}
		\item Решите неравенства:
		\begin{tasks}
			\task \( 3^{4x^2-13x} \le \dfrac{ 1 }{ 27 } \)
			\task \(\sqrt[7]{13^x} < 169  \)
		\end{tasks}
		%\item Вычислите: 
		%\[ \log_327-\log_{\sqrt{3}}27-\log_{1/3}27-\log_{\sqrt{3}/264}\left( \dfrac{64}{27} \right)  \]
	\end{listofex}
\end{class}
%END_FOLD

%BEGIN_FOLD % ====>>_____ Занятие 6 _____<<====
\begin{class}[number=6]
	\begin{definit}
		Для положительных чисел \( a, b, M \) таких, что \( a \neq 1, b \neq 1 \) справедливо следующее равенство: \[ \log_a M = \dfrac{\log_b M}{\log_b a} = \dfrac{1}{\log_M a} \]
	\end{definit}
	\begin{listofex}
		\item Выразите через логарифмы по основанию \(2\) и упростите:
		\begin{tasks}(4)
			\task \( \log_3 5 \)
			\task \( \log_4 9 \)
			\task \( \log_5 9 \)
			\task \( \log_{128} 8 \)
			\task \( \log_5 15 \)
			\task \( \log_3 12 \)
			\task \( \log_{16} 15 \)
			\task \( \log_{0,01} 2 \)
			\task \( \log_{0,25} 7 \)
			\task \( \log_{0,125} 3 \)
			\task \( \log_{\frac{1}{16}} 2 \)
			\task \( \log_{\frac{1}{32}} 5 \)
		\end{tasks}
	\end{listofex}
	\begin{definit}
		Свойства логарифма:
		\begin{tasks}(2)
			\task \( \log_a(M \cdot N) = \log_a M + \log_a N \)
			\task \( \log_a \dfrac{M}{N} = \log_a M - \log_a N \)
			\task \( \log_{a^l} M^k=\dfrac{k}{l}\log_a M \)
			\task \( \log_a b \cdot \log_b a = 1 \)
			\task \( a^{\log_b c}=c^{\log_b a} \)
		\end{tasks}
		
	\end{definit}
	\begin{listofex}[resume]
		\item Вычислите:
		\begin{tasks}(3)
			\task \( \log_{0,5}2 \)
			\task \( \log_{\frac{1}{2}}8^3 \)
			\task \( \log_{0,5}4^2 \)
			\task \( \log_3 \dfrac{1}{3} \)
			\task \( \log_3 \left(  \dfrac{1}{9} \right)^3 \)
			\task \( \log_4 \left(  \dfrac{1}{16} \right)^5 \)
			%\task \( \log_2 \sqrt{2} + \log_{\sqrt{2}}2 \)
			%\task \( \log_3 \sqrt{3^3} + \log^2_{\sqrt{3}}\sqrt{27} \)
			\task \( \log^2_5 \sqrt{5^5} - \log_{\sqrt{5}}5^3 \)
			\task \( 6^{\log_{36} 25} \)
			\task \( 7^{\log_{49} 36} \)
			\task \( 4^{\tfrac{1}{2\log_{625} 16 }} \)
			\task \( 36^{\log_6 2}:4^{\log_2 3} \)
			\task \( 9^{\log_3 5}+25^{\log_5 9} \)
			\task \( 49^{\log_7 3} \)
			\task \( 8^{\log_2 36^{\log_6 2}} \)
			\task \( (\sqrt[3]{5})^{\log_5 2} \)
			%\task \( \log_2 \sqrt[3]{16} + \log_{\sqrt{\frac{1}{16}}}4^{2} \)
			%\task \( \log^2_3 (27\sqrt{3} - \log_{\sqrt{\frac{1}{3}}}9) \)
			%\task \( \log_5 \sqrt{5\sqrt{5}} \)
		\end{tasks}
		%\item Вычислите: % МИМО НАВСЕГДА
		%\begin{tasks}(1)
		%	\task \( 3^{\log_{\sqrt[3]{9}}4} +2^{\tfrac{1}{\log_{16} 4}} \)
		%	\task \( \dfrac{\log_3 135}{\log_{15} 3} - \dfrac{\log_3 5}{\log_{405} 3} \)
		%	\task \( \dfrac{3+\log_{12}27}{3-\log_{12} 27} \cdot \log_6 16 \)
			%\task \( \log_3 27 - \log_{\sqrt{3}}27 - \log_{\frac{1}{3}} 27 - \log_{\frac{\sqrt{3}}{2}} \left( \dfrac{64}{27} \right)  \)
			%\task \( \log_{0,4}(0,2 \cdot \sqrt[3]{50}) + \log_{0,5}\left( \dfrac{\sqrt{15}}{5} \right) + \log_{0,32} \left( \dfrac{2\sqrt{2}}{5} \right)  \)
			%\task \( \left( \log_{0,5} \sqrt[3]{0,25} + 6\log_{0,25} 0,5 - 2 \log_{\frac{1}{16}}0,25 \right) : \log_{\sqrt{2}}\sqrt[5]{8} \)
		%\end{tasks}
		\item Решите уравнения: %По первые 4 с решуегэ простейшие и сложнейшие
		\begin{tasks}(2)
			\task \( \log_2 (4-x)=7 \)
			\task \( \log_5(4+x)=2 \)
			\task \( \log_5(5-x)=\log_5 3 \)
			%\task \( \log_2(15+x)=\log_2 3 \)
			%\task \( \log_5 (2-x) = \log_{25} x^4 \)
			%\task \( \log_2 (x^2-14x)=5 \)
			%\task \( 6 \log^2_8 x -5\log_8 x+1=0 \)
			%\task \( 1+\log_2(9x^2+5)=\log_{\sqrt{2}} \sqrt{8x^4+14} \)
		\end{tasks}
	\end{listofex}
\end{class}
%END_FOLD

%BEGIN_FOLD % ====>>_ Домашняя работа 3 _<<====
\begin{homework}[number=3]
	\begin{listofex}
		\item Вычислите:
		\begin{tasks}(3)
			\task \( \log_2 4 \)
			\task \( \log_2 16 \)
			\task \( \log_3 3 \)
			\task \( \log_3 27 \)
			\task \( \log_4 4^3 \)
			\task \( \log_7 49^8 \)
		\end{tasks}
		\item Вычислите
		\begin{tasks}(3)
			\task \( 2^{\log_2 5} \)
			\task \( 7^{\log_7 9} \)
			\task \( (3^2)^{\log_3 7} \)
			\task \( 10^{3\lg 5} \)
			\task \( 3^{\log_3 90} \)
			\task \( 5^{\log_5 0,5} \)
			\task \( 10^{\lg 3} \)
			\task \( 10^{2\lg 3} \)
			\task \( 10^{-3\lg 2} \)
		\end{tasks}
		\item Вычислите:
		\begin{tasks}(2)
			\task \( \log_3 0,9 + \log_3 30 \)
			\task \( \log_8 \dfrac{8}{7} + \log_8 \dfrac{7}{8} \)
			\task \( \log_4 48 - \log_4 3 \)
			\task \( \log_7 0,98 - \log_7 0,14 \)
			\task \( 4\log_{12} 2 + 2\log_{12} 3 \)
			\task \( \log_5 100 - 2 \log_5 2 \)
		\end{tasks}
		\item Решите уравнения: %По вторые 4 с решуегэ простейшие и сложнейшие
		\begin{tasks}(2)
			\task \( \log_4 (x+3) = \log_4 (4x-15) \)
			\task \( \log_{\tfrac{1}{7}}(7-x) = -2 \)
			\task \( \log_5 (5-x) = 2 \log_5 3 \)
			\task \( \log_5 (x^2+2x) = \log_5 (x^2+10) \)
			%\task \( \log_7 (x+2) = \log_{49} (x^4) \)
			%\task \( \log_3 (x^2-24x)=4 \)
			%\task \( \log_3 (x^2-2x)=1 \)
			%\task \( \log_2^2 (x-4) -6 \log_2 (x-4) = 7 \)
			%\task \( \log_2^2(x^2) - 16\log_2 (2x) +31=0 \)
		\end{tasks}
	\end{listofex}
\end{homework}
%END_FOLD

%BEGIN_FOLD % ====>>_____ Занятие 7 _____<<====
\begin{class}[number=7]
	\title{Подготовка к проверочной}
	\begin{listofex}
		\item Занятие 7
	\end{listofex}
\end{class}
%END_FOLD

=%BEGIN_FOLD % ====>>_ Проверочная работа _<<====
\begin{exam}
	\begin{listofex}
		\item Проверочная
	\end{listofex}
\end{exam}
%END_FOLD