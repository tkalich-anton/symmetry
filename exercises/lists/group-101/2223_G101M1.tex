%Группа 10-1 Модуль 1
\title{Занятие №1}
\begin{listofex}
	\item \exercise{1379}
	\item \exercise{1402}
	\item \exercise{4120}
	\item \exercise{4121}
	\item \exercise{4122}
	\item Вычислить:
	\begin{enumcols}[itemcolumns=3]
		\item \exercise{4123}
		\item \exercise{4124}
		\item \exercise{4125}
	\end{enumcols}
	\item \exercise{4126}
	\item \exercise{4127}
	\item \exercise{4128}
	\item \exercise{4129}
	\item \exercise{4130}
	\item Упростить выражение:
	\begin{enumcols}[itemcolumns=2]
		\item \exercise{4131}
		\item \exercise{4132}
	\end{enumcols}
\end{listofex}
\newpage
\title{Занятие №2}
\begin{listofex}
	\item \exercise{1463}
	\item \exercise{1420}
	\item \exercise{4133}
	\item \exercise{4134}
	\item \exercise{4135}
	\item Вычислить:
	\begin{enumcols}[itemcolumns=3]
		\item \exercise{4136}
		\item \exercise{4137}
		\item \exercise{4138}
	\end{enumcols}
	\item \exercise{4139}
	\item \exercise{4140}
	\item \exercise{4141}
	\item \exercise{4142} 
	\item \exercise{4143}
	\item Упростить выражение:
	\begin{enumcols}[itemcolumns=2]
		\item \exercise{4144}
		\item \exercise{4145}
	\end{enumcols}
	\item \exercise{4146}
\end{listofex}
\newpage
\title{Домашняя работа №1}
\begin{listofex}
	\item \exercise{1471}
	\item \exercise{1431}
	\item \exercise{4147}
	\item \exercise{4148}
	\item \exercise{4149}
	\item Вычислить:
	\begin{enumcols}[itemcolumns=3]
		\item \exercise{4150}
		\item \exercise{4151}
		\item \exercise{4152}
	\end{enumcols}
	\item \exercise{4153}
	\item \exercise{4154}
	\item \exercise{4155}
	\item \exercise{4156}
	\item \exercise{4157}
	\item Упростить выражение:
	\begin{enumcols}[itemcolumns=2]
		\item \exercise{4158}
		\item \exercise{4159}
	\end{enumcols}
	\item \exercise{4160}
\end{listofex}
\newpage
\title{Занятие №3}
\begin{listofex}
	\item Упростить выражение:
	\begin{enumcols}[itemcolumns=1]
		\item \exercise{1456}
		\item \exercise{1453}
	\end{enumcols}
	\item Вычислить:
	\begin{enumcols}[itemcolumns=3]
		\item \( \dfrac{7!}{5!} \)
		\item \( \dfrac{2000!}{1999!} \)
		\item \( \dfrac{5!+6!+7!}{8!-7!} \)
	\end{enumcols}
	\item Докажите, что для любого натурального \( n \) верно равенство:
	\begin{enumcols}[itemcolumns=2]
		\item \( n!+(n+1)! = n!(n+2) \)
		\item \( (n-1)!+n!+(n+1)! = (n+1)^2(n-1)! \)
	\end{enumcols}
	\item Запишите в виде дроби:
	\begin{enumcols}[itemcolumns=2]
		\item \( \dfrac{1}{(n+1)!}-\dfrac{n^2+5n}{(n+3)!} \)
		\item \( \dfrac{1}{(k-1)!}-\dfrac{k}{(k+1)!} \)
	\end{enumcols}
	\item Множество, состоящее из шести элементов \( A_1 \), \( A_2 \), \( A_3 \), \( A_4 \), \( A_5 \), \( A_6 \), упорядочили всеми возможными способами. Сколько таких способов? В скольких случаях:
	\begin{enumcols}
		\item элемент \( A_1 \) будет первым по порядку;
		\item элемент \( A_1 \) не будет ни первым ни последним;
		\item элемент \( A_1 \) будет первым, а \( A_6 \) будет последним.
	\end{enumcols}
	\item Сколькими различными способами можно усадить в ряд трех мальчиков и трех девочек так, чтобы никакие два мальчика и никакие две девочки не оказались рядом?
	\item Вычислить \( P_{12}:P_{10} \)
	\item Моторная лодка прошла против течения реки \( 160  \) км и вернулась в пункт отправления, затратив на обратный путь на \( 6  \) часов меньше. Найдите скорость течения, если скорость лодки в неподвижной воде равна \( 13  \) км/ч. Ответ дайте в км/ч.
\end{listofex}
\newpage
\title{Занятие №4}
	\begin{listofex}
		\item Упростить выражение:
	\begin{enumcols}[itemcolumns=2]
		\item \exercise{1433}
		\item \exercise{1383}
	\end{enumcols}
	\item Вычислить:
	\begin{enumcols}[itemcolumns=2]
		\item \exercise{1098}
		\item \exercise{1775}
		\item \exercise{1756}
	\end{enumcols}
	\item Вычислить:
	\begin{enumcols}[itemcolumns=2]
		\item \exercise{1743}
		\item \exercise{1759}
	\end{enumcols}
	\item Вычислить:
	\begin{enumcols}[itemcolumns=3]
		\item \( \dfrac{8!}{5!} \)\answer{\( 336 \)}
		\item \( \dfrac{500!}{498!} \)\answer{\( 249500 \)}
		\item \( \dfrac{3!+5!+6!}{141\cdot4!-282\cdot3!} \)\answer{\( 0,5 \)}
	\end{enumcols}
	\item Докажите, что для любого натурального \( n \) верно равенство:
	\begin{enumcols}[itemcolumns=2]
		\item \( (n+1)!-n! = n!n \)
		\item \( \dfrac{(n+1)!}{(n-1)!}=n^2+n \)
	\end{enumcols}
	\item \exercise{1864}
\end{listofex}
\newpage
\title{Домашняя работа №2}
\begin{listofex}
	\item Упростить выражение:
	\begin{enumcols}[itemcolumns=2]
		\item \exercise{1472}
		\item \exercise{1512}
		\item \exercise{1467}
	\end{enumcols}
	\item Вычислить:
	\begin{enumcols}[itemcolumns=2]
		\item \exercise{1215}
		\item \exercise{1776}
		\item \exercise{1765}
		\item \exercise{1686}
	\end{enumcols}
	\item Вычислить:
	\begin{enumcols}[itemcolumns=2]
		\item \exercise{1741}
		\item \exercise{1758}
	\end{enumcols}
	\item Вычислить:
	\begin{enumcols}[itemcolumns=3]
		\item \( \dfrac{20!}{22!} \) \answer{\( \dfrac{1}{462} \)}
		\item \( \dfrac{15!}{10!\cdot5!} \) \answer{\( 3003 \)}
		\item \( \dfrac{18!-17\cdot17!-16\cdot16!}{17!-16!} \) \answer{\( \dfrac{579}{16} \)}
	\end{enumcols}
	\item Докажите, что для любого натурального \( n \) верно равенство:
	\begin{enumcols}[itemcolumns=2]
		\item \( (n+1)!-n!+(n-1)! = (n^2+1)(n-1)! \)
		\item \( \dfrac{(n-1)!}{n!}-\dfrac{n!}{(n+1)!}=\dfrac{1}{n(n+1)} \)
	\end{enumcols}
	\item \exercise{1865}
\end{listofex}
\newpage
\title{Занятие №5}
\begin{listofex}
	\item Решить уравнение:
	\begin{enumcols}[itemcolumns=2]
		\item \exercise{3588}
		\item \exercise{3595}
		\item \exercise{3623}
		\item \exercise{3585}
	\end{enumcols}
	\item Решить уравнение:
	\begin{enumcols}[itemcolumns=3]
		\item \exercise{455}
		\item \exercise{468}
		\item \exercise{477}
	\end{enumcols}
	\item Решить уравнение:
	\begin{enumcols}[itemcolumns=2]
		\item \exercise{491}
		\item \exercise{496}
	\end{enumcols}
	\item Решить уравнение:
	\begin{enumcols}[itemcolumns=3]
		\item \exercise{41}
		\item \exercise{543}
		\item \exercise{549}
	\end{enumcols}
	\item Решить уравнение:
	\begin{enumcols}[itemcolumns=2]
		\item \exercise{509}
		\item \exercise{498}
	\end{enumcols}
	\item Решить уравнение:
	\begin{enumcols}[itemcolumns=3]
		\item \exercise{32}
		\item \exercise{16}
		\item \exercise{3670}
	\end{enumcols}
	\item Решить уравнение:
	\begin{enumcols}[itemcolumns=2]
		\item \exercise{3627}
		\item \exercise{3629}
	\end{enumcols}
\end{listofex}
%\newpage
%\title{Занятие №6}
%\begin{listofex}
%	\item 1
%	
%\end{listofex}
\newpage
\title{Домашняя работа №3}
\begin{listofex}
	\item Решить неравенство:
	\begin{enumcols}[itemcolumns=2]
		\item \( \dfrac{3+7x}{4}>2x+1 \)
		\item \( \dfrac{x}{5}+\dfrac{x+2}{3}\ge\dfrac{4x+5}{15}-\dfrac{2}{3} \)
	\end{enumcols}
	\item Решить неравенство:
	\begin{enumcols}[itemcolumns=2]
		\item \( \dfrac{3x^2}{4}\ge\dfrac{4x}{5} \)
		\item \( (12x+6)(x-7)\le0 \)
		\item \( \dfrac{x^2}{\sqrt{2}}<\sqrt{98} \)
		\item \( x^2-17x+16\ge0 \)
		\item \( (5x-4)^2\ge(4x-5)^2 \)
	\end{enumcols}
	\item Решить систему неравенств:
	\begin{enumcols}[itemcolumns=2]
		\item \( \left\{
		\begin{array}{l}
			4x+9\le9x+4,\\
			1,7x\le51
		\end{array}
		\right. \) \answer{\( [1;30] \)}
		\item \( \left\{
		\begin{array}{l}
			6(5x+4)-4(5x+6)\le10x+11,\\
			\dfrac{x+2}{5}+\dfrac{x+5}{2}\ge\dfrac{x+3}{4}+\dfrac{x+4}{3}
		\end{array}
		\right. \) \answer{\( [-7;+\infty) \)}
	\end{enumcols}
	\item \exercise{4012}
	\item Решить неравенство:
	\begin{enumcols}[itemcolumns=2]
		\item \exercise{4060}
		\item \exercise{4064}
	\end{enumcols}
\end{listofex}
\newpage
\title{Занятие №7}
\begin{listofex}
	\item \exercise{1318}
	\item Решить неравенство:
	\begin{enumcols}[itemcolumns=2]
		\item \( \dfrac{4+5x}{2}>3x+1 \) \answer{\( (-\infty;2) \)}
		\item \( \dfrac{x}{3}-\dfrac{3-x}{5}\ge\dfrac{x+12}{15}-\dfrac{9}{5} \) \answer{\( [-\dfrac{6}{7};+\infty) \)}
		\item \( (2x-1)(x+12)\le0 \) \answer{\( [-12;0,5] \)}
		\item \( \dfrac{x^2}{\sqrt{2}}<\sqrt{162} \) \answer{\( (-3\sqrt{2};3\sqrt{2}) \)}
		\item \( x^2-19x+18\ge0 \) \answer{\( (-\infty;1]\cup[18;+\infty) \)}
		\item \( (3x-7)^2\ge(7x-3)^2 \) \answer{\( [-1;1] \)}
	\end{enumcols}
	\item Решить систему неравенств:
	\begin{enumcols}[itemcolumns=2]
		\item \( \left\{
		\begin{array}{l}
			x^2+9x+8\le0,\\
			-0,3x\ge2,4
		\end{array}
		\right. \) \answer{\( x=-8 \)}
		\item \( \left\{
		\begin{array}{l}
			5(4x+3)-4(5x+3)>3x,\\
			\dfrac{2}{3}x<\dfrac{3}{2}x+5
		\end{array}
		\right. \) \answer{\( (-6;1) \)}
	\end{enumcols}
	\item \exercise{4011}
	\item Решить неравенство:
	\begin{enumcols}[itemcolumns=2]
		\item \exercise{4059}
		\item \exercise{4063}
	\end{enumcols}
\end{listofex}
%\newpage
%\title{Домашняя работа №4}
%\begin{listofex}
%	\item 1
%	
%\end{listofex}
\newpage
\title{Проверочная работа}
\begin{listofex}
	\item \exercise{1379}
	\item Вычислить:
	\begin{enumcols}[itemcolumns=3]
		\item \exercise{4123}
		\item \exercise{4124}
		\item \exercise{4125}
	\end{enumcols}
	\item Вычислить:
	\begin{enumcols}[itemcolumns=2]
		\item \exercise{1098}
		\item \exercise{1756}
	\end{enumcols}
	\item \exercise{4126}
	\item \exercise{4145}
	\item Решить уравнение:
	\begin{enumcols}[itemcolumns=2]
		\item \exercise{3595}
		\item \exercise{41}
		\item \exercise{3585}
	\end{enumcols}
	\item \exercise{509}
\end{listofex}