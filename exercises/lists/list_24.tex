%Группа 71 Модуль 8 Занятие №3
\itemmode{on}
\begin{enumcols}[label=\textbf{\arabic*.}]
	\item \exercise{758}
	\item Медиана, проведенная в треугольнике \( ABC \) из угла \( A \), равна половине стороны, к которой проведена. Докажите, что треугольник \( ABC \) -- прямоугольный.
	\item Докажите, что если треугольник вписан в окружность и одна из его сторон является диаметром этой окружности, то такой треугольник может быть только прямоугольным.
	\item Чему равен угол между биссектрисами двух смежных углов? Докажите.
	\item Чему равен угол между биссектрисами двух вертикальных углов? Докажите.
	\item Чему равна сумма углов в выпуклом четырехугольнике? В пятиугольнике? В 102-угольнике? В \( n \)-угольнике?
	\item Чему равна сумма внешних углов в \( n \)-угольнике, образованных путем продолжения каждой стороны \( n \)-угольника один раз?
	\item \exercise{761}
	\item Один из углов треугольника равен \( 40\degree \). Найдите угол между биссектрисами, проведенными из вершин двух других углов.
	\item Один из углов треугольника равен \( 50\degree \). Найдите угол между высотами, проведенными из вершин двух других углов.
	\item Докажите, что в прямоугольном треугольнике катет, лежащий напротив угла \( 30\degree \), равен половине гипотенузы.
	\item Катет прямоугольного треугольника равен половине гипотенузы. Докажите, что угол, противолежащий этому катету, равен \( 30\degree \).
	\item Острый угол прямоугольного треугольника равен \( 30\degree \). Докажите, что высота и медиана, проведенные из вершины прямого угла, делят его на три равные части.
	\item В прямоугольном треугольнике один из углов равен \( 30\degree \). Докажите, что в этом треугольнике отрезок перпендикуляра, проведенного к гипотенузе через его середину до пересечения с катетом, втрое меньше большего катета.
	\item Высота прямоугольного треугольника, опущенная на гипотенузу, равна 1. Один из острых углов равен \( 15\degree \). Найдите длину гипотенузы.
	\item Две различные окружности пересекаются в точках \( A \) и  \( B \). Докажите, что прямая, проходящая через центры окружностей, делит отрезок \( AB \) пополам и перпендикулярна ему.
	\item Две окружности пересекаются в точках \( A \) и \( B \), \( AM \) и \( AN \) -- диаметры окружностей. Докажите, что точки \( M \), \( N \), \( B \) лежат на одной прямой.
\end{enumcols}