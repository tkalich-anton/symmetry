%Группа 6-1 Модуль 1 Занятие №1
\begin{listofex}
	\item Вычислить:
	\begin{enumcols}[itemcolumns=2]
		\item \( 25\cdot(28\cdot105+7236:18)-(4247-1823):6\cdot25 \)
		\item \( 3124:(3\cdot504-4\cdot307)+10403:101 \)
	\end{enumcols}
	\item Лёва с Васей решили купить футбольный мяч. У Лёвы не хватило \( 200 \) рублей, чтобы его
	купить, а у Васи \( 300 \) рублей. Тогда они сложили свои деньги и купили мяч,
	причём \( 600 \) рублей у них осталось. Сколько стоил футбольный мяч?
	\item Найти значение выражения \( (a-b)\cdot(b+c) \), если \( a=247;\; b=189;\; c=127 \).
	\item Найти значение выражения \( a^2+2\cdot a\cdot b + b^2 \), если \( a=217 \) и \( b=83 \).
	\item Туристы были в походе три дня. Во второй день они прошли 18 км, что на 5 км меньше,
	чем в первый день, а в третий день они прошли на 19 км меньше, чем за два предыдущих дня.
	Сколько километров прошли туристы за три дня?
	\item Периметр треугольника равен 63 см. Одна сторона равна 18 см, что на 7 см меньше второй стороны. Найдите третью сторону треугольника.
	\item Сошлись два пастуха Иван и Пётр. Иван говорит Петру: «Отдай мне одну овцу,
	тогда у меня будет ровно вдвое больше овец, чем у тебя.» А Пётр ему отвечает: «Нет!
	Лучше ты мне отдай одну овцу, тогда у нас овец будет поровну». Сколько же было овец у
	каждого?
	
	%Занятие №2
	\item При ремонте шоссе длиной в 69 км в первый день отремонтировали 7 км, а в каждый из
	трех последующих дней ремонтировали на 3 км больше, чем в предыдущий. Во сколько раз
	оставшийся участок шоссе меньше отремонтированного?
	
	% ДЗ №1
	%\item Фермер убрал урожай картофеля за три дня. В первый день он убрал 19 грядок, что на 6 грядок больше, чем в третий день, а во второй день он убрал на 12 грядок меньше, чем за первый и третий дни вместе. Сколько грядок картофеля убрал фермер за три дня?
	%\item Периметр треугольника равен 61 см. Одна сторона равна 16 см, а вторая в два раза больше третьей. Найдите вторую и третью стороны треугольника.
\end{listofex}