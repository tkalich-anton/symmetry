%Список задач для кураторов
\section*{Список задач по классам}
\subsection*{5 класс}
\begin{enumerate}[label=\textbf{\arabic*}.]
	\item Пароход проходит 150 км по озеру за 6 часов, а 84 км против течения реки – за 4 часа.
	Найдите скорость течения реки.
	\item Сплав состоит из 5 частей цинка, 7 частей олова и 9 частей алюминия. Сколько цинка,
	олова и алюминия содержится в 3 кг 150 г сплава?
	\item Юноша и девушка измерили одно и то же расстояние в 141 м шагами. Шаг девушки 50 см,
	а шаг юноши 60 см. Сколько раз их следы совпали?
	\item Одноместная байдарка проплывает дистанцию гребного канала за 28 секунд, а
	двухместная – за 21 секунду. Обе байдарки стартовали одновременно с противоположных
	концов канала. Через сколько секунд они встретятся?
\end{enumerate}
\subsection*{6 класс}
\begin{enumerate}[label=\textbf{\arabic*}.]
	\item Докажите, что число \( 10^{2011}+2015 \) делится на \( 9 \).
	\item В магазине всё для чая продаются 5 разных чашек, 4 разных блюдца и 3 разных ложки. Сколькими способами можно купить комплект из блюдца, чашки и ложки?
	\item А) Сколько существует чётных пятизначных чисел? Б) Сколько существует нечётных четырёхзначных чисел?
	\item Найти \( 2\dfrac{2}{3}\% \) от \( 33 \)
\end{enumerate}
\subsection*{7-9 класс}
\begin{enumerate}[label=\textbf{\arabic*}.]
	\item \exercise{1759}
	\item \exercise{1747}
	\item \exercise{1663}
	\item \exercise{1605}
	\item \exercise{35}
	\item \exercise{1030}
	\item \exercise{1081}
	\item \exercise{3700}
	\item \exercise{3695}
	\item \exercise{3779}
	\item \exercise{4001}
	\item \exercise{4038}
	\item \exercise{4062}
	\item \exercise{3828}
\end{enumerate}
\subsection*{10-11 класс}
\begin{enumerate}[label=\textbf{\arabic*}.]
	\item \exercise{1404}
	\item \exercise{1848}
	\item \exercise{1294}
	\item \exercise{1576}
	\item \exercise{1136}
	\item \exercise{2868}
	\item \exercise{2883}
	\item \exercise{2907}
	\item \exercise{2964}
	\item \exercise{2979}
	\item \exercise{2995}
	\item \exercise{1798}
	\item \exercise{1185}
	\item \exercise{683}
	\item \exercise{3006}
	\item \exercise{3173}
	\item \exercise{3188}
	\item \exercise{3276}
	\item \exercise{3435}
	\item \exercise{3462}
	\item \exercise{3880}
	\item \exercise{3904}
	\item \exercise{3937}
\end{enumerate}