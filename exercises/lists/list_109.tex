%Список задач для кураторов
\title{Список задач по классам}
\title{5 класс}
\begin{listofex}
	\item Пароход проходит 150 км по озеру за 6 часов, а 84 км против течения реки – за 4 часа.
	Найдите скорость течения реки.
	\item Сплав состоит из 5 частей цинка, 7 частей олова и 9 частей алюминия. Сколько цинка,
	олова и алюминия содержится в 3 кг 150 г сплава?
	\item Юноша и девушка измерили одно и то же расстояние в 141 м шагами. Шаг девушки 50 см,
	а шаг юноши 60 см. Сколько раз их следы совпали?
\end{listofex}
\title{6 класс}
\begin{listofex}
	\item Докажите, что число \( 10^{2011}+2015 \) делится на \( 9 \).
	\item В магазине всё для чая продаются 5 разных чашек, 4 разных блюдца и 3 разных ложки. Сколькими способами можно купить комплект из блюдца, чашки и ложки?
	\item Сколько существует чётных пятизначных чисел?
	\item Одноместная байдарка проплывает дистанцию гребного канала за \( 28 \) секунд, а
	двухместная --- за \( 21 \) секунду. Обе байдарки стартовали одновременно с противоположных
	концов канала. Через сколько секунд они встретятся?
	\item Найти:
	\begin{enumcols}[itemcolumns=4]
		\item \( \dfrac{12}{19} \) от \( 76 \);
		\item \( 2\dfrac{2}{3}\% \) от \( 33 \)
		\item НОД\( (50;75) \)
		\item НОК\( (48;72) \)
	\end{enumcols}
\end{listofex}
\title{7 -- 9 класс}
\begin{listofex}
	\item Вычислить:
	\begin{enumcols}[itemcolumns=2]
		\item \exercise{1605}
		\item \exercise{1759}
		\item \exercise{1745}
		\item \exercise{1747}
		\item \exercise{1663}
	\end{enumcols}
	\item \exercise{1471}
	\item Решить уравнение:
	\begin{enumcols}[itemcolumns=2]
		\item \exercise{35}
		\item \exercise{1030}
		\item \exercise{427}
		\item \exercise{1081}
		\item \exercise{3779}
		\item \exercise{3741}
	\end{enumcols}
	\item Решить неравенство или систему неравенств:
	\begin{enumcols}[itemcolumns=2]
		\item \exercise{4001}
		\item \exercise{4038}
		\item \exercise{4062}
		\item \exercise{3828}
	\end{enumcols}
	\item \exercise{2422}
	\item \exercise{2441}
	\item Постройте график функции:
	\begin{enumcols}[itemcolumns=2]
		\item \( y=|x+5|-2x+3 \)
		\item \( y=|x^2-5x+6| \)
		\item \exercise{1238}
	\end{enumcols}
	\item \exercise{26}
	\item Первый насос наполняет бак за \( 20 \) минут, второй --- за \( 30 \) минут, а третий --- за \( 1 \) час. За сколько минут наполнят бак три насоса, работая одновременно?
\end{listofex}
\title{10 -- 11 класс}
\begin{listofex}
	\item Вычислить:
	\begin{enumcols}[itemcolumns=2]
		\item \exercise{1404}
		\item \exercise{1848}
		\item \exercise{594}
		\item \exercise{1136}
		\item \exercise{2964}
		\item \exercise{2979}
	\end{enumcols}
	\item Вычислить:
	\begin{enumcols}[itemcolumns=1]
		\item \exercise{2883}
		\item \exercise{2907}
		\item \( \sqrt{(\sin60\degree-2)^2}-\sqrt{(\ctg30\degree-1)^2} \)
	\end{enumcols}
	\item Докажите тождество:\quad\( \dfrac{1+\frac{1}{\tg x}+\frac{1}{\tg^2x}}{1+\frac{1}{\ctg^x}+\frac{1}{\ctg^2x}}=\ctg^2x \)
	\item \exercise{3173}
	\item \exercise{3188}
	\item Решить уравнение:
	\begin{enumcols}[itemcolumns=2]
		\item \exercise{1185}
		\item \exercise{683}
		\item \exercise{3435}
		\item \exercise{3462}
	\end{enumcols}
	\item \exercise{3982}
	\item Решить неравенство:
	\begin{enumcols}[itemcolumns=2]
		\item \exercise{3880}
		\item \exercise{3904}
		\item \exercise{3937}
		\item \exercise{3952}
	\end{enumcols}
\end{listofex}
\newpage
\title{Импровизация}
\begin{listofex}
	\item Вычислить:
	\begin{enumcols}[itemcolumns=2]
		\item \exercise{1215}
		\item \exercise{2995}
		\item \exercise{1576}
		\item \exercise{1852}
	\end{enumcols}
	\item Построить график функции: \( y=x^2-5|x|+6 \)
	\item \exercise{186}
	\item \exercise{3967}
	\item \exercise{2444}
\end{listofex}