%
%===============>>  ГРУППА 5-1 МОДУЛЬ 6  <<=============
%
\setmodule{6}

%BEGIN_FOLD % ====>>_____ Занятие 1 _____<<====
\begin{class}[number=1]
		\begin{definit}
			Процент --- это \( \dfrac{1}{100} \) от числа. Например, \( 1\% \) от \( 300 \) --- это \( \dfrac{1}{100}\cdot300=3 \).
		\end{definit}
		\begin{listofex}
		\item Найдите один процент от:
		\begin{tasks}(4)
			\task \( 200 \)
			\task \( 1000 \)
			\task \( 350 \)
			\task \( 20 \)
		\end{tasks}
		\item Найдите:
		\begin{tasks}(4)
			\task \( 5\% \) от \( 100 \)
			\task \( 10\% \) от \( 6000 \)
			\task \( 37\% \) от \( 500 \)
			\task \( 120\% \) от \( 980 \)
		\end{tasks}
		\item Найдите \( 10\% \) от
		\begin{tasks}(4)
			\task \( 100 \)
			\task \( 600 \)
			\task \( 1420 \)
			\task \( 39810 \)
		\end{tasks} 
	\end{listofex}
		\begin{definit}
			Чтобы быстро найти \( 10\% \) от числа, достаточно разделить его на \( 10 \).
		\end{definit}
	\begin{listofex}[resume]
		\item Найдите \( 10\% \) от
		\begin{tasks}(4)
			\task \( 180 \)
			\task \( 960 \)
			\task \( 1720 \)
			\task \( 100000 \)
		\end{tasks}
		\item Найдите \( 20\% \) от 
		\begin{tasks}(4)
			\task \( 100 \)
			\task \( 300 \)
			\task \( 2050 \)
			\task \( 18505 \)
		\end{tasks}
		\end{listofex}
		\begin{definit}
		Чтобы быстро найти \( 20\% \) от числа, достаточно разделить его на \( 5 \).
		\end{definit}	
		\begin{listofex}[resume]
		\item Найдите \( 20\% \) от
		\begin{tasks}(4)
			\task \( 200 \)
			\task \( 700 \)
			\task \( 555 \)
			\task \( 12345 \)
		\end{tasks}
		\item Найдите \( 25\% \) от
		\begin{tasks}(4)
			\task \( 100 \)
			\task \( 320 \)
			\task \( 4000 \)
			\task \( 8888 \)
		\end{tasks}
	\end{listofex}
		\begin{definit}
		Чтобы быстро найти \( 25\% \) от числа, достаточно разделить его на \( 4 \).
		\end{definit}	
	\begin{listofex}[resume]
		\item Найдите \( 25\% \) от
		\begin{tasks}(4)
			\task \( 28 \)
			\task \( 1200 \)
			\task \( 4820 \)
			\task \( 40000 \)
		\end{tasks}
		\item Найдите \( 50\% \) от
		\begin{tasks}(4)
			\task \( 100 \)
			\task \( 628 \)
			\task \( 4320 \)
			\task \( 89202 \)
		\end{tasks}
	\end{listofex}
		\begin{definit}
			Чтобы быстро найти \( 50\% \) от числа, достаточно разделить его на \( 2 \).
		\end{definit}
		\begin{listofex}[resume]
			\item Найдите \( 50\% \) от
			\begin{tasks}(4)
				\task \( 580 \)
				\task \( 9030 \)
				\task \( 10000 \)
				\task \( 95200 \)
			\end{tasks}
			\item Организм взрослого человека на \( 70\% \) состоит из воды. Какова масса воды в теле человека, который весит \( 76 \) кг?
			\item Металлический конструктор состоит из \( 300 \) деталей. \( 12\% \) этих деталей гайки. Сколько гаек в металлическом конструкторе?
			\item В грушах сладких сортов содержится \( 15\% \) сахара от их массы. Сколько граммов сахара будет содержаться в \( 6 \) кг груш?
			\item Было посажено \( 200 \) деревьев, \( 75\% \) из них прижилось. Сколько деревьев прижилось?
			\item Необходимо было заготовить \( 150 \) м\( ^3 \) дров. За первую неделю было заготовлено \( 23\% \) этого количества. Сколько м3 дров было заготовлено за неделю?
		\end{listofex}
\end{class}
%END_FOLD

%BEGIN_FOLD % ====>>_____ Занятие 2 _____<<====
\begin{class}[number=2]
	\begin{listofex}
		\item В цехе \( 40\% \) рабочих --- женщины. Сколько процентов составляют мужчины?
		\item Число \( 6300 \) увеличили на \( 10\% \). Какое получилось число?
		\item Число \( 5700 \) уменьшили на \( 20\% \), а потом ещё на \( 10\% \). Какое получилось число?
		\item Сбербанк платит вкладчикам \( 2\% \) годовых. Сколько он заплатит за год, если вклад равен \( 700 \) тысяч рублей?
		\item В сбербанке вкладчикам платят \( 2\% \) годовых, а в Тинькофф --- \( 5\% \). На сколько больше заработает вкладчик, если положит \( 700 \) тысяч в Тинькофф, а не в Сбер.
		\item Программа получения кэшбека в банке Тинькофф такова: за обычные покупки начисляется \( 1\% \) кэшбека от общей суммы, за категорию "Развлечения" --- \( 3\% \), а за покупку техники --- \( 5\% \). Суммарно Антон потратил \( 150000 \) рублей, \( 25\% \) от которых на развлечения, \( 20000 \) на еду, а на оставшиеся деньги купил электрогитару. Сколько денег вернулось кэшбеком Антону?
		\item Билет на концерт стоит \( 2400 \) рублей, а стоимость билета в кино составляет \( 20\% \) от стоимости билета на концерт. Сколько стоит билет в кино?
		\item Скорость велосипедиста на \( 100\% \) больше скорости пешехода. Скорость пешехода \( 5 \) км/ч. Какова скорость велосипедиста?
		\item Рабочий должен был сделать за смену \( 200 \) деталей. Но он перевыполнил план на \( 12\% \). Сколько деталей он сделал сверх плана?
		\item \exercise{1306}
	\end{listofex}
\end{class}
%END_FOLD

%BEGIN_FOLD % ====>>_ Домашняя работа 1 _<<====
\begin{homework}[number=1]
	\begin{listofex}
		\item Найдите двумя способами:
		\begin{tasks}(2)
			\task \( 10\% \) от \( 3100 \)
			\task \( 20\% \) от \( 5000 \)
			\task \( 25\% \) от \( 444 \)
			\task \( 50\% \) от \( 18960 \)
		\end{tasks}
		\item Настя потратила в магазине \( 45\% \) своих денег. Найдите потраченную сумму денег, если у нее всего было \( 800  \) рублей.
		\item На грузовике должны были перевезти \( 30 \) тонн груза, но перевыполнили план на \( 15\% \).Сколько тонн груза перевезли на грузовике?
		\item Нужно окрасить \( 60 \) м\( ^2 \) поверхности стены. Сделано \( 75\% \) работы. Какую площадь осталось окрасить?
		\item В школе \( 600 \) учеников, \( 55\% \) из них мальчики. Сколько мальчиков, и сколько девочек обучается в школе?
		\item \exercise{4107}
	\end{listofex}
\end{homework}
%END_FOLD

%BEGIN_FOLD % ====>>_____ Занятие 3 _____<<====
\begin{class}[number=3]
	\begin{listofex}
		\item Найдите
		\begin{tasks}(2)
			\task \( 50\% \) от \( 340 \)
			\task \( 20\% \) от \( 15 300 \)
			\task \( 10\% \) от \( 520 \)
			\task \( 25\% \) от \( 484 \)
		\end{tasks}
		\item Задумано число. \( 84 \) --- составляет \( 14\% \) от этого числа. Найдите задуманное число.
		\item В классе \( 30 \) человек, из них \( 18 \) девочек. Сколько процентов мальчиков в классе?
		\item Бегун пробежал \( 600 \) м, что составляет \( 40\% \) всей его намеченной дистанции. Найдите длину дистанции.
		\item В младших классах учится \( 200 \) учеников, что составляет \( 40\% \) учеников старших классов. Сколько учеников учится в школе?
		\item В книге \( 3 \) главы. Число страниц в первой главе составляет \( 30\% \) всей книги, число страниц второй главы --- \( 45\% \) книги, а в третьей \( 50 \) страниц. Сколько страниц в книге?
		\item  Цена на товар увеличилась на \( 20\% \), а потом ещё на \( 10\% \). Найдите новую цену, если старая составляла \( 400 \) рублей.
		\item  Цена на товар снизилась на \( 5\% \). Найдите новую цену, если прежняя цена составляла \( 200 \) рублей.
		\item Цена на ботинки выросла на \( 30\% \). Сколько стоят ботинки теперь, если раньше они стоили \( 3100 \) рублей?
		\item После увеличения цена на мобильный телефон на \( 10\% \) он стал стоить \( 6600 \) руб. Определите первоначальную цену телефона.
	\end{listofex}
\end{class}
%END_FOLD

%BEGIN_FOLD % ====>>_____ Занятие 4 _____<<====
\begin{class}[number=4]
	\begin{listofex}
		\item В магазин привезли арбузы. В первый день продали \( 25\% \) всех арбузов, во второй \( 55\% \) арбузов, а остальные \( 60 \) кг арбузов в третий день. Сколько всего килограммов арбузов привезли в магазин?
		\item Банкомат берет комиссию в \( 2\% \) от внесенной суммы денег. Сколько денег необходимо опустить в банкомат, чтобы на счет пришло \( 196 \) рублей?
		\item После снижения цены на \( 15\% \) товар стал стоить \( 255 \) рублей. Найдите начальную его цену.
		\item В младших классах учится \( 200 \) учеников, что составляет \( 40\% \) учеников старших классов. Сколько учеников учится в школе?
		\item В книге \( 3 \) главы. Число страниц в первой главе составляет \( 30\% \) всей книги, число страниц второй главы --- \( 45\% \) книги, а в третьей \( 50 \) страниц. Сколько страниц в книге?
		\item  Цена на товар увеличилась на \( 20\% \), а потом ещё на \( 10\% \). Найдите новую цену, если старая составляла \( 400 \) рублей.
		\item  Цена на товар снизилась на \( 5\% \). Найдите новую цену, если прежняя цена составляла \( 200 \) рублей.
		\item Цена на ботинки выросла на \( 30\% \). Сколько стоят ботинки теперь, если раньше они стоили \( 3100 \) рублей?
		\item После увеличения цена на мобильный телефон на \( 10\% \) он стал стоить \( 6600 \) руб. Определите первоначальную цену телефона.
	\end{listofex}
\end{class}
%END_FOLD

%BEGIN_FOLD % ====>>_ Домашняя работа 2 _<<====
\begin{homework}[number=2]
	\begin{listofex}
		\item 31 декабря елка подешевела на \( 40\% \). Найдите новую стоимость елки, если до \( 31 \) числа она стоила \( 2100 \) рублей.
		\item После снижения цены на товар на \( 25\% \), а потом ещё на \( 10\% \), он стал стоить \( 4200 \) рублей. Найдите его первоначальную цену.
		\item  Банкомат берет \( 3\% \) от положенной в него суммы денег. Сколько денег положить в банкомат, чтобы на счету оказалось \( 776 \) рублей?
		\item Вычилсите: \quad \( \mfrac{2}{7}{12}-\mfrac{4}{5}{12}-\left( \mfrac{20}{3}{4}-\mfrac{19}{3}{4} \right) \)
	\end{listofex}
\end{homework}
%END_FOLD

%BEGIN_FOLD % ====>>_____ Занятие 5 _____<<====
\begin{class}[number=5]
	\begin{listofex}
		\item После увеличения цены на мобильный телефон на \( 10\% \) он стал стоить \( 6600 \) руб. Определите первоначальную цену телефона.
		\item  В магазине продаются два блокнота, один стоит \( 300 \) руб, другой \( 400 \) руб. На первый сделали скидку \( 40\% \), на второй \( 30\% \). Какой блокнот стал стоить дешевле?
		\item Прыжок кузнечика с каждым разом увеличивается на \( 50\% \). На сколько он должен прыгнуть первый раз, чтобы за \( 4 \) прыжка удалиться на \( 65 \) метров?
		\item На двух полках \( 55 \) книг. Когда с одной полки переставили на другую \( 10\% \), то на показ стало поровну книг. Сколько книг стояло на каждой полке первоначально?
		\item Константин купил дом за \( 3 \) млн. рублей. После вложений в капитальный ремонт \( 700 \) тыс. руб., он продал дом на \( 60\% \) дороже первоначальной стоимости. Сколько заработал Константин?
		\item Турист проехал на машине \( \dfrac{4}{9} \) пути, что составило \( 200 \) км. Какова длина намеченного пути?
		\item В корзине были яблоки. Сначала в неё положили ещё \( 15 \) яблок, а затем взяли \( \dfrac{1}{7} \) получившихся там яблок. Сколько было яблок в корзине первоначально, если из корзины взяли \( 20 \) яблок ?
		\item Вычислите:
		\[\mfrac{2}{1}{2}\cdot48-\mfrac{3}{2}{3}:\dfrac{1}{18}+\mfrac{5}{5}{12}:\dfrac{7}{36}\]
	\end{listofex}
\end{class}
%END_FOLD


%BEGIN_FOLD % ====>>_ Домашняя работа 3 _<<====
\begin{homework}[number=3]
	\begin{listofex}
		\item В десятой части початка ветвистой кукурузы \(93\) зерна. Сколько зёрен в целом початке?
		\item Перед тем как брокер продал \( \dfrac{1}{6} \) акций своего клиента, у него было \(1800\) акций. Сколько акций у брокера теперь?
		\item В классе  \(18\)  учащихся, отсутствуют семь человек. Какой процент учащихся присутствует?
		\item Кладовщик выдал первому рабочему \(50 \% \) всей имевшейся проволоки, а второму --- \(4\) метра, после чего у него осталось еще \(30\) м проволоки. Сколько проволоки было первоначально?
		\item Чашка, которая стоила \(90\) рублей, продаётся с \(10\)-процентной скидкой. Покупатель отдал кассиру \(1000\) рублей, желая купить столько чашек, на сколько ему хватит денег. Сколько рублей сдачи он должен получить?
		\item Вычислите:
		\[ \left( \mfrac{5}{3}{8} + \mfrac{3}{7}{8} - \mfrac{8}{1}{4} \right) \cdot \dfrac{1}{2} \]
	\end{listofex}
\end{homework}
%END_FOLD

%BEGIN_FOLD % ====>>_____ Занятие 7 _____<<====
\begin{class}[number=7]
	\title{Подготовка к проверочной}
	\begin{listofex}
		\item Вычислить:
		\[\mfrac{1}{5}{12}\cdot2+4\cdot\mfrac{1}{1}{12}+\mfrac{1}{1}{9}\cdot\mfrac{1}{1}{4}\]
		\item Найти любым способом:
		\begin{tasks}(1)
			\task \( 10\% \) от \( 650 \)
			\task \( 20\% \) от \( 3655 \)
			\task \( 25\% \) от \( 264 \)
			\task \( 50\% \) от \( 8522 \)
		\end{tasks}
		\item Протяжённость реки Волга \( 3500 \) км, а реки Амур --- на \( 20\% \) меньше. На сколько километров протяжённость реки Амур меньше протяжённости реки Волга?
		\item Сколько человек было в кино, если \( 5\% \) всех зрителей составляет \( 15 \) человек?
		\item Слесарь и его ученик изготовили \( 1500 \) деталей. Ученик сделал \( 30\% \) всех деталей. Сколько деталей сделал слесарь?
		\item Геологи проделали путь длиной \( 2450 \) км. \( 10\% \) пути они пролетели на самолёте, \( 60\% \) пути проплыли в лодках, а остальную часть прошли пешком. Сколько километров геологи прошли пешком?
		\item Цена на блузку составляла \( 5000 \) руб. Сначала она подешевела на \( 10\% \), а потом ещё на \( 20\% \). Сколько стоит блузка сейчас?
		\item Клиент Тинькофф выбрал для себя следующие категории кэшбека: покупка еды --- \( 1\% \), покупка техники --- \( 5\% \), билеты на концерт --- \( 20\% \). За февраль он закупился едой на \( 15000 \), подарил родителям на годовщину свадьбу классный телевизор за \( 25000 \) и пошёл на концерт Газманова, посвящённый \( 23 \) февраля, за цену в \( 8 \) раз меньше, чем стоил телевизор. Сколько рублей вернулось клиенту на карту? 
	\end{listofex}
\end{class}
%END_FOLD

%BEGIN_FOLD % ====>>_ Проверочная работа _<<====
\begin{exam}
	\begin{listofex}
		\item Вычислить:
		\[\dfrac{1}{2}+\left( \mfrac{5}{1}{6}-\mfrac{3}{3}{4}+\dfrac{1}{2} \right)\cdot\dfrac{10}{23}\]
		\item Найти любым способом:
		\begin{tasks}(1)
			\task \( 10\% \) от \( 2500 \)
			\task \( 20\% \) от \( 12345 \)
			\task \( 25\% \) от \( 28 \)
			\task \( 50\% \) от \( 5000 \)
		\end{tasks}
		\item В школе 2000 учеников. Из них \( 55\% \) --- девочки. Сколько в школе девочек?
		\item Пирог стоил \( 120 \) р. Он подешевел на \( 10\% \). На сколько рублей подешевел пирог?
		\item Автотурист проехал в первый день \( 150 \) км, что составляет \( 10\% \) всего намеченного пути. Какой длины намеченный путь?
		\item За три дня турист прошёл \( 40 \) км. В первый день он прошёл \( 40\% \), а во второй день --- \( 30\% \) всего пути. Сколько километров прошёл турист в третий день?
		\item Клиент Тинькофф выбрал для себя следующие категории кэшбека: покупка цветов --- \( 5\% \), покупка техники --- \( 10\% \), покупка лекарств --- \( 1\% \). В феврале он заболел и потратил \( 1500 \) рублей на лекарства в аптеке, на \( 14 \) февраля подарил своей возлюбленной \( 9 \) роз (стоимость одной розы --- \( 100 \) рублей) и купил себе классный компьютер в \( 100 \) раз дороже, чем букет. Сколько рублей вернулось клиенту на карту? 
	\end{listofex}
\end{exam}
%END_FOLD

%BEGIN_FOLD % ====>>_ Консультация _<<====
\begin{consultation}
	\begin{listofex}
		\item Вычислите: \quad \( \left( \dfrac{5}{7}\cdot\mfrac{2}{1}{3}\cdot\dfrac{5}{6}-1 \right):1-\dfrac{7}{8}\cdot\mfrac{1}{3}{5}\cdot\dfrac{3}{14} \)
		\item Ребро куба равно \( 14 \). Найдите его объём.
		\item Ширина прямоугольного параллелепипеда, равная \( 54 \) см, на \( 16 \) см больше его высоты и в \( 6 \) раз больше его длины. Найдите объём параллелепипеда.
		\item Аквариум имеет размеры \( 60 \)Х\( 20 \)Х\( 50 \) см. Сколько литров воды нужно влить в аквариум, чтобы уровень воды был ниже верхнего края аквариума на \( 10 \) см? (\( 1 \) л =\( 1000 \) см\( ^3 \))
		\item Два внешних угла при вершинах треугольника равны \( 140\degree \) и \( 150\degree \). Найдите все внутренние углы треугольника.
	\end{listofex}
	\newpage
	\title{Домашняя работа}
	\begin{listofex}
	\item Ребро куба равно \( 20 \). Найдите его объём.
	\item Ширина прямоугольного параллелепипеда, равная \( 64 \) см, на \( 3 \) см больше его высоты и в \( 8 \) раз больше его длины. Найдите объём параллелепипеда.
	\item Аквариум имеет размеры \( 40 \)Х\( 20 \)Х\( 30 \) см. Сколько литров воды нужно влить в аквариум, чтобы уровень воды был ниже верхнего края аквариума на \( 10 \) см? (\( 1 \) л =\( 1000 \) см\( ^3 \))
	\item Два внешних угла при вершинах треугольника равны \( 100\degree \) и \( 90\degree \). Найдите все внутренние углы треугольника.
	\item Вычислите: \quad \( \mfrac{2}{3}{5}:\mfrac{6}{1}{5}+\mfrac{1}{1}{14}-\mfrac{1}{39}{73}\cdot\left( \mfrac{5}{5}{7}-\mfrac{5}{1}{16} \right) \)
	\end{listofex}
\end{consultation}
%END_FOLD

%BEGIN_FOLD % ====>>_ Консультация _<<====
\begin{consultation}
	\begin{listofex}
		\item Вычислите: 
		\begin{tasks}(1)
			\task \( \left( 13-\mfrac{8}{5}{12} \right)+\left( \mfrac{17}{1}{2}-\mfrac{16}{1}{5} \right) \)
			\task \( \left( \mfrac{63}{2}{3}+\mfrac{3}{1}{8} \right)-\left( 13-\mfrac{10}{5}{9} \right) \)
			\task \( \mfrac{2}{1}{3}\cdot\mfrac{3}{1}{5}-\mfrac{8}{4}{9}:2 \)
			\task \( \left( \dfrac{5}{7}\cdot\mfrac{2}{1}{3}\cdot\dfrac{5}{6}-1 \right):1-\dfrac{7}{8}\cdot\mfrac{1}{3}{5}\cdot\dfrac{3}{14} \)
		\end{tasks}
	\end{listofex}
	\newpage
	\title{Домашняя работа}
	\begin{listofex}
		\item Вычислите: \quad \( \mfrac{2}{3}{5}:\mfrac{6}{1}{5}+\mfrac{1}{1}{14}-\mfrac{1}{39}{73}\cdot\left( \mfrac{5}{5}{7}-\mfrac{5}{1}{16} \right) \)
	\end{listofex}
\end{consultation}
%END_FOLD

%BEGIN_FOLD % ====>>_ Консультация _<<====
\begin{consultation}
	\begin{listofex}
		\item Выполните умножение:
		\begin{tasks}(4)
			\task \( \dfrac{9}{10}\cdot1 \)
			\task \( \dfrac{4}{5}\cdot10 \)
			\task \( \dfrac{3}{5}\cdot\dfrac{5}{9} \)
			\task \( \dfrac{1}{2}\cdot\dfrac{1}{7} \)
			\task \( \dfrac{3}{7}\cdot\dfrac{1}{2} \)
			\task \( \dfrac{6}{7}\cdot\dfrac{2}{3} \)
			\task \( \dfrac{2}{4}\cdot\dfrac{15}{16} \)
			\task \( \dfrac{3}{5}\cdot\dfrac{5}{7} \)
		\end{tasks}
		\item Переведите смешанное число в неправильную дробь:
		\begin{tasks}(4)
			\task \( \mfrac{3}{1}{2} \)
			\task \( \mfrac{5}{4}{7} \)
			\task \( \mfrac{7}{1}{5} \)
			\task \( \mfrac{6}{1}{2} \)
			\task \( \mfrac{9}{1}{3} \)
			\task \( \mfrac{10}{10}{11} \)
			\task \( \mfrac{7}{4}{9} \)
			\task \( \mfrac{1}{1}{12} \)
		\end{tasks}
		\item Переведите неправильную дробь в смешанное число:
		\begin{tasks}(4)
			\task \( \dfrac{8}{2} \)
			\task \( \dfrac{7}{3} \)
			\task \( \dfrac{5}{2} \)
			\task \( \dfrac{8}{7} \)
			\task \( \dfrac{10}{3} \)
			\task \( \dfrac{9}{4} \)
			\task \( \dfrac{15}{13} \)
			\task \( \dfrac{20}{7} \)
		\end{tasks}
	\end{listofex}
	\newpage
	\title{Домашняя работа}
	\begin{listofex}
		\item Выполните умножение:
		\begin{tasks}(4)
			\task \( \dfrac{6}{17}\cdot51 \)
			\task \( \dfrac{3}{4}\cdot\dfrac{5}{12} \)
			\task \( \dfrac{2}{3}\cdot\dfrac{9}{10} \)
			\task \( \dfrac{4}{5}\cdot\dfrac{7}{12} \)
		\end{tasks}
		\item Переведите смешанное число в неправильную дробь:
		\begin{tasks}(4)
			\task \( \mfrac{4}{1}{5} \)
			\task \( \mfrac{8}{9}{10} \)
			\task \( \mfrac{3}{4}{7} \)
			\task \( \mfrac{12}{1}{12} \)
		\end{tasks}
		\item Переведите неправильную дробь в смешанное число:
		\begin{tasks}(4)
			\task \( \dfrac{7}{2} \)
			\task \( \dfrac{13}{5} \)
			\task \( \dfrac{99}{10} \)
			\task \( \dfrac{200}{3} \)
		\end{tasks}
	\end{listofex}
\end{consultation}
%END_FOLD

%BEGIN_FOLD % ====>>_____ Занятие 8 _____<<====
\begin{class}[number=8]
	\begin{listofex}
		\item Представить обыкновенную дробь в виде десятичной
		\begin{tasks}(7)
			\task \( \dfrac{6}{10} \)
			\task \( \dfrac{9}{10} \)
			\task \( \dfrac{14}{100} \)
			\task \( \dfrac{14}{1000} \)
			\task \( \dfrac{999}{1000} \)
			\task \( \dfrac{120}{1000} \)
			\task \( \dfrac{1000}{10000} \)
		\end{tasks}
		\item Представить дробь в виде разрядных слагаемых и далее в виде десятичной дроби:\\
		\fbox{\textbf{\textit{Пример:}} \( \dfrac{37}{100}=\dfrac{30}{100}+\dfrac{7}{100}=\dfrac{3}{10}+\dfrac{7}{100}=0,3+0,07=0,37 \)}
		\begin{tasks}(7)
			\task \( \dfrac{56}{100} \)
			\task \( \dfrac{112}{1000} \)
			\task \( \dfrac{543}{10000} \)
			\task \( \dfrac{37}{10} \)
			\task \( \dfrac{479}{10} \)
			\task \( \dfrac{5488}{10} \)
			\task \( \dfrac{5488}{100} \)
		\end{tasks}
		\item Представьте десятичную дробь в виде разрядных слагаемых и далее переведите ее в обыкновенную дробь:\\
		\fbox{\textbf{\textit{Пример:}} \( 1,24=1+0,2+0,04=1+\dfrac{2}{10}+\dfrac{4}{100}=\dfrac{100}{100}+\dfrac{20}{100}+\dfrac{4}{100}=\dfrac{124}{100} \)}
		\begin{tasks}(6)
			\task \( 1,44 \)
			\task \( 2,37 \)
			\task \( 0,199 \)
			\task \( 12,13 \)
			\task \( 15,006 \)
			\task \( 30,03 \)
		\end{tasks}
		\item Вычислить:
		\begin{tasks}(2)
			\task \( 1,2+0,24 \)
			\task \( 23,55+132,05 \)
			\task \( 66,22-37,12 \)
			\task \( 124,05+25,1-67,15 \)
			\task \( 292,04-(25,55+197,4) \)
			\task \( (14,37+67)+(94,87-32,87) \)
		\end{tasks}
		\item Представить обыкновенную дробь в виде десятичной:
		\begin{tasks}(5)
			\task \( \dfrac{1}{2} \)
			\task \( \dfrac{1}{4} \)
			\task \( \dfrac{2}{5} \)
			\task \( \dfrac{3}{4} \)
			\task \( \dfrac{15}{4} \)
			\task \( \dfrac{13}{25} \)
			\task \( \dfrac{33}{50} \)
			\task \( \dfrac{19}{20} \)
			\task \( \dfrac{37}{20} \)
			\task \( \dfrac{1}{8} \)
		\end{tasks}
		\item Сократить и представить в виде десятичной:
		\begin{tasks}(6)
			\task \( \dfrac{3}{12} \)
			\task \( \dfrac{9}{75} \)
			\task \( \dfrac{30}{24} \)
			\task \( \dfrac{12}{60} \)
			\task \( \dfrac{33}{44} \)
			\task \( \dfrac{33}{150} \)
		\end{tasks}
		\item Вычислить:
		\begin{tasks}(3)
			\task \( \dfrac{7}{10}+15,3 \)
			\task \( 2,4+\dfrac{13}{100} \)
			\task \( \dfrac{1}{2}+16,7 \)
			\task \( \dfrac{3}{4}-0,1 \)
			\task \( \mfrac{3}{1}{50}+4,98 \)
			\task \( \mfrac{5}{13}{25}-4,12 \)
		\end{tasks}
		\item Сравнить:
		\begin{tasks}(3)
			\task \( 0,2 \) и \( 0,3 \)
			\task \( 0,2 \) и \( 0,18 \)
			\task \( 0,35 \) и \( 0,350 \)
			\task \( 1,135 \) и \( 1,153 \)
			\task \( 37,506 \) и \( 37,560 \)
			\task \( 12,1212 \) и \( 12,2112 \)
		\end{tasks}
	\end{listofex}
\end{class}
%END_FOLD