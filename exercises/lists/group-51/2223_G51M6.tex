%
%===============>>  ГРУППА 5-1 МОДУЛЬ 6  <<=============
%
\setmodule{6}

%BEGIN_FOLD % ====>>_____ Занятие 1 _____<<====
\begin{class}[number=1]
		\begin{definit}
			Процент --- это \( \dfrac{1}{100} \) от числа. Например, \( 1\% \) от \( 300 \) --- это \( \dfrac{1}{100}\cdot300=3 \).
		\end{definit}
		\begin{listofex}
		\item Найдите один процент от:
		\begin{tasks}(4)
			\task \( 200 \)
			\task \( 1000 \)
			\task \( 350 \)
			\task \( 20 \)
		\end{tasks}
		\item Найдите:
		\begin{tasks}(1)
			\task \( 5\% \) от \( 100 \)
			\task \( 10\% \) от \( 6000 \)
			\task \( 37\% \) от \( 500 \)
			\task \( 120\% \) от \( 980 \)
		\end{tasks}
		\item Найдите \( 10\% \) от
		\begin{tasks}(4)
			\task \( 100 \)
			\task \( 600 \)
			\task \( 1420 \)
			\task \( 39810 \)
		\end{tasks} 
	\end{listofex}
		\begin{definit}
			Чтобы быстро найти \( 10\% \) от числа, достаточно разделить его на \( 10 \).
		\end{definit}
	\begin{listofex}[resume]
		\item Найдите \( 10\% \) от
		\begin{tasks}(4)
			\task \( 180 \)
			\task \( 960 \)
			\task \( 1720 \)
			\task \( 100000 \)
		\end{tasks}
		\item Найдите \( 20\% \) от 
		\begin{tasks}(4)
			\task \( 100 \)
			\task \( 300 \)
			\task \( 2050 \)
			\task \( 18505 \)
		\end{tasks}
		\end{listofex}
		\begin{definit}
		Чтобы быстро найти \( 20\% \) от числа, достаточно разделить его на \( 5 \).
		\end{definit}	
		\begin{listofex}[resume]
		\item Найдите \( 20\% \) от
		\begin{tasks}(4)
			\task \( 200 \)
			\task \( 700 \)
			\task \( 555 \)
			\task \( 12345 \)
		\end{tasks}
		\item Найдите \( 25\% \) от
		\begin{tasks}(4)
			\task \( 100 \)
			\task \( 320 \)
			\task \( 4000 \)
			\task \( 8888 \)
		\end{tasks}
	\end{listofex}
		\begin{definit}
		Чтобы быстро найти \( 25\% \) от числа, достаточно разделить его на \( 4 \).
		\end{definit}	
	\begin{listofex}[resume]
		\item Найдите \( 25\% \) от
		\begin{tasks}(4)
			\task \( 28 \)
			\task \( 1200 \)
			\task \( 4820 \)
			\task \( 40000 \)
		\end{tasks}
		\item Найдите \( 50\% \) от
		\begin{tasks}(4)
			\task \( 100 \)
			\task \( 628 \)
			\task \( 4320 \)
			\task \( 89202 \)
		\end{tasks}
	\end{listofex}
		\begin{definit}
			Чтобы быстро найти \( 50\% \) от числа, достаточно разделить его на \( 2 \).
		\end{definit}
		\begin{listofex}[resume]
			\item Найдите \( 50\% \) от
			\begin{tasks}(4)
				\task \( 580 \)
				\task \( 9030 \)
				\task \( 10000 \)
				\task \( 95200 \)
			\end{tasks}
			\item Организм взрослого человека на \( 70\% \) состоит из воды. Какова масса воды в теле человека, который весит \( 76 \) кг?
			\item Металлический конструктор состоит из \( 300 \) деталей. \( 12\% \) этих деталей гайки. Сколько гаек в металлическом конструкторе?
			\item В грушах сладких сортов содержится \( 15\% \) сахара от их массы. Сколько кг сахара будет содержаться в \( 6 \) кг груш?
			\item Было посажено \( 200 \) деревьев, \( 80\% \) из них прижилось. Сколько деревьев прижилось?
			\item Необходимо было заготовить \( 150 \) м\( ^3 \) дров. За первую неделю было заготовлено \( 23\% \) этого количества. Сколько м3 дров было заготовлено за неделю?
		\end{listofex}
\end{class}
%END_FOLD

%BEGIN_FOLD % ====>>_____ Занятие 2 _____<<====
\begin{class}[number=2]
	\begin{listofex}
		\item В цехе \( 40\% \) рабочих --- женщины. Сколько процентов составляют мужчины?
		\item Число \( 6300 \) увеличили на 10\%. Какое получилось число?
		\item Число \( 5700 \) уменьшили на \( 20\% \). Какое получилось число?
		\item Сбербанк платит вкладчикам \( 2\% \) годовых. Сколько он заплатит за год, если вклад равен \( 700 \) рублей?
		\item Билет на концерт стоит \( 2400 \) рублей, а стоимость билета в кино составляет \( 20\% \) от стоимости билета на концерт. Сколько стоит билет в кино?
		\item Скорость велосипедиста на \( 100\% \) больше скорости пешехода. Скорость пешехода \( 5 \) км/ч. Какова скорость велосипедиста?
		\item Рабочий должен был сделать за смену \( 200 \) деталей. Но он перевыполнил план на \( 12\% \). Сколько деталей он сделал сверх плана?
		\item \exercise{1306}
	\end{listofex}
\end{class}
%END_FOLD

%BEGIN_FOLD % ====>>_ Домашняя работа 1 _<<====
\begin{homework}[number=1]
	\begin{listofex}
		\item Найдите двумя способами:
		\begin{tasks}(2)
			\task \( 10\% \) от \( 3100 \)
			\task \( 20\% \) от \( 5000 \)
			\task \( 25\% \) от \( 444 \)
			\task \( 50\% \) от \( 18960 \)
		\end{tasks}
		\item Настя потратила в магазине \( 45\% \) своих денег. Найдите потраченную сумму денег, если у нее всего было \( 800  \) рублей.
		\item На грузовике должны были перевезти \( 30 \) тонн груза, но перевыполнили план на \( 15\% \).Сколько тонн груза перевезли на грузовике?
		\item Нужно окрасить \( 60 \) м\( ^2 \) поверхности стены. Сделано \( 75\% \) работы. Какую площадь осталось окрасить?
		\item В школе \( 600 \) учеников, \( 55\% \) из них мальчики. Сколько мальчиков, и сколько девочек обучается в школе?
		\item \exercise{4107}
	\end{listofex}
\end{homework}
%END_FOLD

%BEGIN_FOLD % ====>>_____ Занятие 3 _____<<====
\begin{class}[number=3]
	\begin{listofex}
		\item Занятие 3 
	\end{listofex}
\end{class}
%END_FOLD

%BEGIN_FOLD % ====>>_____ Занятие 4 _____<<====
\begin{class}[number=4]
	\begin{listofex}
		\item Занятие 4
	\end{listofex}
\end{class}
%END_FOLD

%BEGIN_FOLD % ====>>_ Домашняя работа 2 _<<====
\begin{homework}[number=2]
	\begin{listofex}
		\item Домашняя работа 2
	\end{listofex}
\end{homework}
%END_FOLD

%BEGIN_FOLD % ====>>_____ Занятие 5 _____<<====
\begin{class}[number=5]
	\begin{listofex}
		\item Занятие 5
	\end{listofex}
\end{class}
%END_FOLD

%BEGIN_FOLD % ====>>_____ Занятие 6 _____<<====
\begin{class}[number=6]
	\begin{listofex}
		\item Занятие 6
	\end{listofex}
\end{class}
%END_FOLD

%BEGIN_FOLD % ====>>_ Домашняя работа 3 _<<====
\begin{homework}[number=3]
	\begin{listofex}
		\item Домашняя работа 3
	\end{listofex}
\end{homework}
%END_FOLD

%BEGIN_FOLD % ====>>_____ Занятие 7 _____<<====
\begin{class}[number=7]
	\title{Подготовка к проверочной}
	\begin{listofex}
		\item Занятие 7
	\end{listofex}
\end{class}
%END_FOLD

%BEGIN_FOLD % ====>>_ Проверочная работа _<<====
\begin{exam}
	\begin{listofex}
		\item Проверочная
	\end{listofex}
\end{exam}
%END_FOLD