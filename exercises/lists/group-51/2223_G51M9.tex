%
%===============>>  ГРУППА 5-1 МОДУЛЬ 8  <<=============
%
\setmodule{9}

%BEGIN_FOLD % ====>>_____ Занятие 1 _____<<====
\begin{class}[number=1]
	\begin{listofex}
		\item Выполните деление:\begin{tasks}(3)
			\task \( 3,24:0,6 \)
			\task \( 4,96 : 0,8 \)
			\task \( 24,804 : 4,77 \)
		\end{tasks}
		\item Решите уравнение \begin{tasks}(2)
			\task \( 16-3,8x = 6,31 \)
			\task \( 13 -7,6x = 8,136 \)
		\end{tasks} 
		\item Округлить:
		\begin{tasks}(1)
			\task до десятых: \( 8,263 \);   \( 12,4398 \);    \( 0,55112 \)
			\task до сотых: \( 3,274 \);   \( 11,958 \);   \( 9,097 \)
			\task до единиц: \( 35,24 \);   \( 41,096 \);   \( 125,608 \)
			\task до сотен: \( 2345,872 \);   \( 35987,9842 \);   \( 3487,842 \)
		\end{tasks} 
		\item Найдите среднее арифметическое чисел \( 0,2 \); \( 5,4 \) и \( 6,1 \).
		\item Среднее арифметическое трех чисел равно 3. Одно число равно \( 2,4 \); второе --- \( 3,6 \). Найдите третье число.
		\item Участница соревнований по фигурному катанию получила оценки \( 5,3 \); \( 4,8 \); \( 5,4 \); \(  5,3 \);  \( 5,2 \); \(  5,1 \). Найдите среднюю оценку этой участницы.                                                                                                                                            
		\item Один участник соревнований по стрельбе стрелял четыре раза и попал в \( 7 \); \( 9 \); \( 10 \); \( 4 \) очка. А второй стрелял пять раз и попал в \( 6 \); \( 7 \); \( 8 \); \( 5 \); \( 6 \) очков. Какой участник победил?    
		\item Мальчики решили устроить турнир по меткой стрельбе в тире. У Вани было с собой \( 19 \) рублей, у Саши \( 17 \), а у Сергея \( 18 \). Один выстрел в тире стоил \( 3 \) рубля. Ребята решили разделить деньги поровну, чтобы каждый сделал одинаковое количество выстрелов. Какое количество выстрелов совершил каждый участник импровизированного турнира? 
		\item Машина ехала час со скоростью 60 км/ч. и два часа со скоростью \( 90 \) км/ч. С какой постоянной скоростью должна ехать машина, чтобы проехать то же самое расстояние за \( 3 \) часа? 
		\item У Игоря было с собой \( 45 \) рублей, у Андрея \( 28 \), а у Дениса \( 17 \). На все свои деньги они купили \( 3 \) билета в кино. Сколько стоил один билет?  
		\item Из двух городов, расстояние между которыми равно \( 855 \) км, выехали друг другу навстречу два автомобиля и встретились через \( 4,5 \) ч. С какой скоростью они идут, если первый автомобиль, скорость которого на 8 км/ч больше второго, вышел на час раньше второго?
		\item Катер по течению реки прошёл \( 87,5 \) км за \( 5 \) ч, а против течения это же расстояние он прошёл за \( 7 \) ч. Какова собственная скорость катера и скорость течения реки?
		\item Из пункта \( A \) в пункт \( B \) по реке отплыл плот. Одновременно с ним из пункта \( B \) в пункт \( A \) вышел катер. Через сколько часов после выхода катер встретил плот, если катер прошёл всё расстояние между \( A \) и \( B \) за \( 6 \) ч, а плот --- за \( 30 \) ч?
		\item Поезд, двигаясь равномерно со скоростью \( 54 \) км/ч, проезжает мимо идущего параллельно путям со скоростью \( 6 \) км/ч навстречу ему пешехода за \( 30 \) секунд. Найдите длину поезда в метрах.
	\end{listofex}
\end{class}
%END_FOLD

%BEGIN_FOLD % ====>>_____ Занятие 2 _____<<====
\begin{class}[number=2]
	\begin{listofex}
		\item Решите уравнения \begin{tasks}(2)
			\task \( x : 4,6 = 2,3 \)
			\task \( x : 6,8 = 3,4 \)
			\task \( 11,88 : (y - 2,9) = 2,7 \)
			\task \( 8,19x - 3,84x - 1,85x = 19,5\)
		\end{tasks}
		\item Найди среднее арифметическое чисел: \( 23,5 \); \( 20,3 \); \( 22,2 \).
		\item Найди среднее арифметическое первых \( 7 \) натуральных чисел.
		\item Найди среднее арифметическое первых \( 50 \) натуральных чисел.
		\item В волейбольной команде двум игрокам по \( 21 \) году, трем – по \( 20 \) лет и одному \( 24 \) года. Каков средний возраст игроков?
		\item Среднее арифметическое четырех чисел равно \( 2,765 \), а среднее арифметическое других трех чисел равно \( 3,01 \). Вычислите среднее арифметическое этих семи чисел.
		\item У торговца сладостями было \( 10 \) кг конфет по цене \( 60 \) рублей за килограмм и \( 15  \) кг конфет по цене \( 40 \) рублей за килограмм. Мешки с конфетами разорвались и все конфеты перемешались. По какой цене ему продавать полученную смесь?
		\item Средний рост пяти баскетболистов равен \( 195 \) см. Какое наибольшее количество из этих игроков может быть ниже, чем \( 191 \) см?
		\item В овощном магазине за \( 3 \) дня было продано \( 680 \) кг картофеля. В первый день продали \( 28\%  \) всего картофеля, во второй \( 25\% \) от остатка. Сколько кг картофеля было продано в третий день?
		\item На ферме было \( 200 \) животных, из них \( 43\% \) составляли овцы. Сколько овец было на ферме? 
		\item В растворе содержится \( 42 \) кг соли. Чему равна масса раствора, если масса соли в нём составляет \( 35\% \)?
	\end{listofex}
\end{class}
%END_FOLD

%BEGIN_FOLD % ====>>_ Домашняя работа 1 _<<====
\begin{homework}[number=1]
	\begin{listofex}
		\item Велосипедист двигался два часа со скоростью \( 10 \) км/ч  и три часа со скоростью \( 15 \) км/ч. С какой постоянной скоростью должен  ехать велосипедист, чтобы преодолеть то же самое расстояние за то же время, \( 5 \) часов?    
		\item Решите уравнения:
		\begin{tasks}(2)
			\task \( 7x +3 - 4x = 6 \)
			\task \( 5x +3x - 28 = 12 \)
		\end{tasks}
		\item Рабочий должен был сделать за смену \( 200 \) деталей. Но он перевыполнил план на \( 12\% \). Сколько деталей он сделал сверх плана?
		\item На грузовике должны были перевезти \( 30 \) тонн груза, но перевыполнили план на \( 15\% \).Сколько тонн груза перевезли на грузовике?
		\item Найдите среднее арифметическое первых десяти натуральных чисел.
		\item Чему равно среднее арифметическое чисел: \( 23,5 \); \( 20,3 \); \( 22,2 \)?
		\item В футбольной команде двум игрокам по \( 18 \) лет, трем – по \( 22 \) года и одному \( 24 \) года. Каков средний возраст игроков?
	\end{listofex}
\end{homework}
%END_FOLD

%BEGIN_FOLD % ====>>_____ Занятие 3 _____<<====
\begin{class}[number=3]
	\begin{listofex}
		\item Вычислите:\begin{tasks}(4)
			\task \( \dfrac{2}{5}\cdot\dfrac{3}{7} \)
			\task \( \dfrac{3}{4}\cdot\dfrac{5}{11} \)
			\task \( \dfrac{2}{9}\cdot\dfrac{11}{7} \)
			\task \( \dfrac{4}{5}\cdot\dfrac{6}{19} \)
			\task \( \dfrac{2}{5}\cdot\dfrac{5}{7} \)
			\task \( \dfrac{5}{6}\cdot\dfrac{2}{15} \)
			\task \( \dfrac{7}{12}\cdot\dfrac{8}{9} \)
			\task \( \dfrac{6}{25}\cdot\dfrac{15}{17} \)
		\end{tasks}
		\item Найдите решение уравнений: \begin{tasks}(2)
			\task \( x +14 = 35 \) 
			\task \( 32 + x = 56 \) 
			\task \(  x + 41 = 80 \) 
			\task \( 52 + x = 93 \) 
			\task \( 132 + x = 306 \) 
			\task \( 232 + x = 403  \)   
		\end{tasks}
		\item Найдите решение уравнений: \begin{tasks}(2)
			\task \( x - 23 = 45 \) 
			\task \( x - 35 = 44 \) 
			\task \( x - 61 = 32 \) 
			\task \( x - 83 = 98\) 
			\task \( x - 142 = 339\) 
			\task \( x - 212 = 437  \)     
		\end{tasks}
		\item Найдите решение уравнений: \begin{tasks}(2)
			\task \( (128 + 49) - x = 28 \)
			\task \( x - (133 + 75) = 32 \)
			\task \( 145 - (x + 45) = 50 \)
			\task \( (39 + x) - 27 = 22 \)
			\task \( 500 - (120 - x) = 479-99 \)
			\task \( 220 + (x - 120) =997 -736 \)
		\end{tasks}
		\item Решите уравнения:
		\begin{tasks}(2)
			\task \( x+\mfrac{2}{7}{16}=\mfrac{5}{3}{16} \)
			\task \( x-\mfrac{2}{1}{4}= \mfrac{6}{3}{4}\)
		\end{tasks}
		\item Света задумала число, умножила его на \( 4 \) и к произведению прибавила \( 8 \). В результате она получила \( 60 \). Какое число задумала Света?
		\item Собрали несколько килограммов свежей вишни. После того, как из \( 7 \) кг сварили варенье, а затем собрали ещё \( 5 \) кг, то свежей вишни стало \( 10 \) кг. Сколько вишни собрали изначально?
		\item В одной корзине в \( 6 \) раз меньше яблок, чем в другой. Сколько яблок в каждой корзине, если в двух корзинах \( 98 \) яблок?
	\end{listofex}
\end{class}
%END_FOLD

%BEGIN_FOLD % ====>>_____ Занятие 4 _____<<====
\begin{class}[number=4]
	\begin{listofex}
		\item а
	\end{listofex}
\end{class}
%END_FOLD

%BEGIN_FOLD % ====>>_ Домашняя работа 2 _<<====
\begin{homework}[number=2]
	\begin{listofex}
		\item Найдите решение уравнений: \begin{tasks}(2)
			\task \( x - 1,7=2,2 \)
			\task \( 6,4x+ 13,6 = 43,2 \)
			\task \( k : 1,6 - 10,9 = 23,1 \)
		\end{tasks}
		\item Собственная скорость лодки \( 8,5 \) км/ч, а скорость течения \( 3,5 \) км/ч. Расстояние между пристанями \( 15 \) км. Сколько времени затратит лодка на путь между пристанями туда и обратно?
		\item Площадь прямоугольника равна \( 5,12 \) м\( ^{2} \), а одна из его сторон равна \( 3,2 \) м. Найдите периметр прямоугольника.
		\item Два велосипедиста выехали одновременно из двух сёл навстречу друг другу и встретились через \( 1,6 \) ч. Скорость первого \( 10 \) км/ч, а второго --- \( 12 \) км/ч. Найдите расстояние между сёлами?
		\item За день туристы прошли \( 15 \) км. После обеда они прошли в \( 4 \) раза меньше, чем до обеда. Сколько километров прошли туристы после обеда?
	\end{listofex}
\end{homework}
%END_FOLD

%BEGIN_FOLD % ====>>_____ Занятие 5 _____<<====
\begin{class}[number=5]
	\begin{listofex}
		\item  Найдите скорость автомобиля, если за \( 3 \) часа он проехал \( 226,5 \) км.
		\item За какое время проедет расстояние \( 266,5 \) автомобиль со скоростью \( 65 \) км/ч? Ответ дайте в часах и минутах.
		\item Катер, собственная скорость которого \( 14,8 \) км/ч, шёл \( 3 \) часа по течению реки и \( 4 \) ч против течения реки. Какой путь проделал катер за всё это время, если скорость течения реки \( 2,3 \) км/ч?
		\item Из города \( N \) выехал велосипедист со скоростью \( 13,4 \) км/ч. Через \( 2 \) часа вслед за ним выехал другой велосипедист, скорость которого равна \( 17,4 \) км/ч. Через какое время после своего выезда второй велосипедист догонит первого? На каком расстоянии от города \( N \) это произойдёт?
		\item Пароход проплыл \( 74,58 \) км по течению реки и \( 131,85 \) км против течения реки. Сколько времени пароход был в пути, если его собственная скорость равна \( 31,6 \) км/ч, а скорость течения \( 2,3 \) км/ч?
		\item Автомобиль выехал из пункта \( A \) со скоростью \( 60 \) км/ч. Через \( 2 \) ч вслед за ним выехал второй автомобиль со скоростью \( 90 \) км/ч. Через какое время и на каком расстоянии от \( A \) второй автомобиль догонит первый?
		\item Собственная скорость катера \( 25,5 \) км/ч, скорость течения \( 2,5 \) км/ч. Какой путь пройдёт 
		катер за полтора часа по течению и \( 2 \) часа против течения?
		\item  Плот и лодка движутся навстречу друг другу по реке. Они находятся на расстоянии \( 20 \) км друг другу по реке. Через какое время они встретятся, если собственная скорость лодки \( 8 \) км/ч, а скорость течения реки \( 2 \) км/ч?
		\item Два велосипедиста одновременно выехали из лагеря в противоположных направлениях со скоростями \( 10 \) км/ч и \( 12 \) км/ч. Какое расстояние будет между ними через \( 2 \) ч? Через \( 3 \) ч \( 6 \) минут? Через какое время расстояние между ними будет равно \( 33 \) км?
		\item Два велосипедиста выехали одновременно из двух сёл навстречу друг другу и встретились через \( 1,6  \) ч. Скорость первого \( 10 \) км/ч, а второго --- \( 12  \) км/ч. Найдите расстояние между сёлами?
		\item  Два поезда одновременно вышли с одной станции в одном направлении. Их скорости \( 60 \) км/ч и \( 70 \) км/ч. Какое расстояние будет между ними через \( 1,5 \) часа? Через \( 2  \) часа \( 25  \) мин? Через сколько часов расстояние между ними будет равно \( 35 \) км?
		\item Города \( A \) и \( B \) расположены на реке, причём \( B \) ниже по течению. Расстояние между 
		ними равно \( 30 \) км. Моторная лодка проходит путь от \( A \) до \( B \) за \( 2 \) ч, а обратно за \( 3 \) ч. За какое время проплывёт от \( A \) до \( B \) плот?
		\item  Пассажир метро, стоя на ступеньке эскалатора, поднимается наверх за \( 3 \) мин. За сколько 
		минут он поднимется вверх по движущемуся эскалатору, если будет идти со скоростью \( 25 \) м/мин? Длина эскалатора \( 150 \) м.
	\end{listofex}
\end{class}
%END_FOLD

%BEGIN_FOLD % ====>>_____ Занятие 6 _____<<====
\begin{class}[number=6]
	\begin{listofex}
		\item Занятие 6
	\end{listofex}
\end{class}
%END_FOLD

%BEGIN_FOLD % ====>>_ Домашняя работа 3 _<<====
\begin{homework}[number=3]
	\begin{listofex}
		\item Скорость течения реки \( 2,87 \) км/ч. Скорость моторной лодки по течению \( 14,42 \) км/ч. Определите собственную скорость моторной лодки.
		\item Собственная скорость лодки \( 12,5 \) км/ч. Скорость течения реки \( 3,02 \) км/ч. Найдите сколько километров прошла лодка по течению реки за 3 часа.
		\item  Найдите скорость автомобиля, если за \( 4 \) часа он проехал \( 261,08 \) км.
		\item Туристы в первый день прошли \( 25 \) км за \( 5 \) часов, а во второй день \( 28 \) км за \( 4 \) часа. На сколько изменилась их скорость во второй день?
		\item Расстояние между городами \( 930 \) км. Одновременно навстречу друг другу вышли \( 2 \) поезда. Скорость одного \( 45 \) км/ч, другого --- \( 48 \) км/ч. Через сколько часов поезда встретились?
	\end{listofex}
\end{homework}
%END_FOLD

%BEGIN_FOLD % ====>>_____ Занятие 7 _____<<====
\begin{class}[number=7]
	\begin{listofex}
		\item а
	\end{listofex}
\end{class}
%END_FOLD

%BEGIN_FOLD % ====>>_ Проверочная работа _<<====
\begin{exam}
	\begin{listofex}
		\item а
	\end{listofex}
\end{exam}
%END_FOLD

%BEGIN_FOLD % ====>>_ Консультация _<<====
\begin{consultation}
	\begin{listofex}
		\item Скорость поезда \( 67 \) км/ч. Какое расстояние поезд проедет за \( 7 \)ч?
		\item Самолет за \( 4 \) ч пролетел \( 2992 \) км. Какова скорость самолета?
		\item Сколько времени понадобится велосипедисту, чтобы проехать \( 78 \) км со скоростью \( 13 \) км/ч?
		\item Собственная скорость лодки \( 15 \) км/ч. Скорость течения реки \( 3 \)км/ч. Чему равна скорость лодки по течению реки?
		\item Скорость теплохода по течению реки \( 28 \) км/ч. Собственная скорость теплохода \( 26 \) км/ч. Какова скорость течения реки?
		\item Мотоциклист \( 2 \) ч ехал со скоростью \( 58 \) км/ч, а потом \( 4 \) ч со скоростью \( 65 \) км/ч. Какое расстояние проехал мотоциклист за это время?
		\item Пять землекопов за \( 5 \) часов выкапывают \( 5 \) метров канавы. Сколько землекопов выкопают \( 100 \) метров канавы за \( 100 \) часов?
		\item Часы бьют три, и, пока они бьют, проходит \( 3 \) секунды. Сколько времени пройдет, пока часы будут бить семь.
		\item Три друга – Алеша, Коля и Саша – сели на скамейку в один ряд. Сколькими способами они могли это сделать?
		\item Догадайтесь, какая цифра в выражении заменена буквой \( A \):\\\ \( 9A : 1A = A \).
		\item За \( 1 \) час турист проходит \( 6 \) км. Сколько метров он проходит за \( 1 \) минуту? Сколько сантиметров за \( 1 \) секунду?
		\item Выразите число \( 100 \) пятью одинаковыми цифрами. Предложите четыре способа решения.
		\item Масса бидона с молоком составляет \( 34 \) кг, а масса бидона, наполненного наполовину, равна \( 18 \) кг. Какова масса пустого бидона?
		\item В магазин привезли \( 6 \) мешков сахара по \( 50 \) кг, \( 8 \) мешков муки по \( 25 \) кг, \( 4 \) мешка риса по \( 10  \) кг, \( 2 \) мешка пшена по \( 12 \) кг. Сколько мешков с продуктами привезли? 
		\item Из чисел \( 21 \), \( 19 \), \( 30 \), \( 25 \), \( 3 \), \( 12 \), \( 9 \), \( 15 \), \( 6 \), \( 27  \) подберите такие три числа, сумма которых будет равна \( 50 \).
	\end{listofex}
\end{consultation}
%END_FOLD