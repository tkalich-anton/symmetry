%
%===============>>  ГРУППА 5-1 МОДУЛЬ 8  <<=============
%
\setmodule{9}

%BEGIN_FOLD % ====>>_____ Занятие 1 _____<<====
\begin{class}[number=1]
	\begin{listofex}
		\item Выполните деление:\begin{tasks}(3)
			\task \( 3,24:0,6 \)
			\task \( 4,96 : 0,8 \)
			\task \( 24,804 : 4,77 \)
		\end{tasks}
		\item Решите уравнение \begin{tasks}(2)
			\task \( 16-3,8x = 6,31 \)
			\task \( 13 -7,6х = 8,136 \)
		\end{tasks} 
		\item Округлить:
		\begin{tasks}(1)
			\task до десятых: 8,263;   12,4398;    0,55112
			\task до сотых: 3,274;   11,958;   9,097
			\task до единиц: 35,24;   41,096;   125,608
			\task до сотен: 2345,872;   35987,9842;   3487,842
		\end{tasks} 
		\item Найдите среднее арифметическое чисел 0,2; 5,4 и 6,1.
		\item Среднее арифметическое трех чисел равно 3. Одно число равно 2,4; второе – 3,6. Найдите третье число.
		\item Участница соревнований по фигурному катанию получила оценки \( 5,3 \); \( 4,8 \); \( 5,4 \); \(  5,3 \);  \( 5,2 \); \(  5,1 \). Найдите среднюю оценку этой участницы.                                                                                                                                            
		\item Один участник соревнований по стрельбе стрелял четыре раза и попал в 7;9; 10;4 очка. А второй стрелял пять раз и попал в 6; 7; 8; 5; 6 очков. Какой участник победил?    
		\item Мальчики решили устроить турнир по меткой стрельбе в тире. У Вани было с собой 19 копеек, у Саши 17, а у Сергея 18. Один выстрел в тире стоил 3 копейки. Ребята решили разделить деньги поровну, чтобы каждый сделал одинаковое количество выстрелов. Какое количество выстрелов совершил каждый участник импровизированного турнира? Посчитайте среднее арифметическое.
		\item Машина ехала час со скоростью 60 км/ч. и два часа со скоростью 90 км/ч. С какой постоянной скоростью должна ехать машина, чтобы проехать то же самое расстояние за 3 часа? 
		\item Велосипедист двигался два часа со скоростью   10 км/ч  и три часа со скоростью   15 км/ч.   С какой постоянной скоростью должен  ехать велосипедист, чтобы преодолеть то же самое расстояние за то же  время, 5 часов?    
		\item У Игоря было с собой   45 рублей, у Андрея   28, а у Дениса   17. На все свои деньги они купили 3 билета в кино. Сколько стоил один билет?  
		
	\end{listofex}
\end{class}
%END_FOLD

%BEGIN_FOLD % ====>>_____ Занятие 2 _____<<====
\begin{class}[number=2]
	\begin{listofex}
		\item а
	\end{listofex}
\end{class}
%END_FOLD

%BEGIN_FOLD % ====>>_ Домашняя работа 1 _<<====
\begin{homework}[number=1]
	\begin{listofex}
		\item а
	\end{listofex}
\end{homework}
%END_FOLD

%BEGIN_FOLD % ====>>_____ Занятие 3 _____<<====
\begin{class}[number=3]
	\begin{listofex}
		\item а
	\end{listofex}
\end{class}
%END_FOLD

%BEGIN_FOLD % ====>>_____ Занятие 4 _____<<====
\begin{class}[number=4]
	\begin{listofex}
		\item а
	\end{listofex}
\end{class}
%END_FOLD

%BEGIN_FOLD % ====>>_ Домашняя работа 2 _<<====
\begin{homework}[number=2]
	\begin{listofex}
		\item а
	\end{listofex}
\end{homework}
%END_FOLD

%BEGIN_FOLD % ====>>_____ Занятие 5 _____<<====
\begin{class}[number=5]
	\begin{listofex}
		\item а
	\end{listofex}
\end{class}
%END_FOLD

%BEGIN_FOLD % ====>>_____ Занятие 6 _____<<====
\begin{class}[number=6]
	\begin{listofex}
		\item а
	\end{listofex}
\end{class}
%END_FOLD

%BEGIN_FOLD % ====>>_ Домашняя работа 3 _<<====
\begin{homework}[number=3]
	\begin{listofex}
		\item а
	\end{listofex}
\end{homework}
%END_FOLD

%BEGIN_FOLD % ====>>_____ Занятие 7 _____<<====
\begin{class}[number=7]
	\begin{listofex}
		\item а
	\end{listofex}
\end{class}
%END_FOLD

%BEGIN_FOLD % ====>>_ Проверочная работа _<<====
\begin{exam}
	\begin{listofex}
		\item а
	\end{listofex}
\end{exam}
%END_FOLD

%BEGIN_FOLD % ====>>_ Консультация _<<====
\begin{consultation}
	\begin{listofex}
		\item а
	\end{listofex}
\end{consultation}
%END_FOLD