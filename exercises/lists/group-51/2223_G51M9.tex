%
%===============>>  ГРУППА 5-1 МОДУЛЬ 8  <<=============
%
\setmodule{9}

%BEGIN_FOLD % ====>>_____ Занятие 1 _____<<====
\begin{class}[number=1]
	\begin{listofex}
		\item Выполните деление:\begin{tasks}(3)
			\task \( 3,24:0,6 \)
			\task \( 4,96 : 0,8 \)
			\task \( 24,804 : 4,77 \)
		\end{tasks}
		\item Решите уравнение \begin{tasks}(2)
			\task \( 16-3,8x = 6,31 \)
			\task \( 13 -7,6x = 8,136 \)
		\end{tasks} 
		\item Округлить:
		\begin{tasks}(1)
			\task до десятых: 8,263;   12,4398;    0,55112
			\task до сотых: 3,274;   11,958;   9,097
			\task до единиц: 35,24;   41,096;   125,608
			\task до сотен: 2345,872;   35987,9842;   3487,842
		\end{tasks} 
		\item Найдите среднее арифметическое чисел 0,2; 5,4 и 6,1.
		\item Среднее арифметическое трех чисел равно 3. Одно число равно 2,4; второе – 3,6. Найдите третье число.
		\item Участница соревнований по фигурному катанию получила оценки \( 5,3 \); \( 4,8 \); \( 5,4 \); \(  5,3 \);  \( 5,2 \); \(  5,1 \). Найдите среднюю оценку этой участницы.                                                                                                                                            
		\item Один участник соревнований по стрельбе стрелял четыре раза и попал в 7;9; 10;4 очка. А второй стрелял пять раз и попал в 6; 7; 8; 5; 6 очков. Какой участник победил?    
		\item Мальчики решили устроить турнир по меткой стрельбе в тире. У Вани было с собой 19 копеек, у Саши 17, а у Сергея 18. Один выстрел в тире стоил 3 копейки. Ребята решили разделить деньги поровну, чтобы каждый сделал одинаковое количество выстрелов. Какое количество выстрелов совершил каждый участник импровизированного турнира? Посчитайте среднее арифметическое.
		\item Машина ехала час со скоростью 60 км/ч. и два часа со скоростью 90 км/ч. С какой постоянной скоростью должна ехать машина, чтобы проехать то же самое расстояние за 3 часа? 
		\item Велосипедист двигался два часа со скоростью   10 км/ч  и три часа со скоростью   15 км/ч.   С какой постоянной скоростью должен  ехать велосипедист, чтобы преодолеть то же самое расстояние за то же  время, 5 часов?    
		\item У Игоря было с собой   45 рублей, у Андрея   28, а у Дениса   17. На все свои деньги они купили 3 билета в кино. Сколько стоил один билет?  
		
	\end{listofex}
\end{class}
%END_FOLD

%BEGIN_FOLD % ====>>_____ Занятие 2 _____<<====
\begin{class}[number=2]
	\begin{listofex}
		\item Решите уравнения \begin{tasks}(2)
			\task \( x : 4,6 = 2,3 \)
			\task \( x : 6,8 = 3,4 \)
			\task \( 11,88 : (y - 2,9) = 2,7 \)
			\task \( 8,19x - 3,84x - 1,85x = 19,5\)
		\end{tasks}
		\item Найди среднее арифметическое чисел: \( 23,5 \); \( 20,3 \); \( 22,2 \).
		\item Найди среднее арифметическое первых \( 7 \) натуральных чисел.
		\item Найди среднее арифметическое первых \( 50 \) натуральных чисел.
		\item В волейбольной команде двум игрокам по \( 21 \) году, трем – по \( 20 \) лет и одному \( 24 \) года. Каков средний возраст игроков?
		\item Среднее арифметическое четырех чисел равно \( 2,765 \), а среднее арифметическое других трех чисел равно \( 3,01 \). Вычислите среднее арифметическое этих семи чисел.
		\item У торговца сладостями было \( 10 \) кг конфет по цене \( 60 \) рублей за килограмм и \( 15  \) кг конфет по цене \( 40 \) рублей за килограмм. Мешки с конфетами разорвались и все конфеты перемешались. По какой цене ему продавать полученную смесь?
		\item Средний рост пяти баскетболистов равен \( 195 \) см. Какое наибольшее количество из этих игроков может быть ниже, чем \( 191 \) см?
		\item В овощном магазине за \( 3 \) дня было продано \( 680 \) кг картофеля. В первый день продали \( 28\%  \) всего картофеля, во второй \( 25\% \) от остатка. Сколько кг картофеля было продано в третий день?
		\item На ферме было \( 200 \) животных, из них \( 43\% \) составляли овцы. Сколько овец было на ферме? 
		\item В растворе содержится \( 42 \) кг соли. Чему равна масса раствора, если масса соли в нём составляет \( 35\% \)?
	\end{listofex}
\end{class}
%END_FOLD

%BEGIN_FOLD % ====>>_ Домашняя работа 1 _<<====
\begin{homework}[number=1]
	\begin{listofex}
		\item а
	\end{listofex}
\end{homework}
%END_FOLD

%BEGIN_FOLD % ====>>_____ Занятие 3 _____<<====
\begin{class}[number=3]
	\begin{listofex}
		\item а
	\end{listofex}
\end{class}
%END_FOLD

%BEGIN_FOLD % ====>>_____ Занятие 4 _____<<====
\begin{class}[number=4]
	\begin{listofex}
		\item а
	\end{listofex}
\end{class}
%END_FOLD

%BEGIN_FOLD % ====>>_ Домашняя работа 2 _<<====
\begin{homework}[number=2]
	\begin{listofex}
		\item а
	\end{listofex}
\end{homework}
%END_FOLD

%BEGIN_FOLD % ====>>_____ Занятие 5 _____<<====
\begin{class}[number=5]
	\begin{listofex}
		\item а
	\end{listofex}
\end{class}
%END_FOLD

%BEGIN_FOLD % ====>>_____ Занятие 6 _____<<====
\begin{class}[number=6]
	\begin{listofex}
		\item а
	\end{listofex}
\end{class}
%END_FOLD

%BEGIN_FOLD % ====>>_ Домашняя работа 3 _<<====
\begin{homework}[number=3]
	\begin{listofex}
		\item а
	\end{listofex}
\end{homework}
%END_FOLD

%BEGIN_FOLD % ====>>_____ Занятие 7 _____<<====
\begin{class}[number=7]
	\begin{listofex}
		\item а
	\end{listofex}
\end{class}
%END_FOLD

%BEGIN_FOLD % ====>>_ Проверочная работа _<<====
\begin{exam}
	\begin{listofex}
		\item а
	\end{listofex}
\end{exam}
%END_FOLD

%BEGIN_FOLD % ====>>_ Консультация _<<====
\begin{consultation}
	\begin{listofex}
		\item Скорость поезда \( 67 \) км/ч. Какое расстояние поезд проедет за \( 7 \)ч?
		\item Самолет за \( 4 \) ч пролетел \( 2992 \) км. Какова скорость самолета?
		\item Сколько времени понадобится велосипедисту, чтобы проехать \( 78 \) км со скоростью \( 13 \) км/ч?
		\item Собственная скорость лодки \( 15 \) км/ч. Скорость течения реки \( 3 \)км/ч. Чему равна скорость лодки по течению реки?
		\item Скорость теплохода по течению реки \( 28 \) км/ч. Собственная скорость теплохода \( 26 \) км/ч. Какова скорость течения реки?
		\item Мотоциклист \( 2 \) ч ехал со скоростью \( 58 \) км/ч, а потом \( 4 \) ч со скоростью \( 65 \) км/ч. Какое расстояние проехал мотоциклист за это время?
		\item Пять землекопов за \( 5 \) часов выкапывают \( 5 \) метров канавы. Сколько землекопов выкопают \( 100 \) метров канавы за \( 100 \) часов?
		\item Часы бьют три, и, пока они бьют, проходит \( 3 \) секунды. Сколько времени пройдет, пока часы будут бить семь.
		\item Три друга – Алеша, Коля и Саша – сели на скамейку в один ряд. Сколькими способами они могли это сделать?
		\item Догадайтесь, какая цифра в выражении заменена буквой \( A \):\\\ \( 9A : 1A = A \).
		\item За \( 1 \) час турист проходит \( 6 \) км. Сколько метров он проходит за \( 1 \) минуту? Сколько сантиметров за \( 1 \) секунду?
		\item Выразите число \( 100 \) пятью одинаковыми цифрами. Предложите четыре способа решения.
		\item Масса бидона с молоком составляет \( 34 \) кг, а масса бидона, наполненного наполовину, равна \( 18 \) кг. Какова масса пустого бидона?
		\item В магазин привезли \( 6 \) мешков сахара по \( 50 \) кг, \( 8 \) мешков муки по \( 25 \) кг, \( 4 \) мешка риса по \( 10  \) кг, \( 2 \) мешка пшена по \( 12 \) кг. Сколько мешков с продуктами привезли? 
		\item Из чисел \( 21 \), \( 19 \), \( 30 \), \( 25 \), \( 3 \), \( 12 \), \( 9 \), \( 15 \), \( 6 \), \( 27  \) подберите такие три числа, сумма которых будет равна \( 50 \).
	\end{listofex}
\end{consultation}
%END_FOLD