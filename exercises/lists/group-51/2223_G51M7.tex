%
%===============>>  ГРУППА 5-1 МОДУЛЬ 7  <<=============
%
\setmodule{7}

%BEGIN_FOLD % ====>>_____ Занятие 1 _____<<====
\begin{class}[number=1]
	\begin{listofex}
		\item Выполните действия:
		\begin{tasks}(1)
			\task \( 11,47+(3,89-2,11)-4,416+3,711 \)
			\task \( 3,16+(7,84-4,181)-3,11+14,816 \)
			\task \( 1,49+(6,13-4,12)-0,5+7,289 \)
		\end{tasks}
		\item Вычислите:
		\begin{tasks}(4)
			\task \( 25,36\cdot1,2 \)
			\task \( 3,5\cdot3,8 \)
			\task \( 63,25\cdot0,3 \)
			\task \( 5,369\cdot0,05 \)
			\task \( 12,03\cdot4,6 \)
			\task \( 23,71\cdot1,8 \)
			\task \( 0,24\cdot5,5 \)
			\task \( 0,7\cdot0,14 \)
		\end{tasks}
		\item Найдите:
		\begin{tasks}(2)
			\task \( 3,5\% \) от \( 1400 \)
			\task \( 12,5\% \) от \( 12500 \)
			\task \( 25,5\% \) от \( 14 \)
			\task \( 0,3\% \) от \( 2 \)
		\end{tasks}
		\item Цену на блузку понизили на \( 11,5\% \). Какой стала её цена, если первоначально она стоила \( 5000 \) рублей?
		\item Максим построил у себя в тетради прямоугольник со сторонами, равными \( 1,8 \) см и \( 5,6 \) см. Найдите периметр и площадь фигуры. Что больше? На сколько?
		\item Костя изобразил треугольник, один угол которого равен \( 23,5\degree \), другой в \( 3,7 \) раза больше. Чему равен третий угол? Чему равны внешние углы этого треугольника?
	\end{listofex}
\end{class}
%END_FOLD

%BEGIN_FOLD % ====>>_____ Занятие 2 _____<<====
\begin{class}[number=2]
	\begin{listofex}
		\item Занятие 2
	\end{listofex}
\end{class}
%END_FOLD

%BEGIN_FOLD % ====>>_ Домашняя работа 1 _<<====
\begin{homework}[number=1]
	\begin{listofex}
		\item Домашняя работа 1
	\end{listofex}
\end{homework}
%END_FOLD

%BEGIN_FOLD % ====>>_____ Занятие 3 _____<<====
\begin{class}[number=3]
	\begin{listofex}
		\item Занятие 3 
	\end{listofex}
\end{class}
%END_FOLD

%BEGIN_FOLD % ====>>_____ Занятие 4 _____<<====
\begin{class}[number=4]
	\begin{listofex}
		\item Занятие 4
	\end{listofex}
\end{class}
%END_FOLD

%BEGIN_FOLD % ====>>_ Домашняя работа 2 _<<====
\begin{homework}[number=2]
	\begin{listofex}
		\item Домашняя работа 2
	\end{listofex}
\end{homework}
%END_FOLD

%BEGIN_FOLD % ====>>_____ Занятие 5 _____<<====
\begin{class}[number=5]
	\begin{listofex}
		\item Занятие 5
	\end{listofex}
\end{class}
%END_FOLD

%BEGIN_FOLD % ====>>_____ Занятие 6 _____<<====
\begin{class}[number=6]
	\begin{listofex}
		\item Занятие 6
	\end{listofex}
\end{class}
%END_FOLD

%BEGIN_FOLD % ====>>_ Домашняя работа 3 _<<====
\begin{homework}[number=3]
	\begin{listofex}
		\item Домашняя работа 3
	\end{listofex}
\end{homework}
%END_FOLD

%BEGIN_FOLD % ====>>_____ Занятие 7 _____<<====
\begin{class}[number=7]
	\title{Подготовка к проверочной}
	\begin{listofex}
		\item Занятие 7
	\end{listofex}
\end{class}
%END_FOLD

=%BEGIN_FOLD % ====>>_ Проверочная работа _<<====
\begin{exam}
	\begin{listofex}
		\item Проверочная
	\end{listofex}
\end{exam}
%END_FOLD