%
%===============>>  ГРУППА 5-1 МОДУЛЬ 8  <<=============
%
\setmodule{8}

%BEGIN_FOLD % ====>>_____ Занятие 1 _____<<====
\begin{class}[number=1]
	\begin{listofex}
		\item Длина прямоугольного участка составляет \( 19,4 \) метра, а ширина на \( 2,8 \) метра меньше. Вычислите периметр участка.
		\item На отрезке \( AB \), равном \( 34,64  \) см, взята точка \( C \).
		Отрезок \( AC \) на \( 6,54 \) см больше отрезка \( CB \). Найти длину отрезка \( AC \).
		\item Из точки \( C \), взятой на прямой \( AB \), проведены лучи \( CD \) и \( CE \) с одной стороны \( AC \) так, что \( \angle ACE = 155,38\degree \), а  \( \angle DCB  \) ---
		прямой. Найдите угол DCE. 
		\item Смежные углы относятся как \( 2:3 \). \begin{tasks}(1)
			\task[а)] Найти величину каждого из углов. 
			\task[б)] Определить, сколько процентов
			развернутого угла составляет меньший угол.
		\end{tasks}
		\item После того как один из смежных углов увеличили на \( 40\% \), другой угол уменьшился
		на \( 60\% \). Найдите, какими были по величине первоначально два данных смежных
		угла.
		\item Одна из сторон треугольника равна \( 30,75 \) см, длинна второй на \( 80\% \) короче первой, а третья – на \( 22,7 \) см длиннее второй. Вычислите периметр треугольника.
		\item Периметр равнобедренного треугольника равен \( 67,02 \) см, а боковая сторона \( 18,46 \) см. Найдите длину его основания.
		\item 
		\begin{tasks}(2)
			\task\( \dfrac{0,19\cdot0,75\cdot10,8\cdot0,4}{0,03\cdot1,2 \cdot2,5\cdot5,7 }\)
			\task \( \dfrac{0,25\cdot3,2\cdot0,9\cdot2,1}{3,5\cdot2,4 \cdot0,04} \)
		\end{tasks}
		\item 
		\begin{tasks}(1)
			\task \( 420,42:8,4+(161,14-4,4\cdot0,35):(4,08\cdot30,5-123,3)-90,06 \)
			\task \( 38,7:(51,03-42,43)-(0,4+4,578:3,27)\cdot2,02 \)
		\end{tasks}
	\end{listofex}
\end{class}
%END_FOLD

%BEGIN_FOLD % ====>>_____ Занятие 2 _____<<====
\begin{class}[number=2]
	\begin{listofex}
		\item Длина дороги \( 84 \) км. За первый день бригада рабочих отремонтировала \(\dfrac{ 5}{12} \) дороги, а за второй день --- \( \dfrac{5}{14} \) дороги. Сколько километров осталось отремонтировать?
		\item Точка \( C \) расположена на прямой между точками \( A \) и \( B \). Длина отрезка \( AC \) равна \( 8,93 \) см, длина отрезка \( CB \) на \( 3,02 \) см больше длины отрезка \( AC \). Найдите длину отрезка \( AB \).
		\item На прямой даны три точки \( A \), \( B \) и \( C \), причем \( AB=13,51 \) см, \( AC=4,67 \) см. Найдите длину отрезка \( BC \). (Задача имеет два решения.)
		\item Луч \( OC \) делит развернутый угол \( AOB \) на два смежных угла \( AOC \) и \( BOC \) так, что угол \( AOC \) на \( 75\% \) больше угла \( BOC \). Найдите \( \angle AOC \) и \( \angle BOC \).
		\item Найдите периметр и площадь прямоугольника со сторонами:\begin{tasks}(2)
			\task \( 12,56 \) см и \( 9,11 \) см
			\task \( 93,45 \) см и \( 13,25 \) см
			\task\( 11,809 \) см и \( 47,47 \) мм
			\task\( 17,34 \) см и \( 4,5 \) дм
		\end{tasks}
		\item Стороны прямоугольника равны \( 46,93 \) см и \( \mfrac{3}{1}{4} \) см. Найдите сторону и площадь квадрата, имеющего такой же периметр, что и данный прямоугольник, а также его площадь.
		\item Вычислите объем прямоугольного параллелепипеда, если его ребра равны:
		\begin{tasks}(2)
			\task \( 18,3 \) см, \( \mfrac{16}{4}{25} \) см, \( 5,2 \) см;
			\task  \( 12,83\) см, \( 45,91 \) см, \( \mfrac{2}{4}{9} \) см;
			\task  \( \mfrac{20}{11}{12} \) см, \( 23,102 \) см, \( 25,25 \) см;
			\task  \( 11,11 \) см, \( \mfrac{3}{1}{7} \) см, \( 11,83 \) см;
		\end{tasks}
		\item Мама купила \( 4 \) пирожных. Расплачиваясь за них она получила \( 40 \) рублей сдачи. Если бы мама купила \( 6 \) пирожных, то ей пришлось бы доплатить еще \( 40 \) рублей. Сколько стоит пирожное?
		\item \( 0,198\cdot\mfrac{9}{1}{11}-\left[ \left( 2,56+\dfrac{3}{4}-2,56-0,125 \right)\cdot\mfrac{2}{2}{3}-\dfrac{1}{15} \right]:16\cdot\left( \mfrac{5}{3}{4}+2,25 \right) \)
	\end{listofex}
\end{class}
%END_FOLD

%BEGIN_FOLD % ====>>_ Домашняя работа 1 _<<====
\begin{homework}[number=1]
	\begin{listofex}
		\item \begin{tasks}(1)
			\task \( (5,27 – 24,9 \cdot (0,48 – 0,38)) : 0,2 \)
			\task \( 4,57\cdot 0,03 + 27,1 \cdot (1,56 – 1,46) : 0,05 \)
		\end{tasks}
		\item Найдите стороны прямоугольника, если его периметр равен \( 7,2 \) см и одна сторона в \( 2,35 \) больше другой.
		\item На отрезке \( AB \), равном \( 41,02  \) см, взята точка \( C \).
		Отрезок \( AC \) на \( 10,111 \) см больше отрезка \( CB \). Найти длину отрезка \( AC \).
		\item Луч \( OK \) делит развернутый угол \( NOM \) на два смежных угла \( NOK \) и \( MOK \) так, что угол \( NOK \) на \( 34\% \) больше угла \( MOK \). Найдите \( \angle NOK \) и \( \angle MOK \).
	\end{listofex}
\end{homework}
%END_FOLD

%BEGIN_FOLD % ====>>_____ Занятие 3 _____<<====
\begin{class}[number=3]
	\begin{listofex}
		\item Занятие 3 
	\end{listofex}
\end{class}
%END_FOLD

%BEGIN_FOLD % ====>>_____ Занятие 4 _____<<====
\begin{class}[number=4]
	\begin{listofex}
		\item Занятие 4
	\end{listofex}
\end{class}
%END_FOLD

%BEGIN_FOLD % ====>>_ Домашняя работа 2 _<<====
\begin{homework}[number=2]
	\begin{listofex}
		\item Домашняя работа 2
	\end{listofex}
\end{homework}
%END_FOLD

%BEGIN_FOLD % ====>>_____ Занятие 5 _____<<====
\begin{class}[number=5]
	\begin{listofex}
		\item Занятие 5
	\end{listofex}
\end{class}
%END_FOLD

%BEGIN_FOLD % ====>>_____ Занятие 6 _____<<====
\begin{class}[number=6]
	\begin{listofex}
		\item Занятие 6
	\end{listofex}
\end{class}
%END_FOLD

%BEGIN_FOLD % ====>>_ Домашняя работа 3 _<<====
\begin{homework}[number=3]
	\begin{listofex}
		\item Домашняя работа 3
	\end{listofex}
\end{homework}
%END_FOLD

%BEGIN_FOLD % ====>>_____ Занятие 7 _____<<====
\begin{class}[number=7]
	\title{Подготовка к проверочной}
	\begin{listofex}
		\item Занятие 7
	\end{listofex}
\end{class}
%END_FOLD

=%BEGIN_FOLD % ====>>_ Проверочная работа _<<====
\begin{exam}
	\begin{listofex}
		\item Проверочная
	\end{listofex}
\end{exam}
%END_FOLD

%BEGIN_FOLD % ====>>_ Консультация _<<====
\begin{consultation}
	\begin{listofex}
		\item Вычислить:
		\begin{tasks}(4)
			\task \( 20,7:9 \)
			\task \( 243,2:8 \)
			\task \( 7,368:24 \)
			\task \( 25:125 \)
			\task \( 1:8 \)
			\task \( 72,57:59 \)
			\task \( 0,7:25 \)
			\task \( 6,78:26 \)
		\end{tasks}
		\item Вычислить:
		\begin{tasks}(4)
			\task \( 45,5:10 \)
			\task \( 45,5:1000 \)
			\task \( 45,5:10000 \)
			\task \( 89:10 \)
			\task \( 89:100 \)
			\task \( 32,2:10 \)
			\task \( 7,98:10 \)
			\task \( 47,7:1000 \)
			\task \( 0,911:1000 \)
		\end{tasks}
		\item Вычислить:
		\begin{tasks}(4)
			\task \( 2:0,4 \)
			\task \( 70:1,75 \)
			\task \( 24:0,2 \)
			\task \( 2:0,5 \)
			\task \( 45:0,05 \)
			\task \( 125:2,5 \)
			\task \( 484:0,004 \)
			\task \( 5,1:0,17 \)
			\task \( 25,2:0,4 \)
			\task \( 200,1:0,69 \)
		\end{tasks}
		\item Путь у Ксюши до дома занимает \( 1,6 \) км. Утром она обычно опаздывает в школу, поэтому бежит. В этом случае время, которое ей необходимо на дорогу до школы, составляет \( 0,25 \) часа. Обратно со школы Ксюша не торопится и идет домой в течение \( 0,5 \) часа. С какой скоростью Ксюша бежит в школу утром и с какой скоростью она идет домой после уроков?
		\item Вася случайно разрезал моток веревки на две части. Длина одной из них --- \( 3,25 \) м, длина другой в \( 1,3 \) раза короче. Необходимо найти общую длину исходной веревки.
		\item Вася живет в комнате, которая имеет форму прямоугольного параллелепипеда. Известно, что объем этой комнаты равен \( 37,84 \) м\( ^3 \). При этом длина комнаты составляет \( 4 \) м, а высота потолка --- \( 2,2 \) м. Нужно найти ширину комнаты. 
		\item Кенгуру ниже жирафа в \( 2,4 \) раза, при этом жираф выше кенгуру на \( 2,52 \) м. Найдите рост жирафа и кенгуру.
	\end{listofex}
\end{consultation}
%END_FOLD