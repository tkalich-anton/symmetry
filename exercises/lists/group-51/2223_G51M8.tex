%
%===============>>  ГРУППА 5-1 МОДУЛЬ 8  <<=============
%
\setmodule{8}

%BEGIN_FOLD % ====>>_____ Занятие 1 _____<<====
\begin{class}[number=1]
	\begin{listofex}
%		\item Занятие 1
		\item Вычислить:
		\[ 50:\left( \mfrac{14}{8}{23}+\mfrac{5}{15}{23} \right)-\left( \mfrac{6}{1}{5}-\mfrac{2}{3}{5} \right):9 \]
		\item Вычислить:
		\begin{tasks}(2)
			\task \( 0,38\cdot2,5 \)
			\task \( 3,5\cdot5,5 \)
			\task \( 1,2:0,2 \)
			\task \( 3,63:0,003 \)
		\end{tasks}
		\item Вычислить:
		\begin{tasks}(2)
			\task \( 2,5\cdot100 \)
			\task \( 13,55\cdot10000 \)
			\task \( 4,12:10 \)
			\task \( 5554,3345:10000 \)
		\end{tasks}
		\item Вычислить:
		\begin{tasks}(2)
			\task \( 35,5\cdot0,1 \)
			\task \( 212,4\cdot0,001 \)
			\task \( 5644,65\cdot0,1\cdot0,001 \)
			\task \( 34,5\cdot0,001\cdot1,2 \)
		\end{tasks}
		\item Вычислить:
		\begin{tasks}(2)
			\task \( 5,212:0,001 \)
			\task \( 95,5995:0,01 \)
			\task \( 3,0013:0,0001 \)
			\task \( 54,4\cdot100:0,01 \)
		\end{tasks}
		\item Вычислить:
		\begin{tasks}(2)
			\task \( (54,3\cdot100)-(34,5:0,1) \)
			\task \( 13,3\cdot100\cdot0,001:100:0,01\cdot1000 \)
			\task! \( 252,1:0,001\cdot10000\cdot0,001:1000 \)
		\end{tasks}
		\item Автомобиль ехал \( 0,9 \) ч по асфальтированной дороге и \( 0,6 \) ч по грунтовой, проехав всего \( 93,6 \) км.
		С какой скоростью двигался автомобиль по асфальтированной дороге,
		если по грунтовой он ехал со скоростью \( 48 \) км/ч?
		\item Моторная лодка плыла \( 1,4 \) ч по течению реки и \( 2,2 \) ч против течения.
		Какой путь преодолела лодка за все время движения, если скорость
		течения равна \( 1,7 \) км/ч, а собственная скорость лодки --- \( 19,8 \) км/ч?
		\item Постройте точки:
		\( A(3,2); \; B(3,9); \; C(2,12); \; D(5,9); \; E(10,9); \; F(12,18);\\
		K(18,8); \; G(13,16); \; H(14,18); \; I(15,15); \; Y(18,11); \; L(15,7); \; M(13,9);\\
		P(10,5); \; \; N(14,2); \; О(11,2); \; Q(6,6); \; R(6,2) \)\\
		Соедините точки:
		\( A-B-C-D-E-F-G-H-I-Y-K-L-M-N-O-P-
		Q-R-A \)
		\item Даны точки \( A (0;2) \), \( B (0;4) \), \( C (3;0) \), \( D (1;0) \). Определите, какие из них находятся на оси \( x \), а какие на оси \( y \).
	\end{listofex}
\end{class}
%END_FOLD

%BEGIN_FOLD % ====>>_____ Занятие 2 _____<<====
\begin{class}[number=2]
	\begin{listofex}
		\item Занятие 2
	\end{listofex}
\end{class}
%END_FOLD

%BEGIN_FOLD % ====>>_ Домашняя работа 1 _<<====
\begin{homework}[number=1]
	\begin{listofex}
		\item Домашняя работа 1
	\end{listofex}
\end{homework}
%END_FOLD

%BEGIN_FOLD % ====>>_____ Занятие 3 _____<<====
\begin{class}[number=3]
	\begin{listofex}
		\item Занятие 3 
	\end{listofex}
\end{class}
%END_FOLD

%BEGIN_FOLD % ====>>_____ Занятие 4 _____<<====
\begin{class}[number=4]
	\begin{listofex}
		\item Занятие 4
	\end{listofex}
\end{class}
%END_FOLD

%BEGIN_FOLD % ====>>_ Домашняя работа 2 _<<====
\begin{homework}[number=2]
	\begin{listofex}
		\item Домашняя работа 2
	\end{listofex}
\end{homework}
%END_FOLD

%BEGIN_FOLD % ====>>_____ Занятие 5 _____<<====
\begin{class}[number=5]
	\begin{listofex}
		\item Занятие 5
	\end{listofex}
\end{class}
%END_FOLD

%BEGIN_FOLD % ====>>_____ Занятие 6 _____<<====
\begin{class}[number=6]
	\begin{listofex}
		%		\item Занятие 1
		\item Вычислить:
		\[ 50:\left( \mfrac{14}{8}{23}+\mfrac{5}{15}{23} \right)-\left( \mfrac{6}{1}{5}-\mfrac{2}{3}{5} \right):9 \]
		\item Вычислить:
		\begin{tasks}(2)
			\task \( 0,38\cdot2,5 \)
			\task \( 3,5\cdot5,5 \)
			\task \( 1,2:0,2 \)
			\task \( 3,63:0,003 \)
		\end{tasks}
		\item Вычислить:
		\begin{tasks}(2)
			\task \( 2,5\cdot100 \)
			\task \( 13,55\cdot10000 \)
			\task \( 4,12:10 \)
			\task \( 5554,3345:10000 \)
		\end{tasks}
		\item Вычислить:
		\begin{tasks}(2)
			\task \( 35,5\cdot0,1 \)
			\task \( 212,4\cdot0,001 \)
			\task \( 5644,65\cdot0,1\cdot0,001 \)
			\task \( 34,5\cdot0,001\cdot1,2 \)
		\end{tasks}
		\item Вычислить:
		\begin{tasks}(2)
			\task \( 5,212:0,001 \)
			\task \( 95,5995:0,01 \)
			\task \( 3,0013:0,0001 \)
			\task \( 54,4\cdot100:0,01 \)
		\end{tasks}
		\item Вычислить:
		\begin{tasks}(2)
			\task \( (54,3\cdot100)-(34,5:0,1) \)
			\task \( 13,3\cdot100\cdot0,001:100:0,01\cdot1000 \)
			\task! \( 252,1:0,001\cdot10000\cdot0,001:1000 \)
		\end{tasks}
		\item Автомобиль ехал \( 0,9 \) ч по асфальтированной дороге и \( 0,6 \) ч по грунтовой, проехав всего \( 93,6 \) км.
		С какой скоростью двигался автомобиль по асфальтированной дороге,
		если по грунтовой он ехал со скоростью \( 48 \) км/ч?
		\item Моторная лодка плыла \( 1,4 \) ч по течению реки и \( 2,2 \) ч против течения.
		Какой путь преодолела лодка за все время движения, если скорость
		течения равна \( 1,7 \) км/ч, а собственная скорость лодки --- \( 19,8 \) км/ч?
		\item Постройте точки:
		\( A(3,2); \; B(3,9); \; C(2,12); \; D(5,9); \; E(10,9); \; F(12,18);\\
		K(18,8); \; G(13,16); \; H(14,18); \; I(15,15); \; Y(18,11); \; L(15,7); \; M(13,9);\\
		P(10,5); \; \; N(14,2); \; О(11,2); \; Q(6,6); \; R(6,2) \)\\
		Соедините точки:
		\( A-B-C-D-E-F-G-H-I-Y-K-L-M-N-O-P-
		Q-R-A \)
		\item Даны точки \( A (0;2) \), \( B (0;4) \), \( C (3;0) \), \( D (1;0) \). Определите, какие из них находятся на оси \( x \), а какие на оси \( y \).
	\end{listofex}
\end{class}
%END_FOLD

%BEGIN_FOLD % ====>>_ Домашняя работа 3 _<<====
\begin{homework}[number=3]
	\begin{listofex}
		\item Домашняя работа 3
	\end{listofex}
\end{homework}
%END_FOLD

%BEGIN_FOLD % ====>>_____ Занятие 7 _____<<====
\begin{class}[number=7]
	\title{Подготовка к проверочной}
	\begin{listofex}
		\item Занятие 7
	\end{listofex}
\end{class}
%END_FOLD

=%BEGIN_FOLD % ====>>_ Проверочная работа _<<====
\begin{exam}
	\begin{listofex}
		\item Проверочная
	\end{listofex}
\end{exam}
%END_FOLD