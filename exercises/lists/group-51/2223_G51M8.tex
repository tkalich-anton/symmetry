%
%===============>>  ГРУППА 5-1 МОДУЛЬ 8  <<=============
%
\setmodule{8}

%BEGIN_FOLD % ====>>_____ Занятие 1 _____<<====
\begin{class}[number=1]
	\begin{listofex}
%		\item Занятие 1
		\item Вычислить:
		\[ 50:\left( \mfrac{14}{8}{23}+\mfrac{5}{15}{23} \right)-\left( \mfrac{6}{1}{5}-\mfrac{2}{3}{5} \right):9 \]
		\item Вычислить:
		\begin{tasks}(2)
			\task \( 0,38\cdot2,5 \)
			\task \( 3,5\cdot5,5 \)
			\task \( 1,2:0,2 \)
			\task \( 3,63:0,003 \)
		\end{tasks}
		\item Вычислить:
		\begin{tasks}(2)
			\task \( 2,5\cdot100 \)
			\task \( 13,55\cdot10000 \)
			\task \( 4,12:10 \)
			\task \( 5554,3345:10000 \)
		\end{tasks}
		\item Вычислить:
		\begin{tasks}(2)
			\task \( 35,5\cdot0,1 \)
			\task \( 212,4\cdot0,001 \)
			\task \( 5644,65\cdot0,1\cdot0,001 \)
			\task \( 34,5\cdot0,001\cdot1,2 \)
		\end{tasks}
		\item Вычислить:
		\begin{tasks}(2)
			\task \( 5,212:0,001 \)
			\task \( 95,5995:0,01 \)
			\task \( 3,0013:0,0001 \)
			\task \( 54,4\cdot100:0,01 \)
		\end{tasks}
		\item Вычислить:
		\begin{tasks}(2)
			\task \( (54,3\cdot100)-(34,5:0,1) \)
			\task \( 13,3\cdot100\cdot0,001:100:0,01\cdot1000 \)
			\task! \( 252,1:0,001\cdot10000\cdot0,001:1000 \)
		\end{tasks}
		\item Автомобиль ехал \( 0,9 \) ч по асфальтированной дороге и \( 0,6 \) ч по грунтовой, проехав всего \( 93,6 \) км.
		С какой скоростью двигался автомобиль по асфальтированной дороге,
		если по грунтовой он ехал со скоростью \( 48 \) км/ч?
		\item Моторная лодка плыла \( 1,4 \) ч по течению реки и \( 2,2 \) ч против течения.
		Какой путь преодолела лодка за все время движения, если скорость
		течения равна \( 1,7 \) км/ч, а собственная скорость лодки --- \( 19,8 \) км/ч?
		\item Постройте точки:
		\( A(3,2); \; B(3,9); \; C(2,12); \; D(5,9); \; E(10,9); \; F(12,18);\\
		K(18,8); \; G(13,16); \; H(14,18); \; I(15,15); \; Y(18,11); \; L(15,7); \; M(13,9);\\
		P(10,5); \; \; N(14,2); \; О(11,2); \; Q(6,6); \; R(6,2) \)\\
		Соедините точки:
		\( A-B-C-D-E-F-G-H-I-Y-K-L-M-N-O-P-
		Q-R-A \)
		\item Даны точки \( A (0;2) \), \( B (0;4) \), \( C (3;0) \), \( D (1;0) \). Определите, какие из них находятся на оси \( x \), а какие на оси \( y \).
	\end{listofex}
\end{class}
%END_FOLD

%BEGIN_FOLD % ====>>_____ Занятие 2 _____<<====
\begin{class}[number=2]
	\begin{listofex}
		\item Занятие 2
	\end{listofex}
\end{class}
%END_FOLD

%BEGIN_FOLD % ====>>_ Домашняя работа 1 _<<====
\begin{homework}[number=1]
	\begin{listofex}
		\item Домашняя работа 1
	\end{listofex}
\end{homework}
%END_FOLD

%BEGIN_FOLD % ====>>_____ Занятие 3 _____<<====
\begin{class}[number=3]
	\begin{listofex}
		\item Запишите цифру, которую можно поставить вместо звѐздочки, чтобы образовалось
		верное неравенство (рассмотрите все возможные случаи):
		\begin{tasks}(2)
			\task \( 3 7,8* < 3 7,84 \)
			\task \( 5 ,8*5>5 ,872 \)
		\end{tasks}  
		\item Напишите три числа, каждое из которых больше 7,87 и меньше 7,89.
		\item В гараже стоят 63 машины, из них \( \dfrac{5}{7} \) составляют легковые. Сколько легковых машин стоит в гараже?
		\item Кирилл прочѐл 56 страниц, что составило \( \dfrac{7}{12} \) книги. Сколько страниц было в книге?
		\item В классе 12 учеников изучают французский язык, что составляет \( \dfrac{2}{5} \) всех учеников класса. Сколько учеников в классе?
		\item Одна сторона треугольника равна 5,6 см, что на 1,4 см больше второй стороны и на 0,7 см меньше третьей. Найдите периметр треугольника.
		\item Скорость катера по течению реки равна 24,2 км/ч, а собственная скорость катера --- 22,8 км/ч. Найдите скорость катера против течения реки.
		\item Моторная лодка плыла 1,4 ч по течению реки и 2,2 ч против течения. Какой путь преодолела лодка за всѐ время движения, если скорость течения равна 1,7 км/ч, а собственная скорость лодки – 19,8 км/ч?
		\item Пѐтр шѐл из села к озеру 0,7 ч по одной дороге, а возвратился по другой дороге за 0,8 ч, пройдя всего 6,44 км. С какой скоростью шѐл Пѐтр к озеру, если
		возвращался он со скоростью 3,5 км/ч?
		\item Точка \( K \) принадлежит отрезку \( ME \), \( MK = 19,45 \) см, отрезок \( KE \) на \( 17,201 \) см больше отрезка \( MK \). Найдите длину отрезка \( ME \). 
		\item На отрезке \( CD \) длиной \( 40,123 \) см отметили точки \( P \) и \( Q \) так, что \( CP = 28,34 \) см, \( QD =26,82 \) см. Чему равна длина отрезка \( PQ \)?
		\item Длина прямоугольного параллелепипеда равна 18 см, ширина – в 2 раза
		меньше длины, а высота --- на 11,46 см больше ширины. Вычислите объем
		параллелепипеда.
		\item Поле прямоугольной формы имеет площадь 6 га. Ширина поля 150 м.
		Вычислите периметр поля.
		\item Сумма длин всех рѐбер прямоугольного параллелепипеда равна 136,64 см, а два его измерения --- 15,02 см и 11,34 см. Найдите третье измерение параллелепипеда.
		\item Ширина прямоугольного параллелепипеда равна 4 см, что составляет  \( \dfrac{8}{15} \) его длины, а высота составляет \( 40\% \) длины. Вычислите объем параллелепипеда.
		\item Если в некоторой десятичной дроби перенести запятую вправо через одну цифру, то она увеличится на 14,31. Найдите эту дробь.
		
	\end{listofex}
\end{class}
%END_FOLD

%BEGIN_FOLD % ====>>_____ Занятие 4 _____<<====
\begin{class}[number=4]
	\begin{listofex}
		\item В двузначном натуральном числе сумма цифр равна 16. Число десятков на 2 меньше числа единиц. Найдите это число.
		\item В одном зале кинотеатра в 2 раза больше зрителей, чем в другом. Если из первого
		зала уйдут 37 человек, а во второй придут 50, то зрителей в обоих залах будет поровну.
		Сколько зрителей в каждом зале?
		\item В первые сутки теплоход прошёл \( \dfrac{
		9}{20} \)
		всего пути, во вторые сутки – на \( \dfrac{1}{15} \) пути больше, чем в первые. Какую часть всего пути теплоход прошел за эти двое суток?
		\item В один пакет насыпали \( \mfrac{2}{4}{5} \) кг пшена, а в другой \( \dfrac{6}{7} \) этого количества. На сколько меньше пшена насыпали во второй пакет, чем в первый?
		\item В овощехранилище привезли 320 т овощей. \( 75\% \) привезенных овощей составлял картофель, а \( \dfrac{11}{16} \) остатка – капуста. Сколько тонн капусты привезли в овощехранилище?
		\item Площадь одного участка земли \( \mfrac{2}{3}{4} \) га, а другого – в \( \mfrac{1}{1}{11} \) раза больше. На сколько гектаров площадь первого участка меньше площади второго?
		\item В книге 240 страниц. Повесть занимает \( 60\% \) книги, а рассказы\( \dfrac{19}{24} \)
		остатка. Сколько
		страниц в книге занимают рассказы?
		\item У Сережи и Пети всего 69 марок. У Пети марок в \( \mfrac{1}{7}{8} \) раза больше, чем у Сережи. Сколько марок у каждого из мальчиков?
		\item Электричкой, автобусом и катером туристы проехали 150 км. На электричке туристы
		проехали \( 60\% \) всего пути, а на автобусе --- \( \dfrac{2}{3} \) оставшегося пути. Сколько км туристы проехали на автобусе?
		\item  С молочной фермы \( 14\% \) всего молока отправили в детский сад и \( \dfrac{3}{7} \) всего молока – в
		школу. Сколько молока отправили в школу, если в детский сад отправили 49 л?
		\item Автомобиль ехал 0,9 ч по асфальтированной дороге и 0,6 ч по грунтовой, проехав всего 93,6 км. С какой скоростью двигался автомобиль по асфальтированной дороге, если по грунтовой он ехал со скоростью 48 км/ч?
		\newpage
		\item Выполните действия: \begin{tasks}(1)
			\task \( (4,1-0,66:1,2)\cdot0,6 \)
			\task \( 50:\left( \mfrac{14}{8}{23}+\mfrac{5}{15}{23} \right)-\left( \mfrac{6}{1}{5}-\mfrac{2}{3}{5} \right):9 \)
		\end{tasks}
	\end{listofex}
\end{class}
%END_FOLD

%BEGIN_FOLD % ====>>_ Домашняя работа 2 _<<====
\begin{homework}[number=2]
	\begin{listofex}
		\item Выполните действия: \begin{tasks}(1)
			\task \( 2,867:0,094+0,31\cdot0,15 \)
			\task \( (0,49:1,4-0,325)\cdot0,8 \)
		\end{tasks}
		\item Точка \( T \) принадлежит отрезку \( MN \), \( MT = 19.14 \) см, отрезок \( TN \) на \( 7.36 \) см меньше отрезка \( MT \). Найдите длину отрезка \( MN \).
		\item У Сережи и Пети всего 56 марок. У Пети марок в \( \mfrac{1}{12}{15} \) раза больше, чем у Сережи. Сколько марок у каждого из мальчиков?
		\item На отрезке \( CD \) длиной \( 54,02 \) см отметили точки \( P \) и \( Q \) так, что \( CP = 17,33 \) см, \( QD = 30,22 \) см. Чему равна длина отрезка \( PQ \)?
		\item В равнобедренном треугольнике основание равно 5,93 см, а боковая сторона в 3,5 раза больше основания. Найдите периметр этого треугольника.
	\end{listofex}
\end{homework}
%END_FOLD

%BEGIN_FOLD % ====>>_____ Занятие 5 _____<<====
\begin{class}[number=5]
	\begin{listofex}
		\item Постройте точки:
		\( F(9,5) \); \( C(17,5) \); \( B(14,2) \); \( E(3,2) \); \( D(2,5) \);
		\( G(9,14) \); \( H(15,6) \); \( I(9,6) \); \( J(9,7) \); \( K(5,7) \); \( L(9,13) \); \(  M(9,16) \). Соедините их в порядке построения.
		\item Постройте точки:
		 \( A(3,2) \); B(3,9); C(2,12); D(5,9); E(10,9);
		F(12,18); G(13,16); H(14,18); I(15,15);
		Y(18,11); K(18,8); L(15,7); M(13,9);
		N(14,2); О(11,2); P(10,5); Q(6,6); R(6,2) 
		Соедините точки:
		\( A-B-C-D-E-F-G-H-I-Y-K-L-M-N-O-P-
		Q-R-A \)
		\item Даны точки \( A (0;2) \), \( B (0;4) \), \( C (3;0) \), \( D (1;0) \). Определите, какие из них находятся на оси \( x \), а какие на оси \( y \).
		\item Даны точки \( A(4;2) \), \( B(1;-5) \), \( C(-3;3) \), \( D(-2;1) \), \( E(-4;-5) \), \( F(1;4) \), \( G(4;-3) \), \( H(-1;-1) \). В каких четвертях они находятся?
		\item Известны координаты пятнадцати точек: 1(4, 1), 2(4, 2), 3(1, 2), 4(4, 5), 5(2, 5), 6(4, 7), 7(3, 7), 8(5, 9), 9(7, 7), 10(6, 7), 11(8, 5), 12(6, 5), \( 13(9, 2) \), \( 14(6, 2) \), 15(6, 1). Отметьте эти точки на координатной плоскости, а затем соедините их отрезками в последовательности 1—2—3—4—5—6—7—8—9—10—11—12 —13—14—15—1. 
		\item Построить животного по координатам:
		1(3,3), 2(0,3), 3(-3,2), \( 4(-5,2) \), \(  5(-7,4) \), \( 6(-8,3) \), \( 7(-7,1) \), \( 8(-8,-1) \), \( 9(-7,-2) \), \( 10(-5,0) \), \( 11(-1,-2) \), \( 12(0,-4) \), \( 13(2,-4) \), \( 14(3,-2) \), \( 15(5,-2) \), \( 16(7,0) \), \( 17(5,2) \), \( 18(3,3) \), \( 19(2,4) \), \( 20(-3,4) \),\(  21(-4,2) \), глаз (5,0).
		\item Даны точки 1(1,1), 2(2,1), 3(2,2), 4(3,2), 5(3,3), 6(7,3), 7(7,1), 8(11,1), 9(11,6), 10(7,6), 11(7,4), 12(1,4), 13(8,2), 14(10,2), 15(10,5), 16(8,5).\\\
		Соединить их в следующем порядке:
		1-2-3-4-5-6-7-8-9-10-11-12-1 и 13-14-15-16-13.
	\end{listofex}
\end{class}
%END_FOLD

%BEGIN_FOLD % ====>>_____ Занятие 6 _____<<====
\begin{class}[number=6]
	\begin{listofex}
		%		\item Занятие 1
		\item Вычислить:
		\[ 50:\left( \mfrac{14}{8}{23}+\mfrac{5}{15}{23} \right)-\left( \mfrac{6}{1}{5}-\mfrac{2}{3}{5} \right):9 \]
		\item Вычислить:
		\begin{tasks}(2)
			\task \( 0,38\cdot2,5 \)
			\task \( 3,5\cdot5,5 \)
			\task \( 1,2:0,2 \)
			\task \( 3,63:0,003 \)
		\end{tasks}
		\item Вычислить:
		\begin{tasks}(2)
			\task \( 2,5\cdot100 \)
			\task \( 13,55\cdot10000 \)
			\task \( 4,12:10 \)
			\task \( 5554,3345:10000 \)
		\end{tasks}
		\item Вычислить:
		\begin{tasks}(2)
			\task \( 35,5\cdot0,1 \)
			\task \( 212,4\cdot0,001 \)
			\task \( 5644,65\cdot0,1\cdot0,001 \)
			\task \( 34,5\cdot0,001\cdot1,2 \)
		\end{tasks}
		\item Вычислить:
		\begin{tasks}(2)
			\task \( 5,212:0,001 \)
			\task \( 95,5995:0,01 \)
			\task \( 3,0013:0,0001 \)
			\task \( 54,4\cdot100:0,01 \)
		\end{tasks}
		\item Вычислить:
		\begin{tasks}(2)
			\task \( (54,3\cdot100)-(34,5:0,1) \)
			\task \( 13,3\cdot100\cdot0,001:100:0,01\cdot1000 \)
			\task! \( 252,1:0,001\cdot10000\cdot0,001:1000 \)
		\end{tasks}
		\item Автомобиль ехал \( 0,9 \) ч по асфальтированной дороге и \( 0,6 \) ч по грунтовой, проехав всего \( 93,6 \) км.
		С какой скоростью двигался автомобиль по асфальтированной дороге,
		если по грунтовой он ехал со скоростью \( 48 \) км/ч?
		\item Моторная лодка плыла \( 1,4 \) ч по течению реки и \( 2,2 \) ч против течения.
		Какой путь преодолела лодка за все время движения, если скорость
		течения равна \( 1,7 \) км/ч, а собственная скорость лодки --- \( 19,8 \) км/ч?
		\item Постройте точки:
		\( A(3,2); \; B(3,9); \; C(2,12); \; D(5,9); \; E(10,9); \; F(12,18);\\
		K(18,8); \; G(13,16); \; H(14,18); \; I(15,15); \; Y(18,11); \; L(15,7); \; M(13,9);\\
		P(10,5); \; \; N(14,2); \; О(11,2); \; Q(6,6); \; R(6,2) \)\\
		Соедините точки:
		\( A-B-C-D-E-F-G-H-I-Y-K-L-M-N-O-P-
		Q-R-A \)
		\item Даны точки \( A (0;2) \), \( B (0;4) \), \( C (3;0) \), \( D (1;0) \). Определите, какие из них находятся на оси \( x \), а какие на оси \( y \).
	\end{listofex}
\end{class}
%END_FOLD

%BEGIN_FOLD % ====>>_ Домашняя работа 3 _<<====
\begin{homework}[number=3]
	\begin{listofex}
		\item Вычислите: 
		\begin{tasks}(2)
			\task \( \mfrac{5}{2}{4}:\dfrac{3}{2}-2\cdot\dfrac{1}{3} \)
			\task \( \dfrac{4}{9}\cdot\dfrac{3}{4}+\dfrac{11}{36}+\mfrac{2}{3}{6}\cdot\mfrac{5}{5}{6} -\dfrac{7}{36}\)
			\task \( \mfrac{5}{2}{4}-\mfrac{2}{1}{2}\cdot\dfrac{1}{2}+\dfrac{14}{8}:\dfrac{7}{8}\cdot3\cdot\dfrac{3}{4}\)
			\task \( 3\cdot\dfrac{3}{9}\cdot\mfrac{4}{11}{15}+\mfrac{3}{44}{45} \)
		\end{tasks}
		\item Вычислите: 
		\begin{tasks}(2)
			\task \( 3,65\cdot5,82 \)
			\task \( 10,34\cdot3-2,4 \)
			\task \( 6,93:3+4,21\cdot1,43 \)
			\task \( 11,6-5,32\cdot2+3,24 \)
		\end{tasks}
		\item Известны координаты 25 точек:
		\( A(7;18) \), \( B(9;18) \), \( C(14;22) \), \( D(14;24) \), \( E(18;19) \), \( F(17;15) \), \( G(20;10) \), \( H(17;3) \), \( I(19;1) \), \( J(15;1) \), \( K(14;3) \), \( L(11;3) \), \( M(12;1) \), \( N(7;1) \), \( O(2;11) \), \( P(1;18) \), \( Q(2;23) \), \( R(5;24) \), \( S(7;22) \), \( T(5;11) \), \( U(8;7) \), \( V(12;7) \), \( W(16;14) \), \( X(16;14) \), \( Y(11;14) \). Отметьте эти точки на координатной плоскости и соедините их в том порядке, в котором даны эти точки.
		
	\end{listofex}
\end{homework}
%END_FOLD

%BEGIN_FOLD % ====>>_____ Занятие 7 _____<<====
\begin{class}[number=7]
	\title{Подготовка к проверочной}
	\begin{listofex}
		\item Выполните действия: \( (82,23-32,61):0,5+92,24\cdot0,25-2,3 \)
		\item Выполните действия: \( \mfrac{6}{12}{27}-\mfrac{4}{5}{27}+\mfrac{6}{16}{27} \)
		\item В олимпиаде участвовало 300 школьников, \( \dfrac{2}{5} \)  из них прошли в следующий тур. Сколько участников будут проходить испытания в следующем туре олимпиады?
		\item В магазин привезли 983,57 кг сахара. В первый день продали \( \dfrac{1}{7} \) всего сахара, во второй день  \( \dfrac{2}{7} \) всего сахара. Сколько килограммов сахара продали за два дня? Cколько осталось?
		\item В магазине купили  1,4 кг шоколадных конфет и \( \mfrac{2}{3}{5}\) кг карамели. Сколько килограммов конфет купили в магазине?
		\item Отметь точки на координатной плоскости. \\\
		(1,13), (2,12), (3,13), (4,13), (5,14), (3,16), (5,16), (7,14),(7,13), (8,13), (10,11), (19,11), (22,8), (23,8), (24,9), (24,7),(22,7), (22,5), (20,3), (20,1), (17,1), (18,2), (17,2),(18,3), (18,4), (10,4), (10,1), (7,1), (8,2), (7,2),(8,3), (8,4), (6,6), (4,9), (3,9), (2,10), (1,9), (1,13), (4,12).\\\
		Соедини точки по порядку.
	\end{listofex}
\end{class}
%END_FOLD

%BEGIN_FOLD % ====>>_ Проверочная работа _<<====
\begin{exam}
	\begin{listofex}
		\item Выполните действия: \( (185,899-54,623):7,3+1,6\cdot1,4+12,43 \)
		\item Выполните действия: \( \mfrac{4}{16}{32}-\mfrac{1}{7}{32}+\mfrac{3}{5}{32} \)
		\item Длина прямоугольника 48 см, что составляет \( \dfrac{3}{8} \) его периметра. Найдите ширину этого прямоугольника.
		\item В магазин привезли 325,271 кг фруктов. В первый день продали \( \dfrac{1}{5} \) всех фруктов, во второй день \( \dfrac{3}{5} \)  всех фруктов. Сколько килограммов фруктов продали за два дня?
		\item В банке  было налито \( 2,5 \)  литра молока, затем из нее вылили  \( 0,9 \) литра. Сколько литров молока осталось в банке?
		\item Отметь точки на координатной плоскости.\\\ \( (2; 3) \), \( (-6; 3) \), \( (-10; 1) \), \( (-10; -1) \), \( (-12; -1) \), \( (-15; -2) \), \( (-15; -6) \), \( (-11; -11) \), \( (-4; -11) \), \( (-4; -9) \), \( (-6;-7) \), \( (-11; -5) \), \( (-11; -4) \), \( (-10;-3) \), \( (-8; -3) \), \( (-7; -5) \), \( (1; -5) \), \( (1; -4) \), \( (5; -4) \), \( (5; -3) \), \( (7; -3) \), \( (7; -2) \), \( (5; -2) \), \( (5; -1) \), \( (8; 1) \), \( (8; 2) \), \( (7; 4) \), \( (7; 5) \), \( (6; 4) \), \( (6; 5) \), \( (5; 4) \), \( (3; 4) \), \( (2; 3) \). Глаз \( (6; 2) \).\\\ Соедини точки по порядку.
	\end{listofex}
\end{exam}
%END_FOLD