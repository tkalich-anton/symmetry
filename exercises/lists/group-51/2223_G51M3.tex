%
%===============>> ГРУППА 5-1 МОДУЛЬ 3 <<=============
%
\setmodule{3}
%
%===============>>  Занятие 1  <<===============
%
\begin{class}[number=1]
	\begin{listofex}
		\item Сложить или вычесть дроби с одинаковым числителем:
		\begin{tasks}(4)
			\task \( \dfrac{7}{9}+\dfrac{5}{9} \)
			\task \( \dfrac{16}{5}-\dfrac{7}{5} \)
			\task \( \dfrac{34}{99}-\dfrac{15}{99}-\dfrac{5}{99} \)
			\task \( \dfrac{13}{5}+\dfrac{37}{5}-\dfrac{40}{5} \)
		\end{tasks}
		\item Вычислить:
		\begin{tasks}(4)
			\task \( 1+\dfrac{5}{12} \)
			\task \( 5+\dfrac{4}{7} \)
			\task \( 13-\dfrac{6}{11} \)
			\task \( 25+\dfrac{31}{2} \)
			\task \( 4+\dfrac{3}{7}-\dfrac{5}{7} \)
			\task \( 101-\dfrac{101}{2} \)
			\task \( 21-\dfrac{21}{12} \)
			\task \( 10-\dfrac{10}{100} \)
		\end{tasks}
		\item Представьте число в виде неправильной дроби:\\
		\textit{Пример:} \( 1\:\dfrac{3}{4}=1+\dfrac{3}{4}=\dfrac{4}{4}+\dfrac{3}{4}=\dfrac{7}{4} \)
		\begin{tasks}(4)
			\task \( 5 \)
			\task \( 3\:\dfrac{7}{8} \)
			\task \( 7\:\dfrac{2}{5} \)
			\task \( 11\:\dfrac{5}{6} \)
			\task \( 7\:\dfrac{11}{10} \)
			\task \( 5\:\dfrac{9}{4} \)
			\task \( 1\:3\dfrac{2}{3} \)
			\task \( 3\:\dfrac{11}{14} \)
		\end{tasks}
		\item Выделите целую часть и представьте в виде смешанного числа:\\
		\textit{Пример:} \( \dfrac{7}{3}=2\:\dfrac{1}{3} \)
		\begin{tasks}(5)
			\task \( \dfrac{5}{2} \)
			\task \( \dfrac{47}{20} \)
			\task \( \dfrac{74}{13} \)
			\task \( \dfrac{45}{16} \)
			\task \( \dfrac{53}{3} \)
			\task \( \dfrac{98}{7} \)
			\task \( \dfrac{47}{4} \)
			\task \( \dfrac{132}{7} \)
			\task \( \dfrac{76}{13} \)
			\task \( \dfrac{563}{12} \)
		\end{tasks}
		\item Вычислить:
		\begin{tasks}(4)
			\task \( \dfrac{1}{2}+\dfrac{3}{10} \)
			\task \( \dfrac{3}{7}+\dfrac{5}{14} \)
			\task \( \dfrac{5}{16}+\dfrac{25}{32} \)
			\task \( \dfrac{5}{24}+\dfrac{17}{12} \)
			\task \( \dfrac{1}{4}+\dfrac{2}{3} \)
			\task \( \dfrac{11}{15}+\dfrac{7}{10} \)
			\task \( \dfrac{17}{70}+\dfrac{16}{7} \)
			\task \( \dfrac{5}{9}+\dfrac{24}{99} \)
		\end{tasks}
		\item Вычислить:
		\begin{tasks}(4)
			\task \( 4\dfrac{2}{5}+3\dfrac{1}{10} \)
			\task \( 6\dfrac{8}{9}+2\dfrac{5}{18} \)
			\task \( 28\dfrac{3}{4}-10\dfrac{2}{7} \)
			\task \( 11\dfrac{15}{17}+9\dfrac{12}{13} \)
		\end{tasks}
	\end{listofex}
\end{class}
%===============>>  Занятие 2  <<===============
\begin{class}[number=2]
	\begin{listofex}
		\item Вычислить:
		\begin{tasks}(4)
			\task \( 4+\dfrac{3}{10} \)
			\task \( 2+\dfrac{12}{17} \)
			\task \( 5-\dfrac{3}{5} \)
			\task \( 22+\dfrac{14}{3} \)
			\task \( 5+\dfrac{7}{9}-\dfrac{4}{9} \)
			\task \( 99-\dfrac{99}{2} \)
			\task \( 12-\dfrac{12}{13} \)
			\task \( 16-\dfrac{16}{160} \)
		\end{tasks}
		\item Представьте число в виде неправильной дроби:
		\begin{tasks}(4)
			\task \( \mfrac{2}{3}{5} \)
			\task \( \mfrac{8}{2}{15} \)
			\task \( \mfrac{4}{8}{25} \)
			\task \( \mfrac{14}{1}{17} \)
			\task \( \mfrac{10}{10}{101} \)
			\task \( \mfrac{3}{5}{6} \)
			\task \( \mfrac{13}{1}{1000} \)
			\task \( \mfrac{5}{9}{99} \)
		\end{tasks}
		\item Выделите целую часть и представьте в виде смешанного числа:
		\begin{tasks}(5)
			\task \( \dfrac{7}{3} \)
			\task \( \dfrac{55}{12} \)
			\task \( \dfrac{56}{6} \)
			\task \( \dfrac{54}{9} \)
			\task \( \dfrac{31}{5} \)
			\task \( \dfrac{101}{10} \)
			\task \( \dfrac{239}{51} \)
			\task \( \dfrac{154}{15} \)
			\task \( \dfrac{101}{100} \)
			\task \( \dfrac{909}{102} \)
		\end{tasks}
		\item Вычислить
		\begin{tasks}(4)
			\task \( \dfrac{2}{7}+\dfrac{4}{21} \)
			\task \( \dfrac{13}{18}-\dfrac{5}{9} \)
			\task \( \dfrac{4}{20}+\dfrac{7}{10} \)
			\task \( \dfrac{6}{51}-\dfrac{1}{17} \)
			\task \( \dfrac{31}{50}+\dfrac{21}{5} \)
			\task \( \dfrac{4}{3}-\dfrac{3}{4} \)
			\task \( \dfrac{6}{15}+\dfrac{9}{10} \)
			\task \( \dfrac{35}{99}-\dfrac{151}{990} \)
		\end{tasks}
		\item Сравнить:
		\begin{tasks}(2)
			\task \( \dfrac{1}{12}+\dfrac{5}{6} \) и \( \dfrac{3}{4}+\dfrac{2}{12} \)
			\task \( \dfrac{12}{30}+\dfrac{2}{5} \) и \( \dfrac{4}{6}+\dfrac{7}{30} \)
		\end{tasks}
		\item Вычислить:
		\begin{tasks}(4)
			\task \( \mfrac{3}{5}{12}+\mfrac{2}{2}{6} \)
			\task \( \mfrac{4}{3}{14}+\mfrac{2}{3}{7} \)
			\task \( \mfrac{4}{17}{20}-\mfrac{2}{12}{20} \)
			\task \( \mfrac{10}{15}{30}-\mfrac{3}{2}{10} \)
		\end{tasks}
		\item Из двух городов одновременно навстречу друг другу выехали автобус и легковая
		машина. Автобус проезжает весь путь за \( 12 \) часов, а легковая машина --- за \( 6 \) часов. Через сколько часов они встретятся?
	\end{listofex}
\end{class}
%===============>>  Домашняя работа 1  <<===============
\begin{homework}[number=1]
	\begin{listofex}
		\item Вычислить:
		\begin{tasks}(4)
			\task \( 1-\dfrac{9}{11} \)
			\task \( \mfrac{6}{3}{4}+\mfrac{2}{5}{8} \)
			\task \( \mfrac{8}{6}{13}-\mfrac{3}{9}{26} \)
			\task \( \mfrac{9}{1}{3}-\mfrac{8}{14}{15} \)
		\end{tasks}
		\item Решить уравнение:
		\begin{tasks}(3)
			\task \( x+\mfrac{3}{1}{5}=\mfrac{5}{2}{5} \)
			\task \( \mfrac{4}{1}{17}+x=\dfrac{5}{68} \)
			\task \( x-\mfrac{7}{5}{18}=\mfrac{9}{1}{18} \)
		\end{tasks}
		\item Вычислить рациональным образом:
		\begin{tasks}(2)
			\task \( \mfrac{7}{13}{14}-\mfrac{4}{17}{25}-\mfrac{2}{13}{14} \)
			\task \( \mfrac{5}{16}{39}+\mfrac{1}{6}{11}-\mfrac{2}{16}{39} \)
		\end{tasks}
		\item Вычислить:
		\begin{tasks}(4)
			\task \( \dfrac{2}{7}\cdot6 \)
			\task \( \dfrac{15}{4}\cdot8 \)
			\task \( \dfrac{3}{11}\cdot11 \)
			\task \( \dfrac{4}{20}\cdot10 \)
			\task \( \dfrac{13}{5}\cdot40 \)
			\task \( 13\cdot\dfrac{1}{11} \)
			\task \( 4\cdot\dfrac{1}{10} \)
			\task \( 200\cdot\dfrac{3}{200} \)
		\end{tasks}
		\item Вычислить:
		\begin{tasks}(4)
			\task \( \mfrac{2}{3}{10}\cdot5 \)
			\task \( \mfrac{5}{4}{21}\cdot3 \)
			\task \( \mfrac{4}{19}{55}\cdot11 \)
			\task \( 3\cdot\mfrac{4}{3}{13} \)
			\task \( \mfrac{1}{7}{30}\cdot45 \)
			\task \( \mfrac{8}{5}{6}\cdot6 \)
			\task \( 19\cdot\mfrac{3}{1}{57} \)
			\task \( \mfrac{5}{17}{21}\cdot3 \)
		\end{tasks}
		\item Вычислить:\quad\( 44\cdot\left( \mfrac{5}{8}{33}-\mfrac{4}{13}{22} \right)-5\cdot\left( \mfrac{8}{7}{15}-\mfrac{7}{9}{10} \right) \)
		\item Сравнить дроби:\quad\( \dfrac{11}{6} \) и \( \dfrac{7}{4} \)
		\item Вычислить:
		\begin{tasks}(3)
			\task \( \dfrac{1}{2}\cdot\dfrac{2}{7} \)
			\task \( \dfrac{2}{3}\cdot\dfrac{7}{5} \)
			\task \( \dfrac{7}{5}\cdot\dfrac{4}{7} \)
			\task \( \dfrac{12}{13}\cdot\dfrac{13}{12} \)
			\task \( \dfrac{7}{5}\cdot\dfrac{15}{14} \)
			\task \( \dfrac{8}{15}\cdot\dfrac{25}{28} \)
		\end{tasks}
	\end{listofex}
\end{homework}
%===============>>  Занятие 3  <<===============
\begin{class}[number=3]
	\begin{listofex}
		\item Вычислить:
		\begin{tasks}(4)
			\task \( 5-\dfrac{6}{7} \)
			\task \( \mfrac{3}{5}{6}-\mfrac{1}{4}{9} \)
			\task \( \mfrac{9}{11}{16}+\mfrac{3}{5}{24} \)
			\task \( \mfrac{27}{3}{8}+\mfrac{19}{63}{64} \)
		\end{tasks}
		\item Решить уравнение:
		\begin{tasks}(3)
			\task \( x+\mfrac{3}{2}{5}=\mfrac{5}{1}{5} \)
			\task \( \mfrac{4}{3}{8}+x=\mfrac{9}{1}{12} \)
			\task \( x-\mfrac{9}{11}{12}=\mfrac{7}{5}{24} \)
		\end{tasks}
		\item Вычислить рациональным образом:
		\begin{tasks}(2)
			\task \( \mfrac{3}{19}{24}+\mfrac{5}{1}{9}+\mfrac{1}{5}{24} \)
			\task \( \mfrac{4}{7}{45}+\mfrac{11}{4}{13}+\mfrac{8}{5}{26}+\mfrac{10}{2}{5} \)
		\end{tasks}
		\item Вычислить:
		\begin{tasks}(4)
			\task \( \dfrac{2}{5}\cdot2 \)
			\task \( \dfrac{11}{3}\cdot3 \)
			\task \( \dfrac{4}{9}\cdot12 \)
			\task \( \dfrac{6}{7}\cdot14 \)
			\task \( \dfrac{7}{30}\cdot45 \)
			\task \( 17\cdot\dfrac{1}{8} \)
			\task \( 15\cdot\dfrac{1}{10} \)
			\task \( 100\cdot\dfrac{1}{200} \)
		\end{tasks}
		\item Вычислить:
		\begin{tasks}(4)
			\task \( \mfrac{3}{5}{12}\cdot15 \)
			\task \( \mfrac{4}{1}{3}\cdot3 \)
			\task \( \mfrac{1}{6}{25}\cdot20 \)
			\task \( \mfrac{3}{1}{30}\cdot27 \)
			\task \( \mfrac{16}{23}{100}\cdot100 \)
			\task \( 87\cdot\mfrac{3}{2}{29} \)
			\task \( \mfrac{11}{2}{16}\cdot5 \)
			\task \( 25\cdot\mfrac{2}{2}{150} \)
		\end{tasks}
		\item Вычислить:
		\begin{tasks}(1)
			\task \( 10\cdot\left( \mfrac{3}{2}{15}-\mfrac{2}{5}{18} \right)+12\cdot\left( \mfrac{1}{5}{6}+\mfrac{5}{3}{4} \right) \)
			\task \( \left( \dfrac{3}{19}+\dfrac{5}{38} \right)\cdot57+\left( \dfrac{7}{36}+\dfrac{5}{54} \right)\cdot18-3\cdot\left( \dfrac{3}{4}+\dfrac{5}{6} \right) \)
		\end{tasks}
		\item Вычислить:
		\begin{tasks}(3)
			\task \( \dfrac{1}{2}\cdot\dfrac{2}{7} \)
			\task \( \dfrac{2}{3}\cdot\dfrac{7}{5} \)
			\task \( \dfrac{7}{5}\cdot\dfrac{4}{7} \)
			\task \( \dfrac{12}{13}\cdot\dfrac{13}{12} \)
			\task \( \dfrac{7}{5}\cdot\dfrac{15}{14} \)
			\task \( \dfrac{8}{15}\cdot\dfrac{25}{28} \)
		\end{tasks}
	\end{listofex}
\end{class}

%===============>>  Занятие 4  <<===============
\begin{class}[number=4]
	\begin{listofex}
		\item Вычислить:
		\begin{tasks}(4)
			\task \( 4-\dfrac{3}{9} \)
			\task \( \mfrac{5}{5}{6}-\mfrac{2}{5}{9} \)
			\task \( \mfrac{7}{3}{10}+\mfrac{4}{6}{15} \)
			\task \( \mfrac{41}{5}{6}+\mfrac{5}{13}{36} \)
		\end{tasks}
		\item Решить уравнение:
		\begin{tasks}(3)
			\task \( x+\mfrac{2}{1}{7}=\mfrac{3}{6}{7} \)
			\task \( \mfrac{13}{4}{11}+x=\mfrac{25}{10}{11} \)
			\task \( x-\mfrac{14}{5}{9}=\mfrac{3}{15}{18} \)
		\end{tasks}
		\item Решить уравнение:
		\begin{tasks}(2)
			\task \( \dfrac{1}{15}x+\mfrac{3}{2}{5}=\mfrac{7}{3}{5} \)
			\task \( \mfrac{2}{3}{9}-\dfrac{5}{18}x=\mfrac{1}{3}{6} \)
		\end{tasks}
		\item Вычислить:
		\begin{tasks}(4)
			\task \( \dfrac{3}{7}\cdot3 \)
			\task \( \dfrac{13}{6}\cdot6 \)
			\task \( \dfrac{12}{11}\cdot11 \)
			\task \( \dfrac{16}{20}\cdot10 \)
			\task \( \dfrac{9}{30}\cdot60 \)
			\task \( 14\cdot\dfrac{5}{21} \)
			\task \( 26\cdot\dfrac{6}{13} \)
			\task \( 1000\cdot\dfrac{3}{2000} \)
		\end{tasks}
		\item Вычислить:
		\begin{tasks}
			\task \( 7\cdot\left( \mfrac{6}{8}{21}+\mfrac{4}{11}{14} \right)-11\cdot\left( \mfrac{3}{3}{22}-\mfrac{2}{37}{44} \right) \)
			\task \( \mfrac{100}{11}{26}\left( \mfrac{73}{3}{13}-\mfrac{69}{25}{26} \right) \)
		\end{tasks}
		\item Вычислить:
		\begin{tasks}(6)
			\task \( \dfrac{1}{3}\cdot\dfrac{3}{5} \)
			\task \( \dfrac{4}{9}\cdot\dfrac{18}{5} \)
			\task \( \dfrac{12}{1}\cdot\dfrac{5}{2} \)
			\task \( \dfrac{3}{17}\cdot\dfrac{51}{7} \)
			\task \( \dfrac{4}{11}\cdot\dfrac{77}{3} \)
			\task \( \dfrac{12}{15}\cdot\dfrac{35}{16} \)
		\end{tasks}
		\item Вычислить:
		\begin{tasks}(2)
			\task \( \dfrac{4}{5}\cdot\dfrac{3}{8}\cdot\dfrac{3}{5}\cdot\dfrac{3}{4}\cdot\dfrac{2}{3} \)
			\task \( \mfrac{3}{1}{2}\cdot\mfrac{8}{1}{3}\cdot\dfrac{3}{25}\cdot5\cdot\mfrac{6}{1}{4}\cdot16 \)
		\end{tasks}
	\end{listofex}
\end{class}

%===============>>  Домашняя работа 2  <<===============
\begin{homework}[number=2]
	\begin{listofex}
		\item Вычислите:
		\begin{tasks}(3)
			\task \( \dfrac{1}{6}+\dfrac{1}{5} \)
			\task \( \dfrac{10}{11}-\dfrac{2}{3} \)
			\task \( \dfrac{2}{9}+\dfrac{1}{18} \)
			\task \( \dfrac{3}{7}+\dfrac{2}{8} \)
			\task \( \dfrac{1}{4}+4 \)
			\task \( 3-\dfrac{5}{16} \)
		\end{tasks}
		\item Вычислите:
		\begin{tasks}(3)
			\task \( \dfrac{1}{8}\cdot\dfrac{1}{6} \)
			\task \( \dfrac{1}{5}\cdot\dfrac{3}{4} \)
			\task \( \dfrac{2}{9}\cdot\dfrac{2}{9} \)
			\task \( \dfrac{3}{4}\cdot\dfrac{61}{100} \)
			\task \( \dfrac{20}{87}\cdot\dfrac{30}{89} \)
			\task \( \dfrac{88}{97}\cdot\dfrac{125}{163} \)
		\end{tasks}
		\item Приведя к общему знаменателю расположите дроби в порядке возрастания:
		\[ \dfrac{1}{6};\;\dfrac{7}{15};\;\dfrac{1}{12};\;\dfrac{3}{10};\;\dfrac{1}{4};\;\dfrac{14}{15};\;\dfrac{1}{3};\;\dfrac{2}{15};\;\dfrac{1}{2};\;\dfrac{1}{10};\;\dfrac{3}{4}. \]
		\item Вычислите:
		\begin{tasks}(3)
			\task \(\mfrac{1}{2}{3}+\mfrac{1}{1}{3} \)
			\task \(\mfrac{2}{5}{7}-\mfrac{1}{4}{7} \)
			\task \( \mfrac{2}{3}{8}+\mfrac{5}{1}{4}\)
			\task \( \mfrac{8}{10}{17}-\mfrac{1}{4}{5}\)
			\task \( \mfrac{3}{2}{71}+\mfrac{2}{1}{142} \)
			\task \( \mfrac{6}{3}{7}-\mfrac{6}{1}{4}\)
		\end{tasks}
		\item Вычислите:
		\begin{tasks}(3)
			\task \(\mfrac{1}{2}{3}\cdot\mfrac{3}{1}{3} \)
			\task \(\mfrac{2}{3}{7}\cdot\mfrac{8}{4}{7} \)
			\task \( \mfrac{2}{3}{4}\cdot\mfrac{10}{1}{8}\)
			\task \( \mfrac{8}{5}{12}\cdot\mfrac{1}{4}{5}\)
			\task \( \mfrac{3}{3}{605}\cdot\mfrac{2}{1}{2} \)
			\task \( \mfrac{6}{1}{7}\cdot\mfrac{40}{1}{4}\)
		\end{tasks}
	\item Из двух городов одновременно навстречу друг другу выехали автобус и легковая
	машина. Автобус проезжает весь путь за \( 18 \) часов, а легковая машина --- за \( 9 \) часов. Через сколько часов они встретятся?
	\end{listofex}
\end{homework}

%
%===============>>  Занятие 5  <<===============
%
\begin{class}[number=5]
	\begin{listofex}
		\item Представьте число в виде неправильной дроби:
		\begin{tasks}(6)
			\task \( 2\:\dfrac{2}{7} \)
			\task \( 6\:\dfrac{5}{12} \)
			\task \( 13\:\dfrac{1}{9} \)
			\task \( 3\:\dfrac{3}{22} \)
			\task \( 9\:\dfrac{9}{99} \)
			\task \( 101\:\dfrac{1}{101} \)
		\end{tasks}
		\item Выделите целую часть и представьте в виде смешанного числа:
		\begin{tasks}(6)
			\task \( \dfrac{13}{4} \)
			\task \( \dfrac{47}{11} \)
			\task \( \dfrac{51}{4} \)
			\task \( \dfrac{333}{101} \)
			\task \( \dfrac{54}{53} \)
			\task \( \dfrac{123}{10} \)
		\end{tasks}
		\item Вычислить:
		\begin{tasks}(4)
			\task \( 3+\dfrac{4}{7} \)
			\task \( 2+\dfrac{2}{5} \)
			\task \( 13-\dfrac{5}{6} \)
			\task \( 9+\dfrac{24}{3} \)
			\task \( 5+\dfrac{4}{5}-\dfrac{3}{5} \)
			\task \( 121-\dfrac{121}{3} \)
			\task \( 2-\dfrac{99}{100} \)
			\task \( 13-\dfrac{2}{3}-\dfrac{5}{3} \)
		\end{tasks}
		\item Вычислить
		\begin{tasks}(4)
			\task \( \dfrac{3}{5}+\dfrac{6}{20} \)
			\task \( \dfrac{15}{24}-\dfrac{4}{8} \)
			\task \( \dfrac{13}{30}+\dfrac{6}{10} \)
			\task \( \dfrac{17}{51}-\dfrac{5}{17} \)
			\task \( \dfrac{22}{50}+\dfrac{12}{5} \)
			\task \( \dfrac{13}{12}-\dfrac{3}{4} \)
			\task \( \dfrac{62}{100}+\dfrac{9}{10} \)
			\task \( \dfrac{123}{99}-\dfrac{114}{990} \)
		\end{tasks}
		\item Вычислить:
		\begin{tasks}(4)
			\task \( \mfrac{3}{7}{20}+\mfrac{4}{3}{20} \)
			\task \( \mfrac{9}{12}{13}+\mfrac{6}{1}{13} \)
			\task \( \mfrac{12}{13}{30}-\mfrac{7}{13}{30} \)
			\task \( \mfrac{5}{35}{100}-\mfrac{2}{21}{100} \)
		\end{tasks}
		\item Вычислить:
		\begin{tasks}(4)
			\task \( \mfrac{2}{5}{16}+\mfrac{2}{3}{4} \)
			\task \( \mfrac{1}{2}{15}+\mfrac{3}{1}{5} \)
			\task \( \mfrac{5}{21}{30}-\mfrac{3}{2}{3} \)
			\task \( \mfrac{10}{41}{50}-\mfrac{4}{5}{10} \)
		\end{tasks}
		\item Найти:
		\begin{tasks}(3)
			\task \( \dfrac{3}{20} \) от \( 100 \)
			\task \( \dfrac{2}{15} \) от \( 60 \)
			\task \( \dfrac{13}{17} \) от \( 102 \)
			\task \( \mfrac{1}{4}{21} \) от \( 42 \)
			\task \( \mfrac{3}{11}{12} \) от \( 48 \)
		\end{tasks}
	\end{listofex}
\end{class}

%===============>>  Занятие 6  <<===============
%
\begin{class}[number=6]
	\begin{listofex}
	\item Вычислите, применяя распределительный закон:
	\begin{tasks}(4)
		\itask \( 37\cdot12+37\cdot88 \)
		\task \( 18\cdot9+18\ \)
		\task \( 7\cdot102-2\cdot7 \)
		\task \( 41\cdot50-50 \)
	\end{tasks}
	\item Найдите НОД следующих пар чисел:
	\begin{tasks}(2)
		\task \( 24 \) и \( 44 \)
		\task \( 72 \) и \( 27 \)
	\end{tasks}
	\item Найдите НОК следующих пар чисел (помните, что НОК -- это НОД \( \cdot \) всё, что в него не вошло)
	\begin{tasks}(2)
		\task \( 36 \) и \( 48 \)
		\task \( 72 \) и \( 45 \)
	\end{tasks}
	\item Рабочий может выполнить заказ за \( 5 \) часов. Какую часть заказа выполнит рабочий за \( 1  \) час?
	\item \( \dfrac{3}{7} \) всех деревьев, растущих в саду, составляют вишни. Сколько вишен в саду, если в нем \( 42  \) дерева?
	\item Сократите дроби:
		\begin{tasks}(4)
		\task \( \dfrac{2}{10} \)
		\task \( \dfrac{15}{60} \)
		\task \( \dfrac{80}{90} \)
		\task \( \dfrac{78}{294} \)
	\end{tasks}
	\item Вычислите:
		\begin{tasks}(3)
		\task \( \dfrac{1}{5}+\dfrac{2}{7} \)
		\task \( \dfrac{7}{10}+\dfrac{1}{100} \)
		\task \( \dfrac{5}{8}+\dfrac{5}{16} \)
		\task \( \dfrac{17}{18}-\dfrac{2}{9} \)
		\task \( \dfrac{3}{5}-\dfrac{1}{4} \)
		\task \( \dfrac{7}{20}-\dfrac{1}{80} \)
	\end{tasks}
	\item Вычислите:
		\begin{tasks}(4)
		\task \( \mfrac{1}{2}{5}-\mfrac{1}{1}{6} \)
		\task \( \mfrac{3}{2}{7}+\mfrac{1}{3}{14} \)
		\task \( \mfrac{1}{1}{2}+\mfrac{2}{3}{4} \)
		\task \( \mfrac{6}{2}{5}-\mfrac{3}{9}{10} \)	
	\end{tasks}
	\item Вычислите:
		\begin{tasks}(4)
		\task \( \dfrac{1}{5}\cdot\dfrac{2}{5} \)
		\task \( \dfrac{2}{3}\cdot\dfrac{5}{6} \)
		\task \( \dfrac{5}{6}\cdot\dfrac{12}{25} \)
		\task \( \dfrac{3}{80}\cdot\dfrac{100}{291} \)
	\end{tasks}
	\item Вычислите:
		\begin{tasks}(4)
		\task \( \mfrac{1}{2}{7}\cdot\mfrac{1}{1}{3} \)
		\task \( \mfrac{4}{2}{3}\cdot\mfrac{2}{2}{5} \)
		\task \( \mfrac{3}{5}{6}\cdot\mfrac{1}{7}{23} \)
		\task \( \mfrac{7}{3}{11}\cdot\mfrac{2}{19}{40} \)
	\end{tasks}
	\item Выделите целую часть и представьте неправильную дробь в виде смешанного числа:
		\begin{tasks}(4)
		\task \( \dfrac{47}{9} \)
		\task \( \dfrac{27}{4} \)
		\task \( \dfrac{147}{11} \)
		\task \( \dfrac{158}{9} \)
	\end{tasks}
\end{listofex}
\end{class}
%
%===============>>  Занятие 8  <<===============
%
\begin{class}[number=7]
	\begin{listofex}
	\item В автобусе \( 51 \) место. \( \dfrac{2}{3} \) мест уже занято. Сколько ещё пассажиров может сесть в автобус на оставшиеся места?
	\item Максим загадал число. Известно, что \( 12 \) составляет \( \dfrac{4}{7} \) от загаданного числа. Какое число загадал Максим?
	\item Вычислите:
	\begin{tasks}(4)
		\task \( 2+\dfrac{3}{8} \)
		\task \( 3+\dfrac{1}{4} \)
		\task \( 1+\dfrac{1}{9} \)
		\task \( 3+\dfrac{9}{10} \)
		\task \( 4-\dfrac{1}{3} \)
		\task \( 5-\dfrac{3}{5} \)
		\task \( 10-\dfrac{3}{7} \)
		\task \( 1-\dfrac{1}{2} \)
	\end{tasks}
	\item Вычислите:
		\begin{tasks}(4)
		\task \( \dfrac{1}{3}+\dfrac{1}{3} \)
		\task \( \dfrac{4}{5}-\dfrac{2}{5} \)
		\task \( \dfrac{2}{5}+\dfrac{1}{5} \)
		\task \( \dfrac{6}{7}-\dfrac{2}{7} \)
		\task \( \dfrac{3}{8}+\dfrac{1}{2} \)
		\task \( \dfrac{2}{7}-\dfrac{1}{14} \)
		\task \( \dfrac{1}{6}+\dfrac{1}{5} \)
		\task \( \dfrac{5}{6}-\dfrac{1}{7} \)
	\end{tasks}
	\item Представьте число в виде неправильной дроби:
		\begin{tasks}(6)
		\task \( \mfrac{1}{3}{4} \)
		\task \( \mfrac{6}{3}{7} \)
		\task \( \mfrac{3}{7}{8} \)
		\task \( \mfrac{11}{11}{11} \)
		\task \( \mfrac{4}{12}{13} \)
		\task \( \mfrac{3}{5}{37} \)
	\end{tasks}
	\item Выделите целую часть и запишите ответ в виде смешанного числа:
		\begin{tasks}(6)
		\task \( \dfrac{3}{2} \)
		\task \( \dfrac{6}{2} \)
		\task \( \dfrac{11}{7} \)
		\task \( \dfrac{54}{5} \)
		\task \( \dfrac{45}{16} \)
		\task \( \dfrac{113}{4} \)
	\end{tasks}
	\item Вычислите:
		\begin{tasks}(2)
		\task \( \mfrac{8}{1}{9}+\mfrac{8}{2}{9}+\mfrac{5}{5}{9} \)
		\task \( \mfrac{14}{1}{23}+\mfrac{12}{4}{23}+\mfrac{1}{11}{23} \)
		\task \( 11+\dfrac{40}{41}+\mfrac{12}{3}{41}+\mfrac{8}{39}{41}+\dfrac{4}{41} \)
		\task \( \mfrac{11}{7}{19}-\mfrac{7}{3}{19} \)
		\task \( \mfrac{17}{39}{43}-\mfrac{17}{5}{43} \)
		\task \( \mfrac{77}{1}{2}-77 \)
	\end{tasks}
		\item Вычислите:
		\begin{tasks}(3)
		\task \( \mfrac{4}{1}{7}+\mfrac{5}{4}{9} \)
		\task \( \mfrac{12}{6}{7}-\mfrac{3}{5}{10} \)
		\task \( \mfrac{12}{1}{6}+\mfrac{8}{16}{21} \)
		\task \( \mfrac{28}{3}{4}-\mfrac{10}{2}{7} \)
		\task \( \mfrac{5}{16}{33}+\mfrac{1}{6}{11} \)
		\task \( \mfrac{188}{4}{9}-\mfrac{56}{3}{10} \)
		\end{tasks}
\end{listofex}
\end{class}

%BEGIN_FOLD % ====>>_ Консультация _<<====
\begin{consultation}
	\begin{listofex}
	\item У фермера были куры и овцы. Сколько было кур и сколько овец, если у
	них 30 голов и 74 ноги?
	\item Лёва стоит на берегу речки. У него есть два кувшина: один на 5 л, а про второй Лёва помнит, что он вмещает то ли 3 л, то ли 4 л. Помогите Лёве определить ёмкость второго
	кувшина.
	\item Писатель без перерыва писал роман \( 100 \) часов. Он заметил, что в первый час в комнату влетел один комар, в следующий час два комара, в третий час --- три комара и так далее. За время написания романа ни один комар из комнаты не вылетел. Сколько комаров было в комнате к концу написания романа?
	\item В трех больших коробках и семи маленьких 49 кг печенья. Сколько кг печенья в большой коробке и сколько в маленькой, если в одной большой коробке на 3 кг печенья больше, чем в
	одной маленькой?
	\item Вычислите \( 1+3+5+7+...+29+31 \).
	\item Учитель истории поставил в 3 раза больше пятерок, чем учитель математики, а учитель
	словесности – на 9 пятерок больше, чем учитель математики. Сколько пятерок поставил
	каждый учитель, если всего они поставили 74 пятерки?
\end{listofex}
\end{consultation}
%END_FOLD