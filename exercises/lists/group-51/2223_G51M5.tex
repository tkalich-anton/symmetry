%
%===============>>  ГРУППА 5-1 МОДУЛЬ 4  <<=============
%
\setmodule{5}

%BEGIN_FOLD % ====>>_____ Занятие 1 _____<<====
\begin{class}[number=1]
	
	\begin{listofex}
		\item Одноместная байдарка проплывает дистанцию гребного канала за \( 28 \) секунд, а двухместная --- за \( 21 \) секунду. Обе байдарки стартовали одновременно с противоположных концов канала. Через сколько секунд они встретятся?
		\item За \( 10 \) дней пират Ерема\\
		Способен выпить бочку рома.\\
		А у пирата, у Емели,\\
		Уйдет на это две недели.\\
		За сколько дней прикончат ром\\
		Пираты, действуя вдвоем?
		\item Объем бассейна \( 1200 \) м\( ^3 \). Первая труба заполняет бассейн за \( 80 \) минут, а вторая --- за \( 48 \) минут. За сколько минут наполнят бассейн обе трубы, работая вдвоем?
		\item От двух пристаней одновременно навстречу друг другу вышли катер и теплоход. Катер проплывает расстояние между пристанями за \( 24 \) часа, а теплоход --- за \( 40 \) часов. Через сколько часов они встретятся?
		\item Две бригады, работая совместно, закончили посадку деревьев за \( 12 \) дней. Сколько дней потребуется на выполнение этой работы одной первой бригаде, если она может выполнить ее \( \mfrac{1}{1}{2} \) раза быстрее второй?
		\item Вася отвечает за час на \( 8 \) вопросов теста, а Аня --- на \( 9 \). Они одновременно начали отвечать на вопросы теста, и Вася закончил позже Ани на \( 10 \) минут. Сколько вопросов содержит тест?
		\item Вычислите: \quad \( \left( \mfrac{4}{4}{65}\cdot \mfrac{8}{28}{55}-\mfrac{16}{1}{5}\cdot\dfrac{21}{25}-\dfrac{6}{125} \right)\cdot\left( \mfrac{14}{8}{11}:\dfrac{26}{77}:405 \right) \)
	\end{listofex}
\end{class}
%END_FOLD

%BEGIN_FOLD % ====>>_____ Занятие 2 _____<<====
\begin{class}[number=2]
	\begin{listofex}
		\item На Новогодние праздники Дед Мороз раздал в детском саду \( 72 \) подарка, а в школе в три раза больше. Сколько всего подарков раздал Дед Мороз в школе и детском саде? На сколько детей больше в школе, если Дед Мороз каждому ребенку давал по одному подарку?
		\item Дед Мороз и Снегурочка созвонились и договорились вместе вручать новогодние подарки в школе села Снегово. Но Снегурочка в это время была в деревне Внуковская, а Дед Мороз в деревне Дедовская. От Внуковской до Снегово \( 16 \) км, а от Дедовской до Снегово \( 12 \) км. Кто первым окажется в Снегово, если Дед Мороз идет против ветра со скоростью \( 4 \) км/час, а Снегурочка по ветру со скоростью в два раза больше?
		\item Снеговик решил прокатиться на аэросанях. А чтобы не было жарко, кабину снял. Покатался он с ветерком! Но каждую минуту ветром из него выдувало \( 200 \) грамм снега. На сколько килограмм «похудел» Снеговик, если катался он ровно один час?
		\item Дед Мороз купил холодильную камеру для хранения подарков. Объем камеры один кубометр. Сколько подарков войдет в эту камеру, если каждый из них занимает место в \( 2 \) кубических дециметра.
		\item Василиса Прекрасная испекла огромный Новогодний торт. Но пока она приводила себя в порядок, Кот в Сапогах съел половину торта, а затем Микки Маус еще половину от оставшегося куска. Найдите вес оставшегося куска, если первоначально торт весил \( 20 \) килограмм.
		\item Баба Яга за новогодние праздники от злости худеет на \( 15 \) килограммов. Но все равно, даже после праздников, она весит на \( 40 \) килограммов больше, чем Кощей. Какой вес был у Бабы Яги до праздников, если она весила ровно в шесть раз больше Кощея. Вес Кощея не изменяется уже сто лет.
		\item С первого по седьмое января Баба Яга ежедневно летала за спичками в Тридевятое царство. Перед первым полетом спидометр на ее метле показывал \( 123445 \) км, а при возвращении седьмого \( 193445 \) км. Каково расстояние от избушки Бабы Яги до	Тридевятого царства?
		\item Кощей Бессмертный первого января посадил в темницу две Снежинки, второго в два раза больше, а третьего в два раза больше, чем второго. Сколько Снежинок оказалось в темнице за три дня?
		\item Устав после Новогодних праздников, Дед Мороз и Снегурочка решили отдохнуть, Дед проспал \( 16 \) часов \( 38 \) минут. Снегурочка на \( 6 \) часов и \( 46 \) минут больше. Сколько времени спала Снегурочка?
		\item Серый Волк и Лиса бежали новогодний марафон. Стартовали они одновременно у Новогодней ёлочки \( 1 \) января в \( 12 \) часов \( 30 \) минут. Серый Волк добежал до Лукоморья \( 4 \) января в \( 14 \) часов, а Лиса \( 5 \) января в \( 6 \) часов. Сколько времени была в пути лиса? На сколько часов Серый Волк опередил Лису?
	\end{listofex}
\end{class}
%END_FOLD

%BEGIN_FOLD % ====>>_ Домашняя работа 1 _<<====
\begin{homework}[number=1]
	\begin{listofex}
		\item Устав после Новогодних праздников, Дед Мороз и Снегурочка решили отдохнуть, Дед проспал \( 16 \) часов \( 38 \) минут. Снегурочка на \( 6 \) часов и \( 46 \) минут больше. Сколько времени спала Снегурочка?
		\item Максим и Костя вскапывают грядку за \( 10 \) минут, а один Максим --- за \( 15 \) минут. За сколько минут	вскапывает грядку один Костя?
		\item Винни-Пух съедает горшочек меда за \( 6 \) минут, Пятачок --- за \( 20 \) минут, а ослик Иа --- за \( 30 \) минут. За сколько минут они съедят горшочек меда втроем?
		\item Вычислите: \quad \( \left( \mfrac{3}{1}{2}:\mfrac{4}{2}{3}+\mfrac{4}{2}{3}:\mfrac{3}{1}{2} \right)\cdot\mfrac{4}{4}{5} \)
	\end{listofex}
\end{homework}
%END_FOLD

%BEGIN_FOLD % ====>>_____ Занятие 3 _____<<====
\begin{class}[number=3]
	\begin{listofex}
		\item Занятие 3
	\end{listofex}
\end{class}
%END_FOLD

%BEGIN_FOLD % ====>>_____ Занятие 4 _____<<====
\begin{class}[number=4]
	\begin{listofex}
		\item Занятие 4
	\end{listofex}
\end{class}
%END_FOLD

%BEGIN_FOLD % ====>>_ Домашняя работа 2 _<<====
\begin{homework}[number=2]
	\begin{listofex}
		\item ДЗ 2
	\end{listofex}
\end{homework}
%END_FOLD

%BEGIN_FOLD % ====>>_____ Занятие 5 _____<<====
\begin{class}[number=5]
	\begin{listofex}
		\item Занятие 5
	\end{listofex}
\end{class}
%END_FOLD

%BEGIN_FOLD % ====>>_____ Занятие 6 _____<<====
\begin{class}[number=6]
	\begin{listofex}
		\item Занятие 6
	\end{listofex}
\end{class}
%END_FOLD

%BEGIN_FOLD % ====>>_ Домашняя работа 3 _<<====
\begin{homework}[number=3]
	\begin{listofex}
		\item ДЗ 3
	\end{listofex}
\end{homework}
%END_FOLD

%BEGIN_FOLD % ====>>_____ Занятие 7 _____<<====
\begin{class}[number=7]
	\begin{listofex}
		\item Занятие 7
	\end{listofex}
\end{class}
%END_FOLD

%BEGIN_FOLD % ====>>_ Проверочная работа _<<====
\begin{exam}
	\begin{listofex}
		\item Проверочная работа
	\end{listofex}
\end{exam}
%END_FOLD