%
%===============>>  ГРУППА 5-1 МОДУЛЬ 4  <<=============
%
\setmodule{5}

%BEGIN_FOLD % ====>>_____ Занятие 1 _____<<====
\begin{class}[number=1]
	
	\begin{listofex}
		\item Одноместная байдарка проплывает дистанцию гребного канала за \( 28 \) секунд, а двухместная --- за \( 21 \) секунду. Обе байдарки стартовали одновременно с противоположных концов канала. Через сколько секунд они встретятся?
		\item За \( 10 \) дней пират Ерема\\
		Способен выпить бочку рома.\\
		А у пирата, у Емели,\\
		Уйдет на это две недели.\\
		За сколько дней прикончат ром\\
		Пираты, действуя вдвоем?
		\item Объем бассейна \( 1200 \) м\( ^3 \). Первая труба заполняет бассейн за \( 80 \) минут, а вторая --- за \( 48 \) минут. За сколько минут наполнят бассейн обе трубы, работая вдвоем?
		\item От двух пристаней одновременно навстречу друг другу вышли катер и теплоход. Катер проплывает расстояние между пристанями за \( 24 \) часа, а теплоход --- за \( 40 \) часов. Через сколько часов они встретятся?
		\item Две бригады, работая совместно, закончили посадку деревьев за \( 12 \) дней. Сколько дней потребуется на выполнение этой работы одной первой бригаде, если она может выполнить ее \( \mfrac{1}{1}{2} \) раза быстрее второй?
		\item Вася отвечает за час на \( 8 \) вопросов теста, а Аня --- на \( 9 \). Они одновременно начали отвечать на вопросы теста, и Вася закончил позже Ани на \( 10 \) минут. Сколько вопросов содержит тест?
		\item Вычислите: \quad \( \left( \mfrac{4}{4}{65}\cdot \mfrac{8}{28}{55}-\mfrac{16}{1}{5}\cdot\dfrac{21}{25}-\dfrac{6}{125} \right)\cdot\left( \mfrac{14}{8}{11}:\dfrac{26}{77}:405 \right) \)
	\end{listofex}
\end{class}
%END_FOLD

%BEGIN_FOLD % ====>>_____ Занятие 2 _____<<====
\begin{class}[number=2]
	\begin{listofex}
		\item На Новогодние праздники Дед Мороз раздал в детском саду \( 72 \) подарка, а в школе в три раза больше. Сколько всего подарков раздал Дед Мороз в школе и детском саде? На сколько детей больше в школе, если Дед Мороз каждому ребенку давал по одному подарку?
		\item Дед Мороз и Снегурочка созвонились и договорились вместе вручать новогодние подарки в школе села Снегово. Но Снегурочка в это время была в деревне Внуковская, а Дед Мороз в деревне Дедовская. От Внуковской до Снегово \( 16 \) км, а от Дедовской до Снегово \( 12 \) км. Кто первым окажется в Снегово, если Дед Мороз идет против ветра со скоростью \( 4 \) км/час, а Снегурочка по ветру со скоростью в два раза больше?
		\item Снеговик решил прокатиться на аэросанях. А чтобы не было жарко, кабину снял. Покатался он с ветерком! Но каждую минуту ветром из него выдувало \( 200 \) грамм снега. На сколько килограмм «похудел» Снеговик, если катался он ровно один час?
		\item Дед Мороз купил холодильную камеру для хранения подарков. Объем камеры один кубометр. Сколько подарков войдет в эту камеру, если каждый из них занимает место в \( 2 \) кубических дециметра.
		\item Василиса Прекрасная испекла огромный Новогодний торт. Но пока она приводила себя в порядок, Кот в Сапогах съел половину торта, а затем Микки Маус еще половину от оставшегося куска. Найдите вес оставшегося куска, если первоначально торт весил \( 20 \) килограмм.
		\item Баба Яга за новогодние праздники от злости худеет на \( 15 \) килограммов. Но все равно, даже после праздников, она весит на \( 40 \) килограммов больше, чем Кощей. Какой вес был у Бабы Яги до праздников, если она весила ровно в шесть раз больше Кощея. Вес Кощея не изменяется уже сто лет.
		\item С первого по седьмое января Баба Яга ежедневно летала за спичками в Тридевятое царство. Перед первым полетом спидометр на ее метле показывал \( 123445 \) км, а при возвращении седьмого \( 193445 \) км. Каково расстояние от избушки Бабы Яги до	Тридевятого царства?
		\item Кощей Бессмертный первого января посадил в темницу две Снежинки, второго в два раза больше, а третьего в два раза больше, чем второго. Сколько Снежинок оказалось в темнице за три дня?
		\item Устав после Новогодних праздников, Дед Мороз и Снегурочка решили отдохнуть, Дед проспал \( 16 \) часов \( 38 \) минут. Снегурочка на \( 6 \) часов и \( 46 \) минут больше. Сколько времени спала Снегурочка?
		\item Серый Волк и Лиса бежали новогодний марафон. Стартовали они одновременно у Новогодней ёлочки \( 1 \) января в \( 12 \) часов \( 30 \) минут. Серый Волк добежал до Лукоморья \( 4 \) января в \( 14 \) часов, а Лиса \( 5 \) января в \( 6 \) часов. Сколько времени была в пути лиса? На сколько часов Серый Волк опередил Лису?
	\end{listofex}
\end{class}
%END_FOLD

%BEGIN_FOLD % ====>>_ Домашняя работа 1 _<<====
\begin{homework}[number=1]
	\begin{listofex}
		\item Устав после Новогодних праздников, Дед Мороз и Снегурочка решили отдохнуть, Дед проспал \( 16 \) часов \( 38 \) минут. Снегурочка на \( 6 \) часов и \( 46 \) минут больше. Сколько времени спала Снегурочка?
		\item Максим и Костя вскапывают грядку за \( 10 \) минут, а один Максим --- за \( 15 \) минут. За сколько минут	вскапывает грядку один Костя?
		\item Винни-Пух съедает горшочек меда за \( 6 \) минут, Пятачок --- за \( 20 \) минут, а ослик Иа --- за \( 30 \) минут. За сколько минут они съедят горшочек меда втроем?
		\item Вычислите: \quad \( \left( \mfrac{3}{1}{2}:\mfrac{4}{2}{3}+\mfrac{4}{2}{3}:\mfrac{3}{1}{2} \right)\cdot\mfrac{4}{4}{5} \)
	\end{listofex}
\end{homework}
%END_FOLD

%BEGIN_FOLD % ====>>_____ Занятие 3 _____<<====
\begin{class}[number=3]
	\begin{definit}
		Сумма внутренних углов в треугольнике равна \( 180\degree \).
	\end{definit}
	\begin{definit}
		\textbf{Внешний угол} --- угол между стороной треугольника и продолжением другой стороны. Внешний угол является смежным с одним из внутренних.
	\end{definit}
	\begin{listofex}
		\item В треугольнике \( ABC \) два угла равны \( 50\) и \( 70 \) градусов. Найдите третий угол.
		\item Один угол треугольника равен \( 26\degree \), а второй в три раза больше. Найдите третий угол.
		\item Один внутренний угол треугольника в два, а второй в три раза больше третьего, найдите все углы треугольника.
		\item Один внешний угол равен \( 40\degree \), а второй --- \( 100 \) градусов. Чему равны углы в треугольнике? 
		\item Угол треугольника равен \( 30\degree \), второй угол в \( 3 \) раза больше первого. Чему равны внешние углы при каждой вершине? Чему равна сумма внешних углов?
		\item В прямоугольном треугольнике один угол равен \( 40 \) градусов. Найдите сумму наибольшего и наименьшего угла. 
		\item В прямоугольном треугольнике один острый угол на \( 17 \) градусов больше другого. Найдите углы треугольника.
		\item В прямоугольном треугольнике два острых угла равны. Какая у них градусная мера?
		\item В треугольнике \( ABC \) \( \angle A=60\degree \). Провели биссектрису \( AM \) из вершины треугольника \( A \). Чему равен угол \( BAM \)?
		\item В треугольнике \( KLM \) \( \angle K=75\degree \), \(\angle L=66\degree \). Провели биссектрисы \( KO \) и \( MP \). Найдите угол \( OMP \).
	\end{listofex}
\end{class}
%END_FOLD

%BEGIN_FOLD % ====>>_____ Занятие 4 _____<<====
\begin{class}[number=4]
	\begin{listofex}
		\item Периметр прямоугольника равен \( 20 \). Чему равна ширина прямоугольника, если его длина --- \( 6 \)?
		\item Длина прямоугольника в \( 3 \) раза меньше его ширины. Найдите их, если периметр равен \( 24 \).
		\item Периметр треугольника \( ABC \) равен \( 12 \). \( AB=4 \), другая сторона на 1 меньше \( AB \). Чему равна третья сторона?
		\item Периметр квадрата равен \( 16 \). Найдите его сторону.
		\item Найдите площадь прямоугольника, стороны которого равны \( 5 \) и \( 15 \).
		\item Найдите периметр прямоугольника, если его площадь равна \( 18 \) см\( ^2 \), а сторона --- \( 90 \) мм.
		\item Площадь квадрата равна \( 36 \) дм\( ^2 \). Каждую сторону квадрата увеличили на \( 2 \) дм. Найдите площадь нового квадрата.
		\item Вычислите: \quad \( \left( \mfrac{3}{3}{7}\cdot\mfrac{2}{5}{8}-16 \right):\mfrac{5}{1}{4}+\mfrac{11}{11}{15}  \)
	\end{listofex}
\end{class}
%END_FOLD

%BEGIN_FOLD % ====>>_ Домашняя работа 2 _<<====
\begin{homework}[number=2]
	\begin{listofex}
		\item В прямоугольном треугольнике один угол равен \( 30 \) градусов. Найдите сумму наибольшего и наименьшего угла. 
		\item В треугольнике \( OPQ \) угол \( O \) равен \( 150\degree \). Чему равен внешний угол при вершине \( O \)?
		\item На сколько площадь прямоугольника со сторонами \( 9 \) и \( 6 \) см больше его периметра?
		\item Периметр квадрата равен \( 64 \). Найдите его площадь.
		\item Вычислите: \quad \( \mfrac{12}{8}{9}:4+\dfrac{2}{3}\cdot\dfrac{1}{24} \)
	\end{listofex}
\end{homework}
%END_FOLD

%BEGIN_FOLD % ====>>_____ Занятие 5 _____<<====
\begin{class}[number=5]
	\begin{listofex}
		\item В прямоугольном треугольнике два острых угла равны. Чему равна градусная мера острых углов?
		\item В треугольнике \( ABC \) \( \angle A=60\degree \). Из вершины угла \( A \) провели биссектрису \( AM \). Чему равен угол \( BAM \)?
		\item В треугольнике \( KLM \) \( \angle K=75\degree \), \(\angle L=66\degree \). Провели биссектрисы \( KO \) и \( MP \). Найдите угол \( OMP \).
		\item Периметр прямоугольника равен \( 34 \) см. Найдите его площадь, если одна сторона на \( 30 \) мм больше другой.
		\item Сторона квадрата равна \( 6 \). Насколько его площадь больше площади прямоугольника со сторонами \( 3 \) и \( 9 \)?
		\item Площадь квадрата равна \( 64 \). Стороны квадрата увеличили так, что получится квадрат с площадью \( 81 \). На сколько увеличился периметр?
		\item Площадь гостиной составляет \( \mfrac{22}{2}{5} \) м\( ^2 \). Площадь спальни в два раза меньше. Коридор имеет форму прямоугольника длиной \( 2 \) м и шириной \( \dfrac{7}{10} \) м. Площадь ванной комнаты и туалета составляет \( 4 \) м\( ^2 \), а кухня меньше спальни на \( 2 \) м\( ^2 \). Найдите площадь всей квартиры.
		\item \exercise{4097}
	\end{listofex}
\end{class}
%END_FOLD

%BEGIN_FOLD % ====>>_____ Занятие 6 _____<<====
\begin{class}[number=6]
	\begin{listofex}
		\item Найдите объём  параллелепипеда в дм\(^3\) и см\( ^3 \), длина которого составляет \( 5 \) дм, ширина --- \( 3 \) дм, а высота --- \( 30 \) см.
		\item Найдите объём параллелепипеда, если его длина равна \( \mfrac{5}{1}{10} \) см, ширина на \( \mfrac{2}{1}{10} \) см больше, а высота равна \( 50 \) мм.
		\item Найдите высоту параллелепипеда, если его объём равен \( \mfrac{3}{4}{5} \) см\( ^3 \), длина --- \( \dfrac{4}{9} \), а ширина --- \( \dfrac{3}{20} \).
		\item Найдите высоту параллелепипеда, если его объём равен \( 80 \), а в основании лежит квадрат со стороной \( 4 \).
		\item Сеновал, имеющий длину \( 14 \) м, ширину \( 6 \) м и высоту \( \mfrac{3}{5}{10} \) м, полон сена. Сколько кг сена хранится на сеновале, если \( 1 \) м\( ^3 \) весит \( 60 \) кг?
		\item На заводе в цехе, имеющем форму прямоугольного параллелепипеда, длина которого равна \( 21 \) м, ширина --- \( 12 \) м, а высота --- \( 5 \) м, работает \( 28 \) рабочих. Сколько м\( ^3 \) приходится на одно рабочее место?
		\item Из бруска дерева, имеющего форму прямоугольного параллелепипеда, стороны которого равны \( 3 \) м, \( 7 \) м и \( 10 \) м выпилили другой брусок, стороны которого равны \( 1 \) м, \( 2 \) м и \( 6 \) м. Найдите объём оставшейся фигуры.
		\item Вычислите: \quad \( \left( \mfrac{1}{1}{2}+\mfrac{2}{2}{3}+\mfrac{3}{3}{4} \right)\cdot\mfrac{3}{3}{5}:\left( 14-\mfrac{15}{1}{8}:\mfrac{2}{1}{5} \right) \)
	\end{listofex}
\end{class}
%END_FOLD

%BEGIN_FOLD % ====>>_ Домашняя работа 3 _<<====
\begin{homework}[number=3]
	\begin{listofex}
		\item Комната имеет длину \( 8 \) м, ширину \( 6 \) м и высоту \( 4 \) м. Найдите площадь потолка и объём комнаты.
		\item В прямоугольном треугольнике острый угол равен \( 40\degree \). Найдите градусную меру всех внешних углов треугольника.
		\item Вычислите: \quad \( 3\cdot\mfrac{2}{7}{15}-\mfrac{5}{1}{2}\cdot\dfrac{7}{9}+\mfrac{1}{7}{48}\cdot\mfrac{2}{2}{11}-\dfrac{2}{5}\cdot\mfrac{6}{5}{9} \)
	\end{listofex}
\end{homework}
%END_FOLD

%BEGIN_FOLD % ====>>_____ Занятие 7 _____<<====
\begin{class}[number=7]
	\title{Подготовка к проверочной}
	\begin{listofex}
		\item В прямоугольном треугольнике один угол больше другого в \( 2 \) раза. Чему равны эти углы треугольника?
		\item Внутренние углы треугольника равны \( 50\degree \), \( 60\degree \) и \( 70\degree \). Чему равны внешние углы?
		\item Прямоугольный садовый участок,  площадь которого равна \(6\) соток огорожен забором. Длина участка  \(15\) м. Какова длина забора ?
		\item Площадь прямоугольника равна \(80\) дм\(^2\) , а его ширина равна \(5\) дм. На сколько надо уменьшить длину прямоугольника, чтобы его площадь уменьшилась на \(35\) дм\(^2\)?
		\item Длина аквариума \(80\) см, ширина \(50\) см, а высота \(45\) см. Сколько литров воды надо влить в этот аквариум, чтобы уровень воды был ниже верхнего края на \(8\) см?
		\item Ребро куба \(11\) см. Найдите его объём.
		\item Объём спортивного зала \(1800\) м\(^3\). Его высота \(5\) м. Какова площадь пола в зале?
		\item Длина прямоугольной грядки равна \(3\) м \(6\) дм , а ширина на \(1\) м \(8\) дм меньше длины. Найди периметр и площадь грядки.
		\item Объём двух прямоугольных параллелепипедов одинаковый. Длина первого параллелепипеда \(24\) см, ширина \(15\) см, высота \(18\) см. Найдите высоту второго параллелепипеда, если его длина \(45\) см, а ширина \(12\) см.
		\item Вычислить: \quad \( \mfrac{2}{1}{2}\cdot48-\mfrac{3}{2}{3}:\dfrac{1}{18}+\mfrac{5}{5}{12}:\dfrac{7}{36} \)
	\end{listofex}
\end{class}
%END_FOLD

%BEGIN_FOLD % ====>>_ Проверочная работа _<<====
\begin{exam}
	\begin{listofex}
		\item Объём ящика \( 13600 \) см\( ^3 \). Найдите площадь дна этого ящика, если его высота \( 16 \) см.
		\item Ребро куба \( 15 \) см. Найдите его объём.
		\item Длина прямоугольника \( 62 \) см. На сколько увеличится площадь этого прямоугольника, если его ширину увеличить на \( 7 \) см?
		\item Ширина прямоугольного параллелепипеда, равная \( 34 \) см, больше высоты на \( 15 \) см, а длина в \( 3 \) раза больше высоты. Найдите объем этого параллелепипеда.
		\item Аквариум имеет размеры \( 60 \)Х\( 20 \)Х\( 50 \) см. Сколько литров воды нужно влить в аквариум, чтобы уровень воды был ниже верхнего края аквариума на \( 10 \) см? (\( 1 \) л =\( 1000 \) см\( ^3 \))
		\item Один угол в прямоугольном треугольнике равен \( 47\degree \). Чему равен второй острый угол? На сколько сумма острых углов больше их разности?
		\item Внешние углы треугольника равны \( 40\degree \) и \( 170\degree \). Чему равны все внутренние углы треугольника?
		\item Вычислить: \quad \( \mfrac{10}{5}{9}-\mfrac{1}{7}{32}\cdot\left( \mfrac{4}{14}{15}+\mfrac{3}{1}{15} \right) \)
	\end{listofex}
\end{exam}
%END_FOLD