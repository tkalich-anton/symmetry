%7 класс Предподготовка Занятие №2
\begin{listofex}
	\item Вычислить:
	\begin{enumcols}[itemcolumns=2]
		\item \( (-0,05-(-0,5)-1,2)\cdot(-4,8) \)
		\item \( -2,321\cdot(-3,2+2,3-4,8+6,7)-1,579 \)
		\item \( \left( 2\dfrac{7}{15}-4 \right)\cdot\left( 8\dfrac{16}{23}-10 \right) \)
	\end{enumcols}
	\item Масса слона \( 5 \) тонн. Это на \( 4 \) т. \( 500 \) кг. больше, чем масса лошади. Во сколько раз слон тяжелее лошади?
	\item Теплоход за два дня прошел \( 350 \) км. В первый день он был в пути \( 8 \) ч., а во второй --- \( 6 \) ч. Какое расстояние он прошел в каждый из дней, если шел с одинаковой скоростью?
	\item От двух пристаней, расстояние между которыми \( 350 \) км, в \( 11 \) ч отправились навстречу друг другу два теплохода. Скорость первого \( 32 \) км/ч, скорость второго \( 38 \) км/ч. В какое время теплоходы встретятся?
	\item Длина конфеты \( 18 \) м. Петя и Маша одновременно начали есть ее с обоих концов. Петя съедает \( 7 \) см конфеты в секунду, а Маша – \( 8 \) см конфеты в секунду.
	\begin{enumerate}[label=\asbuk*), labelsep=0.5em]
		\item Сколько метров конфеты останется через минуту?
		\item Через сколько времени Петя и Маша съедят всю конфету?
		\item Сколько метров конфеты съест при этом каждый?
	\end{enumerate}
	\item Пиджак стоит \( 5000 \) рублей. В связи с поступлением новой коллекции пиджак продают со скидкой \( 60\% \). Сколько стоит пиджак с учётом скидки?
	\item Работа была выполнена за \( 3 \) дня. В первый день было сделано \( \frac{3}{20} \) всей работы, а во второй --- \( \frac{5}{12} \) всей работы. Какая часть работы была выполнена в третий день?
	\item В библиотеке \( 98000 \) книг. Книги на русском языке составляют \( 78\% \) всех книг, из них \( 5\% \) --- учебники. Сколько учебников на русском языке в библиотеке?
	\item В помощь садовому насосу, перекачивающему \( 5 \) л воды за \( 2 \) мин, подключили второй насос, перекачивающий тот же объем воды за \( 3 \) мин. Сколько времени эти два насоса должны работать совместно, чтобы перекачать \( 25 \) л воды?
\end{listofex}