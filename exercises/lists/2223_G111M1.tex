%11 класс Проифль Модуль 1 Занятие №4
\begin{listofex}
	\item Найти значение выражения:
	\begin{enumcols}[itemcolumns=3]
		\item \exercise{2829}
		\item \exercise{2828}
		\item \exercise{1215}
		\item \exercise{2826}
		\item \exercise{2825}
		\item \exercise{2824}
		\item \exercise{2837}
	\end{enumcols}
	\item Вычислить:
	\begin{enumcols}[itemcolumns=3]
		\item \exercise{1744}
		\item \exercise{1738}
		\item \exercise{2827}
	\end{enumcols}
	\item Найти значение выражения:
	\begin{enumcols}[itemcolumns=2]
		\item \exercise{2838}
		\item \exercise{2841}
		\item \exercise{1743}
	\end{enumcols}
	\item Найти значение выражения:
	\begin{enumcols}[itemcolumns=1]
		\item \exercise{1338}
		\item \exercise{1339}
	\end{enumcols}
	\item \exercise{1326}
	\item Найти значение выражения:
	\begin{enumcols}[itemcolumns=1]
		\item \exercise{1328}
		\item \exercise{1334}
	\end{enumcols}
	\item Найти значение выражения:
	
	\textit{Пример:} \( \sqrt{11-4\sqrt{7}}=\sqrt{\displaystyle2\mathstrut^2+\sqrt{7}\mathstrut^2-2\sqrt{2\cdot7}}=\sqrt{\displaystyle(2-\sqrt{7})^2}=\left|2-\sqrt{7}\right|=\sqrt{7} - 2 \)
	\begin{enumcols}[itemcolumns=2]
		\item \exercise{1332}
		\item \exercise{1331}
	\end{enumcols}
	\item Решить уравнения:
	\begin{enumcols}[itemcolumns=2]
		\item \( \sqrt{15-2x}=3 \)
		\item \( \sqrt{\dfrac{6}{4x-54}}=\dfrac{1}{7} \)
		\item \( \sqrt{-72-17x}=-x \)
		\item \( \sqrt[3]{x-4}=3 \)
	\end{enumcols}
\end{listofex}
\newpage
\title{Занятие №4}
\begin{listofex}
	\item Вычислить:
	\begin{enumcols}[itemcolumns=2]
		\item \exercise{1690}
		\item \exercise{1694}
		\item \exercise{1783}
		\item \exercise{1720}
		\item \exercise{1723}
		\item \exercise{1785}
	\end{enumcols}
	\item Вычислить:
	\begin{enumcols}[itemcolumns=2]
		\item \exercise{1733}
		\item \exercise{1737}
		\item \exercise{1773}
		\item \exercise{1756}
		\item \( (\sqrt{2}+1)^2+(\sqrt{2}-1)^2 \) \answer{\( 6 \)}
		\item \( (\sqrt{7}-2)^2+4\sqrt{7} \) \answer{\( 11 \)}
	\end{enumcols}
	\item Упростить выражение:
	\begin{enumcols}[itemcolumns=2]
		\item \( \sqrt{2}+3\sqrt{32}+\dfrac{1}{2}\sqrt{128}-6\sqrt{18} \) \answer{\( -\sqrt{2} \)}
		\item \exercise{1763}
		\item \exercise{1757}
		\item \exercise{1758}
		\item \exercise{1689}
	\end{enumcols}
	%\item Освободитесь от иррациональности в знаменателе:
	%\begin{enumcols}[itemcolumns=5]
	%	\item \( \dfrac{3\sqrt{2}+2\sqrt{2}}{\sqrt{200}} \)
	%	\item \( \dfrac{\sqrt{5}+5}{\sqrt{5}} \)
	%	\item \( \dfrac{1}{\sqrt{2}-1} \)
	%	\item \( \dfrac{2}{\sqrt{3}-1} \)
	%	\item \( \dfrac{\sqrt{5}-\sqrt{3}}{\sqrt{5}+\sqrt{3}} \)
	%\end{enumcols}
	\item Упростить выражение:
	\begin{enumcols}[itemcolumns=3]
		%\item \( \dfrac{5\sqrt{x}-2}{\sqrt{x}}-\dfrac{\sqrt{x}}{x} \)
		\item \( \dfrac{\sqrt{x}}{\sqrt{x}-1}-\dfrac{\sqrt{x}}{x-1} \) \answer{\( \dfrac{x}{x-1} \)}
		%\item \( \dfrac{\sqrt{x}}{\sqrt{x}-6}-\dfrac{3}{\sqrt{x}+6}+\dfrac{x}{36-x} \)
		%\item \( \left( \dfrac{\sqrt{x}}{\sqrt{x}+1}+1 \right):\left( \dfrac{\sqrt{x}+1}{\sqrt{x}-1} \right) \)
		\item \( \dfrac{x-1}{x-2\sqrt{x}+1}-\dfrac{\sqrt{x}+1}{\sqrt{x}-1} \) \answer{\( 0 \)}
		%\item \( \dfrac{x\sqrt{x}-1}{x-4\sqrt{x}+3}-\dfrac{\sqrt{x}+10}{\sqrt{x}-3} \)
	\end{enumcols}
	%\item Упростить выражение:
	%\[ \left( \dfrac{2x\sqrt{y}}{2\sqrt{x}-\sqrt{y}}-\dfrac{y\sqrt{x}}{2\sqrt{x}+\sqrt{y}} \right)\cdot\dfrac{2\sqrt{x}-\sqrt{y}}{4\sqrt{x^3y}+\sqrt{xy^3}} \]
	\item Найти значение выражения \( x-\sqrt{(10-x)^2} \),\quad если \( x>10 \) \answer{\( 10 \)}
	%\begin{enumcols}[itemcolumns=1]
	%	%\item \( 2x-\sqrt{(2x-3)^2} \),\quad если \( x<1,5\)
	%	%\item \( \dfrac{x-16}{\sqrt{x}-4}-\dfrac{x-36}{\sqrt{x}+6} \),\quad если \( x>16 \)
	%	%\item \( \sqrt{x-3}-|\sqrt{x-3}+1| \),\quad если \( x=\pi\)
	%	%\item \( |\sqrt{x+5}-3|+\sqrt{x+5} \),\quad если \( -5\le x< -\pi\)
	%	%\item \( \sqrt{(x+4)^2}-\sqrt{x^2-6x+9} \),\quad если \( -4\le x \le 3\)
	%	%\item \( 4x+\sqrt{9-x^2}+|\sqrt{9-x^2}-3| \),\quad если \( x=2,5\)
	%\end{enumcols}
	\item Вычислить:
	\begin{enumcols}[itemcolumns=2]
		%\item \( \sqrt{11-4\sqrt{7}}-\sqrt{7} \)
		%\item \( \sqrt{17-6\sqrt{8}}+\sqrt{8} \)
		%\item \( \dfrac{\sqrt{2+\sqrt{3}}-\sqrt{2-\sqrt{3}}}{\sqrt{2}} \)
		\item \exercise{1646}
		\item \exercise{1635}
	\end{enumcols}
\end{listofex}
\newpage
\title{Домашняя работа №2}
\begin{listofex}
	\item Вычислить:
	\begin{enumcols}[itemcolumns=2]
		\item \exercise{1757}
		\item \exercise{1661}
		\item \exercise{1775}
		\item \exercise{1638}
	\end{enumcols}
	\item Вычислить:
	\begin{enumcols}[itemcolumns=2]
		\item \exercise{1763}
		\item \exercise{1741}
	\end{enumcols}
	\item Найти значение выражения:
	\begin{enumcols}[itemcolumns=2]
		\item \exercise{1663}
		\item \exercise{1650}
	\end{enumcols}
	\item \exercise{17}
	\item Найти значение выражения:
	\begin{enumcols}[itemcolumns=2]
		\item \exercise{1668}
		\item \exercise{1634}
	\end{enumcols}
	\item \exercise{1522}
	\item Найти значение выражения \( 2x-\sqrt{(2x-3)^2} \),\quad если \( x<1,5\)
	\item Решить уравнения:
	\begin{enumcols}[itemcolumns=2]
		\item \exercise{3696}
		\item \( \sqrt{12-3x}=4 \)
		\item \( \sqrt{\dfrac{4}{2x-21}}=\dfrac{1}{5} \)
		\item \exercise{3403}
	\end{enumcols}
\end{listofex}