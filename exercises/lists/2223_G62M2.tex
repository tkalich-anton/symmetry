%Группа 6-1 Модуль 1
\title{Занятие №1}
\begin{listofex}
	\item Вычислить:
	\begin{enumcols}[itemcolumns=2]
		\item \( 25\cdot(28\cdot105+7236:18)-(4247-1823):6\cdot25 \)
		\item \( 3124:(3\cdot504-4\cdot307)+10403:101 \)
	\end{enumcols}
	\item Что такое простые числа? Назовите первые 10 простых чисел.
	\item Сформулируйте признаки делимости на \( 2;\;3;\;5;\;9;\;10 \).
	\item Замените звездочки двумя цифрами так, чтобы:
	\begin{enumcols}[itemcolumns=2]
		\item число \( 1\:* \) делилось на \( 2 \)
		\item число \( 12\:* \) делилось на \( 5 \)
		\item число \( 8*3\:* \) делилось на \( 3 \)
		\item число \( *\:18\:* \) делилось на \( 9 \)
	\end{enumcols}
	\item Найдите неизвестные цифры числа, если известно, что число делится на \( 6 \):
	\begin{enumcols}[itemcolumns=2]
		\item \( 354*7\:* \)
		\item \( *\:4567\:* \)
	\end{enumcols}
	\item Найдите неизвестные цифры числа, если известно, что число делится на \( 15 \):
	\begin{enumcols}[itemcolumns=2]
		\item \( 11\:*\:* \)
		\item \( *\:22\:* \)
	\end{enumcols}
	\item Найдите неизвестные цифры числа, если известно, что число делится на \( 9 \) и на \( 10 \):
	\begin{enumcols}[itemcolumns=2]
		\item \( 3*6\:* \)
		\item \( 11111** \)
	\end{enumcols}
\end{listofex}
\title{Занятие №2}
\begin{listofex}
	\item Найти значение выражения \( (a-b)\cdot(b+c) \), если \( a=247;\; b=189;\; c=127 \).
	\item Найти значение выражения \( a^2+2\cdot a\cdot b + b^2 \), если \( a=217 \) и \( b=83 \).
	\item Туристы были в походе три дня. Во второй день они прошли 18 км, что на 5 км меньше,
	чем в первый день, а в третий день они прошли на 19 км меньше, чем за два предыдущих дня.
	Сколько километров прошли туристы за три дня?
	\item При ремонте шоссе длиной в 69 км в первый день отремонтировали 7 км, а в каждый из
	трех последующих дней ремонтировали на 3 км больше, чем в предыдущий. Во сколько раз
	оставшийся участок шоссе меньше отремонтированного?
	\item Периметр треугольника равен 63 см. Одна сторона равна 18 см, что на 7 см меньше второй стороны. Найдите третью сторону треугольника.
	\item На лугу пасся табун лошадей. У них ног на \( 27 \) больше, чем голов. Сколько лошадей паслось на лугу.
	\item В двух комнатах было \( 45 \) человек. Из первой вышли \( 9 \), а из второй --- \( 14 \), и людей в комнатах стало поровну. Сколько человек было в комнатах сначала?
\end{listofex}
\newpage
\title{Домашняя работа №1}
\begin{listofex}
	\item Вычислить: \( (410+96)\cdot(1010-31248:62)-170\cdot1500 \)
	\item Фермер убрал урожай картофеля за три дня. В первый день он убрал 19 грядок, что на 6 грядок больше, чем в третий день, а во второй день он убрал на 12 грядок меньше, чем за первый и третий дни вместе. Сколько грядок картофеля убрал фермер за три дня?
	\item Периметр треугольника равен 61 см. Одна сторона равна 16 см, а вторая в два раза больше третьей. Найдите вторую и третью стороны треугольника.
	\item На лугу паслось стадо коров. У них ног на \( 54 \) больше, чем голов. Сколько коров паслось на лугу.
	\item В двух комнатах было \( 56 \) человек. Когда в первую зашли еще \( 12 \), а во вторую --- \( 8 \), то в комнатах людей стало поровну. Сколько человек было в комнатах сначала?
\end{listofex}
\newpage

\title{Занятие №3}
\begin{listofex}
	\item Вычислите, используя распределительный закон:
	\begin{enumcols}[itemcolumns=4]
		\item \( 7\cdot13-7\cdot2 \)
		\item \( 37\cdot12+37\cdot88 \)
		\item \( 27\cdot12+27\cdot8 \)
		\item \( 101\cdot17-17 \)
		\item \( 33\cdot11+11 \)
		\item \( 16+29\cdot16 \)
		\item \( 99\cdot15+15 \)
		\item \( 1001\cdot54-54 \)
	\end{enumcols}
	\item Перепишите заполняя пропуски:
	\begin{enumcols}[itemcolumns=2]
		\item \( {\dots}\cdot(16+14)=7\cdot16+7\cdot14 \)
		\item \( 45\cdot({\dots}-{\dots})=45\cdot15-45\cdot13 \)
		\item \( 14\cdot(15+3)=14\cdot{\dots}+{\dots}\cdot3 \)
		\item \( 7\cdot({\dots}+14)=14\cdot{\dots}+{\dots}\cdot5 \)
	\end{enumcols}
	\item Вынести общий множитель за скобки и вычислить:
	\begin{enumcols}[itemcolumns=2]
		\item \( 61\cdot21+39\cdot21 \)
		\item \( 123\cdot11-22\cdot11 \)
		\item \( 37\cdot59+37\cdot41+63\cdot59+41\cdot63 \)
		\item \( 999\cdot55+55+257\cdot43+43\cdot43 \)
	\end{enumcols}
	\item \textbf{Свойства делимости:}
	\begin{enumcols}[itemcolumns=1]
		\item Если один из множителей делится на некоторое число, то и произведение делится на это число.
		\item Если первое число делится на второе, а второе делится на третье, то первое число делится на третье.
		\item Если каждое из двух чисел делится на некоторое число, то их сумма или разность делятся на это число.
		\item Если одно из двух чисел делится на некоторое число, а другое на него не делится, то их сумма или разность не делится на это число.
	\end{enumcols}
	\item Запишите числа \( 24,\;42,\;36,\;72,\;75 \) в виде произведения и покажите, что
	\begin{enumcols}[itemcolumns=3]
		\item \( 24 \) делится на \( 2 \)
		\item \( 36 \) делится на \( 6 \)
		\item \( 75 \) делится на \( 5 \)
		\item \( 42 \) делится на \( 21 \)
		\item \( 72 \) делится на \( 9 \)
		\item \( 75 \) делится на \( 25 \)
	\end{enumcols}
	\item Объясните, не производя вычислений, почему следующие произведения делятся на \( 12 \)? Каким свойством вы в это случае пользуетесь?
	\begin{enumcols}[itemcolumns=4]
		\item \( 12\cdot47 \)
		\item \( 24\cdot17 \)
		\item \( 120\cdot51 \)
		\item \( 27\cdot8 \)
	\end{enumcols}
	\item Объясните, почему:
	\begin{enumcols}[itemcolumns=2]
		\item сумма \( 45+36 \) делится на \( 9 \)
		\item сумма \( 99+88 \) делится на \( 11 \)
		\item сумма \( 13\cdot2+13\cdot7\) делится на \( 13 \)
	\end{enumcols}
\end{listofex}
\newpage
\title{Занятие №4}
\begin{listofex}
	\item Какую цифру нужно поставить вместо звездочки, чтобы полученное число:
	\begin{enumcols}[itemcolumns=3]
		\item \( 2\:* \) делилось на \( 2 \);
		\item \( 43\:* \) делилось на \( 3 \);
		\item \( 4\:* \) делилось на \( 9 \);
		\item \( 23\:* \) делилось на \( 10 \);
		\item \( 123\:* \) делилось на \( 5 \);
		\item \( 24*0 \) делилось на \( 9 \);
		\item \( 2*22 \) делилось на \( 9 \);
		\item \( 1*4\:* \) делилось на \( 2 \) и \( 3 \);
		\item \( 4*5\:* \) делилось на \( 9 \) и \( 5 \).
	\end{enumcols}
	\item Найдите неизвестные цифры числа, если известно, что число делится на \( 6 \):
	\begin{enumcols}[itemcolumns=2]
		\item \( 354*7\:* \)
		\item \( *\:4567\:* \)
	\end{enumcols}
	\item Разложить на простые множители:
	\begin{enumcols}[itemcolumns=5]
		\item \( 15 \)
		\item \( 24 \)
		\item \( 36 \)
		\item \( 50 \)
		\item \( 86 \)
		\item \( 98 \)
		\item \( 121 \)
		\item \( 164 \)
		\item \( 240 \)
		\item \( 1200 \)
	\end{enumcols}
	\item Найти все делители числа:
	\begin{enumcols}[itemcolumns=4]
		\item \( 40 \)
		\item \( 24 \)
		\item \( 200 \)
		\item \( 96 \)
	\end{enumcols}
	\item Найти наибольший общий делитель чисел:
	\begin{enumcols}[itemcolumns=4]
		\item \( 40 \) и \( 28 \)
		\item \( 24 \) и \( 36 \)
		\item \( 100 \) и \( 60 \)
		\item \( 75 \) и \( 25 \)
		\item \( 7 \) и \( 13 \)
		\item \( 1 \) и \( 15 \)
		\item \( 126 \) и \( 105 \)
		\item \( 70 \) и \( 245 \)
	\end{enumcols}
	\item Велосипедист и мотоциклист выехали одновременно из одного пункта в одном направлении. Скорость мотоциклиста \( 40 \) км/ч, а велосипедиста \( 12 \) км/ч. Какова скорость их удаления друг от друга? Через сколько часов расстояние между ними будет равно \( 56 \) км?
	\item Расстояние между городами \( A \) и \( B \) равно \( 720 \) км. Из \( A \) в \( B \) вышел скорый поезд со скоростью \( 80 \) км/ч. Через \( 2 \) ч навстречу ему из \( B \) в \( A \) вышел товарный поезд со скоростью \( 60 \) км/ч. Через сколько часов после выхода второго поезда они встретятся?
\end{listofex}
\newpage
\title{Занятие №5}
\begin{listofex}
	\item Разложить на простые множители:
	\begin{enumcols}[itemcolumns=5]
		\item \( 20 \)
		\item \( 60 \)
		\item \( 135 \)
		\item \( 1000 \)
		\item \( 212 \)
	\end{enumcols}
	\item Найти все делители числа:
	\begin{enumcols}[itemcolumns=4]
		\item \( 55 \)
		\item \( 57 \)
		\item \( 70 \)
		\item \( 100 \)
	\end{enumcols}
	\item Найдите двузначное число, кратное 45 и делящееся на 6.
	\item Замените звездочки двумя одинаковыми цифрами так, чтобы:
	\begin{enumcols}[itemcolumns=2]
		\item число \( 5**\:5 \) делилось на \( 3 \)
		\item число \( *\:4*5 \) делилось на \( 9 \)
		%\item число \( 7*2\:* \) делилось на \( 90 \)
	\end{enumcols}
	\item Вася принес в класс 93 конфеты и раздал поровну своим одноклассникам. Сколько в классе может быть человек?
	\item Из 12 офицеров и 20 солдат нужно сформировать одинаковые по составу группы для патрулирования. Сколько таких групп можно сделать?
	\item Найти наибольший общий делитель (НОД) чисел:
	\begin{enumcols}[itemcolumns=4]
		\item \( 40 \) и \( 28 \)
		\item \( 24 \) и \( 36 \)
		\item \( 100 \) и \( 60 \)
		\item \( 75 \) и \( 25 \)
		\item \( 7 \) и \( 13 \)
		\item \( 1 \) и \( 15 \)
		\item \( 126 \) и \( 105 \)
		\item \( 70 \) и \( 245 \)
	\end{enumcols}
	%\item Не производя вычислений, докажите, что:
	%\begin{enumcols}[itemcolumns=2]
	%	\item \( 224+32 \) делится на \( 2 \)
	%	\item \( 535-220 \) делится на \( 5 \)
	%	\item \( 13013-1326+130 \) делится на \( 13 \)
	%	\item \( 11\cdot56 \) делится на \( 11 \)
	%	\item \( 49\cdot48 \) делится на \( 7 \)
	%\end{enumcols}
	%\item Не производя вычислений, докажите, что \( 4556\cdot47+57\cdot507-47\cdot114 \) делится на \( 57 \).
	%\item Не производя вычислений, докажите, что:
	%\begin{enumcols}[itemcolumns=2]
	%	\item \( 35\cdot20 \) делится на \( 14 \)
	%	\item \( 5\cdot2^4 \) кратно \( 20 \)
	%\end{enumcols}
	
\end{listofex}
\newpage
\title{Занятие №6}
\begin{listofex}
	\item \textbf{Свойства делимости:}
	\begin{enumcols}[itemcolumns=1]
		\item Если один из множителей делится на некоторое число, то и произведение делится на это число.
		\item Если первое число делится на второе, а второе делится на третье, то первое число делится на третье.
		\item Если каждое из двух чисел делится на некоторое число, то их сумма или разность делятся на это число.
		\item Если одно из двух чисел делится на некоторое число, а другое на него не делится, то их сумма или разность не делится на это число.
	\end{enumcols}
	\item Запишите числа \( 24,\;42,\;36,\;72,\;75 \) в виде произведения и покажите, что
	\begin{enumcols}[itemcolumns=3]
		\item \( 24 \) делится на \( 2 \)
		\item \( 36 \) делится на \( 6 \)
		\item \( 75 \) делится на \( 5 \)
		\item \( 42 \) делится на \( 21 \)
		\item \( 72 \) делится на \( 9 \)
		\item \( 75 \) делится на \( 25 \)
	\end{enumcols}
	\item Объясните, не производя вычислений, почему следующие произведения делятся на \( 12 \)? Каким свойством вы в это случае пользуетесь?
	\begin{enumcols}[itemcolumns=4]
		\item \( 12\cdot47 \)
		\item \( 24\cdot17 \)
		\item \( 120\cdot51 \)
		\item \( 27\cdot8 \)
	\end{enumcols}
	\item Объясните, почему:
	\begin{enumcols}[itemcolumns=2]
		\item сумма \( 45+36 \) делится на \( 9 \)
		\item сумма \( 99+88 \) делится на \( 11 \)
		\item сумма \( 13\cdot2+13\cdot7\) делится на \( 13 \)
	\end{enumcols}
	\item Поезд, двигаясь со скоростью \( 90 \) км/ч, проезжает мимо неподвижного наблюдателя за \( 7 \) секунд. Какова длина поезда?
\end{listofex}
\newpage
\title{Занятие №7}
\begin{listofex}
	\item Объясните, не производя вычислений, почему следующее произведение:
	\begin{enumcols}[itemcolumns=2]
		\item \( 15\cdot20 \) делится на \( 3 \);
		\item \( 35\cdot55 \) делится на \( 7 \);
		\item \( 140\cdot21 \) делится на \( 10 \);
		\item \( 99\cdot17 \) делится на \( 11 \).
	\end{enumcols}
	\item Объясните, почему:
	\begin{enumcols}[itemcolumns=2]
		\item сумма \( 45+36 \) делится на \( 9 \)
		\item сумма \( 99+88 \) делится на \( 11 \)
		\item разность \( 80-30 \) делится на \( 10 \)
		\item сумма \( 300+1200 \) делится на \( 100 \)
		\item разность \( 64-32 \) делится на \( 8 \)
		\item сумма \( 13\cdot2+13\cdot7\) делится на \( 13 \)
		\item сумма \( 12\cdot5+24\cdot9\) делится на \( 12 \)
		\item разность \( 125\cdot33-50\cdot13\) делится на \( 25 \)
	\end{enumcols}
	\item Умножьте "быстро"{ }на \( 11 \):
	\begin{enumcols}[itemcolumns=5]
		\item \( 31\cdot11 \)
		\item \( 22\cdot11 \)
		\item \( 17\cdot11 \)
		\item \( 36\cdot11 \)
		\item \( 62\cdot11 \)
		\item \( 81\cdot11 \)
		\item \( 72\cdot11 \)
		\item \( 53\cdot11 \)
		\item \( 21\cdot11 \)
		\item \( 11\cdot11 \)
		\item \( 34\cdot11 \)
		\item \( 63\cdot11 \)
		\item \( 44\cdot11 \)
		\item \( 15\cdot11 \)
		\item \( 33\cdot11 \)
	\end{enumcols}
	\item Умножьте "быстро"{ }на \( 11 \):
	\begin{enumcols}[itemcolumns=4]
		\item \( 47\cdot11 \)
		\item \( 85\cdot11 \)
		\item \( 92\cdot11 \)
		\item \( 49\cdot11 \)
		\item \( 88\cdot11 \)
		\item \( 38\cdot11 \)
		\item \( 76\cdot11 \)
		\item \( 57\cdot11 \)
		\item \( 77\cdot11 \)
		\item \( 87\cdot11 \)
		\item \( 59\cdot11 \)
		\item \( 66\cdot11 \)
	\end{enumcols}
	\item Переведите:
	\begin{enumcols}[itemcolumns=1]
		\item 3 дм 2 см в сантиметры;
		\item 3 м в 5 см в миллиметры;
		\item 2 ч 20 мин в минуты;
		\item 12 т 125 кг в килограммы;
		\item пол дня и 1ч в секунды;
		\item 2 не високосных года и 5 месяцев в дни.
	\end{enumcols}
	\item Поезд, двигаясь со скоростью \( 90 \) км/ч, проезжает мимо неподвижного наблюдателя за \( 7 \) секунд. Какова длина поезда?
\end{listofex}
\newpage
\title{Занятие №8}
\begin{listofex}
	\item Вычислить \( 4a^3-2a+3ab^3-(2\cdot a+b):5 \), если \( a=3 \) и \( b=4 \)
	\item Найдите:
	\begin{enumcols}[itemcolumns=3]
		\item \( \dfrac{2}{3} \) от \( 15 \)
		\item \( \dfrac{54}{101} \) от \( 505 \)
		\item \( \dfrac{17}{21} \) от \( 63 \)
		\item \( \dfrac{5}{7} \) от \( 35 \)
		\item \( \dfrac{1}{2} \) от \( 24 \)
		\item \( \dfrac{12}{13} \) от \( 26 \)
		\item \( \dfrac{73}{100} \) от \( 400 \)
		\item \( \dfrac{31}{50} \) от \( 750 \)
		\item \( \dfrac{444}{111} \) от \( 555 \)
	\end{enumcols}
	\item Найдите \( \dfrac{2\cdot a}{b} \) часть от числа \( 55 \), если \( a=6 \) и \( b=11 \).
	\item Найдите \( \dfrac{a^3}{3b^2} \) часть от числа \( 108 \), если \( a=2 \) и \( b=3 \).
	\item На свой день рождения Алиса купила \( 560 \) кг фруктов (на весь класс). Из них \( 4/7 \) составляют яблоки, а остальное --- апельсины. Сколько килограммов апельсинов купила Алиса. Какую часть от всех фруктов составляют апельсины?
	
\end{listofex}
%\newpage
%\title{Домашняя работа №2}
%\begin{listofex}
%	\item Разложить на простые множители:
%	\begin{enumcols}[itemcolumns=5]
%		\item \( 30 \)
%		\item \( 68 \)
%		\item \( 190 \)
%		\item \( 121 \)
%		\item \( 520 \)
%	\end{enumcols}
%	\item Найти все делители числа:
%	\begin{enumcols}[itemcolumns=4]
%		\item \( 65 \)
%		\item \( 100 \)
%		\item \( 75 \)
%		\item \( 105 \)
%	\end{enumcols}
%	\item Замените звездочки двумя одинаковыми цифрами так, чтобы:
%	\begin{enumcols}[itemcolumns=1]
%		\item число \( 2**\:2 \) делилось на \( 3 \)
%		\item число \( *\:6*3 \) делилось на \( 9 \)
%		\item число \( 4*2\:* \) делилось на \( 30 \)
%	\end{enumcols}
%	\item Найдите двузначное число, кратное 36 и не делящееся на 8.
%	\item Для контрольной работы было приготовлено 87 листов бумаги, которые поровну раздали ученикам класса. Сколько учеников в классе?
%	\item Из 20 конфет и 16 шоколадок нужно сделать одинаковые наборы. Сколько таких наборов можно сделать?
%	\item Не производя вычислений, докажите, что:
%	\begin{enumcols}[itemcolumns=2]
%		\item \( 648+24 \) делится на \( 2 \)
%		\item \( 1245-339 \) делится на \( 3 \)
%		\item \( 11088+1122-77 \) делится на \( 11 \)
%		\item \( 7\cdot87 \) делится на \( 7 \)
%		\item \( 45\cdot13 \) делится на \( 5 \)
%	\end{enumcols}
%	\item Не производя вычислений, докажите, что \( 39\cdot737+39\cdot281-39\cdot296 \) делится на \( 13 \).
%	\item Не производя вычислений, докажите, что:
%	\begin{enumcols}[itemcolumns=2]
%		\item \( 63\cdot24 \) делится на \( 21 \)
%		\item \( 34\cdot33 \) кратно \( 51 \)
%		\item \( 2^2\cdot3\cdot5^3 \) кратно \( 50 \)
%	\end{enumcols}
%	\item Поезд, двигаясь со скоростью 108 км/ч, проезжает мимо неподвижного наблюдателя за 13 секунд. Какова длина поезда?
%\end{listofex}