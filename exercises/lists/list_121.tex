%11 класс Предподготовка Занятие №2
\begin{center}
	\textbf{Часть 1}
\end{center}
\begin{listofex}
	\item Найдите корень уравнения \( \sqrt{66-5x}=9 \)
	\item На семинар приехали \( 3 \) ученых из Болгарии, \( 4 \) из Австрии и
	\( 5 \) из Финляндии. Порядок докладов определяется жеребьёвкой.
	Найдите вероятность того, что восьмым окажется доклад ученого
	из Болгарии.
	\item Центральный угол на \( 61\degree \) больше
	острого вписанного угла, опирающегося на ту
	же дугу окружности. Найдите вписанный угол.
	Ответ дайте в градусах.
	\item Найдите значение выражения\quad\( \dfrac{5\sin74\degree}{\cos37\degree\cdot\cos53\degree} \)
	\item Объем первого цилиндра равен \( 20 \) м\( ^3 \). У второго цилиндра
	высота в \( 4 \) раза меньше, а радиус основания в \( 3 \) раза больше,
	чем у первого. Найдите объем второго цилиндра. Ответ дайте в
	кубических метрах.
	\item \exercise{1856}
	\item В ходе распада радиоактивного изотопа, его масса уменьшается по закону \( m(t)=m_0\cdot2^{-t/T} \), где \( m_o \) --- начальная масса изотопа, \( t \) (мин) -- прошедшее от начального момента время, \( T \) --- период полураспада в минутах. В лаборатории получили вещество, содержащее в начальный момент времени \( m_0=40 \) мг изотопа \( Z \), период полураспада которого равен \( T=10 \) мин. В течение скольких минут масса изотопа будет не меньше \( 5 \) мг?
	\item Моторная лодка прошла против течения реки \( 255 \) км и
	вернулась в пункт отправления, затратив на обратный путь на \( 2 \) часа меньше. Найдите скорость лодки в неподвижной воде, если скорость течения равна \( 1 \) км/ч. Ответ дайте в км/ч.
	\item \exercise{1857}
	\item Помещение освещается фонарём с тремя лампами.
	Вероятность перегорания одной лампы в течение года равна \( 0,2 \).
	Найдите вероятность того, что в течение года хотя бы одна лампа
	не перегорит.
	\item Найдите точку минимума функции \( y=x^3-300x+19 \).
\end{listofex}
	\begin{center}
		\textbf{Часть 2}
	\end{center}
\begin{listofex}[start=12]
	\item а) Решите уравнение \( \cos2x-3\sin(-x)-2=0 \)\\
	б) Найдите все корни этого уравнения, принадлежащие
	промежутку \( \left[3\pi;\dfrac{9\pi}{2}\right] \)
	\setcounter{listofexi}{13}
	\item Решить неравенство \[ \dfrac{2}{5^x+75}\ge\dfrac{1}{5^x-25} \]
	\item В июле \( 2026 \) года планируется взять кредит на три года
	в размере \( 400 \) тыс. рублей. Условия его возврата таковы:
	
	--- каждый январь долг будет возрастать на 30\% по сравнению с концом предыдущего года;
	
	--- с февраля по июнь каждого года необходимо выплатить часть долга;
	
	--- платежи в \( 2027 \) и \( 2028 \) годах должны быть равны;
	
	--- к июлю \( 2029 \) года долг должен быть выплачен полностью.
	
	Известно, что платеж в \( 2029 \) году составит \( 280,8 \) тыс. рублей. Сколько рублей составит платеж в \( 2027 \) году?
	\setcounter{listofexi}{17}
	\item С натуральным трехзначным числом проводят следующую операцию: из числа вычитают его сумму цифр и полученный результат делят на \( 3 \).
	
	а) Может ли результатом выполнения операции быть число \( 300 \)?
	
	б) Может ли результатом выполнения операции быть число \( 151 \)?
	
	в) Сколько различных результатов можно получить, если применить данную операцию для всех трехзначных чисел от \( 100 \) до \( 600 \)?
	
\end{listofex}