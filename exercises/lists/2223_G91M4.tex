%
%===============>>  ГРУППА 9-1 МОДУЛЬ 4  <<=============
%
\setmodule{4}
%
%===============>>  Занятие 1  <<===============
%
\begin{class}[number=1]
	\begin{listofex}
		\item Вычислить рациональным способом:
		\begin{enumcols}[itemcolumns=3]
			\item \( \sqrt{16+4\cdot4\cdot24} \)
			\item \( \sqrt{83^3\cdot2^2-83^2\cdot2^3} \)
			\item \( \sqrt{50^2-4\cdot7\cdot7} \)
		\end{enumcols}
		\item Построить график функции \( y=2x-5 \).
		\begin{enumcols}[itemcolumns=1]
			\item Проверить \textit{(графическим, а потом аналитическим способом)}, принадлежит ли точка с координатами \( (4;3) \) графику этой функции?
			\item Принадлежит ли точка с координатами \( (112;217) \) графику этой функции?
			\item Найти точку пересечения графика данной функции с графиком функции \( y=4x-1 \).
		\end{enumcols}
		\item Построить график функции \( f(x)=x^2-2x+1 \). Выберите верные утверждения:
		\begin{enumcols}[itemcolumns=1]
			\item График функции возрастает на промежутке \( [0;+\infty) \);
			\item График функции убывает на промежутке \( [-5;-2] \);
			\item \( f(-1)>f(1) \);
			\item \( f(x)=4 \) при \( x=3 \);
			\item Наименьшее значение функция принимает при \( x=-1 \);
			\item При \( x=2 \) значение функции больше чем при \( x=0 \);
			\item \( f(-1)=f(3) \);
			\item Корнем уравнения \( f(x)=9 \) является только \( x=-2 \);
			\item \( f(x)\ge0 \) при любом \( x \);
			\item Данный график функции пересекается с графиком \( y=-2x+5 \) в одной точке.
		\end{enumcols}
		\item Решить систему неравенств:
		\[ \left\{
		\begin{array}{l}
			3(x-1)-2(2-3x)>5x-3,\\
			8x-3(2x+5)<2(x-7).
		\end{array}
		\right. \]
		\item При каких значениях переменной выражение \( \sqrt{4x+1}+\sqrt{2-3x} \) имеет смысл?
		\item Построить график функции \( y=x-|2x+1|-2 \). Найти точки пересечения данного графика с графиком функции \( y=x-5 \).
	\end{listofex}
\end{class}
%
%===============>>  Занятие 2  <<===============
%
\begin{class}[number=2]
	\begin{listofex}
		\item Из закона всемирного тяготения \( F=G\cdot\dfrac{mM}{r^2} \) выразите массу \( m \) и найдите ее величину (в килограммах), если \( F=13,4 \) H, \( r=5 \) м, \( M=5\cdot10^9 \) и гравитационная постоянная \( G=6,7\cdot10^{-11} \) \( \dfrac{\text{м}^3}{\text{кг}\cdot\text{с}^2} \).
		\item Известно, что \( c<-1 \). расположите в порядке убывания числа \( c,\; c^2,\; \dfrac{1}{c}\).
		\begin{enumcols}[itemcolumns=4]
			\item \( c^2,\; c,\; \dfrac{1}{c} \)
			\item \( c^2,\; \dfrac{1}{c},\; c \)
			\item \( c,\; c^2,\; \dfrac{1}{c} \)
			\item \( c,\; \dfrac{1}{c},\; c^2 \)
		\end{enumcols}
		\item Какое из данных чисел принадлежит отрезку \( [3;4] \)?
		\begin{enumcols}[itemcolumns=4]
			\item \( \dfrac{45}{19} \)
			\item \( \dfrac{52}{19} \)
			\item \( \dfrac{68}{19} \)
			\item \( \dfrac{77}{19} \)
		\end{enumcols}
		\item Постройте графики функции \( y=4x-1 \) и \( y=-\dfrac{1}{2}x+2,5 \). Графически определите точку пересечения этих графиков. Во сколько раз ордината точки пересечения больше абсциссы?
		\item Найдите точку пересечения графиков функций \( y=12x-70 \) и \( y=6x+2 \).
		\item Постройте график функции \( f(x)=x^2-6x+8 \). Выберите верные утверждения:
		\begin{enumcols}[itemcolumns=1]
			\item График функции возрастает на промежутке \( [4;+\infty) \);
			\item График функции убывает на промежутке \( [-7;7] \);
			\item \( f(-4)<f(4) \);
			\item \( f(x)=5 \) при \( x=0 \);
			\item Данная функция имеет максимальное значение при \( x=9 \);
			\item При \( x=-2 \) значение функции больше чем при \( x=0 \);
			\item \( f(-1)=f(3) \);
			\item \( f(x)\ge0 \) при \( x\ge4 \);
			\item \( f(x)\ge0 \) только при \( x\ge4 \);
			\item Данный график функции пересекается с графиком \( y=-x-8 \) в двух точках.
		\end{enumcols}
		\item Подставьте вместо знака \( * \) число так, чтобы функция \( y=2x^2-3x+* \) проходила через точку с координатами \( (4;15) \).
		\item Решить систему неравенств:
		\[ \left\{
		\begin{array}{l}
			6(x+2)-4(0,5-2x)>2x-6,\\
			9x+x(2x+5)<2(x^2-7).
		\end{array}
		\right. \]
		\item При каких значениях переменной выражение \( \sqrt{12x-6}+\sqrt{4-5x} \) имеет смысл?
		\item Высота равностороннего треугольника равна \( 15\sqrt{3} \). Найдите его периметр.
	\end{listofex}
\end{class}
%
%===============>>  Домашняя работа 1  <<===============
%
%\begin{homework}[number=1]
%	\begin{listofex}
%		\item Пусто
%	\end{listofex}
%\end{homework}
%
%===============>>  Занятие 3  <<===============
%
%\begin{class}[number=3]
%	\begin{listofex}
%		\item Пусто
%	\end{listofex}
%\end{class}
%
%===============>>  Занятие 4  <<===============
% смещение на одно занятие с прошлого месяца
%\begin{class}[number=4]
%	\begin{listofex}
%		\item Пусто
%	\end{listofex}
%\end{class}
%
%===============>>  Домашняя работа 2  <<===============
%
%\begin{homework}[number=2]
%	\begin{listofex}
%
%	\end{listofex}
%\end{homework}
%
%===============>>  Занятие 5  <<===============
% смещение на одно занятие с прошлого месяца
%\begin{class}[number=5]
%	\begin{listofex}
%		\item Пусто
%	\end{listofex}
%\end{class}
%
%===============>>  Домашняя работа 3  <<===============
%
%\begin{homework}[number=2]
%	\begin{listofex}
%
%	\end{listofex}
%\end{homework}
%\newpage
%\title{Подготовка к проверочной работе}
%\begin{listofex}
%	
%\end{listofex}
%
%===============>>  Занятие 7  <<===============
%
%\begin{class}[number=7]
%	\begin{listofex}
%	
%	\end{listofex}
%\end{class}
%
%===============>>  Провечная работа  <<===============
%
%\begin{exam}
%	\begin{listofex}
%	
%	\end{listofex}
%\end{exam}