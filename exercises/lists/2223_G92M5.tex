%
%===============>>  ГРУППА 9-2 МОДУЛЬ 5  <<=============
%
\setmodule{5}
%
%===============>>  Занятие 1  <<===============
%
\begin{class}[number=1]
	\begin{definit}
		Геометрическая прогрессия
		\[ b_{n+1}=q\cdot b_n  \quad
		b_n=b_1\cdot q^{n+1}  \quad
		b_n=\sqrt{b_{n-1}\cdot b_{n+1}}  \quad
		S_n=\dfrac{b_1(1-q^n)}{1-q} \]
	\end{definit}
	
	\begin{listofex}
		\item Выписаны первые несколько членов геометрической прогрессии: \( 17 \), \( 68 \), \( 272 \)... Найдите её четвёртый член.
		\item Найдите \( b_6 \), \quad если \( b_n=64,5\cdot(-2)^n \).
		\item Найдите \( b_7 \), \quad если \( b_1=-\mfrac{1}{1}{3}	\), \( b_{n+1}=-3b_n \).
		\item Найдите знаменатель геометрической прогрессии, \quad если \( b_5=-14 \), \( b_8=112 \).
		\item Выписаны первые несколько членов геометрической прогрессии: \( -256 \), \( 128 \), \( -64 \)... Найдите сумму первых \( 7 \) членов.
		\item Найдите \( S_4 \), \quad если \( b_n=164\cdot\left( \dfrac{1}{2} \right)^n \)
		\item Бизнесмен Бубликов получил в \( 2000 \) году прибыль в размере \( 5000 \) рублей. Каждый следующий год его прибыль увеличивалась на \( 300\% \) по сравнению с предыдущим годом. Сколько рублей заработал Бубликов за \( 2003 \) год?
		\item Бактерия, попав в живой организм, к концу \( 20 \)-й минуты делится на две бактерии, каждая из них к концу следующих \( 20 \) минут делится опять на две и т. д. Сколько бактерий окажется в организме через \( 4 \) часа, если по истечении четвертого часа в организм из окружающей среды попала еще одна бактерия?
		\item Каждый день больной заражает четырёх человек, каждый из которых, начиная со следующего дня, каждый день также заражает новых четырех и так далее. Болезнь длится \( 14 \) дней. В первый день месяца в город \( N \) приехал заболевший гражданин \( K \), и в это же день он заразил четырех человек. В какой день станет \( 3125 \) заболевших?
		\item Служившему воину дано вознаграждение: за первую рану \( 1 \) копейка, за другую --- \( 2 \) копейки, за третью --- \( 4 \) копейки и т. д. По исчислению нашлось, что воин получил всего вознаграждения \( 655 \) руб. \( 35 \) коп. Спрашивается число его ран.
		\item Клиент взял в банке кредит в размере \( 50 000 \) р. на \( 5 \) лет под \( 20\% \) годовых. Какую сумму он должен вернуть в банк в конце срока, если проценты начисляются ежегодно на текущую сумму долга и весь кредит с процентами возвращается в банк после срока?
		\item Алик, Миша и Вася покупали блокноты и трехкопеечные карандаши. Алик купил \( 2 \) блокнота и \( 4 \) карандаша, Миша --- блокнот и \( 6 \) карандашей, Вася --- блокнот и \( 3 \) карандаша. Оказалось, что суммы, которые уплатили Алик, Миша и Вася, образуют геометрическую прогрессию. Сколько стоит блокнот?
		\item Ваня, Миша, Алик и Вадим ловили рыбу. Оказалось, что количества рыб, пойманных каждым из них, образуют в указанном порядке арифметическую прогрессию. Если бы Алик поймал столько же рыб, сколько Вадим, а Вадим поймал бы на \( 12 \) рыб больше, то количества рыб, пойманных юношами, образовали бы в том же порядке геометрическую прогрессию. Сколько рыб поймал Миша?
		\item В амфитеатре \( 12 \) рядов. В первом ряду \( 15 \) мест, а в каждом следующем на \( 3 \) места больше, чем в предыдущем. Сколько всего мест в амфитеатре?
		\item Занятия йогой начинают с \( 15 \) минут в день и увеличивают на \( 10 \) минут время каждый следующий день. Сколько дней следует заниматься йогой в указанном режиме, чтобы суммарная продолжительность занятий составила \( 2 \) часа?
		\item При свободном падении тело прошло в первую секунду \( 5 \) м, а в каждую следующую на \( 10 \) м больше. Найдите глубину шахты, если свободно падающее тело достигло его дна через \( 5 \) с после начала падения.
		\item Бригада маляров красит забор длиной \( 270 \) метров, ежедневно увеличивая норму покраски на одно и то же число метров. Известно, что за первый и последний день в сумме бригада покрасила \( 90 \) метров забора. Определите, сколько дней бригада маляров красила весь забор.
		\item Васе надо решить \( 434 \) задачи. Ежедневно он решает на одно и то же количество задач больше по сравнению с предыдущим днем. Известно, что за первый день Вася решил \( 5 \) задач. Определите, сколько задач решил Вася в последний день, если со всеми задачами он справился за \( 14 \) дней.
		\item \exercise{1472}
	\end{listofex}
\end{class}
%
%===============>>  Занятие 2  <<===============
%
%
%===============>>  Домашняя работа 3  <<===============
%
%\begin{homework}[number=2]
%	\begin{listofex}
%
%	\end{listofex}
%\end{homework}
%\newpage
%\title{Подготовка к проверочной работе}
%\begin{listofex}
%	
%\end{listofex}

%
%===============>>  Занятие 7  <<===============
%
%
%===============>>  Провечная работа  <<===============
%
%\begin{exam}
%	\begin{listofex}
%	
%	\end{listofex}
%\end{exam}