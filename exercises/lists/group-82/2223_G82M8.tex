%
%===============>>  ГРУППА 8-2 МОДУЛЬ 8  <<=============
%
\setmodule{8}

%BEGIN_FOLD % ====>>_____ Занятие 1 _____<<====
\begin{class}[number=1]
	\begin{listofex}
		\item Решите уравнения:
		\begin{tasks}(2)
			\task \( x^4-20x^2+64=0 \)
			\task \( (x^2+x)^2-8x^2-8x+12=0 \)
		\end{tasks}
		\item Решите уравнения:
		\begin{tasks}(2)
			\task \( \dfrac{x^2-x}{x^2-x+1}-\dfrac{x^2-x+2}{x^2-x-2}=1 \)
			\task \( \dfrac{2x-1}{x}+\dfrac{4x}{2x-1}=5 \)
			\task! \( \dfrac{4}{9x^2-9x+2}-\dfrac{8}{9x^2-9x+8}=1 \)
		\end{tasks}
		\item Решите уравнения:
		\begin{tasks}(2)
			\task \( x^2-7|x|-8=0 \)
			\task \( (x^2-4)|x|+3=0 \)
			\task \( (x-7)^2-|x-7|=30 \)
			\task \( (x-2)^2+2|x-2|-8=0 \)
		\end{tasks}
	\end{listofex}
\end{class}
%END_FOLD

%BEGIN_FOLD % ====>>_____ Занятие 2 _____<<====
\begin{class}[number=2]
	\begin{listofex}
		\item Решите биквадратное уравнение:
		\begin{tasks}(2)
			\task \( 4x^4-41x^2+100=0 \)
			\task \( 6c^4-35=11c^2 \)
			\task \( 10p^4-21=p^2 \)
			\task \( x^4+4x^2-21=0 \)
		\end{tasks}
		\item Решите уравнения:
		\begin{tasks}(2)
			\task \( (2x^2+3)^2-12(2x^2+3)+11=0 \)
			\task \( (x^2+3)^2-11(x^2+3)+28=0 \)
			\task \( (x^2-4x)^2+9(x^2-4x)+20=0 \)
			\task \( (t^2-2t)^2-3=2(t^2-2t) \)
			\task \( (x^2+x-1)(x^2+x+2)=40 \)
			\task \( (2x^2+x-1)(2x^2+x-4)+2=0 \)
		\end{tasks}
	\end{listofex}
\end{class}
%END_FOLD

%BEGIN_FOLD % ====>>_ Домашняя работа 1 _<<====
\begin{homework}[number=1]
	\begin{listofex}
		\item Решите биквадратные уравнения:
		\begin{tasks}(2)
			\task \( 9x^4=9x^2-1 \)
			\task \( 3x^4+21=4x^2 \)
		\end{tasks}
		\item Решите уравнения:
		\begin{tasks}(2)
			\task \( (2x+3)^2=3(2x+3)-2 \)
			\task \( (x^2-2x)^2-4(x^2-2x)+3=0 \)
			\task! \( (x^2+x)(x^2+x-5)=84 \)
		\end{tasks}
		\item Теплоход проходит по течению реки до пункта назначения \( 280 \) км и после стоянки возвращается в пункт отправления. Найдите скорость теплохода в неподвижной воде, если скорость течения равна \( 4 \) км/ч, стоянка длится \( 15 \) часов, а в пункт отправления теплоход возвращается через \( 39 \) часов после отплытия из него.
	\end{listofex}
\end{homework}
%END_FOLD

%BEGIN_FOLD % ====>>_____ Занятие 3 _____<<====
\begin{class}[number=3]
	\begin{listofex}
		\item Решите уравнения:
		\begin{tasks}(2)
			\task \( x^2-7|x|-8=0 \)
			\task \( (x^2-4)|x|+3=0 \)
			\task \( (x-7)^2-|x-7|=30 \)
			\task \( (x-2)^2+2|x-2|-8=0 \)
		\end{tasks}
		\item Решите системы уравнений:
		\begin{itasks}[2]
			\task \exercise{191}
			\task \exercise{190}
			\task \exercise{206}
			\task \exercise{207}
			\task \exercise{219}
			\task \exercise{220}
		\end{itasks}
		\item Решите уравнение: \[ 7\left( x+\dfrac{1}{x} \right)+2\left( x^2+\dfrac{1}{x^2} \right)+9=0 \]
	\end{listofex}
\end{class}
%END_FOLD

%BEGIN_FOLD % ====>>_____ Занятие 4 _____<<====
\begin{class}[number=4]
	\begin{listofex}
		\item Решите системы уравнений:
		\begin{itasks}[2]
			\task \exercise{192}
			\task \exercise{193}
			\task \exercise{194}
			\task \exercise{195}
			\task \exercise{208}
			\task \exercise{209}
			\task \exercise{210}
			\task \exercise{211}
		\end{itasks}
		\item Мать старше дочери на \( 23 \) года, а вместе им \( 51 \) год. Сколько лет дочери?
		\item Девять лет назад брат был вдвое старше сестры. Сколько лет брату и сколько сестре, если брат старше сестры на \( 4 \) года?
		\item Моторная лодка за \( 3 \) часа движения против течения реки и \( 2,5 \) часа по течению проходит \( 98 \) км. Найдите собственную скорость лодки и скорость течения, если за \( 5 \) часов движения по течению она проходит на \( 36 \) км больше, чем за \( 4 \) часа против течения реки.
		\item Вкладчик положил в банк \( 21000 \) р. на два разных счета. По первому из них	банк выплачивает \( 4\% \) годовых, а по второму --- \( 6\% \) годовых. Через год вкладчик получил по процентам \( 1020 \) р. Сколько рублей он положил на каждый счет?
		\item Разность двух натуральных чисел равна \( 48 \) Если первое число разделить на второе, то в частном получится \( 4 \), а в остатке \( 3 \). Найдите эти числа.
		\item Имеется два сплава меди и цинка. Один сплав содержит \( 9\% \), а другой --- \( 30\% \) цинка. Сколько килограммов каждого сплава надо взять, чтобы получить сплав массой \( 300 \) кг, содержащий \( 23\% \) цинка?
	\end{listofex}
\end{class}
%END_FOLD

%BEGIN_FOLD % ====>>_ Домашняя работа 2 _<<====
\begin{homework}[number=2]
	\begin{listofex}
		\item Решите системы уравнений:
		\begin{itasks}[2]
			\task \exercise{197}
			\task \exercise{199}
			\task \exercise{212}
			\task \exercise{214}
		\end{itasks}
		\item Вкладчик положил в банк \( 30000 \) р. на два разных счета. По первому из них банк выплачивает \( 5\% \) годовых, а по второму --- \( 7\% \) годовых. Через год вкладчик получил по первому вкладу на \( 60 \) р. процентных денег больше, чем по второму вкладу. Сколько рублей он положил на каждый счет?
		\item Сумма цифр двузначного числа равна \( 15 \). Если поменять его цифры местами, то получим число, которое меньше данного на \( 9 \). Найдите данное число.
		\item Имеется два водно-солевых раствора. Первый раствор содержит \( 25\% \), а	второй --- \( 40\% \) соли. Сколько килограммов раствора надо взять, чтобы получить \( 50 \) кг раствора, содержащего \( 34\% \) соли?
	\end{listofex}
\end{homework}
%END_FOLD

%BEGIN_FOLD % ====>>_____ Занятие 5 _____<<====
\begin{class}[number=5]
	\begin{listofex}
		\item Решите системы уравнений:
		\begin{tasks}(2)
			\task \( \begin{cases} x(y+1)=16 \\ \dfrac{ x }{ y+1 }=4 \end{cases} \)
			\task \( \begin{cases} \dfrac{ x-1 }{ y+2 }=2 \\ (x-1)^2+(y+2)^2=45 \end{cases} \)
			\task \( \begin{cases} (x+2y)(2x-y+1)=6 \\ \dfrac{ 2x-y+1 }{ x+2y }=\dfrac{ 2 }{ 3 } \end{cases} \)
			\task \( \begin{cases} 3x+y=2(x-y) \\ (3x+y)^2+2(x-y)^2=96 \end{cases} \)
		\end{tasks}
		\item Решите системы уравнений:
		\begin{tasks}(2)
			\task \( \begin{cases} x+y^2=2 \\ 2y^2+x^2=3 \end{cases} \)
			\task \( \begin{cases} x+y^3=2 \\ 2x+x^2+5y^3=8 \end{cases} \)
			\task \( \begin{cases} x+y^2=3 \\ x^4+y^4+6x=29 \end{cases} \)
			\task \( \begin{cases} x^3+y=1 \\ y^3-4y^2+4y+x^6=1 \end{cases} \)
		\end{tasks}
		\item Площадь прямоугольного земельного участка равна \(20\) м\(^2\). Участок обнесен изгородью длиной \(18\) м. Найдите длину и ширину участка
		\item Один комбайнер может убрать урожай пшеницы с участка на \(24\) ч быстрее, чем другой. При совместной работе они закончат уборку урожая за \(35\) часов. Сколько времени потребуется каждому комбайнеру, чтобы одному убрать урожай?
		\item Сумма квад­ра­тов цифр дву­знач­но­го числа равна \(13\). Если от этого числа от­нять \(9\), то по­лу­чит­ся число, за­пи­сан­ное теми же циф­ра­ми, но в об­рат­ном по­ряд­ке. Най­ди­те ис­ход­ное число.
		\item За­ду­ма­ны два на­ту­раль­ных числа, про­из­ве­де­ние ко­то­рых равно \(720\). Если пер­вое число раз­де­лить на вто­рое, то в част­ном по­лу­чит­ся \(3\) и в остат­ке \(3\). Какие числа за­ду­ма­ны?
	\end{listofex}
\end{class}
%END_FOLD

%BEGIN_FOLD % ====>>_____ Занятие 6 _____<<====
\begin{class}[number=6]
	\begin{listofex}
		\item Решите системы уравнеий:
		\begin{tasks}(2)
			\task \exercise{222}
			\task \exercise{223}
			\task \exercise{224}
			\task \exercise{225}
		\end{tasks}
		\item Решите системы уравнений:
		\begin{tasks}(2)
			\task \( \left\{
			\begin{array}{l}
				xy=15\\
				x+y=-5
			\end{array}
			\right. \)
			\task \( \left\{
			\begin{array}{l}
				x-y=2\\
				xy=-13
			\end{array}
			\right. \)
			\task \( \left\{
			\begin{array}{l}
				x+y=3\\
				xy=0
			\end{array}
			\right. \)
			\task \( \left\{
			\begin{array}{l}
				xy=5\\
				x-y=0
			\end{array}
			\right. \)
			\task \( \left\{
			\begin{array}{l}
				x^2+2x+y^2=16\\
				x+y=2
			\end{array}
			\right. \)
			\task \( \left\{
			\begin{array}{l}
				x^2+xy=2\\
				y-3x=7
			\end{array}
			\right. \)
			\task \( \left\{
			\begin{array}{l}
				3x+y=2(x-y)\\
				(3x+y)^2+2(x-y)^2=96
			\end{array}
			\right. \)
			\task \( \left\{
			\begin{array}{l}
				y^2+1-x=0\\
				y^2+y^3=xy
			\end{array}
			\right. \)
		\end{tasks}
		\item Двое рабочих, работая вместе, выполнили всю работу за \(5\)  дней. Если бы первый рабочий работал в два раза быстрее, а второй --- в два раза медленнее, то всю работу они выполнили бы за \(4\) дня. За сколько дней выполнил бы эту работу первый рабочий?
		\item Два трактора различной мощности могут совместно вспахать после за \(9\) ч. Если бы первый трактор работал один \(1,2\) ч, а затем второй -- \(2\) ч, то было бы вспахано только \(20\%\) поля. Сколько часов требуется каждому трактору на вспашку всего поля?
		\item Машинистка рассчитала, что если она будет печатать ежедневно на \(2\) лиса больше установленной нормы, то окончит работу раньше намеченного срока на \(3\) дня. Если же она будет печатать в день на \(4\) листа больше установленной нормы, то окончит работу на \(5\) дней раньше срока. Сколько листов требовалось напечатать машинистке и в какой срок?
	\end{listofex}
\end{class}
%END_FOLD

%BEGIN_FOLD % ====>>_ Домашняя работа 3 _<<====
\begin{homework}[number=3]
	\begin{listofex}
		\item Решите системы уравнений:
		\begin{itasks}[2]
			\task \exercise{226}
			\task \exercise{234}
			\task \exercise{236}
			\task \exercise{237}
		\end{itasks}
		\item Решите системы уравнений:
		\begin{tasks}(2)
			\task \( \left\{
			\begin{array}{l}
				x+y=8\\
				xy=-20
			\end{array}
			\right. \)
			\task \( \left\{
			\begin{array}{l}
				2x^2-y^2=32\\
				2x-y=8
			\end{array}
			\right. \)
			\task \( \left\{
			\begin{array}{l}
				2xy-y=7\\
				x-5y=2
			\end{array}
			\right. \)
			\task \( \left\{
			\begin{array}{l}
				2x+4y=5(x-y)\\
				x^2-y^2=6
			\end{array}
			\right. \)
		\end{tasks}
		\item Одна из сторон прямоугольника на \( 14 \) больше другой. Найдите стороны прямоугольника, если его диагональ равна \( 26 \) см.
		\item После того как смешали \( 12 \) г одной жидкости с \( 14 \) г другой жидкости большей плотности, получили смесь, плотность которой равна \( 1,3 \) г/см\( ^3 \). Какова плотность каждой жидкости, если известно, что плотность одной из них на \( 0,2 \) г/см\( ^3 \) больше плотности другой?
	\end{listofex}
\end{homework}
%END_FOLD

%BEGIN_FOLD % ====>>_____ Занятие 7 _____<<====
\begin{class}[number=7]
	\begin{listofex}
		\item Решите графическим способом системы уравнений:
		\begin{tasks}(2)
			\task \( \begin{cases} y=x-2,\\y=4 \end{cases} \)
			\task \( \begin{cases} y=2x-4,\\y=2-x \end{cases} \)
			\task \( \begin{cases} 3x-y+2=0,\\x+2y+3=0 \end{cases} \)
			\task \( \begin{cases} -4x-4y+2=0,\\2x+2y-1=0 \end{cases} \)
		\end{tasks}
	% ДЛЯ 8-1		\item Определите координаты точек пересечения с осями координат графика функции \( y=2x-7 \).	
		\item Определите координаты точек пересечения с осями координат графика функции \( y=x^2+5x+6 \)
		\item Решите графическим способом системы уравнений:
		\begin{tasks}(2)
			\task \( \begin{cases} y=x^2-2x,\\y=2-x \end{cases} \)
			\task \( \begin{cases} y=\dfrac{6}{x},\\y=x-4 \end{cases} \)
		\end{tasks}
		\item Поезд, выйдя в момент \( t_0=0 \) со станции \( O \), идёт со скоростью \( 100 \) км/ч. Навстречу ему со скоростью \( 80 \) км/ч идёт другой поезд, вышедший со станции \( A \) в тот же момент \( t_0=0 \). Расстояние от \( O \) до \( A \) равно \( 200 \) км. Постройте графики движения этих поездов и по ним определите, когда и на каком расстоянии от станции \( O \) поезда встретятся.
		\item Решите графическим способом систему уравнений:
		\[ \begin{cases} y=|x|+2,\\y=-\dfrac{1}{3}x+4 \end{cases} \]
		Проверьте правильность решения аналитическим способом.
	\end{listofex}
\end{class}
%END_FOLD

%BEGIN_FOLD % ====>>_ Проверочная работа _<<====
\begin{exam}
	\begin{listofex}
		\item Решите уравнения:
		\begin{tasks}(2)
			\task \( x^4-5x^2+6=0 \)
			\task \( x^4+7x^2-30=0 \)
			\task \( (x^2-2x)^2-4(x^2-2x)+3=0 \)
			\task \( (x^2-5x+4)(x^2-5x+6)=120 \)
		\end{tasks}
		\item Решите системы уравнений:
		\begin{tasks}(2)
			\task \( \begin{cases} x^2-y^2=3,\\ x+y=1 \end{cases} \)
			\task \( \begin{cases} x^2-y^2=3,\\ x+y=1 \end{cases} \)
			\task \( \begin{cases} x+y-7=0,\\ x^2+xy+y^2=43 \end{cases} \)
			\task \( \begin{cases} x+y=12,\\ 2xy=9(x-y) \end{cases} \)
		\end{tasks}
		\item Решите графическим способом системы уравнений:
		\begin{tasks}(3)
			\task \( \begin{cases} x-y+1=0,\\ 2x+y=1 \end{cases} \)
			\task \( \begin{cases} 3x+y-1=0,\\ 6x+2y=2 \end{cases} \)
			\task \( \begin{cases} y=2x^2+8x+7,\\ y=-x^2-2x+4 \end{cases} \)
		\end{tasks}
		Проверьте аналитическим способом.
	\end{listofex}
\end{exam}
%END_FOLD

%BEGIN_FOLD % ====>>_____ Консультация 1 _____<<====
\begin{consultation}[number=1]
	\begin{listofex}
		\item 
		\begin{minipage}[t]{\bodywidth}
			Назовите числовой отрезок
		\end{minipage}
		\hspace{0.02\linewidth}
		\begin{minipage}[t]{\picwidth}
			\includegraphics[align=t, width=\linewidth]{\picpath/82M8CONS-1}
		\end{minipage}
		\item 
		\begin{minipage}[t]{\bodywidth}
			Назовите числовой отрезок
		\end{minipage}
		\hspace{0.02\linewidth}
		\begin{minipage}[t]{\picwidth}
			\includegraphics[align=t, width=\linewidth]{\picpath/82M8CONS-2}
		\end{minipage}
		\item Принадлежат ли промежутку \( (-1,8;1,6) \) числа : \( -2 \), \( -1,5 \), \( -1,3 \), \( 0 \), \( 1,4 \), \( 1,6 \)
		\item Изобразите на координатной прямой числовой промежуток, и назовите его:
		\begin{tasks}(3)
			\task \( [3;6] \)
			\task \( (-3;9] \)
			\task \( (-\infty;-6] \)
			\task \( [-5;+\infty) \)
			\task \( [4;16) \)
			\task \( (-\infty;+\infty )\)
		\end{tasks}
		\item Найти пересечение числовых промежутков, используя координатную прямую.
		\begin{tasks}(1)
			\task \( [-2;8] \) и \( (-3;6] \)
			\task \( [-9;3] \) и \( [-5;5] \)
			\task \( (-\infty;-5] \) и \( (-5;+\infty) \)
		\end{tasks}
		\item Найти объединение числовых промежутков, используя координатную прямую.
		\begin{tasks}(1)
			\task \( (-5;2) \) и \( (-2;7] \)
			\task \( [-10;2] \) и \( [-1;10] \)
			\task \( (-\infty;-3] \) и \( (0;+\infty) \)
		\end{tasks}
		\item Изобразите на числовом промежутке:
		\begin{tasks}(2)
			\task \( 2\leq x\leq 10\)
			\task \( -1< x\leq 0\)
			\task \( -\infty < x< 4\)
			\task \( -2\leq x<+\infty \)
		\end{tasks}
		\item Решите неравенства:
		\begin{tasks}(2)
			\task \( x+4<-3 \)
			\task \( x-9\leq 1 \)
			\task \( x>5x-2\)
			\task \( x-4\geq 2x+8\)
		\end{tasks}
	\end{listofex}
\end{consultation}
%END_FOLD