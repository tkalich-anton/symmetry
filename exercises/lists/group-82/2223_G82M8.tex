%
%===============>>  ГРУППА 8-2 МОДУЛЬ 8  <<=============
%
\setmodule{8}

%BEGIN_FOLD % ====>>_____ Занятие 1 _____<<====
\begin{class}[number=1]
	\begin{listofex}
		\item Решите уравнения:
		\begin{tasks}(2)
			\task \( x^4-20x^2+64=0 \)
			\task \( (x^2+x)^2-8x^2-8x+12=0 \)
		\end{tasks}
		\item Решите уравнения:
		\begin{tasks}(2)
			\task \( \dfrac{x^2-x}{x^2-x+1}-\dfrac{x^2-x+2}{x^2-x-2}=1 \)
			\task \( \dfrac{2x-1}{x}+\dfrac{4x}{2x-1}=5 \)
			\task! \( \dfrac{4}{9x^2-9x+2}-\dfrac{8}{9x^2-9x+8}=1 \)
		\end{tasks}
		\item Решите уравнения:
		\begin{tasks}(2)
			\task \( x^2-7|x|-8=0 \)
			\task \( (x^2-4)|x|+3=0 \)
			\task \( (x-7)^2-|x-7|=30 \)
			\task \( (x-2)^2+2|x-2|-8=0 \)
		\end{tasks}
	\end{listofex}
\end{class}
%END_FOLD

%BEGIN_FOLD % ====>>_____ Занятие 2 _____<<====
\begin{class}[number=2]
	\begin{listofex}
		\item Решите биквадратное уравнение:
		\begin{tasks}(2)
			\task 
			\task \( 4x^4-41x^2+100=0 \)
			\task \( 6c^4-35=11c^2 \)
			\task \( 10p^4-21=p^2 \)
		\end{tasks}
	\end{listofex}
\end{class}
%END_FOLD

%BEGIN_FOLD % ====>>_ Домашняя работа 1 _<<====
\begin{homework}[number=1]
	\begin{listofex}
		\item Решите биквадратные уравнения:
		\begin{tasks}(2)
			\task \( 9x^4=9x^2-1 \)
			\task \( 3x^4+21=4x^2 \)
		\end{tasks}
		\item Решите уравнения:
		\begin{tasks}(2)
			\task \( (2x+3)^2=3(2x+3)-2 \)
			\task \( (x^2-2x)^2-4(x^2-2x)+3=0 \)
		\end{tasks}
	\end{listofex}
\end{homework}
%END_FOLD

%BEGIN_FOLD % ====>>_____ Занятие 3 _____<<====
\begin{class}[number=3]
	\begin{listofex}
		\item \( 7\left( x+\dfrac{1}{x} \right)+2\left( x^2+\dfrac{1}{x^2} \right)+9=0 \)
	\end{listofex}
\end{class}
%END_FOLD

%BEGIN_FOLD % ====>>_____ Занятие 4 _____<<====
\begin{class}[number=4]
	\begin{listofex}
		\item Занятие 4
	\end{listofex}
\end{class}
%END_FOLD

%BEGIN_FOLD % ====>>_ Домашняя работа 2 _<<====
\begin{homework}[number=2]
	\begin{listofex}
		\item Домашняя работа 2
	\end{listofex}
\end{homework}
%END_FOLD

%BEGIN_FOLD % ====>>_____ Занятие 5 _____<<====
\begin{class}[number=5]
	\begin{listofex}
		\item Решите систему уравнений:
		\begin{tasks}(2)
			\task
			\(
			\left\{
			\begin{array}{l}
				y^2-x=-1,\\
				x=y+3
			\end{array}
			\right.
			\)
			\task
			\(
			\left\{
			\begin{array}{l}
				xy+x=-4,\\
				x-y=6
			\end{array}
			\right.
			\)
			\task
			\(
			\left\{
			\begin{array}{l}
				xy=x-1,\\
				x^2-2y=26
			\end{array}
			\right.
			\)
			\task
			\(
			\left\{
			\begin{array}{l}
				x=3-y,\\
				y^2-x=39
			\end{array}
			\right.
			\)
			\task
			\(
			\left\{
			\begin{array}{l}
				x-y=3,\\
				xy=-2
			\end{array}
			\right.
			\)
			\task
			\(
			\left\{
			\begin{array}{l}
				x+y=-1,\\
				x^2+y^2=1
			\end{array}
			\right.
			\)
			\task
			\(
			\left\{
			\begin{array}{l}
				x+y=8,\\
				xy=-20
			\end{array}
			\right.
			\)
		\end{tasks}
		\item Решите систему уравнений:
		\begin{tasks}(2)
			\task
			\(
			\left\{
			\begin{array}{l}
				y-2x=2,\\
				5x^2-y=1
			\end{array}
			\right.
			\)
			\task
			\(
			\left\{
			\begin{array}{l}
				x^2-3y^2=52,\\
				y-x=14
			\end{array}
			\right.
			\)
			\task
			\(
			\left\{
			\begin{array}{l}
				x-2y^2=2,\\
				3x+y=7
			\end{array}
			\right.
			\)
			\task
			\(
			\left\{
			\begin{array}{l}
				2xy-y=7,\\
				x-5y=2
			\end{array}
			\right.
			\)
			\task
			\(
			\left\{
			\begin{array}{l}
				2xy-y=7,\\
				x-5y=2
			\end{array}
			\right.
			\)
			\task
			\(
			\left\{
			\begin{array}{l}
				2x+4y=5(x-y),\\
				x^2-y^2=6
			\end{array}
			\right.
			\)
		\end{tasks}
		\item Площадь прямоугольного земельного участка равна \( 20 \) м\(^2\). Участок обнесен изгородью длиной \( 18 \) м. Найдите длину и ширину участка.
		\item Один комбайнер может убрать урожай пшеницы с участка на \( 24 \) ч быстрее, чем другой.
		При совместной работе они закончат уборку урожая за \( 35 \) часов.
		Сколько времени потребуется каждому комбайнеру,
		чтобы одному убрать урожай?
		\item Сумма квадратов цифр двузначного числа равна \( 13 \).
		Если от этого числа отнять \( 9 \), то получится число, записанное теми же цифрами,
		но в обратном порядке. Найдите исходное число.
	\end{listofex}
\end{class}
%END_FOLD

%BEGIN_FOLD % ====>>_____ Занятие 6 _____<<====
\begin{class}[number=6]
	\begin{listofex}
		\item Занятие 6
	\end{listofex}
\end{class}
%END_FOLD

%BEGIN_FOLD % ====>>_ Домашняя работа 3 _<<====
\begin{homework}[number=3]
	\begin{listofex}
		\item Домашняя работа 3
	\end{listofex}
\end{homework}
%END_FOLD

%BEGIN_FOLD % ====>>_____ Занятие 7 _____<<====
\begin{class}[number=7]
	\title{Подготовка к проверочной}
	\begin{listofex}
		\item Занятие 7
	\end{listofex}
\end{class}
%END_FOLD

=%BEGIN_FOLD % ====>>_ Проверочная работа _<<====
\begin{exam}
	\begin{listofex}
		\item Проверочная
	\end{listofex}
\end{exam}
%END_FOLD