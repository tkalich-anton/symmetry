%
%===============>>  ГРУППА 8-2 МОДУЛЬ 8  <<=============
%
\setmodule{8}

%BEGIN_FOLD % ====>>_____ Занятие 1 _____<<====
\begin{class}[number=1]
	\begin{listofex}
		\item Решите уравнения:
		\begin{tasks}(2)
			\task \( x^4-20x^2+64=0 \)
			\task \( (x^2+x)^2-8x^2-8x+12=0 \)
		\end{tasks}
		\item Решите уравнения:
		\begin{tasks}(2)
			\task \( \dfrac{x^2-x}{x^2-x+1}-\dfrac{x^2-x+2}{x^2-x-2}=1 \)
			\task \( \dfrac{2x-1}{x}+\dfrac{4x}{2x-1}=5 \)
			\task! \( \dfrac{4}{9x^2-9x+2}-\dfrac{8}{9x^2-9x+8}=1 \)
		\end{tasks}
		\item Решите уравнения:
		\begin{tasks}(2)
			\task \( x^2-7|x|-8=0 \)
			\task \( (x^2-4)|x|+3=0 \)
			\task \( (x-7)^2-|x-7|=30 \)
			\task \( (x-2)^2+2|x-2|-8=0 \)
		\end{tasks}
	\end{listofex}
\end{class}
%END_FOLD

%BEGIN_FOLD % ====>>_____ Занятие 2 _____<<====
\begin{class}[number=2]
	\begin{listofex}
		\item Решите биквадратное уравнение:
		\begin{tasks}(2)
			\task \( 4x^4-41x^2+100=0 \)
			\task \( 6c^4-35=11c^2 \)
			\task \( 10p^4-21=p^2 \)
			\task \( x^4+4x^2-21=0 \)
		\end{tasks}
		\item Решите уравнения:
		\begin{tasks}(2)
			\task \( (2x^2+3)^2-12(2x^2+3)+11=0 \)
			\task \( (x^2+3)^2-11(x^2+3)+28=0 \)
			\task \( (x^2-4x)^2+9(x^2-4x)+20=0 \)
			\task \( (t^2-2t)^2-3=2(t^2-2t) \)
			\task \( (x^2+x-1)(x^2+x+2)=40 \)
			\task \( (2x^2+x-1)(2x^2+x-4)+2=0 \)
		\end{tasks}
	\end{listofex}
\end{class}
%END_FOLD

%BEGIN_FOLD % ====>>_ Домашняя работа 1 _<<====
\begin{homework}[number=1]
	\begin{listofex}
		\item Решите биквадратные уравнения:
		\begin{tasks}(2)
			\task \( 9x^4=9x^2-1 \)
			\task \( 3x^4+21=4x^2 \)
		\end{tasks}
		\item Решите уравнения:
		\begin{tasks}(2)
			\task \( (2x+3)^2=3(2x+3)-2 \)
			\task \( (x^2-2x)^2-4(x^2-2x)+3=0 \)
		\end{tasks}
	\end{listofex}
\end{homework}
%END_FOLD

%BEGIN_FOLD % ====>>_____ Занятие 3 _____<<====
\begin{class}[number=3]
	\begin{listofex}
		\item \( 7\left( x+\dfrac{1}{x} \right)+2\left( x^2+\dfrac{1}{x^2} \right)+9=0 \)
	\end{listofex}
\end{class}
%END_FOLD

%BEGIN_FOLD % ====>>_____ Занятие 4 _____<<====
\begin{class}[number=4]
	\begin{listofex}
		\item Занятие 4
	\end{listofex}
\end{class}
%END_FOLD

%BEGIN_FOLD % ====>>_ Домашняя работа 2 _<<====
\begin{homework}[number=2]
	\begin{listofex}
		\item Домашняя работа 2
	\end{listofex}
\end{homework}
%END_FOLD

%BEGIN_FOLD % ====>>_____ Занятие 5 _____<<====
\begin{class}[number=5]
	\begin{listofex}
		\item Занятие 5
	\end{listofex}
\end{class}
%END_FOLD

%BEGIN_FOLD % ====>>_____ Занятие 6 _____<<====
\begin{class}[number=6]
	\begin{listofex}
		\item Занятие 6
	\end{listofex}
\end{class}
%END_FOLD

%BEGIN_FOLD % ====>>_ Домашняя работа 3 _<<====
\begin{homework}[number=3]
	\begin{listofex}
		\item Домашняя работа 3
	\end{listofex}
\end{homework}
%END_FOLD

%BEGIN_FOLD % ====>>_____ Занятие 7 _____<<====
\begin{class}[number=7]
	\title{Подготовка к проверочной}
	\begin{listofex}
		\item Занятие 7
	\end{listofex}
\end{class}
%END_FOLD

%BEGIN_FOLD % ====>>_ Проверочная работа _<<====
\begin{exam}
	\begin{listofex}
<<<<<<< HEAD
		\item Проверочная
	\end{listofex}
\end{exam}
%END_FOLD

%BEGIN_FOLD % ====>>_____ Консультация 1 _____<<====
\begin{consultation}[number=1]
	\begin{listofex}
		\item 
		\begin{minipage}[t]{\bodywidth}
			Назовите числовой отрезок
		\end{minipage}
		\hspace{0.02\linewidth}
		\begin{minipage}[t]{\picwidth}
			\includegraphics[align=t, width=\linewidth]{\picpath/82M8CONS-1}
		\end{minipage}
		\item 
		\begin{minipage}[t]{\bodywidth}
			Назовите числовой отрезок
		\end{minipage}
		\hspace{0.02\linewidth}
		\begin{minipage}[t]{\picwidth}
			\includegraphics[align=t, width=\linewidth]{\picpath/82M8CONS-2}
		\end{minipage}
		\item Принадлежат ли промежутку \( (-1,8;1,6) \) числа : \( -2 \), \( -1,5 \), \( -1,3 \), \( 0 \), \( 1,4 \), \( 1,6 \)
		\item Изобразите на координатной прямой числовой промежуток, и назовите его:
		\begin{tasks}(3)
			\task \( [3;6] \)
			\task \( (-3;9] \)
			\task \( (-\infty;-6] \)
			\task \( [-5;+\infty) \)
			\task \( [4;16) \)
			\task \( (-\infty;+\infty )\)
		\end{tasks}
		\item Найти пересечение числовых промежутков, используя координатную прямую.
		\begin{tasks}(1)
			\task \( [-2;8] \) и \( (-3;6] \)
			\task \( [-9;3] \) и \( [-5;5] \)
			\task \( (-\infty;-5] \) и \( (-5;+\infty) \)
		\end{tasks}
		\item Найти объединение числовых промежутков, используя координатную прямую.
		\begin{tasks}(1)
			\task \( (-5;2) \) и \( (-2;7] \)
			\task \( [-10;2] \) и \( [-1;10] \)
			\task \( (-\infty;-3] \) и \( (0;+\infty) \)
		\end{tasks}
		\item Изобразите на числовом промежутке:
		\begin{tasks}(2)
			\task \( 2\leq x\leq 10\)
			\task \( -1< x\leq 0\)
			\task \( -\infty < x< 4\)
			\task \( -2\leq x<+\infty \)
		\end{tasks}
		\item Решите неравенства:
		\begin{tasks}(2)
			\task \( x+4<-3 \)
			\task \( x-9\leq 1 \)
			\task \( x>5x-2\)
			\task \( x-4\geq 2x+8\)
		\end{tasks}
	\end{listofex}
\end{consultation}
=======
	%BEGIN_FOLD % ====>>_ Консультация _<<====
	\begin{consultation}
		\begin{listofex}
			\item 
		\end{listofex}
	\end{consultation}
>>>>>>> b462c8b7b84e7b37a6423b335d847ef2fc3b38cf
%END_FOLD