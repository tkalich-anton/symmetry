%
%===============>>  ГРУППА 8-2 МОДУЛЬ 9  <<=============
%
\setmodule{9}

%BEGIN_FOLD % ====>>_____ Занятие 1 _____<<====
\begin{class}[number=1]
	\begin{listofex}
		\item В прямоугольный треугольник с катетами, равными \(6\) и 8, вписан квадрат, имеющий с треугольником общий прямой угол. Найдите сторону квадрата.
		\item Диагонали \(AC\) и \(BD \) выпуклого четырехугольника \(ABCD\), площадь которого равна \(28\), пересекаются в точке \(O\). Через середины отрезков \(BO\) и \(DO\) проведены прямые, параллельные диагонали \(AC\). Найдите площадь части четырехугольника, заключенной между этими прямыми.
		\item Основания \(AD\) и \(BC\) трапеции \(ABCD\) равны соответственно \(a\) и \(b\). Диагональ \(AC\) разделена на три равные части и через ближайшую к \(A\) точку деления \(M\) проведена прямая, параллельная основаниям. Найдите отрезок этой прямой, заключенный между диагоналями.
		\item Докажите, что медиана \(AM\) треугольника \(ABC\) делит пополам любой отрезок с концами на \(AB\) и \(AC\), параллельный стороне \(BC\). 
		\item Докажите, что точка пересечения диагоналей, точка пересечения продолжений боковых сторон и середины оснований любой трапеции лежат на одной прямой.
	\end{listofex}
\end{class}
%END_FOLD

%BEGIN_FOLD % ====>>_____ Занятие 2 _____<<====
\begin{class}[number=2]
	\begin{listofex}
		\item Докажите, что квадрат высоты прямоугольного треугольника, проведённой к гипотенузе, равен произведению проекций катетов на гипотенузу.
		\item Один из катетов прямоугольного треугольника равен \( 15 \), а проекция второго катета на гипотенузу равна \( 16 \). Найдите гипотенузу и второй катет.
		\item В треугольнике \( ABC \) с прямым углом \( C \) Проведена высота \( CD \). Известно, что на \( BC \) взята точка \( E \) и \( DE \) перпендикулярно \( BC \). Найдите \( DE \), если \( AC=15 \) см, а \( AD=9 \) см.	
		\item Медиана прямоугольного треугольника, проведенная к гипотенузе, равна \( 12 \) и делит прямой угол в отношении \( 1:2 \). Найдите стороны треугольника.
		\item Катеты прямоугольного треугольника равны \( 12 \) и \( 16 \). Найдите медиану, проведенную к гипотенузе.
		\item Найдите высоту трапеции со сторонами, равными \( 10 \), \( 10 \), \( 10 \) и \( 26 \).
		\item Найдите высоту равнобедренного треугольника, проведенную к основанию, если стороны треугольника равны \( 10 \), \( 13 \) и \( 13 \).
		\item Найдите высоту, а также радиусы вписанной и описанной окружностей равностороннего треугольника со стороной, равной \( a \).
		\item Вершина \( M \) правильного треугольника \( ABM \) со стороной \( a \) расположена на стороне \( CD \) прямоугольника \( ABCD \).	Найдите диагональ прямоугольника \( ABCD \).
	\end{listofex}
\end{class}
%END_FOLD

%BEGIN_FOLD % ====>>_ Домашняя работа 1 _<<====
\begin{homework}[number=1]
	\begin{listofex}
		\item Сторона \( AD \) параллелограмма \( ABCD \) разделена на \( n \) равных частей. Первая точка деления \( P \) соединена с вершиной \( B \). Докажите, что прямая \( BP \) отсекает на диагонали \( AC \) часть \( AQ \), которая равна \( \dfrac{1}{n+1} \)	всей диагонали.
		\item Точка \( M \) лежит на боковой стороне \( AC \) равнобедренного треугольника \( ABC \) с основанием \( BC \), причем \( BM=BC \). Найдите \( MC \), если \( BC=1 \) и \( AB=2 \).
		\item Найдите отношение оснований трапеции, если ее	средняя линия делится диагоналями на три равные части.
	\end{listofex}
\end{homework}
%END_FOLD

%BEGIN_FOLD % ====>>_____ Занятие 3 _____<<====
\begin{class}[number=3]
	\begin{listofex}
		\item Прямая, проведенная через вершину \( C \) трапеции \( ABCD \) параллельно диагонали \( BD \), пересекает продолжение основания \( AD \) в точке \( M \). Докажите, что треугольник \( ACM \) равновелик трапеции \( ABCD \).
		\item Проекция диагонали равнобокой трапеции на ее большее основание равна \( a \), боковая сторона равна \( b \). Найдите площадь трапеции, если угол при ее меньшем основании равен \( 150\degree \).
		\item Диагонали четырехугольника разбивают его на четыре треугольника. Известно, что треугольники, прилежащие к двум противоположным сторонам четырехугольника, равновелики. Докажите, что данный четырехугольник --- трапеция или параллелограмм.
		\item Точка внутри параллелограмма соединена со всеми его вершинами. Докажите, что суммы площадей треугольников, прилежащих к противоположным сторонам параллелограмма, равны между собой.
		\item Докажите, что если диагональ какого-нибудь четырехугольника делит другую диагональ пополам, то она разбивает этот четырехугольник на две равновеликие части.
		\item Середины сторон выпуклого четырехугольника последовательно соединены отрезками. Докажите, что площадь	полученного четырехугольника вдвое меньше площади исходного.
	\end{listofex}
\end{class}
%END_FOLD

%BEGIN_FOLD % ====>>_____ Занятие 4 _____<<====
\begin{class}[number=4]
	\begin{listofex}
		\item .
	\end{listofex}
\end{class}
%END_FOLD

%BEGIN_FOLD % ====>>_ Домашняя работа 2 _<<====
\begin{homework}[number=2]
	\begin{listofex}
		\item .
	\end{listofex}
\end{homework}
%END_FOLD

%BEGIN_FOLD % ====>>_____ Занятие 5 _____<<====
\begin{class}[number=5]
	\begin{listofex}
		\item .
	\end{listofex}
\end{class}
%END_FOLD

%BEGIN_FOLD % ====>>_____ Занятие 6 _____<<====
\begin{class}[number=6]
	\begin{listofex}
		\item .
	\end{listofex}
\end{class}
%END_FOLD

%BEGIN_FOLD % ====>>_ Домашняя работа 3 _<<====
\begin{homework}[number=3]
	\begin{listofex}
		\item .
	\end{listofex}
\end{homework}
%END_FOLD

%BEGIN_FOLD % ====>>_____ Занятие 7 _____<<====
\begin{class}[number=7]
	\begin{listofex}
		\item.
	\end{listofex}
\end{class}
%END_FOLD

%BEGIN_FOLD % ====>>_ Проверочная работа _<<====
\begin{exam}
	\begin{listofex}
		\item .
	\end{listofex}
\end{exam}
%END_FOLD