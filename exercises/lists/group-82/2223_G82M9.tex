%
%===============>>  ГРУППА 8-2 МОДУЛЬ 9  <<=============
%
\setmodule{9}

%BEGIN_FOLD % ====>>_____ Занятие 1 _____<<====
\begin{class}[number=1]
	\begin{listofex}
		\item В прямоугольный треугольник с катетами, равными \(6\) и 8, вписан квадрат, имеющий с треугольником общий прямой угол. Найдите сторону квадрата.
		\item Диагонали \(AC\) и \(BD \) выпуклого четырехугольника \(ABCD\), площадь которого равна \(28\), пересекаются в точке \(O\). Через середины отрезков \(BO\) и \(DO\) проведены прямые, параллельные диагонали \(AC\). Найдите площадь части четырехугольника, заключенной между этими прямыми.
		\item Основания \(AD\) и \(BC\) трапеции \(ABCD\) равны соответственно \(a\) и \(b\). Диагональ \(AC\) разделена на три равные части и через ближайшую к \(A\) точку деления \(M\) проведена прямая, параллельная основаниям. Найдите отрезок этой прямой, заключенный между диагоналями.
		\item Докажите, что медиана \(AM\) треугольника \(ABC\) делит пополам любой отрезок с концами на \(AB\) и \(AC\), параллельный стороне \(BC\). 
		\item Докажите, что точка пересечения диагоналей, точка пересечения продолжений боковых сторон и середины оснований любой трапеции лежат на одной прямой.
	\end{listofex}
\end{class}
%END_FOLD

%BEGIN_FOLD % ====>>_____ Занятие 2 _____<<====
\begin{class}[number=2]
	\begin{listofex}
		\item .
	\end{listofex}
\end{class}
%END_FOLD

%BEGIN_FOLD % ====>>_ Домашняя работа 1 _<<====
\begin{homework}[number=1]
	\begin{listofex}
		\item .
	\end{listofex}
\end{homework}
%END_FOLD

%BEGIN_FOLD % ====>>_____ Занятие 3 _____<<====
\begin{class}[number=3]
	\begin{listofex}
		\item .
	\end{listofex}
\end{class}
%END_FOLD

%BEGIN_FOLD % ====>>_____ Занятие 4 _____<<====
\begin{class}[number=4]
	\begin{listofex}
		\item .
	\end{listofex}
\end{class}
%END_FOLD

%BEGIN_FOLD % ====>>_ Домашняя работа 2 _<<====
\begin{homework}[number=2]
	\begin{listofex}
		\item .
	\end{listofex}
\end{homework}
%END_FOLD

%BEGIN_FOLD % ====>>_____ Занятие 5 _____<<====
\begin{class}[number=5]
	\begin{listofex}
		\item .
	\end{listofex}
\end{class}
%END_FOLD

%BEGIN_FOLD % ====>>_____ Занятие 6 _____<<====
\begin{class}[number=6]
	\begin{listofex}
		\item .
	\end{listofex}
\end{class}
%END_FOLD

%BEGIN_FOLD % ====>>_ Домашняя работа 3 _<<====
\begin{homework}[number=3]
	\begin{listofex}
		\item .
	\end{listofex}
\end{homework}
%END_FOLD

%BEGIN_FOLD % ====>>_____ Занятие 7 _____<<====
\begin{class}[number=7]
	\begin{listofex}
		\item.
	\end{listofex}
\end{class}
%END_FOLD

%BEGIN_FOLD % ====>>_ Проверочная работа _<<====
\begin{exam}
	\begin{listofex}
		\item .
	\end{listofex}
\end{exam}
%END_FOLD