%
%===============>>  ГРУППА 8-2 МОДУЛЬ 4  <<=============
%
\setmodule{4}
%
%===============>>  Занятие 1  <<===============
%
\begin{class}[number=1]
	\begin{listofex}
		\item Вычислить рациональным способом:
		\begin{enumcols}[itemcolumns=3]
			\item \( \sqrt{16+4\cdot4\cdot24} \)
			\item \( \sqrt{83^3\cdot2^2-83^2\cdot2^3} \)
			\item \( \sqrt{50^2-4\cdot7\cdot7} \)
		\end{enumcols}
		\item Решить уравнение:
		\begin{enumcols}[itemcolumns=4]
			\item \( 81x^2-16=0 \)
			\item \( 5x^2-25=0 \)
			\item \( -7x^2=-1 \)
			\item \( 50-2x^2=0 \)
		\end{enumcols}
		\item Решить уравнение:
		\begin{enumcols}[itemcolumns=3]
			\item \( 7x^2=-x \)
			\item \( x=x^2 \)
			\item \( 11x-5x^2 \)
		\end{enumcols}
		\item Решить уравнение:
		\begin{enumcols}[itemcolumns=2]
			\item \( (15-x)(x-2)=(x-6)(x+5) \)
			\item \( (x-1)(x-2)+(x+4)(x-4)+3x=0 \)
		\end{enumcols}
		\item Решить уравнение:
		\begin{enumcols}[itemcolumns=2]
			\item \( (x+1)^2=9 \)
			\item \( (2x-3)^2=121 \)
			\item \( \left( \dfrac{1}{2}x+\dfrac{2}{3} \right)^2=\dfrac{4}{9} \)
			\item \( (2,5x-10)^2=0,25 \)
		\end{enumcols}
		\item Решить уравнение:
		\begin{enumcols}[itemcolumns=2]
			\item \( (x-7)^2=3 \)
			\item \( (x+5)^2=5 \)
		\end{enumcols}
		\item Решить уравнение:
		\begin{enumcols}[itemcolumns=3]
			\item \( x^2-10x+25=0 \)
			\item \( 4x^2+20x+25=0 \)
			\item \( 16x^2-24x+9=0 \)
		\end{enumcols}
		\item Решить уравнение:
		\begin{enumcols}[itemcolumns=3]
			\item \( x^2-12x+36=4 \)
			\item \( 4x^2+40x+100=81 \)
			\item \( 9x^2-60x+100=25 \)
		\end{enumcols}
		\item Решить уравнение:
		\begin{enumcols}[itemcolumns=3]
			\item \( x^2-18x+77=0 \)
			\item \( x^2-10x-39=0 \)
			\item \( x^2-22x+72=0 \)
		\end{enumcols}
		\item Решить уравнение: \( (2x-3)^2-(x-5)(x+5)=2(2x+7) \)
		\item Решить уравнение: \( 9x^2-2|x|=0 \)
		\item Площадь прямоугольника равна \( 36 \) м\( ^2 \) . Если его длину увеличить на \( 6 \) м, а ширину уменьшить на \( 1 \) м, то площадь полученного прямоугольника будет равна \( 60 \) м\( ^2 \) . Найдите длину полученного прямоугольника.
	\end{listofex}
\end{class}
%
%===============>>  Занятие 2  <<===============
%
\begin{class}[number=2]
	\begin{listofex}
	\item Вычислить рациональным способом:
	\begin{enumcols}[itemcolumns=3]
		\item \( \sqrt{2^2+4\cdot15} \)
		\item \( \sqrt{90^2-4\cdot25\cdot81} \)
		\item \( \sqrt{4^2+4\cdot5\cdot12} \)
	\end{enumcols}
	\item Решить уравнение:
	\begin{enumcols}[itemcolumns=3]
		\item \exercise{392}
		\item \exercise{397}
		\item \exercise{391}
		\item \exercise{400}
		\item \exercise{393}
		\item \exercise{396}
	\end{enumcols}
	\item Решить уравнение:
	\begin{enumcols}[itemcolumns=3]
		\item \exercise{404}
		\item \exercise{416}
		\item \exercise{418}
		\item \exercise{421}
		\item \exercise{424}
		\item \exercise{425}
	\end{enumcols}
	\item От листа жести, имеющего форму квадрата, отрезали полосу шириной \( 3 \) см. Площадь его оставшейся части равна \( 10 \) см\( ^2 \) . Найдите первоначальные размеры листа жести.
	\item Решить уравнение: \( (2x-3)(x+1)+(x-6)(x+6)+x=0 \)
	\item \exercise{4140}
	\item Решить уравнение:
		\begin{enumcols}[itemcolumns=3]
			\item \( (3x+4)^2=16 \)
			\item \( (17-2x)^2=81 \)
			\item \( (5-3x)^2=16 \)
			\item \( \left( \dfrac{1}{2}-3y \right)^2=100 \)
		\end{enumcols}
		\item Решить уравнение:
		\begin{enumcols}[itemcolumns=3]
			\item \( (x-4)^2=6 \)
			\item \( (0,2b+2)^2=31 \)
			\item \( \left( \dfrac{1}{2}a^2+4 \right)^2=10 \)
		\end{enumcols}
		\item Решить уравнение:
		\begin{enumcols}[itemcolumns=2]
			\item \( 4x^2-24x+36=0 \)
			\item \( 169y^2+572y+484=0 \)
			\item \( -0,6x+0,01+9x^2=0,25 \)
			\item \( 25x^2-x+\dfrac{1}{100}=100 \)
		\end{enumcols}
	\end{listofex}
\end{class}
%
%===============>>  Домашняя работа 1  <<===============
%
%\begin{homework}[number=1]
%	\begin{listofex}
%		\item Пусто
%	\end{listofex}
%\end{homework}
%
%===============>>  Занятие 3  <<===============
%
%\begin{class}[number=3]
%	\begin{listofex}
%		\item Пусто
%	\end{listofex}
%\end{class}
%
%===============>>  Занятие 4  <<===============
%\begin{class}[number=4]
%	\begin{listofex}
%		\item Пусто
%	\end{listofex}
%\end{class}
%
%===============>>  Домашняя работа 2  <<===============
%
\begin{homework}[number=2]
	\begin{listofex}
		\item \begin{enumcols}[itemcolumns=2]
			\item \exercise{414}
			\item \exercise{413}
			\item \exercise{434}
			\item \exercise{435}
		\end{enumcols}
		\item Докажите, что если в четырехугольнике суммы углов, прилежащих к двум смежным сторонам равны по \( 180\degree \), то такой четырехугольник -- параллелограмм.
		\item Через центр параллелограмма \( ABCD \) проведены две прямые. Одна из них пересекает стороны \( АВ \) и \( CD \) соответственно в точках \( М \) и \( К \), вторая -- стороны \( ВС \) и \( AD \) соответственно в точках \( N \) и \( L \). Докажите, что четырехугольник \( MNKL \) -- параллелограмм.
		\item Диагонали параллелограмма \( ABCD \) пересекаются в точке \( O \). Периметр параллелограмма равен \( 12 \), а разность периметров треугольников \( BOC \) и \( COD \) равна \( 2 \). Найдите стороны параллелограмма.
		
		\item \exercise{1765}
		\item \exercise{1758}
	\end{listofex}
\end{homework}
%
%===============>>  Занятие 5  <<===============
\begin{class}[number=5]
	\begin{center}
		СВОЙСТВА ПАРАЛЛЕЛОГРАММА 
	\end{center}
	\begin{tasks}(1)
		\task Противоположные стороны параллелограмма параллельны и равны;
		\task Противоположные углы параллелограмма равны;
		\task Диагонали параллелограмма делятся точкой пересечения пополам;
		\task Сумма углов, прилежащих к одной стороне параллелограмма, равна \( 180\degree \).
	\end{tasks}
	\begin{listofex}
		\item В четырёхугольнике \( ABCD \) известно, что \( AB=BC \) и \( AD=DC \),
		\( \angle ABC = 57\degree \), а \( \angle ADC = 141\degree \). Найдите \( \angle BCD \). Ответ дайте в градусах.
		\item Докажите, что если в четырехугольнике диагонали пересекаются и делятся точкой пересечения пополам,
		то такой четырехугольник --- параллелограмм.
		\item Докажите, что если в четырехугольнике противолежащие стороны попарно равны, то такой четырехугольник --- параллелограмм.
		\item Докажите, что если в четырехугольнике противолежащие углы попарно равны, то такой четырехугольник --- параллелограмм.
		\item Сторона параллелограмма втрое больше другой стороны, а его периметр равен \( 24 \). Найдите его стороны.
		\item Диагональ \( AC \) параллелограмма \( ABCD \) вдвое длиннее его стороны \( AB \).
		Найдите острый угол между диагоналями параллелограмма, если \( \angle ACD=40\degree \).
		Ответ дайте в градусах.
		% В следующий урок
		%BEGIN_FOLD
%			\item Биссектриса угла параллелограмма делит его сторону на отрезки \( 3 \) и \( 4 \).
%			Найдите стороны параллелограмма.
%			\item Сторона \( BC \) параллелограмма \( ABCD \) вдвое больше стороны \( AB \).
%			Биссектрисы углов \( A \) и \( B \) пересекают прямую \( CD \) в точках \( M \) и \( N \), причем \( MN=12 \).
%			Найдите стороны параллелограмма.
%			\item Высота параллелограмма, проведенная из вершины тупого угла,
%			равного \( 150\degree \), равна двум и делит сторону
%			параллелограмма пополам. Найдите диагональ,
%			проведенную из вершины тупого угла, и углы,
%			которые она образует со сторонами.
		%END_FOLD
		\item Решите уравнение:
		\begin{itasks}[4]
			\task \exercise{459}
			\task \exercise{460}
			\task \exercise{457}
			\task \exercise{483}
		\end{itasks}
		\item Решите уравнение:
		\begin{itasks}[2]
			\task \exercise{493}
			\task \exercise{3669}
		\end{itasks}
	\end{listofex}
\end{class}
%
%===============>>  Занятие 6  <<===============
%
\begin{class}[number=6]
	\begin{listofex}
		% С пршлого урока
		%BEGIN_FOLD
			\item Биссектриса угла параллелограмма делит его сторону на отрезки \( 3 \) и \( 4 \).
			Найдите стороны параллелограмма.
			\item Сторона \( BC \) параллелограмма \( ABCD \) вдвое больше стороны \( AB \).
			Биссектрисы углов \( A \) и \( B \) пересекают прямую \( CD \) в точках \( M \) и \( N \), причем \( MN=12 \).
			Найдите стороны параллелограмма.
			\item Высота параллелограмма, проведенная из вершины тупого угла,
			равного \( 150\degree \), равна двум и делит сторону
			параллелограмма пополам. Найдите диагональ,
			проведенную из вершины тупого угла, и углы,
			которые она образует со сторонами.
		%END_FOLD
		
		\item Докажите, что в параллелограмме отрезок с концами на противоположных сторонах, проходящий через точку пересечения диагоналей, делится точкой пересечения диагоналей пополам.
		\item В треугольнике \( АВС \) медиана \( АМ \) продолжена за точку \( М \) до точки \( D \) на расстояние, равное \( AM \) (так что \( AM=MD\)). Докажите, что \( ABCD \) --- параллелограмм.
		\item Докажите, что отрезок, соединяющий середины противоположных сторон параллелограмма, проходит через его центр.
		\item Точки \( K \), \( L \), \( M \) и \( N \) --- середины сторон соответственно \( AB \), \( BC \), \( CD \) и \( AD \) параллелограмма \( ABCD \). Докажите, что четырехугольник с вершинами в точках пересечения прямых \( AL \), \( BM \), \( CN \) и \( DK \) -- параллелограмм.
		\item Окружность, построенная на стороне \( AD \) параллелограмма \( ABCD \) как на диаметре, проходит через вершину \( B \) и середину стороны \( BC \). Найдите углы параллелограмма.
		\item Через каждую вершину параллелограмма проведена прямая, перпендикулярная диагонали, не проходящей через эту вершину. Докажите, что диагонали четырехугольника, образованного пересечениями четырех проведенных таким образом прямых, перпендикулярны сторонам параллелограмма.
		\item Решите уравнение:
		\begin{itasks}[2]
			\task \exercise{458}
			\task \exercise{460}
			\task \exercise{461}
			\task \exercise{468}
		\end{itasks}
	\end{listofex}
\end{class}
%
%===============>>  Домашняя работа 3  <<===============
%
\begin{homework}[number=3]
	\begin{listofex}
		\item В треугольнике \( ABC \) проведена биссектриса \( AL \). Известно,что \( \angle ALC=130\degree \),
		а \( \angle ABC =103\degree \).
		Найдите \( \angle ACB \).
		Ответ дайте в градусах.
		\item Один из углов треугольника равен \( 43\degree \), а другой \( 57\degree \).
		Найдите величину острого угла между высотами треугольника,
		проведёнными из вершин указанных углов.
		Ответ дайте в градусах.
		\item Найдите величину тупого угла между биссектрисами острых
		углов прямоугольного треугольника. Ответ дайте в градусах.
	\end{listofex}
\end{homework}
%\newpage
%\title{Подготовка к проверочной работе}
%\begin{listofex}
%	
%\end{listofex}
%

%
%===============>>  Провечная работа  <<===============
%
\begin{exam}
	\begin{listofex}
		\item В параллелограмме  \(ABCD\) диагональ \(AC\) со сторонами \(AB\) и \(BC\) образует углы, равные соответственно равные \(45\degree\) и \(25\degree\). Чему равна величина угла \( C \)?
		\item В параллелограмме \(ABCD\)  с острым углом \(A\) из вершины \(B\) опущен перпендикуляр \(BK\) к прямой \(AD\), \(AK=BK\). Найдите углы \(C\) и \(D\).
		\item Периметр параллелограмма равен \( 18,6 \) см, а одна из его сторон больше другой стороны в \( 2 \)  раза. Найдите стороны параллелограмма.
		\item Докажите, что если в четырехугольнике противоположные стороны попарно равны, то этот четырехугольник --- параллелограмм. 
		\item Решите уравнения:
		\begin{tasks}(2)
			\task \(x^2-14x+33=0\)
			\task \(-3x^2+10x-3=0\)
			\task \(x^2-11x=42\)
			\task \((3x-1)(3x+1)-(x-1)(x+2)=8\)
		\end{tasks}
		\item Найдите сумму и произведение корней:\quad\(x^2+11x-12=0\)
		\item \( BK \) --- биссектриса треугольника \( ABC \). Известно,
		что \( \angle AKB : \angle AKB = 4 : 5 \).
		Найдите разность углов \( A \) и \( C \) треугольника \( ABC \).
	\end{listofex}
\end{exam}