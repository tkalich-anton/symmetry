%
%===============>>  ГРУППА 8-2 МОДУЛЬ 6  <<=============
%
\setmodule{6}

%BEGIN_FOLD % ====>>_____ Занятие 1 _____<<====
\begin{class}[number=1]
	\begin{listofex}
		\item \exercise{1462}
		\item Постройте график функции \( y=2x-5 \)
		\begin{tasks}(1)
			\task Проверьте \textit{(графическим, а затем аналитическим способом)}, принадлежит ли точка с координатами \( (4;3) \) графику этой функции?
			\task Найдите абсциссу точки на графике функции, ордината которой равна \( 217 \).
			\task Найдите координаты точек пересечения графика данной функции с графиком функции \( y=4x-1 \).
			\task Найдите уравнение прямой, которая параллельна исходной и проходит через начало координат.
			\task Найдите уравнение прямой, которая параллельна исходной и проходит точку \( (-1;1) \).
		\end{tasks}
		\item Постройте график функции \(y=x^2\) и найдите координаты точек пересечения с прямой \( y=2x \) графическим, а затем аналитическим способом.
		\item Найдите аналитическим способом точки пересечения графиков функций \(f(x)=x^2+3x-10\)	и \( g(x)=-3x^2-9x-19 \).
		\item Парабола вида \( y=ax^2 \) проходит через точку \( (-1;3) \). Найдите \( a \).
		\item Найдите такие значения переменной \( x \), при которых значение функции \( y=6x^2-x-12 \) равнялось нулю.
		\item Цена на некоторый товар была снижена дважды --- сначала на \( 15\% \), а потом ещё на \( 20\% \). Каков общий процент снижения цены?
		\item Цена на товар в течение месяца упала сначала на \( 40\% \), а потом увеличилась на \( 50\% \) и составила \( 5130  \) рублей. Найдите первоначальную цену товара.
	\end{listofex}
\end{class}
%END_FOLD

%BEGIN_FOLD % ====>>_____ Занятие 2 _____<<====
\begin{class}[number=2]
	\begin{listofex}
		\item Цена на товар в течение месяца упала сначала на \( 40\% \), а потом увеличилась на \( 50\% \) и составила \( 5130  \) рублей. Найдите первоначальную цену товара.
		\item \exercise{1423}
		\item Постройте график функции \( y=-2x^2 \) и найдите координаты точек пересечения с прямой \( y=-3x+1 \) графическим, а затем аналитическим способом.
		\item Найдите аналитическим способом точки пересечения графиков функций \(f(x)=-3x^2+2x+1\)	и \( g(x)=x^2-6x+1 \).
		\item Постройте график функции \( y=x^2 \) и определите:
		\begin{tasks}(1)
			\task Промежутки возрастания функции;
			\task Промежутки убывания функции;
			\task Область определения функции;
			\task Область значения функции.
		\end{tasks}
		\item Постройте график \( y=-\dfrac{1}{5}x^2 \) и определите:
		\begin{tasks}(1)
			\task Промежутки возрастания функции;
			\task Промежутки убывания функции;
			\task Область определения функции;
			\task Область значения функции.
		\end{tasks}
		\item Найдите координаты вершины параболы 
		\begin{tasks}(2)
			\task \( y=6x^2 \)
			\task \( y=2x^2-16x+37 \)
			\task \( y=-6x^2+12x-6 \)
			\task \( y=x^2+4 \)
			\task \( y=(x+3)^2-3 \)
			\task \( y=-\dfrac{1}{2}(x-5)^2+2 \)
		\end{tasks}
		\item Найдите аналитическим способом точки пересечения графиков функций \(f(x)=x^2+3x-18\)	и \( g(x)=-x^2+x+6 \).
	\end{listofex}
\end{class}
%END_FOLD

%BEGIN_FOLD % ====>>_ Домашняя работа 1 _<<====
\begin{homework}[number=1]
	\begin{listofex}
		\item Найдите такие значения переменной \( x \), при которых значении функции \( y=-7x^2+19x+6 \) равнялось нулю.
		\item Постройте график функции \( y=3x+1 \)
		\begin{tasks}(1)
			\task Проверьте \textit{(графическим, а потом аналитическим способом)}, принадлежит ли точка с координатами \( (2;7) \) графику этой функции?
			\task Найдите точку на графике функции, ордината которой равна \( 214 \).
			\task Найдите точку пересечения графика данной функции с графиком функции \( y=7x-2 \).
			\task Найдите уравнение прямой, которая параллельна исходной и проходит через начало координат.
			\task Найдите уравнение прямой, которая параллельна исходной и проходит точку \( (-1;1) \).
		\end{tasks}
		\item Найдите координаты вершины параболы:
		\begin{tasks}(2)
			\task \( y=-x^2+12 \)
			\task \( y=2(x-5)^2+7 \)
			\task \( y=-\dfrac{1}{3}(x+6)^2-2 \)
			\task \( y=\dfrac{1}{2}(x-1)^2+3 \)
		\end{tasks}
		\item \exercise{1422}
	\end{listofex}
\end{homework}
%END_FOLD

%BEGIN_FOLD % ====>>_____ Занятие 3 _____<<====
\begin{class}[number=3]
	\begin{listofex}
		\item Какие координаты имеет вершина параболы:
		\begin{tasks}(2)
			\task \( y=(x+1)^2 \)
			\task \( y=3(x+9)^2 \)
			\task \( y=-2(x-5)^2 \)
			\task \( y=-4(x-9)^2 \)
		\end{tasks}
		\item Запишите уравнения оси симметрии параболы:
		\begin{tasks}(2)
			\task \( y=(x+5)^2 \)
			\task \( y=-2(x-1)^2 \)
		\end{tasks}
		\item Дана парабола \( y=(x-2)^2 \). 
		\begin{tasks}(1)
			\task Определите координаты вершины параболы;
			\task Запишите уравнение оси симметрии параболы;
			\task Определите область значений и определения функции;
			\task Постройте график функции;
			\task Как изменяется \( y \), если аргумент \( x \) изменяется от \( -\infty  \) до \( 2 \)? от \( 2 \) до \( +\infty \)?
			\task При каком значении \( x \) функция принимает наименьшее значение? Принимает ли функция наибольшее значение?
			\task В каких точках график функции пересекает оси координат?
		\end{tasks} 
		\item Постройте график функции:
		\begin{tasks}(2)
			\task \( y=-(x-4)^2+1 \)
			\task \( y=-3(x+1)^2 \)
			\task \( y=0,5(x+3)^2-1,5 \)
		\end{tasks}
		\item Выделите полный квадрат и постройте график функции:
		\begin{tasks}(2)
			\task \( y=x^2-4x+3 \)
			\task \( y=4x^2-4x-1 \)
			\task \( y=-x^2-6x-5 \)
			\task \( y=x^2-4x+7 \)
		\end{tasks}
		\item Постройте график функции: 
		\[y=\dfrac{(x^2+4x+3)(x-2)}{x+1}\]
	\end{listofex}
\end{class}
%END_FOLD

%BEGIN_FOLD % ====>>_____ Занятие 4 _____<<====
\begin{class}[number=4]
	\begin{listofex}
		\item Занятие 4
	\end{listofex}
\end{class}
%END_FOLD

%BEGIN_FOLD % ====>>_ Домашняя работа 2 _<<====
\begin{homework}[number=2]
	\begin{listofex}
		\item Постройте график функции:
		\begin{tasks}(2)
			\task \( y=(x-13)^2 \)
			\task \( y=-2(x+3)^2+4 \)
			\task \( y=-(x-3)^2-1 \)
			\task \( y=0,5(x+2)^2 \)
		\end{tasks}
		\item Дана функция \( y=2(x-3)^2 \)
		\begin{tasks}(1)
			\task Постройте график функции;
			\task Какова область определения функции?
			\task При каком \( x \) функция принимает наименьшее значение? Принимает ли функция наибольшее значение?
			\task В каких точках график функции пересекает оси координат?
			\task Как изменяется \( y \), если аргумент \( x \) изменяется от \( -\infty  \) до \( -1 \)? от \( 0 \) до \( 1 \)?
		\end{tasks}
		\item Камень брошен вертикально вверх. Пока камень не упал, высота, на которой он находится, описывается формулой \( h(t)=-5t^2+18t \), где \( h \) --- высота в метрах, \( t \) --- время в секундах, прошедшее с момента броска. Постройте график зависимости высоты от времени и графически определите, сколько секунд камень находился на высоте не менее \( 9 \) метров.
	\end{listofex}
\end{homework}
%END_FOLD

%BEGIN_FOLD % ====>>_____ Занятие 5 _____<<====
\begin{class}[number=5]
	\begin{definit}
	\textbf{Числовые промежутки} или просто \textbf{промежутки} --- это числовые множества, которые можно изобразить на координатной прямой. К числовым промежуткам относятся лучи, отрезки, интервалы и полуинтервалы.
	\end{definit}
	\begin{definit}
	\textbf{Открытый луч} --- это множество точек прямой, лежащих по одну сторону от граничной точки, которая не входит в данное множество. Открытым луч называется именно из-за граничной точки, которая ему не принадлежит.
	\end{definit}
	\begin{definit}
	\textbf{Замкнутый луч} --- это множество точек прямой, лежащих по одну сторону от граничной точки, принадлежащей данному множеству. На чертежах граничные точки, принадлежащие рассматриваемому множеству, обозначаются закрашенным кругом.
	\end{definit}
	\begin{definit}
	\textbf{Отрезок} — это множество точек прямой, лежащих между двумя граничными точками, принадлежащими данному множеству. Такие множества задаются двойными нестрогими неравенствами.
	\end{definit}
	\begin{definit}
	\textbf{Интервал} --- это множество точек прямой, лежащих между двумя граничными точками, не принадлежащими данному множеству. Такие множества задаются двойными строгими неравенствами.	
	\end{definit}
	\begin{definit}
	\textbf{Полуинтервал} --- это множество точек прямой, лежащих между двумя граничными точками, одна из которых принадлежит множеству, а другая не принадлежит. Такие множества задаются двойными неравенствами.
	\end{definit}	
	\begin{listofex}
		\item Какому числовому промежутку принадлежит число \( \dfrac{1}{6} \)?
		\begin{tasks}(4)
			\task \( \left( \dfrac{1}{3};\dfrac{1}{2}\right)  \)
			\task \( \left[ 0;\dfrac{1}{12}\right]  \)
			\task \( \left[ \dfrac{1}{14};\dfrac{2}{5}\right]  \)
			\task \( \left( \dfrac{3}{5};\dfrac{7}{10}\right)  \)
		\end{tasks}
		\item Какому числовому промежутку принадлежит число \( \sqrt{19} \)?
		\begin{tasks}(4)
			\task \( (2;3) \)
			\task \( [3;4] \)
			\task \( [4;5) \)
			\task \( [5;6) \)
		\end{tasks}
		\item Изобразите на числовой прямой множество:
		\begin{tasks}(4)
			\task \( x\leq3 \)
			\task \( x>5 \)
			\task \( x\geq\dfrac{1}{5} \)
			\task \( x<-1 \)
		\end{tasks}
		\item Изобразите на числовой прямой множество:
		\begin{tasks}(2)
			\task \( \left\{
			\begin{array}{l}
				x<4,\\
				x\geq-6
			\end{array}
			\right. \)
			\task \( \left\{
			\begin{array}{l}
				x>-2,\\
				x<10
			\end{array}
			\right. \)
			\task \( \left\{
			\begin{array}{l}
				x>0,\\
				x<-12
			\end{array}
			\right. \)
			\task \( \left\{
			\begin{array}{l}
				x>-0,4,\\
				x\geq0,9
			\end{array}
			\right. \)
		\end{tasks}
		\item Решите линейные неравенства:
		\begin{tasks}(4)
			\task \( 3x\geq3 \)
			\task \( 4x<12 \)
			\task \( x+2\leq3 \)
			\task \( x-0,3>0 \)
			\task \( 2x+1>0 \)
			\task \( 3x-5<10\)
			\task \( \dfrac{2}{3}+2x\leq0 \)
			\task \( 7+14x>0 \)
		\end{tasks}
	\end{listofex}
	\begin{definit}
		При делении или умножении каждой части неравенства на отрицательное число знак неравенства меняется на противоположный. 
	\end{definit}
	\begin{listofex}[resume]
		\item Решите линейные неравенства:
		\begin{tasks}(4)
			\task \( -2x>2 \)
			\task \( -3x\leq-9 \)
			\task \( -5x-3<8 \)
			\task \( \dfrac{1}{5}-2x<\dfrac{1}{10} \)
		\end{tasks}
		\item Решите системы линейных неравенств:
		\begin{tasks}(2)
			\task \( \left\{
			\begin{array}{l}
				-x>4,\\
				3x-1\geq0
			\end{array}
			\right. \)
		\task \( \left\{
		\begin{array}{l}
			x>2,\\
			-x>10
		\end{array}
		\right. \)
		\end{tasks}
		\item Используя выделение квадрата двучлена, докажите неравенство:
		\begin{tasks}(2)
			\task \( a^2-6a+14>0 \)
			\task \( b^2+70>16b \)
		\end{tasks}
		\item Решите системы линейных неравенств:
		\begin{tasks}(2)
			\task \( \left\{
			\begin{array}{l}
				2x+7>3-x,\\
				\dfrac{1}{3}x-1>2x-\dfrac{1}{4}
			\end{array}
			\right. \)
			\task \( \left\{
			\begin{array}{l}
				\dfrac{2}{3}x>8,\\[1em]
				\dfrac{3}{4}x-1>\dfrac{3}{5}x-1
			\end{array}
			\right. \)
		\end{tasks}
		\item Решите двойное неравенство:
		\begin{tasks}(2)
			\task \( 0<3x<2 \)
			\task \( -8<0,5x+1<-4 \)
		\end{tasks}
	\end{listofex}
\end{class}
%END_FOLD

%BEGIN_FOLD % ====>>_____ Занятие 6 _____<<====
\begin{class}[number=6]
	\begin{listofex}
		\item Решите систему неравенств:
		\begin{tasks}(2)
			\task \( \left\{
			\begin{array}{l}
				\dfrac{x-1}{2}<1,\\
				4-x>\dfrac{x-5}{3}
			\end{array}
			\right. \)
			\task \( \left\{
			\begin{array}{l}
				\dfrac{2x+1}{3}>\dfrac{3-x}{2},\\
				3x-1\geq0
			\end{array}
			\right. \)
		\end{tasks}
		\item Найдите все \( x \), для которых значение функции \( y=2x-3 \) больше значения функции \( y=-x+4 \).
		\item Найдите все \( x \), для каждого из которых функции \( y=3x \) и \( y=1-x \) одновременно принимают отрицательные значения.
		\item Найдите все значения \( x \), для каждого из которых значения функции \( y=\dfrac{1}{4}x-\dfrac{1}{2} \) меньше значений функций \( y=x \) и \( y=-2x+3 \).
		\item Даны функции \( y=5x-8 \) и \( y=-5x+8 \). Определите интервал оси \( Ox \), на котором:
		\begin{tasks}(1)
			\task обе функции положительны;
			\task обе функции отрицательны;
			\task функция \( y=5x-8 \) больше нуля, а функция \( y=-5x+8 \) меньше нуля;
			\task значения функции \( y=5x-8 \) больше соответствующих значений функции \( y=-5x+8 \).
		\end{tasks}
		\item Решите двойные неравенства:
		\begin{tasks}(2)
			\task \( 1<x+4<2 \)
			\task \( -7<x-6<-2 \)
		\end{tasks}
	\end{listofex}
\end{class}
%END_FOLD

%BEGIN_FOLD % ====>>_ Домашняя работа 3 _<<====
\begin{homework}[number=3]
	\begin{listofex}
		\item Домашняя работа 3
	\end{listofex}
\end{homework}
%END_FOLD

%BEGIN_FOLD % ====>>_____ Занятие 7 _____<<====
\begin{class}[number=7]
	\title{Подготовка к проверочной}
	\begin{listofex}
		\item Занятие 7
	\end{listofex}
\end{class}
%END_FOLD

%BEGIN_FOLD % ====>>_ Проверочная работа _<<====
\begin{exam}
	\begin{listofex}
		\item Проверочная
	\end{listofex}
\end{exam}
%END_FOLD