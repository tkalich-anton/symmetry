%
%===============>>  ГРУППА 8-2 МОДУЛЬ 6  <<=============
%
\setmodule{6}

%BEGIN_FOLD % ====>>_____ Занятие 1 _____<<====
\begin{class}[number=1]
	\begin{listofex}
		\item \exercise{1462}
		\item Постройте график функции \( y=2x-5 \)
		\begin{tasks}(1)
			\task Проверьте \textit{(графическим, а потом аналитическим способом)}, принадлежит ли точка с координатами \( (4;3) \) графику этой функции?
			\task Найдите точку на графике функции, ордината которой равна \( 217 \).
			\task Найдите точку пересечения графика данной функции с графиком функции \( y=4x-1 \).
			\task Найдите уравнение прямой, которая параллельна исходной и проходит через начало координат.
			\task Найдите уравнение прямой, которая параллельна исходной и проходит точку \( (-1;1) \).
		\end{tasks}
		\item Постройте график функции \(y=x^2\) и найдите его точки пересечения с прямой \( y=2x \) графически, а затем аналитически.
		\item Найдите аналитическим способом точки пересечения графиков функций \(f(x)=x^2+3x-10\)	и \( g(x)=-3x^2-9x-19 \).
		\item Парабола вида \( y=ax^2 \) проходит через точку \( (-1;3) \). Найдите \( a \).
		\item Найдите такие иксы, при которых значении функции \( y=6x^2-x-12 \) равнялось нулю.
	\end{listofex}
\end{class}
%END_FOLD

%BEGIN_FOLD % ====>>_____ Занятие 2 _____<<====
\begin{class}[number=2]
	\begin{listofex}
		\item Занятие 2
	\end{listofex}
\end{class}
%END_FOLD

%BEGIN_FOLD % ====>>_ Домашняя работа 1 _<<====
\begin{homework}[number=1]
	\begin{listofex}
		\item Найдите такие иксы, при которых значении функции \( y=-7x^2+19x+6 \) равнялось нулю.
		\item Постройте график функции \( y=3x+1 \)
		\begin{tasks}(1)
			\task Проверьте \textit{(графическим, а потом аналитическим способом)}, принадлежит ли точка с координатами \( (2;7) \) графику этой функции?
			\task Найдите точку на графике функции, ордината которой равна \( 214 \).
			\task Найдите точку пересечения графика данной функции с графиком функции \( y=7x-2 \).
			\task Найдите уравнение прямой, которая параллельна исходной и проходит через начало координат.
			\task Найдите уравнение прямой, которая параллельна исходной и проходит точку \( (-1;1) \).
		\end{tasks}
		\item \exercise{1397}
	\end{listofex}
\end{homework}
%END_FOLD

%BEGIN_FOLD % ====>>_____ Занятие 3 _____<<====
\begin{class}[number=3]
	\begin{listofex}
		\item Занятие 3 
	\end{listofex}
\end{class}
%END_FOLD

%BEGIN_FOLD % ====>>_____ Занятие 4 _____<<====
\begin{class}[number=4]
	\begin{listofex}
		\item Занятие 4
	\end{listofex}
\end{class}
%END_FOLD

%BEGIN_FOLD % ====>>_ Домашняя работа 2 _<<====
\begin{homework}[number=2]
	\begin{listofex}
		\item Домашняя работа 2
	\end{listofex}
\end{homework}
%END_FOLD

%BEGIN_FOLD % ====>>_____ Занятие 5 _____<<====
\begin{class}[number=5]
	\begin{listofex}
		\item Занятие 5
	\end{listofex}
\end{class}
%END_FOLD

%BEGIN_FOLD % ====>>_____ Занятие 6 _____<<====
\begin{class}[number=6]
	\begin{listofex}
		\item Занятие 6
	\end{listofex}
\end{class}
%END_FOLD

%BEGIN_FOLD % ====>>_ Домашняя работа 3 _<<====
\begin{homework}[number=3]
	\begin{listofex}
		\item Домашняя работа 3
	\end{listofex}
\end{homework}
%END_FOLD

%BEGIN_FOLD % ====>>_____ Занятие 7 _____<<====
\begin{class}[number=7]
	\title{Подготовка к проверочной}
	\begin{listofex}
		\item Занятие 7
	\end{listofex}
\end{class}
%END_FOLD

%BEGIN_FOLD % ====>>_ Проверочная работа _<<====
\begin{exam}
	\begin{listofex}
		\item Проверочная
	\end{listofex}
\end{exam}
%END_FOLD