%
%===============>>  ГРУППА 8-2 МОДУЛЬ 6  <<=============
%
\setmodule{6}

%BEGIN_FOLD % ====>>_____ Занятие 1 _____<<====
\begin{class}[number=1]
	\begin{listofex}
		\item \exercise{1462}
		\item Постройте график функции \( y=2x-5 \)
		\begin{tasks}(1)
			\task Проверьте \textit{(графическим, а затем аналитическим способом)}, принадлежит ли точка с координатами \( (4;3) \) графику этой функции?
			\task Найдите абсциссу точки на графике функции, ордината которой равна \( 217 \).
			\task Найдите координаты точек пересечения графика данной функции с графиком функции \( y=4x-1 \).
			\task Найдите уравнение прямой, которая параллельна исходной и проходит через начало координат.
			\task Найдите уравнение прямой, которая параллельна исходной и проходит точку \( (-1;1) \).
		\end{tasks}
		\item Постройте график функции \(y=x^2\) и найдите координаты точек пересечения с прямой \( y=2x \) графическим, а затем аналитическим способом.
		\item Найдите аналитическим способом точки пересечения графиков функций \(f(x)=x^2+3x-10\)	и \( g(x)=-3x^2-9x-19 \).
		\item Парабола вида \( y=ax^2 \) проходит через точку \( (-1;3) \). Найдите \( a \).
		\item Найдите такие значения переменной \( x \), при которых значение функции \( y=6x^2-x-12 \) равнялось нулю.
		\item Цена на некоторый товар была снижена дважды --- сначала на \( 15\% \), а потом ещё на \( 20\% \). Каков общий процент снижения цены?
		\item Цена на товар в течение месяца упала сначала на \( 40\% \), а потом увеличилась на \( 50\% \) и составила \( 5130  \) рублей. Найдите первоначальную цену товара.
	\end{listofex}
\end{class}
%END_FOLD

%BEGIN_FOLD % ====>>_____ Занятие 2 _____<<====
\begin{class}[number=2]
	\begin{listofex}
		\item Цена на товар в течение месяца упала сначала на \( 40\% \), а потом увеличилась на \( 50\% \) и составила \( 5130  \) рублей. Найдите первоначальную цену товара.
		\item \exercise{1423}
		\item Постройте график функции \( y=-2x^2 \) и найдите координаты точек пересечения с прямой \( y=-3x+1 \) графическим, а затем аналитическим способом.
		\item Найдите аналитическим способом точки пересечения графиков функций \(f(x)=-3x^2+2x+1\)	и \( g(x)=x^2-6x+1 \).
		\item Постройте график функции \( y=x^2 \) и определите:
		\begin{tasks}(1)
			\task Промежутки возрастания функции;
			\task Промежутки убывания функции;
			\task Область определения функции;
			\task Область значения функции.
		\end{tasks}
		\item Постройте график \( y=-\dfrac{1}{5}x^2 \) и определите:
		\begin{tasks}(1)
			\task Промежутки возрастания функции;
			\task Промежутки убывания функции;
			\task Область определения функции;
			\task Область значения функции.
		\end{tasks}
		\item Найдите координаты вершины параболы 
		\begin{tasks}(2)
			\task \( y=6x^2 \)
			\task \( y=2x^2-16x+37 \)
			\task \( y=-6x^2+12x-6 \)
			\task \( y=x^2+4 \)
			\task \( y=(x+3)^2-3 \)
			\task \( y=-\dfrac{1}{2}(x-5)^2+2 \)
		\end{tasks}
		\item Найдите аналитическим способом точки пересечения графиков функций \(f(x)=x^2+3x-18\)	и \( g(x)=-x^2+x+6 \).
	\end{listofex}
\end{class}
%END_FOLD

%BEGIN_FOLD % ====>>_ Домашняя работа 1 _<<====
\begin{homework}[number=1]
	\begin{listofex}
		\item Найдите такие значения переменной \( x \), при которых значении функции \( y=-7x^2+19x+6 \) равнялось нулю.
		\item Постройте график функции \( y=3x+1 \)
		\begin{tasks}(1)
			\task Проверьте \textit{(графическим, а потом аналитическим способом)}, принадлежит ли точка с координатами \( (2;7) \) графику этой функции?
			\task Найдите точку на графике функции, ордината которой равна \( 214 \).
			\task Найдите точку пересечения графика данной функции с графиком функции \( y=7x-2 \).
			\task Найдите уравнение прямой, которая параллельна исходной и проходит через начало координат.
			\task Найдите уравнение прямой, которая параллельна исходной и проходит точку \( (-1;1) \).
		\end{tasks}
		\item Найдите координаты вершины параболы:
		\begin{tasks}(2)
			\task \( y=-x^2+12 \)
			\task \( y=2(x-5)^2+7 \)
			\task \( y=-\dfrac{1}{3}(x+6)^2-2 \)
			\task \( y=\dfrac{1}{2}(x-1)^2+3 \)
		\end{tasks}
		\item \exercise{1422}
	\end{listofex}
\end{homework}
%END_FOLD

%BEGIN_FOLD % ====>>_____ Занятие 3 _____<<====
\begin{class}[number=3]
	\begin{listofex}
		\item Какие координаты имеет вершина параболы:
		\begin{tasks}(2)
			\task \( y=(x+1)^2 \)
			\task \( y=3(x+9)^2 \)
			\task \( y=-2(x-5)^2 \)
			\task \( y=-4(x-9)^2 \)
		\end{tasks}
		\item Запишите уравнения оси симметрии параболы:
		\begin{tasks}(2)
			\task \( y=(x+5)^2 \)
			\task \( y=-2(x-1)^2 \)
		\end{tasks}
		\item Дана парабола \( y=(x-2)^2 \). 
		\begin{tasks}(1)
			\task Определите координаты вершины параболы;
			\task Запишите уравнение оси симметрии параболы;
			\task Определите область значений и определения функции;
			\task Постройте график функции;
			\task Как изменяется \( y \), если аргумент \( x \) изменяется от \( -\infty  \) до \( 2 \)? от \( 2 \) до \( +\infty \)?
			\task При каком значении \( x \) функция принимает наименьшее значение? Принимает ли функция наибольшее значение?
			\task В каких точках график функции пересекает оси координат?
		\end{tasks} 
		\item Постройте график функции:
		\begin{tasks}(2)
			\task \( y=-(x-4)^2+1 \)
			\task \( y=-3(x+1)^2 \)
			\task \( y=0,5(x+3)^2-1,5 \)
		\end{tasks}
		\item Выделите полный квадрат и постройте график функции:
		\begin{tasks}(2)
			\task \( y=x^2-4x+3 \)
			\task \( y=4x^2-4x-1 \)
			\task \( y=-x^2-6x-5 \)
			\task \( y=x^2-4x+7 \)
		\end{tasks}
		\item Постройте график функции: 
		\[y=\dfrac{(x^2+4x+3)(x-2)}{x+1}\]
	\end{listofex}
\end{class}
%END_FOLD

%BEGIN_FOLD % ====>>_____ Занятие 4 _____<<====
\begin{class}[number=4]
	\begin{listofex}
		\item Занятие 4
	\end{listofex}
\end{class}
%END_FOLD

%BEGIN_FOLD % ====>>_ Домашняя работа 2 _<<====
\begin{homework}[number=2]
	\begin{listofex}
		\item Постройте график функции:
		\begin{tasks}(2)
			\task \( y=(x-13)^2 \)
			\task \( y=-2(x+3)^2 \)
			\task \( y=-(x-0,3)^2 \)
			\task \( y=0,5(x+2)^2 \)
		\end{tasks}
		\item Дана функция \( y=2(x-3)^2 \)
		\begin{tasks}(1)
			\task Постройте график функции;
			\task Какова область определения функции?
			\task При каком \( x \) функция принимает наименьшее значение? Принимает ли функция наибольшее значение?
			\task В каких точках график функции пересекает оси координат?
			\task Как изменяется \( y \), если аргумент \( x \) изменяется от \( -\infty  \) до \( -1 \)? от \( 0 \) до \( 1 \)?
		\end{tasks}
	\end{listofex}
\end{homework}
%END_FOLD

%BEGIN_FOLD % ====>>_____ Занятие 5 _____<<====
\begin{class}[number=5]
	\begin{listofex}
		\item Занятие 5
	\end{listofex}
\end{class}
%END_FOLD

%BEGIN_FOLD % ====>>_____ Занятие 6 _____<<====
\begin{class}[number=6]
	\begin{listofex}
		\item Занятие 6
	\end{listofex}
\end{class}
%END_FOLD

%BEGIN_FOLD % ====>>_ Домашняя работа 3 _<<====
\begin{homework}[number=3]
	\begin{listofex}
		\item Домашняя работа 3
	\end{listofex}
\end{homework}
%END_FOLD

%BEGIN_FOLD % ====>>_____ Занятие 7 _____<<====
\begin{class}[number=7]
	\title{Подготовка к проверочной}
	\begin{listofex}
		\item Занятие 7
	\end{listofex}
\end{class}
%END_FOLD

%BEGIN_FOLD % ====>>_ Проверочная работа _<<====
\begin{exam}
	\begin{listofex}
		\item Проверочная
	\end{listofex}
\end{exam}
%END_FOLD