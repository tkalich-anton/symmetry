%
%===============>>  ГРУППА 8-2 МОДУЛЬ 1  <<=============
%
\setmodule{1} 

%BEGIN_FOLD % ====>>_____ Занятие 1 _____<<====
\begin{class}[number=1]
	\begin{listofex}
	\item Упростить дробь:
	\begin{itasks}[3]
		\task \exercise{20}
		\task \exercise{49}
		\task \exercise{54}
		\task \exercise{55}
		\task \exercise{60}
		\task \exercise{61}
	\end{itasks}
	\item Упростить дробь:
	\begin{itasks}[2]
		\task \exercise{63}
		\task \exercise{65}
		\task \exercise{67}
		\task \exercise{70}
	\end{itasks}
	\item Упростить дробь:
	\begin{itasks}[3]
		\task \exercise{73}
		\task \exercise{74}
		\task \exercise{75}
		\task \exercise{78}
		\task \exercise{80}
	\end{itasks}
	\item Упростить дробь:
	\begin{itasks}[2]
		\task \exercise{83}
		\task \exercise{86}
		\task \exercise{1346}
		\task \exercise{93}
		\task \exercise{97}
	\end{itasks}
	\item Упростить дробь:
	\begin{itasks}[3]
		\task \exercise{99}
		\task \exercise{103}
		\task \exercise{107}
	\end{itasks}
	\[ \begin{array}{cccc}
		\text{Разность квадратов}&(a+b)(a-b)& =&a^2-b^2,\\
		\text{Квадрат суммы}&(a+b)^2& =&a^2+2ab+b^2,\\
		\text{Квадрат разности}&(a-b)^2& =&a^2-2ab+b^2,\\
		\text{Сумма кубов}&(a+b)(a^2-ab+b^2)& =&a^3+b^3,\\
		\text{Разность кубов}&(a-b)(a^2+ab+b^2)& =&a^3-b^3,\\
		\text{Куб суммы}&(a+b)^3& =&a^3+3a^2b+3ab^2+b^3,\\
		\text{Куб разности}&(a-b)^3& =&a^3-3a^2b+3ab^2-b^3.\\
	\end{array} \]
	\item Упростить дробь:
	\begin{itasks}[3]
		\task \exercise{109}
		\task \exercise{113}
		\task \exercise{118}
		\task \exercise{121}
		\task \exercise{122}
		\task \exercise{126}
	\end{itasks}
\end{listofex}
\end{class}
%END_FOLD

%BEGIN_FOLD % ====>>_____ Занятие 2 _____<<====
\begin{class}[number=2]
\[ \begin{array}{cccc}
	\text{Разность квадратов}&(a+b)(a-b)& =&a^2-b^2,\\
	\text{Квадрат суммы}&(a+b)^2& =&a^2+2ab+b^2,\\
	\text{Квадрат разности}&(a-b)^2& =&a^2-2ab+b^2,\\
	\text{Сумма кубов}&(a+b)(a^2-ab+b^2)& =&a^3+b^3,\\
	\text{Разность кубов}&(a-b)(a^2+ab+b^2)& =&a^3-b^3,\\
	\text{Куб суммы}&(a+b)^3& =&a^3+3a^2b+3ab^2+b^3,\\
	\text{Куб разности}&(a-b)^3& =&a^3-3a^2b+3ab^2-b^3.\\
\end{array} \]
\begin{listofex}
	\item Упростить дробь:
	\begin{tasks}(2)
		\task \exercise{51}
		\task \exercise{56}
		\task \exercise{57}
		\task \exercise{64}
		\task \exercise{76}
		\task \exercise{77}
	\end{tasks}
	\item Упростить дробь:
	\begin{tasks}(2)
		\task \exercise{118}
		\task \exercise{121}
		\task \exercise{122}
		\task \exercise{126}
		\task \exercise{1351}
		\task \exercise{1353}
		\task \exercise{1371}
		\task \exercise{1374}
	\end{tasks}
	\item Вычислить значение выражения: \( \dfrac{3m^2+6mn+3n^2}{6n^2-6m^2} \), если \( m=0,5 \), \( n=\dfrac{2}{3} \)
	\item Представить в виде несократимой дроби:
	\begin{itasks}[2]
		\task \exercise{130}
		\task \exercise{136}
		\task \exercise{827}
		\task \exercise{829}
		\task \exercise{835}
	\end{itasks}
	\item Представить в виде несократимой дроби:
	\begin{itasks}[2]
		\task \exercise{837}
		\task \exercise{842}
		\task \exercise{844}
	\end{itasks}
	\item Представить в виде несократимой дроби:
	\begin{tasks}(2)
		\task \exercise{839}
		\task \exercise{846}
		\task \exercise{847}
		\task \exercise{850}
		\task \exercise{855}
		\task \exercise{856}
	\end{tasks}
	\item Представить в виде несократимой дроби:
	\begin{tasks}(2)
		\task \exercise{858}
		\task \exercise{862}
		\task \exercise{864}
		\task \exercise{867}
		\task \exercise{871}
	\end{tasks}
	\item Представить в виде несократимой дроби:
	\begin{tasks}(2)
		\task \exercise{874}
		\task \exercise{877}
		\task \exercise{879}
		\task \exercise{881}
		\task \exercise{884}
		\task \exercise{885}
		\task \exercise{888}
		\task \exercise{890}
	\end{tasks}
	\item Представить в виде несократимой дроби:
	\begin{itasks}[2]
		\task \exercise{892}
		\task \exercise{894}
		\task \exercise{896}
	\end{itasks}
	\end{listofex}
\end{class}
%END_FOLD

%BEGIN_FOLD % ====>>_ Домашняя работа 1 _<<====
\begin{homework}[number=1]
	\begin{listofex}
	\item Упростить дробь:
	\begin{tasks}(2)
		\task \exercise{53}
		\task \exercise{59}
		\task \exercise{66}
		\task \exercise{69}
		\task \exercise{79}
		\task \exercise{81}
	\end{tasks}
	\item \exercise{1223}
	\item Упростить дробь:
	\begin{itasks}[2]
		\task \exercise{1352}
		\task \exercise{1372}
		\task \exercise{1373}
		\task \exercise{1375}
		\task \exercise{1376}
		\task \exercise{1377}
	\end{itasks}
	\item Представить в виде несократимой дроби:
	\begin{itasks}[2]
		\task \exercise{826}
		\task \exercise{832}
		\task! \exercise{836}
	\end{itasks}
	\end{listofex}
\end{homework}
%END_FOLD

%BEGIN_FOLD % ====>>_____ Занятие 3 _____<<====
\begin{class}[number=3]
	\begin{listofex}
	\item Произвести сложение или вычитание и представить в виде несократимой дроби:
	\begin{itasks}[2]
		\task \exercise{858}
		\task \exercise{862}
		\task \exercise{866}
		\task \exercise{870}
		\task \exercise{871}
	\end{itasks}
	\item Представить в виде несократимой дроби:
	\begin{itasks}[2]
		\task \exercise{874}
		\task \exercise{877}
		\task \exercise{880}
		\task \exercise{883}
		\task \exercise{889}
		\task \exercise{890}
		\task \exercise{892}
		\task \exercise{895}
	\end{itasks}
	\item Представить в виде несократимой дроби:
	\begin{itasks}[2]
		\task \exercise{898}
		\task \exercise{901}
		\task \exercise{904}
		\task \exercise{907}
		\task \exercise{910}
	\end{itasks}
	\item Представить в виде несократимой дроби:
	\begin{itasks}[2]
		\task \exercise{914}
		\task \exercise{918}
		\task \exercise{922}
		\task \exercise{926}
	\end{itasks}
	\item Представить в виде несократимой дроби:
	\begin{itasks}[2]
		\task \exercise{934}
		\task \exercise{936}
		\task \exercise{939}
		\task \exercise{942}
		\task \exercise{1355}
		\task \exercise{1356}
		\task! \exercise{1360}
		\task \exercise{1364}
		\task \exercise{1365}
	\end{itasks}
	\end{listofex}
\end{class}
%END_FOLD

%BEGIN_FOLD % ====>>_____ Занятие 4 _____<<====
\begin{class}[number=4]
	\begin{listofex}
	\item Произвести сложение или вычитание и представить в виде несократимой дроби:
	\begin{tasks}(4)
		\task \exercise{859}
		\task \exercise{863}
		\task \exercise{867}
		\task \exercise{872}
	\end{tasks}
	\item Представить в виде несократимой дроби:
	\begin{itasks}[3]
		\task \exercise{875}
		\task \exercise{881}
		\task! \exercise{878}
		\task \exercise{884}
		\task \exercise{893}
		\task \exercise{896}
	\end{itasks}
	\item Представить в виде несократимой дроби:
	\begin{itasks}[3]
		\task \exercise{899}
		\task \exercise{902}
		\task \exercise{905}
		\task \exercise{908}
		\task \exercise{911}
	\end{itasks}
	\item Представить в виде несократимой дроби:
	\begin{itasks}[2]
		\task \exercise{915}
		\task \exercise{919}
		\task \exercise{923}
		\task \exercise{927}
	\end{itasks}
	\item Представить в виде несократимой дроби:
	\begin{itasks}[2]
		\task \exercise{935}
		\task \exercise{938}
		\task! \exercise{941}
		\task \exercise{1386}
		\task \exercise{1361}
		\task \exercise{1364}
		\task! \exercise{1368}
		\task \exercise{1356}
		\task! \exercise{1360}
		\task \exercise{1365}
	\end{itasks}
	\end{listofex}
\end{class}
%END_FOLD

%BEGIN_FOLD % ====>>_ Домашняя работа 2 _<<====
\begin{homework}[number=2]
	\begin{listofex}
	\item Произвести сложение или вычитание и представить в виде несократимой дроби:
	\begin{itasks}[2]
		\task \exercise{860}
		\task \exercise{864}
		\task \exercise{868}
		\task \exercise{873}
	\end{itasks}
	\item Представить в виде несократимой дроби:
	\begin{itasks}[2]
		\task \exercise{876}
		\task \exercise{879}
		\task \exercise{882}
		\task \exercise{885}
		\task \exercise{891}
		\task \exercise{894}
		\task \exercise{897}
	\end{itasks}
	\item Представить в виде несократимой дроби:
	\begin{itasks}[2]
		\task \exercise{900}
		\task \exercise{903}
		\task \exercise{906}
		\task \exercise{909}
		\task \exercise{912}
	\end{itasks}
	\item Представить в виде несократимой дроби:
	\begin{itasks}[2]
		\task \exercise{916}
		\task \exercise{920}
		\task \exercise{924}
		\task \exercise{929}
	\end{itasks}
	\item Представить в виде несократимой дроби:
	\begin{itasks}[2]
		\task \exercise{937}
		\task \exercise{940}
		\task \exercise{939}
		\task \exercise{1354}
		\task \exercise{1381}
		\task \exercise{1362}
		\task \exercise{1363}
	\end{itasks}
	\end{listofex}
\end{homework}
%END_FOLD

%BEGIN_FOLD % ====>>_____ Занятие 5 _____<<====
\begin{class}[number=5]
	\begin{listofex}
	\item Упростить выражение до несократимой дроби:
	\begin{tasks}(2)
		\task \( \dfrac{a+1}{7x}\cdot\dfrac{2x}{a+1} \)
		\task \( \dfrac{ax-ay}{ac}\cdot\dfrac{cx+cy}{x-y} \)
		\task \( \dfrac{4a}{a^2b}:\dfrac{5ab}{3a-3b} \)
		\task \( \dfrac{a^2-b^2}{2a^2b}\cdot\dfrac{4ab^2}{a+b} \)
	\end{tasks}
	\item Упростить выражение:
	\begin{tasks}(2)
		\task \( \dfrac{(x-y)^2}{3x^2y^2}:\dfrac{x-y}{6xy^2} \)
		\task \( \dfrac{a^2-9b^2}{a^2-ab}:\dfrac{a^2+3ab}{a-b} \)
		\task \( \dfrac{m^3+n^3}{2m}\cdot\dfrac{4mn}{m^2-mn+n^2} \)
		\task \( \dfrac{12a^2+6ab}{8a^3-b^3}\cdot\dfrac{4a^2+2ab+b^2}{3a^2-6ab} \)
	\end{tasks}
	\item Упростить выражение:
	\begin{tasks}(2)
		\task \( \left( \dfrac{1}{a}+\dfrac{1}{b}+\dfrac{1}{c} \right)\cdot abc \)
		\task \( \left( \dfrac{a+x}{a}-\dfrac{2x}{x-a} \right):\dfrac{a^2+x^2}{x-a} \)
		\task \( \left( \dfrac{a}{a-1}+1 \right):\left( 1-\dfrac{a}{a-1} \right) \)
		\task \( \dfrac{3}{5x}-\dfrac{3}{x+y}\cdot\left( \dfrac{x+y}{5x}-x-y \right) \)
	\end{tasks}
	\item Упростить выражение:
	\begin{tasks}(2)
		\task \( \left( a^2-\dfrac{1}{b^2} \right):\left( a-\dfrac{1}{b} \right) \)
		\task \( \dfrac{x+y}{x}-\dfrac{x}{x-y}+\dfrac{y^2}{x^2-xy} \)
		\task \( \dfrac{a-1}{2a}\cdot\left( \dfrac{a+3}{a+1}-\dfrac{a^2-5}{a^2-1} \right) \)
		\task \( \dfrac{4y}{y-1}\cdot\left( \dfrac{y}{8}-\dfrac{1}{4}+\dfrac{1}{8y} \right) \)
	\end{tasks}
	\item Упростить выражение \( \left( \dfrac{a^2}{a+1}-\dfrac{a^3}{a^2+2a+1} \right):\left( \dfrac{a}{a+1}-\dfrac{a^2}{a^2-1} \right) \) и найти значение выражения при \( a=-3 \).
	\item Докажите тождество:
	\begin{tasks}
		\task \( \left( \dfrac{1}{x-1}-\dfrac{1}{x+1} \right)\cdot(x^2-2x+1)=\dfrac{2x-2}{x+1} \)
		\task \( \dfrac{2x}{x^2-y^2}-\dfrac{1}{x-y}-\dfrac{1}{x+y}=0 \)
	\end{tasks}
	\end{listofex}
\end{class}
%END_FOLD

%BEGIN_FOLD % ====>>_____ Занятие 6 _____<<====
\begin{class}[number=6]
	\begin{listofex}
	\item Упростить выражение до несократимой дроби:
	\begin{tasks}(2)
		\task \( \dfrac{x+3}{4x^2}\cdot\dfrac{6x^3}{x+3} \)
		\task \( \dfrac{m-3n}{6m}\cdot\dfrac{3mn}{4m-12n} \)
		\task \( \dfrac{a^2x-a^2y}{a^3c^3}\cdot\dfrac{c^3x+c^3y}{x-y} \)
		\task \( \dfrac{x+y}{10a}:\dfrac{x+y}{16a^2b} \)
	\end{tasks}
	\item Упростить выражение:
	\begin{tasks}(2)
		\task \( \dfrac{2a-4}{b+1}:\dfrac{a^2-4}{(b+1)^2} \)
		\task \( \dfrac{p^2-q^2}{p^2}\cdot\dfrac{pq+q^2}{(p+q)^2} \)
		\task \( \dfrac{m^2-n^2}{(m+n)^2}:\dfrac{4m-4n}{3m+3n} \)
		\task \( \dfrac{m^3-n^3}{m^3+n^3}:\dfrac{(m-n)^2}{m^2-n^2} \)
		\task \( \dfrac{x^2+xy}{6x^2-6y^2}\cdot\dfrac{3x^3+3y^3}{x^2-xy} \)
	\end{tasks}
	\item Упростить выражение:
	\begin{tasks}(2)
		\task \( \left( \dfrac{a}{b}+\dfrac{b}{c}+\dfrac{c}{a} \right)\cdot\dfrac{ab}{c} \)
		\task \( \left( \dfrac{a+x}{a}-\dfrac{x-y}{x} \right)\cdot\dfrac{a^2}{x^2+ay} \)
		\task \( \left( a+\dfrac{a^2}{c} \right):\left( b+\dfrac{bc}{a} \right) \)
		\task \( \left( \dfrac{x^2+1}{2x-1}-\dfrac{x}{2} \right)\cdot\dfrac{1-2x}{x+2} \)
	\end{tasks}
	\item Упростить выражение:
	\begin{tasks}(2)
		\task \( \left( 4x^2-\dfrac{1}{9b^2} \right):\left( 2x-\dfrac{1}{3b} \right) \)
		\task \( \dfrac{1}{m+2}+\dfrac{1}{m-2}-\dfrac{4}{m^2-4} \)
		\task! \( \left( \dfrac{c-d}{c^2+cd}-\dfrac{c}{d^2+cd} \right):\left( \dfrac{d^2}{c^3-cd^2}+\dfrac{1}{c+d} \right) \)
		\task \( \left( \dfrac{14+a^2}{a^2-4}-\dfrac{a-4}{a+2} \right)\cdot\dfrac{a-2}{6} \)
		\task \( \left( \dfrac{a}{a-4}-\dfrac{a-4}{a+4} \right)\cdot\dfrac{a+4}{4} \)
		\task \( \left( \dfrac{a}{8}+\dfrac{1}{3}+\dfrac{1}{6a} \right):\dfrac{a+1}{12a} \)
	\end{tasks}
	\end{listofex}
\end{class}
%END_FOLD

%BEGIN_FOLD % ====>>_ Домашняя работа 3 _<<====
\begin{homework}[number=3]
	\begin{listofex}
	\item Упростить выражение до несократимой дроби:
	\begin{tasks}
		\task \( \dfrac{x+y}{x-y}\cdot\dfrac{x^2-xy}{2x^2-2y^2} \)
	\end{tasks}
	\item Упростить выражение:
	\begin{tasks}(2)
		\task \( \dfrac{16-m^2}{m^2-3m}:\dfrac{m^2+4m}{m^2-9} \)
		\task \( \dfrac{3x^2-3y^2}{x^2+xy}\cdot\dfrac{x+y}{6x-6y} \)
		\task \( \dfrac{2a}{a^3-b^3}:\dfrac{6ab}{a^2-b^2} \)
		\task \( \dfrac{p^2-4a^2}{(p+2q)^2}:\dfrac{p^3-8q^3}{4q^2+2pq+p^2} \)
	\end{tasks}
	\item Упростить выражение:
	\begin{tasks}(2)
		\task \( 5x^2\cdot\left( \dfrac{1}{x^2}-\dfrac{1}{x}+3 \right) \)
		\task \( \left( m-\dfrac{1}{1+m} \right)\cdot\dfrac{m+1}{1-m-m^2} \)
		\task! \( \left( \dfrac{n}{n+x}-\dfrac{n}{n-x} \right):\left( \dfrac{n}{n-x}+\dfrac{n}{n+x} \right) \)
	\end{tasks}
	\item Упростить выражение:
	\begin{tasks}(2)
		\task \( \left( \dfrac{3a^2}{4b^2}-\dfrac{b^2}{3} \right):\left( \dfrac{3a}{2b}+b \right) \)
		\task! \( \dfrac{3x^2+3xy}{4xy+6ay}\cdot\left( \dfrac{x}{ax+ay}+\dfrac{3}{2x+2y} \right) \)
		\task \( \left( \dfrac{c+3}{c-3}-\dfrac{c}{c+3} \right)\cdot\dfrac{c-3}{c+1} \)
		\task \( \left( \dfrac{1+a}{1-a}-\dfrac{1-a}{1+a} \right):\dfrac{2a}{1-a} \)
	\end{tasks}
	\item Упростить выражение \( \dfrac{4xy}{y^2-x^2}:\left( \dfrac{1}{y^2-x^2}+\dfrac{1}{x^2+2xy+y^2} \right) \) и найти значение выражения при \( x=0,35 \) и \( y=7,65 \).
	\item Докажите тождество:
	\begin{tasks} 
		\task \( \left( \dfrac{1}{x-y}+\dfrac{1}{x+y} \right)\cdot(x^2-y^2)=2x \)
		\task \( \dfrac{1}{(a-b)(a-c)}+\dfrac{1}{(b-a)(b-c)}+\dfrac{1}{(c-a)(c-b)}=0 \)
	\end{tasks}
	\end{listofex}
\end{homework}
%END_FOLD

%BEGIN_FOLD % ====>>_____ Занятие 7 _____<<====
\begin{class}[number=7]
	\begin{listofex}
	\item Упростить выражение:
	\begin{itasks}[1]
		\task \exercise{1507}
		\task \exercise{1503}
	\end{itasks}
	\item Упростить выражение \( \left( \dfrac{n-1}{n+1}-\dfrac{n+1}{n-1} \right)\cdot\left( \dfrac{1}{2}-\dfrac{n}{4}-\dfrac{1}{4n} \right) \) и найти значение выражения при \( n=3 \). \answer{ \( \dfrac{n-1}{n+1};\;0,5 \) }
	\item Упростить выражение \( \dfrac{x^2+25}{(x-5)^3}+\dfrac{10x}{(5-x)^3} \) и найти значение выражения при \( x=5,125 \). \answer{ \( \dfrac{1}{x-5};\;8 \) }
	\item Докажите тождество:
	\begin{tasks} 
		\task \( \left( \dfrac{1}{x-2}-\dfrac{1}{x+2} \right)\cdot(x^2-4x+4)=\dfrac{4x-8}{x+2} \)
		\task \( \dfrac{1}{(a-b)(b-c)}+\dfrac{1}{(b-c)(c-a)}+\dfrac{1}{(a-c)(b-a)}=0 \)
	\end{tasks}
	\item Упростить выражение:
	\begin{tasks} 
		\task \( \left( \dfrac{a-b}{2a-b}-\dfrac{a^2+b^2+a}{2a^2+ab-b^2} \right):\dfrac{4b^4+4ab^2+a^2}{2b^2+a}{\cdot(b^2+b+ab+a)} \) \answer{ \( \dfrac{b+1}{b-2a} \) }
		\task \( \left( \left( \dfrac{x^2}{y^3}+\dfrac{1}{x} \right):\left( \dfrac{x}{y^2}-\dfrac{1}{y}+\dfrac{1}{x} \right) \right):\dfrac{(x-y)^2+4xy}{1+\dfrac{y}{x}} \) \answer{ \( \dfrac{1}{xy} \) }
	\end{tasks}
	\end{listofex}
\end{class}
%END_FOLD

%BEGIN_FOLD % ====>>_ Проверочная работа _<<====
\begin{exam}
	\begin{listofex}
	\item Упростить дробь:
	\begin{itasks}[3]
		\task \exercise{49}
		\task \exercise{61}
		\task \exercise{1371}
	\end{itasks}
	\item Упростить дробь:
	\begin{itasks}[3]
		\task \exercise{75}
		\task \exercise{78}
		\task \exercise{97}
		\task \exercise{126}
		\task! \exercise{1353}
	\end{itasks}
	\item \exercise{642}
	\item Представить в виде несократимой дроби:
	\begin{itasks}[3]
		\task \exercise{835}
		\task \exercise{907}
		\task \exercise{1364}
	\end{itasks}
	\item Упростить выражение:
	\begin{tasks}(2)
		\task \( \left( \dfrac{1}{a}+\dfrac{1}{b}+\dfrac{1}{c} \right)\cdot abc \)
		\task \( \dfrac{3}{5x}-\dfrac{3}{x+y}\cdot\left( \dfrac{x+y}{5x}-x-y \right) \)
		\task \( \left( 4x^2-\dfrac{1}{9b^2} \right):\left( 2x-\dfrac{1}{3b} \right) \)
		\task \( \left( \dfrac{a}{a-4}-\dfrac{a-4}{a+4} \right)\cdot\dfrac{a+4}{4} \)
	\end{tasks}
	\item Упростить выражение \( \left( \dfrac{a^2}{a+1}-\dfrac{a^3}{a^2+2a+1} \right):\left( \dfrac{a}{a+1}-\dfrac{a^2}{a^2-1} \right) \) и найти значение выражения при \( a=-3 \).
	\end{listofex}
\end{exam}
%END_FOLD