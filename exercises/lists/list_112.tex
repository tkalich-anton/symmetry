%11 класс Предподготовка Занятие №1
\begin{listofex}
	\item Найдите корень уравнения \( \log_2(4-x)=7 \)
	\item В чемпионате по гимнастике участвуют 20 спортсменок: 8 из России, 7 из
	США, остальные - из Китая. Порядок, в котором выступают гимнастки,
	определяется :жребием. Найдите вероятность того, что спортсменка,
	выступающая первой, окажется из Китая.
	\item В четырехугольник \( ABCD \) с периметром \( 54 \)
	вписана окружность, \( АВ = 18 \). Найдите
	сторону \( DC \) четырехугольника.
	\item Найдите значение выражения\quad\( \dfrac{2^{4,4}\cdot6^{7,4}}{12^{6,4}} \)
	\item Через среднюю линию основания треугольной
	призмы, объем которой равен \( 32 \), проведена
	плоскость, параллельная боковому ребру.
	Найдите объем отсеченной треугольной
	призмы.
	\item \exercise{1854}
	\item В розетку электросети подключены приборы, общее
	сопротивление которых составляет \( R_1=90 \) Ом. Параллельно с
	ними в розетку предполагается подключить электрообогреватель.
	Определите наименьшее возможное сопротивление \( R_2 \) этого
	электрообогревателя, если известно, что при параллельном
	соединении двух проводников с сопротивлениями \( R_1 \) Ом и \( R_2 \) Ом
	их общее сопротивление дается формулой \( R_{\text{\textit{общ}}}=\dfrac{R_1\cdot R_2}{R_1 + R_2} \), а
	для нормального функционирования электросети общее
	сопротивление в ней должно быть не меньше \( 9 \) Ом. Ответ выразите в Омах.
	\item Имеется два сплава. Первый сплав содержит \( 10\% \) никеля,
	второй --- \( 30\% \) никеля. Из этих двух сплавов получили третий
	сплав массой \( 200 \) кг, содержащий \( 25\% \) никеля. На сколько
	килограммов масса первого сплава меньше массы второго?
	\item \exercise{1855}
	\item Биатлонист 4 раза стреляет по мишеням. Вероятность
	попадания в мишень при одном выстреле равна 0,6. Найдите
	вероятность того, что биатлонист первые два раза попал в мишени,
	а последние два раза промахнулся.
	\item Найдите точку максимума функции \( y=7+6x-2x\sqrt{x} \).
	\item а) Решите уравнение \( 4^{\sin x}+4^{\sin (\pi+x)}=\dfrac{5}{2} \)\\
	б) Найдите все корни этого уравнения, принадлежащие
	промежутку \( \left[\dfrac{5\pi}{2};4\pi\right] \)
	\item Вне плоскости равностороннего треугольника \( АВС \)
	отмечена точка \( D \), причем \\ \( \cos\angle DAB = \cos\angle DAC = 0,2 \).\\
	a) Докажите, что прямые \( AD \) и \( BC \) перпендикулярны.
	б) Найдите расстояние между прямыми \( AD \) и \( BC \), если известно, что \( AB=2 \).
	\item Решить неравенство \[ \dfrac{\log_4(64x)-2}{\log^2_4 x-\log_4 x^3}\ge-1 \]
	\item 15-го декабря планируется взять кредит в банке на сумму
	1100 тысяч рублей на 31 месяц. Условия его возврата таковы:\\
	--- 1-го числа каждого месяца долг возрастает на 2\% по
	сравнению с концом предыдущего месяца;\\
	--- со 2-го по 14-е число каждого месяца необходимо
	выплатить часть долга;\\
	--- 15-го числа каждого месяца с 1-го по 30-й долг должен
	быть на одну и ту же сумму меньше долга на 15-е число
	предыдущего месяца;\\
	--- к 15-му числу 31-го месяца кредит должен быть
	полностью погашен. Какой долг будет 15-го числа 30-го месяца,
	если общая сумма выплат после полного погашения кредита
	составит 1503 тысячи рублей?
	\item В треугольник \( ABC \) вписана окружность, которая
	касается \( AB \) в точке \( P \). Точка \( M \) – середина стороны \( AB \).\\
	a) Докажите, что \( MP=\dfrac{|BA-AC|}{2} \)\\
	б) Найдите углы треугольника \( ABC \), если известно, что
	длина отрезка \( МР \) равна половине радиуса вписанной в
	треугольник \( ABC \) окружности, \( BC > AC \), а отрезки \( MC \) и \( MA \)
	равны.
	\item Найдите все значения параметра \( a \), при каждом из которых имеет единственное решение система уравнений
	\[ \left\{
	\begin{array}{l}
		\dfrac{xy^2-2xy-4y+8}{\sqrt{4-y}}=0,\\
		y=ax.
	\end{array}
	\right. \]
	имеет ровно три различных решения.
\end{listofex}