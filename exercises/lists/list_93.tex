%11 класс Проифль Моудль 2 Занятие №1
\begin{listofex}
	\item Вычислить значения синуса и косинуса \( 30\degree \), \( 45\degree \), \( 60\degree \).
	\item Вычислить значения тангенса и котангенса с теми же самыми аргументами.
	\item Доказать следующие факты:\\
		ОТТ: \( \sin^2x+\cos^2x=1 \); \( \tg x = \dfrac{\sin x}{\cos x} \); \( \ctg x = \dfrac{\cos x}{\sin x} \) и \( \tg x \cdot \ctg x = 1 \)\\[1em]
	\fbox{%
	\begin{minipage}[t]{0,9\textwidth}
		\textit{Расширенное понятие синуса и косинуса.}\vspace{1em}
		
		\textbf{Косинус угла \boldmath\( {\alpha} \)} --- абсцисса точки на единичной окружности, соответствующей углу \( \alpha \).
		
		\textbf{Синус угла \boldmath\( \alpha \)} --- ордината точки на единичной окружности, соответствующей углу \( \alpha \).
	\end{minipage}
	}
	\item \exercise{2807}
	\item \fbox{%
		\parbox[t]{0.9\textwidth}{
			\textit{Доказать следующие формулы:}\vspace{1em}\\
			\begin{minipage}[t]{0,3\textwidth}
				\( \sin(x+360\degree\cdot n) = \sin x \)
				
				\( \cos(x+360\degree\cdot n) = \cos x \)
			\end{minipage}
			\begin{minipage}[t]{0,3\textwidth}
				
				\( \tg(x+360\degree\cdot n) = \tg x \)
				
				\( \ctg(x+360\degree\cdot n) = \ctg x \)
				
				\vspace{1em}
			\end{minipage}
			\begin{minipage}[t]{0,3\textwidth}
				
				\( \sin(-x) = -\sin x \)
				
				\( \cos(-x) = \cos x \)
			\end{minipage}
			\begin{minipage}[t]{0,3\textwidth}
				\( \sin(180 - x) = \sin x \)
				
				\( \cos(180 - x) = -\cos x \)
			\end{minipage}
			\begin{minipage}[t]{0,3\textwidth}
				
				\( \sin(180+x) = -\sin x \)
				
				\( \cos(180+x) = -\cos x \)
			\end{minipage}
		}
	}
	\item Вычислить:
	\begin{enumcols}[itemcolumns=1]
		\item \exercise{2808}
		%\item \exercise{2809}
		\item \exercise{2810}
		\item \exercise{2811}
	\end{enumcols}
	\item Вычислить:
	\begin{enumcols}[itemcolumns=2]
		\item \exercise{1791}
		\item \exercise{1792}
		\item \exercise{1795}
		\item \exercise{1798}
	\end{enumcols}
	\item Вычислить:
	\begin{enumcols}[itemcolumns=2]
		\item \exercise{1800}
		\item \exercise{1801}
		\item \exercise{1804}
	\end{enumcols}
	%\item Доказать тождество:
	%\begin{enumcols}[itemcolumns=2]
	%	\item \exercise{2899}
	%	\item \exercise{2900}
	%	\item \exercise{2902}
	%	\item \exercise{2860}
	%	\item \exercise{2861}
	%	\item \exercise{2885}
	%\end{enumcols}
	\item \exercise{2856}
	%\item \exercise{2883}
	%\item \exercise{2890}
\end{listofex}