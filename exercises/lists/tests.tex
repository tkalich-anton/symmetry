%Группа 5-1 Модуль 1 Занятие №1
\title{11 класс}
\begin{listofex}
	\item Вычислить:
	\begin{enumcols}[itemcolumns=3]
		\item \exercise{1287}
		\item \exercise{1306}
		\item \exercise{1610}
		\item \exercise{1614}
	\end{enumcols}
	\item Вычислить:
	\begin{enumcols}[itemcolumns=3]
		\item \exercise{1720}
		\item \exercise{1708}
		\item \exercise{1734}
		\item \exercise{1738}
		\item \exercise{1773}
	\end{enumcols}
	\item Вычислить:
	\begin{enumcols}[itemcolumns=3]
		\item \exercise{1109}
		\item \exercise{562}
		\item \exercise{580}
		\item \exercise{585}
		\item \exercise{1596}
	\end{enumcols}
	\item Вычислить:
	\begin{enumcols}[itemcolumns=3]
		\item \exercise{2816}
		\item \exercise{2851}
		\item \exercise{2969}
	\end{enumcols}
	\item Вычислить:
	\begin{enumcols}[itemcolumns=2]
		\item \exercise{1489}
		\item \exercise{1748}
	\end{enumcols}
	\item Решить уравнение:
	\begin{enumcols}[itemcolumns=4]
		\item \exercise{497}
		\item \exercise{31}
		\item \exercise{1163}
		\item \exercise{673}
	\end{enumcols}
	\item В прямоугольном треугольник один из катетов равен \( 4 \), а гипотенуза равна \( 5 \). Чему равен второй катет?
	\item В равнобедренном треугольнике боковая сторона равна \( 13 \), а основание --- \( 24 \). Найдите площадь этого треугольника.
	\item Найдите объем треугольной пирамиды, у которой площадь основания равна \( 20 \) и высота равна \( 15 \).
\end{listofex}
\newpage
\title{9 класс}
\begin{listofex}
	\item Вычислить:
	\begin{enumcols}[itemcolumns=3]
		\item \exercise{1287}
		\item \exercise{1610}
		\item \exercise{1306}
		\item \exercise{1614}
	\end{enumcols}
	\item Вычислить:
	\begin{enumcols}[itemcolumns=3]
		\item \exercise{1720}
		\item \exercise{1708}
		\item \exercise{1734}
		\item \exercise{1738}
		\item \exercise{1773}
	\end{enumcols}
	\item Вычислить:
	\begin{enumcols}[itemcolumns=2]
		\item \exercise{1489}
		\item \exercise{1748}
	\end{enumcols}
	\item Решить уравнение:
	\begin{enumcols}[itemcolumns=3]
		\item \exercise{293}
		\item \exercise{451}
		\item \exercise{190}
	\end{enumcols}
	\item В треугольнике \( ABC \) углы \( A \) и \( C \) равны \( 35 \) и \( 65 \) соответственно. Найдите внешний гол при вершине \( B \).
	\item В прямоугольном треугольник один из катетов равен \( 4 \), а гипотенуза равна \( 5 \). Чему равен второй катет?
	\item В равнобедренном треугольнике боковая сторона равна \( 13 \), а основание --- \( 24 \). Найдите площадь этого треугольника.
\end{listofex}
\newpage
\title{8 класс}
\begin{listofex}
	\item Сократить дробь:
	\begin{enumcols}[itemcolumns=2]
		\item \( \dfrac{14}{35} \)
		\item \( \dfrac{36}{60} \)
	\end{enumcols}
	\item Перевести обыкновенную дробь в десятичную:
	\begin{enumcols}[itemcolumns=3]
		\item \( \dfrac{12}{100} \)
		\item \( \dfrac{37}{10} \)
		\item \( \dfrac{3}{12} \)
	\end{enumcols}
	\item Перевести смешанное число в дробь или наоборот:
	\begin{enumcols}[itemcolumns=2]
		\item \( 4\dfrac{3}{7} \)
		\item \( \dfrac{78}{5} \)
	\end{enumcols}
	\item Вычислить:
	\begin{enumcols}[itemcolumns=3]
		\item \( \dfrac{3}{15}+\dfrac{9}{15} \)
		\item \( \dfrac{2}{26}+\dfrac{3}{39} \)
		\item \( \dfrac{7}{12}-\dfrac{1}{3} \)
	\end{enumcols}
	\item Вычислить:
	\begin{enumcols}[itemcolumns=3]
		\item \( -2+17 \)
		\item \( -5+(-2)\cdot3 \)
		\item \( (-16):4+(-6)\cdot\left( -\dfrac{1}{2} \right) \)
	\end{enumcols}
	\item Вычислить:
	\begin{enumcols}[itemcolumns=3]
		\item \exercise{1610}
		\item \( \dfrac{3}{2}\cdot\dfrac{5}{6}+\dfrac{3}{2}:\dfrac{9}{10}-\dfrac{3}{2}\cdot\dfrac{13}{18} \)
	\end{enumcols}
	\item Применить формулы сокращенного умножения:
	\begin{enumcols}[itemcolumns=3]
		\item \( (3x+1)^2 \)
		\item \( (0,5x-2)^2 \)
		\item \( 4x^2-9 \)
	\end{enumcols}
	\item Произведите умножение:
	\begin{enumcols}[itemcolumns=3]
		\item \( 3x(2x-1) \)
		\item \( 2x^2y\left( \dfrac{1}{2}x^4-4xy^3 \right) \)
		\item \( (3x^3-5)(2x^2-x) \)
	\end{enumcols}
	\item \( 2x^3-2(x^3-2x^2)+3x^2 \) при \( x=-2 \)
	\item Решить уравнение: \( 2x+3(7x-12)=5x+72 \)
	\item Какие виды треугольников бывают?
	\item Перечислите названия углов при двух параллельных и секущей. Какими свойствами они обладают?
	\item Назовите признаки равенства треугольников.
	\item Какой угол называют смежным? Какой угол называют внешним углом треугольника?
	\item Один из смежных углов на \( 30\degree \) больше другого. Найдите эти углы.
	\item Что такое биссектриса, медиана и высота в треугольнике?
\end{listofex}