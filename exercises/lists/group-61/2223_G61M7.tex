%
%===============>>  ГРУППА 6-1 МОДУЛЬ 7  <<=============
%
\setmodule{7}

%BEGIN_FOLD % ====>>_____ Занятие 1 _____<<====
\begin{class}[number=1]
	\begin{definit}
		Обе части уравнения можно умножить или разделить на одно и то же число, не равное нулю. Слагаемое можно перенести из одной части уравнения в другую, изменив его знак на противоположный.
	\end{definit}
	\begin{listofex}
		\item Решите уравнения: % 5.21(a-и) и 5.24 a в
		\begin{tasks}(2)
			\task \( x+0,9=1,5 \)
			\task \( 2-x=0,3 \)
			\task \( x-3,8=1,4 \)
			\task \( 0,8-x=1,3 \)
			\task \( 2-x=0,3 \)
			\task \( -0,5+x=3,4 \)
			\task \( x-3,6=-5 \)
			\task \( 9-7y=25-3y \)
			\task \( 4x=0 \)
			\task \( -2y=0 \)
			\task \( -40 \cdot (-7+5)=-1600 \)
			\task \( -2,1 \cdot (4-6y)=-42 \)
		\end{tasks}
		\item Решите уравнение, используя основное свойство пропорции:
		\begin{tasks}(2)
			\task \( \dfrac{-3}{9-4a} = \dfrac{40}{200} \)
			\task \( \dfrac{x-3}{6} = \dfrac{7}{3} \)
			\task \( \dfrac{1-2b}{4} = \dfrac{0,8}{0,5} \)
			\task \( \dfrac{5}{2x+3} = \dfrac{2,5}{4,5} \)
		\end{tasks}
		\item Решите уравнения: % 5,28(а-л) 
		\begin{tasks}(2)
			\task \( 11x=36-x \)
			\task \( 9x+4=48-2x \)
			\task \( 8-4x=2x-16 \)
			\task \( 0,4x+3,8=2,6-0,8x \)
			\task \( 6,8-1,3x=0,6x-2,7 \)
			\task \( \dfrac{4}{9}x+14=\dfrac{1}{6}x+9 \)
			\task \( 6x=28-x \)
			\task \( 9x-26=30-5x \)
			\task \( 7-3x=6x-56 \)
			\task \( 0,9x-7,4=-0,4x+4,3 \)
			\task \( 5,8-1,6x=0,3x-1,8 \)
			\task \( \dfrac{3}{8}x+19=\dfrac{7}{12}x+24 \)
		\end{tasks}
		\item Решите уравнение: % 5.30 ц
		\begin{tasks}
			\task \( \dfrac{1}{3} \left( \dfrac{1}{2}z+\mfrac{19}{1}{2} \right) = \left(  \mfrac{2}{7}{9}+\mfrac{3}{1}{3} \right) \cdot \dfrac{3}{5} \)
		\end{tasks}
	\end{listofex}
\end{class}
%END_FOLD

%BEGIN_FOLD % ====>>_____ Занятие 2 _____<<====
\begin{class}[number=2]
	\begin{listofex}
		\item Решите уравнения: %5.30 5.32(о ф ь)
		\begin{tasks}(2)
			\task \( 4x-x=24 \)
			\task \( 8x-8=20-6x \)
			\task \( 9-4x=3x-40 \)
			\task \( 0,6x-5,4=-0,8+5,8 \)
			\task \( 4,7-1,1x=0,5x-3,3 \)
			\task \( \dfrac{5x}{6}+16=\dfrac{4}{9}x+9 \)
			\task \( \dfrac{x}{2}+\dfrac{x}{3}+\mfrac{1}{5}{6}x=2 \)
			\task \( \dfrac{5x}{7}+\dfrac{x}{35}+x=-1 \)
			\task \( \dfrac{x}{2}+\dfrac{x}{3}+\dfrac{x}{4}=12 \)
			\task \( 0,18x-3,54=0,19x-2,89 \)
			\task \( \mfrac{2}{2}{5}x+ \mfrac{3}{2}{15}=\mfrac{3}{1}{5}x + \mfrac{2}{1}{3} \)
			\task \( \dfrac{1}{4} - \dfrac{1}{3}m=\mfrac{4}{1}{4}-3m \)
			\task \( \dfrac{-0,2(6x+1)}{3,6}=\dfrac{0,5x}{-9} \)
			\task \( \dfrac{3x-2,4}{0,02}=\dfrac{8-x}{0,1} \)
			\task \( \dfrac{3,6}{0,2(6y+1)}=\dfrac{9}{0,5y} \)
			\task \( \dfrac{1,4x-3,5}{0,25}=\dfrac{4,6x-18}{-1,5} \)
			\task \( 4(1,2x+3,7)=0,2(2,6x-14) \)
			\task \( 0,3(5x-7)=3(0,2x+3,2) \)
			\task \( \dfrac{x}{5}-4=-0,1x+2 \)
			\task \( 4,37+6,7x=7,75+9,3x \)
		\end{tasks}
		\item Решите уравнения: %5.36
		\begin{tasks}(2)
			\task \( |x|=5 \)
			\task \( |x|=0 \)
			\task \( |x|=-2 \)
			\task \( |x+57|=13 \)
			\task \( |2x-13|=3 \)
			\task \( |4x-5|=9 \)
			\task \( \left| \dfrac{1}{3}x=5 \right|=6 \)
			\task \( \left| \dfrac{2x}{3}-9 \right|=9 \)
			\task \( \left| \dfrac{3}{4}x-\mfrac{1}{3}{7} \right|=\mfrac{2}{4}{9} \)
			\task \( ||x-1|-2|=3 \)
			\task \( 3|x|-8=9-4|x| \)
			\task \( 1-|23x-25|=7 \)
		\end{tasks}
	\end{listofex}
\end{class}
%END_FOLD

%BEGIN_FOLD % ====>>_ Домашняя работа 1 _<<====
\begin{homework}[number=1]
	\begin{listofex}
		\item Разделите число:
		\begin{tasks}(2)
			\task \( 156 \) в отношении \( 2:1 \)
			\task \( 270 \) в отношении \( 5:3:2 \)
			\task \( 2160 \) в отношении \( 4:5:6 \)
			\task \( 1210 \) в отношении \( 2,5:4:5,5 \)
		\end{tasks}
		\item Вычислите:
		\begin{tasks}(3)
			\task \( -17 + 21 \)
			\task \( -8,5 - 0,6 \)
			\task \( \dfrac{1}{6} - 13,5 \)
			\task \( -4 + \mfrac{7}{4}{5} \)
			\task \( -17 + \dfrac{5}{-3} \)
			\task \( -10,33 - (-11,72) \)
			\task \( 15 \cdot (-4) \)
			%\task \( -5 \cdot \dfrac{4}{-5} \)
			%\task \( -11 \cdot (-0,5) \)
			\task \( -\dfrac{1}{3} \cdot \mfrac{1}{1}{2} \)
			\task \( -\dfrac{5}{3}\cdot \dfrac{-1}{-18} \)
			
			\task \( 14 : (-0,7) \)
			\task \( 0,16 : (-0,4) \)
			%\task \( 42 : (-3,5) \)
			%\task \( -\dfrac{8}{25} : (-0,125) \)
			\task \( 7,11 : (-711) \)
		\end{tasks}
		\item Решите уравнения:
		\begin{tasks}(2)
			\task \( \dfrac{-4}{x}=116 \)
			\task \( -\dfrac{-y}{-18}=\dfrac{1}{3} \)
			\task \( -15,6:y=\dfrac{3}{-11} \)
			\task \( \dfrac{x}{-0,72}=1,2 \cdot 0,4 \)
		\end{tasks}
	\end{listofex}
\end{homework}
%END_FOLD

%BEGIN_FOLD % ====>>_____ Занятие 3 _____<<====
\begin{class}[number=3]
	\begin{listofex}
		\item Занятие 3 
	\end{listofex}
\end{class}
%END_FOLD

%BEGIN_FOLD % ====>>_____ Занятие 4 _____<<====
\begin{class}[number=4]
	\begin{listofex}
		\item Занятие 4
	\end{listofex}
\end{class}
%END_FOLD

%BEGIN_FOLD % ====>>_ Домашняя работа 2 _<<====
\begin{homework}[number=2]
	\begin{listofex}
		\item Решите уравнения: % 5.21 й-р и 5.24 б г
		\begin{tasks}(2)
			\task \( 5x=3x \)
			\task \( -2n=5,7n \)
			\task \( 49x=50 \)
			\task \( x-156=16 \)
			\task \( -(11+x)=2,25 \)
			\task \( (-20x-50)\cdot 2 = 100 \)
			\task \( -3 \cdot (2-15x) = -6 \)
			\task \( -6x+2(5-3x)=8 \)
			\task \( |4-|x-5||-1=3 \)
			\task \( |5-|x+6||+1=6 \)
			\task \( 3|x|-5=10 \)
			\task \( |8-9x|=12 \)
		\end{tasks}
		\item Решите уравнение, используя основное свойство пропорции:
		\begin{tasks}(2)
			\task \( \dfrac{5+3x}{12} = \dfrac{4x-3}{18} \)
			\task \( \dfrac{0,9}{7+5y} = \dfrac{0,2}{y-4} \)
			\task \( \dfrac{x+7}{3} = \dfrac{2x-3}{5} \)
			\task \( \dfrac{0,2}{x+3} = \dfrac{0,7}{x-2} \)
		\end{tasks}
		\item Решите уравнение: % 5.30 щ ш
		\begin{tasks}
			\task \( \dfrac{1}{3} \left( \dfrac{5}{12}-4m \right) =\dfrac{4}{9} \left(  \mfrac{1}{1}{1} m-\dfrac{3}{8} \right) \)
			\task \( 0,3(5x-7)=3(0,2x+3,2) \)
		\end{tasks}
	\end{listofex}
\end{homework}
%END_FOLD

%BEGIN_FOLD % ====>>_____ Занятие 5 _____<<====
\begin{class}[number=5]
	\begin{listofex}
		\item Занятие 5
	\end{listofex}
\end{class}
%END_FOLD

%BEGIN_FOLD % ====>>_____ Занятие 6 _____<<====
\begin{class}[number=6]
	\begin{listofex}
		\item Занятие 6
	\end{listofex}
\end{class}
%END_FOLD

%BEGIN_FOLD % ====>>_ Домашняя работа 3 _<<====
\begin{homework}[number=3]
	\begin{listofex}
		\item Домашняя работа 3
	\end{listofex}
\end{homework}
%END_FOLD

%BEGIN_FOLD % ====>>_____ Занятие 7 _____<<====
\begin{class}[number=7]
	\title{Подготовка к проверочной}
	\begin{listofex}
		\item Занятие 7
	\end{listofex}
\end{class}
%END_FOLD

=%BEGIN_FOLD % ====>>_ Проверочная работа _<<====
\begin{exam}
	\begin{listofex}
		\item Проверочная
	\end{listofex}
\end{exam}
%END_FOLD