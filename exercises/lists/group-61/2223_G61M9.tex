\setmodule{9}

%BEGIN_FOLD % ====>>_____ Занятие 1 _____<<====
\begin{class}[number=1]
	\begin{definit}
	Координатная прямая --- это прямая, имеющая направление, начало отсчета и заданный масштаб.
	\end{definit}
	\begin{minipage}[c]{0.9\linewidth}
		\includegraphics[align=t, width=\linewidth]{\picpath/../../bank/graphs/graph_gleb/7/straight}
	\end{minipage}
	\begin{definit}
		Координатная плоскость --- это прямоугольная система координат, состоящая из двух взаимно перпендикулярных осей, имеющих направление, начало отсчета и единичные отрезки.
	\end{definit}
	\begin{definit}
		Начало координат --- точка \(O\).
		Оси координат --- прямые \(x\) и \(y\).
	\end{definit}
	\begin{minipage}[c]{0.9\linewidth}
		\includegraphics[align=t, width=\linewidth]{\picpath/../../bank/graphs/graph_gleb/7/coord}
	\end{minipage}
	\begin{minipage}[c]{0.6\linewidth}
		\begin{definit}
			Строгий знак означает, что точка \textbf{не} включается в промежуток: \(x>a\). \\
			Нестрогий знак означает, что точка включается в промежуток: \(x\ge b\).
		\end{definit}
	\end{minipage}
	\begin{minipage}[c]{0.4\linewidth}
		\includegraphics[align=t, width=\linewidth]{\picpath/../../bank/graphs/graph_gleb/7/points}
	\end{minipage}
	\begin{listofex}
		\item Отметьте на координатной прямой точки, удовлетворяющие неравенству. Запишите какие-нибудь три числа, удовлетворяющие этому неравенству и, если возможно, наибольшее и наименьшее целое число, удовлетворяющие этому неравенству.
		\begin{tasks}(3)
			\task \( x \le 5 \)
			\task \( x \le 2 \)
			\task \( x < 4 \)
			\task \( z<3 \)
			\task \( -2 \le x \le 5 \)
			\task \( -5 \le x \le 1 \)
			\task \( -2 < x < 7 \)
			\task \( -3 < x < 4 \)
		\end{tasks}
		\item Отметьте на координатной прямой точки, удовлетворяющие системе неравенств. Запишите какие-нибудь три числа, удовлетворяющие этой системе неравенств и, если возможно, наибольшее и наименьшее целое число, удовлетворяющие этой системе неравенств.
		\begin{tasks}(2)
			\task \( \begin{cases} x \ge 5 \\ x \le 7 \end{cases} \)
			\task \( \begin{cases} x \ge 1 \\ x \le 5 \end{cases} \)
			\task \( \begin{cases} x>5 \\ x<6 \end{cases} \)
			\task \( \begin{cases} x<7 \\ x>5 \end{cases} \)
		\end{tasks}
		\item Отметьте на координатной прямой точки, удовлетворяющие совокупности неравенств. Запишите какие-нибудь три числа, удовлетворяющие этой совокупности неравенств и, если возможно, наибольшее и наименьшее целое число, удовлетворяющие этой совокупности неравенств.
		\begin{tasks}(2)
			\task \( \left[ 
			\begin{array}{l}
				x>-2\\
				x>6
			\end{array}
			\right. \)
			\task \( \left[
			\begin{array}{l} x<1 \\ x<-3 \end{array} \right. \)
			\task \( \left[
			\begin{array}{l} x \le -4 \\ x>-1 \end{array} \right. \)
			\task \( \left[
			\begin{array}{l} x \le -2 \\ x < -5 \end{array} \right. \)
		\end{tasks}
		
	\end{listofex}
\end{class}
%END_FOLD
%\task \( \left[
%\begin{array}{l}  \\  \end{array} \right. \)
%BEGIN_FOLD % ====>>_____ Занятие 2 _____<<====
\begin{class}[number=2]
	\begin{listofex}
		\item Занятие 2
	\end{listofex}
\end{class}
%END_FOLD

%BEGIN_FOLD % ====>>_ Домашняя работа 1 _<<====
\begin{homework}[number=1]
	\begin{listofex}
		\item Домашняя работа 1
	\end{listofex}
\end{homework}
%END_FOLD

%BEGIN_FOLD % ====>>_____ Занятие 3 _____<<====
\begin{class}[number=3]
	\begin{listofex}
		\item Занятие 3 
	\end{listofex}
\end{class}
%END_FOLD

%BEGIN_FOLD % ====>>_____ Занятие 4 _____<<====
\begin{class}[number=4]
	\begin{listofex}
		\item Занятие 4
	\end{listofex}
\end{class}
%END_FOLD

%BEGIN_FOLD % ====>>_ Домашняя работа 2 _<<====
\begin{homework}[number=2]
	\begin{listofex}
		\item Домашняя работа 2
	\end{listofex}
\end{homework}
%END_FOLD

%BEGIN_FOLD % ====>>_____ Занятие 5 _____<<====
\begin{class}[number=5]
	\begin{listofex}
		\item Занятие 5
	\end{listofex}
\end{class}
%END_FOLD

%BEGIN_FOLD % ====>>_____ Занятие 6 _____<<====
\begin{class}[number=6]
	\begin{listofex}
		\item Занятие 6
	\end{listofex}
\end{class}
%END_FOLD

%BEGIN_FOLD % ====>>_ Домашняя работа 3 _<<====
\begin{homework}[number=3]
	\begin{listofex}
		\item Домашняя работа 3
	\end{listofex}
\end{homework}
%END_FOLD

%BEGIN_FOLD % ====>>_____ Занятие 7 _____<<====
\begin{class}[number=7]
	\title{Подготовка к проверочной}
	\begin{listofex}
		\item Занятие 7
	\end{listofex}
\end{class}
%END_FOLD

%BEGIN_FOLD % ====>>_ Проверочная работа _<<====
\begin{exam}
	\begin{listofex}
		\item Проверочная
	\end{listofex}
\end{exam}
%END_FOLD