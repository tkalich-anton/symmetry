%
%===============>>  ГРУППА 6-1 МОДУЛЬ 8  <<=============
%
\setmodule{8}

%BEGIN_FOLD % ====>>_____ Занятие 1 _____<<====
\begin{class}[number=1]
	\begin{listofex}
		\item Решите уравнения: %5.31 б в е ё л 5.30 щ ш
		\begin{tasks}(2)
			\task \( 49x=50 \)
			\task \( 5x=3x \)
			\task \( 3,2-5a=-1,8a+4 \)
			\task \( x=-\dfrac{ 1 }{ 6 }x \)
			\task \( \dfrac{ 4x-3 }{ 3-5x }=\dfrac{ 0,14 }{ 0,35 } \)
			
			\task \( 1-\dfrac{ a }{7  }=\dfrac{ a }{ 14 }-0,25a \)
			\task \( 3 - \left( \dfrac{ 2 }{ 9 }m+\dfrac{ 1 }{ 6 } \right)=\dfrac{ m }{ 3 }+1,5  \)
			\task \( 0,3(5x-7)=3(0,5x+3,2) \)
			%\task \( |4-|x-5||-1=3 \)
			%\task \( |5-|x+6||+1=6 \)
			%\task \( \dfrac{1}{3} \left( \dfrac{5}{12}-4m \right) =\dfrac{4}{9} \left(  \mfrac{1}{1}{1} m-\dfrac{3}{8} \right) \)
			
		\end{tasks}
		%c46 1.187-1.191
		\item Автомобиль выехал из пункта \(A\) со скоростью \(60\) км/ч. Через \(2\) ч вслед за ним выехал второй автомобиль со скоростью \(90\) км/ч. Через какое время и на каком расстоянии от \(A\) второй автомобиль догонит первый?
		\item Собственная скорость катера \(25,5\) км/ч, скорость течения \(2,5\) км/ч. Какой путь пройдёт катер за полтора часа по течению и против течения?
		\item Собственная скорость лодки \(8,5\) км/ч, а скорость течения \(3,5\) км/ч. Расстояние между пристанями \(15\) км. Сколько времени затратит лодка на путь между пристанями туда и обратно?
		\item Плот и лодка движутся навстречу друг другу по реке. Они находятся на расстоянии \(20\) км друг другу по реке. Они находятся на расстоянии \(20\) км друг от друга. Через какое время они встретятся, если собственная скорость лодки \(8\) км/ч, а скорость течения реки \(2\) км/ч?
		\item Два велосипедиста одновременно выехали из лагеря в противоположных направлениях со скоростями \(10\) км/ч и \(12\) км/ч. Какое расстояние будет между ними через \(2\) ч? Через \(3\) ч \(6\) минут? Через какое время расстояние между ними будет равно \(33\) км?
	\end{listofex}
\end{class}
%END_FOLD

%BEGIN_FOLD % ====>>_____ Занятие 2 _____<<====
\begin{class}[number=2]
	\begin{listofex}
		\item Занятие 2
	\end{listofex}
\end{class}
%END_FOLD

%BEGIN_FOLD % ====>>_ Домашняя работа 1 _<<====
\begin{homework}[number=1]
	\begin{listofex}
		\item Домашняя работа 1
	\end{listofex}
\end{homework}
%END_FOLD

%BEGIN_FOLD % ====>>_____ Занятие 3 _____<<====
\begin{class}[number=3]
	\begin{listofex}
		\item Занятие 3 
	\end{listofex}
\end{class}
%END_FOLD

%BEGIN_FOLD % ====>>_____ Занятие 4 _____<<====
\begin{class}[number=4]
	\begin{listofex}
		\item Занятие 4
	\end{listofex}
\end{class}
%END_FOLD

%BEGIN_FOLD % ====>>_ Домашняя работа 2 _<<====
\begin{homework}[number=2]
	\begin{listofex}
		\item Домашняя работа 2
	\end{listofex}
\end{homework}
%END_FOLD

%BEGIN_FOLD % ====>>_____ Занятие 5 _____<<====
\begin{class}[number=5]
	\begin{listofex}
		\item Занятие 5
	\end{listofex}
\end{class}
%END_FOLD

%BEGIN_FOLD % ====>>_____ Занятие 6 _____<<====
\begin{class}[number=6]
	\begin{listofex}
		\item Занятие 6
	\end{listofex}
\end{class}
%END_FOLD

%BEGIN_FOLD % ====>>_ Домашняя работа 3 _<<====
\begin{homework}[number=3]
	\begin{listofex}
		\item Домашняя работа 3
	\end{listofex}
\end{homework}
%END_FOLD

%BEGIN_FOLD % ====>>_____ Занятие 7 _____<<====
\begin{class}[number=7]
	\title{Подготовка к проверочной}
	\begin{listofex}
		\item Занятие 7
	\end{listofex}
\end{class}
%END_FOLD

=%BEGIN_FOLD % ====>>_ Проверочная работа _<<====
\begin{exam}
	\begin{listofex}
		\item Проверочная
	\end{listofex}
\end{exam}
%END_FOLD