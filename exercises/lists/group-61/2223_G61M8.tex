%
%===============>>  ГРУППА 6-1 МОДУЛЬ 8  <<=============
%
\setmodule{8}

%BEGIN_FOLD % ====>>_____ Занятие 1 _____<<====
\begin{class}[number=1]
	\begin{listofex}
		\item Решите уравнения: %5.31 б в е ё л 5.30 щ ш
		\begin{tasks}(2)
			\task \( 49x=50 \)
			\task \( 5x=3x \)
			\task \( 3,2-5a=-1,8a+4 \)
			\task \( x=-\dfrac{ 1 }{ 6 }x \)
			\task \( \dfrac{ 4x-3 }{ 3-5x }=\dfrac{ 0,14 }{ 0,35 } \)
			
			\task \( 1-\dfrac{ a }{7  }=\dfrac{ a }{ 14 }-0,25a \)
			\task \( 3 - \left( \dfrac{ 2 }{ 9 }m+\dfrac{ 1 }{ 6 } \right)=\dfrac{ m }{ 3 }+1,5  \)
			\task \( 0,3(5x-7)=3(0,5x+3,2) \)
			%\task \( |4-|x-5||-1=3 \)
			%\task \( |5-|x+6||+1=6 \)
			%\task \( \dfrac{1}{3} \left( \dfrac{5}{12}-4m \right) =\dfrac{4}{9} \left(  \mfrac{1}{1}{1} m-\dfrac{3}{8} \right) \)
			
		\end{tasks}
		%c46 1.187-1.191
		\item Автомобиль выехал из пункта \(A\) со скоростью \(60\) км/ч. Через \(2\) ч вслед за ним выехал второй автомобиль со скоростью \(90\) км/ч. Через какое время и на каком расстоянии от \(A\) второй автомобиль догонит первый?
		\item Собственная скорость катера \(25,5\) км/ч, скорость течения \(2,5\) км/ч. Какой путь пройдёт катер за полтора часа по течению и против течения?
		\item Собственная скорость лодки \(8,5\) км/ч, а скорость течения \(3,5\) км/ч. Расстояние между пристанями \(15\) км. Сколько времени затратит лодка на путь между пристанями туда и обратно?
		\item Плот и лодка движутся навстречу друг другу по реке. Они находятся на расстоянии \(20\) км друг другу по реке. Они находятся на расстоянии \(20\) км друг от друга. Через какое время они встретятся, если собственная скорость лодки \(8\) км/ч, а скорость течения реки \(2\) км/ч?
		\item Два велосипедиста одновременно выехали из лагеря в противоположных направлениях со скоростями \(10\) км/ч и \(12\) км/ч. Какое расстояние будет между ними через \(2\) ч? Через \(3\) ч \(6\) минут? Через какое время расстояние между ними будет равно \(33\) км?
	\end{listofex}
\end{class}
%END_FOLD

%BEGIN_FOLD % ====>>_____ Занятие 2 _____<<====
\begin{class}[number=2]
	\begin{listofex}
		\item Решите уравнения:
		\begin{tasks}(2)
			\task \( 5,5x-24\cdot \dfrac{ 1 }{ 3 }=5x-7,5 \)
			\task \( \left( \dfrac{ 1 }{ 3 } - 3x \right)\cdot 6 = 8-0,5x \)
			\task \( 15x-16=-(-8x+11) \)
			\task \( 4-6,17x-18=-5,3x-16,16 \)
			\task \( \dfrac{ 5x }{ 17 }=\dfrac{ 10,5x }{ 11 } \)
			\task \( \dfrac{ 4x-5 }{ 12,5 }=\dfrac{ 8x-3 }{ 0,2 } \)
			\task \( \dfrac{ 3-2x }{ 11x-5,2 }=\dfrac{ 5 }{ 6 } \)
			\task \( \dfrac{ 17 }{ 0,5 }=\dfrac{ 14x-5 }{ 8-5x } \)
		\end{tasks}
		%c46 1.187-1.191 1.192-1.196
		\item Два велосипедиста выехали одновременно из двух сёл навстречу друг другу и встретились через \(1,6\) ч. Скорость первого \(10\) км/ч, а второго --- \(12\) км/ч. Найдите расстояние между сёлами?
		\item Два поезда одновременно вышли с одной станции в одном направлении. Их скорости \(60\) км/ч и \(70\) км/ч. Какое расстояние будет между ними через \(1,5\) часа? через \(2\) часа \(25\) мин? Через сколько часов расстояние между ними будет равно \(35\) км?
		\item Города \(A\) и \(B\) расположены на реке, причём \(B\) ниже по течению. Расстояние между ними равно \(30\) км. Моторная лодка проходит путь от \(A\) до \(B\) за \(2\) ч, а обратно за \(3\) ч. За какое время проплывёт от \(A\) до \(B\) плот?
		\item Пассажир метро, стоя на ступеньке эскалатора, поднимается наверх за \(3\) мин. За сколько минут он поднимется вверх по движущемуся эскалатору, если будет идти со скоростью \(25\) м/мин? Длина эскалатора \(150\) м.
		\item Расстояние между станциями \(350\) км. От этих станций одновременно навстречу друг другу вышли два поезда. Они встретились через \(2,5\) часа. Определите скорость первого поезда, если скорость второго равна \(65\) км.
	\end{listofex}
\end{class}
%END_FOLD

%BEGIN_FOLD % ====>>_ Домашняя работа 1 _<<====
\begin{homework}[number=1]
	\begin{listofex}
		\item Решите уравнения: %5.30 л м т у ф ч
		\begin{tasks}(2)
			\task \( 3(x+2)=-12 \)
			\task \( -2(y-1)-5=4 \)
			\task \( 0,8y+1,4=0,4-2,6 \)
			\task \( 0,18x-3,54=0,19x-2,89 \)
			\task \( \dfrac{ 2x }{ 3 }=\dfrac{ x-1 }{6  } \)
			\task \( \dfrac{ 1,4x-3,5 }{ 0,25 }=\dfrac{ 4,6x-18 }{ -1,5 } \)
		\end{tasks}
		%1.197-1.199
		\item Расстояние между станциями \(A\) и \(B\) равно \(165\) км. От этих станций одновременно навстречу друг другу выходят два поезда и встречаются через \(1,5\) ч на разъезде, который находится в \(90\) км от станции \(A\). С какой скоростью идут поезда?
		\item Из двух городов, расстояние между которыми \(45\) км, одновременно в одном направлении вышли поезда со скоростями \(70\) км/ч и \(60\) км/ч, причём первый поезд догоняет второй. Через сколько времени расстояние между поездами будет равно \(10\) км?
		\item Два поезда выехали одновременно из пунктов \(A\) и \(B\) навстречу друг другу. Расстояние между пунктами \(A\) и \(B\) равно \(350\) км. Скорость первого \(65\) км/ч, второго --- \(75\) км/ч. Через сколько часов расстояние между поездами составит \(70\) км?
	\end{listofex}
\end{homework}
%END_FOLD

%BEGIN_FOLD % ====>>_____ Занятие 3 _____<<====
\begin{class}[number=3]
	\begin{listofex}
		\item Решите уравнения: %5.22 м-с
		\begin{tasks}(2)
			\task \( 4n+2=6n-7 \)
			\task \( 2,15-0,1c=5,5c+1 \)
			\task \( -3d-10=3d-6,6 \)
			\task \( \dfrac{ 5 }{ 6 }m-2=-\dfrac{ 1 }{ 3 }m-0,8 \)
			\task \( -1,6-0,3p=0,9p+0,2 \)
			\task \( \dfrac{ 11 }{ 12 }x-\dfrac{ 2 }{ 3 }=-0,5-0,75x \)
			\task \( \dfrac{ -16+22x }{ 11 }=\dfrac{ -2,3x-5,5 }{ 10 } \)
			\task \( \dfrac{ 16x-1,25 }{ 16,4 }=\dfrac{ -7x-20,1 }{ 8,2 } \)
		\end{tasks}
		%1.200-1.203
		\item Из двух сел одновременно выехали навстречу друг другу два велосипедиста. Скорость одного из них \(19,5\) км/ч, а скорость второго составляет \(\dfrac{ 2 }{ 3 }\) скорости первого. Какое расстояние между селами, если велосипедисты встретились через \(48\) мин? На каком расстоянии друг от друга они были через \(0,5\) ч после выезда? Через полтора часа?
		\item Города \(A\) и \(B\) расположены на одном шоссе. Из этих городов одновременно в одном направлении выехали два автобуса. Первый автобус двигался со скоростью \(54\) км/ч, что составляет \(0,6\) скорости второго автобуса. Второй автобус догнал первый через \(1\) ч \(30\) мин после выезда. Каково расстояние между городами \(A\) и \(B\)? На каком расстоянии друг от друга были автобусы через \(24\) мин после выезда? Через \(2\) ч после выезда?
		\item Из двух городов одновременно выехали навстречу друг другу два автомобиля. Скорость одного из них \(122,5\) км/ч, а скорость второго составляет \(\dfrac{ 5 }{ 7 }\) скорости первого. Какое расстояние между городами, если автомобили встретились через \(40\) мин? На каком расстоянии друг от друга они были через \(0,5\) ч после выезда? Через полтора часа?
		\item Города \(A\) и \(B\) расположены на одном шоссе. Из этих городов одновременно в одном направлении вышли два поезда. Скорость первого поезда \(35\) км/ч, что составляет \(0,7\) скорости второго поезда. Второй поезд догнал первый через \(1\) ч \(30\) мин после выезда. Каково расстояние между городами \(A\) и \(B\)? На каком расстоянии друг от друга были автобусы через \(36\) мин после выезда? Через \(2\) ч после выезда?
	\end{listofex}
\end{class}
%END_FOLD

%BEGIN_FOLD % ====>>_____ Занятие 4 _____<<====
\begin{class}[number=4]
	\begin{listofex}
		\item Решите уравнения: %1.313 а-е
		\begin{tasks}(2)
			\task \( (8,3-x) \cdot 4,7 = -5,64 \)
			\task \( -(9,2-x) - 14,2 = 16 \cdot (-0,25) \)
			\task \( 2,136:(1,9-x)=-7,12 \)
			\task \( 4,2 \cdot (-0,8+x) = -0,5x + \dfrac{ -3 }{ 2 } \)
			\task \( (x-1,2) : (-0,6) = 21,1 \)
			\task \( (-x-26,1) \cdot (-2,3)=70,84 \)
		\end{tasks}
		% до 1.206
		\item Из двух городов одновременно выехали навстречу друг другу два автомобиля. Скорость одного из них \(122,5\) км/ч, а скорость второго составляет \(\dfrac{ 5 }{ 7 }\) скорости первого. Какое расстояние между городами, если автомобили встретились через \(40\) мин? На каком расстоянии друг от друга они были через \(0,5\) ч после выезда? Через полтора часа?
		\item Города \(A\) и \(B\) расположены на одном шоссе. Из этих городов одновременно в одном направлении вышли два поезда. Скорость первого поезда \(35\) км/ч, что составляет \(0,7\) скорости второго поезда. Второй поезд догнал первый через \(1\) ч \(30\) мин после выезда. Каково расстояние между городами \(A\) и \(B\)? На каком расстоянии друг от друга были автобусы через \(36\) мин после выезда? Через \(2\) ч после выезда?
		\item Из двух городов, расстояние между которыми \(206,15\) км, одновременно навстречу друг другу выехали электричка и товарный поезд. Скорость электрички равна \(31,5\) км/ч, а скорость товарного поезда составляет \(\dfrac{ 13 }{ 18 }\) скорости электрички. Какое расстояние проедет товарный поезд до его встречи с электричкой?
		\item Два велосипедиста, находясь друг от друга на расстоянии \(9\) км, выехали одновременно навстречу друг другу и через \(20\) мин встретились. Когда же они выехали из одного пункта в одном направлении, то через \(1\) ч \(40\) мин один отстал от другого на \(5\) км. Какова скорость каждого велосипедиста?
		\item Два пешехода вышли одновременно навстречу друг другу из пунктов \(A\) и \(B\). При встрече оказалось, что первый пешеход прошел \(\dfrac{ 1 }{ 4 }\) всего пути и еще \(3,2\) км, а второй --- в \(2\) раза больше первого. Чему равно расстояние от \(A\) до \(B\)?
	\end{listofex}
\end{class}
%END_FOLD

%BEGIN_FOLD % ====>>_ Домашняя работа 2 _<<====
\begin{homework}[number=2]
	\begin{listofex}
		\item Решите уравнения: %1.313 ё-и
		\begin{tasks}(2)
			\task \( (10,49-x):4,02=0,805 \)
			\task \( 38007:(4223-x)=9 \)
			\task \( 22374:(x-125)=12345 \)
			\task \( 295,1:(x-3)=13 \)
		\end{tasks}
		% 1.208-1.211
		\item Два пешехода вышли одновременно навстречу друг другу из пунктов \(A\) и \(B\). При встрече оказалось, что первый пешеход прошел \(\dfrac{ 1 }{ 4 }\) всего пути и еще \(3,2\) км, а второй --- в \(2\) раза больше первого. Чему равно расстояние от \(A\) до \(B\)?
		\item Из города в \(8\) часов утра выехал велосипедист со скоростью \(20\) км/ч. Через \(4\) часа велосипедист сделал часовой привал, а в этот момент вслед за ним из города выехал мотоциклист со скоростью \(50\) км/ч. В какое время мотоциклист догонит велосипедиста? На каком расстоянии от города это произойдёт? Какое расстояние будет между велосипедистом и мотоциклистом через в \(6\) часов вечера?
		\item Из города \(A\) в город \(B\), расстояние между которыми \(620\) км выехала легковая машина со скоростью \(60\) км/ч. Через два часа из города \(B\) в город \(A\) выехал грузовик со скоростью \(40\) км/ч.
		\begin{tasks}
			\task Через какое время после выезда грузовика автомобили встретились? 
			\task На каком расстоянии от города \(A\) произошла встреча?
			\task Какое расстояние будет между автомобилями через \(7\) часов после выезда грузовика?
		\end{tasks}
	\end{listofex}
\end{homework}
%END_FOLD

%BEGIN_FOLD % ====>>_____ Занятие 5 _____<<====
\begin{class}[number=5]
	\begin{definit}
		Длина окружности --- длина замкнутой кривой, ограничивающей круг. Вычисляется по формуле: \( C = 2 \pi R \), где \(R\) --- радиус окружности.
	\end{definit}
	\begin{definit}
		Площадь окружности: \( S = \pi R^2 \), где \(R\) --- радиус окружности
	\end{definit}
	\begin{listofex}
		%G62M7L3-4
		\item Найдите площадь окружности, если её радиус равен \( 2 \) см, \(\pi\) примите равным \(3,14\).
		\item Диаметр окружности равен \(5\) см, найдите площадь окружности, \(\pi\) примите равным \(3,14\).
		\item Найдите радиус окружности, если её длина равна \( 11\pi \) см.
		\item Найдите площадь и длину окружности, если диаметр окружности равен \( \dfrac{8}{\pi} \).
		\item Найдите площадь окружности, если длина окружности равна \(6\) см, \(\pi\) примите равным \(3,14\).
		\item Две окружности, имеющие общий центр, образуют кольцо. Радиус внешней окружности равен \(10\) см, а внутренней \(8\) см. Найти площадь этого кольца, \(\pi\) примите равным \(3,14\).
		
		\item Найдите площадь сектора окружности, угол которого составляет \(120 \degree\), если:
		\begin{tasks}(2)
			\task площадь окружности равна \( 33 \) см
			\task радиус окружности равен \( 5 \) см
		\end{tasks}
		\(\pi\) примите равным \(3,14\).
		\item Диаметр окружности равен \(15\) см. Найдите площадь сектора окружности, если:
		\begin{tasks}(2)
			\task угол сектора составляет \( 45 \degree \)
			\task угол сектора составляет \( 135 \degree \)
		\end{tasks}
		\(\pi\) примите равным \(3,14\).
		\item Два пешехода вышли одновременно навстречу друг другу из пунктов \(A\) и \(B\). При встрече оказалось, что первый пешеход прошел \(\dfrac{ 1 }{ 5 }\) всего пути и еще \(1,3\) км, а второй --- в \(3\) раза больше первого. Чему равно расстояние от \(A\) до \(B\)?
		
	\end{listofex}
\end{class}
%END_FOLD

%BEGIN_FOLD % ====>>_____ Занятие 6 _____<<====
\begin{class}[number=6]
	\begin{listofex}
		\item Решите уравнения:
		\begin{tasks}(2)
			\task \( 85-x=25 \)
			
			\task \( -\left( -\dfrac{8}{15} + x \right) = -4 \)
			
			\task \( |-15|:(-3)=x+2 \)
			
			\task \( \dfrac{1}{x} = -|6-25,3| \)
			
			%\task \( \dfrac{18}{x} = \dfrac{3}{9} \)
			\task \( \dfrac{x}{4} = -\dfrac{24}{11} \)
			%\task \( \dfrac{17}{x} = \dfrac{34}{-28} \)
			\task \( \dfrac{-x}{15} = \left| -\dfrac{5}{6} \right| \)
		\end{tasks}
		%G62M7H2
		\item Найдите площадь и длину окружности, если радиус окружности равен \( \dfrac{16}{\pi} \) см.
		\item Найдите радиус окружности, если её длина равна \( 6\pi \) см.
		\item Диаметр окружности равен \(10\pi\) см, найдите площадь окружности.
		\item Две окружности имеют общий центр. Диаметр внешней окружности равен \(4\) см, а внутренней \(2,5\) см. Найти площадь кольца, которое образуется между окружностями.
		\item Найдите площадь сектора окружности, угол которого составляет \(120 \degree\), если площадь окружности равна \( 10 \) см, \(\pi\) примите равным \(3,14\).
		\item Найдите площадь сектора окружности, угол которого составляет \(30 \degree\), если диаметр окружности равен \( 5 \) см, \(\pi\) примите равным \(3,14\).
		%1.212
		\item Расстояние между городами \(A\) и \(B\) равно \(720\) км. Из города \(A\) в \(B\) город выезжает товарный поезд со скоростью \(60\) км/ч. Через \(2\) часа навстречу ему из города \(B\) в город \(A\) выезжает скорый поезд со скоростью \(90\) км/ч. На каком расстояния от города поезда встретятся?
		
		
		 \item Города \(A\) и \(B\) расположены на реке, причём \(B\) ниже по течению. Расстояние между ними равно \(30\) км. Моторная лодка проходит путь от \(A\) до \(B\) за \(2\) ч, а обратно за \(3\) ч. За какое время проплывёт от \(A\) до \(B\) плот?
		%1.217
		\item Из пункта \(A\) круговой трассы, длина которой равна \(80\) км, одновременно в одном направлении стартовали два автомобилиста. Скорость первого автомобилиста равна \(92\) км/ч, скорость второго --- \(68\) км/ч. Через сколько минут первый автомобилист будет опережать второго ровно на один круг? На два круга?
	\end{listofex}
\end{class}
%END_FOLD

%BEGIN_FOLD % ====>>_ Домашняя работа 3 _<<====
\begin{homework}[number=3]
	\begin{listofex}
		\item Решите уравнения:
		\begin{tasks}(2)
			\task \( 50,3-2x=-11,7 \)
			\task \( \left( -x+\dfrac{ 5 }{ 3 } \right)=|-4| \)
			\task \( \mfrac{3 }{4 }{5 }+x=-|0,5-2| \)
			\task \( \dfrac{ -x }{ 3 }=\dfrac{ 8 }{ 9 } \)
		\end{tasks}
		\item Найдите диаметр окружности, если её длина равна \( \dfrac{ 9 }{ 2 }\pi \) см.
		\item Две окружности имеют общий центр. Радиус внешней окружности равен \(10\) см, а диаметр внутренней \(3\) см. Найти площадь кольца, которое образуется между окружностями.
		\item Найдите площадь сектора окружности, угол которого составляет \(60 \degree\), если площадь окружности равна \( 15 \) см, \(\pi\) примите равным \(3,14\).
		\item Расстояние между городами \(A\) и \(B\) равно \(760\) км. Из города \(A\) в \(B\) город выезжает пассажирский поезд со скоростью \(80\) км/ч. Через \(2\) часа навстречу ему из города \(B\) в город \(A\) выезжает товарный поезд со скоростью \(40\) км/ч. На каком расстояния от города \(B\) поезда встретятся?
	\end{listofex}
\end{homework}
%END_FOLD

%BEGIN_FOLD % ====>>_____ Занятие 7 _____<<====
\begin{class}[number=7]
	\title{Подготовка к проверочной}
	\begin{listofex}
		\item Занятие 7
	\end{listofex}
\end{class}
%END_FOLD

=%BEGIN_FOLD % ====>>_ Проверочная работа _<<====
\begin{exam}
	\begin{listofex}
		\item Проверочная
	\end{listofex}
\end{exam}
%END_FOLD

%BEGIN_FOLD % ====>>_Консультация Мелёхина _<<====
\begin{consultation}
	\begin{listofex}
		\item Кофейник и две чашки вмещают \(740\) г воды. В кофейник входит на \(380\) г больше, чем в чашку. Сколько граммов воды вмещает кофейник?
		
		\item За три дня было продано \(830\) кг апельсинов. Во второй день продали на \(30\) кг меньше, чем в первый, а в третий --- в \(3\) раза больше, чем во второй. Сколько килограммов апельсинов было продано в первый день? 
		
		\item Велосипедист проехал \(43\) км. По проселочной дороге он проехал в \(3\) раза большее расстояние, чем по лесной тропинке, а по тропинке на \(35\) км меньше, чем по шоссе. Какой длины была каждая часть пути велосипедиста?
		
		\item В двух альбомах \(750\) марок, причем в первом альбоме имевшихся марок составляли иностранные марки. Во втором альбоме иностранные марки составляли \(0,9\) имевшихся там марок. Сколько всего марок было в каждом альбоме, если число иностранных марок в них было одинаково?
		
		\item В одной бочке \(110\) л бензина, а в дугой \(130\) л. После того как из второй бочки взяли в \(2\) раза больше бензина, чем из первой, в первой оказалось на \(5\) л больше, чем во второй. Сколько литров бензина взяли из каждой бочки? 
		
		\item В летние каникулы я проехал на поезде на \(120\) км больше, чем проплыл на теплоходе. Если бы я проехал на поезде в \(4\) раза больше, а на теплоходе проплыл в \(8\) раз больше, чем в действительности, то общий путь составил бы \(1200\) км. Сколько километров я проплыл на теплоходе?
		
		В парке \(20\%\) всех деревьев составляют березы, третью часть --- клены, дубов на \(18\) больше, чем кленов, а остальные \(94\) дерева --- липы. Сколько всего деревьев в этом парке? 
		
		\item Пассажирский поезд проходит расстояние между двумя городами за \(10\) ч, а товарный --- за \(12\) ч \(30\) мин. Товарный поезд идет со скоростью на \(28\) км/ч меньшей, чем пассажирский. Каково расстояние между городами?
	\end{listofex}
\end{consultation}
%END_FOLD