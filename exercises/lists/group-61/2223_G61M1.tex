%Группа 6-1 Модуль 1
\title{Занятие №1}
\begin{listofex}
	\item Вычислить:
	\begin{enumcols}[itemcolumns=2]
		\item \( 25\cdot(28\cdot105+7236:18)-(4247-1823):6\cdot25 \)
		\item \( 3124:(3\cdot504-4\cdot307)+10403:101 \)
	\end{enumcols}
	\item Лёва с Васей решили купить футбольный мяч. У Лёвы не хватило \( 200 \) рублей, чтобы его
	купить, а у Васи \( 300 \) рублей. Тогда они сложили свои деньги и купили мяч,
	причём \( 600 \) рублей у них осталось. Сколько стоил футбольный мяч?
	\item Найти значение выражения \( (a-b)\cdot(b+c) \), если \( a=247;\; b=189;\; c=127 \).
	\item Найти значение выражения \( a^2+2\cdot a\cdot b + b^2 \), если \( a=217 \) и \( b=83 \).
\end{listofex}
\newpage
\title{Занятие №2}
\begin{listofex}
	\item Туристы были в походе три дня. Во второй день они прошли 18 км, что на 5 км меньше,
	чем в первый день, а в третий день они прошли на 19 км меньше, чем за два предыдущих дня.
	Сколько километров прошли туристы за три дня?
	\item При ремонте шоссе длиной в 69 км в первый день отремонтировали 7 км, а в каждый из
	трех последующих дней ремонтировали на 3 км больше, чем в предыдущий. Во сколько раз
	оставшийся участок шоссе меньше отремонтированного?
	\item Периметр треугольника равен 63 см. Одна сторона равна 18 см, что на 7 см меньше второй стороны. Найдите третью сторону треугольника.
	\item На лугу пасся табун лошадей. У них ног на \( 27 \) больше, чем голов. Сколько лошадей паслось на лугу.
	\item Сошлись два пастуха Иван и Пётр. Иван говорит Петру: «Отдай мне одну овцу,
	тогда у меня будет ровно вдвое больше овец, чем у тебя.» А Пётр ему отвечает: «Нет!
	Лучше ты мне отдай одну овцу, тогда у нас овец будет поровну». Сколько же было овец у
	каждого?
	\item В двух комнатах было \( 45 \) человек. Из первой вышли \( 9 \), а из второй --- \( 14 \), и людей в комнатах стало поровну. Сколько человек было в комнатах сначала?
\end{listofex}
\newpage
\title{Домашняя работа №1}
\begin{listofex}
	\item Вычислить: \( (410+96)\cdot(1010-31248:62)-170\cdot1500 \)
	\item Фермер убрал урожай картофеля за три дня. В первый день он убрал 19 грядок, что на 6 грядок больше, чем в третий день, а во второй день он убрал на 12 грядок меньше, чем за первый и третий дни вместе. Сколько грядок картофеля убрал фермер за три дня?
	\item Периметр треугольника равен 61 см. Одна сторона равна 16 см, а вторая в два раза больше третьей. Найдите вторую и третью стороны треугольника.
	\item На лугу паслось стадо коров. У них ног на \( 54 \) больше, чем голов. Сколько коров паслось на лугу.
	\item В двух комнатах было \( 56 \) человек. Когда в первую зашли еще \( 12 \), а во вторую --- \( 8 \), то в комнатах людей стало поровну. Сколько человек было в комнатах сначала?
\end{listofex}
\newpage

\title{Занятие №3}
\begin{listofex}
	\item Разложить на простые множители:
	\begin{enumcols}[itemcolumns=7]
		\item \( 15 \)
		\item \( 24 \)
		\item \( 36 \)
		\item \( 50 \)
		\item \( 98 \)
		\item \( 164 \)
		\item \( 240 \)
	\end{enumcols}
	\item Найти все делители числа:
	\begin{enumcols}[itemcolumns=4]
		\item \( 40 \)
		\item \( 24 \)
		\item \( 200 \)
		\item \( 96 \)
	\end{enumcols}
	\item Найти наибольший общий делитель чисел:
	\begin{enumcols}[itemcolumns=4]
		\item \( 40 \) и \( 28 \)
		\item \( 24 \) и \( 36 \)
		\item \( 100 \) и \( 60 \)
		\item \( 75 \) и \( 25 \)
		\item \( 7 \) и \( 13 \)
		\item \( 1 \) и \( 15 \)
		\item \( 126 \) и \( 105 \)
		\item \( 70 \) и \( 245 \)
	\end{enumcols}
	\item Замените звездочки двумя одинаковыми цифрами так, чтобы:
	\begin{enumcols}[itemcolumns=1]
		\item число \( 8*3* \) делилось на \( 3 \)
		\item число \( *18* \) делилось на \( 9 \)
		\item число \( 11** \) делилось на \( 15 \)
	\end{enumcols}
	\item Найдите неизвестные цифры числа, если известно, что число делится на \( 6 \):
	\begin{enumcols}[itemcolumns=2]
		\item \( 354*7* \)
		\item \( *4567* \)
	\end{enumcols}
	\item Велосипедист и мотоциклист выехали одновременно из одного пункта в одном направлении. Скорость мотоциклиста \( 40 \) км/ч, а велосипедиста \( 12 \) км/ч. Какова скорость их удаления друг от друга? Через сколько часов расстояние между ними будет равно \( 56 \) км?
	\item Расстояние между городами \( A \) и \( B \) равно \( 720 \) км. Из \( A \) в \( B \) вышел скорый поезд со скоростью \( 80 \) км/ч. Через \( 2 \) ч навстречу ему из \( B \) в \( A \) вышел товарный поезд со скоростью \( 60 \) км/ч. Через сколько часов после выхода второго поезда они встретятся?
\end{listofex}
\newpage
\title{Занятие №4}
\begin{listofex}
	\item Разложить на простые множители:
	\begin{enumcols}[itemcolumns=5]
		\item \( 50 \)
		\item \( 44 \)
		\item \( 76 \)
		\item \( 420 \)
		\item \( 198 \)
	\end{enumcols}
	\item Найти все делители числа:
	\begin{enumcols}[itemcolumns=4]
		\item \( 55 \)
		\item \( 57 \)
		\item \( 102 \)
		\item \( 96 \)
	\end{enumcols}
	\item Найдите двузначное число, кратное 45 и делящееся на 6.
	\item Замените звездочки двумя одинаковыми цифрами так, чтобы:
	\begin{enumcols}[itemcolumns=1]
		\item число \( 5**\:5 \) делилось на \( 3 \)
		\item число \( *\:4*5 \) делилось на \( 9 \)
		\item число \( 7*2\:* \) делилось на \( 90 \)
	\end{enumcols}
	\item Вася принес в класс 93 конфеты и раздал поровну своим одноклассникам. Сколько в классе может быть человек?
	\item Из 12 офицеров и 20 солдат нужно сформировать одинаковые по составу группы для патрулирования. Сколько таких групп можно сделать?
	\item Не производя вычислений, докажите, что:
	\begin{enumcols}[itemcolumns=2]
		\item \( 224+32 \) делится на \( 2 \)
		\item \( 535-220 \) делится на \( 5 \)
		\item \( 13013-1326+130 \) делится на \( 13 \)
		\item \( 11\cdot56 \) делится на \( 11 \)
		\item \( 49\cdot48 \) делится на \( 7 \)
	\end{enumcols}
	\item Не производя вычислений, докажите, что \( 4556\cdot47+57\cdot507-47\cdot114 \) делится на \( 57 \).
	\item Не производя вычислений, докажите, что:
	\begin{enumcols}[itemcolumns=2]
		\item \( 35\cdot20 \) делится на \( 14 \)
		\item \( 5\cdot2^4 \) кратно \( 20 \)
	\end{enumcols}
	\item Поезд, двигаясь со скоростью \( 90 \) км/ч, проезжает мимо неподвижного наблюдателя за \( 7 \) секунд. Какова длина поезда?
\end{listofex}
\newpage
\title{Домашняя работа №2}
\begin{listofex}
	\item Разложить на простые множители:
	\begin{enumcols}[itemcolumns=5]
		\item \( 30 \)
		\item \( 68 \)
		\item \( 190 \)
		\item \( 121 \)
		\item \( 520 \)
	\end{enumcols}
	\item Найти все делители числа:
	\begin{enumcols}[itemcolumns=4]
		\item \( 65 \)
		\item \( 100 \)
		\item \( 75 \)
		\item \( 105 \)
	\end{enumcols}
	\item Замените звездочки двумя одинаковыми цифрами так, чтобы:
	\begin{enumcols}[itemcolumns=1]
		\item число \( 2**\:2 \) делилось на \( 3 \)
		\item число \( *\:6*3 \) делилось на \( 9 \)
		\item число \( 4*2\:* \) делилось на \( 30 \)
	\end{enumcols}
	\item Найдите двузначное число, кратное 36 и не делящееся на 8.
	\item Для контрольной работы было приготовлено 87 листов бумаги, которые поровну раздали ученикам класса. Сколько учеников в классе?
	\item Из 20 конфет и 16 шоколадок нужно сделать одинаковые наборы. Сколько таких наборов можно сделать?
	\item Не производя вычислений, докажите, что:
	\begin{enumcols}[itemcolumns=2]
		\item \( 648+24 \) делится на \( 2 \)
		\item \( 1245-339 \) делится на \( 3 \)
		\item \( 11088+1122-77 \) делится на \( 11 \)
		\item \( 7\cdot87 \) делится на \( 7 \)
		\item \( 45\cdot13 \) делится на \( 5 \)
	\end{enumcols}
	\item Не производя вычислений, докажите, что \( 39\cdot737+39\cdot281-39\cdot296 \) делится на \( 13 \).
	\item Не производя вычислений, докажите, что:
	\begin{enumcols}[itemcolumns=2]
		\item \( 63\cdot24 \) делится на \( 21 \)
		\item \( 34\cdot33 \) кратно \( 51 \)
		\item \( 2^2\cdot3\cdot5^3 \) кратно \( 50 \)
	\end{enumcols}
	\item Поезд, двигаясь со скоростью 108 км/ч, проезжает мимо неподвижного наблюдателя за 13 секунд. Какова длина поезда?
\end{listofex}
\newpage
\title{Занятие №5}
\begin{listofex}
	\item Разделить число:
	\begin{enumcols}[itemcolumns=2]
		\item \( 12 \) в отношении \( 1:3 \)
		\item \( 900 \) в отношении \( 5:4 \)
		\item \( 30 \) в отношении \( 1:2:3 \)
	\end{enumcols}
	\item Первая машинистка печатает \( 10 \) страниц в час, а вторая --- \( 8 \) страниц в час. Как разделить между ними рукопись в \( 90 \) страниц, чтобы они закончили работу одновременно?
	\item Скоро велосипедиста в \( 5 \) раз больше скорости пешехода. Однажды они отправились одновременно навстречу друг другу из пунктов, расстояние между которыми \( 30 \) км. Какой путь проедет велосипедист до момента встречи с пешеходом?
	\item В \( 900 \) г воды растворили \( 100 \) г соли. Найдите отношение соли и получившегося раствора? Отношение воды и получившегося раствора?
	\item Автомобиль проехал \( 75 \) км из запланированных \( 300 \) км. Объясните, что означают следующие отношения:
	\begin{enumcols}[itemcolumns=4]
		\item \( 75:300 \)
		\item \( 225:300 \)
		\item \( 225:75 \)
		\item \( 300:75 \)
	\end{enumcols}
	\item Найдите отношения:
	\begin{enumcols}[itemcolumns=3]
		\item 3 дм к 2 см
		\item 3 м к 5 см
		\item 2 ч 20 мин к 40 мин
		\item 9 кг 500 г к 5 ц
		\item 3 мм\( ^3 \) к 2 см\( ^3 \)
	\end{enumcols}
	\item Разделить число \( 125 \) на такие \( 4 \) части, чтобы первая часть относилась ко второй, как \( 2:3 \), вторая к третьей, как \( 3:5 \), а третья к четвертой, как \( 5:6 \).
	\item Разделить число \( 250 \) на такие \( 4 \) части, чтобы первая часть относилась ко второй, как \( 2:3 \), вторая к третьей, как \( 4:5 \), а третья к четвертой, как \( 6:11 \).
	\item Решите пропорцию:
	\begin{enumcols}[itemcolumns=2]
		\item \( \dfrac{5}{7}=\dfrac{x}{21} \)
		\item \( \dfrac{30}{57}=\dfrac{20}{x} \)
	\end{enumcols}
	\item Пользуясь основным свойством пропорции, восстановите пропуски, чтобы получить верные отношения:
	
	\[ \dfrac{160}{180}=\dfrac{16}{*}=\dfrac{*}{90}=\dfrac{40}{*} \]
\end{listofex}
%\newpage
%\title{Занятие №6}
%\begin{listofex}
%	\item 1
%	
%\end{listofex}
\newpage
\title{Занятие №7}
\begin{listofex}
	\item Сократить дробь:
	\begin{enumcols}[itemcolumns=4]
		\item \( \dfrac{4}{12} \)
		\item \( \dfrac{24}{18} \)
		\item \( \dfrac{60}{50} \)
		\item \( \dfrac{34}{51} \)
		\item \( \dfrac{25}{75} \)
		\item \( \dfrac{125}{375} \)
		\item \( \dfrac{120}{60} \)
		\item \( \dfrac{244}{448} \)
	\end{enumcols}
	\item Найти:
	\begin{enumcols}[itemcolumns=3]
		\item \( \dfrac{2}{3} \) от \( 18 \)
		\item \( \dfrac{1}{4} \) от \( 28 \)
		\item \( \dfrac{3}{13} \) от \( 39 \)
		\item \( \dfrac{35}{37} \) от \( 114 \)
		\item \( \dfrac{4}{10} \) от \( 70 \)
		\item \( \dfrac{18}{91} \) от \( 273 \)
	\end{enumcols}
	\item Купили кусок ткани длиной \( 25 \) м \( 50 \) см и из \( 1/5 \) куска сшили платье. Сколько ткани ушло на платье?
	\item Купили \( 5 \) кг \( 600 \) г сахара и израсходовали на варенье \( 7/8 \) всего сахара. Сколько сахара пошло на варенье? Сколько сахара осталось?
	\item Сколько градусов составляют \( 4/15 \) прямого угла?
	\item \( 5/8 \) учеников класса составляют девочки. Сколько девочек в классе, если в нем 32
	ученика?
	\item Три бригады изготовили 6800 деталей. Первая бригада изготовила  \( 5/17 \) всего количества деталей, вторая --- \( 7/25 \) всего количества деталей. Сколько деталей изготовила третья бригада?
	\item \( 3/5 \) от числа \( 12 \) составляет \( 1/4 \) неизвестного числа. Найдите это число.
	\item Какую часть от целого составляет треть от половины этого целого?
	\item Найти целое, если:
	\begin{enumcols}[itemcolumns=4]
		\item \( \dfrac{1}{4} \) составляет \( 4 \)
		\item \( \dfrac{1}{3} \) составляет \( 6 \)
		\item \( \dfrac{7}{8} \) составляет \( 35 \)
		\item \( \dfrac{34}{41} \) составляет \( 340 \)
	\end{enumcols}
\end{listofex}
\newpage
\title{Домашняя работа №3}
\begin{listofex}
	\item Найти:
	\begin{enumcols}[itemcolumns=3]
		\item \( \dfrac{2}{3} \) от \( 15 \)
		\item \( \dfrac{2}{11} \) от \( 77 \)
		\item \( \dfrac{3}{8} \) от \( 120 \)
		\item \( \dfrac{31}{100} \) от \( 700 \)
		\item \( \dfrac{9}{19} \) от \( 95 \)
		\item \( \dfrac{3}{11} \) от \( 594 \)
	\end{enumcols}
	\item На базу в Антарктиду доставили \( 22 \) собаки. Из \( 5/11 \)всех собак составили упряжку, на
	которой отправились в поход. Сколько собак не вошло в упряжку?
\end{listofex}
\newpage
\title{Проверочная работа}
\begin{listofex}
	\item Сократить дробь:
	\begin{enumcols}[itemcolumns=2]
		\item \( \dfrac{55}{77} \)
		\item \( \dfrac{224}{84} \)
	\end{enumcols}
	\item Разложить на простые множители:
	\begin{enumcols}[itemcolumns=2]
		\item \( 60 \)
		\item \( 216 \)
	\end{enumcols}
	\item Не производя вычислений, докажите, что:
	\begin{enumcols}[itemcolumns=1]
		\item \( 200+46 \) делится на \( 2 \)
		\item \( 12012-1200+24 \) делится на \( 12 \)
		\item \( 11\cdot56 \) делится на \( 8 \)
	\end{enumcols}
	\item Найти:
	\begin{enumcols}[itemcolumns=2]
		\item \( \dfrac{2}{7} \) от \( 42 \)
		\item \( \dfrac{11}{101} \) от \( 707 \)
	\end{enumcols}
	\item От дыни массой \( 2 \) кг \( 400 \) г отрезали \( 1/5 \) и \( 1/6 \). Чему равна масса каждого отрезанного куска?
	\item Решите пропорцию:
	\begin{enumcols}[itemcolumns=2]
		\item \( \dfrac{9}{6}=\dfrac{x}{12} \)
		\item \( \dfrac{30}{57}=\dfrac{20}{x} \)
	\end{enumcols}
	\item Вычислить:
	\begin{enumcols}[itemcolumns=2]
		\item \( 7^2-6^2 \)
		\item \( 3^3\cdot(257-2^8) \)
	\end{enumcols}
	\item Сколько градусов составляют \( 7/30 \) прямого угла?
	\item Поезд, двигаясь со скоростью 108 км/ч, проезжает мимо неподвижного наблюдателя за 13 секунд. Какова длина поезда?
	\item Найдите неизвестные цифры числа:
	\begin{enumcols}[itemcolumns=1]
		\item \( 4*7\:* \), если число делится \( 6 \) (Подсказка: \( 6=2\cdot3 \))
		\item \( *\:4069\:* \), если число делится \( 15 \)
	\end{enumcols}
	\item На лугу пасется табун лошадей. У них ног на \( 33 \) больше, чем голов. Сколько лошадей паслось на лугу?
\end{listofex}
%\newpage
%\title{Домашняя работа №4}
%\begin{listofex}
%	\item 1
%	
%\end{listofex}