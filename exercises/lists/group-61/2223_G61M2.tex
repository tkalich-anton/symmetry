%Группа 101-1 Модуль 2
\title{Занятие №1}
\begin{listofex}
	\item Найдите:
	\begin{enumcols}[itemcolumns=3]
		\item \( \dfrac{2}{3} \) от \( 15 \)
		\item \( \dfrac{54}{101} \) от \( 505 \)
		\item \( \dfrac{17}{21} \) от \( 63 \)
	\end{enumcols}
	\item Найдите:
	\begin{enumcols}[itemcolumns=4]
		\item НОД\( (30;\;25) \)
		\item НОД\( (24;\;40) \)
		\item НОК\( (30;\;25) \)
		\item НОК\( (24;\;40) \)
	\end{enumcols}
	\item На свой день рождения Алиса купила \( 560 \) кг фруктов (на весь класс). Из них \( 4/7 \) составляют яблоки, а остальное --- апельсины. Сколько килограммов апельсинов купила Алиса. Какую часть от всех фруктов составляют апельсины?
	\item Сколько градусов составляет \( 4/15 \) прямого угла? Сколько градусов составляет \( 7/20 \) развернутого угла?
	\item Рабочий за \( 4 \) дня окончил некоторую работу, сделав в первый день \( 3/20 \) всей работы, во второй день \( 7/40 \), а в третий --- \( 3/8 \). Какую часть работы он сделал в четвертый день?
	\item В первый день турист прошел \( 42 \) км, что составляет \( 7/11 \) всего пути. Сколько километров осталось пройти туристу?
	\item Решить пропорцию:
	\begin{enumcols}[itemcolumns=4]
		\item \( \dfrac{5}{8}=\dfrac{15}{a} \)
		\item \( \dfrac{x}{7}=\dfrac{5}{14} \)
		\item \( 5:x=15:12 \)
		\item \( \dfrac{x}{7}=12:17 \)
	\end{enumcols}
\end{listofex}
\newpage
\title{Занятие №2}
\begin{listofex}
	\item Найдите:
	\begin{enumcols}[itemcolumns=3]
		\item \( \dfrac{2}{7} \) от \( 28 \)
		\item \( \dfrac{54}{60} \) от \( 420 \)
		\item \( \dfrac{17}{31} \) от \( 124 \)
	\end{enumcols}
	\item Вася прочитал \( 13/15 \) книги. Сколько страниц прочитал Вася, если в книге \( 195 \) страниц?
	\item Федя читает книжку, в которой \( 720 \) страниц. За первый день он прочитал \( 5/12 \) всей книжки, а за второй --- \( 7/18 \) всей книжки. Сколько страниц ему осталось прочитать?
	\item Автомобиль проехал \( 575 \) км, что составляет \( 23/25 \) расстояния между двумя городами. Найдите расстояние между городами.
	\item Решить пропорцию:
	\begin{enumcols}[itemcolumns=4]
		\item \( \dfrac{20}{8}=\dfrac{x}{6} \)
		\item \( \dfrac{x}{8}=\dfrac{9}{4} \)
		\item \( 2:x=5:25 \)
		\item \( \dfrac{x}{1}=2:7 \)
	\end{enumcols}
	\item Найдите:
	\begin{enumcols}[itemcolumns=4]
		\item НОД\( (45;\;60) \)
		\item НОД\( (64;\;96) \)
		\item НОД\( (120;\;260) \)
		\item НОД\( (30;\;150) \)
	\end{enumcols}
	\item Найдите:
	\begin{enumcols}[itemcolumns=4]
		\item НОК\( (16;\;24) \)
		\item НОК\( (45;\;60) \)
		\item НОК\( (27;\;36) \)
		\item НОК\( (125;\;75) \)
	\end{enumcols}
\end{listofex}
\newpage
\title{Домашняя работа №1}
\begin{listofex}
	\item Найдите:
	\begin{enumcols}[itemcolumns=3]
		\item \( \dfrac{5}{6} \) от \( 48 \)
		\item \( \dfrac{99}{100} \) от \( 900 \)
		\item \( \dfrac{31}{28} \) от \( 56 \)
	\end{enumcols}
	\item Длина дороги \( 84 \) км. За первый день бригада рабочих отремонтировала \( 5/12 \) дороги, а за второй день --- \( 5/14 \) дороги. Сколько километров осталось отремонтировать?
	\item Заказанная работа была выполнена в \( 3 \) дня. В первый день было сделано \( 4/15 \) всей работы, во второй --- \( 5/12 \) всей работы. Какая часть работы была сделана в третий день?
	\item Вася прочитал \( 195 \) страниц, что составляет \( 13/15 \) всей книги. Сколько страниц в книге?
	\item Решить пропорцию:
	\begin{enumcols}[itemcolumns=4]
		\item \( \dfrac{12}{8}=\dfrac{15}{a} \)
		\item \( \dfrac{x}{16}=\dfrac{5}{8} \)
		\item \( 35:x=14:6 \)
		\item \( \dfrac{x}{7}=12:17 \)
	\end{enumcols}
	\item Найдите:
	\begin{enumcols}[itemcolumns=4]
		\item НОД\( (48;\;72) \)
		\item НОД\( (36;\;42) \)
		\item НОК\( (48;\;72) \)
		\item НОК\( (36;\;42) \)
	\end{enumcols}
\end{listofex}
\newpage
\title{Занятие №3}
\begin{listofex}
	\item Произвести сложение дробей и, если возможно, упростить дробь:
	\begin{enumcols}[itemcolumns=3]
		\item \( \dfrac{11}{15}+\dfrac{1}{15} \)
		\item \( \dfrac{3}{10}+\dfrac{2}{10} \)
		\item \( \dfrac{17}{20}+\dfrac{3}{20} \)
		\item \( \dfrac{45}{64}+\dfrac{11}{64} \)
		\item \( \dfrac{3}{27}+\dfrac{5}{27}+\dfrac{2}{27} \)
		\item \( \dfrac{14}{38}+\dfrac{1}{38}+\dfrac{4}{38} \)
	\end{enumcols}
	\item Произвести вычитание дробей и, если возможно, упростить дробь:
	\begin{enumcols}[itemcolumns=3]
		\item \( \dfrac{17}{13}-\dfrac{4}{13} \)
		\item \( \dfrac{24}{30}-\dfrac{9}{30} \)
		\item \( \dfrac{15}{20}-\dfrac{10}{20} \)
		\item \( \dfrac{21}{32}-\dfrac{4}{32} \)
		\item \( \dfrac{112}{39}-\dfrac{13}{39}-\dfrac{21}{39} \)
		\item \( \dfrac{55}{24}-\dfrac{13}{24}-\dfrac{12}{24} \)
	\end{enumcols}
	\item Вычислить:
	\begin{enumcols}[itemcolumns=2]
		\item \( \dfrac{15}{20}+\dfrac{32}{20}-\dfrac{13}{20} \)
		\item \( \dfrac{15}{25}-\dfrac{7}{25}-\dfrac{2}{25}+\dfrac{21}{25} \)
		\item \( \left( \dfrac{1}{17}+\dfrac{13}{17} \right)-\dfrac{9}{17}+\dfrac{12}{17} \)
		\item \( \left( \dfrac{1}{12}+\dfrac{10}{12} \right)+\left( \dfrac{27}{12}-\dfrac{14}{12} \right) \)
	\end{enumcols}
	\item Вычислить:
	\begin{enumcols}[itemcolumns=2]
		\item \( 1+\dfrac{11}{12} \)
		\item \( \dfrac{33}{12}-2+\dfrac{5}{12} \)
		\item \( \dfrac{4}{5}+11-\dfrac{7}{5}+3 \)
		\item \( \left( \dfrac{3}{12}+4 \right)-\left( 1+\dfrac{2}{12} \right) \)
	\end{enumcols}
	\item Привести к общему знаменателю:
	\begin{enumcols}[itemcolumns=4]
		\item \( \dfrac{3}{6} \) и \( \dfrac{3}{4} \)
		\item \( \dfrac{15}{20} \) и \( \dfrac{7}{10} \)
		\item \( \dfrac{13}{15} \) и \( \dfrac{2}{1} \)
		\item \( \dfrac{5}{6} \) и \( \dfrac{1}{3} \)
	\end{enumcols}
	\item Сравнить:
	\begin{enumcols}[itemcolumns=4]
		\item \( \dfrac{3}{12} \) и \( \dfrac{5}{12} \)
		\item \( \dfrac{5}{18} \) и \( \dfrac{2}{9} \)
		\item \( \dfrac{25}{12} \) и \( 2 \)
		\item \( \dfrac{3}{16} \) и \( \dfrac{2}{24} \)
	\end{enumcols}
	\item Расположить числа в порядке возрастания:\quad\( \dfrac{4}{6},\;\dfrac{3}{12},\;1,\;\dfrac{35}{24},\;\dfrac{13}{6},\;2 \)
	\item Вычислить:
	\begin{enumcols}[itemcolumns=2]
		\item \( \dfrac{23}{25}+\left( 1-\dfrac{12}{25} \right)-1+\left( \dfrac{15}{25}-\dfrac{1}{5} \right) \)
		\item \( \left( \left( \dfrac{12}{20}-\dfrac{4}{10} \right) + 2\right)-\left( \dfrac{4}{5}-\dfrac{3}{20} \right)+\dfrac{9}{20} \)
	\end{enumcols}
\end{listofex}
\newpage
\title{Занятие №4}
\begin{listofex}
	\item Разложить на простые множители:
	\begin{enumcols}[itemcolumns=4]
		\item \( 54 \)
		\item \( 120 \)
		\item \( 264 \)
		\item \( 2000 \)
	\end{enumcols}
	\item Найдите:
	\begin{enumcols}[itemcolumns=4]
		\item НОД\( (38; 24) \)
		\item НОД\( (100; 90) \)
		\item НОК\( (120; 40) \)
		\item НОК\( (35; 56) \)
	\end{enumcols}
	\item Какие дроби называют правильными, а какие --- неправильными?
	\item Выберите неправильные дроби:
	\begin{enumcols}[itemcolumns=6]
		\item \( \dfrac{4}{7} \)
		\item \( \dfrac{12}{5} \)
		\item \( \dfrac{13}{15} \)
		\item \( \dfrac{1001}{1000} \)
		\item \( \dfrac{1001}{2000} \)
		\item \( \dfrac{12345}{13245} \)
	\end{enumcols}
	\item Что такое смешанное число?
	\item Представьте смешанное число в виде неправильной дроби:
	\begin{enumcols}[itemcolumns=4]
		\item \( 2\:\dfrac{3}{2} \)
		\item \( 7\:\dfrac{12}{15} \)
		\item \( 10\:\dfrac{10}{9} \)
		\item \( 9\:\dfrac{3}{5} \)
	\end{enumcols}
	\item Представьте неправильную дробь в виде смешанного числа:
	\begin{enumcols}[itemcolumns=6]
		\item \( \dfrac{12}{5} \)
		\item \( \dfrac{28}{9} \)
		\item \( \dfrac{112}{25} \)
		\item \( \dfrac{2002}{1000} \)
		\item \( \dfrac{145}{32} \)
		\item \( \dfrac{56}{3} \)
	\end{enumcols}
	\item Произвести сложение или вычитание дробей и, если возможно, упростить дробь:
	\begin{enumcols}[itemcolumns=4]
		\item \( \dfrac{12}{17}+\dfrac{3}{17} \)
		\item \( \dfrac{4}{9}+\dfrac{5}{9} \)
		\item \( \dfrac{15}{21}+\dfrac{16}{21} \)
		\item \( \dfrac{13}{50}+\dfrac{7}{50} \)
		\item \( \dfrac{15}{11}-\dfrac{4}{11} \)
		\item \( \dfrac{68}{30}-\dfrac{8}{30} \)
		\item \( \dfrac{112}{20}-\dfrac{2}{20} \)
		\item \( \dfrac{55}{42}-\dfrac{4}{42}-\dfrac{11}{42} \)
	\end{enumcols}
	\item Вычислить:
	\begin{enumcols}[itemcolumns=4]
		\item \( \dfrac{14}{31}+\dfrac{15}{31}-\dfrac{7}{31} \)
		\item \( \dfrac{4}{17}-\dfrac{3}{17}+\dfrac{48}{17}+\dfrac{2}{17} \)
		\item \( 4+\dfrac{5}{16} \)
		\item \( \dfrac{13}{4}-2+\dfrac{1}{4} \)
	\end{enumcols}
	\item Привести к общему знаменателю и сравнить:
	\begin{enumcols}[itemcolumns=3]
		\item \( \dfrac{5}{9} \) и \( \dfrac{1}{3} \)
		\item \( \dfrac{12}{18} \) и \( \dfrac{7}{12} \)
		\item \( \dfrac{14}{25} \) и \( \dfrac{19}{35} \)
	\end{enumcols}
	\item Расположить числа в порядке возрастания:\quad\( \dfrac{12}{5},\;\dfrac{24}{25},\;3,\;\dfrac{4}{5},\;\dfrac{3}{5},\;\dfrac{27}{50},\;1 \)
	\item Вычислить:\quad\( \dfrac{27}{15}-\left( 1-\dfrac{13}{15} \right)+3-\left( \dfrac{32}{15}-\dfrac{7}{5} \right) \)
\end{listofex}
\newpage
\title{Домашняя работа №2}
\begin{listofex}
	\item Найдите:
	\begin{enumcols}[itemcolumns=5]
		\item НОД\( (125; 75) \)
		\item НОД\( (96; 192) \)
	\end{enumcols}
	
\end{listofex}
%\newpage
%\title{Занятие №5}
%\begin{listofex}
%
%\end{listofex}
\newpage
\title{Занятие №6}
\begin{listofex}
	\item Представить дробь в виде неправильной:
	\begin{enumcols}[itemcolumns=5]
		\item \( 3\:\dfrac{5}{9} \)
		\item \( 6\:\dfrac{3}{8} \)
		\item \( 11\:\dfrac{1}{11} \)
		\item \( 100\:\dfrac{3}{10} \)
		\item \( 123\:\dfrac{123}{1000} \)
	\end{enumcols}
	\item Вычислить:
	\begin{enumcols}[itemcolumns=4]
		\item \( 4\:\dfrac{2}{5}+5\:\dfrac{2}{5} \)
		\item \( 4\:\dfrac{7}{11}+8\:\dfrac{9}{11} \)
		\item \( 3\:\dfrac{7}{8}+15\:\dfrac{1}{8} \)
		\item \( 19\:\dfrac{5}{57}+83\:\dfrac{55}{57} \)
		\item \( 1-\dfrac{1}{2} \)
		\item \( 5\:\dfrac{6}{7}-5\:\dfrac{1}{7} \)
		\item \( 7\:\dfrac{56}{75}-7 \)
		\item \( 34\:\dfrac{7}{9}-6\:\dfrac{7}{9} \)
	\end{enumcols}
	\item Вычислить:
	\begin{enumcols}[itemcolumns=3]
		\item \( 8\:\dfrac{1}{9}+8\:\dfrac{7}{9}-3\:\dfrac{5}{9} \)
		\item \( 17\:\dfrac{15}{17}+5\:\dfrac{13}{17}+19\:\dfrac{11}{17} \)
		\item \( 5\:\dfrac{3}{8}-2\:\dfrac{5}{8} \)
		\item \( 6\:\dfrac{1}{3}-5\:\dfrac{2}{3} \)
		\item \( 4\:\dfrac{7}{12}-1\:\dfrac{5}{12}+2\:\dfrac{11}{12} \)
		\item \( 12\:\dfrac{3}{7}-4\:\dfrac{5}{7}-5\:\dfrac{4}{7} \)
	\end{enumcols}
	\item Представить в виде десятичной дроби:
	\begin{enumcols}[itemcolumns=4]
		\item \( \dfrac{3}{10} \)
		\item \( \dfrac{11}{10} \)
		\item \( \dfrac{54}{10} \)
		\item \( \dfrac{137}{10} \)
		\item \( \dfrac{23}{1000} \)
		\item \( \dfrac{11}{20} \)
		\item \( \dfrac{8}{40} \)
		\item \( \dfrac{30}{50} \)
		\item \( \dfrac{9}{30} \)
		\item \( \dfrac{16}{200} \)
		\item \( \dfrac{15}{1500} \)
	\end{enumcols}
	\item Представить дроби так, чтобы в знаменателе была степень числа \( 10 \) и потом представить в виде десятичной дроби:
	\begin{enumcols}[itemcolumns=8]
		\item \( \dfrac{1}{2} \)
		\item \( \dfrac{2}{5} \)
		\item \( \dfrac{11}{20} \)
		\item \( \dfrac{1}{4} \)
		\item \( \dfrac{7}{4} \)
		\item \( \dfrac{17}{25} \)
		\item \( \dfrac{113}{50} \)
		\item \( \dfrac{24}{5} \)
	\end{enumcols}
	\item Представить десятичную дроби в виде обыкновенной:
	\begin{enumcols}[itemcolumns=4]
		\item \( 0,5 \)
		\item \( 0,23 \)
		\item \( 0,2 \)
		\item \( 1,2 \)
		\item \( 17,3 \)
		\item \( 0,017 \)
		\item \( 5,014 \)
		\item \( 0,00001 \)
	\end{enumcols}
	\item Расположить дроби в порядке возрастания:\( \dfrac{3}{12};\;\dfrac{2}{6};\;\dfrac{5}{4};\dfrac{2}{3};\;\dfrac{16}{24} \).
	\item Расположить дроби в порядке возрастания:\( 0,56;\;0,65\;1,23;\;1,18;\;1,33;\;0,123;\;5,6;\;0,506 \).
\end{listofex}
\newpage
\title{Проверочная работа}
\begin{listofex}
	\item Выберите неправильные дроби:
	\begin{enumcols}[itemcolumns=6]
		\item \( \dfrac{5}{7} \)
		\item \( \dfrac{8}{7} \)
		\item \( \dfrac{7}{7} \)
		\item \( \dfrac{99}{100} \)
		\item \( \dfrac{99}{10} \)
		\item \( \dfrac{53421}{54321} \)
	\end{enumcols}
	\item Найдите:
	\begin{enumcols}[itemcolumns=4]
		\item НОД\( (48; 64) \)
		\item НОД\( (200; 120) \)
		\item НОК\( (35; 140) \)
		\item НОК\( (15; 40) \)
	\end{enumcols}
	\item Представьте смешанное число в виде неправильной дроби:
	\begin{enumcols}[itemcolumns=4]
		\item \( 3\:\dfrac{1}{5} \)
		\item \( 7\:\dfrac{100}{101} \)
		\item \( 22\:\dfrac{22}{23} \)
		\item \( 1\:\dfrac{1}{1000} \)
	\end{enumcols}
	\item Привести к общему знаменателю:
	\begin{enumcols}[itemcolumns=4]
		\item \( \dfrac{7}{12} \) и \( \dfrac{1}{4} \)
		\item \( \dfrac{13}{30} \) и \( \dfrac{4}{15} \)
		\item \( \dfrac{11}{60} \) и \( \dfrac{11}{12} \)
		\item \( \dfrac{5}{12} \) и \( \dfrac{6}{18} \)
	\end{enumcols}
	\item Расположить числа в порядке возрастания:\quad\( \dfrac{4}{9},\;\dfrac{13}{36},\;1,\;\dfrac{25}{18},\;\dfrac{75}{36},\;2 \).
	\item Представить в виде десятичной дроби:
	\begin{enumcols}[itemcolumns=4]
		\item \( \dfrac{6}{10} \)
		\item \( \dfrac{17}{10} \)
		\item \( \dfrac{543}{10} \)
		\item \( \dfrac{137}{100} \)
		\item \( \dfrac{157}{1000} \)
		\item \( \dfrac{1}{4} \)
		\item \( \dfrac{3}{50} \)
		\item \( \dfrac{17}{25} \)
	\end{enumcols}
	\item Сократить (если это возможно) и представить в виде десятичной дроби:
	\begin{enumcols}[itemcolumns=3]
		\item \( \dfrac{6}{12} \)
		\item \( \dfrac{3}{75} \)
		\item \( \dfrac{10}{4} \)
		\item \( \dfrac{18}{30} \)
		\item \( \dfrac{26}{200} \)
		\item \( \dfrac{13}{1300} \)
	\end{enumcols}
	\item Сколько градусов составляет \( 29/30 \) прямого угла?
	\item Решить пропорцию:
	\begin{enumcols}[itemcolumns=4]
		\item \( \dfrac{7}{8}=\dfrac{49}{a} \)
		\item \( \dfrac{x}{12}=\dfrac{5}{6} \)
		\item \( 10:x=6:90 \)
		\item \( \dfrac{x}{45}=3:36 \)
	\end{enumcols}
\end{listofex}