%
%===============>>  ГРУППА 6-1 МОДУЛЬ 4  <<=============
%
\setmodule{4}
%
%===============>>  Занятие 1  <<===============
%
\begin{class}[number=1]
	\begin{definit}
		Чтобы поделить число на десятичную дробь, будем придерживаться следующего алгоритма:
		\begin{tasks}
			\task Делить на десятичную дробь нельзя!
			\task \textbf{Для деления на десятичную дробь нужно перенести в делимом и в
			делителе запятую на столько цифр вправо, сколько их после запятой в
			делителе.}
			\task \textbf{Если в десятичной части делимого (справа от запятой) знаков меньше, чем в делителе, то во время переноса запятой добавляем в делимом нули для каждого недостающего разряда.}
			\task \textbf{Выполнить деление десятичной дроби на натуральное число.}
			\task Еще раз подчеркнем, что избавляться от запятой нужно только в
			делителе, в то время как в делимом запятая может остаться!
		\end{tasks}
	\end{definit}
	\begin{listofex}
		\item Выполнить деление (в столбик):
		\begin{tasks}(4)
			\task \( 3,3:0,3 \)
			\task \( 2:0,5 \)
			\task \( 1:0,1 \)
			\task \( 5,1:0,17 \)
			\task \( 3:0,06 \)
			\task \( 25,2:0,4 \)
			\task \( 200,1:0,69 \)
			\task \( 6:0,0064 \)
		\end{tasks}
		\item Сравнить:
		\begin{tasks}(2)
			\task \( \dfrac{4}{11} \) и \( 0,4 \)
			\task \( \mfrac{2}{4}{11} \) и \( 2,36 \)
		\end{tasks}
	\end{listofex}
	\begin{definit}
		Чтобы найти число, которое составляет часть \( A \) от целого числа \( B \), нужно \( A\cdot B \)
	\end{definit}
	\begin{listofex}[resume]
		\item Найти часть от числа:
		\begin{tasks}(4)
			\task \( 0,3 \) от \( 15 \)
			\task \( 0,12 \) от \( 0,25 \)
			\task \( 0,8 \) от \( \dfrac{5}{64} \)
			\task \( \dfrac{1}{12} \) от \( 1,44 \)
		\end{tasks}
	\end{listofex}
	\begin{definit}
	Чтобы найти целое число, часть \( M \) от которого составляет число \( K \), нужно \( K:M \).
	\end{definit}
	\begin{listofex}[resume]
		\item Найти целое, если:
		\begin{tasks}(2)
			\task \( 0,1 \) целого составляет \( 5 \)
			\task \( 0,7 \) целого составляет \( 49 \)
			\task \( 6,25 \) целого составляет \( 225 \)
			\task \( 0,032 \) целого составляет \( 11 \)
		\end{tasks}
	\end{listofex}
	\begin{definit}
		Чтобы определить, какую часть от первого числа \( A \) составляет второе число \( B \), необходимо \( A:B \).
	\end{definit}
	\begin{listofex}[resume]
		\item Какую часть составляет первое число от второго? Ответ дайте в десятичной дроби:
		\begin{tasks}(4)
			\task \( 2 \) от \( 8 \)
			\task \( 3,6 \) от \( 12 \)
			\task \( 0,45 \) от \( 3,6 \)
			\task \( \mfrac{2}{2}{3} \) от \( \mfrac{5}{1}{3} \)
		\end{tasks}
	\end{listofex}
\end{class}
%
%===============>>  Занятие 2  <<===============
%
\begin{class}[number=2]
	\begin{listofex}
		\item Выполнить деление (в столбик):
		\begin{tasks}(4)
			\task \( 2:0,4 \)
			\task \( 0,48:0,08 \)
			\task \( 70:1,75 \)
			\task \( 23,53:2,6 \)
			\task \( 0,09:0,001 \)
			\task \( 49,56:0,007 \)
			\task \( 56,58:0,0082 \)
			\task \( 648,432:0,0058 \)
		\end{tasks}
		\item Сравнить:
		\begin{tasks}(2)
			\task \( \dfrac{1}{3} \) и \( 0,33 \)
			\task \( \mfrac{3}{13}{24} \) и \( 3,54167 \)
		\end{tasks}
		\item Найти часть от числа:
		\begin{tasks}(4)
			\task \( 0,7 \) от \( 20 \)
			\task \( 0,24 \) от \( 1,15 \)
			\task \( 0,1 \) от \( \dfrac{30}{17} \)
			\task \( \dfrac{1}{20} \) от \( 2,2 \)
		\end{tasks}
		\item Найти целое, если:
		\begin{tasks}(2)
			\task \( 0,5 \) целого составляет \( 14 \)
			\task \( 1,12 \) целого составляет \( 22,4 \)
			\task \( 1,25 \) целого составляет \( 6 \)
			\task \( 0,8 \) целого составляет \( 0,4 \)
		\end{tasks}
		\item Какую часть составляет первое число от второго? Ответ дайте в десятичной дроби:
		\begin{tasks}(4)
			\task \( 7 \) от \( 35 \)
			\task \( 1,89 \) от \( 12,6 \)
			\task \( 1,425 \) от \( 5,7 \)
			\task \( \mfrac{2}{1}{2} \) от \( \mfrac{8}{1}{3} \)
		\end{tasks}
		\item Найдите \( 0,73 \) числа, \( 0,21 \) которого равны \( 1,575 \).
		\item В двух ящиках было \( 38,25 \) кг гвоздей. Если из одного ящика переложить в другой \( 4,75 \) кг гвоздей, то в обоих ящиках гвоздей станет поровну. Сколько кг гвоздей было в каждом ящике?
	\end{listofex}
\end{class}
%
%===============>>  Домашняя работа 1  <<===============
%
\begin{homework}[number=1]
	\begin{listofex}
		\item Выполнить деление (в столбик):
		\begin{tasks}(4)
			\task \( 1,5:0,8 \)
			\task \( 2:0,2 \)
			\task \( 52,5:1,4 \)
			\task \( 3:0,06 \)
			\task \( 58,36:0,1 \)
			\task \( 85,69:41,8 \)
			\task \( 1,006:0,8 \)
			\task \( 397,5:0,53 \)
		\end{tasks}
		\item Сравнить:
		\begin{tasks}(2)
			\task \( 0,3 \) и \( \dfrac{1}{3} \)
			\task \( 1,07 \) и \( \mfrac{1}{7}{101} \)
		\end{tasks}
		\item Найти часть от числа:
		\begin{tasks}(4)
			\task \( 0,13 \) от \( 70 \)
			\task \( 1,2 \) от \( 2,25 \)
			\task \( 0,6 \) от \( \dfrac{5}{6} \)
			\task \( \dfrac{7}{55} \) от \( 5,5 \)
		\end{tasks}
		\item Найти целое, если:
		\begin{tasks}(2)
			\task \( 0,3 \) целого составляет \( 12 \)
			\task \( 2,7 \) целого составляет \( 54 \)
			\task \( 0,75 \) целого составляет \( 15 \)
			\task \( 0,032 \) целого составляет \( 11 \)
		\end{tasks}
		\item Какую часть составляет первое число от второго? Ответ дайте в десятичной дроби:
		\begin{tasks}(4)
			\task \( 4 \) от \( 100 \)
			\task \( 1,425 \) от \( 5,7 \)
			\task \( 1,95 \) от \( 1,3 \)
			\task \( \mfrac{3}{41}{100} \) от \( \mfrac{7}{3}{4} \)
		\end{tasks}
		\item Найдите \( 0,89 \) числа, \( 0,37 \) которого равны \( 425,5 \).
	\end{listofex}
\end{homework}
%
%===============>>  Занятие 3  <<===============
%
\begin{class}[number=3]
	\begin{listofex}
		\item Воздух состоит из азота (\( 78,09\% \) по объёму), кислорода (\( 20,95\% \)), углекислого газа (\( 0,03\% \)). Кроме этих газов, в воздухе содержатся ещё так называемые инертные газы: аргон, неон, гелий, криптон, радон. Каково процентное содержание инертных газов в воздухе?
		\item Оптовая цена товара на складе \( 5500 \) р. Торговая надбавка в магазине составляет \( 12\% \). Сколько стоит этот товар в магазине?
		\item Сплав состоит из \( 10\% \) олова, \( 35\% \) меди и \( 55\% \) свинца. Сколько каждого металла содержится в \( 2 \) кг сплава? Ответ дайте в граммах.
		\item В первый час работы продавец продал \( 40 \) кг яблок. Это составило \( 16\% \) от первоначального количества яблок. Сколько килограммов яблок было у продавца первоначально.
		\item Вкладчик положил \( 300000 \) рублей в банк под \( 11\% \) годовых. Какая сумма будет у него на счету через год?
		\item Какую сумму должен положить вкладчик в банк под \( 12\% \) годовых, чтобы через год на его счету оказалось \( 448000 \) рублей?
		\item До снижения цен товар стоил \( 2700 \) рублей, а после снижения цен стал стоить \( 2322 \) рубля. На сколько процентов была снижена цена товара?
		\item Найдите число, \( 17\% \) которого на \( 27 \) больше \( 14\% \) его.
		\item По определению процента найдите, сколько процентов составляет первое число от второго:
		\begin{tasks}(3)
			\task \( 40 \) от \( 400 \)
			\task \( 80 \) от \( 500 \)
			\task \( \dfrac{8}{9} \) от \( \mfrac{1}{13}{27} \)
		\end{tasks}
	\end{listofex}
\end{class}
%
%===============>>  Занятие 4  <<===============
\begin{class}[number=4]
	\begin{listofex}
		\item Оптовая цена товара на складе \( 2500 \) р. Торговая надбавка в магазине составляет \( 14\% \). Сколько стоит этот товар в магазине?
		\item Свежие грибы содержат \( 90\% \) влаги, сушенные --- \( 12\% \). Cколько сушенных грибов получится из 10 кг свежих?
		\item Сплав меди, цинка и олова содержит \(20\% \) меди и \(45\% \) цинка и весит \(120\) кг. Сколько весит олово в этом сплаве?
		\item В первый час работы продавец продал \( 15 \) кг яблок. Это составило \( 12,5\% \) от первоначального количества яблок. Сколько килограммов яблок было у продавца первоначально.
		\item Вкладчик положил \( 200 000 \) рублей в банк под \( 15\% \) годовых. Какая сумма будет у него на счету через год?
		\item Какую сумму должен положить вкладчик в банк под \( 6\% \) годовых, чтобы через год на его счету оказалось \( 265 000 \) рублей?
		\item До повышения цены товар стоил \( 2700 \) рублей, а после снижения цен стал стоить \( 3186 \) рубля. На сколько процентов была повышена цена товара?
		\item Антикварный магазин купил предмет за \(225\) р., а потом продал его, получив \(40 \%\) прибыли. Сколько рублей заплатил покупатель?
		\item Найдите число, \( 25\% \) которого на \( 60 \) больше \( 13\% \) его.
		\item По определению процента найдите, сколько процентов составляет первое число от второго:
		\begin{tasks}(4)
			\task \( 25 \) от \( 500 \)
			\task \( 80 \) от \( 640 \)
			\task \( \dfrac{2}{7} \) от \( \dfrac{28}{49} \)
			\task \( \dfrac{6}{9} \) от \( \mfrac{3}{10}{18} \)
		\end{tasks}
	\end{listofex}
\end{class}
%
%===============>>  Домашняя работа 2  <<===============
%
%BEGIN_FOLD
	\begin{homework}[number=2]
		\begin{listofex}
		\item Какую сумму должен положить вкладчик в банк под \( 11,5\% \) годовых, чтобы через год на его счету оказалось \( 724\,750 \) рублей?
		\item При перегонке нефти массы \( m \) получается \( \dfrac{1}{8}m \) керосина и \( \dfrac{5}{8}m \) мазута, остальное – потери при переработке. Сколько керосина и мазута получится из \(64\) т нефти?
		\item Кофе при обжаривании становится легче в \(1,5\) раза своего первоначального веса. Сколько килограммов свежего кофе нужно взять, чтобы получить \(42\) кг жареного?
		\item Предприниматель купил товар за \(505\) р., а потом продал его, получив \(16 \%\) прибыли. За какую сумму товар был продан?
		\item Найдите число, пятая часть которого на \( 28 \) больше \( 13\% \) искомого числа.
		\item По определению процента найдите, сколько процентов составляет первое число от второго:
		\begin{tasks}(4)
			\task \( 11 \) от \( 25 \)
			\task \( 80 \) от \( 165 \)
			\task \( \dfrac{1}{16} \) от \( \dfrac{5}{64} \)
			\task \( \dfrac{3}{19} \) от \( \mfrac{1}{26}{38} \)
		\end{tasks}
		\item Вычислить:\quad\( \left( 12,75 - \mfrac{6}{11}{12} + 14,8 - \mfrac{7}{2}{15}\right): \left( \mfrac{10}{2}{3} - \mfrac{3}{11}{12} \right) \)
		\item Записаны подряд \( 20 \) пятёрок: \( 5\;5\dots5\;5 \). Поставьте между некоторыми цифрами знак сложения, чтобы сумма оказалась равна \( 1000 \).
	\end{listofex}
\end{homework}
%END_FOLD

%
%===============>>  Занятие 5  <<===============
%
%BEGIN_FOLD
\begin{class}[number=5]
	\begin{listofex}
		\item Гриша прочитал \(280\) страниц, что составляет \(\dfrac{7}{15}\) книги, которую читает Гриша. Сколько страниц ему осталось прочитать?
		\item После того как брокер продал \( \dfrac{3}{8} \) акций своего клиента, у него осталось ещё \(1200\) акций. Сколько акций было у брокера первоначально?
		\item Сколько чистого спирта надо добавить к \(276\) г воды, чтобы получить восьмипроцентный раствор спирта?
		\item Книги русских писателей XIX века составляют \(\dfrac{1}{10}\) всех книг в библиотеке, из них \(\dfrac{3}{50}\) составляют книги Льва Николаевича Толстого. Книг Льва Николаевича Толстого в библиотеке \(255\) штук. Сколько всего книг в библиотеке?
		\item В библиотеке \(20000\) книг, из них \(1500\) книг – это словари для перевода с одного языка на другой, из них \(75\) книг --- это англо-русские и русско-английские словари. Ответьте на три вопроса:
		\begin{tasks}(1)
			\task какую долю от всех книг в библиотеке составляют словари?
			\task какую долю от всех словарей составляют англо-русские и русско-английские словари?
			\task какую долю от всех книг в библиотеке составляют англо-русские и русско-английские словари?
		\end{tasks}
		\item Цена на товар повысилась на \(\dfrac{2}{5}\) и составила \(2945\) рубля. Найдите первоначальную цену товара.
		\item Выразите в часах и результат запишите десятичной дробью:
		\begin{tasks}(2)
			\task \(1\) час \(30\) минут
			\task \(2\) часа \(15\) минут
			\task \(3\) часа \(12\) минут
			\task \(6\) минут
		\end{tasks}
		\item Сумма двух чисел равна \(2,4\), а их разность равна \(1,63\). Найдите эти числа.
		\item В двух ящиках было \(38,25\) кг гвоздей. Если из одного ящика переложить в другой \(4,75\) кг гвоздей, то в обоих ящиках гвоздей станет поровну. Сколько кг гвоздей было в каждом ящике?
		\item В двух коробках \(7,8\) кг конфет. Когда из одной коробки взяли \(1,25\) кг конфет, то в обеих коробках конфет стало поровну. Сколько конфет было в каждой коробке?
		\item Длина первого отрезка в \(1,4\) раза больше длины второго, а длина третьего – на \(6\) см больше длины второго. Найдите длину каждого отрезка, если сумма их длин \(125\) см.
		\item Вычислить: 
		\[\dfrac{1}{2} \cdot 28,5 + \mfrac{28}{1}{2} \cdot \dfrac{1}{3} - \mfrac{28}{1}{2} \cdot 0,25 + 0,2 \cdot \mfrac{28}{1}{2} + \mfrac{1}{13}{60} \cdot \mfrac{28}{1}{2}\]
		Можно ли посчитать этот пример рациональным способом?
	\end{listofex}
\end{class}
%END_FOLD

%
%===============>>  Занятие 6  <<===============
%
%BEGIN_FOLD
\begin{class}[number=6]
	\begin{listofex}
		\item Длина дороги \(84\) км. За первый день бригада рабочих отремонтировала \(\dfrac{5}{12}\) дороги, а за второй день --- \(\dfrac{5}{14}\) дороги. Сколько километров осталось отремонтировать?
		\item Сколько градусов составляет \(\dfrac{4}{15}\) прямого угла? Сколько градусов составляет \(\dfrac{7}{20}\) развёрнутого угла?
		\item Представьте число \(5\) в виде суммы трех слагаемых так, чтобы первое слагаемое было вдвое больше второго и на \(\dfrac{1}{3}\) меньше третьего.
		\begin{tasks}(1)
			\task Который сейчас час, если оставшаяся часть суток в \(\mfrac{1}{2}{5}\) раза больше истекшей?
			\task Который сейчас час, если оставшаяся часть суток в \(\mfrac{6}{1}{2}\) раза меньше истекшей?
		\end{tasks}
		\item В коробку помещается вдвое меньше яблок, чем в корзину. Сколько яблок помещается в корзину, если в коробке и корзине \(7,2\) кг яблок?
		\item Бревно укоротили сначала на \(0,3\) его длины, а потом на \(\dfrac{2}{5}\)  остатка, после чего длина оставшейся части стала равна \(2,1\) м. Сколько метров отпилили от бревна второй раз?
		\item Вася сначала истратил \(0,7\) своих денег, а потом --- \(\dfrac{2}{3}\) остатка, после чего у него осталось \(54,3\) р. Сколько денег Вася истратил во второй раз?
		\item Весёлый турист отправился на слёт, предполагая каждый день проходить треть всего пути, чтобы за \(3\) дня прибыть на место. В первый день он прошёл треть трети. Во второй день, устав, он прошёл не треть пути, а треть остатка. И в третий день он прошёл треть нового остатка. В результате ему осталось пройти ещё \(32\)] км. Сколько километров от дома до места слёта?
		\item Вычислить:
		\begin{tasks}(1)
			\task \(\left(\mfrac{2}{1}{4}+\mfrac{3}{2}{3}\right):\left(8,5-\mfrac{1}{2}{5}\right) \cdot 1,2\)
			\task \(\left(5,07:\dfrac{1}{20}-23,4: \dfrac{13}{50}\right) \cdot \dfrac{1}{4} + 0,074 \cdot \dfrac{1}{2}\)
		\end{tasks}
	\end{listofex}
\end{class}
%END_FOLD

%
%===============>>  Домашняя работа 3  <<===============
%
%BEGIN_FOLD
\begin{homework}[number=3]
	\begin{listofex}
		\item Какую часть от целого составляет треть от половины этого целого?
		\item Сплав состоит из \(125\) г золота и \(375\) г серебра. Какую долю в сплаве составляет золото и серебро?
		\item В первый день на мельнице смололи \(0,3\) привезенного зерна, во второй --- \(0,3\) остатка, а в третий --- оставшиеся \(10,78\) ц. Сколько зерна смололи на мельнице за три дня?
		\item Известно, \(\dfrac{3}{5}\) от числа \(12\) составляет \(\dfrac{1}{4}\) неизвестного числа. Найдите это число.
		\item Автомобиль проехал \(\dfrac{23}{25}\) пути. Сколько км проехал автомобиль, если весь путь равен \(575\) км?
		\item В классе  \(30\)  учащихся, отсутствуют четверо. Какая часть учащихся отсутствует?
		\item Было  \(600\)  рублей, в первом магазине потратили \(\dfrac{1}{4}\) этой суммы истратили, а потом ещё 450 рублей. Сколько денег истратили?
		\item Сумма двух чисел равна \(1,9\), а их разность равна \(1,27\). Найдите эти числа.
		\item Вычислить:\quad\( \dfrac{9}{10} \cdot \mfrac{1}{1}{14} : \mfrac{2}{4}{7} \cdot 24 - \mfrac{2}{4}{15} : \left( \mfrac{1}{1}{5} - \dfrac{2}{3} \right) \)
	\end{listofex}
\end{homework}
%END_FOLD

%
%===============>>  Занятие 7  <<===============
%
%BEGIN_FOLD
\begin{class}[number=7]
	\begin{listofex}
		\item В корзину помещается в \(1,4\) раза больше яблок, чем в коробку. Сколько яблок в корзине, если в коробке и корзине \( 7,2\) кг яблок?
		\item Бревно укоротили сначала на \(0,5\) его длины, а потом на \(\dfrac{1}{4}\) остатка, после чего длина оставшейся части стала равна \(4\) м. Сколько метров отпилили от бревна второй раз?
		\item Цена овощей была снижена с \(2\) рублей до \(1\) рубля \(7\) копеек за \(1\) кг. На сколько процентов снижена цена овощей по отношению к первоначальной цене?
		\item Вычислить:\quad\(\left(2,05:\dfrac{3}{5}-\dfrac{16}{40}\right) \cdot \dfrac{1}{3} + 0,02 \cdot \dfrac{1}{2}\)
	\end{listofex}
\end{class}
%END_FOLD

%
%===============>>  Провечная работа  <<===============
%
%BEGIN_FOLD
\begin{exam}
	\begin{listofex}
		\item Выразить:
		\begin{tasks}(1)
			\task \(0,1\) часа в минутах;
			\task \(0,3\) км в метрах;
			\task \(15\) гр в килограммах.
		\end{tasks}
		\item Вычислить:\quad\(\left(3,04:\dfrac{1}{30}-16,03: \dfrac{7}{20}\right) \cdot \dfrac{1}{5} + 0,072 \cdot \dfrac{1}{3}\)
		\item Какую сумму должен положить вкладчик в банк под \( 8\% \) годовых, чтобы через год на его счету оказалось \( 783 000 \) рублей?
		\item Цена на товар понизилась на \(15\%\) и составила \(2176\) рублей. Найдите первоначальную цену товара.
		\item При перегонке нефти массы \( m \) получается \(\dfrac{1}{8}m\) керосина и \(\dfrac{5}{8}m \)  мазута, остальное --- потери при переработке. При перегонке нефти получилось \(192\) т мазута. Сколько было нефти? И сколько из нее получилось керосина?
		\item Автомобиль проехал \(225\) км, что составляет \(0,92\) расстояния между двумя городами. Найдите расстояние между городами.
	\end{listofex}
\end{exam}
%END_FOLD