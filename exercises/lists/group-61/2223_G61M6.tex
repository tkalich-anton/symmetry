%
%===============>>  ГРУППА 6-1 МОДУЛЬ 6  <<=============
%
\setmodule{6}

%BEGIN_FOLD % ====>>_____ Занятие 1 _____<<====
\begin{class}[number=1]
	\begin{listofex}
		\item
		\begin{minipage}[t]{\bodywidth}
			Определите координаты точек:
		\end{minipage}
		%\hspace{0.02\linewidth}
		\begin{minipage}[c]{0.45\textwidth}
			\includegraphics[align=t, width=\linewidth]{\picpath/G61M6L1-1}
		\end{minipage}
	\end{listofex}
		\begin{definit}
			Чтобы вычесть из отрицательного числа отрицательное число, нужно сложить их так, будто они положительные, а перед суммой поставить минус.
		\end{definit}
	\begin{listofex}[resume]
		\item Вычислите:
		\begin{tasks}(4)
			\task \( -28-14 \)
			\task \( -144-56 \)
			\task \( -9-17 \)
			\task \( -3-18 \)
			\task \( -12-7 \)
			\task \( -15-8 \)
			\task \( -24-19 \)
			\task \( -8-3 \)
		\end{tasks}
	\end{listofex}
		\begin{definit}
			Для удобства, при сложении отрицательного числа с положительным, сумму можно записать как разность, поставив положительное число перед отрицательным. \\ Например, \(-12+25=25-12=13\).
		\end{definit}
	\begin{listofex}[resume]
		\item Вычислите:
		\begin{tasks}(4)
			\task \( -35 + 92 \)
			\task \( -7+14 \)
			\task \( -17+21 \)
			\task \( -5+65 \)
			\task \( -26+27 \)
			\task \( -8+12 \)
			\task \( -32+32 \)
			\task \( -80+124 \)
		\end{tasks}
	\end{listofex}
		\begin{definit}
			Если перед скобками стоит знак минус, то при их раскрытии знак внутри скобок меняется на противоположный. \\ Например: \(-(-a) = a \) или \( -(b) = -b \)
		\end{definit}
	\begin{listofex}[resume]
		\item Вычислите:
		\begin{tasks}(3)
			\task \( -35 - (-42) \)
			\task \( 8-(6) \)
			\task \( -(19)-(-21) \)
			\task \( -5-(-27) \)
			\task \( -(-42)+18 \)
			\task \( -(-17)-26 \)
			\task \( -(-29)+(-13) \)
			\task \( -(18)+(-12) \)
		\end{tasks}
	\end{listofex}
		\begin{definit}
		Чтобы из меньшего числа вычесть большее, необходимо из большего вычесть меньшее и перед результатом поставить знак минус. \\ Например: \( 17-82 = -(82-17)=-65 \)
		\end{definit}
	\begin{listofex}[resume]
		\item Вычислите:
		\begin{tasks}(4)
			\task \( 15-21 \)
			\task \( 17-66 \)
			\task \( 100-143 \)
			\task \( 42-69 \)
			\task \( 85-98 \)
			\task \( 117-162 \)
			\task \( 31-67 \)
			\task \( 71-143 \)
		\end{tasks}
		\item Вычислите:
		\begin{tasks}(2)
			\task \( -\mfrac{2}{3}{7}-\mfrac{3}{5}{14} \)
			\task \( -\mfrac{4}{8}{9}+\mfrac{6}{7}{15} \)
			\task \( -3,3+\mfrac{4}{8}{7} \)
			\task \( -5,6+\mfrac{5}{3}{5} \)
			\task \( 9,9-\mfrac{11}{4}{15} \)
			\task \( -\mfrac{8}{4}{23}+\mfrac{9}{5}{46}+\left(-\mfrac{18}{13}{69}\right) \)
			\task \( -2,3 + \left(-\mfrac{1}{3}{5}\right) + 9,09 \)
			\task \( -\mfrac{6}{8}{12} + \left(-\dfrac{13}{24}\right) - (-7,11) \)
		\end{tasks}
	\end{listofex}
\end{class}
%END_FOLD

%BEGIN_FOLD % ====>>_____ Занятие 2 _____<<====
\begin{class}[number=2]
	\begin{listofex}
		\item
		\begin{minipage}[t]{\bodywidth}
			Определите координаты всех точек:
		\end{minipage}
		%\hspace{0.02\linewidth}
		\begin{minipage}[c]{0.45\textwidth}
			\includegraphics[align=t, width=\linewidth]{\picpath/G61M6L2-1}
		\end{minipage}
	\end{listofex}
	\begin{definit}
		Рациональные числа, как и целые, можно сравнивать с помощью числовой оси --- \textbf{чем правее расположено число, тем оно больше.}
	\end{definit}
	\begin{listofex}[resume]
		\item С помощью этого же рисунка сравните координаты точек:
		\begin{tasks}(3)
			\task \( A \) и \( D \)
			\task \( B \) и \( C \)
			\task \( B \) и \( E \)
			\task \( O \) и \( B \)
			\task \( E \) и \( C \)
			\task \( O \) и \( C \)
		\end{tasks}
		\item Сравните рациональные числа:
		\begin{tasks}(3)
			\task \( -5 \) и \( -3 \)
			\task \( -7 \) и \( -25 \)
			\task \( 4 \) и \( -1 \)
			\task \( -99 \) и \( -101 \)
			\task \( -10,5 \) и \( -1,5 \)
			\task \( -8,9 \) и \( -9,2 \)
			\task \( -55,3 \) и \( -55,4 \)
			\task \( -0,2 \) и \( 0 \)
			\task \( \dfrac{1}{3} \) и \( -\dfrac{1}{4} \)
			\task \(- \dfrac{4}{7} \) и \( \dfrac{3}{4} \)
			\task \(- \mfrac{1}{3}{15} \) и \( -\dfrac{3}{15} \)
			\task \( -\mfrac{2}{3}{4} \) и \( -\dfrac{2}{5} \)
		\end{tasks}
		\item Вычислите:
		\begin{tasks}(3)
			\task \( -\dfrac{5}{9}-\dfrac{5}{9} \)
			\task \( -\dfrac{14}{15}+\dfrac{5}{6} \)
			\task \( -\dfrac{1}{2}-\dfrac{1}{3} \)
			\task \( \mfrac{3}{5}{6}-4 \)
			\task \( -1,18+\dfrac{33}{50} \)
			\task \( -0,8+4 \)
			\task \( 1,7-7,3 \)
			\task \( -2,4+3,6 \)
			\task \( -5,7-2,15 \)
		\end{tasks}
	\end{listofex}
\end{class}
%END_FOLD

%BEGIN_FOLD % ====>>_ Домашняя работа 1 _<<====
\begin{homework}[number=1]
	\begin{listofex}
		\item Вычислите:
		\begin{tasks}(2)
			\task \( -11 + 21 \)
			\task \( -17+25 \)
			\task \( -37+65 \)
			\task \( -51-14 \)
			\task \( -26-37 \)
			\task \( -83-22 \)
			\task \( -\dfrac{1}{3}+\dfrac{1}{2} \)
			\task \( -\mfrac{2}{1}{6}+11 \)
			\task \( -1,5-\dfrac{3}{7} \)
			\task \( -\mfrac{3}{5}{21}+\mfrac{3}{17}{42}+\left(-\mfrac{18}{45}{66}\right) \)
			\task \( -15,2 + \left(-\mfrac{2}{17}{25}\right) + 12 \)
			\task \( -\mfrac{4}{2}{3} + \left(-\dfrac{22}{24}\right) - (-6) \)
		\end{tasks}
		\item Сравните рациональные числа:
		\begin{tasks}(3)
			\task \( -\dfrac{1}{2} \) и \( -\dfrac{3}{4} \)
			\task \( -\dfrac{5}{11} \) и \( -\dfrac{6}{33} \)
			\task \( -\mfrac{1}{5}{7} \) и \( -\dfrac{13}{8} \)
			\task \( -11,5 \) и \( -17,22 \)
			\task \( -8,85 \) и \( -8,02 \)
			\task \( -14,4 \) и \( -14,02 \)
		\end{tasks}
	\end{listofex}
\end{homework}
%END_FOLD

%BEGIN_FOLD % ====>>_____ Занятие 3 _____<<====
\begin{class}[number=3]
	\begin{listofex}
		\item Занятие 3 
	\end{listofex}
\end{class}
%END_FOLD

%BEGIN_FOLD % ====>>_____ Занятие 4 _____<<====
\begin{class}[number=4]
	\begin{listofex}
		\item Занятие 4
	\end{listofex}
\end{class}
%END_FOLD

%BEGIN_FOLD % ====>>_ Домашняя работа 2 _<<====
\begin{homework}[number=2]
	\begin{listofex}
		\item Домашняя работа 2
	\end{listofex}
\end{homework}
%END_FOLD

%BEGIN_FOLD % ====>>_____ Занятие 5 _____<<====
\begin{class}[number=5]
	\begin{listofex}
		\item Занятие 5
	\end{listofex}
\end{class}
%END_FOLD

%BEGIN_FOLD % ====>>_____ Занятие 6 _____<<====
\begin{class}[number=6]
	\begin{listofex}
		\item Занятие 6
	\end{listofex}
\end{class}
%END_FOLD

%BEGIN_FOLD % ====>>_ Домашняя работа 3 _<<====
\begin{homework}[number=3]
	\begin{listofex}
		\item Домашняя работа 3
	\end{listofex}
\end{homework}
%END_FOLD

%BEGIN_FOLD % ====>>_____ Занятие 7 _____<<====
\begin{class}[number=7]
	\title{Подготовка к проверочной}
	\begin{listofex}
		\item Занятие 7
	\end{listofex}
\end{class}
%END_FOLD

%BEGIN_FOLD % ====>>_ Проверочная работа _<<====
\begin{exam}
	\begin{listofex}
		\item Проверочная
	\end{listofex}
\end{exam}
%END_FOLD\underline{%BEGIN_FOLD % ====>>_____ Занятие 1 _____<<====
\begin{class}[number=1]
	\begin{listofex}
		\item Занятие 1
	\end{listofex}
\end{class}
%END_FOLD

%BEGIN_FOLD % ====>>_____ Занятие 2 _____<<====
\begin{class}[number=2]
	\begin{listofex}
		\item Занятие 2
	\end{listofex}
\end{class}
%END_FOLD

%BEGIN_FOLD % ====>>_ Домашняя работа 1 _<<====
\begin{homework}[number=1]
	\begin{listofex}
		\item Домашняя работа 1
	\end{listofex}
\end{homework}
%END_FOLD

%BEGIN_FOLD % ====>>_____ Занятие 3 _____<<====
\begin{class}[number=3]
	\begin{listofex}
		\item Занятие 3 
	\end{listofex}
\end{class}
%END_FOLD

%BEGIN_FOLD % ====>>_____ Занятие 4 _____<<====
\begin{class}[number=4]
	\begin{listofex}
		\item Занятие 4
	\end{listofex}
\end{class}
%END_FOLD

%BEGIN_FOLD % ====>>_ Домашняя работа 2 _<<====
\begin{homework}[number=2]
	\begin{listofex}
		\item Домашняя работа 2
	\end{listofex}
\end{homework}
%END_FOLD

%BEGIN_FOLD % ====>>_____ Занятие 5 _____<<====
\begin{class}[number=5]
	\begin{listofex}
		\item Занятие 5
	\end{listofex}
	\end{class}
	%END_FOLD
	
	%BEGIN_FOLD % ====>>_____ Занятие 6 _____<<====
	\begin{class}[number=6]
		\begin{listofex}
			\item Занятие 6
		\end{listofex}
	\end{class}
	%END_FOLD
	
	%BEGIN_FOLD % ====>>_ Домашняя работа 3 _<<====
	\begin{homework}[number=3]
		\begin{listofex}
			\item Домашняя работа 3
		\end{listofex}
	\end{homework}
	%END_FOLD
	
	%BEGIN_FOLD % ====>>_____ Занятие 7 _____<<====
	\begin{class}[number=7]
		\title{Подготовка к проверочной}
		\begin{listofex}
			\item Занятие 7
		\end{listofex}
	\end{class}
	%END_FOLD
	
	%BEGIN_FOLD % ====>>_ Проверочная работа _<<====
	\begin{exam}
		\begin{listofex}
			\item Проверочная
		\end{listofex}
	\end{exam}
	%END_FOLD}