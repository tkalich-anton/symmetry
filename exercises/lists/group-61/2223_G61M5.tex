%
%===============>>  ГРУППА 6-1 МОДУЛЬ 5  <<=============
%
\setmodule{5}

%BEGIN_FOLD % ====>>_____ Занятие 1 _____<<====
\begin{class}[number=1]
	\begin{listofex}
		\item Цена на товар снизилась на \(\dfrac{3}{7}\) и составила \(931\) рубля. Найдите первоначальную цену товара.
		\item В десятой части початка ветвистой кукурузы \(93\) зерна. Сколько зёрен в целом початке?
		\item Перед тем как брокер продал \( \dfrac{1}{6} \) акций своего клиента, у него было \(1800\) акций. Сколько акций у брокера теперь?
		\item В классе  \(27\)  учащихся, отсутствуют восемь человек. Какая часть учащихся присутствует?
		\item У нас было  \(325\)  рублей, в первом магазине потратили \(\dfrac{3}{5}\) этой суммы истратили, а потом еще \(\dfrac{7}{10}\) от остатка. Сколько денег у нас осталось?
		\item Два поезда идут навстречу друг другу. Один прошел \(\dfrac{2}{5}\) всего пути, а другой – половину. Сколько километров им осталось идти до встречи, если между ними было \(200\) км?
		\item Сад прямоугольной формы хотят обнести забором. Длина  сада \(800\) м, а ширина составляет \(\dfrac{5}{8}\) длины. Какой длины должен быть весь забор?
		\item В четырех домах \(3672\) жителя. В одном доме \(\dfrac{1}{3}\) всех жителей, во втором --- в \(2\) раза меньше, чем в первом, а остальные живут поровну в третьем и четвёртом домах. По скольку жителей живёт в третьем и четвёртом домах?
	\end{listofex}
\end{class}
%END_FOLD

%BEGIN_FOLD % ====>>_____ Занятие 2 _____<<====
\begin{class}[number=2]
	\begin{listofex}
		%1
		\item Гриша прочитал \(280\) страниц, что составляет \(\dfrac{7}{15}\) книги, которую читает Гриша. Сколько страниц ему осталось прочитать?
		%2
		\item Бронза --- это сплав меди с оловом. Монетная бронза содержит \(3,9 \% \) олова, оружейная бронза --- \(10,2 \%\) олова, а колокольная бронза содержит \(22,4 \% \) олова. В каждом случае определите, сколько процентов меди содержится в сплаве.
		%3
		\item Цена товара уменьшилась на \(35 \%\). Сколько процентов составляет новая цена от старой?
		%4
		\item Вкладчик снял в банке \(234\) тысячи рублей, что составило \(36 \%\) вклада. Определите первоначальную сумму вклада.
		%5
		\item Палку укоротили сначала на \(0,2\) его длины, а потом на \(\dfrac{1}{4}\)  остатка, после чего длина оставшейся части стала равна \(3\) м. Сколько метров отпилили от бревна второй раз?
		%6
		\item Сплав состоит из \(10 \%\) олова, \(35\%\) меди и \(55\%\) свинца. Сколько каждого металла содержится в \(2\) кг сплава. Ответ дайте в граммах.
		%7
		\item Стоимость покупки с учётом четырёхпроцентной скидки по дисконтной карте составила \(1152\) рубля. Сколько рублей пришлось бы заплатить за покупку при отсутствии дисконтной карты?
		%8
		\item Найдите число, если известно, что после прибавления к нему \(12\%\) его получится \(420\).
	\end{listofex}
\end{class}
%END_FOLD

%BEGIN_FOLD % ====>>_____ Занятие 3 _____<<====
\begin{class}[number=3]
	\begin{listofex}
		\item Решить уравнение:
		\begin{tasks}(2)
			\task \( 16x+(5x-3)=4 \)
			\task \( ((3x-1)+4):2=12\)
		\end{tasks}
		\item Решить пропорцию:
		\begin{tasks}(2)
			\task \( \dfrac{x}{12}=\dfrac{4}{9} \)
			\task \( 13:x=1:2\)
		\end{tasks}
		\item Цена на акцию сначала снизилась на \( 10 \% \), потом снизилась ещё на \(10 \% \), а потом увеличилась на \(20\%\). На сколько процентов изменилась цена акции по сравнению с первоначальной? Сколько стоит теперь акция, если первоначально она стоила \(5000\) рублей?
		\item Население города N в среднем увеличивается на \(2\%\) в год. На сколько процентов увеличится население за три года? Сколько жителей будет жить в городе через три года, если первоначально в городе \(900000\) человек?
		\item Вася сначала истратил \(0,7\) своих денег, а потом --- \(\dfrac{2}{3} \) остатка, после чего у него осталось \(54,3\) р. Сколько денег Вася истратил во второй раз?
		\item Мастер и его ученик должны были сделать некоторое количество деталей. По окончании работы выяснилось, что мастер сделал \(\dfrac{2}{3}\) всего задания и еще \(8\) деталей, а его ученик --- \(0,25\) того, что выполнил мастер. Сколько деталей сделали ученик и мастер?
	\end{listofex}
\end{class}
%END_FOLD

%BEGIN_FOLD % ====>>_____ Занятие 4 _____<<====
\begin{class}[number=4]
	\begin{listofex}
		\item Решить уравнение:
		\begin{tasks}(2)
			\task \( 2,75x+(0,25x-3)=12 \)
			\task \( ((2x-4)+3):5=7\)
		\end{tasks}
		\item Решить пропорцию:
		\begin{tasks}(2)
			\task \( \dfrac{x}{18}=\dfrac{4}{5} \)
			\task \( 9:x=3:27\)
		\end{tasks}
		\item Цена на акцию в течении дня поднялась на \(1\%\), потом упала на \(2\%\), а потом снова поднялась на \(3\%\). Уменьшилась или увеличилась цена акции по сравнению с первоначальной? На сколько процентов?
		\item В связи с поступлением новой коллекции одежды цена на старую коллекцию снизилась сначала на \(10\%\), а потом ещё на \(30\%\). На сколько процентов снизилась цена по сравнению с первоначальной? Сколько теперь стоит шляпа, которая раньше стоила \(800\) рублей?
		\item Бревно укоротили сначала на \(0,3\) его длины, а потом на \(\dfrac{2}{5}\) остатка, после чего длина оставшейся части стала равна \(2,1\) м. Сколько метров отпилили от бревна второй раз?
		\item Двое рабочих должны были сделать некоторое количество деталей. По окончании работы выяснилось, что первый рабочий сделал \(\dfrac{4}{5}\) всего задания и еще \(40\) деталей, а второй рабочий --- \(0,15\) того, что выполнил первый. Сколько деталей сделали двое рабочих?
	\end{listofex}
\end{class}
%END_FOLD

%BEGIN_FOLD % ====>>_____ Занятие 5 _____<<====
\begin{class}[number=5]
	\begin{listofex}
		\item Представьте число \(7\) в виде суммы трех слагаемых так, чтобы первое слагаемое было вдвое меньше второго и на \(\dfrac{1}{6}\) больше третьего.
		\item В двух ящиках было \(38,25\) кг гвоздей. Если из одного ящика переложить в другой \(4,75\) кг гвоздей, то в обоих ящиках гвоздей станет поровну. Сколько килограммов гвоздей было в каждом ящике?
		\item Решить уравнение: \( ((5x-1)+15):7=12\)
	\end{listofex}
	\begin{definit}
		Сумма внутренних углов в треугольнике равна \( 180\degree \).
	\end{definit}
	\begin{definit}
		\textbf{Внешний угол} --- угол между стороной треугольника и продолжением другой стороны. Внешний угол является смежным с одним из внутренних.
	\end{definit}
	\begin{listofex}
		\item В треугольнике \( ABC \) два угла равны \( 50\) и \( 70 \) градусов. Найдите третий угол.
		\item Один угол треугольника равен \( 26\degree \), а второй в три раза больше. Найдите третий угол.
		\item Один внутренний угол треугольника в два, а второй в три раза больше третьего, найдите все углы треугольника.
		\item Один внешний угол равен \( 40\degree \), а второй --- \( 100\degree \). Чему равны внутренние углы треугольника?
		\item Угол треугольника равен \( 30\degree \), второй угол в \( 3 \) раза больше первого. Чему равны внешние углы при каждой вершине? Чему равна сумма внешних углов?
		\item В прямоугольном треугольнике один угол равен \( 40 \) градусов. Найдите сумму наибольшего и наименьшего угла.
		\item В прямоугольном треугольнике один острый угол на \( 17 \) градусов больше другого. Найдите углы треугольника.
		\item В прямоугольном треугольнике два острых угла равны. Какая у них градусная мера?
	\end{listofex}
\end{class}
%END_FOLD

%BEGIN_FOLD % ====>>_____ Занятие 6 _____<<====
\begin{class}[number=6]
	\begin{listofex}
		\item Вера в первый день прочитала \( \dfrac{5}{9} \) книги, а во второй день на \( \dfrac{1}{6} \) меньше.
		Какую часть книги прочитала Вера во второй день?
		Успела ли она прочитать книгу за два дня?
		\item В школу привезли \(900\) новых учебников, из них учебники по математике составляли \(\dfrac{8}{25}\) всех книг, учебники по русскому языку \(\dfrac{33}{100}\) всех книг, а остальные книги были по литературе. Сколько привезли книг по литературе?
		\item В треугольнике \( ABC \) два угла равны \( 130\) и \( 15 \) градусов. Найдите третий угол.
		\item Один угол треугольника равен \( 31\degree \), а второй в четыре раза больше. Найдите третий угол треугольника.
		\item Один внешний угол равен \( 15\degree \), а второй --- \( 120\degree \). Чему равны внутренние углы треугольника?
		\item В прямоугольном треугольнике один острый угол равен \( 32 \) градуса. Найдите сумму наибольшего и наименьшего угла.
		\item В прямоугольном треугольнике один угол в два раза меньше другого. Какими могут быть эти углы?
		\item Для возведения стены длиной \( 18 \) м, толщиной \( 0,8 \) м и высотой \( 2,1 \) м требуется \( 16800 \) кирпичей. Какой высоты стену можно возвести при длине её \( 15 \) м, толщине \( 0,6 \) м, имея \( 6000 \) таких же кирпичей?
	\end{listofex}
\end{class}
%END_FOLD

%BEGIN_FOLD % ====>>_____ Занятие 7 _____<<====
\begin{class}[number=7]
	\begin{listofex}
		\item В двух коробках \(7,8\) кг конфет. Когда из одной коробки взяли \(1,25\) кг конфет, то в обеих коробках конфет стало поровну. Сколько конфет было в каждой коробке?
		\item В первый день продали \(\dfrac{1}{3}\), а во второй день \(\dfrac{1}{2}\) поступившего в магазин винограда. Какую часть винограда продали за два дня?
		\item Кладовщик выдал первому рабочему \(0,5\) всей имевшейся проволоки, а второму --- \(4\) метра, после чего у него осталось еще \(30\) м проволоки. Сколько проволоки было первоначально?
		\item Чашка, которая стоила \(90\) рублей, продаётся с \(10\)-процентной скидкой. Покупатель отдал кассиру \(1000\) рублей. Сколько рублей сдачи он должен получить, если он хочет купить максимально возможное количество чашек на эту сумму?
		\item Школьник решил \(40\) задач из учебника. Что составляет \(16 \%\) числа всех задач в книге. Сколько всего задач собрано в этом учебнике?
		\item Двое рабочих должны были сделать некоторое количество деталей. По окончании работы выяснилось, что первый рабочий сделал \(80\%\) всего задания и еще \(40\) деталей, а второй рабочий --- \(15\%\) того, что выполнил первый. Сколько деталей сделали двое рабочих?
		\item Решите уравнения:
		\begin{tasks}(2)
			\task \( ((4x-8)+16):4=22\)
			\task \( 2x-(8x-5)=11 \)
		\end{tasks}
		\item Решите пропорцию:
		\begin{tasks}(2)
			\task \( \dfrac{5x}{15}=\dfrac{6}{8} \)
			\task \( 8:10x=4:25\)
		\end{tasks}
		\item В прямоугольном треугольнике один острый угол меньше другого острого угла в \( 2 \) раза. Найдите все углы треугольника.
		\item В прямоугольном треугольнике один угол на \( 30 \degree \) меньше другого. Какими могут быть эти углы?
	\end{listofex}
\end{class}
%END_FOLD

%BEGIN_FOLD % ====>>_ Домашняя работа 1 _<<====
\begin{homework}[number=1]
	\begin{listofex}
		\item Решить уравнение:
		\begin{tasks}(2)
			\task \( 5x-(3x-1)=5 \)
			\task \( ((3x-1)+4):2=12\)
		\end{tasks}
		\item В двух коробках \(7,8\) кг конфет.
		Когда из одной коробки взяли \(1,25\) кг конфет, то в обеих коробках конфет стало поровну.
		Сколько конфет было в каждой коробке?
		\item Автомобиль стоит \(600000\) рублей.
		В рамках специальной акции автомобиль можно купить по цене \(534000\) рублей.
		На сколько процентов снизилась цена автомобиля?
		\item Цена на товар увеличилась на \(20\%\) и стала равна \(1500\) рублей.
		Сколько стоил товар до подорожания?
	\end{listofex}
\end{homework}
%END_FOLD

%BEGIN_FOLD % ====>>_ Домашняя работа 2 _<<====
\begin{homework}[number=2]
	\begin{listofex}
		\item Решить пропорцию:
		\begin{tasks}(2)
			\task \( \dfrac{2x}{15}=\dfrac{1}{45} \)
			\task \( 15:x=2:3\)
		\end{tasks}
		\item Цена на акцию увеличилась на \(10\%\), потом ещё на \(5\%\), а потом увала на \(20\%\). На сколько процентов изменилась цена акции по сравнению с первоначальной? Сколько стоит теперь акция, если первоначально она стоила \(4000\) рублей?
		\item В первый день на мельнице смололи \(0,3\) привезенного зерна, во второй --- \(0,3\) остатка, а в третий --- оставшиеся \(10,78\) ц. Сколько зерна смололи на мельнице за три дня?
		\item В одном пакете \( \dfrac{1}{2} \) кг, а в другом на \(\dfrac{1}{5}\) кг меньше.
		Сколько килограммов конфет в двух пакетах вместе?
	\end{listofex}
\end{homework}
%END_FOLD

%BEGIN_FOLD % ====>>_ Домашняя работа 3 _<<====
\begin{homework}[number=3]
	\begin{listofex}
		\item В двух магазинах были одинаковые цены. В одном магазине их сначала понизили на \(15\%\), а потом повысили на \(10\%\), а в другом --- сначала повысили на \(10\%\), а потом понизили на \(15\%\). Как изменились цены в этих магазинах по сравнению с первоначальной? В каком из магазинов выгоднее покупать товар?
		\item На сколько процентов увеличится произведение двух чисел, если одно из них увеличить на \(20\%\), а другое --- на \( 40\% \)?
		\item В двух коробках \(10,8\) кг конфет. Когда из одной коробки в другую переложили \(1,4\) кг конфет, то в обеих коробках конфет стало поровну. Сколько конфет было в каждой коробке?
		\item Решить уравнение:
		\begin{tasks}(2)
			\task \( 4x-(x+6,5)=5 \)
			\task \( 5x - (6x-8):2 = 10 \)
		\end{tasks}
		\item Решить пропорцию:
		\begin{tasks}(2)
			\task \( \dfrac{x}{6}=\dfrac{2}{3} \)
			\task \( 50:x=5:2\)
		\end{tasks}
		\item Один угол треугольника равен \( 15\degree \), а второй в два раза больше. Найдите все углы треугольника.
		\item Угол треугольника равен \( 45\degree \), второй угол в \( 1,5 \) раза меньше первого. Чему равна сумма внешних углов по одному при каждой вершине треугольника?
	\end{listofex}
\end{homework}
%END_FOLD

%BEGIN_FOLD % ====>>_ Проверочная работа _<<====
\begin{exam}
	\begin{listofex}
		% В ДЗ \item Цена на акцию сначала снизилась на \(10\%\), потом снизилась ещё на \(10\%\), а потом увеличилась на \(20\%\). На сколько процентов изменилась цена акции по сравнению с первоначальной? Сколько стоит теперь акция, если первоначально она стоила \(5000\) рублей?
		\item Цена на товар была повышена на \( 60\% \). На сколько процентов надо теперь её понизить, чтобы получить первоначальную цену?
		\item Кладовщик выдал первому рабочему \(0,4\) всей имевшейся проволоки, а второму --- \(0,75\) остатка, после чего у него осталось еще \(28,5\) м проволоки. Сколько проволоки было первоначально?
		\item Автомобиль проехал \(575\) км, что составляет \(\dfrac{23}{25}\) расстояния между двумя городами. Найдите расстояние между городами.
		\item Один внешний угол треугольника равен \( 25\degree \), а второй --- \( 120\degree \). Чему равны внутренние углы треугольника?
		\item Длина прямоугольника \(10\) см. Ширина на \(25\%\) меньше. Найдите его периметр.
		\item В треугольнике один угол равен \(70 \degree\), а два других находятся в отношении \(5:6\). Найдите данные углы треугольника.
		\item Решите уравнение: \[ (-(25x+10)+5):10=35\]
		\item Решите пропорцию: \( \dfrac{6x}{27}=\dfrac{1}{5} \)
		
	\end{listofex}
\end{exam}
%END_FOLD
