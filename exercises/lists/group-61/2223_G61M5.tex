%
%===============>>  ГРУППА 6-1 МОДУЛЬ 5  <<=============
%
\setmodule{5}

%BEGIN_FOLD % ====>>_____ Занятие 1 _____<<====
\begin{class}[number=1]
	\begin{listofex}
		\item Цена на товар снизилась на \(\dfrac{3}{7}\) и составила \(931\) рубля. Найдите первоначальную цену товара.
		\item В десятой части початка ветвистой кукурузы \(93\) зерна. Сколько зёрен в целом початке?
		\item Перед тем как брокер продал \( \dfrac{1}{6} \) акций своего клиента, у него было \(1800\) акций. Сколько акций у брокера теперь?
		\item В классе  \(27\)  учащихся, отсутствуют восемь человек. Какая часть учащихся присутствует?
		\item У нас было  \(325\)  рублей, в первом магазине потратили \(\dfrac{3}{5}\) этой суммы истратили, а потом еще \(\dfrac{7}{10}\) от остатка. Сколько денег у нас осталось?
		\item Два поезда идут навстречу друг другу. Один прошел \(\dfrac{2}{5}\) всего пути, а другой – половину. Сколько километров им осталось идти до встречи, если между ними было \(200\) км?
		\item Сад прямоугольной формы хотят обнести забором. Длина  сада \(800\) м, а ширина составляет \(\dfrac{5}{8}\) длины. Какой длины должен быть весь забор?
		\item В четырех домах \(3672\) жителя. В одном доме \(\dfrac{1}{3}\) всех жителей, во втором --- в \(2\) раза меньше, чем в первом, а остальные живут поровну в третьем и четвёртом домах. По скольку жителей живёт в третьем и четвёртом домах?
	\end{listofex}
\end{class}
%END_FOLD

%BEGIN_FOLD % ====>>_____ Занятие 2 _____<<====
\begin{class}[number=2]
	\begin{listofex}
		%1
		\item Гриша прочитал \(280\) страниц, что составляет \(\dfrac{7}{15}\) книги, которую читает Гриша. Сколько страниц ему осталось прочитать?
		%2
		\item Бронза --- это сплав меди с оловом. Монетная бронза содержит \(3,9 \% \) олова, оружейная бронза --- \(10,2 \%\) олова, а колокольная бронза содержит \(22,4 \% \) олова. В каждом случае определите, сколько процентов меди содержится в сплаве.
		%3
		\item Цена товара уменьшилась на \(35 \%\). Сколько процентов составляет новая цена от старой?
		%4
		\item Вкладчик снял в банке \(234\) тысячи рублей, что составило \(36 \%\) вклада. Определите первоначальную сумму вклада.
		%5
		\item Палку укоротили сначала на \(0,2\) его длины, а потом на \(\dfrac{1}{4}\)  остатка, после чего длина оставшейся части стала равна \(3\) м. Сколько метров отпилили от бревна второй раз?
		%6
		\item Сплав состоит из \(10 \%\) олова, \(35\%\) меди и \(55\%\) свинца. Сколько каждого металла содержится в \(2\) кг сплава. Ответ дайте в граммах.
		%7
		\item Стоимость покупки с учётом четырёхпроцентной скидки по дисконтной карте составила \(1152\) рубля. Сколько рублей пришлось бы заплатить за покупку при отсутствии дисконтной карты?
		%8
		\item Найдите число, если известно, что после прибавления к нему \(12\%\) его получится \(420\).
	\end{listofex}
\end{class}
%END_FOLD

%BEGIN_FOLD % ====>>_____ Занятие 3 _____<<====
\begin{class}[number=3]
	\begin{listofex}
		\item Занятие 3
	\end{listofex}
\end{class}
%END_FOLD

%BEGIN_FOLD % ====>>_____ Занятие 4 _____<<====
\begin{class}[number=4]
	\begin{listofex}
		\item Занятие 4
	\end{listofex}
\end{class}
%END_FOLD

%BEGIN_FOLD % ====>>_____ Занятие 5 _____<<====
\begin{class}[number=5]
	\begin{listofex}
		\item Занятие 5
	\end{listofex}
\end{class}
%END_FOLD

%BEGIN_FOLD % ====>>_____ Занятие 6 _____<<====
\begin{class}[number=6]
	\begin{listofex}
		\item Занятие 6
	\end{listofex}
\end{class}
%END_FOLD

%BEGIN_FOLD % ====>>_____ Занятие 7 _____<<====
\begin{class}[number=7]
	\begin{listofex}
		\item Занятие 7
	\end{listofex}
\end{class}
%END_FOLD

%BEGIN_FOLD % ====>>_ Домашняя работа 1 _<<====
\begin{homework}[number=1]
	\begin{listofex}
		\item В двух коробках \(7,8\) кг конфет. Когда из одной коробки взяли \(1,25\) кг конфет, то в обеих коробках конфет стало поровну. Сколько конфет было в каждой коробке?
		\item Автомобиль стоит \(600000\) рублей. В рамках специальной акции автомобиль можно купить по цене \(534000\) рублей. На сколько процентов снизилась цена автомобиля?
		\item Цена на товар увеличилась на \(20\%\) и стала равна \(1500\) рублей. Сколько стоил товар до подорожания? Б) Цена на товар уменьшилась на \(25\%\) и стала равна \(1125\) рублей. Найдите первоначальную цену товара.
		\item Пять рубашек дешевле куртки на \(25\%\). На сколько процентов \(7\) рубашек дороже куртки?
		\item На сколько процентов увеличится произведение двух чисел, если одно из них увеличить на \(20\%\), а другое --- на 40\%?
		\item Цена на акцию сначала снизилась на \(10\%\), потом снизилась ещё на \(10\%\), а потом увеличилась на \(20\%\). На сколько процентов изменилась цена акции по сравнению с первоначальной? Сколько стоит теперь акция, если первоначально она стоила \(5000\) рублей?
	\end{listofex}
\end{homework}
%END_FOLD

%BEGIN_FOLD % ====>>_ Домашняя работа 2 _<<====
\begin{homework}[number=2]
	\begin{listofex}
		\item ДЗ 2
	\end{listofex}
\end{homework}
%END_FOLD

%BEGIN_FOLD % ====>>_ Домашняя работа 3 _<<====
\begin{homework}[number=3]
	\begin{listofex}
		\item ДЗ 3
	\end{listofex}
\end{homework}
%END_FOLD

%BEGIN_FOLD % ====>>_ Проверочная работа _<<====
\begin{exam}
	\begin{listofex}
		\item Проверочная работа
	\end{listofex}
\end{exam}
%END_FOLD
