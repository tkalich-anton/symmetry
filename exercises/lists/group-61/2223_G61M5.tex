%
%===============>>  ГРУППА 6-1 МОДУЛЬ 5  <<=============
%
\setmodule{5}

%BEGIN_FOLD % ====>>_____ Занятие 1 _____<<====
\begin{class}[number=1]
	\begin{listofex}
		\item Цена на товар снизилась на \(\dfrac{3}{7}\) и составила \(931\) рубля. Найдите первоначальную цену товара.
		\item В десятой части початка ветвистой кукурузы \(93\) зерна. Сколько зёрен в целом початке?
		\item Перед тем как брокер продал \( \dfrac{1}{6} \) акций своего клиента, у него было \(1800\) акций. Сколько акций у брокера теперь?
		\item В классе  \(27\)  учащихся, отсутствуют восемь человек. Какая часть учащихся присутствует?
		\item У нас было  \(325\)  рублей, в первом магазине потратили \(\dfrac{3}{5}\) этой суммы истратили, а потом еще \(\dfrac{7}{10}\) от остатка. Сколько денег у нас осталось?
		\item Два поезда идут навстречу друг другу. Один прошел \(\dfrac{2}{5}\) всего пути, а другой – половину. Сколько километров им осталось идти до встречи, если между ними было \(200\) км?
		\item Сад прямоугольной формы хотят обнести забором. Длина  сада \(800\) м, а ширина составляет \(\dfrac{5}{8}\) длины. Какой длины должен быть весь забор?
		\item В четырех домах \(3672\) жителя. В одном доме \(\dfrac{1}{3}\) всех жителей, во втором --- в \(2\) раза меньше, чем в первом, а остальные живут поровну в третьем и четвёртом домах. По скольку жителей живёт в третьем и четвёртом домах?
	\end{listofex}
\end{class}
%END_FOLD

%BEGIN_FOLD % ====>>_____ Занятие 2 _____<<====
\begin{class}[number=2]
	\begin{listofex}
		\item Занятие 2
	\end{listofex}
\end{class}
%END_FOLD

%BEGIN_FOLD % ====>>_____ Занятие 3 _____<<====
\begin{class}[number=3]
	\begin{listofex}
		\item Занятие 3
	\end{listofex}
\end{class}
%END_FOLD

%BEGIN_FOLD % ====>>_____ Занятие 4 _____<<====
\begin{class}[number=4]
	\begin{listofex}
		\item Занятие 4
	\end{listofex}
\end{class}
%END_FOLD

%BEGIN_FOLD % ====>>_____ Занятие 5 _____<<====
\begin{class}[number=5]
	\begin{listofex}
		\item Занятие 5
	\end{listofex}
\end{class}
%END_FOLD

%BEGIN_FOLD % ====>>_____ Занятие 6 _____<<====
\begin{class}[number=6]
	\begin{listofex}
		\item Занятие 6
	\end{listofex}
\end{class}
%END_FOLD

%BEGIN_FOLD % ====>>_____ Занятие 7 _____<<====
\begin{class}[number=7]
	\begin{listofex}
		\item Занятие 7
	\end{listofex}
\end{class}
%END_FOLD

%BEGIN_FOLD % ====>>_ Домашняя работа 1 _<<====
\begin{homework}[number=1]
	\begin{listofex}
		\item ДЗ 1
	\end{listofex}
\end{homework}
%END_FOLD

%BEGIN_FOLD % ====>>_ Домашняя работа 2 _<<====
\begin{homework}[number=2]
	\begin{listofex}
		\item ДЗ 2
	\end{listofex}
\end{homework}
%END_FOLD

%BEGIN_FOLD % ====>>_ Домашняя работа 3 _<<====
\begin{homework}[number=3]
	\begin{listofex}
		\item ДЗ 3
	\end{listofex}
\end{homework}
%END_FOLD

%BEGIN_FOLD % ====>>_ Проверочная работа _<<====
\begin{exam}
	\begin{listofex}
		\item Проверочная работа
	\end{listofex}
\end{exam}
%END_FOLD
