%Группа 11-4 Модуль 1
\title{Занятие №1}
\begin{listofex}
	\item Вычислить:
	\begin{enumcols}[itemcolumns=2]
		\item \exercise{1740}
		\item \exercise{1846}
		\item \exercise{569}
		\item \exercise{596}
		\item \exercise{1582}
		\item \exercise{1576}
	\end{enumcols}
	\item Вычислить:
	\begin{enumcols}[itemcolumns=2]
		\item \exercise{1140}
		\item \exercise{1138}
		\item \exercise{1142}
		\item \exercise{2964}
		\item \exercise{2966}
		\item \exercise{1797}
	\end{enumcols}
	\item Решить уравнение:
	\begin{enumcols}[itemcolumns=2]
		\item \exercise{645}
		\item \exercise{646}
		\item \exercise{1037}
		\item \exercise{998}
		\item \exercise{1002}
	\end{enumcols}
	\item Решить уравнение:
	\begin{enumcols}[itemcolumns=2]
		\item \exercise{31}
		\item \exercise{1082}
		\item \exercise{1084}
		\item \( x^2-10x+25+|x^2-9x+20|=0 \)
	\end{enumcols}
	\item Найдите три последовательных натуральных числа, если удвоенный квадрат
	первого из них на \( 26 \) больше произведения второго и третьего чисел.
\end{listofex}
\newpage
\title{Занятие №2}
\begin{listofex}
	\item Вычислить:
	\begin{enumcols}[itemcolumns=2]
		\item \exercise{1741}
		\item \exercise{1469}
		\item \exercise{570}
		\item \exercise{1578}
		\item \exercise{594}
		\item \exercise{1574}
	\end{enumcols}
	\item Вычислить:
	\begin{enumcols}[itemcolumns=2]
		\item \exercise{1141}
		\item \exercise{1137}
		\item \exercise{2990}
		\item \exercise{2991}
	\end{enumcols}
	\item Найти значение выражения:
	\begin{enumcols}[itemcolumns=2]
		\item \exercise{1322}
		\item \exercise{1564}
		\item \exercise{1108}
		\item \exercise{1859}
	\end{enumcols}
	\item Решить уравнение:
	\begin{enumcols}[itemcolumns=2]
		\item \exercise{972}
		\item \exercise{975}
		\item \exercise{995}
		\item \exercise{997}
		\item \exercise{1038}
	\end{enumcols}
	\item Решить уравнение:
	\begin{enumcols}[itemcolumns=2]
		\item \exercise{1860}
		\item \exercise{1861}
	\end{enumcols}
	\item Найдите четыре последовательных нечетных натуральных числа, если удвоенное
	произведение второго и третьего чисел на \( 107 \) больше произведения первого и четвертого
	чисел.
	\answer{\( 7;\;9;\;11;\;13 \)}
\end{listofex}
\newpage
\title{Домашняя работа №1}
\begin{listofex}
	\item Вычислить:
	\begin{enumcols}[itemcolumns=1]
		\item \exercise{1743}
		\item \exercise{1554}
		\item \exercise{1653}
		\item \exercise{572}
		\item \exercise{1587}
	\end{enumcols}
	\item Вычислить:
	\begin{enumcols}[itemcolumns=2]
		\item \exercise{1139}
		\item \exercise{2980}
		\item \exercise{2992}
		\item \exercise{1793}
	\end{enumcols}
	\item Найти значение выражения:
	\begin{enumcols}[itemcolumns=1]
		\item \exercise{1858}
		\item \exercise{1227}
	\end{enumcols}
	\item Решить уравнение:
	\begin{enumcols}[itemcolumns=2]
		\item \exercise{974}
		\item \exercise{976}
		\item \exercise{996}
		\item \exercise{1003}
		\item \exercise{1040}
	\end{enumcols}
	\item Решить уравнение:
	\begin{enumcols}[itemcolumns=2]
		\item \exercise{1862}
		\item \exercise{1863}
	\end{enumcols}
	\item Найдите три последовательных натуральных числа, если удвоенный квадрат большего из них на \( 79 \) больше суммы квадратов двух других чисел.
	\answer{\( 12;\;13;\;14 \)}
\end{listofex}
\newpage
\title{Занятие №3}
\begin{listofex}
	\item Решить уравнение:
	\begin{enumcols}[itemcolumns=2]
		\item \exercise{1004}
		\item \exercise{1040}
		\item \exercise{1077}
		%\item \exercise{1078}
		\item \exercise{1075}
	\end{enumcols}
	\item Решить уравнение:
	\begin{enumcols}[itemcolumns=2]
		\item \exercise{1163}
		\item \exercise{1171}
		\item \exercise{1172}
		\item \exercise{1173}
		\item \exercise{3397}
	\end{enumcols}
	\item Решить уравнение:
	\begin{enumcols}[itemcolumns=2]
		\item \exercise{3499}
		\item \exercise{3500}
		\item \exercise{3502}
		\item \exercise{3424}
		\item \exercise{3423}
		\item \exercise{3431}
		\item \exercise{3435}
	\end{enumcols}
	\item Решить уравнение:
	\begin{enumcols}[itemcolumns=2]
		\item \exercise{3438}
		\item \exercise{3490}
		\item \exercise{3495}
		\item \exercise{3440}
		\item \exercise{3442}
		\item \exercise{3443}
	\end{enumcols}
	\item Решить уравнение:
	\begin{enumcols}[itemcolumns=1]
		\item \exercise{3445}
		\item \exercise{3507}
		\item \exercise{3510}
	\end{enumcols}
\end{listofex}
\newpage
\title{Занятие №4}
\begin{listofex}
	\item Решить уравнение:
	\begin{enumcols}[itemcolumns=2]
		\item \exercise{3744}
		\item \exercise{3390}
		\item \exercise{3727}
		%\item \exercise{1078}
		\item \exercise{3741}
	\end{enumcols}
	\item Решить уравнение:
	\begin{enumcols}[itemcolumns=2]
		\item \( \sqrt{\dfrac{4}{2x-11}}=\dfrac{1}{3} \)
		\item \exercise{3403}
		\item \exercise{3404}
		%\item \exercise{3404}
		\item \exercise{3406}
		%\item \exercise{1173} на след урок
		%\item \exercise{3397} на след урок
	\end{enumcols}
	\item Решить уравнение:
	\begin{enumcols}[itemcolumns=2]
		\item \exercise{3514}
		\item \exercise{3497}
		\item \exercise{3502}
		\item \exercise{3424}
		\item \exercise{3423}
		%\item \exercise{3431} на след урок
		%\item \exercise{3435} на след урок
	\end{enumcols}
	\item Решить уравнение:
	\begin{enumcols}[itemcolumns=2]
		\item \exercise{3529}
		\item \exercise{3438}
		\item \exercise{3490}
		\item \exercise{3495}
		%\item \exercise{3440}
		%\item \exercise{3442}
		%\item \exercise{3443}
	\end{enumcols}
	%\item Решить уравнение:
	%\begin{enumcols}[itemcolumns=1]
	%	\item \exercise{3445}
	%	\item \exercise{3507}
	%	\item \exercise{3510}
	%\end{enumcols}
\end{listofex}
\newpage
\title{Домашняя работа №2}
\begin{listofex}
	\item Вычислить:
	\begin{enumcols}[itemcolumns=2]
		\item \exercise{1639}
		\item \exercise{1666}
	\end{enumcols}
	\item Решить уравнение:
	\begin{enumcols}[itemcolumns=1]
		\item \exercise{3745}
		\item \exercise{3766}
		\item \exercise{3714}
		\item \exercise{3719}
	\end{enumcols}
	\item Решить уравнение:
	\begin{enumcols}[itemcolumns=2]
		\item \exercise{3407}
		\item \exercise{3404}
		\item \exercise{3408}
	\end{enumcols}
	\item Решить уравнение:
	\begin{enumcols}[itemcolumns=2]
		\item \exercise{3427}
		\item \exercise{3511}
		\item \exercise{3422}
		\item \exercise{3519}
	\end{enumcols}
	\item \exercise{2954}
	\item Вычислить:
	\begin{enumcols}[itemcolumns=1]
		\item \exercise{2981}
		\item \exercise{1119}
		\item \exercise{2920}
		\item \exercise{2924}
	\end{enumcols}
\end{listofex}