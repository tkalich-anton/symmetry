%10 класс Предподготовка Занятие №2
\begin{listofex}
	\item Упростить выражение:
	\begin{enumcols}[itemcolumns=1]
		\item \exercise{1479}
		\item \exercise{1403}
	\end{enumcols}
	\item Вычислить:
	\begin{enumcols}[itemcolumns=2]
		\item \exercise{1550}
		\item \exercise{2837}
		\item \exercise{1768}
		\item \exercise{1646}
	\end{enumcols}
	\item Решить уравнение:
	\begin{enumcols}[itemcolumns=2]
		\item \exercise{496}
		\item \exercise{503}
	\end{enumcols}
	\item Решить уравнение:
	\begin{enumcols}[itemcolumns=2]
		\item \exercise{511}
		\item \exercise{542}
		\item \exercise{1009}
		\item \exercise{1025}
	\end{enumcols}
	\item \exercise{31}
	\item Решить неравенство:
	\begin{enumcols}[itemcolumns=2]
		\item \exercise{644}
		\item \exercise{647}
		\item \exercise{650}
		\item \exercise{653}
	\end{enumcols}
	\item В розетку электросети подключены приборы, общее
	сопротивление которых составляет \( R_1=90 \) Ом. Параллельно с
	ними в розетку предполагается подключить электрообогреватель.
	Определите наименьшее возможное сопротивление \( R_2 \) этого
	электрообогревателя, если известно, что при параллельном
	соединении двух проводников с сопротивлениями \( R_1 \) Ом и \( R_2 \) Ом
	их общее сопротивление дается формулой \( R_{\text{\textit{общ}}}=\dfrac{R_1\cdot R_2}{R_1 + R_2} \), а
	для нормального функционирования электросети общее
	сопротивление в ней должно быть не меньше \( 9 \) Ом. Ответ выразите в Омах.
	\answer{\( 10 \)}
\end{listofex}