%Группа 11-4 Модуль 2
\title{Занятие №1}
\begin{listofex}
	\item На экзамен вынесено 60 вопросов, Андрей не выучил 3 из них. Найдите вероятность того, что ему попадется выученный вопрос.
	\item На рок-фестивале выступают группы — по одной от каждой из заявленных стран. Порядок выступления определяется жребием. Какова вероятность того, что группа из Дании будет выступать после группы из Швеции и после группы из Норвегии? Результат округлите до сотых.
	\item На борту самолёта 12 кресел расположены рядом с запасными выходами и 18 — за перегородками, разделяющими салоны. Все эти места удобны для пассажира высокого роста. Остальные места неудобны. Пассажир В. высокого роста. Найдите вероятность того, что на регистрации при случайном выборе места пассажиру В. достанется удобное место, если всего в самолёте 300 мест.
	\item Механические часы с двенадцатичасовым циферблатом в какой-то момент сломались и перестали идти. Найдите вероятность того, что часовая стрелка остановилась, достигнув отметки 10, но не дойдя до отметки 1.
	\item За круглый стол на 9 стульев в случайном порядке рассаживаются 7 мальчиков и 2 девочки. Найдите вероятность того, что обе девочки будут сидеть рядом.
	\item За круглый стол на 201 стул в случайном порядке рассаживаются 199 мальчиков и 2 девочки. Найдите вероятность того, что между девочками будет сидеть один мальчик.
	\item В случайном эксперименте симметричную монету бросают дважды. Найдите вероятность того, что орел выпадет ровно один раз.
	\item В случайном эксперименте симметричную монету бросают трижды. Найдите вероятность того, что орел выпадет ровно два раза.
	\item Какова вероятность того, что случайно выбранный телефонный номер оканчивается двумя чётными цифрами?
	\item Если шахматист А. играет белыми фигурами, то он выигрывает у шахматиста Б. с вероятностью 0,52. Если А. играет черными, то А. выигрывает у Б. с вероятностью 0,3. Шахматисты А. и Б. играют две партии, причём во второй партии меняют цвет фигур. Найдите вероятность того, что А. выиграет оба раза.
	\item Из городов \( A  \) и \( B  \) навстречу друг другу выехали мотоциклист и велосипедист. Мотоциклист приехал в \(В\)на \(1\) час раньше, чем велосипедист приехал в \( A \), а встретились они через \(40\) минут после выезда. Сколько часов затратил на путь из \(  B  \) в \( A  \) велосипедист?
	\item Смешали некоторое количество \( 18 \)-процентного раствора некоторого вещества с таким же количеством \( 14 \)-процентного раствора этого вещества. Сколько процентов составляет концентрация получившегося раствора?
	\item Имеется два сплава. Первый сплав содержит \( 5\% \) меди, второй – \( 12\% \) меди. Масса второго сплава больше массы первого на \( 9 \) кг. Из этих двух сплавов получили третий сплав, содержащий \( 10\% \) меди. Найдите массу третьего сплава. Ответ дайте в килограммах.
\end{listofex}
\newpage
\title{Занятие №2}
\begin{listofex}
	\item При изготовлении подшипников диаметром 67 мм вероятность того, что диаметр будет отличаться от заданного не больше, чем на 0,01 мм, равна 0,965. Найдите вероятность того, что случайный подшипник будет иметь диаметр меньше чем 66,99 мм или больше чем 67,01 мм.
	\item Вероятность того, что в случайный момент времени температура тела здорового человека окажется ниже чем 36,8 °С, равна 0,81. Найдите вероятность того, что в случайный момент времени у здорового человека температура окажется 36,8 °С или выше.
	\item Вероятность того, что батарейка бракованная, равна 0,06. Покупатель в магазине выбирает случайную упаковку, в которой две таких батарейки. Найдите вероятность того, что обе батарейки окажутся исправными.
	\item Если шахматист А. играет белыми фигурами, то он выигрывает у шахматиста Б. с вероятностью 0,52. Если А. играет черными, то А. выигрывает у Б. с вероятностью 0,3. Шахматисты А. и Б. играют две партии, причём во второй партии меняют цвет фигур. Найдите вероятность того, что А. выиграет оба раза.
	\item В магазине три продавца. Каждый из них занят обслуживанием клиента с вероятностью 0,2 независимо от других продавцов. Найдите вероятность того, что в случайный момент времени все три продавца заняты.
	\item В торговом центре два одинаковых автомата продают кофе. Обслуживание автоматов происходит по вечерам после закрытия центра. Известно, что вероятность события «К вечеру в первом автомате закончится кофе» равна 0,25. Такая же вероятность события «К вечеру во втором автомате закончится кофе». Вероятность того, что кофе к вечеру закончится в обоих автоматах, равна 0,15. Найдите вероятность того, что к вечеру дня кофе останется в обоих автоматах.
	\item Из районного центра в деревню ежедневно ходит автобус. Вероятность того, что в понедельник в автобусе окажется меньше 20 пассажиров, равна 0,94. Вероятность того, что окажется меньше 15 пассажиров, равна 0,56. Найдите вероятность того, что число пассажиров будет от 15 до 19.
	\item Вероятность того, что новый электрический чайник прослужит больше года, равна 0,97. Вероятность того, что он прослужит больше двух лет, равна 0,89. Найдите вероятность того, что он прослужит меньше двух лет, но больше года.
	\item Помещение освещается фонарём с двумя лампами. Вероятность перегорания лампы в течение года равна 0,3. Найдите вероятность того, что в течение года хотя бы одна лампа не перегорит.
	\item В Волшебной стране бывает два типа погоды: хорошая и отличная, причём погода, установившись утром, держится неизменной весь день. Известно, что с вероятностью 0,8 погода завтра будет такой же, как и сегодня. Сегодня 3 июля, погода в Волшебной стране хорошая. Найдите вероятность того, что 6 июля в Волшебной стране будет отличная погода.
	\item Ковбой Джон попадает в муху на стене с вероятностью 0,9, если стреляет из пристрелянного револьвера. Если Джон стреляет из непристрелянного револьвера, то он попадает в муху с вероятностью 0,2. На столе лежит 10 револьверов, из них только 4 пристрелянные. Ковбой Джон видит на стене муху, наудачу хватает первый попавшийся револьвер и стреляет в муху. Найдите вероятность того, что Джон промахнётся.
\end{listofex}
%\newpage
%\title{Домашняя работа №1}
%\begin{listofex}
%
%\end{listofex}
%\newpage
%\title{Занятие №3}
%\begin{listofex}
%
%\end{listofex}
%\newpage
%\title{Занятие №4}
%\begin{listofex}
%
%\end{listofex}
%\newpage
%\title{Домашняя работа №2}
%\begin{listofex}
%
%\end{listofex}
%\newpage
%\title{Занятие №5}
%\begin{listofex}
%
%\end{listofex}
%\newpage
%\title{Занятие №6}
%\begin{listofex}
%
%\end{listofex}
%\newpage
%\title{Занятие №7}
%\begin{listofex}
%
%\end{listofex}
%\newpage
%\title{Проверочная работа}
%\begin{listofex}
%
%\end{listofex}
\newpage
\title{Консультация}
\begin{listofex}
	\item \exercise{1075}
	\item \exercise{31}
	\item \exercise{1081}
	\item \exercise{1073}
	\item \exercise{3707}
	\item \exercise{3723}
\end{listofex}