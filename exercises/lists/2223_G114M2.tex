%Группа 11-4 Модуль 2
\title{Занятие №1}
\begin{listofex}
	\item На экзамен вынесено 60 вопросов, Андрей не выучил 3 из них. Найдите вероятность того, что ему попадется выученный вопрос.
	\item На рок-фестивале выступают группы — по одной от каждой из заявленных стран. Порядок выступления определяется жребием. Какова вероятность того, что группа из Дании будет выступать после группы из Швеции и после группы из Норвегии? Результат округлите до сотых.
	\item На борту самолёта 12 кресел расположены рядом с запасными выходами и 18 — за перегородками, разделяющими салоны. Все эти места удобны для пассажира высокого роста. Остальные места неудобны. Пассажир В. высокого роста. Найдите вероятность того, что на регистрации при случайном выборе места пассажиру В. достанется удобное место, если всего в самолёте 300 мест.
	\item Механические часы с двенадцатичасовым циферблатом в какой-то момент сломались и перестали идти. Найдите вероятность того, что часовая стрелка остановилась, достигнув отметки 10, но не дойдя до отметки 1.
	\item За круглый стол на 9 стульев в случайном порядке рассаживаются 7 мальчиков и 2 девочки. Найдите вероятность того, что обе девочки будут сидеть рядом.
	\item За круглый стол на 201 стул в случайном порядке рассаживаются 199 мальчиков и 2 девочки. Найдите вероятность того, что между девочками будет сидеть один мальчик.
	\item В случайном эксперименте симметричную монету бросают дважды. Найдите вероятность того, что орел выпадет ровно один раз.
	\item В случайном эксперименте симметричную монету бросают трижды. Найдите вероятность того, что орел выпадет ровно два раза.
	\item Какова вероятность того, что случайно выбранный телефонный номер оканчивается двумя чётными цифрами?
	\item Если шахматист А. играет белыми фигурами, то он выигрывает у шахматиста Б. с вероятностью 0,52. Если А. играет черными, то А. выигрывает у Б. с вероятностью 0,3. Шахматисты А. и Б. играют две партии, причём во второй партии меняют цвет фигур. Найдите вероятность того, что А. выиграет оба раза.
	\item Из городов \( A  \) и \( B  \) навстречу друг другу выехали мотоциклист и велосипедист. Мотоциклист приехал в \(В\)на \(1\) час раньше, чем велосипедист приехал в \( A \), а встретились они через \(40\) минут после выезда. Сколько часов затратил на путь из \(  B  \) в \( A  \) велосипедист?
	\item Смешали некоторое количество \( 18 \)-процентного раствора некоторого вещества с таким же количеством \( 14 \)-процентного раствора этого вещества. Сколько процентов составляет концентрация получившегося раствора?
	\item Имеется два сплава. Первый сплав содержит \( 5\% \) меди, второй – \( 12\% \) меди. Масса второго сплава больше массы первого на \( 9 \) кг. Из этих двух сплавов получили третий сплав, содержащий \( 10\% \) меди. Найдите массу третьего сплава. Ответ дайте в килограммах.
\end{listofex}
\newpage
\title{Занятие №2}
\begin{listofex}
	\item При изготовлении подшипников диаметром 67 мм вероятность того, что диаметр будет отличаться от заданного не больше, чем на 0,01 мм, равна 0,965. Найдите вероятность того, что случайный подшипник будет иметь диаметр меньше чем 66,99 мм или больше чем 67,01 мм.
	\item Вероятность того, что в случайный момент времени температура тела здорового человека окажется ниже чем 36,8 °С, равна 0,81. Найдите вероятность того, что в случайный момент времени у здорового человека температура окажется 36,8 °С или выше.
	\item Вероятность того, что батарейка бракованная, равна 0,06. Покупатель в магазине выбирает случайную упаковку, в которой две таких батарейки. Найдите вероятность того, что обе батарейки окажутся исправными.
	\item Если шахматист А. играет белыми фигурами, то он выигрывает у шахматиста Б. с вероятностью 0,52. Если А. играет черными, то А. выигрывает у Б. с вероятностью 0,3. Шахматисты А. и Б. играют две партии, причём во второй партии меняют цвет фигур. Найдите вероятность того, что А. выиграет оба раза.
	\item В магазине три продавца. Каждый из них занят обслуживанием клиента с вероятностью 0,2 независимо от других продавцов. Найдите вероятность того, что в случайный момент времени все три продавца заняты.
	\item В торговом центре два одинаковых автомата продают кофе. Обслуживание автоматов происходит по вечерам после закрытия центра. Известно, что вероятность события «К вечеру в первом автомате закончится кофе» равна 0,25. Такая же вероятность события «К вечеру во втором автомате закончится кофе». Вероятность того, что кофе к вечеру закончится в обоих автоматах, равна 0,15. Найдите вероятность того, что к вечеру дня кофе останется в обоих автоматах.
	\item Из районного центра в деревню ежедневно ходит автобус. Вероятность того, что в понедельник в автобусе окажется меньше 20 пассажиров, равна 0,94. Вероятность того, что окажется меньше 15 пассажиров, равна 0,56. Найдите вероятность того, что число пассажиров будет от 15 до 19.
	\item Вероятность того, что новый электрический чайник прослужит больше года, равна 0,97. Вероятность того, что он прослужит больше двух лет, равна 0,89. Найдите вероятность того, что он прослужит меньше двух лет, но больше года.
	\item Помещение освещается фонарём с двумя лампами. Вероятность перегорания лампы в течение года равна 0,3. Найдите вероятность того, что в течение года хотя бы одна лампа не перегорит.
	\item В Волшебной стране бывает два типа погоды: хорошая и отличная, причём погода, установившись утром, держится неизменной весь день. Известно, что с вероятностью 0,8 погода завтра будет такой же, как и сегодня. Сегодня 3 июля, погода в Волшебной стране хорошая. Найдите вероятность того, что 6 июля в Волшебной стране будет отличная погода.
	\item Ковбой Джон попадает в муху на стене с вероятностью 0,9, если стреляет из пристрелянного револьвера. Если Джон стреляет из непристрелянного револьвера, то он попадает в муху с вероятностью 0,2. На столе лежит 10 револьверов, из них только 4 пристрелянные. Ковбой Джон видит на стене муху, наудачу хватает первый попавшийся револьвер и стреляет в муху. Найдите вероятность того, что Джон промахнётся.
\end{listofex}
\newpage
\title{Домашняя работа №1}
\begin{listofex}
	\item \exercise{3176}
		\item Решить уравнение:
		\begin{enumcols}[itemcolumns=3]
			\item \exercise{3782}
			\item \exercise{3775}
			\item \exercise{3770}
		\end{enumcols}
		\item Чтобы пройти в следующий круг соревнований, футбольной команде нужно набрать хотя бы \( 4 \) очка в двух играх. Если команда выигрывает, она получает \( 3 \) очка, в случае ничьей --- \( 1 \) очко, если проигрывает --- \( 0 \) очков. Найдите вероятность того, что команде удастся выйти в следующий круг соревнований. Считайте, что в каждой игре вероятности выигрыша и проигрыша одинаковы и равны \( 0,4 \).
		\item В Волшебной стране бывает два типа погоды: хорошая и отличная, причём погода, установившись утром, держится неизменной весь день. Известно, что с вероятностью 0,8 погода завтра будет такой же, как и сегодня. Сегодня 3 июля, погода в Волшебной стране хорошая. Найдите вероятность того, что 6 июля в Волшебной стране будет хорошая погода.
		\item Перед началом волейбольного матча капитаны команд тянут честный жребий, чтобы определить, какая из команд начнёт игру с мячом. Команда «Статор» по очереди играет с командами «Ротор», «Мотор» и «Стартер». Найдите вероятность того, что «Статор» будет начинать только первую и последнюю игры.
		\item Вере надо подписать 640 открыток. Ежедневно она подписывает на одно и то же количество открыток больше по сравнению с предыдущим днем. Известно, что за первый день Вера подписала 10 открыток. Определите, сколько открыток было подписано за четвертый день, если вся работа была выполнена за 16 дней.
		\item Весной катер идёт против течения реки в \( 1\:\dfrac{2}{3} \) раза медленнее, чем по течению. Летом течение становится на \( 1 \) км/ч медленнее. Поэтому летом катер идёт против течения в \( 1\:\dfrac{1}{2} \) раза медленнее, чем по течению. Найдите скорость течения весной (в км/ч).
		\item Решить неравенство: \( \dfrac{x^2-3x-2}{x-3}+\dfrac{4}{x-5}\le x \)
\end{listofex}
\newpage
\title{Занятие №3}
\begin{listofex}
	\item Решить уравнение:
	\begin{enumcols}[itemcolumns=2]
		\item \( \left( \dfrac{1}{8} \right)^{-3+x}=512 \) \answer{ \( 0 \) }
		\item \exercise{3422}
		\item \exercise{3430}
		\item \exercise{3358}
		\item \( 2^{3+x}=0,4\cdot5^{3+x} \)
		\item \(\log_5(x^2+2x)=\log_5(x^2+10) \)\answer{ \( 5 \) }
		\item \( \log_{x-5}49=2 \) \answer{ \( 12 \) }
	\end{enumcols}
	\item Решить уравнение:
	\begin{enumcols}[itemcolumns=1]
		\item \exercise{3544}
		\item \exercise{3547}
	\end{enumcols}
	\item Решить уравнение:
	\begin{enumcols}[itemcolumns=2]
		\item \( \sin \left( x+\dfrac{\pi}{2} \right) = \dfrac{\sqrt{3}}{2} \) \answer{ \( \dfrac{\pi}{6}+2\pi n;\;\dfrac{11\pi}{6}+2\pi n \) }
		\item \( \cos \left( 2x-\dfrac{3\pi}{2} \right) = 1 \) \answer{ \( \dfrac{3\pi}{4}+\pi n \) }
		\item \( \sin \left( \dfrac{1}{2}\pi-x \right)=1 \)\answer{ \( 2\pi n \) }
		\item \( \tg\left( 3x-\dfrac{5}{4}\pi \right)=-1 \)\answer{ \( \dfrac{\pi n}{3} \) }
	\end{enumcols}
	\item Решить уравнение \( \cos\dfrac{\pi(x-7)}{3}=\dfrac{1}{2} \). В ответ запишите наибольший отрицательный корень.\answer{ \( -4 \) }
	\item Решить уравнение \( \tg\dfrac{\pi x}{4}=\dfrac{1}{2} \). В ответ запишите наибольший отрицательный корень.\answer{ \( -1 \) }
	\item \exercise{3174}
	\item Два велосипедиста одновременно отправились в 88-километровый пробег. Первый ехал со скоростью, на 3 км/ч большей, чем скорость второго, и прибыл к финишу на 3 часа раньше второго. Найти скорость велосипедиста, пришедшего к финишу вторым. Ответ дайте в км/ч.
	\item Первый садовый насос перекачивает 5 литров воды за 2 минуты, второй насос перекачивает тот же объём воды за 3 минуты. Сколько минут эти два насоса должны работать совместно, чтобы перекачать 25 литров воды?
	\item Решить уравнение:
	\begin{enumcols}[itemcolumns=2]
		\item \exercise{3776}
		\item \exercise{3769}
	\end{enumcols}
\end{listofex}
\newpage
\title{Занятие №4}
\begin{listofex}
		\item Решить уравнение:
	\begin{enumcols}[itemcolumns=2]
		\item \exercise{3502}
		\item \exercise{3500}
		\item \exercise{3498}
		\item \exercise{3429}
		\item \exercise{607}
		\item \exercise{777}
		\item \exercise{3545}
	\end{enumcols}
	\item Решить уравнение:
	\begin{enumcols}[itemcolumns=3]
		\item \( \sin x = \dfrac{1}{3} \)
		\item \( \sin x = \dfrac{3}{2}\)
		\item \( \tg 2x = \dfrac{1}{2}\)
	\end{enumcols}
	\item Решить уравнение:
	\begin{enumcols}[itemcolumns=2]
		\item \( \sin \left( x+\dfrac{\pi}{3} \right) = \dfrac{\sqrt{2}}{2} \)
		\item \( \sin \left( 2x-\dfrac{3\pi}{2} \right) = -1 \)
		\item \( \cos \left( \dfrac{\pi}{4}-x \right)=\dfrac{\sqrt{3}}{2} \)
		\item \( \ctg\left( 2x-\dfrac{3\pi}{4} \right)=-1 \)
	\end{enumcols}
	\item Решить уравнение \( \cos\dfrac{\pi(x-4)}{2}=\dfrac{\sqrt{3}}{2} \). В ответ запишите наибольший отрицательный корень.
	\item Решить уравнение \( \sin\dfrac{2\pi x}{3}=\dfrac{1}{2} \). В ответ запишите наименьший положительный корень.
	\item \exercise{3175}
	\item Улитка ползет от одного дерева до другого. Каждый день она проползает на одно и то же расстояние больше, чем в предыдущий день. Известно, что за первый и последний дни улитка проползла в общей сложности \( 10 \) метров. Определите, сколько дней улитка потратила на весь путь, если расстояние между деревьями равно \( 150 \) метрам.\answer{ \( 30 \) }
	\item Плиточник планирует уложить \( 175 \) м\( ^2 \) плитки. Если он будет укладывать на \( 10 \) м\( ^2 \) в день больше, чем запланировал, то закончит работу на \( 2 \) дня раньше. Сколько квадратных метров плитки в день планирует укладывать плиточник? \answer{ \( 25 \) }
	\item Две фабрики выпускают одинаковые стекла для автомобильных фар. Первая фабрика выпускает \( 45\% \) этих стекол, вторая --- \( 55\% \). Первая фабрика выпускает \( 3\% \) бракованных стекол, а вторая — \( 1\% \). Найдите вероятность того, что случайно купленное в магазине стекло окажется бракованным.\answer{ \( 0,019 \) }
\end{listofex}
\newpage
\title{Домашняя работа №2}
\begin{listofex}
	\item Решить уравнение:
	\begin{enumcols}[itemcolumns=2]
		\item \( \big||2x - 1|-6\big|= 10 \)
		\item \(\log_7(2x^2+4x)=\log_7(2x^2+10x-1) \)
		\item \exercise{3304}
		\item \( \cos \left( 4x-\dfrac{5\pi}{2} \right) = -1 \)
	\end{enumcols}
	\item Решить уравнение \( \cos\dfrac{\pi(x-4)}{2}=-\dfrac{\sqrt{3}}{2} \). В ответ запишите наибольший отрицательный корень.
	\item \exercise{2997}
	\item Решить уравнение:
	\begin{enumcols}[itemcolumns=2]
		\item \exercise{3308}
		\item \exercise{3312}
	\end{enumcols}
	\item Два промышленных фильтра, работая одновременно, очищают цистерну воды за 30 минут.
	Определите, за сколько минут второй фильтр очистит цистерну воды, работая отдельно,
	если известно, что он сделает это на 25 минут быстрее, чем первый.
	\item У Вити в копилке лежит 12 рублёвых, 6 двухрублёвых, 4 пятирублёвых и 3 десятирублёвых монеты. Витя наугад достаёт из копилки одну монету. Найдите вероятность того, что оставшаяся в копилке сумма составит более 70 рублей.
	\item Вероятность того, что новый сканер прослужит больше года, равна 0,9. Вероятность того, что он прослужит больше двух лет, равна 0,88. Найдите вероятность того, что он прослужит меньше двух лет, но больше года.
\end{listofex}
\newpage
\title{Занятие №5}
\begin{listofex}
	\item \exercise{1189}
	\item \exercise{24}
	\item \exercise{1867}
	\item \exercise{1868}
	\item \exercise{1869}
	\item \exercise{1870}
	\item \exercise{1871}
	\item \exercise{1855}
	\item \exercise{1857}
	\item \exercise{1866}
	\item \exercise{185}
	\item Построить графики функций:
	\begin{enumcols}[itemcolumns=2]
		\item \( y=|x^2-5x+6| \)
		\item \( y=x^2-5|x|+6 \)
		\item \( y=\dfrac{2}{|x|}+1 \)
		\item \( y=2x+|x+1|+|x-4| \)
	\end{enumcols}
\end{listofex}
%\newpage
%\title{Занятие №6}
%\begin{listofex}
%
%\end{listofex}
\newpage
\title{Подготовка к проверочной}
\begin{listofex}
	\item Решить уравнение:
	\begin{enumcols}[itemcolumns=2]
		\item \( \big||x - 4|-1\big|= 5 \)
		\item \(\log_5(x^2+2x)=\log_5(x^2+10) \)\answer{ \( 5 \) }
		\item \exercise{3547}
		\item \( \cos \left( 2x-\dfrac{3\pi}{2} \right) = 1 \) \answer{ \( \dfrac{3\pi}{4}+\pi n \) }
	\end{enumcols}
	\item Решить уравнение \( \sin\dfrac{2\pi x}{3}=-\dfrac{\sqrt{2}}{2} \). В ответ запишите наименьший положительный корень.
	\item В Волшебной стране бывает два типа погоды: хорошая и отличная, причём погода, установившись утром, держится неизменной весь день. Известно, что с вероятностью 0,6 погода завтра будет такой же, как и сегодня. Сегодня 25 октября, погода в Волшебной стране хорошая. Найдите вероятность того, что 28 октября в Волшебной стране будет хорошая погода.
	\item В помощь садовому насосу, перекачивающему 8 литров воды за 3 минуты, подключили
	второй насос, перекачивающий тот же объем воды за 6 минут. Сколько минут эти два
	насоса должны работать совместно, чтобы перекачать 24 литра воды?
	\item \exercise{3317}
\end{listofex}
\newpage
\title{Проверочная работа}
\begin{listofex}
	\item Вычислить:
	\begin{enumcols}[itemcolumns=2]
		\item \exercise{1570}
		\item \exercise{595}
		\item \( \dfrac{20\sin13\degree\cdot\cos13\degree}{-\sin26\degree} \)
		\item \( \sqrt{3}-\sqrt{12}\sin^2\dfrac{7\pi}{12} \)
		\item \( \dfrac{7}{\cos^2\left( \dfrac{\vphantom{9}\pi}{8} \right)+\cos^2\left( \dfrac{5\pi}{8} \right)} \)
	\end{enumcols}
	\item Решить уравнение:
	\begin{enumcols}[itemcolumns=2]
		\item \( \big||2x + 1|-6\big|= 8 \)
		\item \(\log_\pi(16x^3-2)=\log_\pi(3x-24x^4) \)
		\item \exercise{3447}
		\item \exercise{781}
	\end{enumcols}
	\item \exercise{1117}
	\item Решить уравнение \( \cos\dfrac{\pi(3x+6)}{3}=\dfrac{\sqrt{2}}{2} \). В ответ запишите наименьший положительный корень.
	\item За круглый стол на 201 стул в случайном порядке рассаживаются 199 мальчиков и 2 девочки. Найдите вероятность того, что между девочками будет сидеть один мальчик.
	\item Если шахматист А. играет белыми фигурами, то он выигрывает у шахматиста Б. с вероятностью 0,54. Если А. играет черными, то А. выигрывает у Б. с вероятностью 0,4. Шахматисты А. и Б. играют две партии, причём во второй партии меняют цвет фигур. Найдите вероятность того, что А. выиграет оба раза.
	\item Смешали некоторое количество \( 20 \)-процентного раствора некоторого вещества с таким же количеством \( 16 \)-процентного раствора этого вещества. Сколько процентов составляет концентрация получившегося раствора?
	\item \exercise{1867}
	\item \exercise{1870}
	\item \exercise{1871}
	\item \exercise{1857}
\end{listofex}
\newpage
\title{Консультация}
\begin{listofex}
	\item \exercise{1075}
	\item \exercise{31}
	\item \exercise{1081}
	\item \exercise{1073}
	\item \exercise{3707}
	\item \exercise{3723}
\end{listofex}
\newpage
\title{Консультация}
\begin{listofex}
	\item Найдите производную функции:
	\begin{enumcols}[itemcolumns=2]
		\item \( y=x^3-3x^2+15x+1 \)
		\item \( y=\dfrac{1}{5}x^5-2x^2-\sqrt{7} \)
		\item \( y=x^3-3x^2+15x+1 \)
		\item \( y=3\sqrt[3]{x^2}+2x\sqrt[3]{x}+\dfrac{1}{x^3} \)
		\item \( y=3\cos x - \dfrac{1}{2} \)
		\item \( y=\dfrac{1}{x} \)
		\item \( y=\dfrac{1}{x^2} \)
		\item \( y=2\sin x + \cos x \)
		\item \( y=\dfrac{4}{3}\sqrt{x}-\dfrac{4}{3}\sqrt[4]{x^3} \)
		\item \( y=\dfrac{x}{x^2+1} \)
		\item \( y=\dfrac{x^2+x-7}{x^2+1} \)
		\item \( y=4^x+8^x \)
	\end{enumcols}
\end{listofex}