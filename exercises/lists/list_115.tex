%9 класс Предподготовка Занятие №2
\begin{listofex}
	\item \exercise{1535}
	\item Упростить выражение:
	\begin{enumcols}[itemcolumns=2]
		\item \exercise{1516}
		\item \exercise{1454}
	\end{enumcols}
	\item \exercise{1223}
	\item Решить уравнение:
	\begin{enumcols}[itemcolumns=2]
		\item \exercise{496}
		\item \exercise{503}
	\end{enumcols}
	\item Решить уравнение:
	\begin{enumcols}[itemcolumns=2]
		\item \exercise{511}
		\item \exercise{542}
		\item \exercise{1009}
		\item \exercise{1025}
	\end{enumcols}
	\item \exercise{31}
	\item Решить неравенство:
	\begin{enumcols}[itemcolumns=3]
		\item \exercise{644}
		\item \exercise{647}
		\item \exercise{653}
	\end{enumcols}
	\item Работа была выполнена за \( 3 \) дня. В первый день было сделано \( \frac{3}{20} \) всей работы, а во второй --- \( \frac{5}{12} \) всей работы. Какая часть работы была выполнена в третий день?
	\answer{\( \dfrac{13}{30} \)}
	\item В помощь садовому насосу, перекачивающему \( 5 \) л воды за \( 2 \) мин, подключили второй насос, перекачивающий тот же объем воды за \( 3 \) мин. Сколько времени эти два насоса должны работать совместно, чтобы перекачать \( 25 \) л воды?
	\answer{\( 6 \)}
\end{listofex}