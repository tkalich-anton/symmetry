%Группа 8-2 Модуль 1 Занятие №1
\begin{listofex}
	\item Упростить дробь:
	\begin{enumcols}[itemcolumns=3]
		\item \exercise{20}
		\item \exercise{49}
		\item \exercise{54}
		\item \exercise{55}
		%\item \exercise{56} Занятие 2
		\item \exercise{60}
		\item \exercise{61}
	\end{enumcols}
	\item Упростить дробь:
	\begin{enumcols}[itemcolumns=4]
		\item \exercise{63}
		\item \exercise{65}
		\item \exercise{67}
		\item \exercise{70}
	\end{enumcols}
	\item Упростить дробь:
	\begin{enumcols}[itemcolumns=3]
		\item \exercise{73}
		\item \exercise{74}
		\item \exercise{75}
		\item \exercise{78}
		\item \exercise{80}
	\end{enumcols}
	\item Упростить дробь:
	\begin{enumcols}[itemcolumns=5]
		\item \exercise{83}
		\item \exercise{86}
		\item \exercise{1346}
		\item \exercise{93}
		\item \exercise{97}
	\end{enumcols}
	\item Упростить дробь:
	\begin{enumcols}[itemcolumns=3]
		\item \exercise{99}
		\item \exercise{103}
		\item \exercise{107}
	\end{enumcols}
	Вспомним:
	\[ \begin{array}{cccc}
		\text{Разность квадратов}&(a+b)(a-b)& =&a^2-b^2,\\
		\text{Квадрат суммы}&(a+b)^2& =&a^2+2ab+b^2,\\
		\text{Квадрат разности}&(a-b)^2& =&a^2-2ab+b^2,\\
		\text{Сумма кубов}&(a+b)(a^2-ab+b^2)& =&a^3+b^3,\\
		\text{Разность кубов}&(a-b)(a^2+ab+b^2)& =&a^3-b^3,\\
		\text{Куб суммы}&(a+b)^3& =&a^3+3a^2b+3ab^2+b^3,\\
		\text{Куб разности}&(a-b)^3& =&a^3-3a^2b+3ab^2-b^3.\\
	\end{array} \]
	\item Упростить дробь:
	\begin{enumcols}[itemcolumns=3]
		\item \exercise{109}
		\item \exercise{113}
		\item \exercise{118}
		\item \exercise{121}
		\item \exercise{122}
		\item \exercise{126}
	\end{enumcols}
	\item Вычислить значение выражения:
	\begin{enumcols}[itemcolumns=1]
		\item \exercise{1223}
		\item \exercise{642}
	\end{enumcols}
	\newpage
	\item Представить в виде несократимой дроби:
	\begin{enumcols}[itemcolumns=2]
		\item \exercise{130}
		\item \exercise{136}
		\item \exercise{826}
		\item \exercise{827}
		\item \exercise{829}
		\item \exercise{832}
		\item \exercise{835}
		\item \exercise{836}
	\end{enumcols}
	\item Представить в виде несократимой дроби:
	\begin{enumcols}[itemcolumns=3]
		\item \exercise{837}
		\item \exercise{842}
		\item \exercise{844}
	\end{enumcols}
	\item Представить в виде несократимой дроби:
	\begin{enumcols}[itemcolumns=3]
		\item \exercise{848}
		\item \exercise{851}
		\item \exercise{855}
	\end{enumcols}
\end{listofex}