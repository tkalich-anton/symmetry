%
%===============>> ГРУППА 10-1 МОДУЛЬ 3 <<=============
%
\setmodule{3}
%
%===============>>  Занятие 5  <<===============
%
\begin{class}[number=5]
	\begin{listofex}
		\item \textbf{Формулы суммы/разности синуса или косинуса:}
		\begin{enumcols}[itemcolumns=2]
			\item \( \sin(x+y)=\sin x\cos y + \sin y \cos x \)
			\item \( \sin(x-y)=\sin x\cos y - \sin y \cos x \)
			\item \( \cos(x+y)=\cos x \cos y - \sin x \sin y \)
			\item \( \cos(x-y)=\cos x \cos y + \sin x \sin y \)
		\end{enumcols}
		\item \textbf{Метод приведения аргумента тригонометрических функций:}
		\begin{enumcols}
			\item[0)] Выносим минус за знак аргумента;
			\item "Убираем"{ }полные круги из аргумента \textit{(в будущем не обязательно);}
			\item Представляем аргумент в виде суммы/разности так, чтобы одно слагаемое было кратно \( 90 \), а другое было табличным значением (\( 30\degree;\;45\degree;\;60\degree \));
			\item Определяем четверть аргумента \textit{(меньшее слагаемое всегда принимаем за острый угол);}
			\item Определяем знак функции в этой четверти;
			\item Меняем или оставляем название тригонометрической функции (\( 0\degree;\;180\degree \) --- не меняем название функции; \( 90\degree;\;270\degree \) --- меняем название функции на противоположное).
		\end{enumcols}
		\item Вычислить с помощью метода приведения:
		\[ \sin135\degree;\;\cos240\degree;\;\sin390\degree;\;\tg150\degree;\;\ctg220\degree;\;\sin(-220\degree);\;\tg840\degree;\;\cos(-240\degree);\;\sin315\degree \]
		\item Перевести градусы в радианы:
		\begin{enumcols}[itemcolumns=5]
			\item \( 90\degree \)
			\item \( 120\degree \)
			\item \( 60\degree \)
			\item \( 45\degree \)
			\item \( 30\degree \)
			\item \( 210\degree \)
			\item \( 270\degree \)
			\item \( 360\degree \)
			\item \( 225\degree \)
			\item \( 330\degree \)
			\item \( 390\degree \)
			\item \( 150\degree \)
			\item \( 810\degree \)
			\item \( 210\degree \)
			\item \( 300\degree \)
		\end{enumcols}
		\item Перевести радианы в градусы:
		\begin{enumcols}[itemcolumns=5]
			\item \( \dfrac{\pi}{2} \)
			\item \( \dfrac{3\pi}{2} \)
			\item \( \dfrac{5\pi}{4} \)
			\item \( \dfrac{7\pi}{6} \)
			\item \( \dfrac{14\pi}{2} \)
			\item \( \dfrac{36\pi}{9} \)
			\item \( \dfrac{11\pi}{3} \)
			\item \( \dfrac{5\pi}{3} \)
			\item \( \dfrac{9\pi}{3} \)
			\item \( \dfrac{45\pi}{6} \)
			\item \( \dfrac{7\pi}{4} \)
			\item \( \dfrac{13\pi}{6} \)
			\item \( \dfrac{55\pi}{4} \)
			\item \( \dfrac{15\pi}{5} \)
			\item \( \dfrac{21\pi}{4} \)
		\end{enumcols}
		\item Вычислить с помощью метода приведения:
		\[ \cos\dfrac{5\pi}{4};\;\sin\dfrac{7\pi}{3};\;\sin\dfrac{3\pi}{2};\;\sin\left( -\dfrac{5\pi}{3} \right);\;\cos\dfrac{7\pi}{6};\;\sin\dfrac{13\pi}{4};\;\sin\left( -\dfrac{7\pi}{6}  \right);\;\cos\dfrac{21\pi}{4};\;\tg\dfrac{16\pi}{6};\;\ctg\dfrac{11\pi}{4} \]
		\item Вычислить:
		\begin{enumcols}[itemcolumns=2]
			\item \( \dfrac{\sqrt{3}}{\sin60\degree}+\dfrac{3}{\sin30\degree} \)
			\item \( \dfrac{-13\sin126\degree}{\sin54\degree} \)
			\item \( \sin^223\degree+9+\cos^223 \)
			\item \( 2\sin30\degree-\sqrt{3}\sin60\degree\ctg45\degree\tg30\degree\)
			\item \( \dfrac{6\sin30\degree\cos30\degree}{\cos^230\degree-\sin^230\degree} \)
			% Занятие 6 Галицкий стр. 187 13.1 б) 13.2 б) формат задач с Решу ЕГЭ
		\end{enumcols}
		\item Найти значение выражения:
		\begin{enumcols}[itemcolumns=1]
			\item \exercise{2965}
			\item \exercise{1116}
			\item \exercise{1117}
			%\item \exercise{2874}
			%\item \exercise{2856}
			%\item \exercise{2883}
			%\item \exercise{2928}
			\item \exercise{2906}
		\end{enumcols}
	\end{listofex}
\end{class}
%
%===============>>  Занятие 6  <<===============
%
%\begin{class}[number=6]
%	\begin{listofex}
%	
%	\end{listofex}
%\end{class}
%
%===============>>  Домашняя работа 3  <<===============
%
%\begin{homework}[number=3]
%	\begin{listofex}
%
%	\end{listofex}
%\end{homework}
%\newpage
%\title{Подготовка к проверочной работе}
%\begin{listofex}
%	
%\end{listofex}
%
%===============>>  Занятие 7  <<===============
%
%\begin{class}[number=7]
%	\begin{listofex}
%	
%	\end{listofex}
%\end{class}
%
%===============>>  Провечная работа  <<===============
%
%\begin{exam}
%	\begin{listofex}
%	
%	\end{listofex}
%\end{exam}