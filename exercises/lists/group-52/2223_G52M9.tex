%
%===============>>  ГРУППА 5-2 МОДУЛЬ 8  <<=============
%
\setmodule{9}

%BEGIN_FOLD % ====>>_____ Занятие 1 _____<<====
\begin{class}[number=1]
	\begin{listofex}
		\item  Округлите :
		\begin{tasks}(1)
			\task до десятых \( 8,174 \); \( 0,32 \); \(  0,2829 \); \( 4,1046 \)
			\task до сотых \( 2,1873 \); \( 127,314 \); \( 103,1309 \);  \( 121,3476 \)
			\task до тысячных \( 2,34516 \); \( 0,2172 \);  \( 7,15671 \); \( 132,78503 \)
			\task до десятков \( 48276,0031 \); \( 58307,8762 \);  \( 78413,9871 \); \( 385125,00403 \)
			\task до сотен \( 436824,464 \); \( 577884,3567 \);  \( 986535,871 \); \( 5707231,752 \)
		\end{tasks}
		\item В магазине куртки продавались по цене \( 8550 \) руб. за одну куртку. Летом на эту цену стала действовать скидка в \( 21\% \). Сколько рублей составляет скидка?
		\item Мама получила премию \( 18000,55 \) руб. На подарок дочери она потратила \( 12\% \) этой премии. Сколько рублей стоит подарок?
		\item Инженер получил за изобретение премию. \( 15\% \) премии он передал в Детский фонд. Какую премию получил инженер, если в Детский фонд он передал \( 120,99 \) рублей?
		\item На поле, площадь которого \( 620,5 \)  га, работали хлопкоуборочные машины. За сутки они убрали \( 17\% \) всего поля. Сколько гектаров хлопка они не убрали за сутки?
		\item Предприятие изготовило за квартал \( 150 \) насосов, из которых \( 60\% \) имели высшую категорию качества. Сколько насосов высшей категории качества изготовило предприятие.
		\item В школе \( 725 \) учащихся. Среди них \( 398 \) мальчиков. Сколько процентов учащихся этой школы составляют мальчики?
	\end{listofex}
\end{class}
%END_FOLD

%BEGIN_FOLD % ====>>_____ Занятие 2 _____<<====
\begin{class}[number=2]
	\begin{listofex}
		\item Округлите до:
		\begin{tasks}(3)
			\task[] \textbf{Десятых:}
			\task[]
			\task[]
			\task \( 345,73456 = \)
			\task \( 875, 954 = \)
			\task \( 24,4567 = \)
			\task[] \textbf{Десятков:}
			\task[]
			\task[]
			\task \( 8865,1021 = \)
			\task \( 7312,8001 = \)
			\task \( 3675,0862 = \)
			\task[] \textbf{Целых:}
			\task[]
			\task[]
			\task \( 6543,5001 = \)
			\task \( 7234,236 = \)
			\task \( 5426,499 = \)
		\end{tasks}
		\item  В голосовании приняли участие \( 64\% \) студентов. Какое количество студентов не пришло на выборы, если всего \( 2000 \) студентов? 
		\item Оптовая цена товара на складе \( 5500 \) р. Торговая надбавка в магазине составляет \( 12\% \). 
		Сколько стоит этот товар в магазине?  
		\item В библиотеке \( 98000  \) книг. Книги на русском языке составляют \( 78\% \) всех книг, из
		них \( 5\% \) --- учебники. Сколько учебников на русском языке в библиотеке?
		\item В школе \( 15 \) учеников учатся только на оценки «отлично». Это составляет \( 5\% \) всех учащихся
		школы. Сколько всего учащихся в школе? 
		\item За год число книг в библиотеке увеличилось на \( 10\% \) и стало равным \( 8800 \). Сколько книг
		было в библиотеке в прошлом году?
		\item Телефон стоит \( 8900 \) рублей. Чему будет равна стоимость телефона, если цену повысить на \( 14\% \)?
		\item Хоккейные коньки стоили \( 4500 \) руб. Сначала цену снизили на \( 20\% \), а потом эту сниженную цену повысили на \( 20\% \). Сколько стали стоить коньки после повышения цены?
		\item Цена на учебник понизилась на \( 9\% \) и стала равна \( 521,43  \) рубля. Сколько стоил учебник до понижения цены?
		\item Цена на чайник повысилась на \( 11\% \) и стала равна \( 1391,94  \) рубля. Сколько стоил чайник до повышения цены?
	\end{listofex}
\end{class}
%END_FOLD

%BEGIN_FOLD % ====>>_ Домашняя работа 1 _<<====
\begin{homework}[number=1]
	\begin{listofex}
		\item Округлите до:
		\begin{tasks}(3)
			\task[] \textbf{Десятых:}
			\task[]
			\task[]
			\task \( 734,725 = \)
			\task \( 9013, 3623 = \)
			\task[]
			\task[] \textbf{Единиц:}
			\task[]
			\task[]
			\task \( 2136,3459 = \)
			\task \( 7312,7427 = \)
			\task[]
			\task[] \textbf{Десятков:}
			\task[]
			\task[]
			\task \( 3568,5672 = \)
			\task \( 4256,0097 = \)
			\task[]
			\task[] \textbf{Сотых:}
			\task[]
			\task[]
			\task \( 8562,4582 = \)
			\task \( 530,3242 = \)
		\end{tasks}
		\item \( 15\% \) числа равны 75. Чему равно всё число?
		\item Организм взрослого человека на \( 60\% \) состоит из воды. Какова масса воды в теле человека, который весит \( 87,3 \) кг?
		\item В грушах сладких сортов содержится сахара \( 15\% \) от их массы. Сколько кг сахара будет содержаться в \( 6,35 \) кг груш?
	\end{listofex}
\end{homework}
%END_FOLD

%BEGIN_FOLD % ====>>_____ Занятие 3 _____<<====
\begin{class}[number=3]
	\begin{listofex}
		\item Мотоцикл стоил \( 56000 \) руб. Сначала цену повысили на \( 24\% \), а затем еще на \( 30\% \). Определите, сколько стал стоить мотоцикл после второго повышения цены.
		\item Цена на лопату резко повысилась на \( 15\% \), после чего понизилась на \( 20\% \). Определите, сколько стоила лопата изначально, если после всех изменений она стала стоить \( 92 \) руб?
		\item Иван собрался в поход на лыжах. В первый день он прошел \( 25\% \) пути, а во второй день — на \( 10\% \) больше, чем в первый. Определите, сколько осталось пройти Ивану, если он запланировал всего пройти \( 540 \) км?
		\item Бригада рабочих за первый день сделала \( 25\% \) от запланированного количества деталей, а во второй день — \( 40\% \) от оставшегося количества. Определите, сколько деталей запланировала сделать бригада рабочих, если во второй день они сделали \( 120 \) деталей.
		\item Яхта «Гайда» прошла в первую неделю \( 28\% \) от запланированного пути, а во вторую — на \( 16\% \) меньше, чем в первую. Определите, сколько км прошла яхта за первые \( 2 \) недели, если всего длина маршрута составляет \( 6400 \) км.
		\item Тарас взял в долг у приятеля в сентябре. Каждый месяц, начиная с октября, он выплачивает \( 25\% \) от оставшейся суммы. Определите, какую сумму взял в долг у своего приятеля Тарас, если он заплатил в ноябре \( 3000 \) руб.
		\item Фома, выполняя свою домашнюю работу, тратит на выполнение задания по математике \( 60\% \) времени, \( 20\% \) от оставшегося времени у него уходит на задание по литературе. Определите, сколько всего времени уходит у Фомы на выполнение домашнего задания, если на выполнение задания по литературе у него уходит 40 минут. Ответ укажите в минутах.
		\item Георгий в декабре весил \( 72 \) кг. В январе он стал весить на \( 10\% \) больше, а в феврале он прибавил в весе еще на \( 18 \) кг. Определите, сколько процентов от декабрьского веса составляет вес Георгия в феврале.
		\item Путник собрался пройти \( 40  \) км. В первый день он прошел \( 25\% \) от назначенного пути, а во второй день — на \( 20\% \) меньше, чем в первый, определите, сколько процентов составляет оставшаяся часть пути от первоначальной.
		
	\end{listofex}
\end{class}
%END_FOLD

%BEGIN_FOLD % ====>>_____ Занятие 4 _____<<====
\begin{class}[number=4]
	\begin{listofex}
		\item Занятие 4
	\end{listofex}
\end{class}
%END_FOLD

%BEGIN_FOLD % ====>>_ Домашняя работа 2 _<<====
\begin{homework}[number=2]
	\begin{listofex}
		\item \begin{tasks}(2)
			\task \( 14-6,7 \)
			\task \( 432,3487\cdot 1000 \)
			\task \( 92,9\cdot4,63 \)
			\task \( 507,36:14 \)
			\task \( 276,288:86,34 \)
		\end{tasks}
		\item  Тракторист должен вспахать поле площадью \( 25 \) га. В первый день он вспахал \( 32\% \) поля. Сколько гектаров он вспахал?
		\item Сумма трех чисел равна \( 520 \). Первое число составляет \( 24\% \) всей суммы, второе число составляет \( 20\% \) всей суммы. Найдите третье число.
	\end{listofex}
\end{homework}
%END_FOLD

%BEGIN_FOLD % ====>>_____ Занятие 5 _____<<====
\begin{class}[number=5]
	\begin{listofex}
		\item По определению процента найдите часть от целого: 
		\begin{tasks}(3)
			\task \( 23\%  \) от \( 300 \) 
			\task \( 76\%  \) от \( 400 \)   
			\task \( 120\%  \) от \( 500 \)   
			\task \( 20\%  \) от \( 150 \)   
			\task \( 90\%  \) от \( 330 \)   
			\task \( 55\%  \) от \( 78 \)   
			\task \( 68\%  \) от \( 1550 \)   
			\task \( 56\%  \) от \( 1025 \)   
			\task \( 50\%  \) от \( 432\)   
		\end{tasks}
		\item Найдите \( 57\% \) от числа:\begin{tasks}(4)
			\task \( 100 \)
			\task \( 300 \)
			\task \( 550 \)
			\task \( 435,3 \)
		\end{tasks}
		Объясните, почему во всех случаях мы находим один и тот же процент, а ответы получаются
		различные?
		\item Найдите часть от целого: \begin{tasks}(3)
			\task \( 10\% \) от \( 13 \)
			\task \( 20\% \) от \( 45 \)
			\task \( 25\% \) от \( 44 \)
			\task \( 50\% \) от \( 114 \)
			\task \( 25\% \) от \( 112 \)
			\task \( 50\% \) от \( 45 \)
			\task \( 20\% \) от \( 115 \)
			\task \( 5\% \) от \( 60 \)
			\task \( 15\% \) от \( 60 \)
		\end{tasks} 
		\item Билет в театр стоит \( 2500 \) рублей, а стоимость билета в кино составляет \( 20\% \) от стоимости билета в театр. Сколько стоит билет в кино?
		\item В магазине \( 240  \) кг фруктов. За день продали \( 45\% \). Сколько килограммов фруктов осталось?
		\item Бегун пробежал  \( 300 \) м, что составляет \( 60\% \) всей его намеченной дистанции. Найдите длину дистанции.
		\item  Пиджак стоит \( 5000  \) рублей. В связи с поступлением новой коллекции пиджак продают со скидкой \( 70\% \). Сколько стоит пиджак с учётом скидки? 
		\item  Сплав состоит из \( 10\% \) олова, \( 35\% \) меди и \( 55\% \) свинца. Сколько каждого металла содержится в \( 3 \) кг сплава. Ответ дайте в граммах. 
		\item  Себестоимость товара составляет \( 75\% \) от его розничной цены. Товар продаётся по цене \( 1500 \) рублей. Какова его себестоимость?
		
	\end{listofex}
\end{class}
%END_FOLD

%BEGIN_FOLD % ====>>_____ Занятие 6 _____<<====
\begin{class}[number=6]
	\begin{listofex}
		\item Переведите проценты в десятичную дробь: 
		\begin{tasks}(4)
			\task \( 2\% \)
			\task \( 75\% \)
			\task \( 30\% \)
			\task \( 1,7\% \)
			\task \( 0,8\% \)
			\task \( 0,03\% \)
			\task \( 0,009\% \)
			\task \( 133\% \)
		\end{tasks}
		\item Переведите дроби в проценты:
		\begin{tasks}(4)
			\task \( 0,01 \)
			\task \( 0,012 \)
			\task \( 0,45 \)
			\task \( 0,03 \)
			\task \( 0,11 \)
			\task \( 0,0003 \)
			\task \( 1 \)
			\task \( 4 \)
		\end{tasks}
		\item По определению процента найдите часть от целого: 
		\begin{tasks}(3)
			\task \( 10\%  \) от \( 500 \) 
			\task \( 25\%  \) от \( 1000 \)   
			\task \( 23\%  \) от \( 2000 \)   
			\task \( 78\%  \) от \( 5000 \)   
			\task \( 34\%  \) от \( 6160 \)   
			\task \( 56\%  \) от \( 12 \)   
			\task \( 12\%  \) от \( 23 \)   
			\task \( 13\%  \) от \( 2 \)   
			\task \( 14\%  \) от \( 3\)   
		\end{tasks}
		\item Найдите \( 45\% \) от числа:\begin{tasks}(4)
			\task \( 100 \)
			\task \( 300 \)
			\task \( 550 \)
			\task \( 435,3 \)
		\end{tasks}
		\item Найдите часть от целого: \begin{tasks}(3)
			\task \( 95\% \) от \( 120 \)
			\task \( 2\% \) от \( 250 \)
			\task \( 4\% \) от \( 125 \)
			\task \( 98\% \) от \( 750 \)
			\task \( 96\% \) от \( 150 \)
			\task \( 90\% \) от \( 45 \)
			\task \( 50\% \) от \( 57 \)
			\task \( 15\% \) от \( 140 \)
			\task \( 25\% \) от \( 520 \)
		\end{tasks} 
		
		\item У брата и сестры \( 90 \) марок. Сколько марок у сестры, если у брата \( 30\% \) всех марок?
		\item Масса овцы \( 86,5 \) кг. Масса одного ягненка составляет \( 20\% \) массы овцы. Какова масса одной овцы и шести ягнят вместе?
		\item На школьной выставке \( 72 \) рисунка. Выполнено акварелью \( \dfrac{5}{6} \) всех рисунков, а \( \dfrac{1}{4} \) от остатка выполнена карандашами. Сколько карандашных рисунков на выставке?
		\item Число жителей города \( 750 000 \) человек. Ежегодно население в нем увеличивается на \(  2\% \). Сколько жителей будет в городе через \( 1 \) год?
		\item По норме рабочий должен изготовить \( 45 \) деталей. Он выполнил норму на \( 120\% \). Сколько деталей изготовил рабочий?
	\end{listofex}
\end{class}
%END_FOLD

%BEGIN_FOLD % ====>>_ Домашняя работа 3 _<<====
\begin{homework}[number=3]
	\begin{listofex}
		\item Найдите: \begin{tasks}(3)
			\task \( 15\% \) от \( 220 \)
			\task \( 50\% \) от \( 1 \)
			\task \( 25\% \) от \( 3 \)
			\task \( 20\% \) от \( 2 \)
			\task \( 75\% \) от \( 32 \)
			\task \( 40\% \) от \( 65 \)
		\end{tasks}
	\item Найдите: 
	\begin{tasks}(3)
		\task \( 98\% \) от \( 4 \)
		\task \( 34\% \) от \( 192 \)
		\task \( 12\% \) от \( 324 \)
		\task \( 4,5 \% \) от \( 98 \)
		\task \( 1,07\% \) от \( 231 \)
		\task \( 121\% \) от \( 341 \)
	\end{tasks} 
		\item У брата и сестры \( 180 \) марок. Сколько марок у сестры, если у брата \( 45\% \) всех марок?
		\item  Пиджак стоит \( 5250 \) рублей. В магазине проводят распродажу со скидкой \( 15\% \). Сколько стоит пиджак на распродаже? 
	\end{listofex}
\end{homework}
%END_FOLD

%BEGIN_FOLD % ====>>_____ Занятие 7 _____<<====
\begin{class}[number=7]
	\begin{listofex}
		\item Найдите: \begin{tasks}(3)
			\task \( 45,5\% \) от \( 32 \)
			\task \( 23,8\% \) от \( 209 \)
			\task \( 12,4\% \) от \( 65 \)
			\task \( 56,2\% \) от \( 124 \)
			\task \( 60,5\% \) от \( 3 \)
			\task \( 29,1\% \) от \( 781 \)
			\task \( 33,05\% \) от \( 1002 \)
			\task \( 2,2\% \) от \( 64 \)
			\task \( 64,6\% \) от \( 133 \)
		\end{tasks}
		\item Найдите целое число, если: \begin{tasks}(3)
			\task \( 1\% \) равен \( 3 \)
			\task \( 2\% \) равены \( 8 \)
			\task \( 3\% \) равены \( 15 \)
			\task \( 4\% \) равены \( 28 \)
			\task \( 15\% \) равены \( 45 \)
			\task \( 19\% \) равены \( 57 \)
			\task \( 20\% \) равены \( 140 \)
			\task \( 35\% \) равены \( 105 \)
			\task \( 23\% \) равены \( 115 \)
		\end{tasks}
		\item Найдите, сколько процентов составляет первое число от второго:
		\begin{tasks}(3)
			\task \( 40 \) от \( 400 \)
			\task \( 60 \) от \( 200 \)
			\task \( 80 \) от \( 500 \)
			\task \( 77 \) от \( 1100 \)
			\task \( 56 \) от \( 1900 \)
			\task \( 25 \) от \( 500 \)
			\task \( 33 \) от \( 400 \)
			\task \( 57 \) от \( 600 \)
			\task \( 300 \) от \( 300 \)
		\end{tasks}
		\item На сколько процентов \begin{tasks}(3)
			\task \( 250 \) больше \( 200 \)
			\task \( 180 \) меньше \( 300 \)
			\task \( 630 \) больше \( 600 \)
			\task \( 382,5 \) меньше \( 450 \)
			\task \( 100 \) больше \( 80 \)
			\task \( 80 \) меньше \( 100 \)
			\task \( 230 \) больше \( 200 \)
			\task \( 451 \) меньше \( 550 \)
			\task \( 305 \) больше \( 250 \)
		\end{tasks}
		\item Себестоимость товара составляет \( 75\% \) от его розничной цены. Товар продаётся по цене \( 1500 \) рублей. Какова его себестоимость?
		\item Цена на чайник повысилась на \( 11\% \) и стала равна \( 1332 \) рубля. Сколько стоил чайник до повышения цены? 
		\item На субботник вышли \( 160  \) человек. Всего \( 75\% \) людей убирали территорию, остальные сажали деревья. Сколько человек сажали деревья?
		\item Туристы прошли \( 27,5 \) км, что составляет \( 25\% \) всего пути. Какова длина всего пути, который прошли туристы?
		\item Завод выпустил \( 150 \) холодильниковю \( \dfrac{2}{5} \) этих холодильников было отправлено в больницы, а \( 60\% \) из того, что осталось, ушло в детские сады. Сколько  холодильников было отправлено в денские сады?
		\item В первый час машина прошла \( 27\% \) намеченного пути, после ччего ей осталось пройти \( 146 \) км. Сколько километров составляет длина намеченного пути?
		\item Фабрика получила некоторый заказ. За неделю заказ был выполнен на \( 40\% \) , причем было выпущено \( 120 \) деталей. Сколько деталей нужно выполнить по плану?
	\end{listofex}
\end{class}
%END_FOLD

%BEGIN_FOLD % ====>>_ Проверочная работа _<<====
\begin{class}[number=8]
	\begin{listofex}
			\item Округлите: 
			\begin{tasks}(2)
				\task \( 123,45 \) до десятых
				\task \(  682,9725 \) до сотых
				\task \( 9525,204 \) до сотых
				\task \( 6675,98731 \) до тысячных
				\task \( 7,0999701 \) до тысячных
				\task \( 11,000875\) до стотысячных
			\end{tasks}
			\item Вычислите и полученный результат округлите: 
			\begin{tasks}(1)
				\task \( 38,49 : 0,1 \), до единиц, до сотен
				\task \( 56,99 : 0,01  \), до десятков, до десятков тысяч
				\task \( 65,3 \cdot3,21 \), до сотых
			\end{tasks}
			\item Вычислите и полученный результат округлите: 
			\begin{tasks}(3)
				\task \( 20,7 :9 \) 
				\task \( 243,2:8 \)
				\task \( 88,298: 7 \)
				\task \( 772,8:12 \)
				\task \( 93,15: 23 \)
				\task \( 0,644:92 \)
			\end{tasks}
			\item  В голосовании приняли участие \( 64,3\% \) студентов. Какое количество студентов не
			пришло на выборы, если всего \( 2000 \) студентов? 
			\item  Морская вода содержит \( 8\% \) соли (по массе). Сколько в морской воде пресной воды? 
			\item Оптовая цена товара на складе \( 5500 \) р. Торговая надбавка в магазине составляет \( 12\% \).Сколько стоит этот товар в магазине?  
			\item В библиотеке \( 98000  \) книг. Книги на русском языке составляют \( 78\% \) всех книг, из
			них \( 5\% \) --- учебники. Сколько учебников на русском языке в библиотеке?
			\item В школе \( 15 \) учеников учатся только на оценки «отлично». Это составляет \( 5\% \) всех учащихся
			школы. Сколько всего учащихся в школе? 
	\end{listofex}
\end{class}
%END_FOLD

%BEGIN_FOLD % ====>>_ Консультация _<<====
\begin{consultation}
	\begin{listofex}
		\item Вычислить:
		\begin{tasks}(4)
			\task \( 20,7:9 \)
			\task \( 243,2:8 \)
			\task \( 7,368:24 \)
			\task \( 25:125 \)
			\task \( 1:8 \)
			\task \( 72,57:59 \)
			\task \( 0,7:25 \)
			\task \( 6,78:26 \)
		\end{tasks}
		\item Вычислить:
		\begin{tasks}(4)
			\task \( 45,5:10 \)
			\task \( 45,5:1000 \)
			\task \( 45,5:10000 \)
			\task \( 89:10 \)
			\task \( 89:100 \)
			\task \( 32,2:10 \)
			\task \( 7,98:10 \)
			\task \( 47,7:1000 \)
			\task \( 0,911:1000 \)
		\end{tasks}
		\item Вычислить:
		\begin{tasks}(4)
			\task \( 2:0,4 \)
			\task \( 70:1,75 \)
			\task \( 24:0,2 \)
			\task \( 2:0,5 \)
			\task \( 45:0,05 \)
			\task \( 125:2,5 \)
			\task \( 484:0,004 \)
			\task \( 5,1:0,17 \)
			\task \( 25,2:0,4 \)
			\task \( 200,1:0,69 \)
		\end{tasks}
		\item Путь у Ксюши до дома занимает \( 1,6 \) км. Утром она обычно опаздывает в школу, поэтому бежит. В этом случае время, которое ей необходимо на дорогу до школы, составляет \( 0,25 \) часа. Обратно со школы Ксюша не торопится и идет домой в течение \( 0,5 \) часа. С какой скоростью Ксюша бежит в школу утром и с какой скоростью она идет домой после уроков?
		\item Вася случайно разрезал моток веревки на две части. Длина одной из них --- \( 3,25 \) м, длина другой в \( 1,3 \) раза короче. Необходимо найти общую длину исходной веревки.
		\item Вася живет в комнате, которая имеет форму прямоугольного параллелепипеда. Известно, что объем этой комнаты равен \( 37,84 \) м\( ^3 \). При этом длина комнаты составляет \( 4 \) м, а высота потолка --- \( 2,2 \) м. Нужно найти ширину комнаты. 
		\item Кенгуру ниже жирафа в \( 2,4 \) раза, при этом жираф выше кенгуру на \( 2,52 \) м. Найдите рост жирафа и кенгуру.
	\end{listofex}
\end{consultation}
%END_FOLD

%BEGIN_FOLD % ====>>_ Домашняя работа 4 _<<====
\begin{homework}[number=4]
	\begin{listofex}
		\item Вычислите: \begin{tasks}(3)
			\task \( 0,65\cdot0,1 \)
			\task \( 4,2031\cdot0,01 \)
			\task \( 333,003\cdot0,001 \)
			\task \( 98,21:1000 \)
			\task \( 3215,005:10000 \)
			\task \( 84,9:100 \)
		\end{tasks}
		\item Выполните деление:
		\begin{tasks}(3)
			\task \( 7,368: 24 \)
			\task \( 138,16:11 \) 
			\task \( 170,986:34 \)  
			\task \( 1,953:14 \)  
			\task \( 0,11611:17 \)  
			\task \( 8,25:33 \)  
		\end{tasks}
		\item Вычислите и полученный результат округлите: 
		\begin{tasks}(1)
			\task \( 23,67\cdot 0,1 \), до сотых
			\task \( 99,7891\cdot 1000  \), до сотен, до десятков тысяч
			\task \( 5,002 \cdot4,003 \), до сотых, до стотысячных
		\end{tasks}
		\item Максим построил у себя в тетради прямоугольник со сторонами, равными \( 1,8 \) см и \( 5,6 \) см. Найдите периметр и площадь фигуры. Что больше? На сколько?
		\item Найдите сумму площадей стен комнаты, длина которой \( 6,4 \) м, ширина \( 3,5 \) м и высота \( 2,69 \) м. 
		\item  Пиджак стоит \( 5000 \) рублей. В связи с поступлением новой коллекции пиджак продают со скидкой \( 70\% \). Сколько стоит пиджак с учётом скидки?
	\end{listofex}
\end{homework}
%END_FOLD