%
%===============>>  ГРУППА 5-2 МОДУЛЬ 8  <<=============
%
\setmodule{8}

%BEGIN_FOLD % ====>>_____ Занятие 1 _____<<====
\begin{class}[number=1]
	\begin{listofex}
		\item .
	\end{listofex}
\end{class}
%END_FOLD

%BEGIN_FOLD % ====>>_____ Занятие 2 _____<<====
\begin{class}[number=2]
	\begin{listofex}
		\item .
	\end{listofex}
\end{class}
%END_FOLD

%BEGIN_FOLD % ====>>_ Домашняя работа 1 _<<====
\begin{homework}[number=1]
	\begin{listofex}
		\item \.
	\end{listofex}
\end{homework}
%END_FOLD

%BEGIN_FOLD % ====>>_____ Занятие 3 _____<<====
\begin{class}[number=3]
	\begin{listofex}
		\item Занятие 3 
	\end{listofex}
\end{class}
%END_FOLD

%BEGIN_FOLD % ====>>_____ Занятие 4 _____<<====
\begin{class}[number=4]
	\begin{listofex}
		\item Занятие 4
	\end{listofex}
\end{class}
%END_FOLD

%BEGIN_FOLD % ====>>_ Домашняя работа 2 _<<====
\begin{homework}[number=2]
	\begin{listofex}
		\item Домашняя работа 2
	\end{listofex}
\end{homework}
%END_FOLD

%BEGIN_FOLD % ====>>_____ Занятие 5 _____<<====
\begin{class}[number=5]
	\begin{listofex}
		\item Занятие 5
	\end{listofex}
\end{class}
%END_FOLD

%BEGIN_FOLD % ====>>_____ Занятие 6 _____<<====
\begin{class}[number=6]
	\begin{listofex}
		\item Занятие 6
	\end{listofex}
\end{class}
%END_FOLD

%BEGIN_FOLD % ====>>_ Домашняя работа 3 _<<====
\begin{homework}[number=3]
	\begin{listofex}
		\item Домашняя работа 3
	\end{listofex}
\end{homework}
%END_FOLD

%BEGIN_FOLD % ====>>_____ Занятие 7 _____<<====
\begin{class}[number=7]
	\title{Подготовка к проверочной}
	\begin{listofex}
		\item Занятие 7
	\end{listofex}
\end{class}
%END_FOLD

=%BEGIN_FOLD % ====>>_ Проверочная работа _<<====
\begin{exam}
	\begin{listofex}
		\item Проверочная
	\end{listofex}
\end{exam}
%END_FOLD

%BEGIN_FOLD % ====>>_ Консультация _<<====
\begin{consultation}
	\begin{listofex}
		\item Вычислить:
		\begin{tasks}(4)
			\task \( 20,7:9 \)
			\task \( 243,2:8 \)
			\task \( 7,368:24 \)
			\task \( 25:125 \)
			\task \( 1:8 \)
			\task \( 72,57:59 \)
			\task \( 0,7:25 \)
			\task \( 6,78:26 \)
		\end{tasks}
		\item Вычислить:
		\begin{tasks}(4)
			\task \( 45,5:10 \)
			\task \( 45,5:1000 \)
			\task \( 45,5:10000 \)
			\task \( 89:10 \)
			\task \( 89:100 \)
			\task \( 32,2:10 \)
			\task \( 7,98:10 \)
			\task \( 47,7:1000 \)
			\task \( 0,911:1000 \)
		\end{tasks}
		\item Вычислить:
		\begin{tasks}(4)
			\task \( 2:0,4 \)
			\task \( 70:1,75 \)
			\task \( 24:0,2 \)
			\task \( 2:0,5 \)
			\task \( 45:0,05 \)
			\task \( 125:2,5 \)
			\task \( 484:0,004 \)
			\task \( 5,1:0,17 \)
			\task \( 25,2:0,4 \)
			\task \( 200,1:0,69 \)
		\end{tasks}
		\item Путь у Ксюши до дома занимает \( 1,6 \) км. Утром она обычно опаздывает в школу, поэтому бежит. В этом случае время, которое ей необходимо на дорогу до школы, составляет \( 0,25 \) часа. Обратно со школы Ксюша не торопится и идет домой в течение \( 0,5 \) часа. С какой скоростью Ксюша бежит в школу утром и с какой скоростью она идет домой после уроков?
		\item Вася случайно разрезал моток веревки на две части. Длина одной из них --- \( 3,25 \) м, длина другой в \( 1,3 \) раза короче. Необходимо найти общую длину исходной веревки.
		\item Вася живет в комнате, которая имеет форму прямоугольного параллелепипеда. Известно, что объем этой комнаты равен \( 37,84 \) м\( ^3 \). При этом длина комнаты составляет \( 4 \) м, а высота потолка --- \( 2,2 \) м. Нужно найти ширину комнаты. 
		\item Кенгуру ниже жирафа в \( 2,4 \) раза, при этом жираф выше кенгуру на \( 2,52 \) м. Найдите рост жирафа и кенгуру.
	\end{listofex}
\end{consultation}
%END_FOLD