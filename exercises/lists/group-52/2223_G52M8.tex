%
%===============>>  ГРУППА 5-2 МОДУЛЬ 8  <<=============
%
\setmodule{8}

%BEGIN_FOLD % ====>>_____ Занятие 1 _____<<====
\begin{class}[number=1]
	\begin{listofex}
		\item d
	\end{listofex}
\end{class}
%END_FOLD

%BEGIN_FOLD % ====>>_____ Занятие 2 _____<<====
\begin{class}[number=2]
	\begin{listofex}
		\item .
	\end{listofex}
\end{class}
%END_FOLD

%BEGIN_FOLD % ====>>_ Домашняя работа 1 _<<====
\begin{homework}[number=1]
	\begin{listofex}
		\item f
	\end{listofex}
\end{homework}
%END_FOLD

%BEGIN_FOLD % ====>>_____ Занятие 3 _____<<====
\begin{class}[number=3]
	\begin{listofex}
		\item Занятие 3 
	\end{listofex}
\end{class}
%END_FOLD

%BEGIN_FOLD % ====>>_____ Занятие 4 _____<<====
\begin{class}[number=4]
	\begin{listofex}
		\item Занятие 4
	\end{listofex}
\end{class}
%END_FOLD

%BEGIN_FOLD % ====>>_ Домашняя работа 2 _<<====
\begin{homework}[number=2]
	\begin{listofex}
		\item Домашняя работа 2
	\end{listofex}
\end{homework}
%END_FOLD

%BEGIN_FOLD % ====>>_____ Занятие 5 _____<<====
\begin{class}[number=5]
	\begin{listofex}
		\item Занятие 5
	\end{listofex}
\end{class}
%END_FOLD

%BEGIN_FOLD % ====>>_____ Занятие 6 _____<<====
\begin{class}[number=6]
	\begin{listofex}
		\item Выполните действия:
		\begin{tasks}(1)
			\task \( 11,47+(3,89-2,11)-4,416+3,711 \)
			\task \( 3,16+(7,84-4,181)-3,11+14,816 \)
			\task \( 1,49+(6,13-4,12)-0,5+7,289 \)
		\end{tasks}
		\item Вычислите:
		\begin{tasks}(3)
			\task \( \mfrac{3}{4}{5}-\mfrac{1}{1}{5} \)
			\task \( \mfrac{21}{2}{6}-\mfrac{14}{1}{6} \)
			\task \( \mfrac{4}{7}{15}-\mfrac{1}{12}{15} \)
			\task \( \mfrac{30}{4}{16}-\mfrac{29}{1}{16} \)
			\task \( \mfrac{5}{3}{8}-\mfrac{5}{1}{8} \)
			\task \( \mfrac{11}{10}{13}-\mfrac{3}{8}{13} \)
		\end{tasks}
		\item Вычислите:
		\begin{tasks}(3)
			\task \( 4-\mfrac{3}{2}{5} \)
			\task \( 12-\mfrac{5}{6}{12} \)
			\task \( 11-\mfrac{6}{23}{25} \)
			\task \( 43-\mfrac{15}{1}{2} \)
			\task \( 22-\mfrac{1}{6}{17} \)
			\task \( 11-\mfrac{10}{2}{3} \)
		\end{tasks}
		\item Вычислите:
		\begin{tasks}(3)
			\task \( 23,4\cdot100 \)
			\task\( 4,5 \cdot100 \)
			\task \( 2,345\cdot100 \)
			\task\( 8,09\cdot0,001 \) 
			\task\( 25,043 \cdot0,001  \)
			\task\( 29,01\cdot0,001   \)
		\end{tasks}
		\item Вычислите:
		\begin{tasks}(3)
			\task \( 66,103:100 \)
			\task \( 123,2:100 \)
			\task \( 493,3:1000 \)
			\task \( 53235,03:10000 \)
			\task \( 23:100 \)
			\task \( 1944,9039:10 \)
		\end{tasks}
		\item Вычислите:
		\begin{tasks}(3)
			\task \( 1,75:0,001 \)
			\task \( 0,48:0,1 \)
			\task \( 86,75:0,1\)
			\task \( 20,7:0,01 \)
			\task \( 243,2:0,001 \)
			\task \( 88,298:0,0001 \)
		\end{tasks}
		\item Максим построил у себя в тетради прямоугольник со сторонами, равными \( 1,8 \) см и \( 5,6 \) см. Найдите периметр и площадь фигуры. Что больше? На сколько?
		
	\end{listofex}
\end{class}
%END_FOLD

%BEGIN_FOLD % ====>>_ Домашняя работа 3 _<<====
\begin{homework}[number=3]
	\begin{listofex}
		\item Вычислите 
		\begin{tasks}(3)
			\task \( 6-\mfrac{4}{1}{7} \)
			\task \( 13-\mfrac{6}{13}{25} \)
			\task \( 9-\mfrac{2}{5}{6} \)
		\end{tasks}
		\item Вычислите:
		\begin{tasks}(2)
			\task \( 0,763-0,321+5,8 \)
			\task \( 10,5-6,957+11,87 \)
		\end{tasks}
		\item Вычислите 
		\begin{tasks}(2)
			\task \( 5,8\cdot100 \)
			\task \( 6,3:0,01 \)
			\task \( 2382,35:1000 \)
			\task \( 440,8\cdot0,001 \)
		\end{tasks}
		\item Бассейн имеет форму прямоугольника со сторонами \( 5,32 \) м и \( 4,74 \) м. Чему равна его площадь?
		\item Все стороны семиугольника имеют длину \( 9,47 \) см. Найдите периметр фигуры.
	\end{listofex}
\end{homework}
%END_FOLD

%BEGIN_FOLD % ====>>_____ Занятие 7 _____<<====
\begin{class}[number=7]
	\begin{listofex}
		\item Округлить до:
		\begin{tasks}(3)
			\task[] \textbf{Десятых:}
			\task[]
			\task[]
			\task \( 6,56743 = \)
			\task \( 12, 6291 = \)
			\task \( 456,0347 = \)
			\task[] \textbf{Сотых:}
			\task[]
			\task[]
			\task \( 768,342 = \)
			\task \( 22,4581 = \)
			\task \( 345,7254 = \)
			\task[] \textbf{Десятков:}
			\task[]
			\task[]
			\task \( 324,563 = \)
			\task \( 7645,2345 = \)
			\task[]
			\task[] \textbf{Тысячных:}
			\task[]
			\task[]
			\task \( 876,5439 = \)
			\task \( 42,9872 = \)
			\task[]
			\task[] \textbf{Целых:}
			\task[]
			\task[]
			\task \( 24,67831 = \)
			\task \( 349,7651 = \)
		\end{tasks}
		\item Найдите:
		\begin{tasks}(2)
			\task \( 35\% \) от \( 1400 \)
			\task \( 12\% \) от \( 12500 \)
			\task \( 48\% \) от \( 14 \)
			\task \( 83\% \) от \( 274 \)
		\end{tasks}
		\item Вычислите:
		\begin{tasks}(3)
			\task \( 53,5\cdot0,1 \)
			\task \( 45,6\cdot0,01 \)
			\task \( 292,8\cdot0,0001 \)
			\task \( 98,23\cdot0,01 \)
			\task \( 801\cdot0,001 \)
			\task \( 44,4\cdot0,1 \)
		\end{tasks}
		\item Вычислите: \begin{tasks}(3)
			\task \( 0,65\cdot100 \)
			\task \( 4,2031\cdot10000 \)
			\task \( 333,003\cdot100000 \)
			\task \( 98,21:1000 \)
			\task \( 3215,005:10000 \)
			\task \( 84,9:100 \)
		\end{tasks}
		\item Площадь прямоугольника равна \( 37,8 \) см\( ^{2} \), а длина равна \( 9 \) см. Найдите ширину и периметр данного прямоугольника.
		\item В двух коробках \( 7,8 \) кг конфет. Когда из одной коробки взяли \( 1,25 \) кг конфет, то в обеих коробках конфет стало поровну.Сколько конфет было в каждой коробке?
		\item Цену на блузку понизили на \( 11\% \). Какой стала её цена, если первоначально она стоила \( 5000 \) рублей?
	\end{listofex}
\end{class}
%END_FOLD

%BEGIN_FOLD % ====>>_ Проверочная работа _<<====
\begin{exam}
	\begin{listofex}
			\item Округлите: 
			\begin{tasks}(2)
				\task \( 123,45 \) до десятых
				\task \(  682,9725 \) до сотых
				\task \( 9525,204 \) до сотых
				\task \( 6675,98731 \) до тысячных
				\task \( 7,0999701 \) до тысячных
				\task \( 11,000875\) до стотысячных
			\end{tasks}
			\item Вычислите и полученный результат округлите: 
			\begin{tasks}(1)
				\task \( 38,49 : 0,1 \), до единиц, до сотен
				\task \( 56,99 : 0,01  \), до десятков, до десятков тысяч
				\task \( 65,3 \cdot3,21 \), до сотых
			\end{tasks}
			\item Вычислите и полученный результат округлите: 
			\begin{tasks}(3)
				\task \( 20,7 :9 \) 
				\task \( 243,2:8 \)
				\task \( 88,298: 7 \)
				\task \( 772,8:12 \)
				\task \( 93,15: 23 \)
				\task \( 0,644:92 \)
			\end{tasks}
			\item  В голосовании приняли участие \( 64,3\% \) студентов. Какое количество студентов не
			пришло на выборы, если всего \( 2000 \) студентов? 
			\item  Морская вода содержит \( 8\% \) соли (по массе). Сколько в морской воде пресной воды? 
			\item Оптовая цена товара на складе \( 5500 \) р. Торговая надбавка в магазине составляет \( 12\% \).Сколько стоит этот товар в магазине?  
			\item В библиотеке \( 98000  \) книг. Книги на русском языке составляют \( 78\% \) всех книг, из
			них \( 5\% \) --- учебники. Сколько учебников на русском языке в библиотеке?
			\item В школе \( 15 \) учеников учатся только на оценки «отлично». Это составляет \( 5\% \) всех учащихся
			школы. Сколько всего учащихся в школе? 
	\end{listofex}
\end{exam}
%END_FOLD

%BEGIN_FOLD % ====>>_ Консультация _<<====
\begin{consultation}
	\begin{listofex}
		\item Максим построил у себя в тетради прямоугольник со сторонами, равными \( 1,8 \) см и \( 5,6 \) см. Найдите периметр и площадь фигуры. Что больше? На сколько?
	\end{listofex}
\end{consultation}
%END_FOLD

%BEGIN_FOLD % ====>>_ Консультация _<<====
\begin{consultation}
	\begin{listofex}
		\item Вычислить:
		\begin{tasks}(4)
			\task \( 20,7:9 \)
			\task \( 243,2:8 \)
			\task \( 7,368:24 \)
			\task \( 25:125 \)
			\task \( 1:8 \)
			\task \( 72,57:59 \)
			\task \( 0,7:25 \)
			\task \( 6,78:26 \)
		\end{tasks}
		\item Вычислить:
		\begin{tasks}(4)
			\task \( 45,5:10 \)
			\task \( 45,5:1000 \)
			\task \( 45,5:10000 \)
			\task \( 89:10 \)
			\task \( 89:100 \)
			\task \( 32,2:10 \)
			\task \( 7,98:10 \)
			\task \( 47,7:1000 \)
			\task \( 0,911:1000 \)
		\end{tasks}
		\item Вычислить:
		\begin{tasks}(4)
			\task \( 2:0,4 \)
			\task \( 70:1,75 \)
			\task \( 24:0,2 \)
			\task \( 2:0,5 \)
			\task \( 45:0,05 \)
			\task \( 125:2,5 \)
			\task \( 484:0,004 \)
			\task \( 5,1:0,17 \)
			\task \( 25,2:0,4 \)
			\task \( 200,1:0,69 \)
		\end{tasks}
		\item Путь у Ксюши до дома занимает \( 1,6 \) км. Утром она обычно опаздывает в школу, поэтому бежит. В этом случае время, которое ей необходимо на дорогу до школы, составляет \( 0,25 \) часа. Обратно со школы Ксюша не торопится и идет домой в течение \( 0,5 \) часа. С какой скоростью Ксюша бежит в школу утром и с какой скоростью она идет домой после уроков?
		\item Вася случайно разрезал моток веревки на две части. Длина одной из них --- \( 3,25 \) м, длина другой в \( 1,3 \) раза короче. Необходимо найти общую длину исходной веревки.
		\item Вася живет в комнате, которая имеет форму прямоугольного параллелепипеда. Известно, что объем этой комнаты равен \( 37,84 \) м\( ^3 \). При этом длина комнаты составляет \( 4 \) м, а высота потолка --- \( 2,2 \) м. Нужно найти ширину комнаты. 
		\item Кенгуру ниже жирафа в \( 2,4 \) раза, при этом жираф выше кенгуру на \( 2,52 \) м. Найдите рост жирафа и кенгуру.
	\end{listofex}
\end{consultation}
%END_FOLD