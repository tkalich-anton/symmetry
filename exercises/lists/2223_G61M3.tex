%===============>> ГРУППА 6-1 МОДУЛЬ 3 <<=============
\setmodule{3}
%===============>>     Занятие 1     <<===============
\begin{class}[number=1]
	\begin{listofex}
		\item Представить обыкновенную дробь в виде десятичной:
		\begin{enumcols}[itemcolumns=6]
			\item \( \dfrac{6}{10} \)
			\item \( \dfrac{9}{10} \)
			\item \( \dfrac{14}{100} \)
			\item \( \dfrac{14}{1000} \)
			\item \( \dfrac{999}{1000} \)
			\item \( \dfrac{10}{1000} \)
		\end{enumcols}
		\item Представить неправильную дробь в виде разрядных слагаемых и далее в виде десятичной дроби:\\
		\textit{Пример:} \( \dfrac{125}{100}=1\dfrac{25}{100}=1+\dfrac{20}{100}+\dfrac{5}{100}=1+\dfrac{2}{10}+\dfrac{5}{100}=1,25 \)\\
		\begin{enumcols}[itemcolumns=4]
			\item \( \dfrac{244}{100} \)
			\item \( \dfrac{312}{10} \)
			\item \( \dfrac{54765}{1000} \)
			\item \( \dfrac{1025}{10} \)
		\end{enumcols}
		\item Представить неправильную дробь в виде десятичной:
		\begin{enumcols}[itemcolumns=5]
			\item \( \dfrac{17}{10} \)
			\item \( \dfrac{27}{10} \)
			\item \( \dfrac{579}{10} \)
			\item \( \dfrac{1001}{10} \)
			\item \( \dfrac{649}{100} \)
			\item \( \dfrac{1247}{100} \)
			\item \( \dfrac{5488}{10} \)
			\item \( \dfrac{5488}{100} \)
			\item \( \dfrac{5488}{1000} \)
		\end{enumcols}
		\item Представить десятичную дробь в виде обыкновенной:
		\begin{enumcols}[itemcolumns=6]
			\item \( 2,3 \)
			\item \( 15,6 \)
			\item \( 24,24 \)
			\item \( 57,24 \)
			\item \( 101,01 \)
			\item \( 444,4 \)
		\end{enumcols}
		\item Вычислить:
		\begin{enumcols}[itemcolumns=2]
			\item \( 1,2+0,24 \)
			\item \( 23,55+132,05 \)
			\item \( 66,22+37,12 \)
			\item \( 124,05+25,1-67,15 \)
			\item \( 212,04-(25,55+197,4) \)
			\item \( (14,37+67,63)+(94,87-32,87) \)
		\end{enumcols}
		\item Представить обыкновенную дробь в виде десятичной:
		\begin{enumcols}[itemcolumns=6]
			\item \( \dfrac{1}{2} \)
			\item \( \dfrac{1}{4} \)
			\item \( \dfrac{3}{4} \)
			\item \( \dfrac{15}{8} \)
			\item \( \dfrac{76}{25} \)
			\item \( \dfrac{37}{50} \)
			\item \( \dfrac{9}{40} \)
			\item \( \dfrac{3}{12} \)
			\item \( \dfrac{33}{150} \)
			\item \( \dfrac{9}{75} \)
			\item \( \dfrac{30}{24} \)
			\item \( \dfrac{7}{50000} \)
		\end{enumcols}
		\item Вычислить:
		\begin{enumcols}[itemcolumns=6]
			\item \( \dfrac{7}{10}+15,3 \)
			\item \( 2,4+\dfrac{13}{100} \)
			\item \( \dfrac{1}{2}+16,7 \)
			\item \( \dfrac{3}{4}-0,1 \)
			\item \( 3\dfrac{1}{50}+4,98 \)
			\item \( 5\dfrac{13}{25}-4,12 \)
		\end{enumcols}
		\item Вычислить:
		\begin{enumcols}[itemcolumns=6]
			\item \( 1,2\cdot1,3 \)
			\item \( 7\cdot0,2 \)
			\item \( 0,07^2 \)
			\item \( 2,3\cdot12,7 \)
			\item \( 85,8\cdot2,3 \)
			\item \( 0,2^3 \)
		\end{enumcols}
	\end{listofex}
\end{class}
%===============>>     Занятие 2     <<===============
\begin{class}[number=2]
		\begin{listofex}
			\item Представить обыкновенную дробь в виде десятичной:
			\begin{enumcols}[itemcolumns=6]
				\item \( \dfrac{7}{10} \)
				\item \( \dfrac{11}{10} \)
				\item \( \dfrac{17}{100} \)
				\item \( \dfrac{19}{100} \)
				\item \( \dfrac{766}{1000} \)
				\item \( \dfrac{30}{1000} \)
			\end{enumcols}
			\item Представить неправильную дробь в виде разрядных слагаемых и далее в виде десятичной дроби:
			\begin{enumcols}[itemcolumns=4]
				\item \( \dfrac{146}{100} \)
				\item \( \dfrac{527}{10} \)
				\item \( \dfrac{6537}{1000} \)
				\item \( \dfrac{2225}{10} \)
			\end{enumcols}
			\item Представить неправильную дробь в виде десятичной:
			\begin{enumcols}[itemcolumns=5]
				\item \( \dfrac{19}{10} \)
				\item \( \dfrac{31}{10} \)
				\item \( \dfrac{623}{10} \)
				\item \( \dfrac{1007}{10} \)
				\item \( \dfrac{753}{100} \)
				\item \( \dfrac{1313}{100} \)
				\item \( \dfrac{6001}{10} \)
				\item \( \dfrac{6321}{100} \)
				\item \( \dfrac{5473}{1000} \)
				\item \( \dfrac{5815}{10} \)
			\end{enumcols}
			\item Представить десятичную дробь в виде неправильной:
			\begin{enumcols}[itemcolumns=6]
				\item \( 2,7 \)
				\item \( 17,2 \)
				\item \( 27,25 \)
				\item \( 63,37 \)
				\item \( 127,35 \)
				\item \( 555,5 \)
			\end{enumcols}
			\item Вычислить:
			\begin{enumcols}[itemcolumns=3]
				\item \( 1,7+0,23 \)
				\item \( 22,45+104,15 \)
				\item \( 57,24+23,26 \)
				\item \( 117,1+25,05-52,15 \)
				\item \( 210,08-24,45+157,4 \)
				\item \( (13,27+67,73)+(94,87-32,87) \)
			\end{enumcols}
			\item Представьте в виде десятичной дроби:
			\begin{enumcols}[itemcolumns=4]
				\item \( \dfrac{3}{2} \)
				\item \( \dfrac{4}{10} \)
				\item \( \dfrac{6}{25} \)
				\item \( \dfrac{13}{1000} \)
				\item \( \dfrac{27}{50} \)
				\item \( \dfrac{13}{40} \)
				\item \( \dfrac{3}{8} \)
				\item \( \dfrac{9}{50000} \)
				\item \( \dfrac{66}{150}\)
				\item \( \dfrac{12}{75} \)
				\item \( \dfrac{18}{3000} \)
				\item \( \dfrac{78}{24} \)
			\end{enumcols}
			\item Вычислить:
			\begin{enumcols}[itemcolumns=6]
				\item \( \dfrac{6}{10}+14,4 \)
				\item \( 2,5+\dfrac{17}{100} \)
				\item \( 0,7+\dfrac{1}{4} \)
				\item \( \dfrac{7}{5}-0,3\)
				\item \( 6\dfrac{1}{50}+3,98 \)
				\item \( 7\dfrac{18}{25}-3,02 \)
			\end{enumcols}
			\item Вычислить:
			\begin{enumcols}[itemcolumns=6]
				\item \( 1,8\cdot1,5 \)
				\item \( 8\cdot0,4 \)
				\item \( 0,06^2 \)
				\item \( 2,6\cdot13,4\)
				\item \( 74,6\cdot1,3 \)
				\item \( 0,3^3 \)
			\end{enumcols}
		\end{listofex}
\end{class}
%===============>> Домашняя работа 1 <<===============
\begin{homework}[number=1]
	\begin{listofex}
		\item Вычислить:
		\begin{enumcols}[itemcolumns=4]
			\item \( 1-\dfrac{9}{11} \)
			\item \( \mfrac{6}{3}{4}+\mfrac{2}{5}{8} \)
			\item \( \mfrac{8}{6}{13}-\mfrac{3}{9}{26} \)
			\item \( \mfrac{9}{1}{3}-\mfrac{8}{14}{15} \)
		\end{enumcols}
		\item Решить уравнение:
		\begin{enumcols}[itemcolumns=3]
			\item \( x+\mfrac{3}{1}{5}=\mfrac{5}{2}{5} \)
			\item \( \mfrac{4}{1}{17}+x=\dfrac{5}{68} \)
			\item \( x-\mfrac{7}{5}{18}=\mfrac{9}{1}{18} \)
		\end{enumcols}
		\item Вычислить рациональным образом:
		\begin{enumcols}[itemcolumns=2]
		\item \( \mfrac{7}{13}{14}-\mfrac{4}{17}{25}-\mfrac{2}{13}{14} \)
		\item \( \mfrac{5}{16}{39}+\mfrac{1}{6}{11}-\mfrac{2}{16}{39} \)
		\end{enumcols}
		\item Найти:
		\begin{enumcols}[itemcolumns=3]
			\item \( \dfrac{4}{9} \) от \( \mfrac{3}{3}{4} \)
			\item \( \dfrac{9}{17} \) от \( \mfrac{15}{1}{9} \)
			\item \( \dfrac{13}{17} \) от \( \mfrac{4}{14}{39} \)
		\end{enumcols}
		\item Вычислить: %4.187 е, ж, р
		\begin{enumcols}[itemcolumns=2]
			\item \( \left( \mfrac{1}{4}{9}+\mfrac{2}{5}{6}-\mfrac{2}{3}{4} \right)\cdot\left( \mfrac{2}{1}{2}-\dfrac{11}{14} \right) \)
			\item \( \left( \mfrac{2}{1}{2}-\mfrac{1}{3}{8} \right)\cdot\left( \mfrac{3}{1}{2}-\dfrac{5}{6} \right)\cdot\mfrac{1}{1}{3} \)
			\item \( \left( \mfrac{5}{7}{12}-\mfrac{3}{17}{36} \right)\cdot\mfrac{2}{1}{2}+\mfrac{4}{1}{3}\cdot\dfrac{3}{26}+\dfrac{1}{2} \)
		\end{enumcols}
		\item Вычислить:
		\begin{enumcols}[itemcolumns=3]
			\item \( 0,99\cdot2,5 \)
			\item \( 0,9\cdot800 \)
			\item \( 74\cdot4,9 \)
			\item \( 3,43\cdot0,12 \)
			\item \( 0,00013\cdot0,5 \)
			\item \( 0,01^4 \)
		\end{enumcols}
		\item Вычислить:
		\begin{enumcols}[itemcolumns=2]
			\item \( 14,3\cdot0,6-5,7\cdot1,4 \)
			\item \( (54-23,42)\cdot0,08 \)
			\item \( (6-4,94)\cdot2,5-2,35 \)
			\item \( 1,2\cdot4,4+2,3\cdot(3,72-2,42)-1,27 \)
		\end{enumcols}
		\item Найдите сумму площадей стен комнаты, длина которой \( 6,4 \) м, ширина \( 3,5 \) м и
		высота \( 2,69 \) м. Найдите объём комнаты.
	\end{listofex}
\end{homework}
%===============>>     Занятие 3     <<===============
\begin{class}[number=3]
	\begin{listofex}
		\item Вычислить:
		\begin{enumcols}[itemcolumns=4]
			\item \( 5-\dfrac{6}{7} \)
			\item \( \mfrac{3}{5}{6}-\mfrac{1}{4}{9} \)
			\item \( \mfrac{9}{11}{16}+\mfrac{3}{5}{24} \)
			\item \( \mfrac{27}{3}{8}+\mfrac{19}{63}{64} \)
		\end{enumcols}
		\item Решить уравнение:
		\begin{enumcols}[itemcolumns=3]
			\item \( x+\mfrac{3}{2}{5}=\mfrac{5}{1}{5} \)
			\item \( \mfrac{4}{3}{8}+x=\mfrac{9}{1}{12} \)
			\item \( x-\mfrac{9}{11}{12}=\mfrac{7}{5}{24} \)
		\end{enumcols}
		\item Вычислить рациональным образом:
		\begin{enumcols}[itemcolumns=2]
			\item \( \mfrac{3}{19}{24}+\mfrac{5}{1}{9}+\mfrac{1}{5}{24} \)
			\item \( \mfrac{4}{7}{45}+\mfrac{11}{4}{13}+\mfrac{8}{5}{26}+\mfrac{10}{2}{5} \)
		\end{enumcols}
		\item Найти:
		\begin{enumcols}[itemcolumns=3]
			\item \( \dfrac{3}{5} \) от \( \mfrac{6}{2}{3} \)
			\item \( \dfrac{9}{25} \) от \( \mfrac{20}{5}{6} \)
			\item \( \dfrac{11}{48} \) от \( \mfrac{13}{1}{11} \)
		\end{enumcols}
		\item Вычислить: %4.187 а, й, н
		\begin{enumcols}[itemcolumns=2]
			\item \( \left( \dfrac{3}{4}+\dfrac{5}{6} \right)\cdot3+\left( \dfrac{5}{6}-\dfrac{3}{4} \right)\cdot4 \)
			\item \( \left( \mfrac{40}{7}{15}-\mfrac{29}{8}{35} \right)\cdot28-\mfrac{8}{4}{7}\cdot\mfrac{4}{1}{5} \)
			\item \( \left( \mfrac{2}{5}{6}-\dfrac{3}{4}-\mfrac{1}{1}{10}+\dfrac{8}{15} \right)\cdot\mfrac{4}{1}{2}\cdot\left( \mfrac{1}{5}{12}-\dfrac{1}{2} \right) \)
		\end{enumcols}
		\item Вычислить:
		\begin{enumcols}[itemcolumns=3]
			\item \( 2,3\cdot12,7 \)
			\item \( 60\cdot0,03 \)
			\item \( 85,8\cdot3,2 \)
			\item \( 2,749\cdot0,48 \)
			\item \( 0,00016\cdot0,004 \)
			\item \( 0,5^3 \)
		\end{enumcols}
		\item Вычислить:
		\begin{enumcols}[itemcolumns=2]
			\item \( (4,125-1,6)\cdot(0,12+7,3) \)
			\item \( (8,4\cdot0,55+3,28)\cdot9,2-43,78 \)
			\item \( 67,45-7,45\cdot(3,8+4,2) \)
			\item \( 28,6+11,4\cdot(6,595+3,405) \)
		\end{enumcols}
		\item Катер, собственная скорость которого \( 14,8 \) км/ч, шёл \( 3 \) часа по течению реки и \( 4 \) ч против течения реки. Какой путь проделал катер за всё это время, если скорость течения реки \( 2,3 \) км/ч?
	\end{listofex}
\end{class}
%===============>>     Занятие 4     <<===============
\begin{class}[number=4]
	\begin{listofex}
		\item Вычислить:
		\begin{enumcols}[itemcolumns=4]
			\item \( 5-\dfrac{6}{7} \)
			\item \( \mfrac{5}{7}{12}-\mfrac{2}{2}{15} \)
			\item \( \mfrac{4}{11}{14}-\mfrac{3}{2}{7} \)
			\item \( \mfrac{49}{4}{5}-\mfrac{13}{61}{65} \)
		\end{enumcols}
		\item Решить уравнение:
		\begin{enumcols}[itemcolumns=3]
			\item \( \mfrac{8}{7}{8}-x=\mfrac{5}{1}{4} \)
			\item \( \mfrac{13}{1}{19}+x=\mfrac{47}{4}{19} \)
			\item \( x-\mfrac{6}{2}{3}=\mfrac{7}{11}{15} \)
		\end{enumcols}
		\item Вычислить рациональным образом:
		\begin{enumcols}[itemcolumns=2]
			\item \( \mfrac{1}{5}{8}+\mfrac{4}{8}{17}+\dfrac{9}{17}+\mfrac{2}{3}{8} \)
			\item \( \mfrac{4}{1}{7}+\mfrac{5}{4}{9}+\mfrac{12}{6}{7}+\mfrac{3}{5}{11}+\mfrac{10}{5}{9}+\mfrac{11}{6}{11}+\dfrac{4}{7} \)
		\end{enumcols}
		\item Найти:
		\begin{enumcols}[itemcolumns=3]
			\item \( \dfrac{3}{1} \) от \( \mfrac{3}{2}{3} \)
			\item \( \dfrac{7}{18} \) от \( \mfrac{4}{1}{2} \)
			\item \( \mfrac{2}{3}{4} \) от \( \mfrac{1}{2}{3} \)
		\end{enumcols}
		\item Вычислить: %4.187 в, ё, и
		\begin{enumcols}[itemcolumns=1]
			\item \( \left( \mfrac{2}{3}{5}+\mfrac{1}{5}{7} \right)\cdot14-\left( \mfrac{2}{1}{2}-\dfrac{3}{8} \right)\cdot4 \)
			\item \( \mfrac{13}{5}{8}-\mfrac{5}{5}{42}\cdot\left( \mfrac{1}{1}{6}+\dfrac{7}{12} \right) \)
			\item \( \left( \dfrac{5}{18}+\dfrac{7}{12}+\dfrac{4}{9} \right)\cdot\left( 1-\dfrac{20}{47} \right)\cdot\left( \mfrac{1}{1}{4}-\dfrac{17}{20} \right) \)
		\end{enumcols}
		\item Вычислить:
		\begin{enumcols}[itemcolumns=3]
			\item \( 0,27\cdot1,8 \)
			\item \( 32,15\cdot0,6 \)
			\item \( 27\cdot3,5 \)
			\item \( 0,156\cdot1,7 \)
			\item \( 0,00157\cdot0,002 \)
			\item \( 0,11^3 \)
		\end{enumcols}
		\item Вычислить:
		\begin{enumcols}[itemcolumns=2]
			\item \( 20,4\cdot6,5+3,8\cdot18 \)
			\item \( 7,2\cdot3,6-4,8\cdot5,4 \)
			\item \( (1,13-0,5)\cdot(1,34+3,4)-0,02\cdot49,31 \)
			\item \( 9,8\cdot8,8\cdot2,5-0,05\cdot1312 \)
		\end{enumcols}
		\item Найдите площадь прямоугольника, если ширина его \( 13,4 \) м, а длина в \( 4 \) раза больше ширины.
	\end{listofex}
\end{class}
%===============>> Домашняя работа 2 <<===============
\begin{homework}[number=2]
	\begin{listofex}
		\item Решить уравнение:
		\begin{enumcols}[itemcolumns=3]
			\item \( x+\mfrac{4}{1}{7}=\mfrac{9}{5}{7} \)
			\item \( \mfrac{9}{5}{12}-x=\mfrac{7}{20}{21} \)
			\item \( x-\mfrac{4}{3}{11}=\mfrac{2}{5}{22} \)
		\end{enumcols}
		\item Найти:
		\begin{enumcols}[itemcolumns=3]
			\item \( \dfrac{11}{19} \) от \( \mfrac{4}{3}{4} \)
			\item \( \mfrac{3}{2}{3} \) от \( \dfrac{4}{5} \)
			\item \( \mfrac{3}{7}{9} \) от \( \mfrac{1}{1}{8} \)
		\end{enumcols}
		\item Вычислить: %4.187 з, у, п
		\begin{enumcols}[itemcolumns=2]
			\item \( \mfrac{15}{4}{7}-\mfrac{4}{3}{8}\cdot\left( \mfrac{1}{3}{7}-\dfrac{34}{35} \right) \)
			\item \( \left( \mfrac{8}{1}{2}-\mfrac{7}{3}{8} \right)\cdot\mfrac{5}{2}{3}-\mfrac{1}{4}{5}\cdot\left( \mfrac{3}{1}{3}-\mfrac{2}{7}{9} \right) \)
		\end{enumcols}
		\begin{enumcols}[itemcolumns=1, resume]
			\item \( \mfrac{8}{2}{11}\cdot\left( \mfrac{4}{3}{4}\cdot\dfrac{4}{57}+\mfrac{7}{2}{3}\cdot\dfrac{9}{46} \right)+15\cdot\left( \mfrac{5}{7}{8}\cdot\mfrac{3}{3}{47}-\mfrac{3}{2}{3}\cdot\mfrac{2}{1}{22} \right) \)
		\end{enumcols}
		\item Вычислить:
		\begin{enumcols}[itemcolumns=2]
			\item \( 5,2\cdot1,3+3,1\cdot(5,42-4,12)-1,79 \)
			\item \( (4,56+4,1)\cdot(1,12-0,99)-0,04\cdot3,145 \)
			\item \( (4,9\cdot(8,9-7,6)-5,5)\cdot10,1-3,087 \)
			\item \( 7,7\cdot5,6\cdot3,5-0,04\cdot1273 \)
		\end{enumcols}
	\end{listofex}
\end{homework}
%\newpage
%\title{Занятие №5}
%\begin{listofex}
%	
%\end{listofex}
%\newpage
%\title{Занятие №6}
%\begin{listofex}
%	
%\end{listofex}
%\newpage
%\title{Домашняя работа №3}
%\begin{listofex}
%	
%\end{listofex}
%\newpage
%\title{Подготовка к проверочной работе}
%\begin{listofex}
%	
%\end{listofex}
%\newpage
%\title{Проверочная работа}
%\title{Вариант 1}
%\begin{listofex}
%	
%\end{listofex}
%\newpage
%\title{Проверочная работа}
%\begin{listofex}
%	
%\end{listofex}