%
%===============>>  ГРУППА 11-3 МОДУЛЬ 4  <<=============
%
\setmodule{4}
%
%===============>>  Занятие 1  <<===============
%
%\begin{class}[number=1]
%	\begin{listofex}
%		\item Пусто
%	\end{listofex}
%\end{class}
%
%===============>>  Занятие 2  <<===============
%
%\begin{class}[number=2]
%	\begin{listofex}
%		\item Пусто
%	\end{listofex}
%\end{class}
%
%===============>>  Домашняя работа 1  <<===============
%
%\begin{homework}[number=1]
%	\begin{listofex}
%		\item Пусто
%	\end{listofex}
%\end{homework}
%
%===============>>  Занятие 3  <<===============
%
%\begin{class}[number=3]
%	\begin{listofex}
%		\item Пусто
%	\end{listofex}
%\end{class}
%
%===============>>  Занятие 4  <<===============
% смещение на одно занятие с прошлого месяца
%\begin{class}[number=4]
%	\begin{listofex}
%		\item Пусто
%	\end{listofex}
%\end{class}
%
%===============>>  Домашняя работа 2  <<===============
%
\begin{homework}[number=2]
	\begin{listofex}
		\item Прямая \( y=9x+5 \) является касательной к графику функции \( y=18x^2+bx+7 \). Найдите \( b \), учитывая, что абсцисса точки касания меньше \( 0 \).
		\item Прямая \( y=3x-2 \) является касательной к графику функции \( y=x^3-5x^2+6x+7 \). Найдите абсциссу точки касания.
		\item Основания равнобедренной трапеции равны \( 51 \) и \( 65 \), а боковые стороны равны \( 25 \). Найдите синус острого угла трапеции.
		\item Диагональ прямоугольника в полтора раза длиннее одной из его сторон. Другая сторона прямоугольника равна \( 3\sqrt{5} \). Какова длина диагонали?
		\item Наименьший угол равнобедренного треугольника равен \( 40\degree \). Найдите (в градусах) его наибольший угол. Если задача
		имеет несколько решений, в ответе запишите их сумму.
		\item Стороны треугольника равны \( 10 \), \( 17 \) и \( 21 \). Найдите высоту треугольника, проведённую из вершины наибольшего угла.
		\item В треугольнике \( ABC \) известно, что \( \angle C = 90\degree \),
		а медиана \( CM \) и биссектриса \( AL \) пересекаются в точке \( T \), причём \( LT = CL \).
		Найдите наибольший острый угол треугольника ABC.
		Ответ дайте в градусах.
		\item Прямая \( y=5x-8 \) явлется касательной к графику функции \( y=4x^2-15x+c \). Найдите \( c \).
		\item В прямоугольном треугольнике \( ABC \) из вершины прямого
		угла \( C \) проведены медиана \( CM \) и высота \( CH \).\\
		а) Докажите, что биссектриса \( CL \) треугольника \( ABC \) является
		также биссектрисой треугольника \( CMH \).\\
		б) Найдите \( CL \), если \( CM = 10 \), \( CH = 6 \).
		\item \exercise{2929}
		\item \exercise{1842}
	\end{listofex}
\end{homework}
%
%===============>>  Занятие 5  <<===============
% 
%\begin{class}[number=5]
%	\begin{listofex}
%		\item Пусто
%	\end{listofex}
%\end{class}
%
%===============>>  Занятие 6  <<===============
% 
\begin{class}[number=6]
	\begin{listofex}
		\item В треугольнике \( ABC \) угол \( C \) равен \( 90\degree \), \( AC=8 \), \( \tg A=0,5 \). Найдите \( BC \).
		\item Параллелограмм и прямоугольник имеют одинаковые стороны. Найдите острый угол параллелограмма, если его площадь равна половине площади прямоугольника. Ответ дайте в градусах.
		\item Диагонали четырехугольника равны \( 4 \) и \( 5 \). Найдите периметр четырехугольника, вершинами которого являются середины сторон данного четырехугольника.
		\item Площадь параллелограмма \( ABCD \) равна \( 176 \). Точка \( E \) --- середина стороны \( CD \).
		Найдите площадь треугольника \( ADE \).
		\item Найдите вписанный угол, опирающийся на дугу, которая составляет \( \dfrac{1}{5} \) окружности.
		Ответ дайте в градусах.
		\item Угол \( \angle ACB = 42\degree \).
		Градусная мера дуги \( AB \) окружности, не содержащей точек \( D \) и \( E \), равна \( 124\degree \).
		Найдите угол \( DAE \).
		Ответ дайте в градусах.
		\item В треугольнике \( ABC \) сторона \( AB \) равна \( 3\sqrt{2} \),
		угол \( C \) равен \( 135\degree \).
		Найдите радиус описанной около этого треугольника окружности.
		\item Хорда \( AB \) делит окружность на две части, градусные величины которых относятся как \( 5:7 \).
		Под каким углом видна эта хорда из точки \( C \), принадлежащей меньшей дуге окружности?
		Ответ дайте в градусах.
		\item Касательные \( CA \) и \( CB \) к окружности образуют угол \( ACB \), равный \( 122\degree \).
		Найдите величину меньшей дуги \( AB \), стягиваемой точками касания. Ответ дайте в градусах.
		\item Угол \( ACO \) равен \( 28\degree \), где \( O \) --- центр окружности.
		Его сторона \( CA \) касается окружности.
		Найдите величину меньшей дуги \( AB \) окружности, заключенной внутри этого угла.
		Ответ дайте в градусах.
		\item В равнобедренном треугольнике с боковой стороной,
		равной \( 4 \), проведена медиана к боковой стороне.
		Найдите основание треугольника, если эта медиана равна \( 3 \).
		\item Медианы треугольника \( ABC \), проведенные из вершин \( B \) и \( C \),
		равны \( 6 \) и \( 9 \) и пересекаются в точке \( M \).
		Известно, что \( \angle BMC = 120\degree \).
		Найдите стороны треугольника.
		\item В выпуклом четырехугольнике \( ABCD \) отрезок,
		соединяющий середины сторон \( AB \) и \( CD \), равен \( 1 \).
		Прямые \( BC \) и \( AD \) перпендикулярны.
		Найдите отрезок, соединяющий середины диагоналей \( AC \) и \( BD \).
	\end{listofex}
\end{class}
%
%===============>>  Домашняя работа 3  <<===============
%
%\begin{homework}[number=2]
%	\begin{listofex}
%
%	\end{listofex}
%\end{homework}
%\newpage
%\title{Подготовка к проверочной работе}
%\begin{listofex}
%	
%\end{listofex}
%
%===============>>  Занятие 7  <<===============
%
%\begin{class}[number=7]
%	\begin{listofex}
%	
%	\end{listofex}
%\end{class}
%
%===============>>  Провечная работа  <<===============
%
%\begin{exam}
%	\begin{listofex}
%	
%	\end{listofex}
%\end{exam}