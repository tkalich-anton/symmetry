%
%===============>>  ГРУППА 9-1 МОДУЛЬ 10  <<=============
%
\setmodule{10}

%BEGIN_FOLD % ====>>_____ Занятие 1 _____<<====
\begin{class}[number=1]
	\begin{listofex}
		\item Решите уравнения:
		\begin{tasks}(2)
			\task \( (1-2x)^4=x^2 \)
			\task \( \dfrac{(x+2)^4-2(x+2)^2+1}{x^2-3x-4}=0 \)
		\end{tasks}
		\item Решите неравенство: \( (2+x)^3\ge\sqrt{8}(2+x)^2 \)
		\item Окружности с центрами в точках \( E \) и \( F \) пересекаются в точках \( C \) и \( D \),	причём точки \( E \) и \( F \) лежат по одну сторону от прямой \( CD \). Докажите, что \( CD \) и \( EF \) перпендикулярны.      
		\item В окружности с центром \( O \) проведены две хорды \( AB \) и \( CD \) так, что центральные углы \( AOB \) и \( COD \) равны. На эти хорды опущены перпендикуляры \( OK \) и \( OL \). Докажите, что \( OK \) и \( OL \) равны.
		\item В остроугольном треугольнике \( ABC \) угол \( B \) равен \( 60\degree \). Докажите, что точки \( A \), \( C \), центр описанной окружности треугольника \( ABC \) и центр вписанной окружности треугольника \( ABC \) лежат на одной окружности.
		\item В трапеции \( ABCD \) с основаниями \( AD \) и \( BC \) диагонали пересекаются в точке \( O \). Докажите, что площади треугольников \( AOB \) и \( COD \) равны.
		\item В параллелограмме \( ABCD \) проведена диагональ \( AC \). Точка \( O \) является центром окружности, вписанной в треугольник \( ABC \). Расстояния от точки \( O \) до точки \( A \) и прямых \( AD \) и \( AC \) соответственно равны \( 10 \), \( 8 \) и \( 6 \). Найдите площадь параллелограмма \( ABCD \).
		\item Из вершины прямого угла \( C \) треугольника \( ABC \) проведена высота \( CP \). Радиус окружности, вписанной в треугольник \( BCP \), равен \( 96 \), тангенс угла	\( BAC \) равен \( \dfrac{8}{15} \). Найдите радиус окружности, вписанной в треугольник \( ABC \).
		\item Три окружности, радиусы которых равны \( 2 \), \( 3 \) и \( 10 \), попарно касаются внешним образом. Найдите радиус окружности вписанной в треугольник, вершинами которого являются центры этих трёх окружностей
		\item Окружность, вписанная в треугольник \( ABC \), касается его сторон в точках \( M \), \( K \) и \( P \). Найдите углы треугольника \( ABC \), если углы треугольника \( MKP \) равны \( 49\degree \), \( 69\degree \) и \( 62\degree \).
	\end{listofex}
\end{class}
%END_FOLD

%BEGIN_FOLD % ====>>_____ Занятие 2 _____<<====
\begin{class}[number=2]
	\begin{listofex}
		\item .
	\end{listofex}
\end{class}
%END_FOLD

%BEGIN_FOLD % ====>>_ Домашняя работа 1 _<<====
\begin{homework}[number=1]
	\begin{listofex}
		\item .
	\end{listofex}
\end{homework}
%END_FOLD

%BEGIN_FOLD % ====>>_____ Занятие 3 _____<<====
\begin{class}[number=3]
	\begin{listofex}
		\item .
	\end{listofex}
\end{class}
%END_FOLD

%BEGIN_FOLD % ====>>_____ Занятие 4 _____<<====
\begin{class}[number=4]
	\begin{listofex}
		\item .
	\end{listofex}
\end{class}
%END_FOLD

%BEGIN_FOLD % ====>>_ Домашняя работа 2 _<<====
\begin{homework}[number=2]
	\begin{listofex}
		\item.
	\end{listofex}
\end{homework}
%END_FOLD

%BEGIN_FOLD % ====>>_____ Занятие 5 _____<<====
\begin{class}[number=5]
	\begin{listofex}
		\item .
	\end{listofex}
\end{class}
%END_FOLD

%BEGIN_FOLD % ====>>_____ Занятие 6 _____<<====
\begin{class}[number=6]
	\begin{listofex}
		\item .
	\end{listofex}
\end{class}
%END_FOLD

%BEGIN_FOLD % ====>>_ Домашняя работа 3 _<<====
\begin{homework}[number=3]
	\begin{listofex}
		\item .
	\end{listofex}
\end{homework}
%END_FOLD

%BEGIN_FOLD % ====>>_____ Занятие 7 _____<<====
\begin{class}[number=7]
	\begin{listofex}
		\item .
	\end{listofex}
\end{class}
%END_FOLD

%BEGIN_FOLD % ====>>_ Проверочная работа _<<====
\begin{class}[number=8]
	\begin{listofex}
		\item .
	\end{listofex}
\end{class}
%END_FOLD

%BEGIN_FOLD % ====>>_ Домашняя работа 4 _<<====
\begin{homework}[number=4]
	\begin{listofex}
		\item .
	\end{listofex}
\end{homework}
%END_FOLD