%
%===============>>  ГРУППА 9-1 МОДУЛЬ 9  <<=============
%
\setmodule{9}

%BEGIN_FOLD % ====>>_____ Занятие 1 _____<<====
\begin{class}[number=1]
	\begin{listofex}
		\item Найдите значение выражения: \(\dfrac{ 9,4 }{ 4,1+5,3 }\).
		\item На координатной прямой отмечены числа \( a \) и \( b \). Какое из следующих утверждений неверно?
		\begin{center}
			\includegraphics[align=t, width=0.5\linewidth]{\picpath/G93M8L6-6}
		\end{center}
		\begin{tasks}(2)
			\task \( a+b<0 \)
			\task \( -2<b-1<-1 \)
			\task \( a^2b<0 \)
			\task \( -a<0 \)
		\end{tasks}
		\item Найдите значение выражения \( \dfrac{a^{17}\cdot(b^5)^3}{(a\cdot b)^{15}} \) при \( a=7 \) и \( b=\sqrt{7} \).
		\item Решите уравнения: \(\dfrac{ 3 }{ x-19 }=\dfrac{ 19 }{ x-3 }\). \\
		Если корней несколько, запишите их в ответ без пробелов в порядке возрастания.
		\item В фирме такси в данный момент свободно \(15\) машин: \(3\) чёрных, \(6\) жёлтых и \(6\) зелёных. По вызову выехала одна из машин, случайно оказавшаяся ближе всего к заказчику. Найдите вероятность того, что к нему приедет жёлтое такси.
		\item На одном из рисунков изображен график функции \(y=x^2-2x+3\). Укажите номер этого рисунка.
		\begin{center}
			\includegraphics[align=t, width=0.5\linewidth]{\picpath/G91M9L1}
		\end{center}
		\item Объём пирамиды вычисляют по формуле \(V=\dfrac{ 1 }{ 3 }Sh\),  где \(S\) --- площадь основания пирамиды, \(h\) --- её высота. Объём пирамиды равен \(40\), площадь основания \(15\). Чему равна высота пирамиды?
		\item Решите неравенство: \(-x^2-2x\le0\).
		\item В амфитеатре \(14\) рядов. В первом ряду \(20\) мест, а в каждом следующем на \(3\) места больше, чем в предыдущем. Сколько мест в десятом ряду амфитеатра?
		\item В равностороннем треугольнике \(ABC\) биссектрисы \(CN\) и \(AM\) пересекаются в точке \(P\). Найдите \(\angle MPN\).
		\item К окружности с центром в точке \(O\) проведены касательная \(AB\) и секущая \(AO\). Найдите радиус окружности, если \(AB=12\) см, \(AO=13\) см.
		\item В треугольнике одна из сторон равна \(10\), другая равна \(10\sqrt{3}\), а угол между ними равен \(60\degree\). Найдите площадь треугольника.
		\item Основания трапеции равны \(18\) и \(12\), одна из боковых сторон равна \(4\sqrt{2}\), а угол между ней и одним из оснований равен \(135\degree\). Найдите площадь трапеции.
		\item Укажите номера верных утверждений.
		\begin{tasks}
			\task Если три угла одного треугольника равны трем углам другого треугольника, то такие треугольники подобны. 
			\task Сумма смежных углов равна \(180\degree\).
			\task Любая медиана равнобедренного треугольника является его биссектрисой.
		\end{tasks}
		\item Решите уравнение: \(x^2-2x+\sqrt{3-x}=\sqrt{3-x}+8\)
		\item Из пункта \(A\) в пункт \(B\), расстояние между которыми \(13\) км, вышел пешеход. Одновременно с ним из \(B\) в \(A\) выехал велосипедист. Велосипедист ехал со скоростью, на \(11\) км/ч большей скорости пешехода, и сделал в пути получасовую остановку. Найдите скорость пешехода, если известно, что они встретились в \(8\) км от пункта \(B\).
		\item Постройте график функции \(y=\dfrac{ (x+1)(x^2+7x+12) }{ x+3 }\) и определите, при каких значениях \(m\) прямая \(y=m\) имеет с графиком ровно одну общую точку.
		\item Периметр прямоугольника равен \(30\), а диагональ равна \(14\). Найдите площадь этого прямоугольника.
		\item Докажите, что медиана треугольника делит его на два треугольника, площади которых равны между собой.
	\end{listofex}
\end{class}
%END_FOLD

%BEGIN_FOLD % ====>>_____ Занятие 2 _____<<====
\begin{class}[number=2]
	\begin{listofex}
		\item Найдите значение выражения: \(0,6\cdot(-10)^3+50\).
		\item Какое из данных ниже чисел принадлежит промежутку \( [3;4] \)?
		\begin{tasks}(4)
			\task \( \dfrac{45}{19} \)
			\task \( \dfrac{52}{19} \)
			\task \( \dfrac{68}{19} \)
			\task \( \dfrac{77}{19} \)
		\end{tasks}
		\item Найдите значение выражения \( \dfrac{2c-4}{cd-2d} \) при \( c=0.5 \) и \(d=5 \).
		\item Решите уравнениe: \((x-4)^2+(x+9)^2\).
		\item Записан рост (в сантиметрах) пяти учащихся: \( 158 \), \( 166 \), \( 134 \), \( 130 \), \( 132 \). На сколько отличается среднее арифметическое этого набора чисел от его медианы?
		\item Найдите значение \( k \) по графику функции y=\(\dfrac{k}{x}\),  изображенному на рисунке.
		\begin{center}
			\includegraphics[align=t, width=0.3\linewidth]{\picpath/G91M9L2}
		\end{center}
		\item Радиус описанной около треугольника окружности можно найти по формуле \( R=\dfrac{a}{2\sin\alpha} \),  где \( a \) --- сторона треугольника, \( \alpha \) --- противолежащий этой стороне угол, а \( R \) --- радиус описанной около этого треугольника окружности. Пользуясь этой формулой, найдите \( \sin\alpha \), если \( a=0,6 \), а \( R=0,75 \).
		\item При каких значениях \( a \) выражение \( 5a+9 \) принимает отрицательные значения?
		\begin{tasks}(4)
			\task \( a>-\dfrac{9}{5} \)
			\task \( a<-\dfrac{5}{9} \)
			\task \( a>-\dfrac{5}{9} \)
			\task \( a><-\dfrac{9}{5} \)
		\end{tasks}
		\item У Кати есть попрыгунчик (каучуковый шарик). Она со всей силы бросила его об асфальт. После первого отскока попрыгунчик подлетел на высоту 400 см, а после каждого следующего отскока от асфальта подлетал на высоту в два раза меньше предыдущей. После какого по счёту отскока высота, на которую подлетит попрыгунчик, станет меньше \( 20 \) см?
		\item Площадь ромба равна \( 27 \), а периметр равен \( 36 \). Найдите высоту ромба.
		\item Прямоугольный треугольник с катетами \( 5 \) см и \( 12 \) см вписан в окружность. Чему равен радиус этой окружности?
		\item Основания равнобедренной трапеции равны \( 5 \) и \( 17 \), а ее боковые стороны равны \( 10 \). Найдите площадь трапеции.
		\item
		\begin{minipage}[t]{\bodywidth}
		На клетчатой бумаге с размером клетки \( 1X1 \) изображён треугольник \( ABC \). Найдите длину его средней линии, параллельной стороне \( AC \).
			\foranswer
		\end{minipage}
		\gapwidth
		\begin{minipage}[t]{\picwidth}
			\includegraphics[align=t, width=\linewidth]{\picpath/G91M9L2-1}
		\end{minipage}
		\item Какие из следующих утверждений верны?
		\begin{tasks}
			\task Сумма углов выпуклого четырехугольника равна \( 180\degree \).
			\task Если один из углов параллелограмма равен 60°, то противоположный ему угол равен \( 120\degree \).
			\task Диагонали квадрата делят его углы пополам.
			\task Если в четырехугольнике две противоположные стороны равны, то этот четырехугольник --- параллелограмм.
		\end{tasks}
		\item Решите неравенство : \((\sqrt{3}-1,5)(3-2x)>0\)
		\item При смешивании первого раствора соли, концентрация которого \( 40\% \), и второго раствора этой же соли, концентрация которого \( 48\% \), получился раствор с концентрацией \( 42\% \). В каком отношении были взяты первый и второй растворы?
		\item Постройте график функции \(y=\dfrac{x-2}{(\sqrt{x^2-2x})^2}\) и найдите все значения \( k \), при которых прямая \( y=kx \) имеет с графиком данной функции ровно одну общую точку.
		\item Прямая, параллельная основаниям \( MP \) и \( NK \) трапеции \( MNKP \), проходит через точку пересечения диагоналей трапеции и пересекает её боковые стороны \( MN \) и \( KP \) в точках  \( A \) и \( B \) соответственно. Найдите длину отрезка \( AB \), если \( MP=40 \) см, \( NK=24 \) см.
		\item Дан правильный восьмиугольник. Докажите, что если его вершины последовательно соединить отрезками через одну, то получится квадрат.
	\end{listofex}
\end{class}
%END_FOLD

%BEGIN_FOLD % ====>>_ Домашняя работа 1 _<<====
\begin{homework}[number=1]
	\begin{listofex}
		\item .
	\end{listofex}
\end{homework}
%END_FOLD

%BEGIN_FOLD % ====>>_____ Занятие 3 _____<<====
\begin{class}[number=3]
	\begin{listofex}
		\item.
	\end{listofex}
\end{class}
%END_FOLD

%BEGIN_FOLD % ====>>_____ Занятие 4 _____<<====
\begin{class}[number=4]
	\begin{listofex}
		\item Найдите значение выражения: \(\left( \dfrac{ 17 }{ 15 }-\dfrac{ 1 }{ 12 }\right)\cdot5 \).
		\item Какому промежутку принадлежит число \(\sqrt{66}\)?
		\begin{tasks}(4)
			\task \( [7;\,8] \)
			\task \( [8;\,9] \)
			\task \( [9;\,10] \)
			\task \( [10;\,11] \)
		\end{tasks}
		\item Найдите значение выражения \( \dfrac{ 4x-25y }{ 2\sqrt{x}-5\sqrt{y} }-3\sqrt{y} \), если \( \sqrt{x}+\sqrt{y}=4 \).
		\item Уравнение \(x^2+px=q=0\) имеет корни \(-6\); \(4\). Найдите \(q\).
		\item В среднем из каждых \(200\) поступивших в продажу аккумуляторов \(196\) аккумуляторов заряжены. Найдите вероятность того, что купленный аккумулятор не заряжен.
		\item На рисунке изображены графики функций вида \(ax^2+bx+c\). Установите соответствие между графиками функций и знаками коэффициентов \(a\) и \(c\).
		\begin{center}
			\includegraphics[align=t, width=0.7\linewidth]{\picpath/G91M9L4}
		\end{center}
	\begin{tasks}
		\task \( a>0, c>0 \)
		\task \( a<0, c>0 \)
		\task \( a>, c<0 \)
	\end{tasks}
		\item Закон Менделеева-Клапейрона можно записать в виде \(PV=vRT\), где \(P\) --- давление (в паскалях), \(V\) --- объём (в м\(^3\)), \(v\)  --- количество вещества (в молях), \(T\) --- температура (в градусах Кельвина), а \(R\) --- универсальная газовая постоянная, равная \(8,31\) Дж/(К\(\cdot\) моль). Пользуясь этой формулой, найдите температуру \(T\) (в градусах Кельвина), если \(v=68,2\) моль, \(P=37 782,8\) Па, \(V=6\) м\(^3\).
		\item Решите неравенство: \(x^2+x\le0\).
		\item Васе надо решить \(434\) задачи. Ежедневно он решает на одно и то же количество задач больше по сравнению с предыдущим днем. Известно, что за первый день Вася решил \(5\) задач. Определите, сколько задач решил Вася в последний день, если со всеми задачами он справился за \(14\) дней.
		\item В параллелограмм вписана окружность. Найдите периметр параллелограмма, если одна из его сторон равна \(6\).
		\item Точки \(A\) и \(B\) делят окружность на две дуги, длины которых относятся как \(9:11\). Найдите величину центрального угла, опирающегося на меньшую из дуг. Ответ дайте в градусах.
		\item Найдите площадь прямоугольника, если его периметр равен \(44\) и одна сторона на \(2\) больше другой.
		\item На рисунке изображен ромб \(ABCD\). Используя рисунок, найдите  тангенс \(\angle OBC\).
		\begin{center}
			\includegraphics[align=t, width=0.5\linewidth]{\picpath/G91M9L4-1}
		\end{center}
		\item Укажите номера верных утверждений.
		\begin{tasks}
			\task  Биссектриса равнобедренного треугольника, проведённая из вершины, противолежащей основанию, делит основание на две равные части.
			\task В любом прямоугольнике диагонали взаимно перпендикулярны.
			\task Для точки, лежащей на окружности, расстояние до центра окружности равно радиусу.
		\end{tasks}
		\item Решите неравенство: \(\dfrac{ x^2 }{ 3 }<\dfrac{ 3x+3 }{ 4 }\)
		\item При смешивании первого раствора кислоты, концентрация которого \(20\%\), и второго раствора этой же кислоты, концентрация которого \(50\%\), получили раствор, содержащий \(30\%\) кислоты. В каком отношении были взяты первый и второй растворы?
		\item Постройте график функции \(y=\dfrac{ x^4-13x^2+36 }{ (x-3)(x+2) }\) и определите, при каких значениях \(m\) прямая \(y=m\) имеет с графиком ровно одну общую точку.
		\item Прямая, параллельная основаниям \(MP\) и \(NK\) трапеции \(MNKP\), проходит через точку пересечения диагоналей трапеции и пересекает её боковые стороны \(MN\) и \(KP\) в точках  \(A\) и \(B\) соответственно. Найдите длину отрезка \(AB\), если \(MP=40\) см, \(NK=24\) см.
		\item Дан правильный восьмиугольник. Докажите, что если последовательно соединить отрезками середины его сторон, то получится правильный восьмиугольник.
	\end{listofex}
\end{class}
%END_FOLD

%BEGIN_FOLD % ====>>_ Домашняя работа 2 _<<====
\begin{homework}[number=2]
	\begin{listofex}
		\item .
	\end{listofex}
\end{homework}
%END_FOLD

%BEGIN_FOLD % ====>>_____ Занятие 5 _____<<====
\begin{class}[number=5]
	\begin{listofex}
		\item .
	\end{listofex}
\end{class}
%END_FOLD

%BEGIN_FOLD % ====>>_____ Занятие 6 _____<<====
\begin{class}[number=6]
	\begin{listofex}
		\item .
	\end{listofex}
\end{class}
%END_FOLD

%BEGIN_FOLD % ====>>_ Домашняя работа 3 _<<====
\begin{homework}[number=3]
	\begin{listofex}
		\item .
	\end{listofex}
\end{homework}
%END_FOLD

%BEGIN_FOLD % ====>>_____ Занятие 7 _____<<====
\begin{class}[number=7]
	\begin{listofex}
		\item .
	\end{listofex}
\end{class}
%END_FOLD

%BEGIN_FOLD % ====>>_ Проверочная работа _<<====
\begin{class}[number=8]
	\begin{listofex}
		\item .
	\end{listofex}
\end{class}
%END_FOLD