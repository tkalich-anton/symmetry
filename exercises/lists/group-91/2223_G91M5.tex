%
%===============>>  ГРУППА 9-1 МОДУЛЬ 5  <<=============
%
\setmodule{5}

%BEGIN_FOLD % ====>>_____ Занятие 1 _____<<====
\begin{class}[number=1]
	\begin{definit}
		Геометрическая прогрессия
		\[ b_{n+1}=q\cdot b_n  \quad
		 b_n=b_1\cdot q^{n-1}  \quad
		 b_n=\sqrt{b_{n-1}\cdot b_{n+1}}  \quad
		 S_n=\dfrac{b_1(1-q^n)}{1-q} \]
	\end{definit}
		
	\begin{listofex}
		\item Выписаны первые несколько членов геометрической прогрессии: \( 17 \), \( 68 \), \( 272 \)... Найдите её четвёртый член.
		\item Найдите \( b_6 \), \quad если \( b_n=64,5\cdot(-2)^n \).
		\item Найдите \( b_7 \), \quad если \( b_1=-\mfrac{1}{1}{3}	\), \( b_{n+1}=-3b_n \).
		\item Найдите знаменатель геометрической прогрессии, \quad если \( b_5=-14 \), \( b_8=112 \).
		\item Выписаны первые несколько членов геометрической прогрессии: \( -256 \), \( 128 \), \( -64 \)... Найдите сумму первых \( 7 \) членов.
		\item Найдите \( S_4 \), \quad если \( b_n=164\cdot\left( \dfrac{1}{2} \right)^n \)
		\item Бизнесмен Бубликов получил в \( 2000 \) году прибыль в размере \( 5000 \) рублей. Каждый следующий год его прибыль увеличивалась на \( 300\% \) по сравнению с предыдущим годом. Сколько рублей заработал Бубликов за \( 2003 \) год?
		\item Бактерия, попав в живой организм, к концу \( 20 \)-й минуты делится на две бактерии, каждая из них к концу следующих \( 20 \) минут делится опять на две и т. д. Сколько бактерий окажется в организме через \( 4 \) часа, если по истечении четвертого часа в организм из окружающей среды попала еще одна бактерия?
		\item Каждый день больной заражает четырёх человек, каждый из которых, начиная со следующего дня, каждый день также заражает новых четырех и так далее. Болезнь длится \( 14 \) дней. В первый день месяца в город \( N \) приехал заболевший гражданин \( K \), и в это же день он заразил четырех человек. В какой день станет \( 3125 \) заболевших?
		\item Служившему воину дано вознаграждение: за первую рану \( 1 \) копейка, за другую --- \( 2 \) копейки, за третью --- \( 4 \) копейки и т. д. По исчислению нашлось, что воин получил всего вознаграждения \( 655 \) руб. \( 35 \) коп. Спрашивается число его ран.
		\item В амфитеатре \( 12 \) рядов. В первом ряду \( 15 \) мест, а в каждом следующем на \( 3 \) места больше, чем в предыдущем. Сколько всего мест в амфитеатре?
		\item Занятия йогой начинают с \( 15 \) минут в день и увеличивают на \( 10 \) минут время каждый следующий день. Сколько дней следует заниматься йогой в указанном режиме, чтобы суммарная продолжительность занятий составила \( 2 \) часа?
		\item При свободном падении тело прошло в первую секунду \( 5 \) м, а в каждую следующую на \( 10 \) м больше. Найдите глубину шахты, если свободно падающее тело достигло его дна через \( 5 \) с после начала падения.
		\item Бригада маляров красит забор длиной \( 270 \) метров, ежедневно увеличивая норму покраски на одно и то же число метров. Известно, что за первый и последний день в сумме бригада покрасила \( 90 \) метров забора. Определите, сколько дней бригада маляров красила весь забор.
		\item Васе надо решить \( 434 \) задачи. Ежедневно он решает на одно и то же количество задач больше по сравнению с предыдущим днем. Известно, что за первый день Вася решил \( 5 \) задач. Определите, сколько задач решил Вася в последний день, если со всеми задачами он справился за \( 14 \) дней.
	\end{listofex}
\end{class}
%END_FOLD

%BEGIN_FOLD % ====>>_____ Занятие 2 _____<<====
\begin{class}[number=2]
	\begin{listofex}
		\item Бизнесмен Коржов получил в \( 2000 \) году прибыль в размере \( 1 400 000 \) рублей. Каждый следующий год его прибыль увеличивалась на \( 20\% \) по сравнению с предыдущим годом. Сколько рублей составила прибыль Коржова за \( 2004 \) год?
		\item Однажды богач заключил выгодную, как ему казалось, сделку с человеком, который в течение \( 15 \) дней ежедневно должен был приносить по \( 1000 \) р., а взамен в первый день богач должен был отдать \( 10 \) р., во второй --- \( 20 \) р., в третий --- \( 40 \) р., в четвертый --- \( 80 \) р. и т. д. в течение \( 15 \) дней. Сколько денег получил богач и сколько он отдал? Кто выиграл от этой сделки? В ответ запишите, сколько рублей потерял богач за \( 15 \) дней.
		\item Клиент взял в банке кредит в размере \( 50 000 \) р. на \( 5 \) лет под \( 20\% \) годовых. Какую сумму он должен вернуть в банк в конце срока, если проценты начисляются ежегодно на текущую сумму долга и весь кредит с процентами возвращается в банк после срока?
		\item На биржевых торгах в понедельник вечером цена акции банка «Городской» повысилась на некоторое количество процентов, а во вторник произошло снижение стоимости акции на то же число процентов. В результате во вторник вечером цена акции составила \( 99\% \) от ее первоначальной цены в понедельник утром. На сколько процентов менялась котировка акции в понедельник и во вторник?
		\item Алик, Миша и Вася покупали блокноты и трехкопеечные карандаши. Алик купил \( 2 \) блокнота и \( 4 \) карандаша, Миша --- блокнот и \( 6 \) карандашей, Вася --- блокнот и \( 3 \) карандаша. Оказалось, что суммы, которые уплатили Алик, Миша и Вася, образуют геометрическую прогрессию. Сколько стоит блокнот?
		\item Три конькобежца, скорости которых в некотором порядке образуют геометрическую прогрессию, одновременно стартуют (из одного места) по кругу. Через некоторое время второй конькобежец обгоняет первого, пробежав на \( 400 \) метров больше его. Третий конькобежец пробегает то расстояние, который пробежал первый к моменту обгона его вторым, за время на \( \dfrac{2}{3} \) мин больше, чем первый. Найдите скорость первого конькобежца в м/мин.
		\item Ваня, Миша, Алик и Вадим ловили рыбу. Оказалось, что количества рыб, пойманных каждым из них, образуют в указанном порядке арифметическую прогрессию. Если бы Алик поймал столько же рыб, сколько Вадим, а Вадим поймал бы на \( 12 \) рыб больше, то количества рыб, пойманных юношами, образовали бы в том же порядке геометрическую прогрессию. Сколько рыб поймал Миша?
		\item В полночь в организме начало накапливаться ядовитое вещество, причем каждые три часа количество попадающего в организм вещества увеличивается вдвое. Сколько граммов вещества накопится в организме за сутки (начиная с нуля часов), если в период с \( 6 \) до \( 9 \) часов утра в организм попало \( 0,0008 \) г вещества?
		\item Давление воздуха под колоколом равно \( 625 \) мм ртутного столба. Каждую минуту насос откачивает из-под колокола \( 20\% \) находящегося там воздуха. Определите давление (в мм рт. ст.) через \( 5 \) минут после начала работы насоса.
		\item Мощности пяти различных электромоторов составляют возрастающую геометрическую прогрессию. Мощность самого слабого электромотора --- \( 5 \) кВт, а третьего по мощности --- \( 20 \) кВт. Найдите мощность самого мощного электромотора, ответ дайте в кВт.
	\end{listofex}
\end{class}
%END_FOLD

%BEGIN_FOLD % ====>>_ Домашняя работа 1 _<<====
\begin{homework}[number=1]
	\begin{listofex}
		\item Выписаны первые несколько членов геометрической прогрессии: \( 18 \), \( -54 \), \( 162 \)... Найдите её пятый член.
		\item Найдите \( b_6 \), \quad если \( b_n=64\cdot\dfrac{3}{2}^n \).
		\item Найдите \( b_4 \), \quad если \( b_1=5 \), \( b_{n+1}=3b_n \).
		\item Выписаны первые несколько членов геометрической прогрессии: \( -750 \), \( 150 \), \( -30 \)... Найдите сумму первых \( 7 \) членов.
		\item Бактерия, попав в живой организм, к концу \( 30 \)-й минуты делится на две бактерии, каждая из них к концу следующих \( 30 \) минут делится опять на две и т. д. Сколько бактерий окажется в организме через \( 3 \) часа, если по истечении четвертого часа в организм из окружающей среды попала еще одна бактерия?
		\item Вычислите: \quad \(\left( \dfrac{36x}{x^2-81}+\dfrac{x-9}{x+9} \right)\cdot\dfrac{x}{x+9}-\dfrac{x}{x-9}\)
	\end{listofex}
\end{homework}
%END_FOLD

%BEGIN_FOLD % ====>>_____ Занятие 3 _____<<====
\begin{class}[number=3]
	\begin{listofex}
		\item Турист, поднимаясь в гору, достиг в первый час высоту \( 800 \) м, а каждый следующий час поднимался на высоту, на \( 25 \) м меньшую, чем в предыдущий. За сколько часов он достигнет высоты в \( 5700 \) м?
		\item За установку самого нижнего железобетонного кольца колодца заплатили \( 2600 \) р., а за каждое следующее кольцо платили на \( 200 \) р. меньше, чем за предыдущее. Кроме того, по окончании работы было уплачено ещё \( 4000 \) р. Средняя стоимость установки одного кольца оказалась равной \( \mfrac{2244}{4}{9} \) р. Сколько колец было установлено?
		\item Первый член арифметической прогрессии равен \( 429 \), разность её равна \( -22 \). Сколько членов этой прогрессии нужно взять, чтобы их сумма была равна \( 3069 \)?
		\item В соревновании по стрельбе за каждый промах в серии из \( 25 \) выстрелов стрелок получал штрафные очки: за первый промах --- одно штрафное очко, а за каждый последующий --- на \( 0,5 \) очка больше, чем за предыдущий. Сколько раз попал в цель стрелок, получивший \( 7 \) штрафных очков?
		\item Постройте график функции \( y=-2-\dfrac{x^4-x^3}{x^2-x} \) и определите, при каких значениях \( m \) прямая \( y=m \) имеет с графиком ровно две общие точки.
		\item Постройте график функции \( y=\dfrac{(x^2+7x+12)(x^2-x-2)}{x^2+5x+4} \) и определите, при каких значениях \( m \) прямая \( y=m \) имеет с графиком ровно одну общую точку.
	\end{listofex}
\end{class}
%END_FOLD

%BEGIN_FOLD % ====>>_____ Занятие 4 _____<<====
\begin{class}[number=4]
	\begin{listofex}
		\item Занятие 4
	\end{listofex}
\end{class}
%END_FOLD

%BEGIN_FOLD % ====>>_ Домашняя работа 2 _<<====
\begin{homework}[number=2]
	\begin{listofex}
		\item Постройте график функции \( y=\dfrac{(x+4)(x^2+3x+2)}{x+1} \) определите, при каких значениях \( m \) прямая \( y=m \) имеет с графиком ровно одну общую точку.
		\item Найдите сумму всех натуральных чисел от \( 2 \) до \( 98 \) включительно, используя арифметическую прогрессию.
		\item При проведении опыта вещество равномерно охлаждали в течение \( 10 \) минут. При этом каждую минуту температура вещества уменьшалась на \( 8\degree \) C. Найдите температуру вещества (в градусах Цельсия) через \( 6 \) минут после начала проведения опыта, если его начальная температура составляла \( -6\degree \) C.
	\end{listofex}
\end{homework}
%END_FOLD

%BEGIN_FOLD % ====>>_____ Занятие 5 _____<<====
\begin{class}[number=5]
	\begin{listofex}
		\item Занятие 5
	\end{listofex}
\end{class}
%END_FOLD

%BEGIN_FOLD % ====>>_____ Занятие 6 _____<<====
\begin{class}[number=6]
	\begin{listofex}
		\item Занятие 6
	\end{listofex}
\end{class}
%END_FOLD

%BEGIN_FOLD % ====>>_ Домашняя работа 3 _<<====
\begin{homework}[number=3]
	\begin{listofex}
		\item ДЗ 3
	\end{listofex}
\end{homework}
%END_FOLD

%BEGIN_FOLD % ====>>_____ Занятие 7 _____<<====
\begin{class}[number=7]
	\begin{listofex}
		\item Занятие 7
	\end{listofex}
\end{class}
%END_FOLD

%BEGIN_FOLD % ====>>_ Проверочная работа _<<====
\begin{exam}
	\begin{listofex}
		\item Проверочная работа
	\end{listofex}
\end{exam}
%END_FOLD