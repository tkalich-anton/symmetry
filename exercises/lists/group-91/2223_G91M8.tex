%
%===============>>  ГРУППА 9-1 МОДУЛЬ 8  <<=============
%
\setmodule{8}

%BEGIN_FOLD % ====>>_____ Занятие 1 _____<<====
\begin{class}[number=1]
	\begin{listofex}
		\item Какие из следующих утверждений верны?
		\begin{tasks}(1)
			\task Через точку, не лежащую на данной прямой, можно провести прямую, параллельную этой
			прямой.
			\task Треугольник со сторонами \( 1 \), \( 2 \), \( 4 \) существует.
			\task В любом параллелограмме есть два равных угла.
		\end{tasks}
		\item Какие из следующих утверждений верны?
		\begin{tasks}(1)
			\task Любые два прямоугольных треугольника подобны.
			\task Если катет и гипотенуза прямоугольного треугольника равны соответственно \( 6 \) и \( 10 \), то второй катет этого треугольника равен \( 8 \).
			\task Стороны треугольника пропорциональны косинусам противолежащих углов.
			\task Квадрат любой стороны треугольника равен сумме квадратов двух других сторон без удвоенного произведения этих сторон на косинус угла между ними.
		\end{tasks}	
	\item Укажите номера верных утверждений.
	\begin{tasks}(1)
		\task Если угол равен \( 45\degree \), то вертикальный с ним угол равен \( 45\degree \).
		\task Любые две прямые имеют ровно одну общую точку.
		\task Через любые три точки проходит ровно одна прямая.
		\task Если расстояние от точки до прямой меньше \( 1 \), то и длина любой наклонной, проведенной из данной точки к прямой, меньше \( 1 \).
	\end{tasks}
	\item Укажите номера верных утверждений.
	\begin{tasks}(1)
		\task Если при пересечении двух прямых третьей прямой соответственные углы равны \( 65\degree \), то эти две прямые параллельны.
		\task Любые две прямые имеют не менее одной общей точки.
		\task Через любую точку проходит более одной прямой.
		\task Любые три прямые имеют не менее одной общей точки.
	\end{tasks}
	\item Укажите номера верных утверждений.
	\begin{tasks}(1)
		\task  Если при пересечении двух прямых третьей прямой внутренние накрест лежащие углы составляют в сумме \( 90\degree \), то эти две прямые параллельны.
		\task Если угол равен \( 60\degree \), то смежный с ним равен \( 120\degree \).
		\task Если при пересечении двух прямых третьей прямой внутренние односторонние углы равны \( 70\degree \) и \( 110\degree \), то эти две прямые параллельны.
		\task Через любые три точки проходит не более одной прямой.
	\end{tasks}
	\item Укажите номера верных утверждений.
	\begin{tasks}(1)
		\task Вписанные углы, опирающиеся на одну и ту же хорду окружности, равны.
		\task Если радиусы двух окружностей равны \( 5 \) и \( 7 \), а расстояние между их центрами равно \( 3 \), то эти окружности не имеют общих точек.
		\task Если радиус окружности равен \( 3 \), а расстояние от центра окружности до прямой равно \( 2 \), то эти прямая и окружность пересекаются.
		\task Если вписанный угол равен \( 30\degree \), то дуга окружности, на которую опирается этот угол, равна \( 60\degree \).
	\end{tasks}
	\item Укажите номера верных утверждений.
	\begin{tasks}(1)
		\task Через любые три точки проходит не более одной окружности.
		\task Если расстояние между центрами двух окружностей больше суммы их диаметров, то эти окружности не имеют общих точек.
		\task Если радиусы двух окружностей равны \( 3 \) и \( 5 \), а расстояние между их центрами равно \( 1 \), то эти окружности пересекаются.
		\task Если дуга окружности составляет \( 80\degree \), то вписанный угол, опирающийся на эту дугу окружности, равен \( 40\degree \).
	\end{tasks}
	\item Решите системы уравнений:
	\begin{tasks}(2)
		\task \( \begin{cases}
			(2x+3)^2=5y,\\
			(3x+2)^2=5y
		\end{cases} \)
		\task \( \begin{cases}
			x-y=-5,\\
			x^2-2xy-y^2=17
		\end{cases} \)
		\task \( \begin{cases}
			x^2+3x+y^2=2,\\
			x^2+3x-y^2=-6
		\end{cases} \)
		\task \( \begin{cases}
			3x-y=2,\\
			x^2-4x+8=y
		\end{cases} \)
		\task \( \begin{cases}
			3x+y=5,\\
			\dfrac{x+2}{y}+\dfrac{y}{2}=-1
		\end{cases} \)
	\end{tasks}
	\item Решите уравнение \( (x^2-9)^2+(x^2-2x-15)^2=0 \)
	\end{listofex}
\end{class}
%END_FOLD

%BEGIN_FOLD % ====>>_____ Занятие 2 _____<<====
\begin{class}[number=2]
	\begin{listofex}
			\item Укажите номера верных утверждений.
		\begin{tasks}(1)
			\task Около всякого треугольника можно описать не более одной окружности.
			\task В любой треугольник можно вписать не менее одной окружности.
			\task Центром окружности, описанной около треугольника, является точка пересечения биссектрис.
			\task Центром окружности, вписанной в треугольник, является точка пересечения серединных перпендикуляров к его сторонам.
		\end{tasks}
		\item Укажите номера верных утверждений.
		\begin{tasks}(1)
			\task Около любого правильного многоугольника можно описать не более одной окружности.
			\task Центр окружности, описанной около треугольника со сторонами, равными \( 3 \), \( 4 \), \( 5 \), находится на стороне этого треугольника.
			\task Центром окружности, описанной около квадрата, является точка пересечения его диагоналей.
			\task Около любого ромба можно описать окружность.
		\end{tasks}
		\item Укажите номера верных утверждений.
		\begin{tasks}(1)
			\task Окружность имеет бесконечно много центров симметрии.
			\task Прямая не имеет осей симметрии.
			\task Правильный пятиугольник имеет пять осей симметрии.
			\task Квадрат не имеет центра симметрии.
		\end{tasks}
		\item Укажите номера верных утверждений.
		\begin{tasks}(1)
			\task Правильный шестиугольник имеет шесть осей симметрии.
			\task Прямая не имеет осей симметрии.
			\task Центром симметрии ромба является точка пересечения его диагоналей.
			\task Равнобедренный треугольник имеет три оси симметрии.
		\end{tasks}
	\newpage
		\item Укажите номера верных утверждений.
		\begin{tasks}(1)
			\task Центром симметрии прямоугольника является точка пересечения диагоналей.
			\task Центром симметрии ромба является точка пересечения его диагоналей.
			\task Правильный пятиугольник имеет пять осей симметрии.
			\task Центром симметрии равнобедренной трапеции является точка пересечения ее диагоналей.
		\end{tasks}
		\item Укажите номера верных утверждений.
		\begin{tasks}(1)
			\task Если катет и гипотенуза прямоугольного треугольника равны соответственно \( 6 \) и \( 10 \), то второй катет этого треугольника равен \( 8 \).
			\task Любые два равнобедренных треугольника подобны.
			\task Любые два прямоугольных треугольника подобны.
			\task Треугольник \( ABC \), у которого \( AB=3 \), \( BC=4 \), \( AC=5 \), является тупоугольным.
		\end{tasks}
		\item Решите системы неравенств:
		\begin{tasks}(2)
			\task \( \begin{cases}
				 7(3x+2)-3(7x+2)>2x ,\\
				(x-5)(x+8)<0
			\end{cases} \)
			\task \( \begin{cases}
				4(9x+3)-9(4x+3)>3x,\\
				(x-2)(x+9)<0
			\end{cases} \)
			\task \( \begin{cases}
			(6x+2)-6(x+2)>2x,\\
			(x-7)(x+6)<0
			\end{cases} \)
			\task \( \begin{cases}
			5(2x+3)-2(5x+3)>3x,\\
			(x-6)(x+2)<0
			\end{cases} \)
			\task \( \begin{cases}
			\dfrac{10-2x}{3+(5-2x)^2}\ge0,\\
			2-7x\le14-3x
			\end{cases} \)
			\task \( \begin{cases}
			\dfrac{3-x}{1+(5-x)^2}\ge0,\\
			8-7x\le24-3x
			\end{cases} \)
		\end{tasks}
		\item Решите уравнение \( (x^2-9)^2+(x^2-2x-15)^2=0 \)
	\end{listofex}
\end{class}
%END_FOLD

%BEGIN_FOLD % ====>>_ Домашняя работа 1 _<<====
\begin{homework}[number=1]
	\begin{listofex}
		\item Какие из следующих утверждений верны?
		\begin{tasks}(1)
			\task Около всякого треугольника можно описать не более одной окружности.
			\task В любой треугольник можно вписать не менее одной окружности.
			\task Центром окружности, описанной около треугольника, является точка пересечения биссектрис.
			\task Центром окружности, вписанной в треугольник, является точка пересечения серединных перпендикуляров к его сторонам.
		\end{tasks}
		\item Какие из следующих утверждений верны?
		\begin{tasks}(1)
			\task Косинус острого угла прямоугольного треугольника равен отношению гипотенузы к прилежащему к этому углу катету.
			\task Диагонали ромба перпендикулярны.
			\task Существуют три прямые, которые проходят через одну точку.
		\end{tasks}
		\item Какие из следующих утверждений верны?
		\begin{tasks}(1)
			\task Все хорды одной окружности равны между собой.
			\task Треугольника со сторонами \( 1 \), \( 2 \), \( 4 \) не существует.
			\task Все углы прямоугольника равны.
		\end{tasks}
		\item Решите системы уравнений:
		\begin{tasks}(2)
			\task \( \begin{cases}
				5x+y=-13,\\
				x^2+y^2=13
			\end{cases} \)
			\task \( \begin{cases}
				2x^2+y^2=36,\\
				8x^2+4y^2=36x
			\end{cases} \)
		\end{tasks}
		\item Решите системы неравенств:
		\begin{tasks}(2)
			\task \( \begin{cases}
				\dfrac{6-3x}{4+(9-2x)^2}\ge0,\\
			 	5-8x\le23-5x
			\end{cases} \)
			\task \( \begin{cases}
				8(3x+5)-3(8x+5)>5,\\
				(x-8)(x+1)<0
			\end{cases} \)
		\end{tasks}
		\item Имеются два сосуда, содержащие \( 30 \) кг и \( 20 \) кг раствора кислоты различной концентрации. Если их слить вместе, то получим раствор, содержащий \( 81\% \) кислоты. Если же слить равные массы этих растворов, то полученный раствор будет содержать \( 83\% \) кислоты. Сколько килограммов кислоты содержится во втором растворе?
	\end{listofex}
\end{homework}
%END_FOLD

%BEGIN_FOLD % ====>>_____ Занятие 3 _____<<====
\begin{class}[number=3]
	\begin{listofex}
		\item Один из корней уравнения \( 3x^2+5x-2m=0 \) равен \( -1 \). Найдите второй корень.
		\item Решите уравнения:
		\begin{tasks}(2)
			\task \( \dfrac{2x^2+7x+3}{x^2-9}=1 \)
			\task \( x^6=(6x-5)^3 \)
			\task \( (x+2)^4-4(x+2)^2-5=0 \)
			\task \( \dfrac{1}{(x-2)^2}-\dfrac{1}{x-2}-6=0 \)
		\end{tasks}
		\item Один мастер может выполнить заказ за \( 12 \) часов, а другой --- за \( 6 \) часов. За сколько часов выполнят заказ оба мастера, работая вместе?
		\item На изготовление \( 99 \) деталей первый рабочий тратит на \( 2 \) часа меньше, чем второй рабочий на изготовление \( 110 \) таких же деталей. Известно, что первый рабочий за час делает на \( 1 \) деталь больше, чем второй. Сколько деталей в час делает второй рабочий?
		\item На изготовление \( 231 \) детали ученик тратит на \( 11 \) часов больше, чем мастер на изготовление \( 462 \) таких же деталей. Известно, что ученик за час делает на \( 4 \) детали меньше, чем мастер. Сколько деталей в час делает ученик?
		\item Первая труба пропускает на \( 1 \) литр воды в минуту меньше, чем вторая. Сколько литров воды в минуту пропускает первая труба, если резервуар объемом \( 110 \) литров она заполняет на \( 1 \) минуту дольше, чем вторая труба?
		\item Первая труба пропускает на \( 5 \) литров воды в минуту меньше, чем вторая. Сколько литров воды в минуту пропускает вторая труба, если резервуар объемом \( 375 \) литров она заполняет на \( 10 \) минут быстрее, чем первая труба заполняет резервуар объемом \( 500 \) литров?
	\end{listofex}
\end{class}
%END_FOLD

%BEGIN_FOLD % ====>>_____ Занятие 4 _____<<====
\begin{class}[number=4]
	\begin{listofex}
		\item Постройте график:
		\[ y=	 \left\{
		\begin{array}{l}
			1,5x-3, \quad x<2,\\
			-1,5x+3, \quad 2\leq x\leq3,\\
			3x-10,5, \quad x>3.
		\end{array}
		\right. \]
		и определите, при каких значениях \( m \) прямая \( y=m \) имеет с графиком ровно две общие точки.
		\item  Постройте график функции
		\[y=	 \left\{
		\begin{array}{l}
			2x+1, \quad x<0,\\
			-1,5x+1, \quad 0\leq x<2,\\
			x-4, \quad x\geq 2
		\end{array}
		\right. \]
		и определите, при каких значениях \( m \) прямая \( y=m \) имеет с графиком ровно две общие точки.
		\item Постройте график функции
		\[y=	 \left\{
		\begin{array}{l}
			x^2+2x+3, \quad x\leq-3,\\
			x+9, \quad x>-3
		\end{array}
		\right. \]
		и определите, при каких значениях \( m \) прямая \( y=m \) имеет с графиком ровно две общие точки.
		\item Построить график функции \( y=x-|2x+1|-2 \). Найти точки пересечения данного графика с графиком функции \( y=x-5 \).
		\item Постройте график функции \( y=x^2-3|x|-x \)  и определите, при каких значениях \( c \)  прямая \( y=c \)  имеет с графиком три общие точки.
	\end{listofex}
\end{class}
%END_FOLD

%BEGIN_FOLD % ====>>_ Домашняя работа 2 _<<====
\begin{homework}[number=2]
	\begin{listofex}
		\item Домашняя работа 2
	\end{listofex}
\end{homework}
%END_FOLD

%BEGIN_FOLD % ====>>_____ Занятие 5 _____<<====
\begin{class}[number=5]
	\begin{listofex}
		\item Занятие 5
	\end{listofex}
\end{class}
%END_FOLD

%BEGIN_FOLD % ====>>_____ Занятие 6 _____<<====
\begin{class}[number=6]
	\begin{listofex}
		\item Занятие 6
	\end{listofex}
\end{class}
%END_FOLD

%BEGIN_FOLD % ====>>_ Домашняя работа 3 _<<====
\begin{homework}[number=3]
	\begin{listofex}
		\item Домашняя работа 3
	\end{listofex}
\end{homework}
%END_FOLD

%BEGIN_FOLD % ====>>_____ Занятие 7 _____<<====
\begin{class}[number=7]
	\title{Подготовка к проверочной}
	\begin{listofex}
		\item Занятие 7
	\end{listofex}
\end{class}
%END_FOLD

=%BEGIN_FOLD % ====>>_ Проверочная работа _<<====
\begin{exam}
	\begin{listofex}
		\item Проверочная
	\end{listofex}
\end{exam}
%END_FOLD