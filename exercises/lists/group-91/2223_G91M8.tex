%
%===============>>  ГРУППА 9-1 МОДУЛЬ 8  <<=============
%
\setmodule{8}

%BEGIN_FOLD % ====>>_____ Занятие 1 _____<<====
\begin{class}[number=1]
	\begin{listofex}
		\item Какие из следующих утверждений верны?
		\begin{tasks}(1)
			\task Через точку, не лежащую на данной прямой, можно провести прямую, параллельную этой
			прямой.
			\task Треугольник со сторонами \( 1 \), \( 2 \), \( 4 \) существует.
			\task В любом параллелограмме есть два равных угла.
		\end{tasks}
		\item Какие из следующих утверждений верны?
		\begin{tasks}(1)
			\task Любые два прямоугольных треугольника подобны.
			\task Если катет и гипотенуза прямоугольного треугольника равны соответственно \( 6 \) и \( 10 \), то второй катет этого треугольника равен \( 8 \).
			\task Стороны треугольника пропорциональны косинусам противолежащих углов.
			\task Квадрат любой стороны треугольника равен сумме квадратов двух других сторон без удвоенного произведения этих сторон на косинус угла между ними.
		\end{tasks}
	\item Укажите номера верных утверждений.
	\begin{tasks}(1)
		\task Если угол равен \( 45\degree \), то вертикальный с ним угол равен \( 45\degree \).
		\task Любые две прямые имеют ровно одну общую точку.
		\task Через любые три точки проходит ровно одна прямая.
		\task Если расстояние от точки до прямой меньше \( 1 \), то и длина любой наклонной, проведенной из данной точки к прямой, меньше \( 1 \).
	\end{tasks}
	\item Укажите номера верных утверждений.
	\begin{tasks}(1)
		\task Если при пересечении двух прямых третьей прямой соответственные углы равны \( 65\degree \), то эти две прямые параллельны.
		\task Любые две прямые имеют не менее одной общей точки.
		\task Через любую точку проходит более одной прямой.
		\task Любые три прямые имеют не менее одной общей точки.
	\end{tasks}
	\item Укажите номера верных утверждений.
	\begin{tasks}(1)
		\task  Если при пересечении двух прямых третьей прямой внутренние накрест лежащие углы составляют в сумме \( 90\degree \), то эти две прямые параллельны.
		\task Если угол равен \( 60\degree \), то смежный с ним равен \( 120\degree \).
		\task Если при пересечении двух прямых третьей прямой внутренние односторонние углы равны \( 70\degree \) и \( 110\degree \), то эти две прямые параллельны.
		\task Через любые три точки проходит не более одной прямой.
	\end{tasks}
	\item Укажите номера верных утверждений.
	\begin{tasks}(1)
		\task Вписанные углы, опирающиеся на одну и ту же хорду окружности, равны.
		\task Если радиусы двух окружностей равны \( 5 \) и \( 7 \), а расстояние между их центрами равно \( 3 \), то эти окружности не имеют общих точек.
		\task Если радиус окружности равен \( 3 \), а расстояние от центра окружности до прямой равно \( 2 \), то эти прямая и окружность пересекаются.
		\task Если вписанный угол равен \( 30\degree \), то дуга окружности, на которую опирается этот угол, равна \( 60\degree \).
	\end{tasks}
	\item Укажите номера верных утверждений.
	\begin{tasks}(1)
		\task Через любые три точки проходит не более одной окружности.
		\task Если расстояние между центрами двух окружностей больше суммы их диаметров, то эти окружности не имеют общих точек.
		\task Если радиусы двух окружностей равны \( 3 \) и \( 5 \), а расстояние между их центрами равно \( 1 \), то эти окружности пересекаются.
		\task Если дуга окружности составляет \( 80\degree \), то вписанный угол, опирающийся на эту дугу окружности, равен \( 40\degree \).
	\end{tasks}
	\item Решите системы уравнений:
	\begin{tasks}(2)
		\task \( \begin{cases}
			(2x+3)^2=5y,\\
			(3x+2)^2=5y
		\end{cases} \)
		\task \( \begin{cases}
			x-y=-5,\\
			x^2-2xy-y^2=17
		\end{cases} \)
		\task \( \begin{cases}
			x^2+3x+y^2=2,\\
			x^2+3x-y^2=-6
		\end{cases} \)
		\task \( \begin{cases}
			3x-y=2,\\
			x^2-4x+8=y
		\end{cases} \)
		\task \( \begin{cases}
			3x+y=5,\\
			\dfrac{x+2}{y}+\dfrac{y}{2}=-1
		\end{cases} \)
	\end{tasks}
	\item Решите уравнение \( (x^2-9)^2+(x^2-2x-15)^2=0 \)
	\end{listofex}
\end{class}
%END_FOLD

%BEGIN_FOLD % ====>>_____ Занятие 2 _____<<====
\begin{class}[number=2]
	\begin{listofex}
		\item Занятие 2
	\end{listofex}
\end{class}
%END_FOLD

%BEGIN_FOLD % ====>>_ Домашняя работа 1 _<<====
\begin{homework}[number=1]
	\begin{listofex}
		\item Домашняя работа 1
	\end{listofex}
\end{homework}
%END_FOLD

%BEGIN_FOLD % ====>>_____ Занятие 3 _____<<====
\begin{class}[number=3]
	\begin{listofex}
		\item Занятие 3 
	\end{listofex}
\end{class}
%END_FOLD

%BEGIN_FOLD % ====>>_____ Занятие 4 _____<<====
\begin{class}[number=4]
	\begin{listofex}
		\item Занятие 4
	\end{listofex}
\end{class}
%END_FOLD

%BEGIN_FOLD % ====>>_ Домашняя работа 2 _<<====
\begin{homework}[number=2]
	\begin{listofex}
		\item Домашняя работа 2
	\end{listofex}
\end{homework}
%END_FOLD

%BEGIN_FOLD % ====>>_____ Занятие 5 _____<<====
\begin{class}[number=5]
	\begin{listofex}
		\item Занятие 5
	\end{listofex}
\end{class}
%END_FOLD

%BEGIN_FOLD % ====>>_____ Занятие 6 _____<<====
\begin{class}[number=6]
	\begin{listofex}
		\item Занятие 6
	\end{listofex}
\end{class}
%END_FOLD

%BEGIN_FOLD % ====>>_ Домашняя работа 3 _<<====
\begin{homework}[number=3]
	\begin{listofex}
		\item Домашняя работа 3
	\end{listofex}
\end{homework}
%END_FOLD

%BEGIN_FOLD % ====>>_____ Занятие 7 _____<<====
\begin{class}[number=7]
	\title{Подготовка к проверочной}
	\begin{listofex}
		\item Занятие 7
	\end{listofex}
\end{class}
%END_FOLD

=%BEGIN_FOLD % ====>>_ Проверочная работа _<<====
\begin{exam}
	\begin{listofex}
		\item Проверочная
	\end{listofex}
\end{exam}
%END_FOLD