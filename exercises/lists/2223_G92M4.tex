%
%===============>>  ГРУППА 9-2 МОДУЛЬ 4  <<=============
%
\setmodule{4}
%
%===============>>  Занятие 1  <<===============
%
\begin{class}[number=1]
	\begin{listofex}
		%		В ДЗ №1
		%		\item Вычислить рациональным способом:
		%		\begin{enumcols}[itemcolumns=3]
		%			\item \( \sqrt{16+4\cdot4\cdot24} \)
		%			\item \( \sqrt{83^3\cdot2^2-83^2\cdot2^3} \)
		%			\item \( \sqrt{50^2-4\cdot7\cdot7} \)
		%		\end{enumcols}
		\item Построить график функции \( y=2x-5 \).
		\begin{enumcols}[itemcolumns=1]
			\item Принадлежит ли точка с координатами \( (112;217) \) графику этой функции?
			\item Найти точку пересечения графика данной функции с графиком функции \( y=4x-1 \).
		\end{enumcols}
		\item Построить графики функций \( f(x)=x^2-2x+1 \) и \( g(x)=x^2+6x+8 \).\\
		Выберите верные утверждения:
		\begin{enumcols}[itemcolumns=4]
			\item \( f(-1)>f(1) \);
			\item \( g(-1)<f(-1) \);
			\item \( g(-2)=f(1) \);
			\item \( f(3)<g(-5) \);
		\end{enumcols}
		\begin{enumcols}[itemcolumns=1, resume]
			\item график функции \( f(x) \) возрастает на промежутке \( [0;+\infty) \);
			\item график функции \( g(x) \) убывает на промежутке \( [-9;-4] \);
			\item промежуток возрастания функции \( f(x) \) входит в промежуток возрастания функции \( g(x) \);
			\item \( f(x)=4 \) при \( x=3 \);
			\item наименьшее значение функция \( f(x) \) принимает в \( x=-1 \);
			\item наименьшее значение \( f(x) \) меньше наименьшего значения \( g(x) \);
			\item корнем уравнения \( f(x)=9 \) является только \( x=-2 \);
			\item \( f(x)\ge0 \) при любом \( x \);
			\item \( g(x)\le0 \) при \( x\in[-4;-2] \);
			\item график функции \( g(x) \) пересекается c осью \( Y \) в точке \( (0;4) \).
		\end{enumcols}
		\item Решить систему неравенств:
		\begin{enumcols}[itemcolumns=2]
			\item
			\( \left\{
			\begin{array}{l}
				3(x-1)-2(2-3x)>5x-3,\\
				8x-3(2x+5)<2(x-7).
			\end{array}
			\right. \)
			\item
			\( \left\{
			\begin{array}{l}
				12x-3(x-5)>2x+1,\\
				(x-7)(x+12)\le0.
			\end{array}
			\right. \)
		\end{enumcols}
		
		\item При каких значениях переменной выражение \( \sqrt{4x+1}+\sqrt{2-3x} \) имеет смысл?
		\item Построить график функции \( f(x)=\sqrt{x-4}+1 \).
		\begin{enumcols}[itemcolumns=1]
			\item определить область определения и область значений функции;
			\item сравнить \( f(7) \) и \( f(12) \);
			\item вычислить \( f(x)=26 \)
			\item существует ли \( x \), при котором \( f(x)=-1 \)? Объясните это графически и аналитически.
		\end{enumcols}
	\end{listofex}
	\textbf{Дежурные задачи}
	\begin{listofex}
		\item Первые \( 300 \) км автомобиль ехал со скоростью \( 60 \) км/ч, следующие 300 км --- со скоростью \( 100 \) км/ч, а последние \( 300 \) км --- со скоростью \( 75 \) км/ч. Найдите среднюю скорость автомобиля на протяжении всего пути.
		\item Велосипедист выехал с постоянной скоростью из города \( А \) в город \( В \), расстояние между которыми равно \( 60 \) км. Отдохнув, он отправился обратно в \( А \), увеличив скорость на \( 10 \) км/ч. По пути он сделал остановку на \( 3 \) часа, в
		результате чего затратил на обратный путь столько же времени, сколько на путь из \( А \) в \( В \). Найдите скорость велосипедиста на пути из \( А \) в \( В \).
		\item Баржа прошла по течению реки 48 км и, повернув обратно, прошла ещё 36 км, затратив на весь путь 6 часов. Найдите собственную скорость баржи, если скорость течения реки равна 5 км/ч.
	\end{listofex}
\end{class}
%
%===============>>  Занятие 2  <<===============
%
%\begin{class}[number=2]
%	\begin{listofex}
%		\item Построить график функции \( f(x)=\dfrac{(x-7)(x-5)^2}{x-7} \). Найти область определения и область значений данной функции.
%	\end{listofex}
%\end{class}
%
%===============>>  Домашняя работа 1  <<===============
%
%\begin{homework}[number=1]
%	\begin{listofex}
%		\item Пусто
%	\end{listofex}
%\end{homework}
%
%===============>>  Занятие 3  <<===============
%
%\begin{class}[number=3]
%	\begin{listofex}
%		\item Пусто
%	\end{listofex}
%\end{class}
%
%===============>>  Занятие 4  <<===============
% смещение на одно занятие с прошлого месяца
%\begin{class}[number=4]
%	\begin{listofex}
%		\item Пусто
%	\end{listofex}
%\end{class}
%
%===============>>  Домашняя работа 2  <<===============
%
%\begin{homework}[number=2]
%	\begin{listofex}
%
%	\end{listofex}
%\end{homework}
%
%===============>>  Занятие 5  <<===============
% смещение на одно занятие с прошлого месяца
%\begin{class}[number=5]
%	\begin{listofex}
%		\item Пусто
%	\end{listofex}
%\end{class}
%
%===============>>  Домашняя работа 3  <<===============
%
%\begin{homework}[number=2]
%	\begin{listofex}
%
%	\end{listofex}
%\end{homework}
%\newpage
%\title{Подготовка к проверочной работе}
%\begin{listofex}
%	
%\end{listofex}
%
%===============>>  Занятие 7  <<===============
%
%\begin{class}[number=7]
%	\begin{listofex}
%	
%	\end{listofex}
%\end{class}
%
%===============>>  Провечная работа  <<===============
%
%\begin{exam}
%	\begin{listofex}
%	
%	\end{listofex}
%\end{exam}