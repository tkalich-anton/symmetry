%Группа 9-1 Модуль 2
\title{Занятие №1}
\begin{listofex}
	\item \exercise{2350}
	\item \exercise{2095}
	\item \exercise{2390}
	\item \exercise{2381}
	\item \exercise{2380}
	\item \exercise{2393}
	\item \exercise{2559}
	\item \exercise{2040}
\end{listofex}
\newpage
\title{Занятие №2}
\begin{listofex}
	\item \exercise{1115}
	\item \exercise{2365}
	\item \exercise{2562}
	\item \exercise{2385}
	\item \exercise{2685}
	\item \exercise{2391}
	\item \exercise{2396}
	\item \exercise{2424}
	\item \exercise{2369}
\end{listofex}
\newpage
\title{Домашняя работа №1}
\begin{listofex}
	\item \exercise{1484}
	\item \exercise{1114}
	\item \exercise{4140}
	\item \exercise{2408}
	\item \exercise{2389}
	\item \exercise{2401}
	\item \exercise{2366}
\end{listofex}
\newpage
\title{Занятие №3}
\begin{listofex}
	\item Докажите следующие свойства окружности:
	\begin{enumcols}[itemcolumns=1]
		\item диаметр, перпендикулярный хорде, делит ее пополам;
		\item диаметр, проходящий через середину хорды, не являющейся диаметром, перпендикулярен этой хорде;
		\item хорды, удаленные от центра окружности на равные расстояния, равны.
	\end{enumcols}
	\item \exercise{2437}
	\item \exercise{2439}(без использования свойств центральных и вписанных углов)
	\item \exercise{2442}
	\item \exercise{2444}
	\item \exercise{2445}
	\item Докажите, что если треугольник вписан в окружность и одна из его сторон является диаметром этой окружности, то такой треугольник является прямоугольным.
	\item Центр окружности, описанной около треугольника, симметричен центру окружности, вписанной в этот треугольник, относительно одной из сторон. Найдите углы треугольника.
	\item Через точку \( A \) проведена прямая, пересекающая
	окружность с диаметром \( AB \) в точке \( K \), отличной от \( A \), а
	окружность с центром \( B \) --- в точках \( M \) и \( N \). Докажите, что \( MK = KN \).
\end{listofex}
\newpage
\title{Занятие №4}
\begin{listofex}
	\item \exercise{2438}
	\item \exercise{2441}
	\item \exercise{2443}
	\item \exercise{2514}
	\item \exercise{2460}
	\item \exercise{2468}
	\item \exercise{2453}
	\item Окружность, построенная на биссектрисе \( AD \) треугольника \( ABC \) как на диаметре, пересекает стороны \( AB \) и \( AC \) соответственно в точках \( M \) и \( N \), отличных от \( A \). Докажите, что \( AM = AN \).
\end{listofex}
\newpage
\title{Домашняя работа №2}
\begin{listofex}
	\item \exercise{1522}
	\item \exercise{1317}
	\item \exercise{2436}
	\item \exercise{2440}
	\item \exercise{2454}
	\item \exercise{2457}
	\item \exercise{2459}
\end{listofex}
\newpage
\title{Занятие №5}
\begin{listofex}
	\item \exercise{2472}
	\item \exercise{2473}
	\item \exercise{2474}
	\item \exercise{2483}
	\item \exercise{2493}
	\item \exercise{2508}
	\item \exercise{1608}
	\item \exercise{3664}
\end{listofex}
\newpage
\title{Занятие №6}
\begin{listofex}
	\item \exercise{2477}
	\item \exercise{2481}
	\item \exercise{2484}
	\item \exercise{2486}
	\item \exercise{2479}
	\item \exercise{2500}
	\item \exercise{2502}
	\item \exercise{2506}
\end{listofex}
\newpage
\title{Домашняя работа №3}
\begin{listofex}
	\item \exercise{2475}
	\item \exercise{2476}
	\item \exercise{2478}
	\item \exercise{2485}
	\item \exercise{2502}
	\item \exercise{2480}
	\item \exercise{2501}
\end{listofex}
%\newpage
%\title{Занятие №7}
%\begin{listofex}
%
%\end{listofex}
%\newpage
%\title{Проверочная работа}
%\begin{listofex}
%
%\end{listofex}