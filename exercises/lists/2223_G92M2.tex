%Группа 9-1 Модуль 2
\title{Занятие №1}
\begin{listofex}
	\item \exercise{2350}
	\item \exercise{2095}
	\item \exercise{2390}
	\item \exercise{2381}
	\item \exercise{2380}
	\item \exercise{2393}
	\item \exercise{2559}
	\item \exercise{2040}
\end{listofex}
\newpage
\title{Занятие №2}
\begin{listofex}
	\item \exercise{1115}
	\item \exercise{2365}
	\item \exercise{2562}
	\item \exercise{2385}
	\item \exercise{2685}
	\item \exercise{2391}
	\item \exercise{2396}
	\item \exercise{2424}
	\item \exercise{2369}
\end{listofex}
\newpage
\title{Домашняя работа №1}
\begin{listofex}
	\item \exercise{1484}
	\item \exercise{1114}
	\item \exercise{4140}
	\item \exercise{2408}
	\item \exercise{2389}
	\item \exercise{2401}
	\item \exercise{2366}
\end{listofex}
\newpage
\title{Занятие №3}
\begin{listofex}
	\item Докажите следующие свойства окружности:
	\begin{enumcols}[itemcolumns=1]
		\item диаметр, перпендикулярный хорде, делит ее пополам;
		\item диаметр, проходящий через середину хорды, не являющейся диаметром, перпендикулярен этой хорде;
		\item окружность симметрична относительно каждого своего
		диаметра;
		\item дуги окружности, заключенные между параллельными
		хордами, равны;
		\item хорды, удаленные от центра окружности на равные расстояния, равны.
	\end{enumcols}
	\item \exercise{2437}
	\item \exercise{2439}
	\item \exercise{2442}
	\item \exercise{2444}
	\item \exercise{2445}
\end{listofex}
\newpage
\title{Занятие №4}
\begin{listofex}
	\item \exercise{2438}
	\item \exercise{2441}
	\item \exercise{2443}
	\item \exercise{2460}
	\item \exercise{2468}
\end{listofex}
\newpage
\title{Домашняя работа №2}
\begin{listofex}
	\item \exercise{1522}
	\item \exercise{1317}
	\item \exercise{2436}
	\item \exercise{2440}
	\item \exercise{2440}
	\item \exercise{2454}
	\item \exercise{2457}
	\item \exercise{2459}
\end{listofex}
%\newpage
%\title{Занятие №5}
%\begin{listofex}
%
%\end{listofex}
%\newpage
%\title{Занятие №6}
%\begin{listofex}
%
%\end{listofex}
%\newpage
%\title{Занятие №7}
%\begin{listofex}
%
%\end{listofex}
%\newpage
%\title{Проверочная работа}
%\begin{listofex}
%
%\end{listofex}