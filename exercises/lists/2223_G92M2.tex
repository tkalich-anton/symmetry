%Группа 9-1 Модуль 2
\title{Занятие №1}
\begin{listofex}
	\item \exercise{2350}
	\item \exercise{2095}
	\item \exercise{2390}
	\item \exercise{2381}
	\item \exercise{2380}
	\item \exercise{2393}
	\item \exercise{2559}
	\item \exercise{2040}
\end{listofex}
\newpage
\title{Занятие №2}
\begin{listofex}
	\item \exercise{1115}
	\item \exercise{2365}
	\item \exercise{2562}
	\item \exercise{2385}
	\item \exercise{2685}
	\item \exercise{2391}
	\item \exercise{2396}
	\item \exercise{2424}
	\item \exercise{2369}
\end{listofex}
\newpage
\title{Домашняя работа №1}
\begin{listofex}
	\item \exercise{1484}
	\item \exercise{1114}
	\item \exercise{4140}
	\item \exercise{2408}
	\item \exercise{2389}
	\item \exercise{2401}
	\item \exercise{2366}
\end{listofex}
\newpage
\title{Занятие №3}
\begin{listofex}
	\item Докажите следующие свойства окружности:
	\begin{enumcols}[itemcolumns=1]
		\item диаметр, перпендикулярный хорде, делит ее пополам;
		\item диаметр, проходящий через середину хорды, не являющейся диаметром, перпендикулярен этой хорде;
		\item хорды, удаленные от центра окружности на равные расстояния, равны.
	\end{enumcols}
	\item \exercise{2437}
	\item \exercise{2439}(без использования свойств центральных и вписанных углов)
	\item \exercise{2442}
	\item \exercise{2444}
	\item \exercise{2445}
	\item Докажите, что если треугольник вписан в окружность и одна из его сторон является диаметром этой окружности, то такой треугольник является прямоугольным.
	\item Центр окружности, описанной около треугольника, симметричен центру окружности, вписанной в этот треугольник, относительно одной из сторон. Найдите углы треугольника.
	\item Через точку \( A \) проведена прямая, пересекающая
	окружность с диаметром \( AB \) в точке \( K \), отличной от \( A \), а
	окружность с центром \( B \) --- в точках \( M \) и \( N \). Докажите, что \( MK = KN \).
\end{listofex}
\newpage
\title{Занятие №4}
\begin{listofex}
	\item \exercise{2438}
	\item \exercise{2441}
	\item \exercise{2443}
	\item \exercise{2514}
	\item \exercise{2460}
	\item \exercise{2468}
	\item \exercise{2453}
	\item Окружность, построенная на биссектрисе \( AD \) треугольника \( ABC \) как на диаметре, пересекает стороны \( AB \) и \( AC \) соответственно в точках \( M \) и \( N \), отличных от \( A \). Докажите, что \( AM = AN \).
\end{listofex}
\newpage
\title{Домашняя работа №2}
\begin{listofex}
	\item \exercise{1522}
	\item \exercise{1317}
	\item \exercise{2436}
	\item \exercise{2440}
	\item \exercise{2454}
	\item \exercise{2457}
	\item \exercise{2459}
\end{listofex}
\newpage
\title{Занятие №5}
\begin{listofex}
	\item \exercise{2472}
	\item \exercise{2473}
	\item \exercise{2474}
	\item \exercise{2483}
	\item \exercise{2493}
	\item \exercise{2508}
	\item \exercise{1608}
	\item \exercise{3664}
\end{listofex}
\newpage
\title{Занятие №6}
\begin{listofex}
	\item \exercise{2477}
	\item \exercise{2481}
	\item \exercise{2484}
	\item \exercise{2486}
	\item \exercise{2479}
	\item \exercise{2500}
	\item \exercise{2502}
	\item \exercise{2506}
\end{listofex}
\newpage
\title{Домашняя работа №3}
\begin{listofex}
	\item \exercise{2475}
	\item \exercise{2476}
	\item \exercise{2478}
	\item \exercise{2485}
	\item \exercise{2502}
	\item \exercise{2480}
	\item \exercise{2501}
\end{listofex}
\newpage
\title{Подготовка к проверочной работе}
\begin{listofex}
	\item Чему равен угол между биссектрисами двух смежных углов?
	\item Чему равен угол между биссектрисами двух внутренних односторонних углов при параллельных прямых? Докажите это.
	\item Сформулируйте и докажите теорему о внешнем угле треугольника.
	\item Чему равна сумма всех внешних углов треугольника?
	\item Докажите, что биссектриса внешнего угла при вершине равнобедренного треугольника, параллельна основанию.
	\item Докажите, что если в треугольнике один угол равен сумме двух других, то такое треугольник прямоугольный.
	\item Докажите, что если медиана равна половине стороны, к которой она проведена, то такой треугольник прямоугольный.
	\item Докажите, что если треугольник вписан в окружность и одна из его сторон является диаметром этой окружности, то такой треугольник прямоугольный.
	\item Сформулируйте теорему об угле в \( 30\degree \) в прямоугольном треугольнике. Сформулируйте обратную теорему.
	\item Сформулируйте теорему о диаметре, перпендикулярном хорде.
	\item Сформулируйте теорему о диаметре, проходящем через середину хорды.
	\item Где лежит центр вписанной в треугольник окружности? Где лежит центр описанной окружности?
	\item Сформулируйте теорему о двух касательных, проведенных из одной точки к окружности.
	\item \exercise{2472}
	\item Угол между биссектрисами двух углов треугольника равен \( 120\degree \). Чему равен третий угол треугольника?
	\item Угол треугольника равен \( 50\degree \). Найдите угол между высотами, проведенными из двух других углов.
	\item В треугольнике \( ABC \) угол \( \angle B=60\degree \). Найдите угол между биссектрисами двух других внешних углов.
	\item \exercise{2456}
	\item \exercise{2459}
	\item \exercise{2475}
	\item \exercise{2478}
	\item \exercise{2485}
	\item \exercise{2480}
	\item \exercise{2464}
	\item В треугольнике \( ABC \) медиана \( AM \) продолжена за точку \( M \) на расстояние, равное \( AM \). Найдите расстояние от полученной точки до вершин \( B  \) и \( C\), если \( AB = 5\), \( AC = 12\).
	\item \exercise{2441}
	\item \exercise{2514}
	\item \exercise{2470}
\end{listofex}
\newpage
\title{Проверочная работа}
\title{Вариант 1}
\begin{listofex}
	\item
	\begin{enumcols}[itemcolumns=1]
		\item Чему равен угол между биссектрисами двух смежных углов?
		\item Сформулируйте и докажите теорему о внешнем угле треугольника.
		\item Докажите, что биссектриса внешнего угла при вершине равнобедренного треугольника, параллельна основанию.
		\item Докажите, что если медиана равна половине стороны, к которой она проведена, то такой треугольник прямоугольный.
		\item Докажите, что если треугольник вписан в окружность и одна из его сторон является диаметром этой окружности, то такой треугольник прямоугольный.
		\item Сформулируйте теорему об угле в \( 30\degree \) в прямоугольном треугольнике. Сформулируйте обратную теорему.
		\item Сформулируйте теорему о диаметре, проходящем через середину хорды.
		\item Где лежит центр вписанной в треугольник окружности?
	\end{enumcols}
	\item В треугольнике \( ABC \) обе стороны \( AB \) и \( BC \) равны \( 15 \). Чему равна сторона \( AC \), если \( \angle BAC = 60 \degree \)?
	\item Угол между биссектрисами двух углов треугольника равен \( 100\degree \). Чему равен третий угол треугольника?
	\item \exercise{2456}
	\item \exercise{2478}
	\item \exercise{2485}
	\item В треугольнике \( ABC \) медиана \( AM \) продолжена за точку \( M \) на расстояние, равное \( AM \). Найдите расстояние от полученной точки до вершин \( B  \) и \( C\), если \( AB = 5\), \( AC = 12\).
	\item \exercise{2441}
	\item \exercise{37}
	\item Найти значение выражения \( 61a-11b+67 \), если \( \dfrac{2a-7b+5}{7a-2b+5}=9 \)
\end{listofex}
\newpage
\title{Проверочная работа}
\title{Вариант 2}
\begin{listofex}
	\item
	\begin{enumcols}[itemcolumns=1]
		\item Чему равен угол между биссектрисами двух внутренних односторонних углов при параллельных прямых?
		\item Сформулируйте и докажите теорему о внешнем угле треугольника.
		\item Докажите, что если в треугольнике один угол равен сумме двух других, то такое треугольник прямоугольный.
		\item Докажите, что если треугольник вписан в окружность и одна из его сторон является диаметром этой окружности, то такой треугольник прямоугольный.
		\item Сформулируйте теорему об угле в \( 30\degree \) в прямоугольном треугольнике. Сформулируйте обратную теорему.
		\item Сформулируйте теорему о диаметре, перпендикулярном хорде.
		\item Сформулируйте теорему о двух касательных, проведенных из одной точки к окружности.
	\end{enumcols}
	\item В треугольнике \( ABC \) обе стороны \( AB \) и \( BC \) равны \( 30 \). Чему равна сторона \( AC \), если \( \angle BAC = 60 \degree \)?
	\item Угол треугольника равен \( 80\degree \). Найдите угол между высотами, проведенными из двух других углов.
	\item \exercise{2456}
	\item \exercise{2478}
	\item \exercise{2485}
	\item \exercise{2441}
	\item В треугольнике \( ABC \) медиана \( AM \) продолжена за точку \( M \) на расстояние, равное \( AM \). Найдите расстояние от полученной точки до вершин \( B  \) и \( C\), если \( AB = 6\), \( AC = 17\).
	\item \exercise{638}
	\item Найти значение выражения \( 61a-11b+78 \), если \( \dfrac{2a-7b+5}{7a-2b+5}=9 \)
	
\end{listofex}