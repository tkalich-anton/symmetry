%Математиечкие встречи для преподавателй и кураторов

\title{Встреча №1}
\begin{listofex}
	\item Точка \( B \) лежит на отрезке \( AC \) длиной \( 5 \). Найдите расстояние между серединами отрезков \( AB \) и \( BC \).
	\item Найдите угол между биссектрисами двух смежных углов.
	\item \exercise{2387}
	\item \exercise{2347}
	\item Через данную точку проведите прямую, пересекающую две данные прямые под равными углами.
	\item \exercise{2358}
	\item Биссектрисы \( BB_1 \) и \( CC_1 \) треугольника \( ABC \) пересекаются в точке \( M \), биссектрисы \( B_1B_2 \) и \( C_1C_2 \) треугольника \( AB_1C_1 \) пересекаются в точке \( N \). Докажите, что точки A, M и N лежат на одной прямой.
	\item Докажите, что биссектриса внешнего угла при вершине равнобедренного треугольника параллельна основанию.
	\item Постройте равнобедренный треугольник, если даны две прямые, на которых лежат биссектрисы его углов при вершине и при основании, и по точке на каждой из боковых сторон.
	\item Постройте треугольник, если заданы сторона, противолежащий ей угол и сумма двух других сторон.
	
	\item \exercise{2402}
	\item \exercise{2403}
	
	\item Угол треугольника равен сумме двух других его углов. Докажите, что треугольник прямоугольный.
	\item Докажите, что если медиана равна половине стороны, к которой она проведена, то этот треугольник прямоугольный. Доказать обратное.
	\item В треугольнике \( ABC \) угол \( \angle C = 30\degree \) и \( AC = 10 \). И вершины \( B \) провели медиану, которая равна \( 5 \). Найдите другие стороны треугольника.
	\item Докажите, что если треугольник вписан в окружность и одна из его сторон -- диаметр, то такой треугольник прямоугольный.
	\item Постройте равнобедренный треугольник по основанию и радиусу описанной окружности.
	\item Докажите, что центр окружности, описанной около прямоугольного треугольника, --- середина гипотенузы.
	
	\item Докажите, что, если в треугольнике один угол равен \( 120\degree \), то треугольник, образованный основаниями его биссектрис, прямоугольный.
	
	% ========= ДЗ№1 ============
	
	%\item Точка K отрезка AB, равного 12, расположена на 5 бли-
	%же к A, чем к B. Найдите AK и BK.
	
	%\item \exercise{2369}
	
	%\item Дана прямая l и точки A и B по разные стороны от нее.
	%Постройте на прямой l такую точку C, чтобы прямая l делила
	%угол ACB пополам.
	
	%\item Дана прямая l и точки A и B по одну сторону от нее.
	%Луч света, выпущенный из точки A, отразившись от этой пря-
	%мой в точке C, попадает в точку B. Постройте точку C.
	
	% \item Постройте биссектрису угла, вершина которого недоступна.
	
	%\item Внутри острого угла даны точки M и N. Как из точ-
	%ки M направить луч света, чтобы он, отразившись последова-
	%тельно от сторон угла, попал в точку N?
	
	%\item \exercise{2404}
	%\item \exercise{2412}
	%\item \exercise{2409}
	%\item \exercise{2420}
	
	%Постройте окружность данного радиуса, высекаю-
	%щую на данной прямой отрезок, равный данному.
	
	%Постройте треугольник, если дана одна его вершина
	%и две прямые, на которых лежат биссектрисы, проведенные из
	%двух других вершин.
	
	%1.41
\end{listofex}

\title{Домашняя работа №1}
\begin{listofex}
	
	\item Точка \( K \) отрезка \( AB \), равного \( 12 \), расположена на \( 5 \) ближе к \( A \), чем к \( B \). Найдите \( AK \) и \( BK \).
	
	\item \exercise{2369}
	
	\item Дана прямая \( l \) и точки \( A \) и \( B \) по разные стороны от нее.
	Постройте на прямой \( l \) такую точку \( C \), чтобы прямая \( l \) делила
	угол \( ACB \) пополам.
	
	\item Дана прямая \( l \) и точки \( A \) и \( B \) по одну сторону от нее.
	Луч света, выпущенный из точки \( A \), отразившись от этой прямой в точке \( C \), попадает в точку \( B \). Постройте точку \( C \).
	
	\item Постройте биссектрису угла, вершина которого недоступна.
	\item \exercise{2404}
	\item \exercise{2412}
	\item \exercise{2409}
	\item \exercise{2420}
	
	\item Постройте окружность данного радиуса, высекающую на данной прямой отрезок, равный данному.
	
	\item Постройте треугольник, если дана одна его вершина и две прямые, на которых лежат биссектрисы, проведенные из двух других вершин.
	
	\item Внутри острого угла даны точки \( M \) и \( N \). Как из точки \( M \) направить луч света, чтобы он, отразившись последовательно от сторон угла, попал в точку \( N \)?
	
	%1.41
\end{listofex}
