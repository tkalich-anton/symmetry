%Математиечкие встречи для преподавателй и кураторов

\title{Встреча 1}
\begin{listofex}
	\item Точка \( B \) лежит на отрезке \( AC \) длиной \( 5 \). Найдите расстояние между серединами отрезков \( AB \) и \( BC \).
	\item Найдите угол между биссектрисами двух смежных углов.
	\item \exercise{2387}
	\item \exercise{2347}
	\item Через данную точку проведите прямую, пересекающую две данные прямые под равными углами.
	\item \exercise{2358}
	\item Биссектрисы \( BB_1 \) и \( CC_1 \) треугольника \( ABC \) пересекаются в точке \( M \), биссектрисы \( B_1B_2 \) и \( C_1C_2 \) треугольника \( AB_1C_1 \) пересекаются в точке \( N \). Докажите, что точки A, M и N лежат на одной прямой.
	\item Докажите, что биссектриса внешнего угла при вершине равнобедренного треугольника параллельна основанию.
	\item Постройте равнобедренный треугольник, если даны две прямые, на которых лежат биссектрисы его углов при вершине и при основании, и по точке на каждой из боковых сторон.
	\item Постройте треугольник, если заданы сторона, противолежащий ей угол и сумма двух других сторон.
	
	\item \exercise{2402}
	\item \exercise{2403}
	
	\item Угол треугольника равен сумме двух других его углов. Докажите, что треугольник прямоугольный.
	\item Докажите, что если медиана равна половине стороны, к которой она проведена, то этот треугольник прямоугольный. Доказать обратное.
	\item В треугольнике \( ABC \) угол \( \angle C = 30\degree \) и \( AC = 10 \). И вершины \( B \) провели медиану, которая равна \( 5 \). Найдите другие стороны треугольника.
	\item Докажите, что если треугольник вписан в окружность и одна из его сторон -- диаметр, то такой треугольник прямоугольный.
	\item Постройте равнобедренный треугольник по основанию и радиусу описанной окружности.
	\item Докажите, что центр окружности, описанной около прямоугольного треугольника, --- середина гипотенузы.
	
	\item Докажите, что, если в треугольнике один угол равен \( 120\degree \), то треугольник, образованный основаниями его биссектрис, прямоугольный.
\end{listofex}
\newpage
\title{Домашняя работа 1}
\begin{listofex}
	
	\item Точка \( K \) отрезка \( AB \), равного \( 12 \), расположена на \( 5 \) ближе к \( A \), чем к \( B \). Найдите \( AK \) и \( BK \).
	
	\item \exercise{2369}
	
	\item Дана прямая \( l \) и точки \( A \) и \( B \) по разные стороны от нее.
	Постройте на прямой \( l \) такую точку \( C \), чтобы прямая \( l \) делила
	угол \( ACB \) пополам.
	
	\item Дана прямая \( l \) и точки \( A \) и \( B \) по одну сторону от нее.
	Луч света, выпущенный из точки \( A \), отразившись от этой прямой в точке \( C \), попадает в точку \( B \). Постройте точку \( C \).
	
	\item Постройте биссектрису угла, вершина которого недоступна.
	\item \exercise{2404}
	\item \exercise{2412}
	\item \exercise{2409}
	\item \exercise{2420}
	
	\item Постройте окружность данного радиуса, высекающую на данной прямой отрезок, равный данному.
	
	\item Постройте треугольник, если дана одна его вершина и две прямые, на которых лежат биссектрисы, проведенные из двух других вершин.
	
	\item Внутри острого угла даны точки \( M \) и \( N \). Как из точки \( M \) направить луч света, чтобы он, отразившись последовательно от сторон угла, попал в точку \( N \)?
	
	%1.41
\end{listofex}
\newpage
\title{Домашняя работа 2}
\begin{listofex}
	\item Точка \( A \) лежит вне данной окружности с центром \( O \).
	Окружность с диаметром \( OA \) пересекается с данной в точках \( B \)
	и \( C \). Докажите, что прямые \( AB \) и \( AC \) --- касательные к данной
	окружности.
	\item Постройте хорду данной окружности, равную и параллельную заданному отрезку.
	\item \( CH \) --- высота прямоугольного треугольника \( ABC \),
	проведенная из вершины прямого угла. Докажите, что сумма
	радиусов окружностей, вписанных в треугольники \( ACH \), \( BCH \)
	и \( ABC \), равна \( CH \).
	\item Окружность касается стороны \( BC \) треугольника
	\( ABC \) в точке \( M \), а продолжений сторон \( AB \) и \( AC \) ---
	в точках \( N \) и \( P \) соответственно. Вписанная в этот треугольник
	окружность касается стороны \( BC \) в точке \( K \), а стороны \( AB \) ---
	в точке \( L \). Докажите, что:\\
	\textit{а)} отрезок \( AN \) равен полупериметру треугольника \( ABC \);\\
	\textit{б)} \( BK = CM \);\\
	\textit{в)} \( NL = BC \).
	\item В острый угол, равный \( 60\degree \), вписаны две окружности, касающиеся друг друга внешним образом. Радиус меньшей окружности равен \( r \). Найдите радиус большей окружности.
	\item Угловые величины дуг, заключенных между двумя хордами, продолжения которых
	пересекаются вне круга, равны \( \alpha \) и \( \beta\;(\alpha>\beta)\). Под каким углом пересекаются продолжения хорд?
	\item Треугольник с вершинами в основаниях высот треугольника \( ABC \) называется ортотреугольником треугольника \( ABC \). Докажите, что высоты остроугольного треугольника \( ABC \) являются биссектрисами его ортотреугольника.
	\item Две окружности пересекаются в точках \( A \) и \( B \). Через
	точку \( B \) проводится прямая, пересекающая окружности в точках \( C \) и \( D \), а затем через точки \( C \) и \( D \) проводятся касательные
	к этим окружностям. Докажите, что точки \( A \), \( C \), \( D \) и точка \( P \)
	пересечения касательных лежат на одной окружности.
	\item Решить уравнения:
	\begin{enumcols}[itemcolumns=2]
		\item \( 6x^4+7x^3-36x^2-7x+6=0 \)
		\item \( 5\sqrt{12-x}+|4x-3|=5x+|4\sqrt{12-x}-3| \)
		\item \( |x^2-x-5|+|x^2-x-9|=10  \)
		\item \( 5\sin x + 2\cos x = 0 \)
	\end{enumcols}
\end{listofex}
\newpage
\cheadbf{Домашняя работа 3}
\begin{listofex}
	\item Точка \( M \) --- середина ребра \( AB \) треугольной призмы \( ABCA_1B_1C_1 \).\\
	а) Постройте сечение призмы плоскостью, проходящей через прямую \( A_1M \) параллельно прямой \( AC \).\\
	б) В каком отношении плоскость сечения делит отрезок, соединяющий точку \( B_1 \) с серединой ребра \( AC \)?
	\item Основание шестиугольной пирамиды \( SABCDEF \) --- правильный шестиугольник \( ABCDEF \). Постройте сечение пирамиды плоскостью, проходящей через следующие точки:\\
	а) центр основания параллельно плоскости \( ASB \);\\
	б) \( B \), \( C \) и середину отрезка, соединяющего вершину пирамиды с центром основания.
	\item В треугольнике \( ABC \) известно, что \( AB = 12 \), \( AC = 15 \),
	\( BC = 18 \). Найдите биссектрису треугольника, проведенную из вершины наибольшего угла.
	\item Три окружности равных радиусов проходят через точку \( M \) и попарно пересекаются в трех других точках \( A \), \( B \) и \( C \). Докажите, что точки \( A \), \( B \) и \( C \) лежат на окружности того же радиуса, а M --- точка пересечения высот треугольника \( ABC \).
\end{listofex}
%\newpage
%\cheadbf{Дополнительные уроки}
%\begin{listofex}
%	\item При всех значениях параметра \( a \) решить уравнение: \[ (2a-4)x=3a+1 \]
%	\item При всех значениях параметра \( a \) решить уравнение: \[ a^2x-5a=9x-15 \]
%	\item При каких \( a \) уравнение: \[ (a^6+12a-40)x=a^2+a-2 \] имеет бесконечное число решений.
%	\item При всех значениях параметра \( a \) решить неравенство: \[ (a^2-a)x<3-3a \]
%	\item При каких \( a \) неравенство \( ax+2-\dfrac{a}{3}>0 \) выполняется при всех \( x\in(1;2) \)?
%	\item При каких \( a \) уравнение: \[ x^2+2(a+1)x+9a-5=0 \] имеет один корень; два корня; ни одного корня?
%	\item Найдите все значения параметра \( a \), при которых уравнение \[ (2a+3)x^2-2ax+a-2=0 \] имеет два решения.
%	\item Найдите все значения параметра \( a \), при которых уравнение: \[ (a+1)x^2-ax+a-3=0 \] имеет не более одного корня.
%	\item При всех значениях параметра \( a \) определить число решений уравнения: \[ |2x+8|+|2x-6|=a \]
%\end{listofex}
