%Группа 103 Модуль 1 Занятие №4
\setlength\fboxrule{1.2pt}
\begin{listofex}
	\item Найти значение выражения:
	\begin{enumcols}[itemcolumns=3]
		\item \( (\sqrt{63}-\sqrt{28}) \cdot \sqrt{7} \)
		\item \( \dfrac{(8\sqrt{3})^2}{8} \)
		\item \( (\sqrt{10}-\sqrt{12})(\sqrt{10}+\sqrt{12}) \)
		\item \( \dfrac{(\sqrt{3}+\sqrt{11})^2}{7+\sqrt{33}} \)
		\item \( \dfrac{4\sqrt{7}+5\sqrt{7}}{\sqrt{63}} \)
		\item \( \dfrac{\sqrt{2,8} \cdot \sqrt{4,2}}{\sqrt{0,24}} \)
		\item \( \left( \sqrt{62\dfrac{1}{2}}-\sqrt{22\dfrac{1}{2}} \right)\cdot\sqrt{\dfrac{5}{8}} \)
	\end{enumcols}
	\item Найти значение выражения:
	\begin{enumcols}[itemcolumns=2]
		\item \( \dfrac{1-\sqrt{10}}{\sqrt{2}+\sqrt{5}}-(11-5\sqrt{5})(2+\sqrt{5}) \)
		\item \( \dfrac{5\sqrt{x}+2}{\sqrt{x}}-\dfrac{2\sqrt{x}}{x} \)
	\end{enumcols}
	\item Найти значение выражения:
	\begin{enumcols}[itemcolumns=1]
		\item \exercise{1338}
		\item \exercise{1339}
	\end{enumcols}
	\item \exercise{1326}
	\item Найти значение выражения:
	\begin{enumcols}[itemcolumns=2]
		\item \exercise{1328}
		\item \exercise{1333}
		\item \exercise{1337}
		\item \exercise{1334}
	\end{enumcols}
	\item \exercise{1330}
	\item Найти значение выражения:
	
	\textit{Пример:} \[ \sqrt{11-4\sqrt{7}}=\sqrt{4+7-2\cdot2\cdot\sqrt{7}}=\sqrt{\displaystyle2\mathstrut^2+\sqrt{7}\mathstrut^2-2\sqrt{2\cdot7}}=\sqrt{\displaystyle(2-\sqrt{7})^2}=\left|2-\sqrt{7}\right|=\sqrt{7} - 2 \]
	\begin{enumcols}[itemcolumns=2]
		\item \exercise{1332}
		\item \exercise{1331}
	\end{enumcols}
\end{listofex}