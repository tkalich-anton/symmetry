%Группа 9-2 Модуль 1 Занятие №1
\begin{listofex}
	\item Упростить выражение:
	\begin{enumcols}[itemcolumns=2]
		\item \exercise{1402}
		%\item \exercise{1420} % Занятие №2
		%\item \exercise{1431} % ДЗ №1
		\item \exercise{1379}
		%\item \exercise{1463} % Занятие №2
		%\item \exercise{1471} % ДЗ №1
	\end{enumcols}
	
	\item Упростить выражение \( \dfrac{p\cdot q}{p+q}\cdot\left( \dfrac{q}{p}-\dfrac{p}{q} \right) \) и найдите значение выражения, если \( p=3-2\sqrt{2} \) и \( q=-2\sqrt{2} \)
	
	\item Найти значение выражения \( x^2+\dfrac{1}{x^2} \), если известно, что \( x-\dfrac{1}{x}=5 \)
	%\item Найти значение выражения \( 4x^2+\dfrac{1}{x^2} \), если известно, что \( 2x+\dfrac{1}{x}=7 \) % Занятие №2
	%\item Найти значение выражения \( 25x^2+\dfrac{1}{x^2} \), если известно, что \( 5x+\dfrac{1}{x}=4 \) % ДЗ №1
	
	\item Из формулы \( \dfrac{1}{F}=\dfrac{1}{f}+\dfrac{1}{d} \) выразить: а) \( F \); б) \( d \)
	%\item Из формулы \( S_n=\dfrac{2a_1+d(n+1)}{2}\cdot n \) выразить: а) \( a_1 \); б) \( d \) % Занятие №2
	%\item Из формулы \( S=\dfrac{abc}{4R} \) выразить: а) \( c \); б) \( R \) % ДЗ №1
	
	\item Из формулы \( F=\gamma\cdot\dfrac{m_1m_2}{r^2} \) выразить \( r \). Все величины положительны.
	%\item Из формулы \( P=\dfrac{U^2}{R} \) выразить \( U \). Все величины положительны. % Занятие №2
	%\item Из формулы \( Q=I^2Rt \) выразить \( I \). Все величины положительны. % ДЗ №1
	
	\item Вычислить:
	\begin{enumcols}[itemcolumns=3]
		\item \( \sqrt{77\cdot24\cdot33\cdot14} \)
		%\item \( \sqrt{5\cdot6\cdot8\cdot20\cdot27} \) % Занятие №2
		%\item \( \sqrt{21\cdot65\cdot39\cdot35} \) % ДЗ №1
		\item \( \sqrt{21}\cdot\sqrt{3\dfrac{6}{7}} \)
		%\item \( \sqrt{15}\cdot\sqrt{6\dfrac{2}{3}} \) % Занятие №2
		%\item \( \sqrt{12}\cdot\sqrt{5\dfrac{1}{3}} \) % ДЗ №1
		\item \( \dfrac{(3\sqrt{5})^2}{15} \)
		%\item \( \dfrac{6}{(2\sqrt{3})^2} \) % Занятие №2
		%\item \( \dfrac{(5\sqrt{7})^2}{35} \) % ДЗ №1
	\end{enumcols}
	
	\item Расположить числа в порядке возрастания:\quad\( 4;\;3,8;\;\sqrt{15};\;\sqrt{5};\;4,3 \)
	%\item Расположить числа в порядке возрастания:\quad\( 5;\;\sqrt{26};\;,7;\;\sqrt{6};\;,1 \) % Занятие №2
	%\item Расположить числа в порядке возрастания:\quad\( 7;\;\sqrt{46};\;6,8;\;5\sqrt{2};\;7,2 \) % ДЗ №1
	
	\item Найти значение выражения \( 3x^2-2x-1 \), если \( x=\dfrac{1-\sqrt{2}}{3} \)
	%\item Найти значение выражения \( 2x^2-6x+3 \), если \( x=\dfrac{3-\sqrt{5}}{2} \) % Занятие №2
	%\item Найти значение выражения \( a^2-6\sqrt{5}-1 \), если \( a=\sqrt{5}+4 \) % ДЗ №1
	
	\item Упростить выражение:
	\begin{enumcols}[itemcolumns=2]
		\item \( \dfrac{a}{a-1}-\dfrac{\sqrt{a}}{\sqrt{a}+1} \)
		%\item \( \dfrac{c}{c-4}-\dfrac{\sqrt{c}}{\sqrt{c}-2} \) % Занятие №2
		%\item \( \dfrac{x}{x-16}-\dfrac{\sqrt{x}}{\sqrt{x}+4} \) % ДЗ №1
		\item \( \left( \dfrac{\sqrt{a}-5}{\sqrt{a}+5}+\dfrac{20\sqrt{a}}{a-25} \right):\dfrac{\sqrt{a}+5}{a-5\sqrt{a}} \)
		%\item \( \left( \dfrac{\sqrt{y}+7}{\sqrt{y}-7}-\dfrac{28\sqrt{y}}{y-49} \right):\dfrac{\sqrt{y}-7}{y+7\sqrt{y}} \) % Занятие №2
		%\item \( \left( \dfrac{\sqrt{m}-2}{\sqrt{m}+2}+\dfrac{8\sqrt{m}}{m-4} \right):\dfrac{\sqrt{m}+2}{m-2\sqrt{m}} \) % ДЗ №1
	\end{enumcols}

	\item Известно, что \( \sqrt{8-x}+\sqrt{x+3}=4 \). Найдите значение выражения \( \sqrt{(8-x)(x+3)} \)
	%\item Известно, что \( \sqrt{y-1}+\sqrt{8-y}=2 \). Найдите значение выражения \( \sqrt{(y-1)(8-y)} \) % Занятие №2
	%\item Известно, что \( \sqrt{7-x}+\sqrt{x-2}=3 \). Найдите значение выражения \( \sqrt{(7-x)(x-2)} \) % ДЗ №1
	
	\item Найдите три последовательных натуральных числа, если удвоенный квадрат
	первого из них на \( 26 \) больше произведения второго и третьего чисел.
\end{listofex}