%Группа 6-1 Модуль 1
%\title{Занятие №1}
%\begin{listofex}
%	
%\end{listofex}
%\newpage
%\title{Занятие №2}
%\begin{listofex}
%	\item 1
%\end{listofex}
%\newpage
%\title{Домашняя работа №1}
%\begin{listofex}
%	\item 1
%\end{listofex}
%\newpage
%
\title{Занятие №3}
\begin{listofex}
		\item Упростить дробь:
	\begin{enumcols}[itemcolumns=5]
		\item \( \dfrac{9}{36} \)
		\item \( \dfrac{45}{54} \)
		\item \( \dfrac{12}{28} \)
		\item \( \dfrac{144}{120} \)
		\item \( \dfrac{72}{24} \)
	\end{enumcols}
	\item Представить обыкновенную дробь в виде десятичной делением числителя на знаменатель в столбик:
	\begin{enumcols}[itemcolumns=5]
		\item \( \dfrac{7}{5} \)
		\item \( \dfrac{3}{16} \)
		\item \( \dfrac{28}{140} \)
		\item \( \dfrac{17}{200} \)
		\item \( \dfrac{324}{25} \)
	\end{enumcols}
	\item Вычислить:
	\begin{enumcols}[itemcolumns=4]
		\item \( \dfrac{2}{26}+\dfrac{3}{39} \)
		\item \( \dfrac{7}{12}-\dfrac{1}{3} \)
		\item \( \dfrac{3}{20}+\dfrac{7}{30}+\dfrac{2}{40} \)
		\item \( \dfrac{31}{80}+\left( \dfrac{3}{16}+\dfrac{39}{80} \right) \)
	\end{enumcols}
	\item Вычислить:
	\begin{enumcols}[itemcolumns=2]
		\item \( \left( \dfrac{1}{2}:\dfrac{3}{4}-\dfrac{4}{9} \right):\dfrac{3}{5} \)
		\item \( \dfrac{3}{2}\cdot\dfrac{5}{6}+\dfrac{3}{2}:\dfrac{9}{10}-\dfrac{3}{2}\cdot\dfrac{13}{18} \)
	\end{enumcols}
	\item Вычислить:\qquad\( \dfrac{\cfrac{3}{20}\cdot\left( \cfrac{7}{12}-\cfrac{1}{2} \right)+\cfrac{79}{80}}{\cfrac{13}{24}:\left( \cfrac{7}{12}+\cfrac{1}{2} \right)-\cfrac{1}{4}} \)
	\item Вычислить:
	\begin{enumcols}[itemcolumns=3]
		\item \( (12-27)\cdot(-1) \)
		\item \( (-25)\cdot(45-100)+25\cdot45 \)
		\item \( 8\cdot(-8+100-22+45) \)
	\end{enumcols}
	\item Вычислить удобным способом: \( 392\cdot23-492\cdot23+392\cdot77-492\cdot77 \)
	\item Сократить дробь:
	\begin{enumcols}[itemcolumns=3]
		\item \( \dfrac{36\cdot(-112)}{126\cdot(-63)} \)
		\item \( \dfrac{-3\cdot8\cdot(-6)}{18\cdot(-4)}\)
		\item \( \dfrac{-128\cdot(-92)}{-256\cdot(-48)} \)
	\end{enumcols}
	\item Вычислить:
	\begin{enumcols}[itemcolumns=2]
		\item \( \dfrac{28}{63}:\left( -\dfrac{9}{7} \right) \)
		\item \( -3\dfrac{8}{19}+\left( -1\dfrac{11}{19} \right) \)
		\item \( \left( -1\dfrac{1}{3} \right)\cdot\dfrac{9}{10} \)
		\item \( 7\dfrac{2}{9}\cdot8\dfrac{2}{3}-7\dfrac{2}{9}\cdot6\dfrac{2}{3} \)
		\item \( 7\dfrac{1}{2}\cdot\left( -\dfrac{1}{5} \right)+\left( -1\dfrac{2}{3} \right)\cdot\left( -\dfrac{9}{10} \right)-17\dfrac{29}{30} \)
		\item \( 12,8\cdot\dfrac{1}{4}:\left( \dfrac{3}{4}-0,125 \right) \)
	\end{enumcols}
\end{listofex}
\newpage
\title{Занятие №4}
\begin{listofex}
	\item 1
\end{listofex}
\newpage
\title{Домашняя работа №2}
\begin{listofex}
	\item Вычислить:
	\begin{enumcols}[itemcolumns=3]
		\item \exercise{4199}
		\item \exercise{4204}
		\item \exercise{4207}
	\end{enumcols}
\end{listofex}
\newpage
\title{Занятие №5}
\begin{listofex}
	\item Вычислить:
	\begin{enumcols}[itemcolumns=1]
		\item \exercise{4189}
	\end{enumcols}
	\begin{enumcols}[itemcolumns=2, start=2]
		\item \exercise{4202}
		\item \exercise{4208}
	\end{enumcols}
	\item Вычислить значение выражения \( 4ab+a^2-3b \) при \( a=2\dfrac{1}{3} \) и \( b=1,2 \).
	\item Вычислить значение выражения \( a^2+2ab+b^2 \) при \( a=2\dfrac{1}{7} \) и \( b=\dfrac{6}{7} \).
	\item Решить пропорцию:
	\begin{enumcols}[itemcolumns=3]
		\item \( x:5=16:0,8 \)
		\item \( 0,2:x=\dfrac{1}{2}:2\dfrac{1}{2} \)
		\item \( \dfrac{720}{91,2}=\dfrac{c}{0,513} \)
		\item \( \dfrac{8,75}{3\dfrac{3}{4}}=\dfrac{x}{0,75} \)
	\end{enumcols}
	\item Решить пропорцию:
	\begin{enumcols}[itemcolumns=2]
		\item \( \dfrac{x}{1,7}=\dfrac{3,5\cdot6,3}{5,1\cdot0,21} \)
		\item \( \dfrac{0,74\cdot4,5}{0,03\cdot7,5}=\dfrac{3,7\cdot2,4}{x} \)
	\end{enumcols}
	\item Решить уравнение:
	\begin{enumcols}[itemcolumns=3]
		\item \exercise{289}
		\item \exercise{346}
		\item \exercise{358}
	\end{enumcols}
\end{listofex}
\newpage
%\title{Занятие №6}
%\begin{listofex}
%	\item 1
%	
%\end{listofex}
%\newpage
%\title{Домашняя работа №3}
%\begin{listofex}
%	\item 1
%	
%\end{listofex}
%\newpage
%\title{Занятие №7}
%\begin{listofex}
%	\item 1
%	
%\end{listofex}
%\newpage
%\title{Занятие №8}
%\begin{listofex}
%	\item 1
%	
%\end{listofex}
%\newpage
%\title{Домашняя работа №4}
%\begin{listofex}
%	\item 1
%	
%\end{listofex}
%\newpage
%\title{Проверочная работа}
%\begin{listofex}
%	\item 1
%	
%\end{listofex}