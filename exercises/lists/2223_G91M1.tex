%Группа 9-1 Модуль 1 Занятие №1
\title{Занятие №1}
\begin{listofex}
	\item Упростить выражение:
	\begin{enumcols}[itemcolumns=2]
		\item \exercise{117}
		\item \exercise{120}
		\item \exercise{126}
		\item \exercise{853}
		\item \exercise{1363}
		\item \exercise{1368}
	\end{enumcols}
	
	\item \exercise{1379}
	\item \exercise{1402}
	\item Найдите значение выражения \( x^2+\dfrac{1}{x^2} \), если известно, что \( x-\dfrac{1}{x}=5 \)
	\item Из формулы \( \dfrac{1}{F}=\dfrac{1}{f}+\dfrac{1}{d} \) выразите: а) \( F \); б) \( d \)
	\item Из формулы \( F=\gamma\cdot\dfrac{m_1m_2}{r^2} \) выразите \( r \). Все величины положительны.
	\item Вычислить:
	\begin{enumcols}[itemcolumns=3]
		\item \( \sqrt{77\cdot24\cdot33\cdot14} \)
		\item \( \sqrt{21}\cdot\sqrt{3\dfrac{6}{7}} \)
		\item \( \dfrac{(3\sqrt{5})^2}{15} \)
	\end{enumcols}
	\item Расположите числа в порядке возрастания:\quad\( 4;\;3,8;\;\sqrt{15};\;\sqrt{5};\;4,3 \)
	\item Найдите значение выражения \( 3x^2-2x-1 \), если \( x=\dfrac{1-\sqrt{2}}{3} \)
	
\end{listofex}
\newpage
\title{Занятие №2}
\[ \begin{array}{cccc}
	\text{Разность квадратов}&(a+b)(a-b)& =&a^2-b^2,\\
	\text{Квадрат суммы}&(a+b)^2& =&a^2+2ab+b^2,\\
	\text{Квадрат разности}&(a-b)^2& =&a^2-2ab+b^2,\\
	\text{Сумма кубов}&(a+b)(a^2-ab+b^2)& =&a^3+b^3,\\
	\text{Разность кубов}&(a-b)(a^2+ab+b^2)& =&a^3-b^3,\\
	\text{Куб суммы}&(a+b)^3& =&a^3+3a^2b+3ab^2+b^3,\\
	\text{Куб разности}&(a-b)^3& =&a^3-3a^2b+3ab^2-b^3.\\
\end{array} \]
\begin{listofex}
	\item Упростить выражение:
	\begin{enumcols}[itemcolumns=2]
		\item \exercise{116}
		\item \exercise{121}
		\item \exercise{1352}
		\item \exercise{1351}
		\item \exercise{1364}
		\item \exercise{1387}
	\end{enumcols}
	\item \exercise{1463}
	\item \exercise{1420}
	\item Найдите значение выражения \( 4x^2+\dfrac{1}{x^2} \), если известно, что \( 2x+\dfrac{1}{x}=7 \)
	\item Из формулы \( S_n=\dfrac{2a_1+d(n+1)}{2}\cdot n \) выразите: а) \( a_1 \); б) \( d \)
	\item Из формулы \( P=\dfrac{U^2}{R} \) выразите \( U \). Все величины положительны.
	
	\item Вычислить:
	\begin{enumcols}[itemcolumns=3]
		\item \( \sqrt{5\cdot6\cdot8\cdot20\cdot27} \)
		\item \( \sqrt{15}\cdot\sqrt{6\dfrac{2}{3}} \)
		\item \( \dfrac{6}{(2\sqrt{3})^2} \)
	\end{enumcols}
	\item Расположите числа в порядке возрастания:\quad\( 5;\;\sqrt{26};\;,7;\;\sqrt{6};\;,1 \)
	\item Найдите значение выражения \( 2x^2-6x+3 \), если \( x=\dfrac{3-\sqrt{5}}{2} \)
\end{listofex}
\newpage
\title{Домашняя работа №1}
\[ \begin{array}{cccc}
	\text{Разность квадратов}&(a+b)(a-b)& =&a^2-b^2,\\
	\text{Квадрат суммы}&(a+b)^2& =&a^2+2ab+b^2,\\
	\text{Квадрат разности}&(a-b)^2& =&a^2-2ab+b^2,\\
	\text{Сумма кубов}&(a+b)(a^2-ab+b^2)& =&a^3+b^3,\\
	\text{Разность кубов}&(a-b)(a^2+ab+b^2)& =&a^3-b^3,\\
	\text{Куб суммы}&(a+b)^3& =&a^3+3a^2b+3ab^2+b^3,\\
	\text{Куб разности}&(a-b)^3& =&a^3-3a^2b+3ab^2-b^3.\\
\end{array} \]
\begin{listofex}
	\item Упростить выражение:
	\begin{enumcols}[itemcolumns=2]
		\item \exercise{122}
		\item \exercise{123}
		\item \exercise{124}
		\item \exercise{127}
		\item \exercise{855}
		\item \exercise{1362}
	\end{enumcols}
	\item \exercise{1471}
	\item \exercise{1431}
	\item Найдите значение выражения \( 25x^2+\dfrac{1}{x^2} \), если известно, что \( 5x+\dfrac{1}{x}=4 \)
	\item Из формулы \( S=\dfrac{abc}{4R} \) выразите: а) \( c \); б) \( R \)
	\item Из формулы \( Q=I^2Rt \) выразите \( I \). Все величины положительны.
	\item Вычислить:
	\begin{enumcols}[itemcolumns=3]
		\item \( \sqrt{21\cdot65\cdot39\cdot35} \)
		\item \( \sqrt{12}\cdot\sqrt{5\dfrac{1}{3}} \)
		\item \( \dfrac{(5\sqrt{7})^2}{35} \)
	\end{enumcols}
	\item Расположите числа в порядке возрастания:\quad\( 7;\;\sqrt{46};\;6,8;\;5\sqrt{2};\;7,2 \)
	\item Найдите значение выражения \( a^2-6\sqrt{5}-1 \), если \( a=\sqrt{5}+4 \)
\end{listofex}
\newpage

\title{Занятие №3}
\begin{listofex}
	\item Упростить выражение:
	\begin{enumcols}[itemcolumns=1]
		\item \exercise{1456}
		\item \exercise{1453}
	\end{enumcols}
	\item Вычислить:
	\begin{enumcols}[itemcolumns=3]
		\item \( \dfrac{24^3}{18^4} \)
		\item \( \dfrac{32^7}{16^4\cdot64^4} \)
		\item \( \dfrac{100^8}{2^{}\cdot5^{14}} \)
	\end{enumcols}
	\item Вычислить:
	\begin{enumcols}[itemcolumns=3]
		\item \( 78^2-77^2 \)
		\item \( \dfrac{73^2-54^2}{19} \)
		\item \( \dfrac{83^2-19^2}{39^2-25^2} \)
	\end{enumcols}
	\item Вычислить:
	\begin{enumcols}[itemcolumns=3]
		\item \( \sqrt{2,7}\cdot\sqrt{1,2} \)
		\item \( \sqrt{10\cdot20\cdot48\cdot36\cdot75\cdot98} \)
		\item \( \sqrt{\dfrac{165^2-124^2}{164}} \)
	\end{enumcols}
	\item Сократить дробь:
	\begin{enumcols}[itemcolumns=2]
		\item \( \dfrac{x^2-11}{x+\sqrt{11}} \)
		\item \( \dfrac{m+3\sqrt{m}}{m-9} \)
		\item \( \dfrac{21+\sqrt{21}}{\sqrt{21}} \)
		\item \( \dfrac{36-c}{c-12\sqrt{c}+36} \)
	\end{enumcols}
	\item Упростить выражение \( \sqrt{(\sqrt{7}-3)^2} \)
\end{listofex}
\newpage
\title{Занятие №4}
\begin{listofex}
	\item Упростить выражение:
	\begin{enumcols}[itemcolumns=2]
		\item \exercise{1433}
		\item \exercise{1383}
	\end{enumcols}
	\item Вычислить:
	\begin{enumcols}[itemcolumns=3]
		\item \( \dfrac{45^6}{75^3} \)
		\item \( \dfrac{2^{14}\cdot3^{11}}{72^6} \)
		\item \( \dfrac{5^{32}\cdot7^{30}}{35^{31}} \)
	\end{enumcols}
	\item Вычислить:
	\begin{enumcols}[itemcolumns=3]
		\item \( 65^2-64^2 \)
		\item \( \dfrac{65^2-91^2}{26} \)
		\item \( \dfrac{57^2-33^2}{43^2-67^2} \)
		\item \( \dfrac{73^3-37^3}{36}+73\cdot37 \)
	\end{enumcols}
	\item Вычислить:
	\begin{enumcols}[itemcolumns=3]
		\item \( \sqrt{5,7}\cdot\sqrt{1,9} \)
		\item \( \sqrt{\dfrac{149^2-76^2}{457^2-384^2}} \)
		\item \exercise{1744}
	\end{enumcols}
	\item Сократить дробь:
	\begin{enumcols}[itemcolumns=2]
		\item \( \dfrac{a-\sqrt{17}}{17-a^2} \)
		\item \( \dfrac{x-7\sqrt{x}}{49-x} \)
		\item \( \dfrac{\sqrt{13}}{\sqrt{13}-13} \)
		\item \( \dfrac{y-10\sqrt{y}+25}{y-25} \)
	\end{enumcols}
	\item Упростить выражение \( \sqrt{(5-\sqrt{33})^2} \)
	\item Вычислить:
	\begin{enumcols}[itemcolumns=2]
		\item \exercise{1603}
		\item \exercise{1606}
	\end{enumcols}
\end{listofex}
\newpage
\title{Домашняя работа №2}
\begin{listofex}
	\item Упростить выражение:
	\begin{enumcols}[itemcolumns=2]
		\item \exercise{1472}
		\item \exercise{1512}
		\item \exercise{1467}
		\item \exercise{1464}
	\end{enumcols}
	\item Вычислить:
	\begin{enumcols}[itemcolumns=3]
		\item \( \dfrac{27^6\cdot81^5}{3^{40}} \)
		\item \( \dfrac{33^{14}}{11^{15}\cdot3^{11}} \)
		\item \( \dfrac{49^{11}\cdot32^{4}}{196^{12}} \)
	\end{enumcols}
	\item Вычислить:
	\begin{enumcols}[itemcolumns=3]
		\item \( 453^2-452^2 \)
		\item \( \dfrac{45^2-73^2}{56} \)
		\item \( \dfrac{32^3+17^3}{49}-32\cdot17 \)
	\end{enumcols}
	\item Вычислить:
	\begin{enumcols}[itemcolumns=3]
		\item \( \sqrt{50}\cdot\sqrt{4,5} \)
		\item \( \sqrt{\dfrac{145,5^2-96,5^2}{193,5^2-31,5^2}} \)
		\item \exercise{1756}
	\end{enumcols}
	\item Сократить дробь:
	\begin{enumcols}[itemcolumns=4]
		\item \( \dfrac{t-\sqrt{5}}{5-t^2} \)
		\item \( \dfrac{9-a}{a-3\sqrt{a}} \)
		\item \( \dfrac{24+\sqrt{24}}{\sqrt{24}} \)
		\item \( \dfrac{x+16\sqrt{x}+64}{64-x} \)
	\end{enumcols}
	\item Упростить выражение \( \sqrt{(\sqrt{9}-12)^2} \)
	\item Вычислить:
	\begin{enumcols}[itemcolumns=2]
		\item \exercise{1604}
		\item \exercise{1605}
	\end{enumcols}
\end{listofex}
\newpage
\title{Занятие №5}
\begin{listofex}
	\item Решить уравнение:
	\begin{enumcols}[itemcolumns=2]
		\item \exercise{3588}
		\item \exercise{3595}
		\item \exercise{3623}
		\item \exercise{3585}
	\end{enumcols}
	\item Решить уравнение:
	\begin{enumcols}[itemcolumns=3]
		\item \exercise{455}
		\item \exercise{468}
		\item \exercise{477}
	\end{enumcols}
	\item Решить уравнение:
	\begin{enumcols}[itemcolumns=2]
		\item \exercise{491}
		\item \exercise{496}
	\end{enumcols}
	\item Решить уравнение:
	\begin{enumcols}[itemcolumns=3]
		\item \exercise{41}
		\item \exercise{543}
		\item \exercise{549}
	\end{enumcols}
	\item Решить уравнение:
	\begin{enumcols}[itemcolumns=2]
		\item \exercise{3662}
		\item \exercise{498}
	\end{enumcols}
	\item Решить уравнение:
	\begin{enumcols}[itemcolumns=3]
		\item \exercise{32}
		\item \exercise{16}
		\item \exercise{3670}
	\end{enumcols}
	\item Решить уравнение:
	\begin{enumcols}[itemcolumns=2]
		\item \exercise{3627}
		\item \exercise{3629}
	\end{enumcols}
\end{listofex}
%\newpage
%\title{Занятие №6}
%\begin{listofex}
%	\item 1
%	
%\end{listofex}
%\newpage
%\title{Домашняя работа №3}
%\begin{listofex}
%	\item 1
%	
%\end{listofex}
%\newpage
%\title{Занятие №7}
%\begin{listofex}
%	\item 1
%	
%\end{listofex}
%\newpage
%\title{Занятие №8}
%\begin{listofex}
%	\item 1
%	
%\end{listofex}
%\newpage
%\title{Домашняя работа №4}
%\begin{listofex}
%	\item 1
%	
%\end{listofex}
%\newpage
%\title{Проверочная работа}
%\begin{listofex}
%	\item 1
%	
%\end{listofex}