%
%===============>>  Сторожук Модуль 4 <<=============
%
\setmodule{4}
%
%===============>>  Занятие 1  <<===============
%
\begin{class}[number=1]
	\begin{listofex}
		\item Некоторая компания продает свою продукцию по цене p=400 руб. за единицу, переменные затраты на производство одной единицы продукции составляют \( v=200 \) руб., постоянные расходы предприятия \( f= 600000 \) руб. в месяц. Месячная операционная прибыль предприятия (в рублях) вычисляется по формуле \( \pi(q)=q(p-v)-f \). Определите месячный объeм производства \( q \) (единиц продукции), при котором месячная операционная прибыль предприятия будет равна \( 900 000 \) руб.
		\item Если достаточно быстро вращать ведeрко с водой на верeвке в вертикальной плоскости, то вода не будет выливаться. При вращении ведeрка сила давления воды на дно не остаeтся постоянной: она максимальна в нижней точке и минимальна в верхней. Вода не будет выливаться, если сила еe давления на дно будет положительной во всех точках траектории кроме верхней, где она может быть равной нулю. В верхней точке сила давления, выраженная в ньютонах, равна \( P=m \left( \dfrac{v^2}{L}-g \right) \), где \( m \) -- масса воды в килограммах, \( v \) скорость движения ведерка в м/с, \( L \) -- длина вервеки в метрах, \( g \) -- ускорение свободного падения (считайте \( g=10 \) м/с\( ^2 \)). С какой наименьшей скоростью надо вращать ведeрко, чтобы вода не выливалась, если длина верeвки равна \( 40 \) см? Ответ выразите в м/с.
		\item Камнеметательная машина выстреливает камни под некоторым острым углом к горизонту. Траектория полeта камня описывается формулой \( y=ax^2+bx \), где \( a=-\dfrac{1}{100} \) м\( ^{-1} \), \( b=1 \) -- постоянные параметры,\( x(M) \) -- смещение камня по горизонтали, \( y(M) \) -- высота камня над землeй. На каком наибольшем расстоянии (в метрах) от крепостной стены высотой \( 8 \) м нужно расположить машину, чтобы камни пролетали над стеной на высоте не менее \( 1 \) метра?
		\item Моторная лодка прошла против течения реки \( 112  \) км и вернулась в пункт отправления, затратив на обратный путь на \( 6 \) часов меньше. Найдите скорость течения, если скорость лодки в неподвижной воде равна \( 11 \) км/ч. Ответ дайте в км/ч.
		\item Теплоход проходит по течению реки до пункта назначения \( 200 \) км и после стоянки возвращается в пункт отправления. Найдите скорость течения, если скорость теплохода в неподвижной воде равна \( 15 \) км/ч, стоянка длится \( 10 \) часов, а в пункт отправления теплоход возвращается через \( 40 \) часов после отплытия из него. Ответ дайте в км/ч.
		\item Пристани \( A \) и \( B \) расположены на озере, расстояние между ними 390 км. Баржа отправилась с постоянной скоростью из \( A \) в \( B \). На следующий день после прибытия она отправилась обратно со скоростью на \( 3 \) км/ч больше прежней, сделав по пути остановку на \( 9 \) часов. В результате она затратила на обратный путь столько же времени, сколько на путь из \( A \) в \( B \). Найдите скорость баржи на пути из \( A \) в \( B \). Ответ дайте в км/ч.
		\item Из пункта \( A \) в пункт \( B \), расстояние между которыми \( 75 \) км, одновременно выехали автомобилист и велосипедист. Известно, что за час автомобилист проезжает на \( 40 \) км больше, чем велосипедист. Определите скорость велосипедиста, если известно, что он прибыл в пункт \( B \) на \( 6 \) часов позже автомобилиста. Ответ дайте в км/ч.
		\item Велосипедист выехал с постоянной скоростью из города \( A \) в город \( B \), расстояние между которыми равно \( 70 \) км. На следующий день он отправился обратно в \( A \) со скоростью на \( 3 \) км/ч больше прежней. По дороге он сделал остановку на \( 3 \) часа. В результате велосипедист затратил на обратный путь столько же времени, сколько на путь из \( A \) в \( B \). Найдите скорость велосипедиста на пути из \( B \) в \( A \). Ответ дайте в км/ч.
	\end{listofex}
\end{class}
%
%===============>>  Занятие 2  <<===============
%
\begin{class}[number=2]
	\begin{listofex}
		\item Два тела массой \( m=2 \) кг каждое, движутся с одинаковой скоростью  \( v =10  \)м/с под углом \( 2\alpha \)  друг к другу. Энергия (в джоулях), выделяющаяся при их абсолютно неупругом соударении определяется выражением \( Q=mv^2\sin^2\alpha \). Под каким наименьшим углом \( 2\alpha \) (в градусах) должны двигаться тела, чтобы в результате соударения выделилось не менее \( 50 \) джоулей?
		\item Моторная лодка прошла против течения реки \( 255 \) км и вернулась в пункт отправления, затратив на обратный путь на \( 2 \) часа меньше. Найдите скорость лодки в неподвижной воде, если скорость течения равна \( 1 \) км/ч. Ответ дайте в км/ч.
		\item Теплоход проходит по течению реки до пункта назначения \( 255 \) км и после стоянки возвращается в пункт отправления. Найдите скорость теплохода в неподвижной воде, если скорость течения равна \( 1 \) км/ч, стоянка длится \( 2 \) часа, а в пункт отправления теплоход возвращается через \( 34 \) часа после отплытия из него. Ответ дайте в км/ч.
		\item Баржа в \( 10:00 \) вышла из пункта \( A \) в пункт \( B \), расположенный в \( 15 \) км от \( A \). Пробыв в пункте B \( 1 \) час \( 20 \) минут, баржа отправилась назад и вернулась в пункт \( A \) в \( 16:00 \) того же дня. Определите (в км/час) скорость течения реки, если известно, что собственная скорость баржи равна \( 7 \) км/ч.
		\item Моторная лодка в \( 10:00 \) вышла из пункта \( А \) в пункт \( В \), расположенный в \( 30 \) км от \( А \). Пробыв в пункте \( В \) \( 2 \) часа \( 30 \) минут, лодка отправилась назад и вернулась в пункт \( А \) в \( 18:00 \) того же дня. Определите (в км/ч) собственную скорость лодки, если известно, что скорость течения реки \( 1 \) км/ч.
		\item Первые \( 190 \) км автомобиль ехал со скоростью \( 50 \) км/ч, следующие \( 180 \) км -- со скоростью \( 90 \) км/ч, а затем \( 170 \) км -- со скоростью \( 100 \) км/ч. Найдите среднюю скорость автомобиля на протяжении всего пути. Ответ дайте в км/ч.
		\item Первые два часа автомобиль ехал со скоростью \( 50 \) км/ч, следующий час -- со скоростью \( 100 \) км/ч, а затем два часа -- со скоростью \( 75 \) км/ч. Найдите среднюю скорость автомобиля на протяжении всего пути. Ответ дайте в км/ч.
		\item Из городов \( A \) и \( B \), расстояние между которыми равно \( 330 \) км, навстречу друг другу одновременно выехали два автомобиля и встретились через \( 3 \) часа на расстоянии \( 180 \) км от города \( B \). Найдите скорость автомобиля, выехавшего из города \( A \). Ответ дайте в км/ч.
		\item Два велосипедиста одновременно отправились в \( 88 \)-километровый пробег. Первый ехал со скоростью, на \( 3 \) км/ч большей, чем скорость второго, и прибыл к финишу на \( 3 \) часа раньше второго. Найти скорость велосипедиста, пришедшего к финишу вторым. Ответ дайте в км/ч.
	\end{listofex}
\end{class}
%
%===============>>  Домашняя работа 1  <<===============
%
\begin{homework}[number=1]
	\begin{listofex}
		\item Велосипедист выехал с постоянной скоростью из города \( A \) в город \( B \), расстояние между которыми равно \( 154 \) км. На следующий день он отправился обратно со скоростью на \( 3 \) км/ч больше прежней. По дороге он сделал остановку на \( 3 \) часа. В результате он затратил на обратный путь столько же времени, сколько на путь из \( A \) в \( B \). Найдите скорость велосипедиста на пути из \( A \) в \( B \). Ответ дайте в км/ч.
		\item Из городов \( A \) и \( B \), расстояние между которыми равно \( 300 \) км, навстречу друг другу одновременно выехали два автомобиля и встретились через \( 2 \) часа на расстоянии \( 180 \) км от города \( B \). Найдите скорость автомобиля, выехавшего из города \( A \). Ответ дайте в км/ч.
		\item От пристани \( A \) к пристани \( B \), расстояние между которыми равно \( 224 \) км, отправился с постоянной скоростью первый теплоход, а через \( 2 \) часа после этого следом за ним со скоростью на \( 2 \) км/ч большей отправился второй. Найдите скорость второго теплохода, если в пункт \( B \) он прибыл одновременно с первым. Ответ дайте в км/ч.
		\item Ёмкость высоковольтного конденсатора в телевизоре \( C=2\cdot10^{-6} \) Ф. Параллельно с конденсатором подключeн резистор с сопротивлением \( R=5\cdot10^6 \) Ом. Во время работы телевизора напряжение на конденсаторе \(  U_0 = 16 \) кВ. После выключения телевизора напряжение на конденсаторе убывает до значения \( U \) (кВ) за время, определяемое выражением \( t=\alpha RC\log_2\dfrac{U}{U_0} \) (с), где \( \alpha=0,7 \) -- постоянная. Определите напряжение на конденсаторе, если после выключения телевизора прошло \( 21 \) с. Ответ дайте в киловольтах.
		\item \exercise{1244}
	\end{listofex}
\end{homework}
%
%===============>>  Занятие 3  <<===============
%
\begin{class}[number=3]
	\begin{listofex}
		\item \exercise{1234}
		\item \exercise{1242}
		\item \exercise{1260}
		\item \exercise{1264}
		\item Два мотоциклиста стартуют одновременно в одном направлении из двух диаметрально противоположных точек круговой трассы, длина которой равна \( 14 \) км. Через сколько минут мотоциклисты поравняются в первый раз, если скорость одного из них на \( 21 \) км/ч больше скорости другого?
		\item Два гонщика участвуют в гонках. Им предстоит проехать \( 60 \) кругов по кольцевой трассе протяжённостью \( 3 \) км. Оба гонщика стартовали одновременно, а на финиш первый пришёл раньше второго на \( 10 \) минут. Чему равнялась средняя скорость второго гонщика, если известно, что первый гонщик в первый раз обогнал второго на круг через \( 15 \) минут? Ответ дайте в км/ч.
		\item Часы со стрелками показывают \( 8 \) часов ровно. Через сколько минут минутная стрелка в четвертый раз поравняется с часовой?
		\item Из пункта \( A \) круговой трассы выехал велосипедист. Через \( 30 \) минут он еще не вернулся в пункт \( A \) и из пункта \( A \) следом за ним отправился мотоциклист. Через \( 10 \) минут после отправления он догнал велосипедиста в первый раз, а еще через \( 30 \) минут после этого догнал его во второй раз. Найдите скорость мотоциклиста, если длина трассы равна \( 30 \) км. Ответ дайте в км/ч.
		\item Рабочие прокладывают тоннель длиной \( 500 \) метров, ежедневно увеличивая норму прокладки на одно и то же число метров. Известно, что за первый день рабочие проложили \( 3 \) метра тоннеля. Определите, сколько метров тоннеля проложили рабочие в последний день, если вся работа была выполнена за \( 10 \) дней.
		\item Васе надо решить \( 434 \) задачи. Ежедневно он решает на одно и то же количество задач больше по сравнению с предыдущим днем. Известно, что за первый день Вася решил \( 5 \) задач. Определите, сколько задач решил Вася в последний день, если со всеми задачами он справился за \( 14 \) дней.
		\item Компания «Альфа» начала инвестировать средства в перспективную отрасль в \( 2001 \) году, имея капитал в размере \( 5000 \) долларов. Каждый год, начиная с \( 2002 \) года, она получала прибыль, которая составляла \( 200\% \) от капитала предыдущего года. А компания «Бета» начала инвестировать средства в другую отрасль в \( 2003 \) году, имея капитал в размере \( 10 000 \) долларов, и, начиная с \( 2004 \) года, ежегодно получала прибыль, составляющую \( 400\% \) от капитала предыдущего года. На сколько долларов капитал одной из компаний был больше капитала другой к концу \( 2006 \) года, если прибыль из оборота не изымалась?
		\item \exercise{1249}
	\end{listofex}
\end{class}
%
%===============>>  Занятие 4  <<===============
%
%\begin{class}[number=4]
%	\begin{listofex}
%		\item Пусто
%	\end{listofex}
%\end{class}
%
%===============>>  Домашняя работа 2  <<===============
%
\begin{homework}[number=2]
	\begin{listofex}
		\item Из одной точки круговой трассы, длина которой равна \( 14 \) км, одновременно в одном направлении стартовали два автомобиля. Скорость первого автомобиля равна \( 80 \) км/ч, и через \( 40 \) минут после старта он опережал второй автомобиль на один круг. Найдите скорость второго автомобиля. Ответ дайте в км/ч.
		\item Бригада маляров красит забор длиной \( 240 \) метров, ежедневно увеличивая норму покраски на одно и то же число метров. Известно, что за первый и последний день в сумме бригада покрасила \( 60 \) метров забора. Определите, сколько дней бригада маляров красила весь забор.
		\item Компания "Альфа" начала инвестировать средства в перспективную отрасль в \( 2001 \) году, имея капитал в размере \( 3000 \) долларов. Каждый год, начиная с \( 2002 \) года, она получала прибыль, которая составляла \( 100\% \) от капитала предыдущего года. А компания "Бета" начала инвестировать средства в другую отрасль в \( 2004 \) году, имея капитал в размере \( 4500 \) долларов, и, начиная с \( 2005 \) года, ежегодно получала прибыль, составляющую \( 300\% \) от капитала предыдущего года. На сколько долларов капитал одной из компаний был больше капитала другой к концу \( 2008 \) года, если прибыль из оборота не изымалась?
		\item При адиабатическом процессе для идеального газа выполняется закон \( pV^k=10^5 \)  Па\( \cdot \)на м\( ^5 \), где \( p \) -- давление газа в паскалях, \( V \) -- объeм газа в кубических метрах, \( k=\dfrac{5}{3} \).  Найдите, какой объём \( V \) (в куб. м) будет занимать газ при давлении \( p \), равном \( 3,2\cdot10^6 \) Па.
		\item \exercise{1263}
	\end{listofex}
\end{homework}
%
%===============>>  Занятие 5  <<===============
%
%\begin{class}[number=6]
%	\begin{listofex}
%		\item Пусто
%	\end{listofex}
%\end{class}
%
%===============>>  Занятие 6  <<===============
%
%\begin{class}[number=6]
%	\begin{listofex}
%		\item Пусто
%	\end{listofex}
%\end{class}
%
%===============>>  Домашняя работа 3  <<===============
%
%\begin{homework}[number=3]
%	\begin{listofex}
%		\item Пусто
%	\end{listofex}
%\end{homework}
%
%===============>>  Занятие 7  <<===============
%
%\begin{class}[number=7]
%	\begin{listofex}
%		\item Пусто
%	\end{listofex}
%\end{class}
%
%===============>>  Проверочная работа  <<===============
%
%\begin{exam}
%	\begin{listofex}
%		\item Пусто
%	\end{listofex}
%\end{exam}