%
%===============>>  Сторожук Модуль 4 <<=============
%
\setmodule{4}
%
%===============>>  Занятие 1  <<===============
%
\begin{class}[number=1]
	\begin{listofex}
		\item Некоторая компания продает свою продукцию по цене p=400 руб. за единицу, переменные затраты на производство одной единицы продукции составляют \( v=200 \) руб., постоянные расходы предприятия \( f= 600000 \) руб. в месяц. Месячная операционная прибыль предприятия (в рублях) вычисляется по формуле \( \pi(q)=q(p-v)-f \). Определите месячный объeм производства \( q \) (единиц продукции), при котором месячная операционная прибыль предприятия будет равна \( 900 000 \) руб.
		\item Если достаточно быстро вращать ведeрко с водой на верeвке в вертикальной плоскости, то вода не будет выливаться. При вращении ведeрка сила давления воды на дно не остаeтся постоянной: она максимальна в нижней точке и минимальна в верхней. Вода не будет выливаться, если сила еe давления на дно будет положительной во всех точках траектории кроме верхней, где она может быть равной нулю. В верхней точке сила давления, выраженная в ньютонах, равна \( P=m \left( \dfrac{v^2}{L}-g \right) \), где \( m \) -- масса воды в килограммах, \( v \) скорость движения ведерка в м/с, \( L \) -- длина вервеки в метрах, \( g \) -- ускорение свободного падения (считайте \( g=10 \) м/с\( ^2 \)). С какой наименьшей скоростью надо вращать ведeрко, чтобы вода не выливалась, если длина верeвки равна \( 40 \) см? Ответ выразите в м/с.
		\item Камнеметательная машина выстреливает камни под некоторым острым углом к горизонту. Траектория полeта камня описывается формулой \( y=ax^2+bx \), где \( a=-\dfrac{1}{100} \) м\( ^{-1} \), \( b=1 \) -- постоянные параметры,\( x(M) \) -- смещение камня по горизонтали, \( y(M) \) -- высота камня над землeй. На каком наибольшем расстоянии (в метрах) от крепостной стены высотой \( 8 \) м нужно расположить машину, чтобы камни пролетали над стеной на высоте не менее \( 1 \) метра?
		\item Моторная лодка прошла против течения реки \( 112  \) км и вернулась в пункт отправления, затратив на обратный путь на \( 6 \) часов меньше. Найдите скорость течения, если скорость лодки в неподвижной воде равна \( 11 \) км/ч. Ответ дайте в км/ч.
		\item Теплоход проходит по течению реки до пункта назначения \( 200 \) км и после стоянки возвращается в пункт отправления. Найдите скорость течения, если скорость теплохода в неподвижной воде равна \( 15 \) км/ч, стоянка длится \( 10 \) часов, а в пункт отправления теплоход возвращается через \( 40 \) часов после отплытия из него. Ответ дайте в км/ч.
		\item Пристани \( A \) и \( B \) расположены на озере, расстояние между ними 390 км. Баржа отправилась с постоянной скоростью из \( A \) в \( B \). На следующий день после прибытия она отправилась обратно со скоростью на \( 3 \) км/ч больше прежней, сделав по пути остановку на \( 9 \) часов. В результате она затратила на обратный путь столько же времени, сколько на путь из \( A \) в \( B \). Найдите скорость баржи на пути из \( A \) в \( B \). Ответ дайте в км/ч.
		\item Из пункта \( A \) в пункт \( B \), расстояние между которыми \( 75 \) км, одновременно выехали автомобилист и велосипедист. Известно, что за час автомобилист проезжает на \( 40 \) км больше, чем велосипедист. Определите скорость велосипедиста, если известно, что он прибыл в пункт \( B \) на \( 6 \) часов позже автомобилиста. Ответ дайте в км/ч.
		\item Велосипедист выехал с постоянной скоростью из города \( A \) в город \( B \), расстояние между которыми равно \( 70 \) км. На следующий день он отправился обратно в \( A \) со скоростью на \( 3 \) км/ч больше прежней. По дороге он сделал остановку на \( 3 \) часа. В результате велосипедист затратил на обратный путь столько же времени, сколько на путь из \( A \) в \( B \). Найдите скорость велосипедиста на пути из \( B \) в \( A \). Ответ дайте в км/ч.
	\end{listofex}
\end{class}
%
%===============>>  Занятие 2  <<===============
%
%\begin{class}[number=2]
%	\begin{listofex}
%		\item Пусто
%	\end{listofex}
%\end{class}
%
%===============>>  Домашняя работа 1  <<===============
%
%\begin{homework}[number=1]
%	\begin{listofex}
%		\item Пусто
%	\end{listofex}
%\end{homework}
%
%===============>>  Занятие 3  <<===============
%
%\begin{class}[number=3]
%	\begin{listofex}
%		\item Пусто
%	\end{listofex}
%\end{class}
%
%===============>>  Занятие 4  <<===============
%
%\begin{class}[number=4]
%	\begin{listofex}
%		\item Пусто
%	\end{listofex}
%\end{class}
%
%===============>>  Домашняя работа 2  <<===============
%
%\begin{homework}[number=2]
%	\begin{listofex}
%		\item Пусто
%	\end{listofex}
%\end{homework}
%
%===============>>  Занятие 5  <<===============
%
%\begin{class}[number=6]
%	\begin{listofex}
%		\item Пусто
%	\end{listofex}
%\end{class}
%
%===============>>  Занятие 6  <<===============
%
%\begin{class}[number=6]
%	\begin{listofex}
%		\item Пусто
%	\end{listofex}
%\end{class}
%
%===============>>  Домашняя работа 3  <<===============
%
%\begin{homework}[number=3]
%	\begin{listofex}
%		\item Пусто
%	\end{listofex}
%\end{homework}
%
%===============>>  Занятие 7  <<===============
%
%\begin{class}[number=7]
%	\begin{listofex}
%		\item Пусто
%	\end{listofex}
%\end{class}
%
%===============>>  Проверочная работа  <<===============
%
%\begin{exam}
%	\begin{listofex}
%		\item Пусто
%	\end{listofex}
%\end{exam}