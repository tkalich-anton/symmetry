%
%===============>>  ГРУППА 11-6 МОДУЛЬ 8  <<=============
%
\setmodule{Вспомнить всё}

%BEGIN_FOLD % ====>>_____ Занятие 1 _____<<====
\begin{class}[number=1]
	\begin{listofex}
		\item Вычислить:
		\begin{enumcols}[itemcolumns=2]
			\item \( \sin150\degree\sin(-120\degree)\ctg(-225\degree) \)
			\item \( \tg\left( -\dfrac{17\pi}{4} \right)\sin\dfrac{14\pi}{3} \cos\left( -\dfrac{25\pi}{4}\right)\ \)
		\end{enumcols}
		\item Вычислите:
		\begin{tasks}(2)
			\task \(\dfrac{4 \cdot \sin61\degree}{\cos29\degree}\)
			\task \(\dfrac{2 \cdot \sin136\degree}{\sin68\degree \cdot \sin22\degree}\)
		\end{tasks}
		\item Вычислить:
		\begin{enumcols}[itemcolumns=1]
			\item \exercise{1117}
			\item \exercise{1119}
		\end{enumcols}
		\item Решить уравнение:
		\begin{tasks}(2)
			\task \( \sin x = \dfrac{1}{2} \)
			\task \( 2\sin x = -\sqrt{2} \)
			\task \( 3\tg x = \sqrt{3} \)
			\task \( \sin^2 x = \dfrac{1}{2} \)
		\end{tasks}
%		\item Решить неравенство:
%		\begin{tasks}(2)
%			\task \( \sin x > \dfrac{1}{2} \)
%			\task \( \cos x \le \dfrac{\sqrt{2}}{2} \)
%			\task \( \sin x < 1 \)
%			\task \( \sin x \ge -\dfrac{3}{2}\)
%		\end{tasks}
		\item  Решите уравнение \( 8\sin x + 4\cos^2 x = 7 \)
		\item  Решите уравнение \( \tg x-2\ctg x=1 \)
		\item  Решите уравнение \( \sqrt{2}\cos {(\pi-x)} + 2\cos^2{(\pi+x)}=0 \)
		\item  Решите уравнение \( 4\cos^4 x - 15\cos 2x -1= 0 \)
		\item а) Решить уравнение: \( 9^{\sin x}+9^{-\sin x}=\dfrac{10}{3} \)\\
		б) Укажите корни этого уравнения, принадлежащие отрезку \( \left[ -\dfrac{7\pi}{2};-2\pi \right] \)
		\item Теплоход проходит по течению реки до пункта назначения \( 255 \) км и после стоянки возвращается в пункт отправления. Найдите скорость теплохода в неподвижной воде, если скорость течения равна \( 1 \) км/ч, стоянка длится \( 2 \) часа, а в пункт отправления теплоход возвращается через \( 34 \) часа после отплытия из него. Ответ дайте в км/ч.
		\item Баржа в \( 10:00 \) вышла из пункта \( A \) в пункт \( B \), расположенный в \( 15 \) км от \( A \). Пробыв в пункте \( B \) \( 1 \) час \( 20 \) минут, баржа отправилась назад и вернулась в пункт \( A \) в \( 16:00 \) того же дня. Определите (в км/час) скорость течения реки, если известно, что собственная скорость баржи равна \( 7 \) км/ч.
		\item Путешественник переплыл море на яхте со средней скоростью \( 20 \) км/ч. Обратно он летел на спортивном самолете со скоростью \( 480 \) км/ч. Найдите среднюю скорость путешественника на протяжении всего пути. Ответ дайте в км/ч.
	\end{listofex}
\end{class}
%END_FOLD

%BEGIN_FOLD % ====>>_ Домашняя работа 1 _<<====
\begin{homework}[number=1]
		\begin{listofex}
			\item Домашняя работа
		\end{listofex}
\end{homework}
%END_FOLD

%BEGIN_FOLD % ====>>_____ Занятие 2 _____<<====
\begin{class}[number=2]
	\begin{listofex}
		\item Вычислить:
		\begin{tasks}(2)
			\task \( 2^{\log_2 3} \)
			\task \( 9^{\log_3 5} \)
			\task \( 5^{\log_{\sqrt[3]{5}} 2} \)
			\task \( (\sqrt[3]{5})^{\log_5 8} \)
		\end{tasks}
		\item Вычислите:
		\begin{tasks}(2)
			\task \( \log_3 0,9 + \log_3 30 \)
			\task \( \log_8 \dfrac{8}{7} + \log_8 \dfrac{7}{8} \)
			\task \( \log_4 48 - \log_4 3 \)
			\task \( \log_5 100 - 2 \log_5 2 \)
		\end{tasks}
		\item Решите уравнения: %По первые 4 с решуегэ простейшие и сложнейшие
		\begin{tasks}(2)
			\task \( \log_2 (4-x)=7 \)
			\task \( \log_5(4+x)=2 \)
			\task \( \log_5(5-x)=\log_5 3 \)
			\task \( \log_2(15+x)=\log_2 3 \)
			\task \( \log_5 (2-x) = \log_{25} x^4 \)
			\task \( \log_2 (x^2-14x)=5 \)
		\end{tasks}
		\item Решите неравенства: %299 1-8 A 34-37
		\begin{tasks}(2)
			\task \( \log_5 x<2 \)
			\task \( \log_3 x \le 3 \)
			\task \( \log_4 x \ge -0,5 \)
			\task \( \log_{0,04}x \ge -1 \)
			\task \( \log_5 (4x+5) < 0 \)
			\task \( \log_{0,1} (3x+25) < -2 \)
		\end{tasks}
		\item Решите систему неравенств: %Диагност с300 1 вар 10-11 2 вар 10-11
		\[ \begin{cases} \log_{0,2}(2x-7) \le -2 \\ \log_3 (x-5) \le 3 \end{cases} \]
	\end{listofex}
\end{class}
%END_FOLD

%BEGIN_FOLD % ====>>_ Домашняя работа 2 _<<====
\begin{homework}[number=2]
	\begin{listofex}
		\item Домашняя работа
	\end{listofex}
\end{homework}
%END_FOLD

%BEGIN_FOLD % ====>>_____ Занятие 3 _____<<====
\begin{class}[number=3]
	\begin{listofex}
		\item Занятие 3
	\end{listofex}
\end{class}
%END_FOLD

%BEGIN_FOLD % ====>>_ Домашняя работа 3 _<<====
\begin{homework}[number=3]
	\begin{listofex}
		\item Домашняя работа
	\end{listofex}
\end{homework}
%END_FOLD

%BEGIN_FOLD % ====>>_____ Занятие 4 _____<<====
\begin{class}[number=4]
	\begin{listofex}
		\item Пусто
	\end{listofex}
\end{class}
%END_FOLD