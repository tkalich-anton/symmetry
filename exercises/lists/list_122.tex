%Группа 9-1 Модуль 1 Занятие №1
\begin{listofex}
	\item Упростить выражение:
	\begin{enumcols}[itemcolumns=2]
		\item \exercise{117}
		\item \exercise{120}
		\item \exercise{126}
		\item \exercise{853}
		\item \exercise{1363}
		\item \exercise{1368}
		
		%\item \exercise{116} % Занятие №2
		%\item \exercise{121} % Занятие №2
		%\item \exercise{1352} % Занятие №2
		%\item \exercise{1351} % Занятие №2
		%\item \exercise{1364} % Занятие №2
		%\item \exercise{1387} % Занятие №2
		
		%\item \exercise{122} % ДЗ №1
		%\item \exercise{123} % ДЗ №1
		%\item \exercise{124} % ДЗ №1
		%\item \exercise{127} % ДЗ №1
		%\item \exercise{855} % ДЗ №1
		%\item \exercise{1362} % ДЗ №1
	\end{enumcols}

	\item \exercise{1379}
	%\item \exercise{1463} % Занятие №2
	%\item \exercise{1471} % ДЗ №1
	
	\item \exercise{1402}
	%\item \exercise{1420} % Занятие №2
	%\item \exercise{1431} % ДЗ №1
	
	\item Найдите значение выражения \( x^2+\dfrac{1}{x^2} \), если известно, что \( x-\dfrac{1}{x}=5 \)
	%\item Найдите значение выражения \( 4x^2+\dfrac{1}{x^2} \), если известно, что \( 2x+\dfrac{1}{x}=7 \) % Занятие №2
	%\item Найдите значение выражения \( 25x^2+\dfrac{1}{x^2} \), если известно, что \( 5x+\dfrac{1}{x}=4 \) % ДЗ №1
	
	\item Из формулы \( \dfrac{1}{F}=\dfrac{1}{f}+\dfrac{1}{d} \) выразите: а) \( F \); б) \( d \)
	%\item Из формулы \( S_n=\dfrac{2a_1+d(n+1)}{2}\cdot n \) выразите: а) \( a_1 \); б) \( d \) % Занятие №2
	%\item Из формулы \( S=\dfrac{abc}{4R} \) выразите: а) \( c \); б) \( R \) % ДЗ №1
	
	\item Из формулы \( F=\gamma\cdot\dfrac{m_1m_2}{r^2} \) выразите \( r \). Все величины положительны.
	%\item Из формулы \( P=\dfrac{U^2}{R} \) выразите \( U \). Все величины положительны. % Занятие №2
	%\item Из формулы \( Q=I^2Rt \) выразите \( I \). Все величины положительны. % ДЗ №1
	
	\item Вычислить:
	\begin{enumcols}[itemcolumns=3]
		\item \( \sqrt{77\cdot24\cdot33\cdot14} \)
		%\item \( \sqrt{5\cdot6\cdot8\cdot20\cdot27} \) % Занятие №2
		%\item \( \sqrt{21\cdot65\cdot39\cdot35} \) % ДЗ №1
		\item \( \sqrt{21}\cdot\sqrt{3\dfrac{6}{7}} \)
		%\item \( \sqrt{15}\cdot\sqrt{6\dfrac{2}{3}} \) % Занятие №2
		%\item \( \sqrt{12}\cdot\sqrt{5\dfrac{1}{3}} \) % ДЗ №1
		\item \( \dfrac{(3\sqrt{5})^2}{15} \)
		%\item \( \dfrac{6}{(2\sqrt{3})^2} \) % Занятие №2
		%\item \( \dfrac{(5\sqrt{7})^2}{35} \) % ДЗ №1
	\end{enumcols}
	
	\item Расположите числа в порядке возрастания:\quad\( 4;\;3,8;\;\sqrt{15};\;\sqrt{5};\;4,3 \)
	%\item Расположите числа в порядке возрастания:\quad\( 5;\;\sqrt{26};\;,7;\;\sqrt{6};\;,1 \) % Занятие №2
	%\item Расположите числа в порядке возрастания:\quad\( 7;\;\sqrt{46};\;6,8;\;5\sqrt{2};\;7,2 \) % ДЗ №1
	
	\item Найдите значение выражения \( 3x^2-2x-1 \), если \( x=\dfrac{1-\sqrt{2}}{3} \)
	%\item Найдите значение выражения \( 2x^2-6x+3 \), если \( x=\dfrac{3-\sqrt{5}}{2} \) % Занятие №2
	%\item Найдите значение выражения \( a^2-6\sqrt{5}-1 \), если \( a=\sqrt{5}+4 \) % ДЗ №1
\end{listofex}