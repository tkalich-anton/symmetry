%Группа 8-2 Модуль 2
\title{Занятие №1}
\begin{listofex}
	\item \exercise{2350}
	\item \exercise{2390}
	\item \exercise{2380}
	\item \exercise{2385}
	\item \exercise{2381}
	\item \exercise{2393}
	\item \exercise{2040}
	\item \exercise{2424}
\end{listofex}
\newpage
\title{Занятие №2}
\begin{listofex}
		\item Дан треугольник \( ABC \), причем \( AB = AC \) и \( \angle A = 110\degree \). Внутри треугольника взята точка \( M \) такая, что \( \angle MBC = 30\degree \), а \( \angle MCB = 25\degree \). Найдите \( \angle AMC \).
	\item Докажите, что если медиана равна половине стороны, к которой она проведена, то такой треугольник прямоугольный.
	\item \exercise{2415}
	\item Докажите, что если треугольник вписан в окружность и одна из его сторон является диаметром этой окружности, то такой треугольник является прямоугольным.
	\item Докажите обратное, что если треугольник прямоугольный и вписан в окружность, то гипотенуза будет являться диаметром окружности.
	\item \exercise{2455}
	\item Докажите, что отличная от \( A \) точка пересечения окружностей, построенных на сторонах \( AB \) и \( AC \) треугольника \( ABC \) как на диаметрах, лежит на прямой \( BC \).
	\item Окружность, построенная на катете прямоугольного треугольника как на диаметре, делит гипотенузу пополам. Найдите углы треугольника.
	\item \exercise{2418}
	\item \exercise{2424}
\end{listofex}
\newpage
\title{Домашняя работа №1}
	\begin{listofex}
		\item Вычислить:
	\begin{enumcols}[itemcolumns=2]
		\item \( 3^7\cdot3^9:3^{14} \)
		\item \( \dfrac{10^8}{2^9\cdot2^8} \)
	\end{enumcols}
	\item \exercise{1489}
	\item \exercise{1308}
	\item Докажите, что в равных треугольниках соответствующие биссектрисы равны.
	\item В равностороннем треугольнике \( ABC \) биссектрисы \( CN \) и \( AM \) пересекаются в точке \( P \). Найдите \( \angle MPN \).
	\item \exercise{2347}
	\item \exercise{2423}
	\item \exercise{2456}
	\item \exercise{2412}
\end{listofex}
\newpage
\title{Занятие №3}
\begin{listofex}
	\item Докажите следующие свойства окружности:
	\begin{enumcols}[itemcolumns=1]
		\item диаметр, перпендикулярный хорде, делит ее пополам;
		\item диаметр, проходящий через середину хорды, не являющейся диаметром, перпендикулярен этой хорде;
		\item окружность симметрична относительно каждого своего
		диаметра;
		\item дуги окружности, заключенные между параллельными
		хордами, равны;
		\item хорды, удаленные от центра окружности на равные расстояния, равны.
	\end{enumcols}
	\item \exercise{2437}
	\item \exercise{2439}
	\item \exercise{2442}
	\item \exercise{2444}
	\item \exercise{2445}
\end{listofex}
\newpage
\title{Занятие №4}
\begin{listofex}
	\item \exercise{2438}
	\item \exercise{2441}
	\item \exercise{2443}
	\item \exercise{2460}
	\item \exercise{2468}
\end{listofex}
\newpage
\title{Домашняя работа №2}
\begin{listofex}
	\item \exercise{1522}
	\item \exercise{1317}
	\item \exercise{2436}
	\item \exercise{2440}
	\item \exercise{2440}
	\item \exercise{2454}
	\item \exercise{2457}
	\item \exercise{2459}
\end{listofex}
%\newpage
%\title{Занятие №5}
%\begin{listofex}
%
%\end{listofex}
%\newpage
%\title{Занятие №6}
%\begin{listofex}
%
%\end{listofex}
%\newpage
%\title{Занятие №7}
%\begin{listofex}
%
%\end{listofex}
%\newpage
%\title{Проверочная работа}
%\begin{listofex}
%
%\end{listofex}