%
%===============>>  Кузнецов Модуль 4-5 <<=============
%
\setmodule{5}
%
%===============>>  Занятие 1  <<===============
%
%\begin{class}[number=7]
%	\begin{listofex}
	%		\item Пусто
	%	\end{listofex}
%\end{class}
%
%===============>>  Занятие 2  <<===============
%
%\begin{class}[number=7]
%	\begin{listofex}
	%		\item Пусто
	%	\end{listofex}
%\end{class}
%
%
%===============>>  Занятие 3  <<===============
%
%\begin{class}[number=7]
%	\begin{listofex}
	%		\item Пусто
	%	\end{listofex}
%\end{class}
%
%===============>>  Занятие 4  <<===============
%
%\begin{class}[number=7]
%	\begin{listofex}
	%		\item Пусто
	%	\end{listofex}
%\end{class}
%
%===============>>  Занятие 5  <<===============
%
%\begin{class}[number=7]
%	\begin{listofex}
	%		\item Пусто
	%	\end{listofex}
%\end{class}
%
%===============>>  Занятие 6  <<===============
%
%\begin{class}[number=6]
%	\begin{listofex}
	%		\item Пусто
	%	\end{listofex}
%\end{class}
%
%===============>>  Занятие 7  <<===============
%
%\begin{class}[number=7]
%	\begin{listofex}
	%		\item Пусто
	%	\end{listofex}
%\end{class}
%
%===============>>  Домашняя работа 1  <<===============
%
\begin{homework}[number=1]
	\begin{listofex}
		\item Имеется два сплава. Первый содержит \(15\% \) никеля, второй --- \(35\%\) никеля. Из этих двух сплавов получили третий сплав массой \(140\) кг, содержащий \(30 \%\) никеля. На сколько килограммов масса первого сплава была меньше массы второго?
		\item Смешав \(30\)-процентный и \(60\)-процентный растворы кислоты и добавив \(10\) кг чистой воды, получили \(36\)-процентный раствор кислоты. Если бы вместо \(10\) кг воды добавили \(10\) кг \(50\)-процентного раствора той же кислоты, то получили бы \(41\)-процентный раствор кислоты. Сколько килограммов \(30\)-процентного раствора использовали для получения смеси?
		\item Решите неравенства:
		\begin{tasks}
			\task \( x \cdot \sqrt{6+5x} \leq 0 \)
			\task \( \sqrt{\dfrac{5x-3}{x-3}} \leq 3 \)
			\task \( \dfrac{\sqrt{-72-17x}}{-x} \leq 1 \)
		\end{tasks}
	\end{listofex}
\end{homework}
%
%===============>>  Домашняя работа 2  <<===============
%
%\begin{homework}[number=2]
%	\begin{listofex}
	%		\item Пусто
	%	\end{listofex}
%\end{homework}
%
%===============>>  Домашняя работа 3  <<===============
%
%\begin{homework}[number=3]
%	\begin{listofex}
	%		\item Пусто
	%	\end{listofex}
%\end{homework}
%
%===============>>  Проверочная работа  <<===============
%
%\begin{exam}
%	\begin{listofex}
	%		\item Пусто
	%	\end{listofex}
%\end{exam}