%Группа 5-1 Модуль 1 Занятие №1
\begin{listofex}
	\item Вычислить:
	\begin{enumcols}[itemcolumns=2]
		\item \exercise{4161}
		\item \exercise{4162}
		\item \exercise{4163}
		\item \exercise{4164}
	\end{enumcols}
	\item \exercise{4165}
\end{listofex}
\newpage
\title{Занятие №2}
\begin{listofex}
	\item \exercise{4166}
	\item \exercise{4167}
	\item \exercise{4168}
	\item \exercise{4169}
	\item \exercise{4170}
	\item \exercise{4171}
	\item \exercise{4172}
	\item \exercise{4173}
\end{listofex}
\newpage
\title{Домашняя работа №1}
\begin{listofex}
	\item \exercise{4174} 
	\item \exercise{4175}
	\item \exercise{4176}
	\item \exercise{4177}
	\item \exercise{4178}
\end{listofex}
\newpage
\title{Занятие №3}
\begin{listofex}
	\item Вычислите, используя распределительный закон:
	\begin{enumcols}[itemcolumns=2]
		\item \( 7\cdot13-7\cdot2 \)
		\item \( 37\cdot12+37\cdot88 \)
		\item \( 101\cdot17-17 \)
		\item \( 33\cdot11+11 \)
	\end{enumcols}
	\item Перепишите заполняя пропуски:
	\begin{enumcols}[itemcolumns=2]
		\item \( {\dots}\cdot(16+14)=7\cdot16+7\cdot14 \)
		\item \( 45\cdot({\dots}-{\dots})=45\cdot15-45\cdot13 \)
		\item \( 14\cdot(15+3)=14\cdot{\dots}+{\dots}\cdot3 \)
		\item \( 7\cdot({\dots}+14)=14\cdot{\dots}+{\dots}\cdot5 \)
	\end{enumcols}
	\item Вынести общий множитель за скобки и вычислить:
	\begin{enumcols}[itemcolumns=2]
		\item \( 61\cdot21+39\cdot21 \)
		\item \( 123\cdot11-22\cdot11 \)
		\item \( 37\cdot59+37\cdot41+63\cdot59+41\cdot63 \)
		\item \( 999\cdot55+55+257\cdot43+43\cdot43 \)
	\end{enumcols}
	\item В магазине «Спортмастер» цена футбольного мяча равна \( a \) рублей, а цена баскетбольного
	мяча \( b \) рублей. Какой смысл имеют следующие выражения:
	\begin{enumcols}[itemcolumns=5]
		\item \( a+b \)
		\item \( a-b \)
		\item \( 2\cdot a + 3\cdot b \)
		\item \( 7\cdot a - 2\cdot b \)
		\item \( 5000 - (a+b) \)
	\end{enumcols}
	\item В двух комнатах было \( 45 \) человек. Из первой вышли \( 9 \), а из второй --- \( 14 \), и людей в комнатах стало поровну. Сколько человек было в комнатах сначала?
	\item Сошлись два пастуха Иван и Пётр. Иван говорит Петру: «Отдай мне одну овцу,
	тогда у меня будет ровно вдвое больше овец, чем у тебя.» А Пётр ему отвечает: «Нет!
	Лучше ты мне отдай одну овцу, тогда у нас овец будет поровну». Сколько же было овец у
	каждого?
	\item Зная, что \( 25\cdot4=100 \), вычислите устно:
	\begin{enumcols}[itemcolumns=4]
		\item \( 16\cdot25 \)
		\item \( 25\cdot80 \)
		\item \( 52\cdot25 \)
		\item \( 25\cdot844 \)
	\end{enumcols}
	\item Зная, что \( 125\cdot8=1000 \), вычислите устно:
	\begin{enumcols}[itemcolumns=4]
		\item \( 125\cdot24 \)
		\item \( 125\cdot80 \)
		\item \( 64\cdot125 \)
		\item \( 248\cdot125 \)
	\end{enumcols}
\end{listofex}
\newpage
\title{Занятие №4}
\begin{listofex}
	\item Вычислите, используя распределительный закон:
	\begin{enumcols}[itemcolumns=2]
		\item \( 18\cdot9-18\cdot7 \)
		\item \( 37\cdot12-37\cdot2 \)
		\item \( 99\cdot15+15 \)
		\item \( 201\cdot44-44 \)
	\end{enumcols}
	\item Перепишите заполняя пропуски:
	\begin{enumcols}[itemcolumns=2]
		\item \( {\dots}\cdot(13-2)=8\cdot2-8\cdot2 \)
		\item \( 16\cdot({\dots}-\dots)=16\cdot3-16 \)
		\item \( 2\cdot(6+17)=2\cdot{\dots}+{\dots}\cdot2 \)
		\item \( 7\cdot({\dots}+1)=15\cdot{\dots}+{\dots} \)
	\end{enumcols}
	\item Вынести общий множитель за скобки и вычислить:
	\begin{enumcols}[itemcolumns=3]
		\item \( 47\cdot42+42\cdot153 \)
		\item \( 35\cdot36-35\cdot34 \)
		\item \( 7\cdot55+7\cdot45+3\cdot45+3\cdot55 \)
	\end{enumcols}
	\item В двух комнатах было \( 64 \) человек. Из первой вышли \( 12 \), а из второй --- \( 15 \), и людей в комнатах стало поровну. Сколько человек было в комнатах сначала?
	\item Сошлись два пастуха Иван и Пётр. Иван говорит Петру: «Отдай мне две овцы,
	тогда у меня будет ровно вдвое больше овец, чем у тебя.» А Пётр ему отвечает: «Нет!
	Лучше ты мне отдай две овцы, тогда у нас овец будет поровну». Сколько же было овец у
	каждого?
	\item Придумайте (и запишите в тетради!) задачи, математической моделью которых могут
	являться следующие числовые и буквенные выражения:
	\begin{enumcols}[itemcolumns=3]
		\item \( 5\cdot15-2 \)
		\item \( 4000:2+2000:5 \)
		\item \( a+(3+b):c \)
	\end{enumcols}
	\item Зная, что \( 25\cdot4=100 \), вычислите устно:
	\begin{enumcols}[itemcolumns=4]
		\item \( 8\cdot25 \)
		\item \( 25\cdot40 \)
		\item \( 64\cdot25 \)
		\item \( 25\cdot420 \)
	\end{enumcols}
	\item Зная, что \( 125\cdot8=1000 \), вычислите устно:
	\begin{enumcols}[itemcolumns=4]
		\item \( 125\cdot16 \)
		\item \( 125\cdot800 \)
		\item \( 160\cdot125 \)
		\item \( 88\cdot125 \)
	\end{enumcols}
\end{listofex}
\newpage
\title{Домашняя работа №2}
\begin{listofex}
	\item Вычислите, используя распределительный закон:
	\begin{enumcols}[itemcolumns=2]
		\item \( 5\cdot23-5\cdot8 \)
		\item \( 54\cdot36-54\cdot6 \)
		\item \( 199\cdot87+87 \)
		\item \( 501\cdot70-70 \)
	\end{enumcols}
	\item Перепишите заполняя пропуски:
	\begin{enumcols}[itemcolumns=2]
		\item \( {\dots}\cdot(27+3)=4\cdot27-4\cdot3 \)
		\item \( 11\cdot({\dots}+{\dots})=11\cdot13+15 \)
		\item \( 33\cdot(4+11)=33\cdot{\dots}+33\cdot{\dots} \)
		\item \( 12\cdot({\dots}-1)=10\cdot{\dots}-{\dots} \)
	\end{enumcols}
	\item Вынести общий множитель за скобки и вычислить:
	\begin{enumcols}[itemcolumns=3]
		\item \( 51\cdot43+12\cdot43 \)
		\item \( 51\cdot81-39\cdot81 \)
		\item \( 8\cdot2+2\cdot92+8\cdot98+2\cdot8 \)
	\end{enumcols}
	\item Вычислите рациональным образом:
	\begin{enumcols}[itemcolumns=2]
		\item \( (5486+3578)+1422 \)
		\item \( (357+768+589)+(332+211+643) \)
	\end{enumcols}
	\item У Максима и Кости коллекции редких монет. Максим говорит Косте: «Отдай мне три монеты, тогда у меня будет в три раза больше
	монет, чем у тебя.» А Костя отвечает: «Нет! Лучше отдай ты мне три монеты, тогда у нас будет монет поровну. Сколько монет у каждого?
	\item Придумайте (и запишите в тетради!) задачи, математической моделью которых могут
	являться следующие числовые и буквенные выражения:
	\begin{enumcols}[itemcolumns=3]
		\item \( 7\cdot18+23 \)
		\item \( 2500 - 3\cdot x \)
		\item \( (a+b):7+c \)
	\end{enumcols}
\end{listofex}
\newpage
\title{Занятие №5}
\begin{listofex}
	\item Сформулировать признаки делимости на \( 2 \); \( 3 \); \( 5 \); \( 9 \); \( 10 \).
	\item Воспользуйтесь признаками делимости из предыдущего задания и определите, на что делятся данные числа:
	\begin{enumcols}[itemcolumns=5]
		\item \( 368 \)
		\item \( 585 \)
		\item \( 2450 \)
		\item \( 12321 \)
		\item \( 303030 \)
	\end{enumcols}
	\item \textbf{Свойства делимости:}
	\begin{enumcols}[itemcolumns=1]
		\item Если один из множителей делится на некоторое число, то и произведение делится на это число.
		\item Если первое число делится на второе, а второе делится на третье, то первое число делится на третье.
		\item Если каждое из двух чисел делится на некоторое число, то их сумма или разность делятся на это число.
		\item Если одно из двух чисел делится на некоторое число, а другое на него не делится, то их сумма или разность не делится на это число.
	\end{enumcols}
	\item Объясните, не производя вычислений, почему следующие произведения делятся на \( 12 \)? Каким свойством вы в это случае пользуетесь?
	\begin{enumcols}[itemcolumns=4]
		\item \( 12\cdot47 \)
		\item \( 24\cdot17 \)
		\item \( 120\cdot51 \)
		\item \( 27\cdot8 \)
	\end{enumcols}
	\item Запишите чиста \( 24,\;42,\;36,\;72,\;75 \) в виде произведения и покажите, что
	\begin{enumcols}[itemcolumns=3]
		\item \( 24 \) делится на \( 2 \)
		\item \( 36 \) делится на \( 6 \)
		\item \( 75 \) делится на \( 5 \)
		\item \( 42 \) делится на \( 21 \)
		\item \( 72 \) делится на \( 9 \)
		\item \( 75 \) делится на \( 25 \)
	\end{enumcols}
	\item Объясните, почему:
	\begin{enumcols}[itemcolumns=2]
		\item сумма \( 45+36 \) делится на \( 9 \)
		\item сумма \( 99+88 \) делится на \( 11 \)
		\item сумма \( 13\cdot2+13\cdot7\) делится на \( 13 \)
	\end{enumcols}
	\item Доказать, что:
	\begin{enumcols}[itemcolumns=1]
		\item произведение четного числа и любого натурального числа является четным числом;
		\item сумма двух четных чисел является четным числом;
		\item сумма двух нечетных чисел является четным числом;
		\item сумма четного и нечетного числа является нечетным числом.
	\end{enumcols}
	\item Какую цифру нужно поставить вместо звездочки, чтобы полученное число:
	\begin{enumcols}[itemcolumns=3]
		\item \( 2\:* \) делилось на \( 2 \);
		\item \( 43\:* \) делилось на \( 3 \);
		\item \( 4\:* \) делилось на \( 9 \);
		\item \( 23\:* \) делилось на \( 10 \);
		\item \( 123\:* \) делилось на \( 5 \);
		\item \( 24*0 \) делилось на \( 9 \);
		\item \( 2*22 \) делилось на \( 9 \);
		\item \( 1*4\:* \) делилось на \( 2 \) и \( 3 \);
		\item \( 4*5\:* \) делилось на \( 9 \) и \( 5 \).
	\end{enumcols}
	%\item Что такое простые числа? Что такое составные числа?
\end{listofex}
\newpage
\title{Занятие №6}
\begin{listofex}
	\item 1
\end{listofex}
\newpage
\title{Домашняя работа №3}
\begin{listofex}
	\item 1
\end{listofex}
%\newpage
%\title{Занятие №7}
%\begin{listofex}
%	\item 1
%\end{listofex}
%\newpage
%\title{Занятие №8}
%\begin{listofex}
%	\item 1
%\end{listofex}
%\newpage
%\title{Домашняя работа №4}
%\begin{listofex}
%	\item 1
%\end{listofex}
%\newpage
%\title{Проверочная работа}
%\begin{listofex}
%	\item 1
%\end{listofex}