%Группа 5-1 Модуль 1 Занятие №1
\begin{listofex}
	\item Вычислить:
	\begin{enumcols}[itemcolumns=2]
		\item \( 25\cdot(28\cdot105+7236:18)-(4247-1823):6\cdot25 \)
		\item \( ((451-17\cdot3)\cdot3-200):500+46\cdot60 \)
		\item \( 15+(12322:(24+37)-12\cdot15):(35\cdot2-59) \)
		\item \( 3124:(3\cdot504-4\cdot307)+10403:101 \)
	\end{enumcols}
	\item Лёва с Васей решили купить футбольный мяч. У Лёвы не хватило \( 200 \) рублей, чтобы его
	купить, а у Васи \( 300 \) рублей. Тогда они сложили свои деньги и купили мяч,
	причём \( 600 \) рублей у них осталось. Сколько стоил футбольный мяч?
\end{listofex}
\newpage
\title{Занятие №2}
\begin{listofex}
	\item До отхода поезда остаётся \( 2 \) минуты. Расстояние до вокзала \( 2 \) км. Если максимальная скорость бега пассажира \( 30 \) км в час, то можно ли успеть на поезд?
	\item Чтобы подняться с первого этажа на третий этаж дома, надо пройти \( 52 \) ступеньки.
	Сколько ступенек надо пройти, чтобы подняться с первого этажа на шестой этаж
	этого же дома?
	\item Решить буквенное выражение \( (a-b)\cdot(b+c) \), если \( a=247;\; b=189;\; c=127 \).
	\item Решить буквенное выражение \( a^2+2\cdot a\cdot b + b^2 \), если \( a=217 \) и \( b=83 \).
	\item В книжном шкафу три полки. На первой стоит 30 книг, на второй – на 2 книги больше,
	чем на первой, а на третьей – в два раза меньше, чем на первой и второй полках вместе.
	Сколько всего книг в шкафу?
	\item В книжном шкафу три полки. На первой стоит a книг, на
	второй – на b книги больше, чем на первой, а на третьей – в два раза меньше, чем на первой и
	второй полках вместе. Сколько всего книг в шкафу?
	\item Заполнить цепочку:
	\begin{center}
		\begin{tikzpicture}[-latex ,auto, on grid ,
			semithick, state/.style ={ rectangle, draw , minimum width =1 cm, minimum height =1.2 cm}, pil/.style={ ->, thick, shorten <=2pt, shorten >=2pt,}]
			\node[state] (A1) at (0,0) {$90$};
			\node[state] (A2) at (2.5,0) {$\phantom{10}$};
			\node[state] (A3) at (5,0) {$\phantom{10}$};
			\node[state] (A4) at (7.5,0) {$\phantom{10}$};
			\node[state] (A5) at (10,0) {$\phantom{10}$};
			\node[state] (A6) at (12.5,0) {$\phantom{10}$};
			\node[state] (A7) at (15,0) {$\phantom{10}$};
			\node[state] (A8) at (15,-3) {$\phantom{10}$};
			\node[state] (A9) at (12.5,-3) {$\phantom{10}$};
			\node[state] (A10) at (10,-3) {$\phantom{10}$};
			\node[state] (A11) at (7.5,-3) {$\phantom{10}$};
			\node[state] (A12) at (5,-3) {$\phantom{10}$};
			\node[state] (A13) at (2.5,-3) {$\phantom{10}$};
			\node[state] (A14) at (0,-3) {$\phantom{10}$};
			\path (A1) edge [bend left =25] node[above] {$-45$} (A2);
			\path (A2) edge [bend left =25] node[above] {$:15$} (A3);
			\path (A3) edge [bend left =25] node[above] {$+13$} (A4);
			\path (A4) edge [bend left =25] node[above] {$\times\;5$} (A5);
			\path (A5) edge [bend left =25] node[above] {$:20$} (A6);
			\path (A6) edge [bend left =25] node[above] {$+26$} (A7);
			\path (A7) edge node[right] {$\times\;3$} (A8);
			\path (A8) edge [bend left =25] node[below] {$\times\;16$} (A9);
			\path (A9) edge [bend left =25] node[below] {$+560$} (A10);
			\path (A10) edge [bend left =25] node[below] {$\times\;7$} (A11);
			\path (A11) edge [bend left =25] node[below] {$:140$} (A12);
			\path (A12) edge [bend left =25] node[below] {$+180$} (A13);
			\path (A13) edge [bend left =25] node[below] {$:7$} (A14);
		\end{tikzpicture}
	\end{center}
\end{listofex}
\newpage
\title{Домашняя работа №1}
\begin{listofex}
	\item 1
\end{listofex}
\newpage
\title{Занятие №3}
\begin{listofex}
	\item 1
\end{listofex}
\newpage
\title{Занятие №4}
\begin{listofex}
	\item 1
\end{listofex}
\newpage
\title{Домашняя работа №2}
\begin{listofex}
	\item 1
\end{listofex}
\newpage
\title{Занятие №5}
\begin{listofex}
	\item 1
\end{listofex}
\newpage
\title{Занятие №6}
\begin{listofex}
	\item 1
\end{listofex}
\newpage
\title{Домашняя работа №3}
\begin{listofex}
	\item 1
\end{listofex}
\newpage
\title{Занятие №7}
\begin{listofex}
	\item 1
\end{listofex}
\newpage
\title{Занятие №8}
\begin{listofex}
	\item 1
\end{listofex}
\newpage
\title{Домашняя работа №4}
\begin{listofex}
	\item 1
\end{listofex}
\newpage
\title{Проверочная работа}
\begin{listofex}
	\item 1
\end{listofex}