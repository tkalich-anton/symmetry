\documentclass[12pt, a4paper]{article}
\usepackage{cmap} % Улучшенный поиск русских слов в полученном pdf-файле
\usepackage[T2A]{fontenc} % Поддержка русских букв
\usepackage[utf8]{inputenc} % Кодировка utf8
\usepackage[english, russian]{babel} % Языки: русский, английский
\usepackage{enumitem}
\usepackage{pscyr} % Нормальные шрифты
\usepackage{amsmath}
\usepackage{amsthm}
\usepackage{amssymb}
\usepackage{scrextend}
\usepackage{titling}
\usepackage{indentfirst}
\usepackage{cancel}
\usepackage{soulutf8}
\usepackage{wrapfig}
\usepackage{gensymb}
\usepackage[dvipsnames,table,xcdraw]{xcolor}
\usepackage{tikz}

%Русские символы в списке
\makeatletter
\AddEnumerateCounter{\asbuk}{\russian@alph}{щ}
\makeatother

%Дублирование знаков при переносе
\newcommand*{\hm}[1]{#1\nobreak\discretionary{}%
	{\hbox{$\mathsurround=0pt #1$}}{}}

\usepackage{graphicx}
\graphicspath{{pic/}}
\DeclareGraphicsExtensions{.pdf,.png,.jpg}

%Изменеие параметров листа
\usepackage[left=15mm,right=15mm,
top=2cm,bottom=2cm,bindingoffset=0cm]{geometry}

\usepackage{fancyhdr}
\pagestyle{fancy}
\usepackage{multicol}

\setlength\parindent{1,5em}
\usepackage{indentfirst}
\begin{document}
	
\chead{Сборник заданий по алгебре}
\rhead{Школа <<Симметрия>>}
Шахмейстер. Дроби.
\begin{enumerate}
	\item Тренировочная карточка 2
	\begin{enumerate}[label=\asbuk*)]
		\item 
		\item 
		\item 
		\item 
		\item 
		\item 
		\item 
		\item 
		\item 
		\item 
	\end{enumerate}
	\item Тренировочная карточка 3
		\begin{enumerate}[label=\asbuk*)]
		\item 
		\item 
		\item 
		\item 
		\item 
		\item 
		\item 
		\item 
		\item 
		\item 
	\end{enumerate}
	\item Тренировочная карточка 4
		\begin{enumerate}[label=\asbuk*)]
			\item 
			\item 
			\item 	
			\item 
			\item 
			\item 
			\item 
			\item 
			\item 
			\item 
		\end{enumerate}
	\item Тренировочная карточка 5
		\begin{enumerate}[label=\asbuk*)]
			\item 
			\item 
			\item 
			\item 
			\item 
			\item 
			\item 
			\item 
			\item 
			\item 
		\end{enumerate}
		\item Тренировочная карточка 6
		\begin{enumerate}[label=\asbuk*)]
			\item 
			\item 
			\item 
			\item
			\item 
			\item 
			\item 
			\item
			\item 
			\item 
		\end{enumerate}
	\item Тренировочная карточка 7
	\begin{enumerate}[label=\asbuk*)]
		\item 
		\item 
		\item
		\item 
		\item 
		\item 
		\item 
		\item 
		\item 
		\item 
	\end{enumerate}
	\item Тренировочная карточка 8
	\begin{enumerate}[label=\asbuk*)]
		\item 
		\item 
		\item 
		\item 
		\item 
		\item 
		\item 
		\item 
		\item 
		\item 
	\end{enumerate}
	\item Зачётная карточка 1
	\begin{enumerate}[label=\asbuk*)]
		\item 
		\item 
		\item 
		\item 
		\item 
		\item 
		\item 
		\item 
		\item 
		\item 
	\end{enumerate}
	\item Зачётная карточка 2
	\begin{enumerate}[label=\asbuk*)]
		\item 
		\item 
		\item 
		\item 
		\item 
		\item 
		\item 
		\item 
		\item 
		\item 
	\end{enumerate}
	\item Зачётная карточка 3
	\begin{enumerate}[label=\asbuk*)]
		\item 
		\item 
		\item 
		\item 
		\item 
		\item 
		\item 
		\item 
		\item 
		\item 
	\end{enumerate}
	\item Зачётная карточка 4
	\begin{enumerate}[label=\asbuk*)]
		\item 
		\item 
		\item 
		\item 
		\item 
		\item
		\item 
		\item 
		\item 
		\item 
	\end{enumerate}
	\item Зачётная карточка 5
	\begin{enumerate}[label=\asbuk*)]
		\item 
		\item 
		\item 
		\item 
		\item 
		\item 
		\item 
		\item 
		\item 
		\item 
	\end{enumerate}
	\item Зачётная карточка 6
	\begin{enumerate}[label=\asbuk*)]
		\item 
		\item 
		\item 
		\item 
		\item 
		\item 
		\item 
		\item 
		\item 
		\item 
	\end{enumerate}
	\item Зачётная карточка 7
	\begin{enumerate}[label=\asbuk*)]
		\item 
		\item 
		\item 
		\item 
		\item 
		\item 
		\item 
		\item 
		\item 
		\item 
		\end{enumerate}
	\item Зачётная карточка 8
	\begin{enumerate}[label=\asbuk*)]
		\item 
		\item 
		\item 
		\item 
		\item 
		\item 
		\item 
		\item 
		\item
		\item
	\end{enumerate}
\end{enumerate}
\end{document}