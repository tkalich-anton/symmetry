\documentclass[12pt, a4paper]{article}
\usepackage{cmap} % Улучшенный поиск русских слов в полученном pdf-файле
\usepackage[T2A]{fontenc} % Поддержка русских букв
\usepackage[utf8]{inputenc} % Кодировка utf8
\usepackage[english, russian]{babel} % Языки: русский, английский
\usepackage{enumitem}
\usepackage{pscyr} % Нормальные шрифты
\usepackage{amsmath}
\usepackage{amsthm}
\usepackage{amssymb}
\usepackage{scrextend}
\usepackage{titling}
\usepackage{indentfirst}
\usepackage{cancel}
\usepackage{soulutf8}
\usepackage{wrapfig}
\usepackage{gensymb}
\usepackage[dvipsnames,table,xcdraw]{xcolor}
\usepackage{tikz}

%Русские символы в списке
\makeatletter
\AddEnumerateCounter{\asbuk}{\russian@alph}{щ}
\makeatother

%Дублирование знаков при переносе
\newcommand*{\hm}[1]{#1\nobreak\discretionary{}%
	{\hbox{$\mathsurround=0pt #1$}}{}}

\usepackage{graphicx}
\graphicspath{{pic/}}
\DeclareGraphicsExtensions{.pdf,.png,.jpg}

%Изменеие параметров листа
\usepackage[left=15mm,right=15mm,
top=2cm,bottom=2cm,bindingoffset=0cm]{geometry}

\usepackage{fancyhdr}
\pagestyle{fancy}
\usepackage{multicol}

\setlength\parindent{1,5em}
\usepackage{indentfirst}
\begin{document}
	
	\chead{Сборник заданий по алгебре}
	\rhead{Школа <<Симметрия>>}
	Шахмейстер. Степени с натуральными показателями.
\begin{enumerate}
	\item Вычилсите:
		\begin{enumerate}[label=\textbf{\arabic*)}]
			\item \( \dfrac{18^2\cdot12^3\cdot8^2}{24^3\cdot6^2} \)
			\item \( \dfrac{72^3\cdot48^3}{36^5\cdot16^3} \) 
			\item \( \dfrac{(9\cdot16^{n-1}+16^n)^2}{(4^{n-1}+4^{n-2})^4} \)
			\item \( \left( \dfrac{7^4}{15^2} \right)^3\cdot\left( \dfrac{5}{7} \right)^6\cdot\left( \dfrac{3}{7} \right)^5 \)
			\item \( \dfrac{(4\cdot3^{22}+7\cdot3^{21})\cdot57}{(19\cdot27^4)^2} \)
			\item \( \dfrac{5(3\cdot7^{15}-19\cdot7^{14})}{7^{16}+3\cdot7^{15}} \)
			\item \( \dfrac{6\cdot2^8-9\cdot2^{10}+3\cdot2^{12}}{4\cdot2^{10}+4\cdot2^{12}-8\cdot2^{11}} \)
			\item \( \dfrac{3^{n+2}-2\cdot3^n}{3^{n-1}}-\dfrac{36^{n+1}}{6^{2n-1}} \)
			\item Упростите \( \dfrac{a^{3n}-a^{n-2}}{a^{2n-2}-a^{n-3}} \)
			\item Сравните 
			\begin{enumerate}[label=\textbf{\asbuk*)}]
				\item \( 75^{10} \) и \( 15^{15} \)
				\item \( 200^6 \) и \( 14^{12} \)
			\end{enumerate}
		\end{enumerate}
\end{enumerate}
\end{document}