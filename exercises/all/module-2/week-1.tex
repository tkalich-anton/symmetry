\documentclass[12pt, a4paper]{article}
\usepackage{cmap} % Улучшенный поиск русских слов в полученном pdf-файле
\usepackage[T2A]{fontenc} % Поддержка русских букв
\usepackage[utf8]{inputenc} % Кодировка utf8
\usepackage[english, russian]{babel} % Языки: русский, английский
\usepackage{enumitem}
\usepackage{pscyr} % Нормальные шрифты
\usepackage{amsmath}
\usepackage{amsthm}
\usepackage{amssymb}
\usepackage{scrextend}
\usepackage{titling}
\usepackage{indentfirst}
\usepackage{cancel}
\usepackage{soulutf8}
\usepackage{wrapfig}
\usepackage{gensymb}
\usepackage[dvipsnames,table,xcdraw]{xcolor}
\usepackage{tikz}

%Русские символы в списке
\makeatletter
\AddEnumerateCounter{\asbuk}{\russian@alph}{щ}
\makeatother

%Дублирование знаков при переносе
\newcommand*{\hm}[1]{#1\nobreak\discretionary{}%
	{\hbox{$\mathsurround=0pt #1$}}{}}

\usepackage{graphicx}
\graphicspath{{pic/}}
\DeclareGraphicsExtensions{.pdf,.png,.jpg}

%Изменеие параметров листа
\usepackage[left=15mm,right=15mm,
top=2cm,bottom=2cm,bindingoffset=0cm]{geometry}

\usepackage{fancyhdr}
\pagestyle{fancy}
\usepackage{multicol}

\setlength\parindent{1,5em}
\usepackage{indentfirst}
\begin{document}
	
	\chead{Модуль 2}
	\rhead{Школа <<Симметрия>>}
	\begin{enumerate}
		\item При температуре \( 0^{\circ}  \) рельс имеет длину \(  l_0=10\)м. При возрастании температуры происходит тепловое расширение рельса, и его длина, выраженная в метрах, меняется по закону \( l(t^{\circ})=l_0(1+\alpha\cdot t^{\circ}) \), где \( \alpha = 1,2\cdot10^{-5}(^{\circ} C)^{-1} \) – коэффициент теплового расширения, \( t^{\circ} \) – температура (в градусах Цельсия). При какой температуре рельс удлинится на \( 3 \)мм? Ответ выразите в градусах Цельсия.
		\item При температуре \( 0^{\circ}  \) рельс имеет длину \( l_0=12,5 \)м. При возрастании температуры происходит тепловое расширение рельса, и его длина, выраженная в метрах, меняется по закону \( l(t^{\circ})=l_0(1+\alpha\cdot t^{\circ}) \), где  \( \alpha=1,2\cdot 10^{-5}(^{\circ}C)^{-1} \) – коэффициент теплового расширения, \( t^{\circ} \) – температура (в градусах Цельсия). При какой температуре рельс удлинится на \( 6 \) мм? Ответ выразите в градусах Цельсия.
		\item Некоторая компания продает свою продукцию по цене\(  p=500  \)руб. за единицу, переменные затраты на производство одной единицы продукции составляют  \( v=300 \) руб., постоянные расходы предприятия \( f=700000 \) руб. в месяц. Месячная операционная прибыль предприятия (в рублях) вычисляется по формуле  \( \pi(q)=q(p-v)-f \). Определите месячный объeм производства\(  q \) (единиц продукции), при котором месячная операционная прибыль предприятия будет равна \( 300000 \) руб.
		\item Некоторая компания продаёт свою продукцию по цене \( p = 600 \) руб. за единицу, переменные затраты на производство одной единицы продукции составляют \( v = 300 \) руб., постоянные расходы предприятия \( f = 700 000 \) руб. в месяц. Месячная операционная прибыль предприятия (в рублях) вычисляется по формуле \(g(q)=q(p-v)-f\). Определите месячный объём производства \( q \) (единиц продукции), при котором месячная операционная прибыль предприятия будет равна \( 500 000 \) руб.
		\item Зависимость объёма спроса \( q \) (единиц в месяц) на продукцию предприятия – монополиста от цены \( p \) (тыс. руб.) задаётся формулой \( q=100-10p \). Выручка предприятия за месяц \( r \) (в тыс. руб.) вычисляется по формуле \( r(p)=q\cdot p \). Определите наибольшую цену \( p \), при которой месячная выручка \( r(p) \) составит не менее \( 240 \) тыс. руб. Ответ приведите в тыс. руб.
		\item Зависимость объёма спроса \( q \) (единиц в месяц) на продукцию предприятия – монополиста от цены \( p \) (тыс. руб.) задаётся формулой \( q=85-5p \). Выручка предприятия за месяц \( r \) (в тыс. руб.) вычисляется по формуле \( r(p)=q\cdot p \). Определите наибольшую цену \( p \), при которой месячная выручка \( r(p) \) составит не менее \( 350 \) тыс. руб. Ответ приведите в тыс. руб.
		\item Для получения на экране увеличенного изображения лампочки в лаборатории используется собирающая линза с главным фокусным расстоянием\(  f = 30 \) см. Расстояние \( d_1  \)от линзы до лампочки может изменяться в пределах от \( 30 \) до \( 50 \) см, а расстояние \( d_2 \) от линзы до экрана – в пределах от \( 150 \) до \( 180 \) см. Изображение на экране будет четким, если выполнено соотношение \( \dfrac{1}{d_1}+\dfrac{1}{d_2}=\dfrac{1}{f} \). Укажите, на каком наименьшем расстоянии от линзы можно поместить лампочку, чтобы еe изображение на экрана было чётким. Ответ выразите в сантиметрах.
		\item Для получения на экране увеличенного изображения лампочки в лаборатории используется собирающая линза с главным фокусным расстоянием\(  f = 80 \) см. Расстояние \( d_1  \)от линзы до лампочки может изменяться в пределах от \( 33	0 \) до \( 350 \) см, а расстояние \( d_2 \) от линзы до экрана – в пределах от \( 80 \) до \( 105 \) см. Изображение на экране будет четким, если выполнено соотношение \( \dfrac{1}{d_1}+\dfrac{1}{d_2}=\dfrac{1}{f} \). Укажите, на каком наименьшем расстоянии от линзы можно поместить лампочку, чтобы еe изображение на экрана было чётким. Ответ выразите в сантиметрах.
		\item По закону Ома для полной цепи сила тока, измеряемая в амперах, равна \( I=\dfrac{\sigma}{R+r} \), где \(\sigma\) – ЭДС источника (в вольтах), \(r=2\) Ом – его внутреннее сопротивление, \(R\) – сопротивление цепи (в омах). При каком наименьшем сопротивлении цепи сила тока будет составлять не более \(40\% \) от силы тока короткого замыкания \( I_{кз} =\dfrac{\sigma}{r}\)? (Ответ выразите в омах).
		\item По закону Ома для полной цепи сила тока, измеряемая в амперах, равна \( I=\dfrac{\sigma}{R+r} \), где \(\sigma\) – ЭДС источника (в вольтах), \(r=1\) Ом – его внутреннее сопротивление, \(R\) – сопротивление цепи (в омах). При каком наименьшем сопротивлении цепи сила тока будет составлять не более \( 40\% \) от силы тока короткого замыкания \( I_{кз} =\dfrac{\sigma}{r}\)? (Ответ выразите в омах).
		\item Если достаточно быстро вращать ведeрко с водой на верeвке в вертикальной плоскости, то вода не будет выливаться. При вращении ведeрка сила давления воды на дно не остаeтся постоянной: она максимальна в нижней точке и минимальна в верхней. Вода не будет выливаться, если сила еe давления на дно будет положительной во всех точках траектории кроме верхней, где она может быть равной нулю. В верхней точке сила давления, выраженная в ньютонах, равна \( P=m\left( \dfrac{v^2}{L}-g \right) \), где \( m \) – масса воды в килограммах, \( v \) – скорость движения ведeрка в м/с, \( L \) – длина верeвки в метрах, \( g \) – ускорение свободного падения (считайте \( g=10 \) м/с\(^2\)). С какой наименьшей скоростью надо вращать ведeрко, чтобы вода не выливалась, если длина верeвки равна \( 40 \)см? Ответ выразите в м/с.
		\item Если достаточно быстро вращать ведeрко с водой на верeвке в вертикальной плоскости, то вода не будет выливаться. При вращении ведeрка сила давления воды на дно не остаeтся постоянной: она максимальна в нижней точке и минимальна в верхней. Вода не будет выливаться, если сила еe давления на дно будет положительной во всех точках траектории кроме верхней, где она может быть равной нулю. В верхней точке сила давления, выраженная в ньютонах, равна \( P=m\left( \dfrac{v^2}{L}-g \right) \), где \( m \) – масса воды в килограммах, \( v \) – скорость движения ведeрка в м/с, \( L \) – длина верeвки в метрах, \( g \) – ускорение свободного падения (считайте \( g=10 \) м/с\(^2\)). С какой наименьшей скоростью надо вращать ведeрко, чтобы вода не выливалась, если длина верeвки равна \( 160 \)см? Ответ выразите в м/с.
		\item Для определения эффективной температуры звёзд используют закон Стефана–Больцмана, согласно которому \( P=\sigma ST^4 \), где \( P \) – мощность излучения звезды (в ваттах),\( \sigma =5,7\cdot10^{-8} \dfrac{BT}{M^2\cdot K^4} \) – постоянная, \( S \) – площадь поверхности звезды (в квадратных метрах), а \( T \) – температура (в кельвинах). Известно, что площадь поверхности некоторой звезды равна \( \dfrac{1}{16}\cdot 10^{20} M	0^2\), а мощность её излучения равна \( 9,12\cdot 10^{25} \)Вт. Найдите температуру этой звезды в кельвинах.
		\item Для определения эффективной температуры звёзд используют закон Стефана–Больцмана, согласно которому \( P=\sigma ST^4 \), где \( P \) – мощность излучения звезды (в ваттах),\( \sigma =5,7\cdot10^{-8} \dfrac{BT}{M^2\cdot K^4} \) – постоянная, \( S \) – площадь поверхности звезды (в квадратных метрах), а \( T \) – температура (в кельвинах). Известно, что площадь поверхности некоторой звезды равна \( \dfrac{1}{125}\cdot 10^{20} M^2\), а мощность её излучения равна \( 2,85 \cdot 10^{25} \)Вт. Найдите температуру этой звезды в кельвинах.
		\item Амплитуда колебаний маятника зависит от частоты вынуждающей силы, определяемой по формуле \( A(\omega)=\dfrac{A_0\omega _p ^2}{|\omega _p ^2-\omega^2|} \), где \( \omega \) – частота вынуждающей силы (в \( c^{-1} \)), \( A_0 \) – постоянный параметр, \( \omega_p = 360c^{-1}\) – резонансная частота. Найдите максимальную частоту \( \omega  \), меньшую резонансной, для которой амплитуда колебаний превосходит величину\(  A_0  \) не более чем на \( 12,5\% \). Ответ выразите в \( c^{-1} \).
		\item Амплитуда колебаний маятника зависит от частоты вынуждающей силы, определяемой по формуле \( A(\omega)=\dfrac{A_0\omega _p ^2}{|\omega _p ^2-\omega^2|} \), где \( \omega \) – частота вынуждающей силы (в \( c^{-1} \)), \( A_0 \) – постоянный параметр, \( \omega_p = 360c^{-1}\) – резонансная частота. Найдите максимальную частоту \( \omega  \), меньшую резонансной, для которой амплитуда колебаний превосходит величину\(  A_0  \) не более чем на одну треть. Ответ выразите в \( c^{-1} \).
		\item Гоночный автомобиль разгоняется на прямолинейном участке шоссе с постоянным ускорением \( a \) км/ч\( ^2 \). Скорость \( v \)  в конце пути вычисляется по формуле \( v=\sqrt{2la} \), где \( l \) – пройденный автомобилем путь в км. Определите ускорение, с которым должен двигаться автомобиль, чтобы, проехав \( 250 \) метров, приобрести скорость \( 60 \)км/ч. Ответ выразите в км/ч\( ^2 \).
		\item Расстояние (в км) от наблюдателя, находящегося на небольшой высоте \( h \) километров над землeй, до наблюдаемой им линии горизонта вычисляется по формуле \( l=\sqrt{2Rh} \), где \( R = 6400 \) (км) – радиус Земли. С какой высоты горизонт виден на расстоянии \(4\) километра? Ответ выразите в километрах.
		\item Автомобиль массой \( m \) кг начинает тормозить и проходит до полной остановки путь \( S \) м. Сила трения \(  F \) (в \(  Н \)), масса автомобиля \( m \) (в кг), время \( t \) (в с) и пройденный путь \( S \) (в м) связаны соотношением \( F=\dfrac{2mS}{t^2} \). Определите, сколько секунд заняло торможение, если известно, что сила трения равна \( 2000\) Н , масса автомобиля – \( 1500 \) кг, путь – \( 600 \) м.
		\item Уравнение процесса, в котором участвовал газ, записывается в виде \( pV^a=const \), где \( p \) (Па) – давление газа, \( V \) – объeм газа в кубических метрах, \( a \) – положительная константа. При каком наименьшем значении константы \( a \) уменьшение в два раза объeма газа, участвующего в этом процессе, приводит к увеличению давления не менее, чем в \( 4 \) раза?
		\item Установка для демонстрации адиабатического сжатия представляет собой сосуд с поршнем, резко сжимающим газ. При этом объeм и давление связаны соотношением \( pV^{1,4}=const\), где \( p \) (атм.) – давление газа, \( V \) – объeм газа в литрах. Изначально объeм газа равен\(  1,6 \) л, а его давление равно одной атмосфере. В соответствии с техническими характеристиками поршень насоса выдерживает давление не более \( 128 \) атмосфер. Определите, до какого минимального объeма можно сжать газ. Ответ выразите в литрах.
		\item При адиабатическом процессе для идеального газа выполняется закон \( pV^k=1,25 \cdot 10^8 \) Па\( \cdot \)м\( ^4 \), где \( p \) – давление газа (в Па), \( V \) – объём газа (в м\( ^3 \)), \( k=\dfrac{4}{3}\). Найдите, какой объём \( V \) (в м\( ^3 \)) будет занимать газ при давлении \( p \), равном \( 2 \cdot 10^5 \) Па.
		\item При адиабатическом процессе для идеального газа выполняется закон \( pV^k=1,25 \cdot 10^7 \) Па\( \cdot \)м\( ^5 \), где \( p \) – давление газа (в Па), \( V \) – объём газа (в м\( ^3 \)), \( k=\dfrac{5}{3}\). Найдите, какой объём \( V \) (в м\( ^3 \)) будет занимать газ при давлении \( p \), равном \( 3 \cdot 10^5 \) Па.
		\item Для обогрева помещения, температура в котором поддерживается на уровне \( T_n=20^{\circ}C \), через радиатор отопления пропускают воду по проходящей через трубу воды \( m=0,3 \) кг/с. Проходя по трубе расстояние \( x \), вода охлаждается от начальной температуры \( T_b=60^{\circ }C \) до температуры \( T^{\circ} C \), причем \( x=\alpha \dfrac{cm}{\gamma}log_2\dfrac{T_b-T_n}{T-T_b}\), где \( c=4200\dfrac{Дж}{кг \cdot ^{\circ}C} \) – теплоемкость воды, \( \gamma=21\dfrac{Вт}{м \cdot ^{circ}C} \) – коэффициент теплообмена, а \( \alpha =0,7 \) – постоянная. Найдите, до какой температуры (в градусах Цельсия) охладится вода, если длина трубы радиатора равна \(  84 \) м.
		\item Для обогрева помещения, температура в котором поддерживается на уровне \( T_n=15^{\circ}C \), через радиатор отопления пропускают воду по проходящей через трубу воды \( m=0,6 \) кг/с. Проходя по трубе расстояние \( x \), вода охлаждается от начальной температуры \( T_b=91^{\circ }C \) до температуры \( T^{\circ} C \), причем \( x=\alpha \dfrac{cm}{\gamma}log_2\dfrac{T_b-T_n}{T-T_b}\), где \( c=4200\dfrac{Дж}{кг \cdot ^{\circ}C} \) – теплоемкость воды, \( \gamma=28\dfrac{Вт}{м \cdot ^{circ}C} \) – коэффициент теплообмена, а \( \alpha =0,8 \) – постоянная. Найдите, до какой температуры (в градусах Цельсия) охладится вода, если длина трубы радиатора равна \(  144 \) м.
		\item Мяч бросили под углом \( \alpha \) к плоской горизонтальной поверхности земли. Время полeта мяча (в секундах) определяется по формуле \( t=\dfrac{2V_0sin\alpha}{g} \). При каком значении угла \( \alpha \) (в градусах) время полeта составит \( 3 \) секунды, если мяч бросают с начальной скоростью \( v_0=30 \)м/с? Считайте, что ускорение свободного падения \( g=10 \) м/с\( ^2 \).
		\item Мяч бросили под углом \( \alpha \) к плоской горизонтальной поверхности земли. Время полeта мяча (в секундах) определяется по формуле \( t=\dfrac{2V_0sin\alpha}{g} \). При каком значении угла \( \alpha \) (в градусах) время полeта составит \( 2,2 \) секунды, если мяч бросают с начальной скоростью \( v_0=22 \)м/с? Считайте, что ускорение свободного падения \( g=10 \) м/с\( ^2 \).
		\item Небольшой мячик бросают под острым углом \( \alpha \) к плоской горизонтальной поверхности земли. Расстояние, которое пролетает мячик, вычисляется по формуле\( L=\dfrac{v_0 ^2}{g} sin2\alpha\) (м), где \( v_0=20 \)м/с – начальная скорость мячика, а \(g\) – ускорение свободного падения (считайте \(  g=10 \) м/с\( ^2 \)). При каком наименьшем значении угла (в градусах) мячик перелетит реку шириной \( 20 \) м?
		\item Небольшой мячик бросают под острым углом \( \alpha \) к плоской горизонтальной поверхности земли. Расстояние, которое пролетает мячик, вычисляется по формуле\( L=\dfrac{v_0 ^2}{g} sin2\alpha\) (м), где \( v_0=14 \)м/с – начальная скорость мячика, а \(g\) – ускорение свободного падения (считайте \(  g=10 \) м/с\( ^2 \)). При каком наименьшем значении угла (в градусах) мячик перелетит реку шириной \( 9,8 \) м?
		\item Плоский замкнутый контур площадью \( S = 0,5 \) м\( ^2 \) находится в магнитном поле, индукция которого равномерно возрастает. При этом согласно закону электромагнитной индукции Фарадея в контуре появляется ЭДС индукции, значение которой, выраженное в вольтах, определяется формулой \( \epsilon _i = aS\cos \alpha \), где  \( \alpha \) – острый угол между направлением магнитного поля и перпендикуляром к контуру, \(  a=4\cdot10^{-4}\)Тл/с – постоянная, \( S \) – площадь замкнутого контура, находящегося в магнитном поле (в м\( ^2 \)). При каком минимальном угле\( \alpha \) (в градусах) ЭДС индукции не будет превышать \( 10^{-4} \) В?
		\item Плоский замкнутый контур площадью \( S = 2,5 \) м\( ^2 \) находится в магнитном поле, индукция которого равномерно возрастает. При этом согласно закону электромагнитной индукции Фарадея в контуре появляется ЭДС индукции, значение которой, выраженное в вольтах, определяется формулой \( \epsilon _i = aS\cos \alpha \), где  \( \alpha \) – острый угол между направлением магнитного поля и перпендикуляром к контуру, \(  a=4\cdot10^{-4}\)Тл/с – постоянная, \( S \) – площадь замкнутого контура, находящегося в магнитном поле (в м\( ^2 \)). При каком минимальном угле\( \alpha \) (в градусах) ЭДС индукции не будет превышать \( 5\cdot 10^{-4} \)В?
		\item Рейтинг \( R \) интернет-магазина вычисляется по формуле \\
		\( R=r_{pok}-\dfrac{r_{pok}-r_{eks}}{(K+1)^m} \),\\
		где \( m=\dfrac{0,02K}{r_{pok}+0,1} \)б \( r_{pok} \) – средняя оценка магазина покупателями, \( r_{eks} \) – оценка магазина, данная экспертами, \( K \) – число покупателей, оценивших магазин. Найдите рейтинг интернет-магазина, если число покупателей, оценивших магазин, равно \( 26 \), их средняя оценка равна \(0,68\), а оценка экспертов равна \(0,23\).
		\item Рейтинг \( R \) интернет-магазина вычисляется по формуле \( R=r_{pok}-\dfrac{r_{pok}-r_{eks}}{(K+1)^m} \), где \( m=\dfrac{0,02K}{r_{pok}+0,1} \), \( r_{pok} \) – средняя оценка магазина покупателями,\( r_{eks} \) – оценка магазина, данная экспертами, \( K \) – число покупателей, оценивших магазин. Найдите рейтинг интернет-магазина, если число покупателей, оценивших магазин, равно\(  15 \), их средняя оценка равна \(  0,5 \), а оценка экспертов равна \( 0,42 \).
		\item Из пункта \( A \) в пункт \( B \) одновременно выехали два автомобиля. Первый проехал с постоянной скоростью весь путь. Второй проехал первую половину пути со скоростью \( 24 \)км/ч, а вторую половину пути – со скоростью, на \(16\) км/ч больше скорости первого, в результате чего прибыл в пункт \( B \) одновременно с первым автомобилем. Найдите скорость первого автомобиля. Ответ дайте в км/ч.
		\item Из пункта \( A \) в пункт \( B \) одновременно выехали два автомобиля. Первый проехал с постоянной скоростью весь путь. Второй проехал первую половину пути со скоростью \( 72 \)км/ч, а вторую половину пути – со скоростью, на \(10\) км/ч больше скорости первого, в результате чего прибыл в пункт \( B \) одновременно с первым автомобилем. Найдите скорость первого автомобиля. Ответ дайте в км/ч.
		\item Из пункта \( A \) в пункт \( B \), расстояние между которыми \(75\), км одновременно выехали автомобилист и велосипедист. Известно, что за час автомобилист проезжает на \(40\) км больше, чем велосипедист. Определите скорость велосипедиста, если известно, что он прибыл в пункт \(B\) на \(6\) часов позже автомобилиста. Ответ дайте в км/ч.
		\item Из пункта \( A \) в пункт \( B \), расстояние между которыми \(50\), км одновременно выехали автомобилист и велосипедист. Известно, что за час автомобилист проезжает на \(40\) км больше, чем велосипедист. Определите скорость велосипедиста, если известно, что он прибыл в пункт \(B\) на \(4\) часа позже автомобилиста. Ответ дайте в км/ч.
		\item Товарный поезд каждую минуту проезжает на \( 750 \) метров меньше, чем скорый, и на путь в \( 180 \) км тратит времени на \( 2 \) часа больше, чем скорый. Найдите скорость товарного поезда. Ответ дайте в км/ч.
		\item Товарный поезд каждую минуту проезжает на \( 450 \) метров меньше, чем скорый, и на путь в \( 240 \) км тратит времени на \( 2 \) часа больше, чем скорый. Найдите скорость товарного поезда. Ответ дайте в км/ч.
		\item Из городов \( A \) и \( B  \) навстречу друг другу выехали мотоциклист и велосипедист. Мотоциклист приехал в \(B\)на \(3\)часа раньше, чем велосипедист приехал в \( A \), а встретились они через \(48\)минут после выезда. Сколько часов затратил на путь из \( B  \) в \( A  \)велосипедист?
		\item Из городов \( A  \) и \( B  \) навстречу друг другу выехали мотоциклист и велосипедист. Мотоциклист приехал в \(В\)на \(1\) час раньше, чем велосипедист приехал в \( A \), а встретились они через \(40\) минут после выезда. Сколько часов затратил на путь из \(  B  \) в \( A  \) велосипедист?
		\item Автомобиль, движущийся с постоянной скоростью \( 70  \) км/ч по прямому шоссе, обгоняет другой автомобиль, движущийся в ту же сторону с постоянной скоростью \( 40  \) км/ч. Каким будет расстояние (в километрах) между этими автомобилями через \( 15  \) минут после обгона?
		\item По двум параллельным железнодорожным путям друг навстречу другу следуют скорый и пассажирский поезда, скорости которых равны соответственно \( 65  \) км/ч и \( 35  \) км/ч. Длина пассажирского поезда равна \( 700  \) метрам. Найдите длину скорого поезда, если время, за которое он прошел мимо пассажирского поезда, равно \( 36  \) секундам. Ответ дайте в метрах.
		\item По двум параллельным железнодорожным путям друг навстречу другу следуют скорый и пассажирский поезда, скорости которых равны соответственно \( 75  \) км/ч и \( 60  \) км/ч. Длина пассажирского поезда равна \( 400  \) метрам. Найдите длину скорого поезда, если время, за которое он прошел мимо пассажирского поезда, равно \( 16  \) секундам. Ответ дайте в метрах.
		\item Два мотоциклиста стартуют одновременно в одном направлении из двух диаметрально противоположных точек круговой трассы, длина которой равна \( 14  \) км. Через сколько минут мотоциклисты поравняются в первый раз, если скорость одного из них на \( 21  \) км/ч больше скорости другого?
		\item Два мотоциклиста стартуют одновременно в одном направлении из двух диаметрально противоположных точек круговой трассы, длина которой равна \( 19  \)км. Через сколько минут мотоциклисты поравняются в первый раз, если скорость одного из них на \( 15  \) км/ч больше скорости другого?
		\item Часы со стрелками показывают \( 8  \) часов ровно. Через сколько минут минутная стрелка в четвертый раз поравняется с часовой?
		\item Часы со стрелками показывают \( 6  \) часов \( 45  \) минут. Через сколько минут минутная стрелка в пятый раз поравняется с часовой?
		\item Два гонщика участвуют в гонках. Им предстоит проехать \(60\) кругов по кольцевой трассе протяжённостью \(3\) км. Оба гонщика стартовали одновременно, а на финиш первый пришёл раньше второго на \(10\) минут. Чему равнялась средняя скорость второго гонщика, если известно, что первый гонщик в первый раз обогнал второго на круг через \(15\) минут? Ответ дайте в км/ч.
		\item  Два гонщика участвуют в гонках. Им предстоит проехать \(46\) кругов по кольцевой трассе протяжённостью \(4\) км. Оба гонщика стартовали одновременно, а на финиш первый пришёл раньше второго на \(5\)минут. Чему равнялась средняя скорость второго гонщика, если известно, что первый гонщик в первый раз обогнал второго на круг через \(60\) минут? Ответ дайте в км/ч.
		\item Моторная лодка прошла против течения реки \( 112  \)км и вернулась в пункт отправления, затратив на обратный путь на \(6\) часов меньше. Найдите скорость течения, если скорость лодки в неподвижной воде равна \( 11  \) км/ч. Ответ дайте в км/ч.
		\item Моторная лодка прошла против течения реки \( 160  \) км и вернулась в пункт отправления, затратив на обратный путь на \( 6  \) часов меньше. Найдите скорость течения, если скорость лодки в неподвижной воде равна \( 13  \) км/ч. Ответ дайте в км/ч.
		\item Баржа в \( 10:00  \) вышла из пункта \( А  \)в пункт \( В \), расположенный в \( 30  \) км от \( A \). Пробыв в пункте \( В 1  \) час \( 30  \) минут, баржа отправилась назад и вернулась в пункт \( А  \) в \( 22:00  \) того же дня. Определите (в км/ч) собственную скорость баржи, если известно, что скорость течения реки \( 3  \) км/ч.
		\item Лодка в \( 5:00  \) вышла из пункта \( А  \) в пункт \( В \), расположенный в \( 30  \) км от \( A \). Пробыв в пункте \(  В 2  \) часа, лодка отправилась назад и вернулась в пункт \( А  \) в \( 23:00  \) того же дня. Определите (в км/ч) собственную скорость лодки, если известно, что скорость течения реки \( 1  \) км/ч.
		\item Пристани \( A  \) и \( B  \) расположены на озере, расстояние между ними \( 390  \) км. Баржа отправилась с постоянной скоростью из \( A  \) в \( B \). На следующий день после прибытия она отправилась обратно со скоростью на \( 3  \) км/ч больше прежней, сделав по пути остановку на \( 9  \) часов. В результате она затратила на обратный путь столько же времени, сколько на путь из \( A  \) в \( B \). Найдите скорость баржи на пути из \( A  \) в \( B \). Ответ дайте в км/ч.
		\item Пристани \( A  \) и \( B  \) расположены на озере, расстояние между ними \(234\) км. Баржа отправилась с постоянной скоростью из \( A  \) в \( B \). На следующий день после прибытия она отправилась обратно со скоростью на \(4\) км/ч больше прежней, сделав по пути остановку на \(8\) часов. В результате она затратила на обратный путь столько же времени, сколько на путь из \(A\) в \(B\). Найдите скорость баржи на пути из \(A\)  в \(B\). Ответ дайте в км/ч.
		\item Весной катер идёт против течения реки в \( 1\dfrac{2}{3} \) раза медленнее, чем по течению. Летом течение становится на \( 1  \) км/ч медленнее. Поэтому летом катер идёт против течения в \( 1\dfrac{1}{2} \) раза медленнее, чем по течению. Найдите скорость течения весной (в км/ч).
		\item По морю параллельными курсами в одном направлении следуют два сухогруза: первый длиной \(120\) метров, второй – длиной \(80\) метров. Сначала второй сухогруз отстает от первого, и в некоторый момент времени расстояние от кормы первого сухогруза до носа второго составляет \(400  \) метров. Через \(12\) минут после этого уже первый сухогруз отстает от второго так, что расстояние от кормы второго сухогруза до носа первого равно \( 600  \) метрам. На сколько километров в час скорость первого сухогруза меньше скорости второго?
		\item По морю параллельными курсами в одном направлении следуют два сухогруза: первый длиной \(140\) метров, второй – длиной \( 60  \) метров. Сначала второй сухогруз отстает от первого, и в некоторый момент времени расстояние от кормы первого сухогруза до носа второго составляет \( 800  \) метров. Через \( 15  \) минут после этого уже первый сухогруз отстает от второго так, что расстояние от кормы второго сухогруза до носа первого равно \( 1000  \) метрам. На сколько километров в час скорость первого сухогруза меньше скорости второго?
	\end{enumerate}	
	\end{document}