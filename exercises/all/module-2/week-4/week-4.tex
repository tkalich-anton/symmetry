\documentclass[12pt, a4paper]{article}
\usepackage{cmap} % Улучшенный поиск русских слов в полученном pdf-файле
\usepackage[T2A]{fontenc} % Поддержка русских букв
\usepackage[utf8]{inputenc} % Кодировка utf8
\usepackage[english, russian]{babel} % Языки: русский, английский
\usepackage{enumitem}
\usepackage{pscyr} % Нормальные шрифты
\usepackage{amsmath}
\usepackage{amsthm}
\usepackage{amssymb}
\usepackage{scrextend}
\usepackage{titling}
\usepackage{indentfirst}
\usepackage{cancel}
\usepackage{soulutf8}
\usepackage{wrapfig}
\usepackage{gensymb}
\usepackage[dvipsnames,table,xcdraw]{xcolor}
\usepackage{tikz}

%Русские символы в списке
\makeatletter
\AddEnumerateCounter{\asbuk}{\russian@alph}{щ}
\makeatother

%Дублирование знаков при переносе
\newcommand*{\hm}[1]{#1\nobreak\discretionary{}%
	{\hbox{$\mathsurround=0pt #1$}}{}}

\usepackage{graphicx}
\graphicspath{{pic/}}
\DeclareGraphicsExtensions{.pdf,.png,.jpg}

%Изменеие параметров листа
\usepackage[left=15mm,right=15mm,
top=2cm,bottom=2cm,bindingoffset=0cm]{geometry}

\usepackage{fancyhdr}
\pagestyle{fancy}
\usepackage{multicol}

\setlength\parindent{1,5em}
\usepackage{indentfirst}
\begin{document}
	
	\chead{Модуль 2}
	\rhead{Школа <<Симметрия>>}
	\begin{enumerate}
		\item Какова вероятность того, что случайно выбранный телефонный номер оканчивается двумя чётными цифрами?
		\item Какова вероятность того, что в случайно выбранном телефонном номере последняя цифра чётная, а предпоследняя – нечётная?
		\item Если шахматист А. играет белыми фигурами, то он выигрывает у шахматиста Б. с вероятностью \( 0,52 \). Если А. играет черными, то А. выигрывает у Б. с вероятностью \( 0,3 \). Шахматисты А. и Б. играют две партии, причём во второй партии меняют цвет фигур. Найдите вероятность того, что А. выиграет оба раза.
		\item Если шахматист А. играет белыми фигурами, то он выигрывает у шахматиста Б. с вероятностью \( 0,5 \). Если А. играет чёрными, то А. выигрывает у Б. с вероятностью \( 0,32 \). Шахматисты А. и Б. играют две партии, причём во второй партии меняют цвет фигур. Найдите вероятность того, что А. выиграет оба раза.
		\item На рисунке изображён лабиринт. Паук заползает в лабиринт в точке «Вход». Развернуться и ползти назад паук не может, поэтому на каждом разветвлении паук выбирает один из путей, по которому ещё не полз. Считая, что выбор дальнейшего пути чисто случайный, определите, с какой вероятностью паук придёт к выходу D.
		\item На рисунке изображён лабиринт. Паук заползает в лабиринт в точке «Вход». Развернуться и ползти назад паук не может. На каждом разветвлении паук выбирает путь, по которому ещё не полз. Считая выбор дальнейшего пути случайным, определите, с какой вероятностью паук придёт к выходу A.
		\item Вероятность того, что в случайный момент времени температура тела здорового человека окажется ниже, чем \( 36,8^{\circ}C \), равна \( 0,81 \). Найдите вероятность того, что в случайный момент времени у здорового человека температура окажется \( 36,8^{\circ}C \) или выше.
		\item Вероятность того, что в случайный момент времени температура тела здорового человека окажется ниже, чем \( 36,8^{\circ}C \), равна \( 0,94 \). Найдите вероятность того, что в случайный момент времени у здорового человека температура окажется \( 36,8^{\circ}C \) или выше.
		\item При изготовлении подшипников диаметром \( 67 \) мм вероятность того, что диаметр будет отличаться от заданного не больше, чем на \( 0,01 \) мм, равна \( 0,965 \). Найдите вероятность того, что случайный подшипник будет иметь диаметр меньше чем \( 66,99 \) мм или больше чем \( 67,01 \) мм.
		\item При изготовлении подшипников диаметром \( 72 \) мм вероятность того, что диаметр будет отличаться от заданного не больше, чем на \( 0,01 \) мм, равна \( 0,971 \). Найдите вероятность того, что случайный подшипник будет иметь диаметр меньше, чем \( 71,99 \) мм, или больше, чем \( 72,01 \) мм.
		\item Вероятность того, что батарейка бракованная, равна \( 0,06 \). Покупатель в магазине выбирает случайную упаковку, в которой две таких батарейки. Найдите вероятность того, что обе батарейки окажутся исправными.
		\item Вероятность того, что батарейка бракованная, равна \( 0,02 \). Покупатель в магазине выбирает случайную упаковку, в которой две таких батарейки. Найдите вероятность того, что обе батарейки окажутся исправными.
		\item В торговом центре два одинаковых автомата продают кофе. Обслуживание автоматов происходит по вечерам после закрытия центра. Известно, что вероятность события «К вечеру в первом автомате закончится кофе» равна \(  0,25 \). Такая же вероятность события «К вечеру во втором автомате закончится кофе». Вероятность того, что кофе к вечеру закончится в обоих автоматах, равна \( 0,15 \). Найдите вероятность того, что к вечеру кофе останется в обоих автоматах.
		\item В торговом центре два одинаковых автомата продают жвачку. Вероятность того, что к концу дня в автомате закончится жвачка, равна \( 0,4 \). Вероятность того, что жвачка закончится в обоих автоматах, равна \( 0,14 \). Найдите вероятность того, что к концу дня жвачка останется в обоих автоматах.
		\item Вероятность того, что новый электрический чайник прослужит больше года, равна \( 0,93 \). Вероятность того, что он прослужит больше двух лет, равна \( 0,87 \). Найдите вероятность того, что он прослужит меньше двух лет, но больше года.
		\item Вероятность того, что новый сканер прослужит больше года, равна \( 0,94 \). Вероятность того, что он прослужит больше двух лет, равна \( 0,87 \). Найдите вероятность того, что он прослужит меньше двух лет, но больше года.
		\item Из районного центра в деревню ежедневно ходит автобус. Вероятность того, что в понедельник в автобусе окажется меньше \( 18 \) пассажиров, равна \( 0,82 \). Вероятность того, что окажется меньше \( 10 \) пассажиров, равна \( 0,51 \). Найдите вероятность того, что число пассажиров будет от \( 10 \) до \( 17 \).
		\item Из районного центра в деревню ежедневно ходит автобус. Вероятность того, что в автобусе окажется меньше \( 20 \) пассажиров, равна \( 0,81 \). Вероятность того, что окажется меньше \( 12 \) пассажиров, равна \( 0,56 \). Найдите вероятность того, что число пассажиров будет от \( 12 \) до \( 19 \).
		\item Помещение освещается фонарём с двумя лампами. Вероятность перегорания лампы в течение года равна \( 0,3 \). Найдите вероятность того, что в течение года хотя бы одна лампа не перегорит.
		\item Помещение освещается фонарём с тремя лампами. Вероятность перегорания одной лампы в течение года равна \( 0,18 \). Найдите вероятность того, что в течение года хотя бы одна лампа не перегорит.
		\item При артиллерийской стрельбе автоматическая система делает выстрел по цели. Если цель не уничтожена, то система делает повторный выстрел. Выстрелы повторяются до тех пор, пока цель не будет уничтожена. Вероятность уничтожения некоторой цели при первом выстреле равна \( 0,4 \), а при каждом последующем – \( 0,6 \). Сколько выстрелов потребуется для того, чтобы вероятность уничтожения цели была не менее \( 0,98 \)?
		\item При артиллерийской стрельбе автоматическая система делает выстрел по цели. Если цель не уничтожена, то система делает повторный выстрел. Выстрелы повторяются до тех пор, пока цель не будет уничтожена. Вероятность уничтожения некоторой цели при первом выстреле равна \( 0,8 \), а при каждом последующем – \( 0,9 \). Сколько выстрелов потребуется для того, чтобы вероятность уничтожения цели была не менее \( 0,99 \)?
		\item В Волшебной стране бывает два типа погоды: хорошая и отличная, причём погода, установившись утром, держится неизменной весь день. Известно, что с вероятностью \( 0,8  \) погода завтра будет такой же, как и сегодня. Сегодня \( 3 \) июля, погода в Волшебной стране хорошая. Найдите вероятность того, что 6 июля в Волшебной стране будет отличная погода.
		\item В Волшебной стране бывает два типа погоды: хорошая и отличная, причём погода, установившись утром, держится неизменной весь день. Известно, что с вероятностью \( 0,8 \) погода завтра будет такой же, как и сегодня. \( 3  \) августа погода в Волшебной стране хорошая. Найдите вероятность того, что 6 августа в Волшебной стране будет отличная погода.
		\item В магазине стоят два платёжных автомата. Каждый из них может быть неисправен с вероятностью \( 0,05 \) независимо от другого автомата. Найдите вероятность того, что хотя бы один автомат исправен.
		\item В магазине стоят два платёжных автомата. Каждый из них может быть неисправен с вероятностью \( 0,07 \) независимо от другого автомата. Найдите вероятность того, что хотя бы один автомат исправен.
		\item Две фабрики выпускают одинаковые стекла для автомобильных фар. Первая фабрика выпускает \( 45\% \) этих стекол, вторая – \( 55\% \). Первая фабрика выпускает \( 3\% \) бракованных стекол, а вторая – \( 1\% \). Найдите вероятность того, что случайно купленное в магазине стекло окажется бракованным.
		\item Две фабрики выпускают одинаковые стекла для автомобильных фар. Первая фабрика выпускает \( 65\% \) этих стекол, вторая – \( 35\% \). Первая фабрика выпускает \( 5\% \) бракованных стекол, а вторая – \( 3\% \). Найдите вероятность того, что случайно купленное в магазине стекло окажется бракованным.
		\item Всем пациентам с подозрением на гепатит делают анализ крови. Если анализ выявляет гепатит, то результат анализа называется положительным. У больных гепатитом пациентов анализ даёт положительный результат с вероятностью \( 0,9 \). Если пациент не болен гепатитом, то анализ может дать ложный положительный результат с вероятностью \( 0,01 \). Известно, что \( 5\% \) пациентов, поступающих с подозрением на гепатит, действительно больны гепатитом. Найдите вероятность того, что результат анализа у пациента, поступившего в клинику с подозрением на гепатит, будет положительным.
		\item Всем пациентам с подозрением на гепатит делают анализ крови. Если анализ выявляет гепатит, то результат анализа называется положительным. У больных гепатитом пациентов анализ даёт положительный результат с вероятностью \( 0,9 \). Если пациент не болен гепатитом, то анализ может дать ложный положительный результат с вероятностью \( 0,03 \). Известно, что \( 2\% \) пациентов, поступающих с подозрением на гепатит, действительно больны гепатитом. Найдите вероятность того, что результат анализа у пациента, поступившего в клинику с подозрением на гепатит, будет положительным.
		\item Агрофирма закупает куриные яйца в двух домашних хозяйствах. \( 40\%\) яиц из первого хозяйства – яйца высшей категории, а из второго хозяйства – \( 20\% \) яиц высшей категории. Всего высшую категорию получает \( 35\% \) яиц. Найдите вероятность того, что яйцо, купленное у этой агрофирмы, окажется из первого хозяйства.
		\item Агрофирма закупает куриные яйца в двух домашних хозяйствах. \( 65\% \) яиц из первого хозяйства – яйца высшей категории, а из второго хозяйства – \( 85\% \) яиц высшей категории. Всего высшую категорию получает \( 80\% \) яиц. Найдите вероятность того, что яйцо, купленное у этой агрофирмы, окажется из первого хозяйства.
		\item В торговом центре два одинаковых автомата продают кофе. Вероятность того, что к концу дня в автомате закончится кофе, равна \( 0,3 \). Вероятность того, что кофе закончится в обоих автоматах, равна \( 0,12 \). Найдите вероятность того, что к концу дня кофе останется в обоих автоматах.
		\item Чтобы поступить в институт на специальность «Лингвистика», абитуриент должен набрать на ЕГЭ не менее \( 70 \) баллов по каждому из трёх предметов – математика, русский язык и иностранный язык. Чтобы поступить на специальность «Коммерция», нужно набрать не менее \( 70 \) баллов по каждому из трёх предметов – математика, русский язык и обществознание.
		
		Вероятность того, что абитуриент З. получит не менее \( 70 \) баллов по математике, равна \( 0,6 \), по русскому языку – \( 0,8 \), по иностранному языку – \( 0,7 \) и по обществознанию – \( 0,5 \).
		
		Найдите вероятность того, что З. сможет поступить хотя бы на одну из двух упомянутых специальностей.
		\item Чтобы поступить в институт на специальность «Лингвистика», абитуриент должен набрать на ЕГЭ не менее \( 68 \) баллов по каждому из трёх предметов – математика, русский язык и иностранный язык. Чтобы поступить на на специальность «Менеджмент», нужно набрать не менее \( 68 \) баллов по каждому из трёх предметов – математика, русский язык и обществознание.
		
		Вероятность того, что абитуриент Р. получит не менее \( 68 \) баллов по математике, равна \( 0,7 \), по русскому языку – \(  0,7 \), по иностранному языку – \( 0,8  \) и по обществознанию – \( 0,5 \).
		\item Из районного центра в деревню ежедневно ходит автобус. Вероятность того, что в понедельник в автобусе окажется меньше \( 20 \) пассажиров, равна \(  0,94 \). Вероятность того, что окажется меньше \( 15 \) пассажиров, равна \( 0,56 \). Найдите вероятность того, что число пассажиров будет от \( 15 \) до \( 19 \).
		\item Вероятность того, что на тестировании по биологии учащийся О. верно решит больше \( 11 \) задач, равна \( 0,67 \). Вероятность того, что О. верно решит больше \( 10 \) задач, равна \( 0,74 \). Найдите вероятность того, что О. верно решит ровно \( 11 \) задач.
		\item Вероятность того, что на тестировании по математике учащийся У. верно решит больше \( 12 \) задач, равна \( 0,78 \). Вероятность того, что У. верно решит больше \( 11 \) задач, равна \( 0,88 \). Найдите вероятность того, что У. верно решит ровно \( 12 \) задач.
		\item Перед началом волейбольного матча капитаны команд тянут честный жребий, чтобы определить, какая из команд начнёт игру с мячом. Команда «Статор» по очереди играет с командами «Ротор», «Мотор» и «Стартер». Найдите вероятность того, что «Статор» будет начинать только первую и последнюю игры.
		\item В кармане у Пети было \( 2 \) монеты по \( 5 \) рублей и \( 4 \) монеты по \( 10 \) рублей. Петя, не глядя, переложил какие-то \( 3 \) монеты в другой карман. Найдите вероятность того, что пятирублевые монеты лежат теперь в разных карманах.
		\item В кармане у Пети было \( 4 \) монеты по рублю и \( 2 \) монеты по два рубля. Петя, не глядя, переложил какие-то \( 3 \) монеты в другой карман. Найдите вероятность того, что обе двухрублёвые монеты лежат в одном кармане.
		\item Стрелок стреляет по мишени один раз. В случае промаха стрелок делает второй выстрел по той же мишени. Вероятность попасть в мишень при одном выстреле равна \( 0,7 \). Найдите вероятность того, что мишень будет поражена (либо первым, либо вторым выстрелом).
		\item Стрелок при каждом выстреле поражает мишень с вероятностью \( 0,3 \), независимо от результатов предыдущих выстрелов. Какова вероятность того, что он поразит мишень, сделав не более \( 3 \) выстрелов?
		\item Игральный кубик бросают дважды. Известно, что в сумме выпало \( 8 \) очков. Найдите вероятность того, что во второй раз выпало \( 3 \) очка.
		\item Игральный кубик бросают дважды. Известно, что в сумме выпало \( 9 \) очков. Найдите вероятность того, что во второй раз выпало \( 5 \) очков.
		\item При двукратном бросании игральной кости в сумме выпало \( 9 \) очков. Какова вероятность того, что хотя бы раз выпало \( 5 \) очков?
		\item При двукратном бросании игральной кости в сумме выпало \( 8 \) очков. Какова вероятность того, что хотя бы раз выпало \( 6 \) очков?
		\item Игральную кость бросили один или несколько раз. Оказалось, что сумма всех выпавших очков равна \( 4 \). Какова вероятность того, что был сделан один бросок? Ответ округлите до сотых.
		\item Игральную кость бросили один или несколько раз. Оказалось, что сумма всех выпавших очков равна \( 3 \). Какова вероятность того, что было сделано два броска? Ответ округлите до сотых.
		\item Игральную кость бросили один или несколько раз. Оказалось, что сумма всех выпавших очков равна \( 3 \). Какова вероятность того, что был сделан один бросок? Ответ округлите до сотых.
		\item Первый игральный кубик обычный, а на гранях второго кубика нет чисел, больших, чем \( 2 \), а числа \( 1 \) и \( 2 \) встречаются по три раза. В остальном кубики одинаковые.
		
		Один случайно выбранный кубик бросают два раза. Известно, что в каком-то порядке выпали \( 1 \) и \( 2 \) очков. Какова вероятность того, что бросали второй кубик?
		\item Первый игральный кубик обычный, а на гранях второго кубика числа \( 5 \) и \( 6 \) встречаются по три раза. В остальном кубики одинаковые.
		
		Один случайно выбранный кубик бросают два раза. Известно, что в каком-то порядке выпали \( 5  \) и \( 6 \) очков. Какова вероятность того, что бросали второй кубик?
		\item Первый игральный кубик обычный, а на гранях второго кубика числа \( 1 \) и \( 2 \) встречаются по три раза. В остальном кубики одинаковые.
		
		Один случайно выбранный кубик бросают два раза. Известно, что в каком-то порядке выпали \( 1 \) и \( 2 \) очков. Какова вероятность того, что бросали первый кубик?
		\item Маша коллекционирует принцесс из Киндер-сюрпризов. Всего в коллекции \( 10 \) разных принцесс, и они равномерно распределены, то есть в каждом очередном Киндер-сюрпризе может с равными вероятностями оказаться любая из \( 10 \) принцесс. У Маши уже есть две разные принцессы из коллекции. Какова вероятность того, что для получения следующей принцессы Маше придётся купить ещё \( 2 \) или \( 3 \) шоколадных яйца?
		\item Маша коллекционирует принцесс из Киндер-сюрпризов. Всего в коллекции \( 10 \) разных принцесс, и они равномерно распределены, то есть в каждом очередном Киндер-сюрпризе может с равными вероятностями оказаться любая из \( 10 \) принцесс.
		
		У Маши уже есть четыре разные принцессы из коллекции. Какова вероятность того, что для получения следующей принцессы Маше придётся купить ещё \( 1 \) или \( 2 \) шоколадных яйца?
		\item Артём гуляет по парку. Он выходит из точки S и, дойдя до очередной развилки, с равными шансами выбирает следующую дорожку, но не возвращается обратно. Найдите вероятность того, что таким образом он выйдет к пруду или фонтану.
		\item Артём гуляет по парку. Он выходит из точки S и, дойдя до очередной развилки, с равными шансами выбирает следующую дорожку, но не возвращается обратно. Найдите вероятность того, что таким образом он выйдет к детской площадке.
		\item Симметричную игральную кость бросили \( 3 \) раза. Известно, что в сумме выпало \( 6 \) очков. Какова вероятность события «хотя бы раз выпало \( 3 \) очка»?
		\item В городе \( 48\% \) взрослого населения – мужчины. Пенсионеры составляют \( 12,6\% \) взрослого населения, причём доля пенсионеров среди женщин равна \( 15\% \). Для социологического опроса выбран случайным образом мужчина, проживающий в этом городе. Найдите вероятность события «выбранный мужчина является пенсионером».
		\item В коробке \( 8 \) синих, \( 6 \) красных и \( 11 \) зелёных фломастеров. Случайным образом выбирают два фломастера. Какова вероятность того, что окажутся выбраны один синий и один красный фломастер?
		\item В коробке \( 11 \) синих, \( 6 \) красных и \( 8 \) зелёных фломастеров. Случайным образом выбирают два фломастера. Какова вероятность того, что окажутся выбраны один синий и один красный фломастер?
		\item Платежный терминал в течение рабочего дня может выйти из строя. Вероятность этого события \( 0,07 \). В торговом центре независимо друг от друга работают два таких платёжных терминала. Найдите вероятность того, что хотя бы один из них в течение рабочего дня будет исправен.
	\end{enumerate}	
	\end{document}