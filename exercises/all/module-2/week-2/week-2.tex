\documentclass[12pt, a4paper]{article}
\usepackage{cmap} % Улучшенный поиск русских слов в полученном pdf-файле
\usepackage[T2A]{fontenc} % Поддержка русских букв
\usepackage[utf8]{inputenc} % Кодировка utf8
\usepackage[english, russian]{babel} % Языки: русский, английский
\usepackage{enumitem}
\usepackage{pscyr} % Нормальные шрифты
\usepackage{amsmath}
\usepackage{amsthm}
\usepackage{amssymb}
\usepackage{scrextend}
\usepackage{titling}
\usepackage{indentfirst}
\usepackage{cancel}
\usepackage{soulutf8}
\usepackage{wrapfig}
\usepackage{gensymb}
\usepackage[dvipsnames,table,xcdraw]{xcolor}
\usepackage{tikz}

%Русские символы в списке
\makeatletter
\AddEnumerateCounter{\asbuk}{\russian@alph}{щ}
\makeatother

%Дублирование знаков при переносе
\newcommand*{\hm}[1]{#1\nobreak\discretionary{}%
	{\hbox{$\mathsurround=0pt #1$}}{}}

\usepackage{graphicx}
\graphicspath{{pic/}}
\DeclareGraphicsExtensions{.pdf,.png,.jpg}

%Изменеие параметров листа
\usepackage[left=15mm,right=15mm,
top=2cm,bottom=2cm,bindingoffset=0cm]{geometry}

\usepackage{fancyhdr}
\pagestyle{fancy}
\usepackage{multicol}

\setlength\parindent{1,5em}
\usepackage{indentfirst}
\begin{document}
	
	\chead{Модуль 2}
	\rhead{Школа <<Симметрия>>}
	\begin{enumerate}
		\item Заказ на \( 110 \) деталей первый рабочий выполняет на \( 1 \) час быстрее, чем второй. Сколько деталей за час изготавливает второй рабочий, если известно, что первый за час изготавливает на \( 1 \) деталь больше?
		\item Заказ на \( 156 \) деталей первый рабочий выполняет на \( 1 \) час быстрее, чем второй. Сколько деталей за час изготавливает второй рабочий, если известно, что первый за час изготавливает на \( 1 \) деталь больше?
		\item На изготовление\(667\) деталей первый рабочий тратит на \( 6 \) часов меньше, чем второй рабочий на изготовление \(754\) таких же деталей. Известно, что первый рабочий за час делает на \(3\) детали больше, чем второй. Сколько деталей в час делает первый рабочий?
		\item На изготовление \(713\) деталей первый рабочий тратит на \(8\) часов меньше, чем второй рабочий на изготовление \(837\) таких же деталей. Известно, что первый рабочий за час делает на \(4\) детали больше, чем второй. Сколько деталей в час делает первый рабочий?
		\item Двое рабочих, работая вместе,могут выполнить работу за \( 15 \) дней. За сколько дней, работая отдельно, выполнит эту работу первый  рабочий, если он за \(  2 \) дня выполняет такую же часть работы, какую второй – за \( 3 \) дня?
		\item Двое рабочих, работая вместе,могут выполнить работу за \( 8 \) дней. За сколько дней, работая отдельно, выполнит эту работу первый  рабочий, если он за \( 3 \) дня выполняет такую же часть работы, какую второй – за \( 2 \) дня?
		\item Первая труба пропускает на \( 1 \) литр воды в минуту меньше, чем вторая. Сколько литров воды в минуту пропускает вторая труба, если резервуар объемом \(110 \) литров она заполняет на \( 1 \) минуту быстрее, чем первая труба?
		\item Первая труба пропускает на \( 4 \) литра воды в минуту меньше, чем вторая. Сколько литров воды в минуту пропускает вторая труба, если резервуар объемом \(621 \) литр она заполняет на \( 2 \) минуты дольше, чем первая труба?
		\item Первая труба пропускает на \(1\) литр воды в минуту меньше, чем вторая. Сколько литров воды в минуту пропускает первая труба, если резервуар объемом \(110\) литров она заполняет на \( 2 \) минуты дольше, чем вторая заполняет резервуар объемом \(99\) литров?
		\item Первая труба пропускает на \(1\) литр воды в минуту меньше, чем вторая. Сколько литров воды в минуту пропускает первая труба, если резервуар объемом \(420\) литров она заполняет на \( 2 \) минуты дольше, чем вторая заполняет резервуар объемом \(399\) литров?
		\item Каждый из двух рабочих одинаковой квалификации может выполнить заказ за \( 15 \) часов. Через \( 3 \) часа после того, как один из них приступил к выполнению заказа, к нему присоединился второй рабочий, и работу над заказом они довели до конца уже вместе. Сколько часов потребовалось на выполнение всего заказа?
		\item Каждый из двух рабочих одинаковой квалификации может выполнить заказ за \( 12 \) часов. Через \( 4 \) часа после того, как один из них приступил к выполнению заказа, к нему присоединился второй рабочий, и работу над заказом они довели до конца уже вместе. Сколько часов потребовалось на выполнение всего заказа?
		\item Один мастер может выполнить заказ за \( 12 \) часов, а другой – за \( 6 \) часов. За сколько часов выполнят заказ оба мастера, работая вместе?
		\item Один мастер может выполнить заказ за \( 40 \) часов, а другой – за \( 24 \) часа. За сколько часов выполнят заказ оба мастера, работая вместе?
		\item Первый насос наполняет бак за \( 20 \) минут, второй – за \( 30 \) минут, а третий – за \( 1 \) час. За сколько минут наполнят бак три насоса, работая одновременно?
		\item Первый насос наполняет бак за \( 18 \) минут, второй – за \( 24 \) минуты, а третий – за \( 36 \) минут. За сколько минут наполнят бак три насоса, работая одновременно?
		\item Игорь и Паша красят забор за \( 15 \) часов. Паша и Володя красят этот же забор за \( 21 \) час, а Володя и Игорь – за \( 35 \) часов. За сколько часов мальчики покрасят забор, работая втроем?
		\item Игорь и Паша красят забор за \( 21 \) час. Паша и Володя красят этот же забор за \( 28 \) часов, а Володя и Игорь – за \( 60 \) часов. За сколько часов мальчики покрасят забор, работая втроем?
		\item Первый садовый насос перекачивает \( 5 \) литров воды за \( 2 \) минуты, второй насос перекачивает тот же объём воды за \( 3 \) минуты. Сколько минут эти два насоса должны работать совместно, чтобы перекачать \( 25 \) литров воды?
		\item В помощь садовому насосу, перекачивающему \( 8 \) литров воды за \( 3 \) минуты, подключили второй насос, перекачивающий тот же объем воды за \( 6 \) минут. Сколько минут эти два насоса должны работать совместно, чтобы перекачать \( 24 \) литра воды?
		\item Плиточник планирует уложить \( 175 \) м\( ^2\) плитки. Если он будет укладывать на \( 10 \) м\( ^2 \) в день больше, чем запланировал, то закончит работу на \( 2 \) дня раньше. Сколько квадратных метров плитки в день планирует укладывать плиточник?
		\item  Плиточник планирует уложить \( 324 \) м\( ^2\) плитки. Если он будет укладывать на \( 6 \) м\( ^2 \) в день больше, чем запланировал, то закончит работу на \( 9 \) дней раньше. Сколько квадратных метров плитки в день планирует укладывать плиточник?
		\item Два промышленных фильтра, работая одновременно, очищают цистерну воды за \( 30 \) минут. Определите, за сколько минут второй фильтр очистит цистерну воды, работая отдельно, если известно, что он сделает это на \( 25 \) минут быстрее, чем первый.
		\item При двух одновременно работающих принтерах расход бумаги составляет \( 1 \) пачку за \( 12 \) минут. Определите, за сколько минут израсходует пачку бумаги первый принтер, если известно, что он сделает это на \( 10 \) минут быстрее, чем второй.
		\item В \( 2008 \) году в городском квартале проживало \( 40 000 \) человек. В \( 2009 \) году, в результате строительства новых домов, число жителей выросло на \( 8 \% \), а в \( 2010 \) году на \( 9 \% \) по сравнению с \( 2009 \) годом. Сколько человек стало проживать в квартале в \( 2010  \) году?
		\item В \( 2008 \) году в городском квартале проживало \( 40 000 \) человек. В \( 2009 \) году, в результате строительства новых домов, число жителей выросло на \( 6 \% \), а в \( 2010 \) году – на \( 7\% \) по сравнению с \( 2009 \) годом. Сколько человек стало проживать в квартале в \( 2010 \) году?
		\item В понедельник акции компании подорожали на некоторое количество процентов, а во вторник подешевели на то же самое количество процентов. В результате они стали стоить на \( 4 \%  \) дешевле, чем при открытии торгов в понедельник. На сколько процентов подорожали акции компании в понедельник?
		\item В среду акции компании подорожали на некоторое количество процентов, а в четверг подешевели на то же самое количество процентов. В результате они стали стоить на \( 64\% \) дешевле, чем при открытии торгов в среду. На сколько процентов подорожали акции компании в среду?
		\item Четыре одинаковые рубашки дешевле куртки на \( 8\% \). На сколько процентов пять таких же рубашек дороже куртки?
		\item Девять одинаковых рубашек дешевле куртки на \( 10\% \). На сколько процентов четырнадцать таких же рубашек дороже куртки?
		\item Семья состоит из мужа, жены и их дочери студентки. Если бы зарплата мужа увеличилась вчетверо, общий доход семьи вырос бы на \( 210\% \). Если бы стипендия дочери уменьшилась вдвое, общий доход семьи сократился бы на \( 1\% \). Сколько процентов от общего дохода семьи составляет зарплата жены?
		\item Семья состоит из мужа, жены и их дочери студентки. Если бы зарплата мужа увеличилась вчетверо, общий доход семьи вырос бы на \( 204\% \). Если бы стипендия дочери уменьшилась вдвое, общий доход семьи сократился бы на \( 4\% \). Сколько процентов от общего дохода семьи составляет зарплата жены?
		\item Цена холодильника в магазине ежегодно уменьшается на одно и то же число процентов от предыдущей цены. Определите, на сколько процентов каждый год уменьшалась цена холодильника, если, выставленный на продажу за \(20 000\) рублей, через два года был продан за \(15 842\) рублей.
		\item Цена холодильника в магазине ежегодно уменьшается на одно и то же число процентов от предыдущей цены. Определите, на сколько процентов каждый год уменьшалась цена холодильника, если, выставленный на продажу за \(19 600\) рублей, через два года был продан за \(17 689\) рублей.
		\item Дима, Артем, Гриша и Игорь учредили компанию с уставным капиталом \( 100000 \) рублей. Дима внес \( 19\% \) уставного капитала, Артем –  \( 50000 \) рублей, Гриша – \( 0,14 \) уставного капитала, а оставшуюся часть капитала внес Игорь. Учредители договорились делить ежегодную прибыль пропорционально внесенному в уставной капитал вкладу. Какая сумма от прибыли \( 700000 \) рублей причитается Игорю? Ответ дайте в рублях.
		\item Дима, Артем, Никита и Денис учредили компанию с уставным капиталом \( 100000 \) рублей. Дима внес \( 20\% \) уставного капитала, Артем –  \( 50000 \) рублей, Никита – \( 0,26 \) уставного капитала, а оставшуюся часть капитала внес Денис. Учредители договорились делить ежегодную прибыль пропорционально внесенному в уставной капитал вкладу. Какая сумма от прибыли \( 700000 \) рублей причитается Денису? Ответ дайте в рублях.
		\item Смешали некоторое количество \( 18 \)-процентного раствора некоторого вещества с таким же количеством \( 14 \)-процентного раствора этого вещества. Сколько процентов составляет концентрация получившегося раствора?
		\item Смешали некоторое количество \( 15\%\)-го раствора некоторого вещества с таким же количеством \( 19\% \)-го раствора этого вещества. Сколько процентов составляет концентрация получившегося раствора?
		\item Изюм получается в процессе сушки винограда. Сколько килограммов винограда потребуется для получения \( 14 \) килограммов изюма, если виноград содержит \( 90\% \) воды, а изюм содержит \( 5\% \) воды?
		\item Изюм получается в процессе сушки винограда. Сколько килограммов винограда потребуется для получения \( 56 \) килограммов изюма, если виноград содержит \( 90\% \) воды, а изюм содержит \( 5\% \) воды?
		\item Имеется два сплава. Первый сплав содержит \( 5\% \) меди, второй – \( 12\% \) меди. Масса второго сплава больше массы первого на \( 9 \) кг. Из этих двух сплавов получили третий сплав, содержащий \( 10\% \) меди. Найдите массу третьего сплава. Ответ дайте в килограммах.
		\item Имеется два сплава. Первый сплав содержит \( 5\% \) меди, второй – \( 14\% \) меди. Масса второго сплава больше массы первого на \( 7 \) кг. Из этих двух сплавов получили третий сплав, содержащий \( 13\% \) меди. Найдите массу третьего сплава. Ответ дайте в килограммах.
		\item Клиент А сделал вклад в банке в размере \( 2500 \) рублей. Проценты по вкладу начисляются раз в год и прибавляются к текущей сумме вклада. Ровно через год на тех же условиях такой же вклад в том же банке сделал клиент Б. Ещё ровно через год клиенты А и Б закрыли вклады и забрали все накопившиеся деньги. При это клиента А \( 216 \) рублей больше клиента Б. Какой процент годовых начислял банк по этим вкладам?
		\item Клиент А сделал вклад в банке в размере \( 4300 \) рублей. Проценты по вкладу начисляются раз в год и прибавляются к текущей сумме вклада. Ровно через год на тех же условиях такой же вклад в том же банке сделал клиент Б. Ещё ровно через год клиенты А и Б закрыли вклады и забрали все накопившиеся деньги. При это клиента А \( 473 \) рубля                 больше клиента Б. Какой процент годовых начислял банк по этим вкладам?
		\item Имеется два сплава. Первый содержит \( 15\% \) никеля, второй – \( 35\% \) никеля. Из этих двух сплавов получили третий сплав массой \( 140 \) кг, содержащий \( 30\%  \) никеля. На сколько килограммов масса первого сплава была меньше массы второго?
	\end{enumerate}	
	\end{document}