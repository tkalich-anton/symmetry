\documentclass[12pt, a4paper]{article}
\usepackage{cmap} % Улучшенный поиск русских слов в полученном pdf-файле
\usepackage[T2A]{fontenc} % Поддержка русских букв
\usepackage[utf8]{inputenc} % Кодировка utf8
\usepackage[english, russian]{babel} % Языки: русский, английский
\usepackage{enumitem}
\usepackage{pscyr} % Нормальные шрифты
\usepackage{amsmath}
\usepackage{amsthm}
\usepackage{amssymb}
\usepackage{scrextend}
\usepackage{titling}
\usepackage{indentfirst}
\usepackage{cancel}
\usepackage{soulutf8}
\usepackage{wrapfig}
\usepackage{gensymb}
\usepackage[dvipsnames,table,xcdraw]{xcolor}
\usepackage{tikz}

%Русские символы в списке
\makeatletter
\AddEnumerateCounter{\asbuk}{\russian@alph}{щ}
\makeatother

%Дублирование знаков при переносе
\newcommand*{\hm}[1]{#1\nobreak\discretionary{}%
	{\hbox{$\mathsurround=0pt #1$}}{}}

\usepackage{graphicx}
\graphicspath{{pic/}}
\DeclareGraphicsExtensions{.pdf,.png,.jpg}

%Изменеие параметров листа
\usepackage[left=15mm,right=15mm,
top=2cm,bottom=2cm,bindingoffset=0cm]{geometry}

\usepackage{fancyhdr}
\pagestyle{fancy}
\usepackage{multicol}

\setlength\parindent{1,5em}
\usepackage{indentfirst}
\begin{document}
	
	\chead{Модуль 2}
	\rhead{Школа <<Симметрия>>}
	\begin{enumerate}
		\item В кармане у Миши было четыре конфеты – «Грильяж», «Белочка», «Коровка» и «Ласточка», а также ключи от квартиры. Вынимая ключи, Миша случайно выронил из кармана одну конфету. Найдите вероятность того, что потерялась конфета «Грильяж».
		\item В кармане у Серёжи было четыре конфеты – «Маска», «Василёк», «Коровка» и «Белочка», а так же ключи от квартиры. Вынимая ключи, Серёжа случайно выронил из кармана одну конфету. Найдите вероятность того, что потерялась конфета «Василёк».
		\item На экзамен вынесено \( 60 \) вопросов, Андрей не выучил \( 3 \) из них. Найдите вероятность того, что ему попадется выученный вопрос.
		\item Маша включает телевизор. Телевизор включается на случайном канале. В это время по девяти каналам из сорока пяти показывают новости. Найдите вероятность того, что Маша попадет на канал, где новости не идут.
		\item Фабрика выпускает сумки. В среднем на \( 190 \) качественных сумок приходится восемь сумок со скрытыми дефектами. Найдите вероятность того, что купленная сумка окажется качественной. Результат округлите до сотых.
		\item Фабрика выпускает сумки. В среднем на \( 110 \) качественных сумок приходится одиннадцать сумок со скрытыми дефектами. Найдите вероятность того, что купленная сумка окажется качественной. Результат округлите до сотых.
		\item На рок-фестивале выступают группы – по одной от каждой из заявленных стран. Порядок выступления определяется жребием. Какова вероятность того, что группа из Канады будет выступать после группы из Англии и после группы из Вьетнама? Результат округлите до сотых.
		\item На рок-фестивале выступают группы – по одной от каждой из заявленных стран. Порядок выступления определяется жребием. Какова вероятность того, что группа из Китая будет выступать после группы из Вьетнама и после группы из Канады? Результат округлите до сотых.
		\item В некотором городе из \( 5000 \) появившихся на свет младенцев \( 2512 \) мальчиков. Найдите частоту рождения девочек в этом городе. Результат округлите до тысячных.
		\item В некотором городе из \( 5000 \) появившихся на свет младенцев \( 2480 \) девочек. Найдите частоту рождения мальчиков в этом городе. Результат округлите до тысячных.
		\item На борту самолёта \( 12 \) кресел расположены рядом с запасными выходами и \( 18 \) – за перегородками, разделяющими салоны. Все эти места удобны для пассажира высокого роста. Остальные места неудобны. Пассажир В. высокого роста. Найдите вероятность того, что на регистрации при случайном выборе места пассажиру В. достанется удобное место, если всего в самолёте \( 300 \) мест.
		\item На борту самолёта \( 13 \) мест рядом с запасными выходами и \( 19 \) мест за перегородками, разделяющими салоны. Остальные места неудобны для пассажира высокого роста. Пассажир В. высокого роста. Найдите вероятность того, что на регистрации при случайном выборе места пассажиру В. достанется удобное место, если всего в самолёте \( 200 \) мест.
		\item Вероятность того, что новый DVD-проигрыватель в течение года поступит в гарантийный ремонт, равна \( 0,045 \). В некотором городе из \( 1000 \) проданных DVD-проигрывателей в течение года в гарантийную мастерскую поступила \( 51 \) штука. На сколько отличается частота события «гарантийный ремонт» от его вероятности в этом городе?
		\item Вероятность того, что новый блендер в течение года поступит в гарантийный ремонт, равна \( 0,058 \). В некотором городе из \( 1000 \) проданных блендеров в течение года в гарантийную мастерскую поступило 59 штук. На сколько отличается частота события «гарантийный ремонт» от его вероятности в этом городе?
		\item Механические часы с двенадцатичасовым циферблатом в какой-то момент сломались и перестали идти. Найдите вероятность того, что часовая стрелка остановилась, достигнув отметки \( 10 \), но не дойдя до отметки \( 1 \).
		\item Механические часы с двенадцатичасовым циферблатом в какой-то момент остановились. Найдите вероятность того, что часовая стрелка остановилась, достигнув отметки \( 2 \), но не дойдя до отметки \( 5 \).
		\item За круглый стол на \( 9 \) стульев в случайном порядке рассаживаются \( 7 \) мальчиков и \( 2 \) девочки. Найдите вероятность того, что обе девочки будут сидеть рядом.
		\item За круглый стол на \( 26 \) стульев в случайном порядке рассаживаются \( 24 \) мальчика и \( 2 \) девочки. Найдите вероятность того, что девочки будут сидеть рядом.
		\item Проводится жеребьёвка Лиги Чемпионов. На первом этапе жеребьёвки восемь команд, среди которых команда «Барселона», распределились случайным образом по восьми игровым группам – по одной команде в группу. Затем по этим же группам случайным образом распределяются еще восемь команд, среди которых команда «Зенит». Найдите вероятность того, что команды «Барселона» и «Зенит» окажутся в одной игровой группе.
		\item В сборнике билетов по биологии всего \( 25 \) билетов, в двух из них встречается вопрос о грибах. На экзамене школьнику достаётся один случайно выбранный билет из этого сборника. Найдите вероятность того, что в этом билете не будет вопроса о грибах.
		\item В соревновании по биатлону участвуют спортсмены из \(25\) стран, одна из которых – Россия. Всего на старт вышло \(60\) участников, из которых \(6\) – из России. Порядок старта определяется жребием, стартуют спортсмены друг за другом. Какова вероятность того, что десятым стартовал спортсмен из России?
		\item В сборнике билетов по истории всего \( 50 \) билетов, в \( 13 \) из них встречается вопрос о Великой Отечественной войне. Найдите вероятность того, что в случайно выбранном на экзамене билете школьнику достанется вопрос о Великой Отечественной войне.
		\item У Вити в копилке лежит \( 12 \) рублёвых, \( 6 \) двухрублёвых, \( 4 \) пятирублёвых и \( 3 \) десятирублёвых монеты. Витя наугад достаёт из копилки одну монету. Найдите вероятность того, что оставшаяся в копилке сумма составит более \( 70 \) рублей.
		\item В случайном эксперименте симметричную монету бросают трижды. Найдите вероятность того, что выпадет хотя бы две решки.
		\item В случайном эксперименте симметричную монету бросают трижды. Найдите вероятность того, что решка выпадет все три раза.
		\item В случайном эксперименте бросают две игральные кости. Найдите вероятность того, что в сумме выпадет \( 8 \) очков. Результат округлите до сотых.
		\item В случайном эксперименте бросают три игральные кости. Найдите вероятность того, что в сумме выпадет \( 15 \) очков. Результат округлите до сотых.
		\item В случайном эксперименте бросают три игральные кости. Найдите вероятность того, что в сумме выпадет \( 13 \) очков. Результат округлите до десятых.
		\item Научная конференция проводится в \( 5 \) дней. Всего запланировано \( 75 \) докладов – первые три дня по \( 17 \) докладов, остальные распределены поровну между четвертым и пятым днями. Порядок докладов определяется жеребьёвкой. Какова вероятность, что доклад профессора М. окажется запланированным на последний день конференции?
		\item В сборнике билетов по математике всего \( 25 \) билетов, в \( 10 \) из них встречается вопрос по теме "Неравенства". Найдите вероятность того, что в случайно выбранном на экзамене билете школьнику не достанется вопроса по теме "Неравенства".
		\item Вася, Петя, Коля и Лёша бросили жребий – кому начинать игру. Найдите вероятность того, что начинать игру должен будет Петя.
		\item В чемпионате мира участвуют \( 16 \) команд. С помощью жребия их нужно разделить на четыре группы по четыре команды в каждой. В ящике вперемешку лежат карточки с номерами групп:
		
		\( 1, 1, 1, 1, 2, 2, 2, 2, 3, 3, 3, 3, 4, 4, 4, 4 \).
		
		Капитаны команд тянут по одной карточке. Какова вероятность того, что команда России окажется во второй группе?
		\item В чемпионате мира участвуют \( 20 \) команд. С помощью жребия их нужно разделить на четыре группы по пять команд в каждой. В ящике вперемешку лежат карточки с номерами групп:
		
		
		\( 1, 1, 1, 1, 1, 2, 2, 2, 2, 2, 3, 3, 3, 3, 3, 4, 4, 4, 4, 4 \).
		
		
		Капитаны команд тянут по одной карточке. Какова вероятность того, что команда Франции окажется в первой группе?
		\item На клавиатуре телефона \( 10 \) цифр, от \( 0 \) до \( 9 \). Какова вероятность того, что случайно нажатая цифра будет чётной?
		\item Из множества натуральных чисел от \( 10 \) до \( 19 \) наудачу выбирают одно число. Какова вероятность того, что оно делится на \( 3 \)?
		\item Из множества натуральных чисел от \( 25 \) до \( 40 \) наудачу выбирают одно число. Какова вероятность того, что оно делится на \( 4 \)?
		\item В группе туристов \( 5  \) человек. С помощью жребия они выбирают двух человек, которые должны идти в село в магазин за продуктами. Какова вероятность того, что турист Д., входящий в состав группы, пойдёт в магазин?
		\item В группе туристов \( 6 \) человек. С помощью жребия они выбирают трёх человек, которые должны идти в село в магазин за продуктами. Какова вероятность того, что турист К., входящий в состав группы, пойдёт в магазин?
		\item Перед началом футбольного матча судья бросает монетку, чтобы определить, какая из команд начнёт игру с мячом. Команда «Физик» играет три матча с разными командами. Найдите вероятность того, что в этих играх «Физик» выиграет жребий ровно два раза.
		\item В случайном эксперименте симметричную монету бросают дважды. Найдите вероятность того, что наступит исход ОР (в первый раз выпадает орёл, во второй – решка).
	\end{enumerate}	
	\end{document}