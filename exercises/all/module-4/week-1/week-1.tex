\documentclass[12pt, a4paper]{article}
\usepackage{cmap} % Улучшенный поиск русских слов в полученном pdf-файле
\usepackage[T2A]{fontenc} % Поддержка русских букв
\usepackage[utf8]{inputenc} % Кодировка utf8
\usepackage[english, russian]{babel} % Языки: русский, английский
\usepackage{enumitem}
\usepackage{pscyr} % Нормальные шрифты
\usepackage{amsmath}
\usepackage{amsthm}
\usepackage{amssymb}
\usepackage{scrextend}
\usepackage{titling}
\usepackage{indentfirst}
\usepackage{cancel}
\usepackage{soulutf8}
\usepackage{wrapfig}
\usepackage{gensymb}
\usepackage[dvipsnames,table,xcdraw]{xcolor}
\usepackage{tikz}

%Русские символы в списке
\makeatletter
\AddEnumerateCounter{\asbuk}{\russian@alph}{щ}
\makeatother

%Дублирование знаков при переносе
\newcommand*{\hm}[1]{#1\nobreak\discretionary{}%
	{\hbox{$\mathsurround=0pt #1$}}{}}

\usepackage{graphicx}
\graphicspath{{pic/}}
\DeclareGraphicsExtensions{.pdf,.png,.jpg}

%Изменеие параметров листа
\usepackage[left=15mm,right=15mm,
top=2cm,bottom=2cm,bindingoffset=0cm]{geometry}

\usepackage{fancyhdr}
\pagestyle{fancy}
\usepackage{multicol}

\setlength\parindent{1,5em}
\usepackage{indentfirst}
\begin{document}
	
	\chead{Модуль 4}
	\rhead{Школа <<Симметрия>>}
	\begin{enumerate}
		\item Найдите площадь треугольника, две стороны которого равны \( 8 \) и \( 12 \), а угол между ними равен \( 30\degree \).
		\item Найдите площадь треугольника, две стороны которого равны \( 48  \) и \( 16 \), а угол между ними равен \( 30  \) градусов.
		\item Площадь треугольника \( ABC \) равна \( 4 \), \( DE\) – средняя линия, параллельная стороне \( AB \). Найдите площадь треугольника \( CDE \).
		\item Площадь треугольника \( ABC  \) равна \( 200 \). \( DE \) – средняя линия. Найдите площадь треугольника \( CDE \).
		\item У треугольника со сторонами \( 9 \) и \( 6  \) проведены высоты к этим сторонам. Высота, проведенная к первой стороне, равна \( 4 \). Чему равна высота, проведенная ко второй стороне?
		\item У треугольника со сторонами \( 6 \) и \( 2 \) проведены высоты к этим сторонам. Высота, проведенная к первой стороне, равна \( 1 \). Чему равна высота, проведенная ко второй стороне?
		\item В треугольнике \( ABC \) угол \( A \) равен \( 40 \degree\), внешний угол при вершине \( B \) равен \( 102 \degree\). Найдите угол \( C \). Ответ дайте в градусах.
		\item В треугольнике \( ABC  \) угол \( A \) равен \( 48 \degree\), внешний угол при вершине \( B \) равен \( 118 \degree\) . Найдите угол \( C \). Ответ дайте в градусах.
		\item Углы треугольника относятся как \(2:3:4\). Найдите меньший из них. Ответ дайте в градусах.
		\item Углы треугольника относятся как \(2:13:30\). Найдите меньший из них. Ответ дайте в градусах.
		\item В треугольнике \( ABC \) угол \( A \) равен \( 30\degree\), угол \( B \) – тупой, \(CH\) – высота, угол \( BCH \) равен \( 22 \degree \). Найдите угол \( ACB \). Ответ дайте в градусах.
		\item В треугольнике \( ABC \) угол \( A \) равен \( 70 \degree\), \( CH \) – высота, угол \( BCH \) равен \( 10\degree \). Найдите угол \( ACB \). Ответ дайте в градусах.
		\item В треугольнике \( ABC \) \( AD \) – биссектриса, угол \( C \) равен \( 50\degree \), угол \( CAD \) равен \( 28\degree \). Найдите угол \( B \). Ответ дайте в градусах.
		\item В треугольнике \( ABC \) \( AD \) – биссектриса, угол \( C \) равен \( 42\degree \), угол \( CAD \) равен \( 23 \degree\). Найдите угол \( B \). Ответ дайте в градусах.
		\item  В треугольнике \( ABC \) \( AD \) – биссектриса, угол \( C \) равен \( 30\degree \), угол \( BAD \) равен \( 22\degree \). Найдите угол \( ADB \). Ответ дайте в градусах.
		\item В треугольнике \( ABC \) \( AD \) – биссектриса, угол \( C \) равен \( 90\degree \), угол \( BAD \) равен \( 21\degree \). Найдите угол \( ADB \). Ответ дайте в градусах.
		\item В треугольнике \( ABC \) угол \( A \) равен \( 46\degree \), углы \( B \) и \( C \) – острые, высоты \( BD \) и \( CE \) пересекаются в точке \( O \). Найдите угол \( DOE \). Ответ дайте в градусах.
		\item В треугольнике \( ABC \) угол \( A \) равен \( 43\degree \), углы \( B \) и \( C \)  – острые, высоты \( BD \) и \( CE \) пересекаются в точке \( O \). Найдите угол \( DOE \). Ответ дайте в градусах.
		\item В треугольнике \( ABC \) угол \( A \) равен \( 41 \degree\), а углы \( B \) и \( C \) – острые, \( BD \) и \( CE \) – высоты, пересекающиеся в точке \(O\). Найдите угол \(DOE\). Ответ дайте в градусах.
		\item В треугольнике \( ABC \) угол \( A \) равен \( 135\degree \). Продолжения высот \( BD \) и \( CE \) пересекаются в точке M. Найдите угол \( DOE \). Ответ дайте в градусах.
		\item В треугольнике \( ABC \) угол \( B \) – тупой, \(AB=5\), \(BC=6\). Найдите величину угла, противолежащего стороне \( AC \), если площадь треугольника равна \( 7,5 \). Ответ дайте в градусах.
		\item В треугольнике \( ABC \) отрезок \( DE \) – средняя линия. Площадь треугольника \( CDE \) равна \( 38 \). Найдите площадь треугольника \( ABC \).
		\item В треугольнике \( ABC \) \(  DE \) – средняя линия. Площадь треугольника \( ADE \) равна \( 4 \). Найдите площадь треугольника \( ABC \).
		\item В треугольнике \( ABC \) угол \( C \) равен \( 90\degree \), \( AC=4,8 \),  \( \sin A=\dfrac{7}{25} \). Найдите \( AB \).
		\item В треугольнике \( ABC \) угол \( C \) равен \( 90 \degree \), \( AC=2 \),  \( \sin A=\dfrac{\sqrt{17}}{17} \). Найдите \( BC \).
		\item В треугольнике \( ABC \) угол \( C \) равен \( 90 \degree \), \( \tg A=\dfrac{33}{4\sqrt{33}} \), \( AC=4 \). Найдите \( АВ \).
		\item В треугольнике \( ABC \) угол \( C \) равен \( 90 \degree \), \( AC=14 \).  \( \cos A=0,7 \). Найдите \( AB \).
		\item В треугольнике \( ABC \) угол \( C \) равен \( 90\degree \), \( AC=3 \),  \( \tg A=\dfrac{12}{5} \). Найдите \( AB \).
		\item В треугольнике \( ABC \) угол \( C \) равен \( 90\degree \), \( CH \) – высота, \( AB=13 \),  \( \tg A=\dfrac{1}{5} \). Найдите \( AH \).
		\item В треугольнике \( ABC \) угол \( C \) равен \( 90 \degree \), \( CH \) – высота, \( AB=5 \),  \( \tg A=\dfrac{1}{7} \). Найдите \( AH \).
		\item В треугольнике \( АВС \) угол \( С \) равен \( 90 \degree \), \( CH \) – высота, \( BC=3 \),  \( \sin A=\dfrac{1}{6} \). Найдите \( АН \).
		\item В треугольнике \( ABC \) угол \( C \) равен \( 90 \degree \), \( CH \) – высота, \( BC=4 \),  \( \sin A = \dfrac{1}{4} \). Найдите \( AH \).
		\item В треугольнике \( ABC \) угол \( C \) равен \( 90 \degree \), \( CH \) – высота, \( BC=3 \),  \( \cos A =\dfrac{\sqrt{35}}{6} \). Найдите \( АН \).
		\item В треугольнике \( ABC \) угол \( C \) равен \( 90 \degree \), \( CH \) – высота, \( BC=35 \),  \( \cos A =\dfrac{\sqrt{33}}{7} \). Найдите \( AH \).
		\item В треугольнике \( ABC \) угол \( C \) равен \( 90\degree \), \( CH  \) – высота, \( BH = 12 \),  \( \tg A = \dfrac{2}{3} \). Найдите \( AH \).
		\item В треугольнике \( ABC \) угол \( C \) равен \( 90\degree \), \( CH  \) – высота, \(AH=3\),  \( \cos A = \dfrac{1}{2}\). Найдите \( AB \).
		\item Острые углы прямоугольного треугольника равны \( 24 \degree \) и \( 66 \degree \). Найдите угол между биссектрисой и медианой, проведенными из вершины прямого угла. Ответ дайте в градусах.
		\item Острые углы прямоугольного треугольника равны \( 81\degree \) и \( 9\degree \). Найдите угол между биссектрисой и медианой, проведенными из вершины прямого угла. Ответ дайте в градусах.
		\item Один острый угол прямоугольного треугольника на \( 32\degree \) больше другого. Найдите больший острый угол. Ответ дайте в градусах.
		\item Один острый угол прямоугольного треугольника на \( 1\degree \) больше другого. Найдите больший острый угол. Ответ дайте в градусах.
		\item Угол между биссектрисой и медианой прямоугольного треугольника, проведенными из вершины прямого угла, равен \( 14\degree \). Найдите меньший угол этого треугольника. Ответ дайте в градусах.
		\item В треугольнике \( ABC AC=BC\), \(AB=10\), высота \(AH\) равна \(3\). Найдите синус угла \(BAC\).
		\item Острый угол прямоугольного треугольника равен \( 32\degree \). Найдите острый угол, образованный биссектрисами этого и прямого углов треугольника. Ответ дайте в градусах.
		\item Острый угол прямоугольного треугольника равен \( 34\degree \). Найдите острый угол, образованный биссектрисами этого и прямого углов треугольника. Ответ дайте в градусах.
		\item В треугольнике \( ABC \) угол \( ACB \) равен \( 90\degree \), угол \( B \) равен \( 58 \degree\), \( CD \) – медиана. Найдите угол \( ACD \). Ответ дайте в градусах.
		\item В треугольнике \( ABC \) угол \( ACB \) равен \( 90\degree \), угол \( B \) равен \( 23\degree \), \( CD \) – медиана. Найдите угол \( ACD \). Ответ дайте в градусах.
	\end{enumerate}	
	\end{document}