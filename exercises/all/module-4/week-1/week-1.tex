\documentclass[12pt, a4paper]{article}
\usepackage{cmap} % Улучшенный поиск русских слов в полученном pdf-файле
\usepackage[T2A]{fontenc} % Поддержка русских букв
\usepackage[utf8]{inputenc} % Кодировка utf8
\usepackage[english, russian]{babel} % Языки: русский, английский
\usepackage{enumitem}
\usepackage{pscyr} % Нормальные шрифты
\usepackage{amsmath}
\usepackage{amsthm}
\usepackage{amssymb}
\usepackage{scrextend}
\usepackage{titling}
\usepackage{indentfirst}
\usepackage{cancel}
\usepackage{soulutf8}
\usepackage{wrapfig}
\usepackage{gensymb}
\usepackage[dvipsnames,table,xcdraw]{xcolor}
\usepackage{tikz}

%Русские символы в списке
\makeatletter
\AddEnumerateCounter{\asbuk}{\russian@alph}{щ}
\makeatother

%Дублирование знаков при переносе
\newcommand*{\hm}[1]{#1\nobreak\discretionary{}%
	{\hbox{$\mathsurround=0pt #1$}}{}}

\usepackage{graphicx}
\graphicspath{{pic/}}
\DeclareGraphicsExtensions{.pdf,.png,.jpg}

%Изменеие параметров листа
\usepackage[left=15mm,right=15mm,
top=2cm,bottom=2cm,bindingoffset=0cm]{geometry}

\usepackage{fancyhdr}
\pagestyle{fancy}
\usepackage{multicol}

\setlength\parindent{1,5em}
\usepackage{indentfirst}
\begin{document}
	
	\chead{Модуль 4}
	\rhead{Школа <<Симметрия>>}
	Треугольники.
	\begin{enumerate}
		\item Найдите площадь треугольника, две стороны которого равны \( 8 \) и \( 12 \), а угол между ними равен \( 30\degree \).
		\item Найдите площадь треугольника, две стороны которого равны \( 48  \) и \( 16 \), а угол между ними равен \( 30  \) градусов.
		\item Площадь треугольника \( ABC \) равна \( 4 \), \( DE\) – средняя линия, параллельная стороне \( AB \). Найдите площадь треугольника \( CDE \).
		\item Площадь треугольника \( ABC  \) равна \( 200 \). \( DE \) – средняя линия. Найдите площадь треугольника \( CDE \).
		\item Площадь треугольника \( ABC  \) равна \( 100 \). \( DE \) – средняя линия. Найдите площадь треугольника \( CDE \).
		\item У треугольника со сторонами \( 9 \) и \( 6  \) проведены высоты к этим сторонам. Высота, проведенная к первой стороне, равна \( 4 \). Чему равна высота, проведенная ко второй стороне?
		\item У треугольника со сторонами \( 6 \) и \( 2 \) проведены высоты к этим сторонам. Высота, проведенная к первой стороне, равна \( 1 \). Чему равна высота, проведенная ко второй стороне?
		\item В треугольнике \( ABC \) угол \( A \) равен \( 40 \degree\), внешний угол при вершине \( B \) равен \( 102 \degree\). Найдите угол \( C \). Ответ дайте в градусах.
		\item В треугольнике \( ABC  \) угол \( A \) равен \( 48 \degree\), внешний угол при вершине \( B \) равен \( 118 \degree\) . Найдите угол \( C \). Ответ дайте в градусах.
		\item Углы треугольника относятся как \(2:3:4\). Найдите меньший из них. Ответ дайте в градусах.
		\item Углы треугольника относятся как \(5:6:7\). Найдите больший из них. Ответ дайте в градусах.
		\item Углы треугольника относятся как \(2:13:30\). Найдите меньший из них. Ответ дайте в градусах.
		\item Углы треугольника относятся как \(2:13:30\). Найдите больший внешний угол. Ответ дайте в градусах.
		\item В треугольнике \( ABC \) угол \( A \) равен \( 30\degree\), угол \( B \) – тупой, \(CH\) – высота, угол \( BCH \) равен \( 22 \degree \). Найдите угол \( ACB \). Ответ дайте в градусах.
		\item В треугольнике \( ABC \) угол \( A \) равен \( 70 \degree\), \( CH \) – высота, угол \( BCH \) равен \( 10\degree \). Найдите угол \( ACB \). Ответ дайте в градусах.
		\item В треугольнике \( ABC \) \( AD \) – биссектриса, угол \( C \) равен \( 50\degree \), угол \( CAD \) равен \( 28\degree \). Найдите угол \( B \). Ответ дайте в градусах.
		\item В треугольнике \( ABC \) \( AD \) – биссектриса, угол \( C \) равен \( 42\degree \), угол \( CAD \) равен \( 23 \degree\). Найдите угол \( B \). Ответ дайте в градусах.
		\item  В треугольнике \( ABC \) \( AD \) – биссектриса, угол \( C \) равен \( 30\degree \), угол \( BAD \) равен \( 22\degree \). Найдите угол \( ADB \). Ответ дайте в градусах.
		\item В треугольнике \( ABC \) \( AD \) – биссектриса, угол \( C \) равен \( 90\degree \), угол \( BAD \) равен \( 21\degree \). Найдите угол \( ADB \). Ответ дайте в градусах.
		\item В треугольнике \( ABC \) угол \( A \) равен \( 46\degree \), углы \( B \) и \( C \) – острые, высоты \( BD \) и \( CE \) пересекаются в точке \( O \). Найдите угол \( DOE \). Ответ дайте в градусах.
		\item В треугольнике \( ABC \) угол \( A \) равен \( 43\degree \), углы \( B \) и \( C \)  – острые, высоты \( BD \) и \( CE \) пересекаются в точке \( O \). Найдите угол \( DOE \). Ответ дайте в градусах.
		\item В треугольнике \( ABC \) угол \( A \) равен \( 41 \degree\), а углы \( B \) и \( C \) – острые, \( BD \) и \( CE \) – высоты, пересекающиеся в точке \(O\). Найдите угол \(DOE\). Ответ дайте в градусах.
		\item В треугольнике \( ABC \) угол \( A \) равен \( 135\degree \). Продолжения высот \( BD \) и \( CE \) пересекаются в точке M. Найдите угол \( DOE \). Ответ дайте в градусах.
		\item В треугольнике \( ABC \) угол \( B \) – тупой, \(AB=5\), \(BC=6\). Найдите величину угла, противолежащего стороне \( AC \), если площадь треугольника равна \( 7,5 \). Ответ дайте в градусах.
		\item В треугольнике \( ABC \) отрезок \( DE \) – средняя линия. Площадь треугольника \( CDE \) равна \( 38 \). Найдите площадь треугольника \( ABC \).
		\item В треугольнике \( ABC \) \(  DE \) – средняя линия. Площадь треугольника \( ADE \) равна \( 4 \). Найдите площадь треугольника \( ABC \).
		\item В треугольнике \( ABC \) угол \( C \) равен \( 90\degree \), \( AC=4,8 \),  \( \sin A=\dfrac{7}{25} \). Найдите \( AB \).
		\item В треугольнике \( ABC \) угол \( C \) равен \( 90 \degree \), \( AC=2 \),  \( \sin A=\dfrac{\sqrt{17}}{17} \). Найдите \( BC \).
		\item В треугольнике \( ABC \) угол \( C \) равен \( 90 \degree \), \( \tg A=\dfrac{33}{4\sqrt{33}} \), \( AC=4 \). Найдите \( АВ \).
		\item В треугольнике \( ABC \) угол \( C \) равен \( 90 \degree \), \( AC=14 \).  \( \cos A=0,7 \). Найдите \( AB \).
		\item В треугольнике \( ABC \) угол \( C \) равен \( 90\degree \), \( AC=3 \),  \( \tg A=\dfrac{12}{5} \). Найдите \( AB \).
		\item В треугольнике \( ABC \) угол \( C \) равен \( 90\degree \), \( CH \) – высота, \( AB=13 \),  \( \tg A=\dfrac{1}{5} \). Найдите \( AH \).
		\item В треугольнике \( ABC \) угол \( C \) равен \( 90 \degree \), \( CH \) – высота, \( AB=5 \),  \( \tg A=\dfrac{1}{7} \). Найдите \( AH \).
		\item В треугольнике \( АВС \) угол \( С \) равен \( 90 \degree \), \( CH \) – высота, \( BC=3 \),  \( \sin A=\dfrac{1}{6} \). Найдите \( АН \).
		\item В треугольнике \( ABC \) угол \( C \) равен \( 90 \degree \), \( CH \) – высота, \( BC=4 \),  \( \sin A = \dfrac{1}{4} \). Найдите \( AH \).
		\item В треугольнике \( ABC \) угол \( C \) равен \( 90 \degree \), \( CH \) – высота, \( BC=3 \),  \( \cos A =\dfrac{\sqrt{35}}{6} \). Найдите \( АН \).
		\item В треугольнике \( ABC \) угол \( C \) равен \( 90 \degree \), \( CH \) – высота, \( BC=35 \),  \( \cos A =\dfrac{\sqrt{33}}{7} \). Найдите \( AH \).
		\item В треугольнике \( ABC \) угол \( C \) равен \( 90\degree \), \( CH  \) – высота, \( BH = 12 \),  \( \tg A = \dfrac{2}{3} \). Найдите \( AH \).
		\item В треугольнике \( ABC \) угол \( C \) равен \( 90\degree \), \( CH  \) – высота, \(AH=3\),  \( \cos A = \dfrac{1}{2}\). Найдите \( AB \).
		\item Острые углы прямоугольного треугольника равны \( 24 \degree \) и \( 66 \degree \). Найдите угол между биссектрисой и медианой, проведенными из вершины прямого угла. Ответ дайте в градусах.
		\item Острые углы прямоугольного треугольника равны \( 81\degree \) и \( 9\degree \). Найдите угол между биссектрисой и медианой, проведенными из вершины прямого угла. Ответ дайте в градусах.
		\item Один острый угол прямоугольного треугольника на \( 32\degree \) больше другого. Найдите больший острый угол. Ответ дайте в градусах.
		\item Один острый угол прямоугольного треугольника на \( 8\degree \) меньше другого. Найдите больший острый угол. Ответ дайте в градусах.
		\item Острые углы прямоугольного треугольника относятся друг к другу как \( 2:1 \). Найдите меньший острый угол. Ответ дайте в градусах.
		\item Один острый угол прямоугольного треугольника на \( 1\degree \) больше другого. Найдите больший острый угол. Ответ дайте в градусах.
		\item Угол между биссектрисой и медианой прямоугольного треугольника, проведенными из вершины прямого угла, равен \( 14\degree \). Найдите меньший угол этого треугольника. Ответ дайте в градусах.
		\item Угол между биссектрисой и медианой прямоугольного треугольника, проведенными из вершины прямого угла, равен \( 29\degree \). Найдите меньший угол этого треугольника. Ответ дайте в градусах.
		\item В треугольнике \( ABC AC=BC\), \(AB=10\), высота \(AH\) равна \(3\). Найдите синус угла \(BAC\).
		\item Острый угол прямоугольного треугольника равен \( 32\degree \). Найдите острый угол, образованный биссектрисами этого и прямого углов треугольника. Ответ дайте в градусах.
		\item Острый угол прямоугольного треугольника равен \( 34\degree \). Найдите острый угол, образованный биссектрисами этого и прямого углов треугольника. Ответ дайте в градусах.
		\item В треугольнике \( ABC \) угол \( ACB \) равен \( 90\degree \), угол \( B \) равен \( 58 \degree\), \( CD \) – медиана. Найдите угол \( ACD \). Ответ дайте в градусах.
		\item В треугольнике \( ABC \) угол \( ACB \) равен \( 90\degree \), угол \( B \) равен \( 23\degree \), \( CD \) – медиана. Найдите угол \( ACD \). Ответ дайте в градусах.
		\item В треугольнике \( ABC \) \(AC=BC=5\),  \( \sin A=\dfrac{7}{25} \).  Найдите \( AB \).
		\item В треугольнике  \( ABC \) \(AC=BC=16\),  \( \sin B=\dfrac{\sqrt{7}}{4} \).  Найдите \( AB \).
		\item В треугольнике \(ABC AC=BC\), \(AB=9,6\),  \( \sin A=\dfrac{7}{25} \).  Найдите \( AC \).
		\item В треугольнике \( ABC AC=BC \), \( AB = 8\sqrt{3}  \),  \( \sin A=0,5 \). Найдите \( AC \).
		\item В треугольнике \(ABC AC=BC=8\),  \( \cos A=0,5 \). Найдите \( АВ \).
		\item В треугольнике \( ABC AC=BC=18 \),  \( \cos A=0,5 \). Найдите \( AB \).
		\item В треугольнике \(ABC AC=BC\), \( AB=8 \),  \( \cos A=0,5 \). Найдите \( AC \).
		\item В треугольнике \(ABC AC=BC\), \( AB=10 \),  \( \cos A=0,5 \). Найдите \( AC \).
		\item В треугольнике \(ABC AC=BC=7\),  \( \tg A=\dfrac{33}{4\sqrt{33}} \).  Найдите \( AB \).
		\item В треугольнике \(ABC AC=BC=12 \),  \( \tg A=\dfrac{5}{\sqrt{20}} \).  Найдите \( AB \).
		\item В треугольнике \(ABC AC=BC\), \( AB=8 \),  \( \sin BAC=0,5 \). Найдите высоту \( AH \).
		\item В треугольнике \(ABC AC=BC\), \( AB=15\),  \( \sin BAC=0,6 \). Найдите высоту \( AH \).
		\item В треугольнике \(ABC AC=BC\), \( AH – \) высота, \( AB=8 \),  \( \cos BAC=0,5 \). Найдите \( BH \).
		\item В треугольнике \(ABC AC=BC\), \( AH – \) высота, \( AB=20 \),  \( \cos BAC=0,5 \). Найдите \( BH \).
		\item В треугольнике \(ABC AC=BC=4\sqrt{15}\),  \( \sin BAC=0,25 \). Найдите высоту \( AH \).
		\item В треугольнике \(ABC AC=BC=75\),  \( \sin BAC=0,96 \). Найдите высоту \( AH \).
		\item Угол при вершине, противолежащей основанию равнобедренного треугольника, равен \(30\degree \). Найдите боковую сторону треугольника, если его площадь равна \(25\).
		\item Угол при вершине, противолежащей основанию равнобедренного треугольника, равен \(30\degree \). Найдите боковую сторону треугольника, если его площадь равна \(676\).
		\item В треугольнике \(ABC AC=BC=6\), высота \( AH=3 \). Найдите угол \( C \). Ответ дайте в градусах.
		\item В треугольнике \(ABC AC=BC=28\), высота \( AH=14 \). Найдите угол \( C \). Ответ дайте в градусах.
		\item Один угол равнобедренного треугольника на \( 90 \degree\) больше другого. Найдите меньший угол. Ответ дайте в градусах.
		\item Один угол равнобедренного треугольника на \( 135 \) градусов больше другого. Найдите меньший угол. Ответ дайте в градусах.
		\item В треугольнике \(ABC AC=BC\), угол \( C \) равен \( 52\degree \). Найдите внешний угол \( CBD \). Ответ дайте в градусах.
		\item В треугольнике \(ABC AC=BC\), угол \( C \) равен \( 16\degree \). Найдите внешний угол \( CBD \). Ответ дайте в градусах.
		\item В треугольнике \(ABC AC=BC=2\sqrt{3} \), угол \( C =110\degree\) . Найдите высоту \( AH \).
		\item В треугольнике \(ABC AC=BC=28\sqrt{3} \), угол \( C =120\degree\) . Найдите высоту \( AH \).
		\item В треугольнике \(ABC AC=BC=10,2 \),  \( \tg A=\dfrac{8}{15} \).  Найдите \( AB \).
		\item В треугольнике \( ABC AC = BC \), \( AH – \) высота, \(AB=7\),  \( \tg BAC=\dfrac{33}{4\sqrt{33}} \).  Найдите \( BH \).
		\item В треугольнике \( ABC AC = BC \), \( AH – \) высота, \( AB=12 \),  \( \tg BAC=\dfrac{5}{\sqrt{20}} \).  Найдите BH.
		\item В треугольнике \( ABC AC = BC = 48\sqrt{3} \), угол \( C \) равен \( 120  \) градусов. Найдите высоту \( AH \).
		\item В треугольнике \( ABC AC = BC = 27 \), \( AH – \) высота,  \( \cos BAC=\dfrac{2}{3} \).  Найдите \( BH \).
		\item В треугольнике \( ABC AC = BC = 12 \), \( AH – \) высота,  \( \cos BAC=\dfrac{1}{2} \).  Найдите \( BH \).
		\item В треугольнике \(ABC\) известно, что \(AC=BC=21\),  \( \tg A=2\sqrt{2} \). Найдите длину стороны \( AB \).
		\item В треугольнике \( ABC AC = BC \), угол \( C \) равен \( 120 \) градусов, \( AC = 2\sqrt{3} \). Найдите \( AB \).
		\item В треугольнике \( ABC AC = BC \), угол \( C \) равен \( 120 \) градусов, \( AC =28\sqrt{3}  \). Найдите \( AB \).
	\end{enumerate}
		Параллелограммы.
	\begin{enumerate}
		\item В параллелограмме \(ABCD AB=3\), \(AD=21\),  \( \sin A=\dfrac{6}{7} \).  Найдите большую высоту параллелограмма.
		\item В параллелограмме \(ABCD AB=1\), \(AD=6\),  \( \sin A=\dfrac{2}{3} \).  Найдите большую высоту параллелограмма.
		\item Найдите площадь квадрата, если его диагональ равна \( 1 \).
		\item Найдите площадь квадрата, если его диагональ равна \( 6 \).
		\item Площадь прямоугольника равна \( 18 \). Найдите его большую сторону, если она на \( 3 \) больше меньшей стороны.
		\item Площадь прямоугольника равна \( 204 \). Найдите его большую сторону, если она на \( 5 \) больше меньшей стороны.
		\item Найдите периметр прямоугольника, если его площадь равна \( 18 \), а отношение соседних сторон равно \( 1:2 \).
		\item Найдите периметр прямоугольника, если его площадь равна \( 96 \), а отношение соседних сторон равно \( 3:8 \).
		\item Периметр прямоугольника равен \( 42 \), а площадь \( 98 \). Найдите большую сторону прямоугольника.
		\item Периметр прямоугольника равен \( 12 \), а площадь \( 8 \). Найдите большую сторону прямоугольника.
		\item Периметр прямоугольника равен \( 28 \), а диагональ равна \( 10 \). Найдите площадь этого прямоугольника.
		\item Периметр прямоугольника равен \( 8 \), а диагональ равна \( 3 \). Найдите площадь этого прямоугольника.
		\item Периметр прямоугольника равен \( 34 \), а площадь равна \( 60 \). Найдите диагональ этого прямоугольника.
		\item Периметр прямоугольника равен \( 60 \), а площадь равна \( 29,5 \). Найдите диагональ этого прямоугольника.
		\item Параллелограмм и прямоугольник имеют одинаковые стороны. Найдите острый угол параллелограмма, если его площадь равна половине площади прямоугольника. Ответ дайте в градусах.
		\item Стороны параллелограмма равны \( 9 \) и \( 15 \). Высота, опущенная на первую сторону, равна 10. Найдите высоту, опущенную на вторую сторону параллелограмма.
		\item Стороны параллелограмма равны \( 10 \) и \( 45 \). Высота, опущенная на первую сторону, равна 27. Найдите высоту, опущенную на вторую сторону параллелограмма.
		\item Площадь параллелограмма равна \( 40 \), две его стороны равны \( 5 \) и \( 10 \). Найдите большую высоту этого параллелограмма.
		\item Площадь параллелограмма равна \( 120 \), две его стороны равны \( 40 \) и \( 80 \). Найдите большую высоту этого параллелограмма.
		\item Найдите площадь ромба, если его высота равна \( 2 \), а острый угол \( 30\degree \).
		\item Найдите площадь ромба, если его высота равна \( 48 \), а острый угол \( 30 \degree \).
		\item Найдите площадь ромба, если его диагонали равны \( 4 \) и \( 12 \).
		\item Найдите площадь ромба, если его диагонали равны \( 4 \) и \( 6 \).
		\item Площадь ромба равна \( 18 \). Одна из его диагоналей равна \( 12 \). Найдите другую диагональ.
		\item Площадь ромба равна \( 47 \). Одна из его диагоналей равна \( 2 \). Найдите другую диагональ.
		\item Площадь ромба равна \( 6 \). Одна из его диагоналей в \( 3 \) раза больше другой. Найдите меньшую диагональ.
		\item Площадь ромба равна \(8\). Одна из его диагоналей в \(4\) раза больше другой. Найдите меньшую диагональ.
		\item Сумма двух углов параллелограмма равна \( 100\degree \). Найдите один из оставшихся углов. Ответ дайте в градусах.
		\item Сумма двух углов параллелограмма равна \( 88\degree \). Найдите один из оставшихся углов. Ответ дайте в градусах.
		\item Один угол параллелограмма больше другого на \( 70 \degree \). Найдите больший угол. Ответ дайте в градусах.
		\item Один угол параллелограмма больше другого на \( 64 \degree \). Найдите больший угол. Ответ дайте в градусах.
		\item Диагональ параллелограмма образует с двумя его сторонами углы \( 26 \degree \) и \( 34 \degree \). Найдите больший угол параллелограмма. Ответ дайте в градусах.
		\item Диагональ параллелограмма образует с двумя его сторонами углы \( 29 \degree \) и \( 12 \degree \). Найдите больший угол параллелограмма. Ответ дайте в градусах.
		\item Периметр параллелограмма равен \( 46 \). Одна сторона параллелограмма на \( 3 \) больше другой. Найдите меньшую сторону параллелограмма.
		\item Периметр параллелограмма равен \( 12 \). Одна сторона параллелограмма на \( 3 \) больше другой. Найдите меньшую сторону параллелограмма.
		\item Диагональ прямоугольника вдвое больше одной из его сторон. Найдите больший из углов, который образует диагональ со сторонами прямоугольника? Ответ выразите в градусах.
		\item Найдите высоту ромба, сторона которого равна \( \sqrt{3} \), а острый угол равен \( 60\degree\).
		\item Найдите высоту ромба, сторона которого равна \( 39\sqrt{3} \), а острый угол равен \( 60 \degree \).
		\item Найдите больший угол параллелограмма, если два его угла относятся как \( 3 : 7 \). Ответ дайте в градусах.
		\item Найдите больший угол параллелограмма, если два его угла относятся как \( 1 : 2 \). Ответ дайте в градусах.
		\item Найдите угол между биссектрисами углов параллелограмма, прилежащих к одной стороне. Ответ дайте в градусах.
		\item Две стороны параллелограмма относятся как \(3:4\), а периметр его равен \( 70 \). Найдите большую сторону параллелограмма.
		\item Две стороны параллелограмма относятся как \(3:7\), а периметр его равен \( 60 \). Найдите большую сторону параллелограмма.
		\item Биссектриса тупого угла параллелограмма делит противоположную сторону в отношении \(4:3\), считая от вершины острого угла. Найдите большую сторону параллелограмма, если его периметр равен \( 88 \).
		\item Биссектриса тупого угла параллелограмма делит противоположную сторону в отношении \( 1 : 3 \), считая от вершины острого угла. Найдите большую сторону параллелограмма, если его периметр равен \( 35 \).
		\item Точка пересечения биссектрис двух углов параллелограмма, прилежащих к одной стороне, принадлежит противоположной стороне. Меньшая сторона параллелограмма равна \( 5 \). Найдите его большую сторону.
		\item Точка пересечения биссектрис двух углов параллелограмма, прилежащих к одной стороне, принадлежит противоположной стороне. Меньшая сторона параллелограмма равна \( 50 \). Найдите его большую сторону.
		\item Найдите большую диагональ ромба, сторона которого равна \( \sqrt{3} \), а острый угол равен \( 60\degree \).
		\item Найдите большую диагональ ромба, сторона которого равна \( 0,5\sqrt{3} \), а острый угол равен \( 60 \degree\).
		\item Диагонали ромба относятся как \( 3:4 \). Периметр ромба равен \( 200 \). Найдите высоту ромба.
		\item Диагонали ромба относятся как \( 2 : 5 \). Периметр ромба равен \( 29 \). Найдите высоту ромба.
		\item Диагонали четырехугольника равны \( 4  \) и \( 5 \). Найдите периметр четырехугольника, вершинами которого являются середины сторон данного четырехугольника.
		\item Диагонали четырехугольника равны \( 34  \) и \( 7 \). Найдите периметр четырехугольника, вершинами которого являются середины сторон данного четырехугольника.
		\item В ромбе \( ABCD  \) угол \( ABC  \) равен \( 122\degree \). Найдите угол \( ACD \). Ответ дайте в градусах.
		\item В ромбе \( ABCD \) угол \( CDA \) равен \( 38\degree \). Найдите угол \( CAB \). Ответ дайте в градусах.
		\item В ромбе \( ABCD \) угол \( ACD \) равен \( 43\degree \). Найдите угол \( ABC \). Ответ дайте в градусах.
		\item В ромбе \( ABCD \) угол \( DBC \) равен \( 28\degree \). Найдите угол \( DAB \). Ответ дайте в градусах.
		\item Площадь параллелограмма \( ABCD \) равна \( 189 \). Точка \( E \) – середина стороны \( AD \). Найдите площадь трапеции \( AECB \).
		\item Площадь параллелограмма \( ABCD \) равна \(20\). Точка \( E \) – середина стороны \(CD\). Найдите площадь трапеции \(ABED\).
		\item Площадь параллелограмма \(ABCD\) равна \(153\). Найдите площадь параллелограмма \( A^{'} B^{'} C^{'} D^{'} \), вершинами которого являются середины сторон данного параллелограмма.
		\item Площадь параллелограмма \(ABCD\) равна \( 164 \). Найдите площадь параллелограмма \( A^{'} B^{'} C^{'} D^{'} \), вершинами которого являются середины сторон данного параллелограмма.
		\item Площадь параллелограмма \( ABCD \) равна \( 176 \). Точка \( E \) – середина стороны \( CD \). Найдите площадь треугольника \( ADE \).
		\item Площадь параллелограмма \( ABCD \) равна \( 106 \). Точка \( E \) – середина стороны \( CD \). Найдите площадь треугольника \( ADE \).
		\item Угол между стороной и диагональю ромба равен \( 54\degree \). Найдите острый угол ромба.
	\end{enumerate}
		Трапепции
	\begin{enumerate}
		\item Основания равнобедренной трапеции равны \(56\) и \( 65 \). Боковые стороны равны \( 25 \). Найдите синус острого угла трапеции.
		\item Основания равнобедренной трапеции равны \( 12  \) и \( 28 \). Боковые стороны равны \( 10 \). Найдите синус острого угла трапеции.
		\item Основания равнобедренной трапеции равны \( 43  \) и \( 73 \). Косинус острого угла трапеции равен \( \dfrac{5}{7} \).  Найдите боковую сторону.
		\item Основания равнобедренной трапеции равны \( 29  \) и \( 37 \). Косинус острого угла трапеции равен \( \dfrac{4}{9} \).  Найдите боковую сторону.
		\item Большее основание равнобедренной трапеции равно \( 34 \) . Боковая сторона равна \( 14 \). Синус острого угла равен  \( \dfrac{2\sqrt{10}}{7} \).  Найдите меньшее основание.
		\item Большее основание равнобедренной трапеции равно \( 26 \). Боковая сторона равна \( 18 \). Синус острого угла равен \( \dfrac{\sqrt{77}}{9} \).  Найдите меньшее основание.
		\item Основания равнобедренной трапеции равны \( 7  \) и \( 51 \). Тангенс острого угла равен \( \dfrac{5}{11} \).  Найдите высоту трапеции.
		\item Основания равнобедренной трапеции равны \( 76  \) и \( 48 \). Тангенс острого угла равен  \( \dfrac{17}{14} \).  Найдите высоту трапеции.
		\item Меньшее основание равнобедренной трапеции равно \( 23 \). Высота трапеции равна \( 39 \). Тангенс острого угла равен \( \dfrac{13}{8} \).  Найдите большее основание.
		\item Меньшее основание равнобедренной трапеции равно \( 6 \). Высота трапеции равна \( 33 \). Тангенс острого угла равен \( \dfrac{11}{5}\).  Найдите большее основание.
		\item Основания равнобедренной трапеции равны \( 17  \) и \( 87 \). Высота трапеции равна \( 14 \). Найдите тангенс острого угла.	
		\item Основания равнобедренной трапеции равны \( 26  \) и \( 23 \). Высота трапеции равна \( 0,9 \). Найдите тангенс острого угла.
		\item Основания равнобедренной трапеции равны \( 14  \) и \( 26 \), а ее периметр равен \( 60 \). Найдите площадь трапеции.
		\item Основания равнобедренной трапеции равны \( 3  \) и \( 13 \), а ее периметр равен \( 42 \). Найдите площадь трапеции.
		\item Основания равнобедренной трапеции равны \( 7  \) и \( 13 \), а ее площадь равна \( 40 \). Найдите периметр трапеции.
		\item Основания равнобедренной трапеции равны \( 8  \) и \( 20 \), а ее площадь равна \( 112 \). Найдите периметр трапеции.
		\item Найдите площадь прямоугольной трапеции, основания которой равны \( 6  \) и  \( 2 \), большая боковая сторона составляет с основанием угол \( 45\degree \).
		\item Найдите площадь прямоугольной трапеции, основания которой равны \( 12  \) и \( 16 \), большая боковая сторона составляет с основанием угол \( 45 \degree\).
		\item Основания прямоугольной трапеции равны \( 12  \) и \( 4 \). Ее площадь равна \( 64 \). Найдите острый угол этой трапеции. Ответ дайте в градусах.
		\item Основания прямоугольной трапеции равны \( 13  \) и \( 21 \). Ее площадь равна \( 136 \). Найдите острый угол этой трапеции. Ответ дайте в градусах.
		\item Основания равнобедренной трапеции равны \( 14  \) и \( 26 \), а ее боковые стороны равны \( 10 \). Найдите площадь трапеции.
		\item Основания равнобедренной трапеции равны \( 10  \) и \( 22 \), а ее боковые стороны равны \( 10 \). Найдите площадь трапеции.
		\item Основания равнобедренной трапеции равны \( 7  \) и \( 13 \), а ее площадь равна \( 40 \). Найдите боковую сторону трапеции.
		\item  Основания равнобедренной трапеции равны \( 3  \) и \( 9 \), а ее площадь равна \( 24 \). Найдите боковую сторону трапеции.
		\item Основания трапеции равны \( 18  \) и \( 6 \), боковая сторона, равная \( 7 \), образует с одним из оснований трапеции угол \( 150\degree\). Найдите площадь трапеции.
		\item Основания трапеции равны \( 10  \) и \( 22 \), боковая сторона, равная \( 9 \), образует с одним из оснований трапеции угол \( 150 \degree \). Найдите площадь трапеции.
		\item Основания трапеции равны \( 27  \) и \( 9 \), боковая сторона равна \( 8 \). Площадь трапеции равна \( 72 \). Найдите острый угол трапеции, прилежащий к данной боковой стороне. Ответ выразите в градусах.
		\item Основания трапеции равны \( 17  \) и \( 23 \), боковая сторона равна \( 12 \). Площадь трапеции равна \( 120 \). Найдите острый угол трапеции, прилежащий к данной боковой стороне. Ответ дайте в градусах.
		\item Чему равен больший угол равнобедренной трапеции, если известно, что разность противолежащих углов равна \( 50\degree\)? Ответ дайте в градусах.
		\item Чему равен больший угол равнобедренной трапеции, если известно, что разность противолежащих углов равна \( 66\degree \)? Ответ дайте в градусах.
		\item Средняя линия трапеции равна \( 43 \), а меньшее основание равно \( 35 \). Найдите большее основание трапеции.
		\item Средняя линия трапеции равна \( 28 \), а меньшее основание равно \( 18 \). Найдите большее основание трапеции.
		\item Основания трапеции равны \( 4  \) и \( 10 \). Найдите больший из отрезков, на которые делит среднюю линию этой трапеции одна из ее диагоналей.
		\item Основания трапеции равны \( 5  \) и \( 9 \). Найдите больший из отрезков, на которые делит среднюю линию этой трапеции одна из ее диагоналей.
		\item В равнобедренной трапеции большее основание равно \( 25 \), боковая сторона равна \( 10 \), угол между ними \( 60\degree  \). Найдите меньшее основание.
		\item В равнобедренной трапеции большее основание равно \( 28 \), боковая сторона равна \( 20 \), угол между ними \( 60 \degree\) . Найдите меньшее основание.
		\item В равнобедренной трапеции основания равны \( 12  \) и \( 27 \), острый угол равен \( 60 \degree\). Найдите ее периметр.
		\item В равнобедренной трапеции основания равны \( 13  \) и \( 26 \), острый угол равен \( 60\degree  \). Найдите ее периметр.
		\item Прямая, проведенная параллельно боковой стороне трапеции через конец меньшего основания, равного \( 4 \), отсекает треугольник, периметр которого равен \( 15 \). Найдите периметр трапеции.
		\item Прямая, проведенная параллельно боковой стороне трапеции через конец меньшего основания, равного \( 27 \), отсекает треугольник, периметр которого равен \( 55 \). Найдите периметр трапеции.
		\item Перпендикуляр, опущенный из вершины тупого угла на большее основание равнобедренной трапеции, делит его на части, имеющие длины \( 10  \) и \( 4 \). Найдите среднюю линию этой трапеции.
		\item Перпендикуляр, опущенный из вершины тупого угла на большее основание равнобедренной трапеции, делит его на части, имеющие длины \( 98  \) и \( 53 \). Найдите среднюю линию этой трапеции.
		\item Основания трапеции равны \( 3  \) и \( 2 \). Найдите отрезок, соединяющий середины диагоналей трапеции.
		\item Основания равнобедренной трапеции равны \( 15  \) и \( 9 \), один из углов равен \( 45  \degree\). Найдите высоту трапеции.
		\item Основания равнобедренной трапеции равны \( 21  \) и \( 15 \), один из углов равен \( 45  \degree\). Найдите высоту трапеции.
		\item Основания трапеции равны \( 6  \) и \( 16 \). Найдите отрезок, соединяющий середины диагоналей трапеции.
		\item В равнобедренной трапеции диагонали перпендикулярны. Высота трапеции равна \( 12 \). Найдите ее среднюю линию.
		\item В равнобедренной трапеции диагонали перпендикулярны. Высота трапеции равна \( 15 \). Найдите ее среднюю линию.
		\item Основания равнобедренной трапеции равны \( 6  \) и \( 12 \). Синус острого угла трапеции равен \( 0,8 \). Найдите боковую сторону.
		\item Основания равнобедренной трапеции равны \( 4 \)  и \( 16 \). Синус острого угла трапеции равен \( 0,6 \). Найдите боковую сторону.
		\item Высота трапеции равна \( 5 \), площадь равна \( 75 \). Найдите среднюю линию трапеции.
		\item Высота трапеции равна \( 9 \), площадь равна \( 45 \). Найдите среднюю линию трапеции.
	\end{enumerate}
		Окружности
	\begin{enumerate}
		\item Треугольник \( ABC \)  вписан в окружность с центром \( O \). Найдите угол \( BOC \), если угол \( BAC  \) равен \( 32\degree\).
		\item Найдите центральный угол \( AOB \), если он на \( 15 \degree\)  больше вписанного угла \( ACB \), опирающегося на ту же дугу. Ответ дайте в градусах.
		\item Найдите центральный угол \( AOB \), если он на \( 36  \degree\)  больше вписанного угла \( ACB \), опирающегося на ту же дугу. Ответ дайте в градусах.
		\item Чему равен острый вписанный угол, опирающийся на хорду, равную радиусу окружности? Ответ дайте в градусах.
		\item Радиус окружности равен \( 1 \). Найдите величину острого вписанного угла, опирающегося на хорду, равную \( \sqrt{3} \) . Ответ дайте в градусах.
		\item Чему равен тупой вписанный угол, опирающийся на хорду, равную радиусу окружности? Ответ дайте в градусах.
		\item Радиус окружности равен \( 1 \) . Найдите величину тупого вписанного угла, опирающегося на хорду, равную \( \sqrt{3} \) . Ответ дайте в градусах.
		\item Найдите вписанный угол, опирающийся на дугу, которая составляет \( \dfrac{1}{5} \)  окружности. Ответ дайте в градусах.
		\item Найдите вписанный угол, опирающийся на дугу, которая составляет \( \dfrac{17}{36} \)  окружности. Ответ дайте в градусах.
		\item Дуга окружности \( AC \), не содержащая точки \( B \), составляет \( 200\degree \). А дуга окружности \( BC \), не содержащая точки \( A \), составляет \( 80\degree \). Найдите вписанный угол \( ACB \) . Ответ дайте в градусах.
		\item Дуга окружности \( AC \), не содержащая точки \( B \), составляет \( 170\degree \). А дуга окружности \( BC \), не содержащая точки \( A \), составляет \( 52 \degree\). Найдите вписанный угол \( ACB \). Ответ дайте в градусах.
		\item В окружности с центром \( O AC \) и \( BD \) – диаметры. Вписанный угол \( ACB \)  равен \( 38 \degree \). Найдите центральный угол \( AOD \). Ответ дайте в градусах.
		\item В окружности с центром \( O AC \) и \( BD \) – диаметры. Вписанный угол \( ACB \)  равен \( 16 \degree \). Найдите центральный угол \( AOD \). Ответ дайте в градусах.
		\item В окружности с центром \(O AC\) и \( BD \) – диаметры. Центральный угол \( AOD \) равен \( 110\degree\). Найдите вписанный угол \( ACB \). Ответ дайте в градусах.
		\item В окружности с центром \( O AC \) и \( BD \) – диаметры. Центральный угол \( AOD \)  равен \( 132 \degree \). Найдите вписанный угол \( ACB \). Ответ дайте в градусах.
		\item Найдите угол \( ACB \), если вписанные углы \( ADB \) и \( DAE \) опираются на дуги окружности, градусные величины которых равны соответственно \( 118 \degree \)  и \(  38 \degree \) . Ответ дайте в градусах.
		\item Угол \( ACB \)  равен \( 42 \degree \). Градусная величина дуги \( AB \)  окружности, не содержащей точек \( D \)  и \( E \), равна \( 124 \) градусов. Найдите угол \( DAE \). Ответ дайте в градусах.
		\item Угол \( ACB \)  равен \( 29 \degree \). Градусная величина дуги \( AB \)  окружности, не содержащей точек \( D \)  и \( E \), равна \( 106 \) градусов. Найдите угол \( DAE \). Ответ дайте в градусах.
		\item Четырёхугольник \( ABCD \)  вписан в окружность. Угол \( ABD \)  равен\(  61\degree \), угол \( CAD \)  равен \( 37\degree \)  Найдите угол \( ABC \). Ответ дайте в градусах.
		\item Угол \( ABD \)  равен \( 53\degree \). Угол \( ВСА \)  равен \( 38\degree \). Найдите вписанный угол \( BCD \). Ответ дайте в градусах.
		\item Угол между двумя соседними сторонами правильного многоугольника, равен \( 160\degree \). Найдите число вершин многоугольника.
		\item Угол между двумя соседними сторонами правильного многоугольника, равен \( 156\degree\). Найдите число вершин многоугольника.
		\item Четырёхугольник \( ABCD \)  вписан в окружность. Угол \( ABC \)  равен \( 102\degree\), угол \( CAD \)  равен \( 46\degree \). Найдите угол \( ABD \). Ответ дайте в градусах.
		\item В треугольнике \( ABC \)  сторона \( AB  \) равна \( 3\sqrt{2} \), угол \( С \)  равен \( 135\degree\) . Найдите радиус описанной около этого треугольника окружности.
		\item В треугольнике \( ABC  \) сторона \( AB  \) равна \( 2\sqrt{3} \), угол \( С \)  равен \( 120\degree\). Найдите радиус описанной около этого треугольника окружности.
		\item Найдите хорду, на которую опирается угол \( 30\degree \), вписанный в окружность радиуса \( 3 \).
		\item Найдите хорду, на которую опирается угол \( 30\degree\), вписанный в окружность радиуса \( 19 \).
		\item Найдите хорду, на которую опирается угол \( 120 \degree\), вписанный в окружность радиуса \( \sqrt{3} \).
		\item Найдите хорду, на которую опирается угол \( 120\degree \), вписанный в окружность радиуса \( 2\sqrt{3} \).
		\item Хорда \( AB \)  делит окружность на две части, градусные величины которых относятся как \( 5:7 \). Под каким углом видна эта хорда из точки \( C \), принадлежащей меньшей дуге окружности? Ответ дайте в градусах.
		\item Хорда \( AB \)  делит окружность на две части, градусные величины которых относятся как \( 7:11 \). Под каким углом видна эта хорда из точки \( C \), принадлежащей меньшей дуге окружности? Ответ дайте в градусах.
		\item Хорда \( AB \)  стягивает дугу окружности в \( 92\degree\) . Найдите угол \( ABC \)  между этой хордой и касательной к окружности, проведенной через точку \( B \). Ответ дайте в градусах.
		\item Хорда \( AB \)  стягивает дугу окружности в \( 84 \degree\). Найдите угол \( ABC \)  между этой хордой и касательной к окружности, проведенной через точку \( B \). Ответ дайте в градусах.
		\item Через концы \( А \) и \( В \) дуги окружности с центром \( О  \) проведены касательные \( АС \)  и \( ВС \). Угол \( СAB \)  равен \( 32\degree\). Найдите угол \( AОB \). Ответ дайте в градусах.
		\item Через концы \( А \)  и \( В \)  дуги окружности с центром \( О \)  проведены касательные \( АС \)  и \( ВС \). Угол \( СAB \)  равен \(  61\degree \). Найдите угол \( AОB \). Ответ дайте в градусах.
		\item Через концы \( A \), \( B \)  дуги окружности в \( 62\degree \) проведены касательные \( AC \)  и \( BC \). Найдите угол \( ACB \). Ответ дайте в градусах.
		\item Через концы \( A \), \( B \)  дуги окружности в \( 34\degree \) проведены касательные \( AC \)  и \( BC \). Найдите угол \( ACB \). Ответ дайте в градусах.
		\item Касательные \( CA \)  и \( CB \) к окружности образуют угол \( ACB \), равный \( 122\degree\). Найдите величину меньшей дуги \( AB \), стягиваемой точками касания. Ответ дайте в градусах.
		\item Касательные \( CA \)  и \( CB \)  к окружности образуют угол \( ACB \), равный \( 58 \degree\). Найдите величину меньшей дуги \( AB \), стягиваемой точками касания. Ответ дайте в градусах.
		\item Найдите угол \( ACO \), если его сторона \( CA \) касается окружности, \( O \) – центр окружности, сторона \( CO \)  пересекает окружность в точке \( B \), дуга \( АВ \)  окружности, заключённая внутри этого угла равна \( 64\degree \). Ответ дайте в градусах.
		\item Найдите угол \( ACO \), если его сторона \( CA \) касается окружности, \( O \) – центр окружности, а меньшая дуга окружности \( AB\),  заключенная внутри этого угла, равна \( 19\degree \) . Ответ дайте в градусах.
		\item Угол \( ACO \) равен \( 28\degree \), где \( O \) – центр окружности. Его сторона \( CA \) касается окружности. Найдите величину меньшей дуги \( AB \) окружности, заключенной внутри этого угла. Ответ дайте в градусах.
		\item Угол \( ACO \) равен \( 48\degree \), где \( O \) – центр окружности. Его сторона \( CA \) касается окружности. Найдите величину меньшей дуги \( AB \) окружности, заключенной внутри этого угла. Ответ дайте в градусах.
		\item Найдите угол \( ACO \), если его сторона \( CA \) касается окружности, \( O \) – центр окружности, сторона \( CO \) пересекает окружность в точках \( B \) и \( D \), а дуга \( AD \) окружности, заключенная внутри этого угла, равна \( 116\degree \). Ответ дайте в градусах.
		\item Найдите угол \( ACO \), если его сторона \( CA \) касается окружности, \( O \) – центр окружности, а большая дуга \( AD \) окружности, заключенная внутри этого угла, равна \( 118 \degree\). Ответ дайте в градусах.
		\item Угол \( ACO \) равен \( 24\degree\). Его сторона \( CA \) касается окружности с центром в точке \( О \). Сторона \( CO \) пересекает окружность в точках \( B \) и \( D \). Найдите градусную меру дуги \( AD \) окружности, заключенной внутри этого угла. Ответ дайте в градусах.
		\item Угол \( ACO \) равен \( 30 \degree\). Его сторона \( CA \) касается окружности. Найдите градусную величину дуги \( AD \) окружности, заключенной внутри этого угла. Ответ дайте в градусах.
		\item Периметр треугольника равен \(12\), а радиус вписанной окружности равен \( 1 \). Найдите площадь этого треугольника.
		\item Периметр треугольника равен \( 6 \), а радиус вписанной окружности равен \( 1 \). Найдите площадь этого треугольника.
		\item Площадь треугольника равна \( 24 \), а радиус вписанной окружности равен \( 2 \). Найдите периметр этого треугольника.
		\item Площадь треугольника равна \( 16 \), а радиус вписанной окружности равен \( 2 \). Найдите периметр этого треугольника.
		\item Около окружности, радиус которой равен \( 3 \), описан многоугольник, периметр которого равен \( 20 \). Найдите его площадь.
		\item Около окружности, радиус которой равен \( 3 \), описан многоугольник, периметр которого равен \( 62 \). Найдите его площадь.
		\item Найдите радиус окружности, вписанной в правильный треугольник, высота которого равна \( 6 \).
		\item Найдите радиус окружности, вписанной в правильный треугольник, высота которого равна \( 123 \).
		\item Радиус окружности, вписанной в правильный треугольник, равен \( 6 \). Найдите высоту этого треугольника.
		\item Радиус окружности, вписанной в правильный треугольник, равен \( 17 \). Найдите высоту этого треугольника.
		\item Сторона правильного треугольника равна \( \sqrt{3} \). Найдите радиус окружности, вписанной в этот треугольник.
		\item Сторона правильного треугольника равна \( 30\sqrt{3} \). Найдите радиус окружности, вписанной в этот треугольник.
		\item Радиус окружности, вписанной в правильный треугольник, равен\( \dfrac{\sqrt{3}}{6} \).  Найдите сторону этого треугольника.
		\item Радиус окружности, вписанной в правильный треугольник, равен  \( \dfrac{11\sqrt{3}}{6} \).  Найдите сторону этого треугольника.
		\item Сторона ромба равна \( 1 \), острый угол равен \( 30 \degree\). Найдите радиус вписанной окружности этого ромба.
		\item Сторона ромба равна \( 74 \), острый угол равен \( 30 \degree \). Найдите радиус вписанной окружности этого ромба.
		\item Острый угол ромба равен \( 30 \degree\). Радиус вписанной в этот ромб окружности равен \( 2 \). Найдите сторону ромба.
		\item Острый угол ромба равен \( 30 \degree\). Радиус вписанной в этот ромб окружности равен \( 21,5 \). Найдите сторону ромба.
		\item Найдите сторону правильного шестиугольника, описанного около окружности, радиус которой равен \( \sqrt{3} \).
		\item Найдите сторону правильного шестиугольника, описанного около окружности, радиус которой равен \( 25\sqrt{3} \).
		\item Найдите радиус окружности, вписанной в правильный шестиугольник со стороной  \( \sqrt{3} \).
		\item Найдите радиус окружности, вписанной в правильный шестиугольник со стороной \( 44\sqrt{3} \).
		\item Катеты равнобедренного прямоугольного треугольника равны \( 2+\sqrt{2} \). Найдите радиус окружности, вписанной в этот треугольник.
		\item Катеты равнобедренного прямоугольного треугольника равны \( 70+35\sqrt{2} \). Найдите радиус окружности, вписанной в этот треугольник.
		\item В треугольнике \( ABC \) стороны \( AC = 4 \), \( BC = 3 \), угол \( C \) равен \( 90\degree\). Найдите радиус вписанной окружности.
		\item В треугольнике \( ABC AC = 20 \), \( BC = 4,5 \), угол \( C \) равен \( 90\degree\). Найдите радиус вписанной окружности.
		\item Боковые стороны равнобедренного треугольника равны \( 5 \), основание равно \( 6 \). Найдите радиус вписанной окружности.
		\item Боковые стороны равнобедренного треугольника равны \( 125 \), основание равно \( 150 \). Найдите радиус вписанной окружности.
		\item Окружность, вписанная в равнобедренный треугольник, делит в точке касания одну из боковых сторон на два отрезка, длины которых равны \( 5  \) и \( 3 \), считая от вершины, противолежащей основанию. Найдите периметр треугольника.
		\item Окружность, вписанная в равнобедренный треугольник, делит в точке касания одну из боковых сторон на два отрезка, длины которых равны \( 19  \) и \( 2 \), считая от вершины, противолежащей основанию. Найдите периметр треугольника.
		\item Боковые стороны трапеции, описанной около окружности, равны \( 3  \) и \(  5 \). Найдите среднюю линию трапеции.
		\item Боковые стороны трапеции, описанной около окружности, равны \( 13 \) и \( 4 \). Найдите среднюю линию трапеции.
		\item Около окружности описана трапеция, периметр которой равен \( 40 \). Найдите длину её средней линии.
		\item Около окружности описана трапеция, периметр которой равен \( 36 \). Найдите длину её средней линии.
		\item Периметр прямоугольной трапеции, описанной около окружности, равен \( 22 \), ее большая боковая сторона равна \( 7 \). Найдите радиус окружности.
		\item Периметр прямоугольной трапеции, описанной около окружности, равен \( 100 \), ее большая боковая сторона равна \( 49 \). Найдите радиус окружности.
		\item В четырехугольник \(ABCD\)вписана окружность, \(AB=10\), \(CD=16\). Найдите периметр четырехугольника \(ABCD\).
		\item В четырехугольник \(ABCD\)вписана окружность, \(AB=4\), \(CD=10\). Найдите периметр четырехугольника \(ABCD\).
		\item Периметр четырехугольника, описанного около окружности, равен \( 24 \), две его стороны равны \( 5  \) и \( 6 \). Найдите большую из оставшихся сторон.
		\item Периметр четырехугольника, описанного около окружности, равен \( 48 \), две его стороны равны \( 1  \) и \( 7 \). Найдите большую из оставшихся сторон.
		\item В четырехугольник \(ABCD\)вписана окружность, \(AB=10\), \( BC=11\) и \( CD=15 \). Найдите четвертую сторону четырехугольника.
		\item В четырёхугольник \( ABCD  \) вписана окружность, \( AB=10 \), \( BC=8 \), \( CD=16 \). Найдите длину стороны \( AD \).
		\item К окружности, вписанной в треугольник \( ABC \), проведены три касательные. Периметры отсеченных треугольников равны \( 6 \), \( 8 \), \( 10 \). Найдите периметр данного треугольника.
		\item К окружности, вписанной в треугольник \( ABC \), проведены три касательные. Периметры отсеченных треугольников равны \( 7 \), \( 18 \), \( 34 \). Найдите периметр данного треугольника.
		\item В треугольнике \( ABC  \) известно, что \( АС = 36 \), \( ВС = 15 \), а угол \( C=90 \degree \). Найдите радиус вписанной в этот треугольник окружности.
		\item В четырёхугольник \(ABCD\), периметр которого равен \( 54 \), вписана окружность, \(AB=18\). Найдите длину стороны \(CDГипотенуза прямоугольного треугольника равна 12. Найдите радиус описанной окружности этого треугольника.\).
		\item Точки \( A \), \( B \), \( C \), расположенные на окружности, делят ее на три дуги, градусные величины которых относятся как \( 1:3:5 \). Найдите больший угол треугольника \( ABC \). Ответ дайте в градусах.
		\item Точки \( A \), \( B \), \( C \), расположенные на окружности, делят ее на три дуги, градусные величины которых относятся как \( 3:13:20 \). Найдите больший угол треугольника \( ABC \). Ответ дайте в градусах.
		\item Угол \( A  \) четырехугольника \( ABCD \), вписанного в окружность, равен \( 58\degree\). Найдите угол \( C  \) этого четырехугольника. Ответ дайте в градусах.
		\item Угол \( A  \) четырехугольника \( ABCD \), вписанного в окружность, равен \( 132\degree  \). Найдите угол \( C  \) этого четырехугольника. Ответ дайте в градусах.
		\item Стороны четырехугольника \( ABCD  \) \( AB \), \( BC \), \( CD  \)и \( AD  \) стягивают дуги описанной окружности, градусные величины которых равны соответственно \( 95 \degree\), \( 49 \degree \), \( 71\degree \), \( 145 \degree\). Найдите угол \( B  \) этого четырехугольника. Ответ дайте в градусах.
		\item Стороны четырехугольника \( ABCD  \) \( AB \), \( BC \), \( CD  \)и \( AD  \) стягивают дуги описанной окружности, градусные величины которых равны соответственно \( 60  \degree\), \( 53  \degree\), \( 75  \degree\), \( 172  \degree\). Найдите угол \( B  \) этого четырехугольника. Ответ дайте в градусах.
		\item Точки \( A \), \( B \), \( C \), \( D \), расположенные на окружности, делят эту окружность на четыре дуги \( AB \), \( BC \), \( CD  \) и \( AD \), градусные величины которых относятся соответственно как \( 4:2:3:6 \). Найдите угол \( A  \) четырехугольника \( ABCD \). Ответ дайте в градусах.
		\item Точки \( A \), \( B \), \( C \), \( D \), расположенные на окружности, делят эту окружность на четыре дуги \( AB \), \( BC \), \( CD  \) и \( AD \), градусные величины которых относятся соответственно как \( 2:5:8:21 \). Найдите угол \( A  \) четырехугольника \( ABCD \). Ответ дайте в градусах.
		\item Четырехугольник \( ABCD  \) вписан в окружность. Угол \( ABD  \) равен \( 75\degree \), угол \( CAD  \) равен \( 35\degree\). Найдите угол \( ABC \). Ответ дайте в градусах.
		\item Четырехугольник \( ABCD  \) вписан в окружность. Угол \( ABD  \) равен \( 23  \degree\), угол \( CAD  \) равен \( 39  \degree\). Найдите угол \( ABC \). Ответ дайте в градусах.
		\item Четырехугольник \( ABCD  \) вписан в окружность. Угол \( ABC  \) равен \( 80  \degree\), угол \( ABD  \) равен \( 48  \degree\). Найдите угол \( CAD \). Ответ дайте в градусах. 
		\item Сторона правильного треугольника равна \( \sqrt{3} \). Найдите радиус окружности, описанной около этого треугольника.
		\item Сторона правильного треугольника равна \( 40\sqrt{3}\). Найдите радиус окружности, описанной около этого треугольника.
		\item Радиус окружности, описанной около правильного треугольника, равен \( \sqrt{3} \). Найдите сторону этого треугольника.
		\item Радиус окружности, описанной около правильного треугольника, равен \( 7\sqrt{3} \). Найдите сторону этого треугольника.
		\item Высота правильного треугольника равна \( 3 \). Найдите радиус окружности, описанной около этого треугольника.
		\item Высота правильного треугольника равна \( 141 \). Найдите радиус окружности, описанной около этого треугольника.
		\item Радиус окружности, описанной около правильного треугольника, равен \( 3 \). Найдите высоту этого треугольника.
		\item Радиус окружности, описанной около правильного треугольника, равен \( 14 \). Найдите высоту этого треугольника.
		\item Гипотенуза прямоугольного треугольника равна \( 12 \). Найдите радиус описанной окружности этого треугольника.
		\item Гипотенуза прямоугольного треугольника равна \( 62 \). Найдите радиус описанной окружности этого треугольника.
		\item Радиус окружности, описанной около прямоугольного треугольника, равен \( 4 \). Найдите гипотенузу этого треугольника.
		\item Радиус окружности, описанной около прямоугольного треугольника, равен \( 21 \). Найдите гипотенузу этого треугольника.
		\item В треугольнике \( ABC  \) \( AC=4 \), \( BC=3 \), угол \( C \) равен \( 90\degree\). Найдите радиус описанной окружности этого треугольника.
		\item В треугольнике \( ABC  \) \( AC=34 \), \( BC=\sqrt{365} \), угол \( C \) равен \( 90\degree  \). Найдите радиус описанной окружности этого треугольника.
		\item Боковая сторона равнобедренного треугольника равна \( 1 \), угол при вершине, противолежащей основанию, равен \( 120\degree\). Найдите диаметр описанной окружности этого треугольника.
		\item Боковая сторона равнобедренного треугольника равна \( 6 \), угол при вершине, противолежащей основанию, равен \( 120  \degree\). Найдите диаметр описанной окружности этого треугольника.
		\item Чему равна сторона правильного шестиугольника, вписанного в окружность, радиус которой равен \( 6\)?
		\item Чему равна сторона правильного шестиугольника, вписанного в окружность, радиус которой равен \( 43 \)?
		\item Сторона \( AB  \) треугольника \( ABC  \) равна \( 1 \). Противолежащий ей угол \( C  \) равен \( 30\degree\). Найдите радиус окружности, описанной около этого треугольника.
		\item Сторона \( AB  \) треугольника \( ABC  \) равна \(  33 \). Противолежащий ей угол \( C  \) равен \( 30  \degree\). Найдите радиус окружности, описанной около этого треугольника.
		\item Одна сторона треугольника равна радиусу описанной окружности. Найдите острый угол треугольника, противолежащий этой стороне. Ответ дайте в градусах
		\item Угол \( C \) треугольника \( ABC \), вписанного в окружность радиуса \( 3 \), равен \( 30\degree\). Найдите сторону \( AB  \) этого треугольника.
		\item Угол \( C  \) треугольника \( ABC \), вписанного в окружность радиуса 33, равен \( 30  \degree\). Найдите сторону \( AB  \) этого треугольника.
		\item Сторона \( AB  \) треугольника \( ABC  \) равна \( 1 \). Противолежащий ей угол \( C  \) равен \( 150\degree\). Найдите радиус окружности, описанной около этого треугольника.
		\item Сторона \( AB  \) треугольника \( ABC  \) равна \( 40 \). Противолежащий ей угол \( C  \) равен \( 150  \degree\). Найдите радиус окружности, описанной около этого треугольника.
		\item Сторона \( AB  \) треугольника \( ABC  \) c тупым углом \( C  \) равна радиусу описанной около него окружности. Найдите угол \( C \). Ответ дайте в градусах.
		\item Боковые стороны равнобедренного треугольника равны \( 40 \), основание равно \( 48 \). Найдите радиус описанной окружности этого треугольника.
		\item Боковые стороны равнобедренного треугольника равны \( 20 \), основание равно \( 24 \). Найдите радиус описанной окружности этого треугольника.
		\item Около трапеции описана окружность. Периметр трапеции равен \( 22 \), средняя линия равна \( 5 \). Найдите боковую сторону трапеции.
		\item Около трапеции описана окружность. Периметр трапеции равен \( 24 \), средняя линия равна \( 4 \). Найдите боковую сторону трапеции.
		\item Боковая сторона равнобедренной трапеции равна ее меньшему основанию, угол при основании равен \( 60\degree\), большее основание равно \( 12 \). Найдите радиус описанной окружности этой трапеции.
		\item Боковая сторона равнобедренной трапеции равна ее меньшему основанию, угол при основании равен \( 60  \degree\) большее основание равно \( 30 \). Найдите радиус описанной окружности этой трапеции.
		\item Основания равнобедренной трапеции равны \( 8  \) и \( 6 \). Радиус описанной окружности равен \( 5 \). Центр окружности лежит внутри трапеции. Найдите высоту трапеции.
		\item Основания равнобедренной трапеции равны \( 192  \) и \( 56 \). Радиус описанной окружности равен \( 100 \). Найдите высоту трапеции.
		\item Два угла вписанного в окружность четырехугольника равны \( 82\degree \) и \( 58\degree \). Найдите больший из оставшихся углов. Ответ дайте в градусах.
		\item Два угла вписанного в окружность четырехугольника равны \( 17\degree \) и \( 45  \degree\). Найдите больший из оставшихся углов. Ответ дайте в градусах.
		\item Периметр правильного шестиугольника равен \( 72 \). Найдите диаметр описанной окружности.
		\item Периметр правильного шестиугольника равен \( 54 \). Найдите диаметр описанной окружности.
		\item Угол между двумя соседними сторонами правильного многоугольника, вписанного в окружность, равен \( 108\degree \). Найдите число вершин многоугольника.
		\item Угол между двумя соседними сторонами правильного многоугольника, вписанного в окружность, равен \( 165\degree \). Найдите число вершин многоугольника.
		\item Одна сторона треугольника равна \( \sqrt{2} \), радиус описанной окружности равен \( 1 \). Найдите острый угол треугольника, противолежащий этой стороне. Ответ дайте в градусах.
	\end{enumerate}
	\end{document}