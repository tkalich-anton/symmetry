\begin{ex}
	\begin{condition}
		В четырехугольнике \( MNPQ  \) расположены две непересекающиеся окружности так, что одна из них касается сторон \( MN  \), \( NP  \) и \( PQ \), а другая – сторон \( MN  \), \( MQ  \) и \( PQ \). Точки \( B \) и \( A  \) лежат соответственно на сторонах \( MN  \) и \( PQ \), причем отрезок \( AB  \) касается обеих окружностей. Найдите сторону \( MQ \), если \( NP = b  \) и периметр четырехугольника \( BAQM  \) больше периметра четырехугольника \( ABNP  \) на \( 2p \).
	\end{condition}
	\answer{?}
\end{ex}