\begin{ex}
	\begin{condition}
	15-го января планируется взять кредит в банке на 19 месяцев. Условия его возврата таковы:
	
	--- 1-го числа каждого месяца долг возрастёт на \( r \% \) по сравнению с концом предыдущего месяца;
	
	--- со 2-го по 14-е число каждого месяца необходимо выплатить часть долга;
	
	--- 15-го числа каждого месяца долг должен быть на одну и ту же сумму меньше долга на 15-е число предыдущего месяца.
	
	Известно, что общая сумма выплат после полного погашения кредита на 30\% больше суммы, взятой в кредит. Найдите \( r \).
	\end{condition}
	\answer{3\%}
\end{ex}