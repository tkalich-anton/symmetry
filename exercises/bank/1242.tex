\begin{ex}
	\begin{condition}
		Для получения на экране увеличенного изображения лампочки в лаборатории используется собирающая линза с главным фокусным расстоянием\(  f = 30 \) см. Расстояние \( d_1  \)от линзы до лампочки может изменяться в пределах от \( 30 \) до \( 50 \) см, а расстояние \( d_2 \) от линзы до экрана – в пределах от \( 150 \) до \( 180 \) см. Изображение на экране будет четким, если выполнено соотношение \( \dfrac{1}{d_1}+\dfrac{1}{d_2}=\dfrac{1}{f} \). Укажите, на каком наименьшем расстоянии от линзы можно поместить лампочку, чтобы еe изображение на экрана было чётким. Ответ выразите в сантиметрах.
	\end{condition}
	\answer{?}
\end{ex}