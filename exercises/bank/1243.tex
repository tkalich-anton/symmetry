\begin{ex}
	\begin{condition}
		Гоночный автомобиль разгоняется на прямолинейном участке шоссе с постоянным ускорением \( a \) км/ч\( ^2 \). Скорость \( v \)  в конце пути вычисляется по формуле \( v=\sqrt{2la} \), где \( l \) -- пройденный автомобилем путь в км. Определите ускорение, с которым должен двигаться автомобиль, чтобы, проехав \( 250 \) метров, приобрести скорость \( 60 \) км/ч. Ответ выразите в \( \text{км/ч}^2 \).
	\end{condition}
	\answer{\( 7200\hspace{1mm}\text{км/ч}^2 \)}
\end{ex}