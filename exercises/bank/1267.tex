\begin{ex}
	\begin{condition}
		Перед отправкой тепловоз издал гудок с частотой \( f_0 = 440 \) Гц. Чуть позже издал гудок подъезжающий к платформе тепловоз. Из-за эффекта Доплера частота второго гудка f больше первого: она зависит от скорости тепловоза по закону \( f(\upsilon )=\dfrac{f_o}{1-\frac{\upsilon}{c}} \) (Гц), где \( c \) -- скорость звука (в м/с). Человек, стоящий на платформе, различает сигналы по тону, если они отличаются не менее чем на \( 10 \) Гц. Определите, с какой минимальной скоростью приближался к платформе тепловоз, если человек смог различить сигналы, а \( c = 315 \) м/с. Ответ выразите в м/с.
	\end{condition}
	\answer{\( 7 \)}
\end{ex}