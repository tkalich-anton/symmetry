\begin{ex}
	\begin{condition}
	31 декабря 2013 года Сергей взял в банке 9 930 000 рублей в кредит под 10\% годовых. Схема выплаты кредита следующая: 31 декабря каждого следующего года банк начисляет проценты на оставшуюся сумму долга (то есть увеличивает долг на 10\%), затем Сергей переводит в банк определённую сумму ежегодного платежа. Какой должна быть сумма ежегодного платежа, чтобы Сергей выплатил долг тремя равными ежегодными платежами?
	\end{condition}
	\answer{3 993 000 рублей}
\end{ex}