\begin{ex}
	\begin{condition}
		Некоторая компания продает свою продукцию по цене \(  p=500  \) руб. за единицу, переменные затраты на производство одной единицы продукции составляют  \( v=300 \) руб., постоянные расходы предприятия \( f=700000 \) руб. в месяц. Месячная операционная прибыль предприятия (в рублях) вычисляется по формуле  \( \pi(q)=q(p-v)-f \). Определите месячный объем производства \(  q \) (единиц продукции), при котором месячная операционная прибыль предприятия будет равна \( 300000 \) руб.
	\end{condition}
	\answer{?}
\end{ex}