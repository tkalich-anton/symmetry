\begin{ex}
	\begin{condition}
		Плоский замкнутый контур площадью \( S = 0,5 \) м\( ^2 \) находится в магнитном поле, индукция которого равномерно возрастает. При этом согласно закону электромагнитной индукции Фарадея в контуре появляется ЭДС индукции, значение которой, выраженное в вольтах, определяется формулой \( E = aS\cos\alpha \), где  \( \alpha \) – острый угол между направлением магнитного поля и перпендикуляром к контуру, \(  a=4\cdot10^{-4}\) Тл/с – постоянная, \( S \) – площадь замкнутого контура, находящегося в магнитном поле (в м\( ^2 \)). При каком минимальном угле \( \alpha \) (в градусах) ЭДС индукции не будет превышать \( 10^{-4} \) В?
	\end{condition}
	\answer{\( 60\degree \)}
\end{ex}