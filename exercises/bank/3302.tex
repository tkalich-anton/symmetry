\begin{ex}[type=equation]
	\begin{condition}
		$\ \sin 4x =\dfrac{\sqrt{3}}{2}. $
	\end{condition}
	\answer{$(-1)^n \dfrac{\pi}{12}+\dfrac{\pi n}{4}.$}
\end{ex}