\begin{ex}
	\begin{condition}
		Докажите, что диагонали четырехугольника с равными сторонами взаимно перпендикулярны.
	\end{condition}
	\answer{?}
\end{ex}