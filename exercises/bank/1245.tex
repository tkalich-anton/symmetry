\begin{ex}
	\begin{condition}
		Небольшой мячик бросают под острым углом \( \alpha \) к плоской горизонтальной поверхности земли. Расстояние, которое пролетает мячик, вычисляется в метрах по формуле
		\[ L=\dfrac{V_0 ^2}{g}\cdot \sin2\alpha, \]
		где \( V_0=14 \) м/с -- начальная скорость мячика, а \( g=10\hspace{1mm}\text{м/с}^2 \) -- ускорение свободного падения. При каком наименьшем значении угла (в градусах) мячик перелетит реку шириной \( 9,8 \) м?
	\end{condition}
	\answer{\( 15\degree \)}
\end{ex}