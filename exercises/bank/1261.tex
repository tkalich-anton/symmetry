\begin{ex}
	\begin{condition}
		Для обогрева помещения, температура в котором поддерживается на уровне \( T_n=20\hspace{1mm}\degree C \), через радиатор отопления пропускают воду по проходящей через трубу воды \( m=0,3 \) кг/с. Проходя по трубе расстояние \( x \), вода охлаждается от начальной температуры \( T_b=60\degree C \) до температуры \( T\hspace{1mm}\degree C \), причем
		\[ x= \dfrac{\alpha\cdot c\cdot m}{\gamma}\cdot\log_2\dfrac{T_b-T_n}{T-T_b}, \]
		где \( c=4200 \) \( \dfrac{\text{Дж}}{\text{кг} \cdot \degree C} \) – теплоемкость воды, \( \gamma=21\hspace{1mm}\dfrac{\text{Вт}}{\text{м} \cdot\degree C} \) – коэффициент теплообмена, а \( \alpha =0,7 \) – постоянная. Найдите, до какой температуры (в градусах Цельсия) охладится вода, если длина трубы радиатора равна \( 84 \) м.
	\end{condition}
	\answer{\( 30\degree C \)}
\end{ex}