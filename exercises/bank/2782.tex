\begin{ex}
	\begin{condition}
		В треугольнике \( ABC  \) известно, что \( AB=AC  \) и угол \( BAC  \) тупой. Пусть \( BD \) – биссектриса треугольника \( ABC \), \( M \) – основание перпендикуляра, опущенного из точки \( A  \) на сторону \( BC \), \( E \) – основание перпендикуляра, опущенного из точки \( D  \) на сторону \( BC \). Через точку \( D  \) проведён также перпендикуляр к \( BD  \) до пересечения со стороной \( BC  \) в точке \( F \). Известно, что \( ME=FC=12 \). Найдите площадь треугольника \( ABC \).
	\end{condition}
	\answer{?}
\end{ex}