\begin{ex}
	\begin{condition}
		Датчик сконструирован таким образом, что его антенна ловит радиосигнал, который затем преобразуется в электрический сигнал, изменяющийся со временем по закону \( U=U_0\sin(\omega t+\phi) \), где \( t \) -- время в секундах, амплитуда \( U_0=2 \) В, частота \( \omega=120\degree\text{/с} \), фаза \( \phi=-30\degree \). Датчик настроен так, что если напряжение в нем не ниже, чем \( 1 \) В, загорается лампочка. Какую часть времени (в процентах) на протяжении первой секунды после начала работы лампочка будет гореть?
	\end{condition}
	\answer{\( 50\% \)}
\end{ex}