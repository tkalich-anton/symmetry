\begin{ex}
	\begin{condition}
		В правильной треугольной призме \( ABCA_1B_1C_1 \) сторона основания равна \( 4 \), а боковое ребро равно \( 2 \). Точка \( M \) --- середина ребра \( A_1C_1 \), а точка \( O \) --- точка пересечения диагоналей боковой грани \( ABB_1A_1 \).
		\begin{enumcols}[label=\asbuk*)]
			\item Докажите, что точка пересечения диагоналей четырёхугольника, являющегося сечением призмы \( ABCA_1B_1C_1 \) плоскостью \( AMB \), лежит на отрезке ,\( OC_1 \).
			\item Найдите угол между прямой \( OC_1 \), и плоскостью \( AMB \).
		\end{enumcols}
	\end{condition}
	\answer{\( \arcsin\dfrac{3\sqrt{3}}{\sqrt{91}} \)}
\end{ex}