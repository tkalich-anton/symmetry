\begin{ex}
	\begin{condition}
		В розетку электросети подключены приборы, общее сопротивление которых составляет \( R_1=90 \) Ом. Параллельно с ними в розетку предполагается подключить электрообогреватель. Определите наименьшее возможное сопротивление \( R_2 \) этого электрообогревателя, если известно, что при параллельном соединении двух проводников с сопротивлениями \( R_1 \) Ом и \( R_2 \) Ом их общее сопротивление даeтся формулой \( R_{\text{общ}}=\dfrac{R_1R_2}{R_1+R_2} \) (Ом), а для нормального функционирования электросети общее сопротивление в ней должно быть не меньше \( 9 \) Ом. Ответ выразите в омах.
	\end{condition}
	\answer{10}
\end{ex}