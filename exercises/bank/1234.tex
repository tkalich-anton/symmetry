\begin{ex}
	\begin{condition}
		При температуре \( 0\degree  \) рельс имеет длину \(  l_0=10\) м. При возрастании температуры происходит тепловое расширение рельса, и его длина, выраженная в метрах, меняется по закону \( l(t)=l_0(1+\alpha\cdot t) \), где \( \alpha = 1,2\cdot10^{-5} \) \( (\degree C)^{-1} \) – коэффициент теплового расширения, \( t \) – температура (в градусах Цельсия). При какой температуре рельс удлинится на \( 3 \) мм? Ответ выразите в градусах Цельсия.
	\end{condition}
	\answer{?}
\end{ex}