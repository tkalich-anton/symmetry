\begin{ex}
	\begin{condition}
		При адиабатическом процессе для идеального газа выполняется закон \( pV^k=1,25 \cdot 10^8 \) Па\( \cdot \)м\( ^4 \), где \( p \) – давление газа (в Па), \( V \) – объём газа (в м\( ^3 \)), \( k=\dfrac{4}{3}\). Найдите, какой объём \( V \) (в м\( ^3 \)) будет занимать газ при давлении \( p \), равном \( 2 \cdot 10^5 \) Па.
	\end{condition}
	\answer{\( 125\hspace{1mm}\text{м}^3 \)}
\end{ex}