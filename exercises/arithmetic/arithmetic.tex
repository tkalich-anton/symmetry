\chapter{Числа на прямой}
	\section{Дроби на числовой прямой}
		\begin{enumerate}
	\item Медиана $AM$ треугольника $ABC$ перпендикулярна его биссектрисе $BK$. Найдите $AB$, если $BC = 12$.\ranswer{$3$}
	\item Прямая,  проведенная  через  вершину  $A$  треугольника $ABC$ перпендикулярно его медиане $BD$, 
делит эту медиану пополам. Найдите отношение сторон $AB$ и $AC$.\ranswer{$3$}
	\item $x+4=9$
\ranswer{$4$}
	\item Докажите, что в равных треугольниках соответствующие медианы равны.\ranswer{$3$}
\end{enumerate}
	\section{Корни на числовой прямой}
		\begin{enumerate}
\item 
\begin{ex}{6}{$2$}
	\subimport{../bank}{ex_6}
\end{ex}

\item 
\begin{ex}{7}{$4$}
	\subimport{../bank}{ex_7}
\end{ex}

\item 
\begin{ex}{8}{$3$}
	\subimport{../bank}{ex_8}
\end{ex}

\item 
\begin{ex}{9}{$2$}
	\subimport{../bank}{ex_9}
\end{ex}

\item 
\begin{ex}{10}{$4$}
	\subimport{../bank}{ex_10}
\end{ex}
\item 
\begin{ex}{11}{$2$}
	\subimport{../bank}{ex_11}
\end{ex}

\item 
\begin{ex}{12}{$1$}
	\subimport{../bank}{ex_12}
\end{ex}

\item 
\begin{ex}{13}{$3$}
	\subimport{../bank}{ex_13}
\end{ex}

\item 
\begin{ex}{14}{$4$}
	\subimport{../bank}{ex_14}
\end{ex}

\end{enumerate}
	\section{Сравнение чисел на числовой прямой}
		%\begin{enumerate}
	\item Медиана $AM$ треугольника $ABC$ перпендикулярна его биссектрисе $BK$. Найдите $AB$, если $BC = 12$. \ranswer{$?$}
\end{enumerate}
\chapter{Рациональные числа}
%\section{Арифметические корни}
%\begin{enumerate}
%	\item \textit{(Никольский 8кл. с.52 Пример 4, 155, 167)} Вычислить:
%	\begin{enumerate}[label=\asbuk*)]
%		\item $(2\sqrt{8}+3\sqrt{5}-7\sqrt{2})(\sqrt{72}+\sqrt{20}-4\sqrt{2})$
%		\item  $\sqrt{245 \cdot 27 \cdot 60}$
%		\item $\sqrt{6 \cdot 30 \cdot 245}$
%		\item $(2\sqrt{6}+5\sqrt{3}-7\sqrt{2})(\sqrt{6}-2\sqrt{3}+4\sqrt{2})$
%	\end{enumerate}
%\end{enumerate}
%\section{Тригонометрия}
%\begin{enumerate}
%	\item \textit{(Никольский 10кл. 7.46, 7.47, 7.61)} Вычислить:
%	\begin{enumerate}[label=\asbuk*)]
%		\item $3\cos0+2\sin\dfrac{\pi}{2}-4\cos\dfrac{\pi}{2}-7\sin(-\pi)$
%		\item $\cos\dfrac{\pi}{2}-3\sin\left(-\dfrac{3\pi}{4}\right)+4\cos(-2\pi)-2\sin(-3\pi)$
%		\item $\sin\dfrac{\pi}{4}+\cos\left(-\dfrac{3\pi}{4}\right)+4\cos(-2\pi)-2\sin(-3\pi)$
%		\item $3\cos\dfrac{7\pi}{4}+2\sin\dfrac{3\pi}{4}-\sin\left(-\dfrac{9\pi}{4}\right)+7\cos\dfrac{13\pi}{2}$
%		\item $3\sin\left(-\dfrac{3\pi}{2}\right)-4\cos\left(-\dfrac{11\pi}{2}\right)+5\sin7\pi+\cos(-11\pi)$
%		\item $3\cos\dfrac{\pi}{3}-2\sin\dfrac{2\pi}{3}+7\cos\left(-\dfrac{2\pi}{3}\right)-\sin\left(-\dfrac{5\pi}{4}\right)$
%		\item $2\sin\left(-\dfrac{5\pi}{6}\right)+11\cos\left(-\dfrac{7\pi}{3}\right)+\sin\dfrac{7\pi}{6}-8\cos\dfrac{2\pi}{3}$
%		\item $-6\cos\left(-\dfrac{\pi}{6}\right)-2\sin\left(-\dfrac{\pi}{2}\right)-5\sin\left(-\dfrac{5\pi}{6}\right)+\cos\dfrac{7\pi}{6}$
%	\end{enumerate}
%\end{enumerate}

\chapter{Иррациональные числа}
	\section{Логарифмы}
		\exercise[mode=notext, type=calculate]{559}
		\exercise[mode=notext, type=calculate]{560}
		\exercise[mode=notext, type=calculate]{561}
		\mexercise{_25}
		\mexercise{_26}
		\mexercise{_27}
		\exercise{597}
		\exercise{598}
		\exercise{599}