\documentclass[12pt, a4paper]{article}
\usepackage{../../style}
\begin{document}
	
\lhead{Подготовка к ЕГЭ 2022-2023 г.}
\center
\title{Модуль 1}
\textsc{Алгебраические преобразования. Простейшие уравнения.}\\[0.5em]
\textit{№4; №1; №7; №12 (введение)}
\begin{enumerate}[label=\textbf{\arabic*})]
	\item Преобразование рациональных и иррациональных выражений. Свойства степени. Логарифмические выражения.
	\item Введение в тригонометрию. Преобразования и вычисления значений тригонометрических выражений.
	\item Простейшие тригонометрические уравнения.
	\item Решение задач с прикладным содержанием.
\end{enumerate}
\title{Модуль 2}
\textsc{Текстовые задачи. Теория вероятностей. Более сложные уравнения и неравенства.}\\[0.5em]
\textit{№7; №8; №2; №10}
\begin{enumerate}[label=\textbf{\arabic*})]
	\item Задачи на движение (по местности, по воде, по кругу)
	\item Задачи на работу. Задачи на проценты, смеси и сплавы.
	\item Классическая теория вероятностей. Теоремы теории вероятностей. Задачи на сложную теорию вероятностей.
	\item Логарифмические и показательные неравенства.
\end{enumerate}
\title{Модуль 3}
\textsc{Графики функций. Производная.}\\[0.5em]
\textit{№9; №6; №11}
\begin{enumerate}[label=\textbf{\arabic*})]
	\item Построение элементарных графиков функций. Преобразования графиков функций.
	\item Практикум по решению Задания №9.
	\item Понятие производной функции. Задачи на графики функций (Задание №6).
	\item Правила дифференцирования функций. Исследование функции на возрастание/убывание (Задание №11).
\end{enumerate}
\title{Модуль 4}
\textsc{Геометрия. Введение.}\\[0.5em]
\textit{№3; №5; №16 (начало)}
\begin{enumerate}[label=\textbf{\arabic*})]
	\item Геометрия треугольника. Касательные к окружности. Параллельность.
	\item Геометрия четырехугольника. Вписанные и описанные окружности.
	\item Подобие треугольников. Площади. Отношение сторон и площадей треугольников.
	\item Комбинации окружностей.
\end{enumerate}
\newpage
\title{Модуль 5}
\textsc{Более сложные уравнения. Смешанные уравнения.}\\[0.5em]
\textit{№12}
\begin{enumerate}[label=\textbf{\arabic*})]
	\item Решение сложных тригонометрических уравнений.
	\item Отбор корней. Оформление задачи.
	\item Иррациональные уравнения. Равносильные переходы.
	\item Введение новой переменной. Решение смешанных уравнений.
\end{enumerate}
\title{Модуль 6}
\textsc{Неравенства. Равносильные переходы.}\\[0.5em]
\textit{№14}
\begin{enumerate}[label=\textbf{\arabic*})]
	\item Рациональные неравенства. Метод интервалов.
	\item Иррациональные неравенства. Неравенства с модулем.
	\item Логарифмические и показательные неравенства.
	\item Смешанные неравенства. Метод рационализации. Решение задач из ЕГЭ прошлых лет.
\end{enumerate}
\title{Модуль 7}
\textsc{Введение в задачи с параметрами. Аналитический метод решения.}\\[0.5em]
\textit{№17}
\begin{enumerate}[label=\textbf{\arabic*})]
	\item Решение линейных и квадратных уравнений с параметром аналитическим способом.
	\item Решение линейных и квадратных неравенств с параметром.
	\item Уравнения высших степеней. Введение новой переменной. Перебор случаев.
	\item Применение монотонности функций и инвариантности в решении задач с параметром.
\end{enumerate}
\title{Модуль 8}
\textsc{Продвинутая планиметрия. Часть 1.}\\[0.5em]
\textit{№16}
\begin{enumerate}[label=\textbf{\arabic*})]
	\item Нахождение медиан, биссектрис, высот в треугольнике.
	\item Важные теоремы на отношение отрезков.
	\item Касательная к окружности. Касающиеся и пересекающиеся окружности.
	\item Решение задач из ЕГЭ прошлых лет.
\end{enumerate}
\newpage
\title{Модуль 9}
\textsc{Финансовая математика.}\\[0.5em]
\textit{№15}
\begin{enumerate}[label=\textbf{\arabic*})]
	\item Задачи на вклады. Формула сложного процента.
	\item Теория по задачам на кредиты. Понятие дифференцированных и аннуитетных платежей.
	\item Решение более сложных задач на кредиты. Оформление задачи.
	\item Задачи на оптимальный выбор.
\end{enumerate}
\title{Модуль 10}
\textsc{Стереометрия.}\\[0.5em]
\textit{№13}
\begin{enumerate}[label=\textbf{\arabic*})]
	\item Сечение многогранников. Расстояние между прямыми и плоскостями. Расстояние от точки до прямой и от точки до плоскости.
	\item Вычисление углов между плоскостями; скрещивающимися прямыми; прямой и плоскостью.
	\item Обем многогранника.
	\item Растояние между скрещивающимися прямыми. Решение задач из ЕГЭ прошлых лет.
\end{enumerate}
\title{Модуль 11}
\textsc{Параметры. Графический метод решения.}\\[0.5em]
\textit{№17}
\begin{enumerate}[label=\textbf{\arabic*})]
	\item Построение множеств точек на плоскости. Задание на плоскости сложных фигур.
	\item Решение параметрических задач графическим методом.
	\item Координатно-параметрический метод решения параметрических задач.
	\item Координатно-параметрический метод. Продолжение.
\end{enumerate}
\title{Модуль 12}
\textsc{Параметры. Продвинутый уровень.}\\[0.5em]
\textit{№17}
\begin{enumerate}[label=\textbf{\arabic*})]
	\item Обобщение всех пройденных методов. Решение задач с помощью свойств функций.
	\item Сложная аналитика.
	\item Решение сложных параметрических задач.
	\item Решение задач из ЕГЭ прошлых лет.
\end{enumerate}
\newpage
\title{Модуль 13}
\textsc{Теория чисел.}\\[0.5em]
\textit{№18}
\begin{enumerate}[label=\textbf{\arabic*})]
	\item Делимость и ее свойства. Признаки делимости.
	\item Остатки.
	\item Последовательности и прогрессии.
	\item Неравенства и оценки в задачах теории чисел.
\end{enumerate}
\end{document}