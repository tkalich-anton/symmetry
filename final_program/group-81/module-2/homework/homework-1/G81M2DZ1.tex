\documentclass[12pt, a4paper]{article}
\usepackage{cmap} % Улучшенный поиск русских слов в полученном pdf-файле
\usepackage[T2A]{fontenc} % Поддержка русских букв
\usepackage[utf8]{inputenc} % Кодировка utf8
\usepackage[english, russian]{babel} % Языки: русский, английский
\usepackage{enumitem}
\usepackage{pscyr} % Нормальные шрифты
\usepackage{soulutf8}
\usepackage{amsmath}
\usepackage{amsthm}
\usepackage{amssymb}
\usepackage{scrextend}
\usepackage{titling}
\usepackage{indentfirst}
\usepackage{cancel}
\usepackage{soulutf8}
\usepackage{wrapfig}
\usepackage{gensymb}
\usepackage[dvipsnames,table,xcdraw]{xcolor}
\usepackage{tikz}

%Русские символы в списке
\makeatletter
\AddEnumerateCounter{\asbuk}{\russian@alph}{щ}
\makeatother

%Дублирование знаков при переносе
\newcommand*{\hm}[1]{#1\nobreak\discretionary{}%
	{\hbox{$\mathsurround=0pt #1$}}{}}

\usepackage{graphicx}
\graphicspath{{pic/}}
\DeclareGraphicsExtensions{.pdf,.png,.jpg}

%Изменеие параметров листа
\usepackage[left=15mm,right=15mm,
top=2cm,bottom=2cm,bindingoffset=0cm]{geometry}


\usepackage{fancyhdr}
\pagestyle{fancy}
\usepackage{multicol}

\setlength\parindent{1,5em}
\usepackage{indentfirst}

\begin{document}
		
\lhead{Группа 81}
\chead{Модуль 2 ДЗ№1}
\rhead{Школа <<Симметрия>>}

\section*{Уравнения. Углы.}
\begin{enumerate}
	\item \textit{(2 балла)} Один из двух смежных углов на $30\degree$ меньше другого. Найдите эти углы.
	\item \textit{(2 балла)} Углы $\alpha$ (альфа), $\beta$ (бетта), $\gamma$ (гамма) — смежные. Известно, что угол $\alpha$ в два раза больше угла $\beta$, а угол $\beta$ в три раза больше угла $\gamma$. Чему равны эти углы?
	\item \textit{(2 балла)} Прямой угол разделен двумя лучами на три угла. Один из них на 10\degree больше другого и на 10\degree меньше третьего. Найдите эти углы.
	\item \textit{(2 балла)} Из точки $O$ на плоскости выходят три луча $OA$, $OB$, $OC$. Известно, что $\angle AOB=91\degree$, $\angle BOC=90\degree$. Найдите $\angle AOC$
	\item \textit{(2 балла)} Решите уравнение:
	\begin{enumerate}[label=\asbuk*)]
		\item $(x+1)(x+2)-(x-3)(x+4)=6$
		\item $-x - 2 + 3(x-3) = 3(4 - x) - 3$
		\item $3(x+ 2) + x = 6 + 4x$
		\item $\dfrac{1}{4} - \dfrac{1}{3}x = 4\dfrac{1}{4} - 3x$
	\end{enumerate}
\end{enumerate}
\end{document}