\documentclass[12pt, a4paper]{article}
\usepackage{cmap} % Улучшенный поиск русских слов в полученном pdf-файле
\usepackage[T2A]{fontenc} % Поддержка русских букв
\usepackage[utf8]{inputenc} % Кодировка utf8
\usepackage[english, russian]{babel} % Языки: русский, английский
\usepackage{enumitem}
\usepackage{pscyr} % Нормальные шрифты
\usepackage{soulutf8}
\usepackage{amsmath}
\usepackage{amsthm}
\usepackage{amssymb}
\usepackage{scrextend}
\usepackage{titling}
\usepackage{indentfirst}
\usepackage{cancel}
\usepackage{soulutf8}
\usepackage{wrapfig}
\usepackage{gensymb}
\usepackage[dvipsnames,table,xcdraw]{xcolor}
\usepackage{tikz}

%Русские символы в списке
\makeatletter
\AddEnumerateCounter{\asbuk}{\russian@alph}{щ}
\makeatother

%Дублирование знаков при переносе
\newcommand*{\hm}[1]{#1\nobreak\discretionary{}%
	{\hbox{$\mathsurround=0pt #1$}}{}}

\usepackage{graphicx}
\graphicspath{{pic/}}
\DeclareGraphicsExtensions{.pdf,.png,.jpg}

%Изменеие параметров листа
\usepackage[left=15mm,right=15mm,
top=2cm,bottom=2cm,bindingoffset=0cm]{geometry}


\usepackage{fancyhdr}
\pagestyle{fancy}
\usepackage{multicol}

\setlength\parindent{1,5em}
\usepackage{indentfirst}

\begin{document}
		
\lhead{Группа 81}
\chead{Модуль 2 Урок 2}
\rhead{Школа <<Симметрия>>}
	
\subsection*{Практика}	
\begin{enumerate}
	\item На прямой выбраны три точки $A$, $B$ и $C$, причем $AB = 1$, $BC = 3$. Чему может быть равно $AC$? Укажите все возможные варианты.
	\item Углы $a$, $b$ и $c$ — смежные. Известно, что угол $a$ в два раза больше угла $b$, а угол $b$ на $20\degree$ больше угла $c$. Найдите градусные величины углов $a$, $b$ и $c$.
	\item Прямой угол $ADB$ разделен лучом $DC$ на два угла, причем один угол на $90$ больше другого. Найдите градусные меры этих углов.
	\item Луч BM делит развернутый угол ABC в отношении $5:1$, считая от луча $BA$. Найдите угол $ABK$, если $BK$ – биссектриса угла $MBC$.
	\item Разность двух смежных углов равна $54\degree$. Найдите эти углы.
	\item Один из внешних углов равнобедренного треугольника равен $126\degree$. Найдите углы треугольника.
	\item В треугольнике $ABC$ угол $A$ равен $70\degree$, внешний угол при вершине $B$ равен  $79\degree$. Найдите угол $C$. Ответ дайте в градусах.
	\item В треугольнике $ABC$ угол $A$ равен $39\degree$, $АС=ВС$. Найдите угол $C$. Ответ дайте в градусах.
	\item Сумма двух углов треугольника и внешнего угла к третьему равна $120\degree$. Найдите этот третий угол. Ответ дайте в градусах.
	\item Один острый угол прямоугольного треугольника на $42\degree$ больше другого. Найдите больший острый угол. Ответ дайте в градусах.
	\item Угол при основании равнобедренного треугольника равен $70\degree$, чему равен внешний угол при при основании треугольника, не смежный с данным углом?
	\item Через вершину B треугольника ABC проведена прямая, параллельная прямой $AC$. Образовавшиеся при этом три угла с вершиной B относятся как 3 : 10 : 5. Найдите углы треугольника ABC.
\end{enumerate}

\end{document}