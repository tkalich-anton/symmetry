\documentclass[12pt, a4paper]{article}
\usepackage{cmap} % Улучшенный поиск русских слов в полученном pdf-файле
\usepackage[T2A]{fontenc} % Поддержка русских букв
\usepackage[utf8]{inputenc} % Кодировка utf8
\usepackage[english, russian]{babel} % Языки: русский, английский
\usepackage{enumitem}
\usepackage{pscyr} % Нормальные шрифты
\usepackage{amsmath}
\usepackage{amsthm}
\usepackage{amssymb}
\usepackage{scrextend}
\usepackage{titling}
\usepackage{indentfirst}
\usepackage{cancel}
\usepackage{soulutf8}
\usepackage{wrapfig}
\usepackage{gensymb}
\usepackage[dvipsnames,table,xcdraw]{xcolor}
\usepackage{tikz}

%Русские символы в списке
\makeatletter
\AddEnumerateCounter{\asbuk}{\russian@alph}{щ}
\makeatother

%Дублирование знаков при переносе
\newcommand*{\hm}[1]{#1\nobreak\discretionary{}%
	{\hbox{$\mathsurround=0pt #1$}}{}}

\usepackage{graphicx}
\graphicspath{{pic/}}
\DeclareGraphicsExtensions{.pdf,.png,.jpg}

%Изменеие параметров листа
\usepackage[left=15mm,right=15mm,
top=2cm,bottom=2cm,bindingoffset=0cm]{geometry}

\usepackage{fancyhdr}
\pagestyle{fancy}
\usepackage{multicol}

\setlength\parindent{1,5em}
\usepackage{indentfirst}
\begin{document}
	
	\lhead{Группа 92}
	\chead{Модуль 5 Домашняя работа №2}
	\rhead{Школа <<Симметрия>>}
	\section*{Домашняя работа №2}
	\begin{enumerate}
		\item \textit{(1 балл)} Вычислите:\\\\
		$\dfrac{\left( 5\dfrac{4}{45}-4\dfrac{1}{6}\right) :5\dfrac{8}{15}}{\left( 4\dfrac{2}{3}+0,75\right)\cdot3\dfrac{9}{13} }\cdot34\dfrac{2}{7}+\dfrac{0,3:0,01}{70}+\dfrac{2}{7}$
		\item \textit{(2 балла)} Решите уравнения:
		\begin{multicols}{2}
			\begin{enumerate}[label=\asbuk*)]
				\item $7-0,2x-(21,28-1,6)=0$
				\item $5-3(x+5)=7-(2+3x)$
				\item $1\dfrac{1}{5}-0,5x-0,4+\dfrac{2}{5}x=0$
				\item $x^2-6x+8=0$
				\item $x^2-5\dfrac{1}{5}x+1=0$
				\item $x^4+5x^2+6=0$
			\end{enumerate}
		\end{multicols}
		\item \textit{(2 балла)} Решите системы неравенств:
		\begin{multicols}{2}
			\begin{enumerate}[label=\asbuk*)]
				\item $
				\left\{
				\begin{aligned}
					5x-23<0,\\
					12x-13>0
				\end{aligned}
				\right.
				$
				\item $
				\left\{
				\begin{aligned}
					x^2-3x+2<0,\\
					2x^2-3x-5 \geqslant 0
				\end{aligned}
				\right.
				$
			\end{enumerate}
		\end{multicols}
		\item \textit{(1 балл)} Вычислите:
			\begin{multicols}{2}
			\begin{enumerate}[label=\asbuk*)]
				\item $(2\sqrt{8}+3\sqrt{5}-7\sqrt{2})(\sqrt{72}+\sqrt{20}-4\sqrt{2})$
				\item  $\sqrt{245 \cdot 27 \cdot 60}$
			\end{enumerate}
		\end{multicols}
		\item \textit{(1 балл)} Саша с папой решили покататься на колесе обозрения. Всего на колесе десять кабинок, из них $5$ – синие, $2$ – зелёные, остальные – красные. Кабинки по очереди подходят к платформе для посадки. Найдите вероятность того, что Саша прокатится в красной кабинке.
		\item \textit{(1 балл)} Определите вероятность того, что при бросании игрального кубика (правильной кости) выпадет нечетное число очков.
		\item \textit{(1 балл)} Грузовик перевозит партию щебня массой $90$ тонн, ежедневно увеличивая норму перевозки на одно и то же число тонн. Известно, что за первый день было перевезено $2$ тонны щебня. Определите, сколько тонн щебня было перевезено за десятый день, если вся работа была выполнена за $12$ дней.
		\item \textit{(1 балл)} Васе надо решить $245$ задач. Ежедневно он решает на одно и то же количество задач больше по сравнению с предыдущим днём. Известно, что за первый день Вася решил $11$ задач. Определите, сколько задач решил Вася в последний день, если со всеми задачами он справился за $7$ дней.
	\end{enumerate}
\end{document}