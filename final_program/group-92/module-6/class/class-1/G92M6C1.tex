\documentclass[12pt, a4paper]{article}
\usepackage{cmap} % Улучшенный поиск русских слов в полученном pdf-файле
\usepackage[T2A]{fontenc} % Поддержка русских букв
\usepackage[utf8]{inputenc} % Кодировка utf8
\usepackage[english, russian]{babel} % Языки: русский, английский
\usepackage{enumitem}
\usepackage{pscyr} % Нормальные шрифты
\usepackage{amsmath}
\usepackage{amsthm}
\usepackage{amssymb}
\usepackage{scrextend}
\usepackage{titling}
\usepackage{indentfirst}
\usepackage{cancel}
\usepackage{soulutf8}
\usepackage{wrapfig}
\usepackage{gensymb}
\usepackage[dvipsnames,table,xcdraw]{xcolor}
\usepackage{tikz}

%Русские символы в списке
\makeatletter
\AddEnumerateCounter{\asbuk}{\russian@alph}{щ}
\makeatother

%Дублирование знаков при переносе
\newcommand*{\hm}[1]{#1\nobreak\discretionary{}%
	{\hbox{$\mathsurround=0pt #1$}}{}}

\usepackage{graphicx}
\graphicspath{{pic/}}
\DeclareGraphicsExtensions{.pdf,.png,.jpg}

%Изменеие параметров листа
\usepackage[left=15mm,right=15mm,
top=2cm,bottom=2cm,bindingoffset=0cm]{geometry}

\usepackage{fancyhdr}
\pagestyle{fancy}
\usepackage{multicol}
\setlength\parindent{1,5em}
\usepackage{indentfirst}
\begin{document}
	
	\lhead{Группа 92}
	\chead{Модуль 6 Урок 1}
	\rhead{Школа <<Симметрия>>}
	\begin{enumerate}
		\item Найдите $20\%$ от $84$ килограммов. Ответ дайте в килограммах.
		\item В магазин привезли $2500$кг лука. Продали $30\%$ всего лука. Сколько килограммов лука осталось продать?
		\item У Алёши $80$ марок, у Бори на $20\%$ больше, чем у Алёши. У Вовы на $25\%$ меньше, чем у Алёши. Сколько марок у Бори и Вовы в отдельности?
		\item Мясо при варке теряет $40\%$ своей массы. Сколько свежего мяса нужно взять, чтобы получить $6$ кг свежего?
		\item Одна таблетка лекарства весит $20$мг и содержит $9\%$ активного вещества. Ребёнку в возрасте $6$ месяцев врач прописывает $1,35$мг активного вещества на каждый килограмм веса в сутки. Сколько таблеток этого лекарства следует дать ребёнку в возрасте четырёх месяцев и весом 8 кг в течение суток?
		\item Налог на доходы составляет $13\%$ от заработной платы. После удержания налога на доходы Иван Иванович получил $26 100$ рублей. Сколько рублей составляет заработная плата Ивана Ивановича?
		\item В июле товар стоил $500$ рублей. В ноябре цену на товар снизили на $7\%$, а в декабре подняли на $8\%$. Сколько рублей стоил товар после повышения цены в декабре?
		\item Десять рубашек дороже куртки на $10\%$. На сколько процентов одиннадцать рубашек дороже куртки?
		\item Во время распродажи Паша купил четыре одинаковые по цене футболки со скидкой $40\%$. Сколько таких футболок он мог бы купить на ту же сумму, если бы скидка составила $60\%$?
		\item На фабрике керамической посуды $5\%$ произведённых тарелок имеют дефект. При контроле качества продукции выявляется $80\%$ дефектных тарелок. Остальные тарелки поступают в продажу. Найдите вероятность, что случайно выбранная тарелка при покупке окажется с дефектом. Результат округлите до сотых.
	\end{enumerate}
\end{document}