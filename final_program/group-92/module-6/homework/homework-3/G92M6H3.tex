\documentclass[12pt, a4paper]{article}
\usepackage{cmap} % Улучшенный поиск русских слов в полученном pdf-файле
\usepackage[T2A]{fontenc} % Поддержка русских букв
\usepackage[utf8]{inputenc} % Кодировка utf8
\usepackage[english, russian]{babel} % Языки: русский, английский
\usepackage{enumitem}
\usepackage{pscyr} % Нормальные шрифты
\usepackage{soulutf8}
\usepackage{amsmath}
\usepackage{amsthm}
\usepackage{amssymb}
\usepackage{scrextend}
\usepackage{titling}
\usepackage{indentfirst}
\usepackage{cancel}
\usepackage{soulutf8}
\usepackage{wrapfig}
\usepackage{gensymb}
\usepackage[dvipsnames,table,xcdraw]{xcolor}
\usepackage{tikz}

%Русские символы в списке
\makeatletter
\AddEnumerateCounter{\asbuk}{\russian@alph}{щ}
\makeatother

%Дублирование знаков при переносе
\newcommand*{\hm}[1]{#1\nobreak\discretionary{}%
	{\hbox{$\mathsurround=0pt #1$}}{}}

\usepackage{graphicx}
\graphicspath{{pic/}}
\DeclareGraphicsExtensions{.pdf,.png,.jpg}

%Изменеие параметров листа
\usepackage[left=15mm,right=15mm,
top=2cm,bottom=2cm,bindingoffset=0cm]{geometry}


\usepackage{fancyhdr}
\pagestyle{fancy}
\usepackage{multicol}

\setlength\parindent{1,5em}
\usepackage{indentfirst}

\begin{document}
	
	\lhead{Группа 92}
	\chead{Модуль 6 Домашняя работа 3}
	\rhead{Школа <<Симметрия>>}
	
	\begin{enumerate}
		\item \textit{(2 балла)} Решите уравнения:
		\begin{multicols}{2}
			\begin{enumerate}[label=\asbuk*)]
				\item $\dfrac{7}{x^2+x+12}-\dfrac{6}{x^2+2x-8}=0$
				\item $\dfrac{2}{x}+\dfrac{10}{x^2-2x}=\dfrac{1+2x}{x-2}$
				\item $x^4+2x^2-8=0$
				\item $x^4+9x^2=400$
			\end{enumerate}
		\end{multicols}
		\item \textit{(2 балла)} Решите неравенства:
		\begin{multicols}{2}
			\begin{enumerate}[label=\asbuk*)]
				\item $\left(1\dfrac{1}{3}x+\dfrac{1}{12} \right)(0,7x+4)>0 $
				\item $x^2-4,8x-1<0$
				\item $(x+3)(x-2)>3+10-(x+2)^2$
				\item $\dfrac{3x-x^2}{x-2}\geqslant0$
				\item $\dfrac{25x^2-1}{5x^2-26x+5}<0$
				\item $\dfrac{(x-2)^2(x+4)}{x-4}<0$
			\end{enumerate}
		\end{multicols}
		\item \textit{(2 балла)} 
		\item \textit{(2 балла)} 
		\item \textit{(2 балла)} 
	\end{enumerate}
\end{document}