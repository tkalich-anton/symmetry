\documentclass[12pt, a4paper]{article}
\usepackage{cmap} % Улучшенный поиск русских слов в полученном pdf-файле
\usepackage[T2A]{fontenc} % Поддержка русских букв
\usepackage[utf8]{inputenc} % Кодировка utf8
\usepackage[english, russian]{babel} % Языки: русский, английский
\usepackage{enumitem}
\usepackage{pscyr} % Нормальные шрифты
\usepackage{soulutf8}
\usepackage{amsmath}
\usepackage{amsthm}
\usepackage{amssymb}
\usepackage{scrextend}
\usepackage{titling}
\usepackage{indentfirst}
\usepackage{cancel}
\usepackage{soulutf8}
\usepackage{wrapfig}
\usepackage{gensymb}
\usepackage[dvipsnames,table,xcdraw]{xcolor}
\usepackage{tikz}

%Русские символы в списке
\makeatletter
\AddEnumerateCounter{\asbuk}{\russian@alph}{щ}
\makeatother

%Дублирование знаков при переносе
\newcommand*{\hm}[1]{#1\nobreak\discretionary{}%
	{\hbox{$\mathsurround=0pt #1$}}{}}

\usepackage{graphicx}
\graphicspath{{pic/}}
\DeclareGraphicsExtensions{.pdf,.png,.jpg}

%Изменеие параметров листа
\usepackage[left=15mm,right=15mm,
top=2cm,bottom=2cm,bindingoffset=0cm]{geometry}


\usepackage{fancyhdr}
\pagestyle{fancy}
\usepackage{multicol}

\setlength\parindent{1,5em}
\usepackage{indentfirst}

\begin{document}
	
	\lhead{Группа 92}
	\chead{Модуль 6 Домашняя работа 1}
	\rhead{Школа <<Симметрия>>}
	
	\begin{enumerate}
		\item \textit{(1 балл)} Токарь до обеденного перерыва обточил $24$ детали, что составляет $60\%$ сменной нормы. Сколько деталей должен обточить токарь за смену?
		\item \textit{(1 балл)} На птицеферме разводят куриц, уток и гусей. Известно, что уток в $1,5$ раза больше, чем гусей, и на $40\%$ меньше, чем куриц. Найдите вероятность того, что случайно увиденная на этой птицефарме птица окажется гусём. Выразите в процентах.
		\item \textit{(2 балла)} Цена холодильника в магазине ежегодно уменьшается на одно и то же число процентов от предыдущей цены. Определите, на сколько процентов каждый год уменьшалась цена холодильника, если, выставленный на продажу за $19 800$ рублей, через два года был продан за $16 038$ рублей.
		\item \textit{(2 балла)} Десять одинаковых рубашек дешевле куртки на $4\%$. На сколько процентов пятнадцать таких же рубашек дороже куртки?
		\item \textit{(2 балла)} Семья состоит из мужа, жены и их дочери студентки. Если бы зарплата мужа увеличилась втрое, общий доход семьи вырос бы на $112\%$. Если бы стипендия дочери уменьшилась вдвое, общий доход семьи сократился бы на $3\%$. Сколько процентов от общего дохода семьи составляет зарплата жены?
		\item \textit{(2 балла)} В $2008$ году в городском квартале проживало $40000$ человек. В $2009$ году, в результате строительства новых домов, число жителей выросло на $1\%$, а в $2010$ году – на $9\%$ по сравнению с $2009$ годом. Сколько человек стало проживать в квартале в $2010$ году?
		
	\end{enumerate}
\end{document}