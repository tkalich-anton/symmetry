\documentclass[12pt, a4paper]{article}
\usepackage{cmap} % Улучшенный поиск русских слов в полученном pdf-файле
\usepackage[T2A]{fontenc} % Поддержка русских букв
\usepackage[utf8]{inputenc} % Кодировка utf8
\usepackage[english, russian]{babel} % Языки: русский, английский
\usepackage{enumitem}
\usepackage{pscyr} % Нормальные шрифты


\usepackage{amsmath,amsthm,amssymb,scrextend, cancel}
\usepackage[dvipsnames,table,xcdraw]{xcolor}
\usepackage{fancyhdr}
\usepackage{multicol}
\usepackage{indentfirst}
\usepackage{graphicx}

%Изменеие параметров листа
\usepackage[left=10mm,right=10mm,
top=2cm,bottom=2cm,bindingoffset=0cm]{geometry}

%Русские символы в списке
\makeatletter
\AddEnumerateCounter{\asbuk}{\russian@alph}{щ}
\makeatother
%Дублирование знаков при переносе
\newcommand*{\hm}[1]{#1\nobreak\discretionary{}%
	{\hbox{$\mathsurround=0pt #1$}}{}}

\setlength\parindent{1,5em}
\setlength{\parskip}{0cm}
\pagestyle{fancy}

\begin{document}
		
\lhead{Группа 111}
\chead{Модуль 1 ДЗ №2}
\rhead{<<Симметрия>>}

\section*{Задачи на теорию вероятностей}
\textbf{Задание №1} \textit{(0,5 балла)} Набирая номер телефона, абонент забыл две последние цифры, но
помнит, что одна из них – ноль, а другая – нечетная. Найти вероятность того,
что он наберет правильный номер.

\textbf{Задание №2} \textit{(0,5 балла)} На семиместную скамейку случайным образом рассаживается 7 человек. Какова вероятность того, что два определенных человека окажутся рядом?

\textbf{Задание №3} \textit{(0,5 балла)} Какова вероятность, что взятое наудачу четырехзначное число кратно 5?

\textbf{Задание №4} \textit{(0,5 балла)} В урне имеется 20 белых шаров и 5 черных. Наудачу последовательно, без возвращения извлекают по одному шару до появления белого. Найти
вероятность, что придется производить третье извлечение.

\textbf{Задание №5} \textit{(0,5 балла)} Два радиста пытаются принять сигнал передатчика. Первый из них
сможет это сделать с вероятностью 60\%, а второй – с вероятностью 80\%, независимо друг от друга. Найти вероятность, что хотя бы одному из них удастся
принять сигнал.

\textbf{Задание №6} \textit{(0,5 балла)} В партии лампочек в среднем 4\% брака. Найти вероятность, что среди наугад выбранных двух лампочек окажется хотя бы одна неисправная.

\textbf{Задание №7} \textit{(0,5 балла)} Прибор содержит генератор и осциллограф. За время работы генератор может выйти из строя с вероятностью 30\%, а осциллограф – с вероятностью 20 \%. Отказы осциллографа и генератора не связаны друг с другом. Найти
вероятность, что прибор будет работать исправно.

\textbf{Задание №8} \textit{(0,5 балла)} Точку наудачу бросили на отрезок [0; 2]. Какова вероятность ее попадания в отрезок [0,5; 1,4]?

\textbf{Задание №9} \textit{(0,5 балла)} В случайном эксперименте симметричную монету бросают четырежды. Найдите вероятность того, что решка не выпадет ни разу.

\textbf{Задание №10} \textit{(0,5 балла)} Определите вероятность того, что при бросании игрального кубика (правильной кости) выпадет нечетное число очков.

\textbf{Задание №11} \textit{(0,5 балла)} В ящике 10 красных и 5 синих пуговиц. Вынимаются наудачу две пуговицы. Какова вероятность, что пуговицы будут одноцветными? 

\textbf{Задание №12} \textit{(0,5 балла)} В семье трое детей. Какова вероятность того, что хотя бы двое из них — девочки?

\textbf{Задание №13} \textit{(0,5 балла)} Фирма имеет три источника поставки комплектующих – фирмы А, B, С. На долю фирмы А приходится 50\% общего объема поставок, В – 30\% и С – 20\%. Из практики известно, что среди поставляемых фирмой А деталей 10\% бракованных, фирмой В – 5\% и фирмой С – 6\%. Какова вероятность, что взятая наугад деталь окажется годной?

\textbf{Задание №14} \textit{(0,5 балла)} Три экзаменатора принимают экзамен по некоторому предмету у группы в 30 человек, причем первый опрашивает 6 студентов, второй — 3 студентов, а третий — 21 студента (выбор студентов производится случайным образом из списка). Отношение трех экзаменаторов к слабо подготовившимся различное: шансы таких студентов сдать экзамен у первого преподавателя равны 40\%, у второго — только 10\%, у третьего — 70\%. Найти вероятность того, что слабо подготовившийся студент сдаст экзамен.

\textbf{Задание №15} \textit{(0,5 балла)} Курс акции за день может подняться на 1 пункт с вероятностью 50\%, опуститься на 1 пункт с вероятностью 30\% и остаться неизменным с вероятностью 20\%. Найти вероятность того, что за 5 дней торгов курс поднимется на 2 пункта.

\textbf{Задание №16} \textit{(0,5 балла)} Аудитор обнаруживает финансовые нарушения у проверяемой фирмы с вероятностью 0,9. Найти вероятность того, что среди 4 фирм — нарушителей будет выявлено больше половины.
\newpage

\textbf{Задание №17} \textit{(1 балл за каждый пример)} Вычислите:
\begin{enumerate}[label=\asbuk*)]
	\item $\dfrac{\cos{\dfrac{21\pi}{10}}\sin{\dfrac{3\pi}{20}}+\cos{\dfrac{3\pi}{20}}\sin{\dfrac{\pi}{10}}}{\sin{\dfrac{7\pi}{8}}\sin{\dfrac{7\pi}{24}}+\cos{\dfrac{7\pi}{24}}\cos{\dfrac{\pi}{8}}}$
	\item $\sin^4{\dfrac{3\pi}{2}-2\alpha}$, если $\cos{\pi-4\alpha}=-\dfrac{1}{3}$
	\item $\dfrac{\cos{a}}{2-3\sin{a}}$, если $\tg{\dfrac{a}{2}}=3$
	\item $8\sin^2{\dfrac{15\pi}{16}}\cdot\cos^2{\dfrac{17\pi}{16}}-1$
\end{enumerate}

\end{document}