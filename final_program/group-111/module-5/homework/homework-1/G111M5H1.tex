\documentclass[10pt, a4paper]{article}
\usepackage{cmap}
\usepackage[T2A]{fontenc}
\usepackage[utf8]{inputenc}
\usepackage[english, russian]{babel}
\usepackage[dvipsnames,table,xcdraw]{xcolor}
\usepackage{
	amsmath,
	amssymb,
	scrextend,
	enumitem,
	pscyr,
	multicol,
	cmap,
	titling,
	indentfirst,
	cancel,
	wrapfig,
	gensymb,
	tikz,
	graphicx,
	fancyhdr,
	mathrsfs,
	graphbox,
	indentfirst,
	array
}
%Параметры страницы
\usepackage[left=15mm,right=15mm,
top=2cm,bottom=2cm]{geometry}
\pagestyle{fancy}
%Путь к картинкам
\graphicspath{{pic/}}
\DeclareGraphicsExtensions{.pdf,.png,.jpg}
%Числа в списке второго уровня по умолчанию
\renewcommand{\labelenumii}{\arabic{enumii})}
%Новые команды
\definecolor{silver}{rgb}{0.7, 0.7, 0.7}
\definecolor{dark}{rgb}{0.3, 0.3, 0.3}
\definecolor{harvestgold}{rgb}{0.85, 0.57, 0.0}
\newcommand{\answer}[1]{\textcolor{silver}{\fbox{#1}}}
\newcommand{\source}[1]{\textcolor{silver}{\textit{(#1)}}}
\newcommand{\ranswer}[1]{\textcolor{silver}{\begin{flushright}\vspace{-1em}\fbox{#1}\end{flushright}}}
\newcommand{\leveli}{\textcolor{dark}{$\blacksquare\square\square$}\hspace{0.5em}}
\newcommand{\levelii}{\textcolor{dark}{$\blacksquare\blacksquare\square$}\hspace{0.5em}}
\newcommand{\leveliii}{\textcolor{dark}{$\blacksquare\blacksquare\blacksquare$}\hspace{0.5em}}

%Русские символы в списке
\AddEnumerateCounter{\asbuk}{\russian@alph}{щ}

%Сеттеры
\setlength{\parindent}{5ex}
\setlength{\parskip}{1em}
\begin{document}
	
	\lhead{Группа 111}
	\chead{Модуль 5 Домашняя работа №1}
	\rhead{Школа <<Симметрия>>}
	\section*{Домашняя работа №2}
	\begin{enumerate}
		\item Анатолий решил взять кредит в банке 331000 рублей на 3 месяца под 10\% в месяц.
		Существуют две схемы выплаты кредита.
		
		По первой схеме банк в конце каждого месяца начисляет проценты на оставшуюся сумму долга
		(то есть увеличивает долг на 10\%), затем Анатолий переводит в банк фиксированную сумму и в
		результате выплачивает весь долг тремя равными платежами (аннуитетные платежи).
		
		По второй схеме тоже сумма долга в конце каждого месяца увеличивается на 10\%, а затем
		уменьшается на сумму, уплаченную Анатолием. Суммы, выплачиваемые в конце каждого месяца,
		подбираются так, чтобы в результате сумма долга каждый месяц уменьшалась равномерно, то есть
		на одну и ту же величину (дифференцированные платежи). Какую схему выгоднее выбрать
		Анатолию? Сколько рублей будет составлять эта выгода?
		\item Сергей взял кредит в банке на срок 9 месяцев. В конце каждого месяца общая сумма
		оставшегося долга увеличивается на 12\%, а затем уменьшается на сумму, уплаченную Сергеем.
		Суммы, выплачиваемые в конце каждого месяца, подбираются так, чтобы в результате сумма
		долга каждый месяц уменьшалась равномерно, то есть на одну и ту же величину. Сколько
		процентов от суммы кредита составила сумма, уплаченная Сергеем банку сверх кредита?
		\item 15-го января планируется взять кредит в банке на 19 месяцев. Условия его возврата таковы:
		\begin{itemize}
			\item 1-го числа каждого месяца долг возрастёт на r\% по сравнению с концом предыдущего
			месяца;
			\item со 2-го по 14-е число каждого месяца необходимо выплатить часть долга;
			\item 15-го числа каждого месяца долг должен быть на одну и ту же сумму меньше долга на 15-е
			число предыдущего месяца. Известно, что общая сумма выплат после полного погашения кредита
			на 30\% больше суммы, взятой в кредит. Найдите r.
		\end{itemize}
		\item 15-го января был выдан полугодовой кредит на развитие бизнеса. В таблице представлен
		график его погашения.
		\setlength{\tabcolsep}{10pt}
		\begin{center}
			\begin{tabular}{| m{4em}| m{4em} | {3em}m{3em} |  c | c | c | c | c |}
				\hline
				Дата & 15.01 & 15.02 & 15.03 & 15.04 & 15.05 & 15.06 & 15.07 \\ \hline
				Долг (в процентах от кредита) & 100\% & 90\% & 80\% & 70\% & 60\% & 50\% & 0\% \\
				\hline
			\end{tabular}
		\end{center}
	\end{enumerate}
\end{document}