\documentclass[12pt, a4paper]{article}
\usepackage{cmap} % Улучшенный поиск русских слов в полученном pdf-файле
\usepackage[T2A]{fontenc} % Поддержка русских букв
\usepackage[utf8]{inputenc} % Кодировка utf8
\usepackage[english, russian]{babel} % Языки: русский, английский
\usepackage{enumitem}
\usepackage{pscyr} % Нормальные шрифты
\usepackage{amsmath}
\usepackage{amsthm}
\usepackage{amssymb}
\usepackage{scrextend}
\usepackage{titling}
\usepackage{indentfirst}
\usepackage{cancel}
\usepackage{soulutf8}
\usepackage{wrapfig}
\usepackage{gensymb}
\usepackage[dvipsnames,table,xcdraw]{xcolor}
\usepackage{tikz}

%Русские символы в списке
\makeatletter
\AddEnumerateCounter{\asbuk}{\russian@alph}{щ}
\makeatother

%Дублирование знаков при переносе
\newcommand*{\hm}[1]{#1\nobreak\discretionary{}%
	{\hbox{$\mathsurround=0pt #1$}}{}}

\usepackage{graphicx}
\graphicspath{{pic/}}
\DeclareGraphicsExtensions{.pdf,.png,.jpg}

%Изменеие параметров листа
\usepackage[left=15mm,right=15mm,
top=2cm,bottom=2cm,bindingoffset=0cm]{geometry}

\usepackage{fancyhdr}
\pagestyle{fancy}
\usepackage{multicol}

\setlength\parindent{1,5em}
\usepackage{indentfirst}
\begin{document}
	
	\lhead{Группа 111}
	\chead{Модуль 3 Домашняя работа №2}
	\rhead{Школа <<Симметрия>>}
	\section*{Домашняя работа №2}
	\begin{enumerate}
		\item \textit{(1 балл)} Упростите выражения
		\begin{enumerate}[label=\asbuk*)]
			\item $\dfrac{\cos (\alpha+\beta)+\cos (\alpha-\beta) }{\cos (\alpha-\beta)-\cos(\alpha+\beta)}$
			\item $\cos(45^{\circ}+\alpha)\cos(45^{\circ}-\alpha)-\sin(45^{\circ}-\alpha)\sin(45^{\circ}+\alpha)$
			\item $cos^2(60^{\circ}+\beta)+\cos^2(60^{\circ}-\beta)+\cos^{2}\beta$
		\end{enumerate}
	\item \textit{(1 балл)} Вычислите
		\begin{enumerate}[label=\asbuk*)]
			\begin{multicols}{2}
				\item $\dfrac{\cos2^{\circ}\cos28^{\circ}-\sin28^{\circ}\sin2^{\circ}}{\cos47^{\circ}\cos2^{\circ}+\sin47^{\circ}\sin2^{\circ}}$
				\item $\dfrac{\sin\dfrac{2\pi}{5}\sin\dfrac{3\pi}{5}-\cos\dfrac{2\pi}{5}\cos\dfrac{3\pi}{5}}{\sin\dfrac{\pi}{8}\sin\dfrac{7\pi}{8}-\cos\dfrac{\pi}{8}\cos\dfrac{7\pi}{8}}$
			\end{multicols}
		\end{enumerate}
	\item \textit{(1 балл)} Косинус острого угла равен $0,2$. Найдите косинус смежного угла.
	\item \textit{(1 балл)} Синус острого угла равен $\dfrac{1}{3}$. Найдите синус смежного угла.
	\item \textit{(1 балл)} Найдите $\cos\alpha\cos\beta$, если $\cos(\alpha+\beta)=\dfrac{1}{5}, \cos(\alpha-\beta)=\dfrac{1}{2}$
	\item \textit{(1 балл)} Найдите $\cos(\alpha+\beta)$, если $0^{\circ}<\alpha<90^{\circ}, 180^{\circ}<\beta<270^{\circ}, \cos\alpha=\dfrac{1}{2}, \sin\beta=-\dfrac{1}{2}$.
	\item \textit{(4 балла)} Решить уравнения
	\begin{enumerate}[label=\asbuk*)]
		\begin{multicols}{2}
			\item $2\sqrt{3}\sin^2 {\frac{x}{2}}+2=2\sin^2 x + \sqrt{3}$
			\item $\dfrac{1-3\sin^2 x}{\sin^2 x}=5\ctg x$
			\item $\cos 4x + 4\sin^2 x=1+2\sin^2 2x$
			\item $\sin^4 x + \cos^4 x = \dfrac{3}{4}$
			\item $\sin3x\cos x=\dfrac{3}{2}\tg x$
		\end{multicols}
	\end{enumerate}
	
\end{enumerate}
\end{document}