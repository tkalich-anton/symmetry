\documentclass[12pt, a4paper]{article}
\usepackage{cmap} % Улучшенный поиск русских слов в полученном pdf-файле
\usepackage[T2A]{fontenc} % Поддержка русских букв
\usepackage[utf8]{inputenc} % Кодировка utf8
\usepackage[english, russian]{babel} % Языки: русский, английский
\usepackage{enumitem}
\usepackage{pscyr} % Нормальные шрифты

%Русские символы в списке
\makeatletter
\AddEnumerateCounter{\asbuk}{\russian@alph}{щ}
\makeatother
%Дублирование знаков при переносе
\newcommand*{\hm}[1]{#1\nobreak\discretionary{}%
	{\hbox{$\mathsurround=0pt #1$}}{}}

\usepackage{graphicx}
\graphicspath{{pic/}}
\DeclareGraphicsExtensions{.pdf,.png,.jpg}

%Изменеие параметров листа
\usepackage[left=15mm,right=15mm,
top=2cm,bottom=2cm,bindingoffset=0cm]{geometry}

\usepackage{amsmath,amsthm,amssymb,scrextend}
\usepackage{fancyhdr}
\pagestyle{fancy}
\usepackage{multicol}

\setlength\parindent{1,5em}
\usepackage{indentfirst}

\begin{document}
		
	\lhead{Группа 71}
	\chead{Модуль 1}
	\rhead{Школа <<Симметрия>>}
	
	\section*{Домашняя работа №1}
		\begin{enumerate}
			\item Являются ли тождественно равными выражения?\\(Если да, то объясните почему. Если нет, приведите значение переменной(-ых), при котором(-ых) значения выражений различны.)
			\begin{multicols}{3}
				\begin{enumerate}[label=\asbuk*)]
					\item $(5a)(6b)$ и $31ab$
					\item $15x+5y$ и $5(3x+y)$
					\item $24xy$ и $24xy\cdot 1$
					\item $a\cdot2\cdot b$ и $2ab$
					\item $3x+2$ и $2x+3$
					\item $15y-13y$ и $y+y$
				\end{enumerate}
			\end{multicols}	
			\item Упростить выражение и найти значение выражения:
			\begin{enumerate}[label=\asbuk*)]
				\item $0,9\cdot(p-7)+2p-7$ при $p=0,5$
				\item $\dfrac{2}{3}a-(\dfrac{8}{6}a+4)-\dfrac{1}{3}a+12$ при $a=-3$
				\item $12(x+7)-2(2x+5)-x+13$ при $x=-10$
				\item $3x+(2x+(x+7))-7$ при $x=\dfrac{1}{3}$
				\item $5,5x-1,2y+3(0,4x-0,2y)+0,5(\dfrac{1}{2}x+0,33y)+\dfrac{3}{5}$ при $x=1, y=0,1$
				\item $\dfrac{1}{7}(x+1,4)+3(\dfrac{1}{7}x+0,5)+\dfrac{4}{21}x\cdot3$ при $x=\dfrac{14}{6}$
			\end{enumerate}	
		\end{enumerate}
\end{document}