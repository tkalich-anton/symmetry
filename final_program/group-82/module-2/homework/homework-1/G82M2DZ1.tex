\documentclass[12pt, a4paper]{article}
\usepackage{cmap} % Улучшенный поиск русских слов в полученном pdf-файле
\usepackage[T2A]{fontenc} % Поддержка русских букв
\usepackage[utf8]{inputenc} % Кодировка utf8
\usepackage[english, russian]{babel} % Языки: русский, английский
\usepackage{enumitem}
\usepackage{pscyr} % Нормальные шрифты
\usepackage{soulutf8}
\usepackage{amsmath}
\usepackage{amsthm}
\usepackage{amssymb}
\usepackage{scrextend}
\usepackage{titling}
\usepackage{indentfirst}
\usepackage{cancel}
\usepackage{soulutf8}
\usepackage{wrapfig}
\usepackage{gensymb}
\usepackage[dvipsnames,table,xcdraw]{xcolor}
\usepackage{tikz}

%Русские символы в списке
\makeatletter
\AddEnumerateCounter{\asbuk}{\russian@alph}{щ}
\makeatother

%Дублирование знаков при переносе
\newcommand*{\hm}[1]{#1\nobreak\discretionary{}%
	{\hbox{$\mathsurround=0pt #1$}}{}}

\usepackage{graphicx}
\graphicspath{{pic/}}
\DeclareGraphicsExtensions{.pdf,.png,.jpg}

%Изменеие параметров листа
\usepackage[left=15mm,right=15mm,
top=2cm,bottom=2cm,bindingoffset=0cm]{geometry}


\usepackage{fancyhdr}
\pagestyle{fancy}
\usepackage{multicol}

\setlength\parindent{1,5em}
\usepackage{indentfirst}

\begin{document}
		
\lhead{Группа 82}
\chead{Модуль 2 ДЗ№1-2}
\rhead{Школа <<Симметрия>>}

\section*{Домашняя работа №1-2}
\begin{enumerate}
	\item \textit{(1 балл)} Из точки, данной на окружности, проведены диаметр и хорда, равная радиусу. Найдите угол между ними.
	\item \textit{(1 балл)} Радиус окружности равен 13, хорда равна 10. Найдите её расстояние от центра.
	\item \textit{(2 балла)} Через вершины A и C треугольника ABC проведены прямые, перпендикулярные биссектрисе угла $ABC$, пересекающие прямые $CB$ и $BA$ в точках $K$ и $M$ соответственно. Найдите $AB$, если $BM = 8$, $KC = 1$.
	\item \textit{(2 балла)} $AD$ — биссектриса треугольника $ABC$. Точка $M$ лежит на стороне $AB$, причем $AM = MD$. Докажите, что $MD||AC$.
	\item \textit{(2 балла)} Острый угол прямоугольного треугольника равен $30\degree$. Докажите, что высота и медиана, проведенные из вершины прямого угла, делят его на три равные части.
	\item \textit{(1 балл)} Биссектрисы двух углов треугольника пересекаются под углом $140\degree$. Найдите третий угол треугольника.
	\item \textit{(1 балла)} Упростите выражение и найдите значение выражения:
	\begin{multicols}{2}
		\begin{enumerate}[label=\asbuk*)]
			\item $\dfrac{x^2-10x+25}{3x+12}\cdot\dfrac{x^2-16}{2x-10}$ при $x=-1$
			\item $\left(\dfrac{b}{a}-\dfrac{a}{b}\right)\cdot\dfrac{1}{b+a}$ при $a=1$, $b=\dfrac{1}{3}$
		\end{enumerate}
	\end{multicols}
	\item \textit{(2 балла)} На продолжениях гипотенузы $AB$ прямоугольного треугольника $AB$C за точки $A$ и $B$ соответственно взяты точки $K$ и $M$, причем $AK = AC$ и $BM = BC$. Найдите $\angle MCK$.
	\item \textit{(2 балла)} В треугольнике $ABC$ угол $B$ равен $20\degree$, угол $C$ равен $40\degree$. Биссектриса $AD$ равна $2$. Найдите разность сторон $BC$ и $AB$.
	\item \textit{(2 балла)} Даны две концентрические окружности и пересекающая их прямая. Докажите, что отрезки этой прямой, заключенные между окружностями, равны.
	\item \textit{(2 балла)} Две прямые, пересекающиеся в точке $C$, касаются окружности в точках $A$ и $B$. Известно, что $\angle ACB = 120\degree$. Докажите, что сумма отрезков $AC$ и $BC$ равна отрезку $OC$.
	\item \textit{(2 балла)} К окружности, вписанной в равносторонний треугольник со стороной, равной a, проведена касательная, пересекающая две его стороны. Найдите периметр отсеченного треугольника.
	\end{enumerate}
\end{document}