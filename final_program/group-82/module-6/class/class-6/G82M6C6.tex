\documentclass[12pt, a4paper]{article}
\usepackage{cmap} % Улучшенный поиск русских слов в полученном pdf-файле
\usepackage[T2A]{fontenc} % Поддержка русских букв
\usepackage[utf8]{inputenc} % Кодировка utf8
\usepackage[english, russian]{babel} % Языки: русский, английский
\usepackage{enumitem}
\usepackage{pscyr} % Нормальные шрифты
\usepackage{amsmath}
\usepackage{amsthm}
\usepackage{amssymb}
\usepackage{scrextend}
\usepackage{titling}
\usepackage{indentfirst}
\usepackage{cancel}
\usepackage{soulutf8}
\usepackage{wrapfig}
\usepackage{gensymb}
\usepackage[dvipsnames,table,xcdraw]{xcolor}
\usepackage{tikz}

%Русские символы в списке
\makeatletter
\AddEnumerateCounter{\asbuk}{\russian@alph}{щ}
\makeatother

%Дублирование знаков при переносе
\newcommand*{\hm}[1]{#1\nobreak\discretionary{}%
	{\hbox{$\mathsurround=0pt #1$}}{}}

\usepackage{graphicx}
\graphicspath{{pic/}}
\DeclareGraphicsExtensions{.pdf,.png,.jpg}

%Изменеие параметров листа
\usepackage[left=15mm,right=15mm,
top=2cm,bottom=2cm,bindingoffset=0cm]{geometry}

\usepackage{fancyhdr}
\pagestyle{fancy}
\usepackage{multicol}

\setlength\parindent{1,5em}
\usepackage{indentfirst}
\begin{document}
	
	\lhead{Группа 82}
	\chead{Модуль 6 Урок №5-8}
	\rhead{Школа <<Симметрия>>}
	\begin{enumerate}
		\item \textit{} Площадь прямоугольника равна $24$. Найдите площадь четырехугольника с вершинами в серединах сторон прямоугольника.
		\item \textit{} Средняя линия треугольника разбивает его на треугольник и четырехугольник. Какую часть составляет площадь полученного треугольника от площади исходного?
		\item \textit{} Докажите, что медиана разбивает треугольник на два равновеликих треугольника.
		\item \textit{} Точки, делящие сторону треугольника на $n$ равных частей, соединены отрезками с противоположной вершиной. Докажите, что при этом треугольник также разделился на $n$ равновеликих частей.
		\item \textit{} Пусть $M$ — точка на стороне $AB$ треугольника $ABC$, причем $AM : MB = m : n$. Докажите, что площадь треугольника $CAM$ относится к площади треугольника $CBM$ как $m : n$.
		\item \textit{} Докажите, что площадь выпуклого четырехугольника со взаимно перпендикулярными диагоналями равна половине произведения диагоналей.
		\item \textit{} На сторонах $AB$ и $AC$ треугольника $ABC$, площадь которого равна $50$, взяты соответственно точки $M$ и $K$ так, что $AM : MB = 1 : 5$, а $AK : KC = 3 : 2$. Найдите площадь треугольника $AMK$.
		\item \textit{} Вершины одного квадрата расположены на сторонах другого и делят эти стороны в отношении $1 : 2$, считая по часовой стрелке. Найдите отношение площадей квадратов.
	\end{enumerate}
\end{document}