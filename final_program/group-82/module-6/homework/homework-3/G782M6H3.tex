\documentclass[12pt, a4paper]{article}
\usepackage{cmap} % Улучшенный поиск русских слов в полученном pdf-файле
\usepackage[T2A]{fontenc} % Поддержка русских букв
\usepackage[utf8]{inputenc} % Кодировка utf8
\usepackage[english, russian]{babel} % Языки: русский, английский
\usepackage{enumitem}
\usepackage{pscyr} % Нормальные шрифты
\usepackage{amsmath}
\usepackage{amsthm}
\usepackage{amssymb}
\usepackage{scrextend}
\usepackage{titling}
\usepackage{indentfirst}
\usepackage{cancel}
\usepackage{soulutf8}
\usepackage{wrapfig}
\usepackage{gensymb}
\usepackage[dvipsnames,table,xcdraw]{xcolor}
\usepackage{tikz}

%Русские символы в списке
\makeatletter
\AddEnumerateCounter{\asbuk}{\russian@alph}{щ}
\makeatother

%Дублирование знаков при переносе
\newcommand*{\hm}[1]{#1\nobreak\discretionary{}%
	{\hbox{$\mathsurround=0pt #1$}}{}}

\usepackage{graphicx}
\graphicspath{{pic/}}
\DeclareGraphicsExtensions{.pdf,.png,.jpg}

%Изменеие параметров листа
\usepackage[left=15mm,right=15mm,
top=2cm,bottom=2cm,bindingoffset=0cm]{geometry}

\usepackage{fancyhdr}
\pagestyle{fancy}
\usepackage{multicol}

\setlength\parindent{1,5em}
\usepackage{indentfirst}
\begin{document}
	
	\lhead{Группа 82}
	\chead{Модуль 6 Домашняя работа №3}
	\rhead{Школа <<Симметрия>>}
	\section*{Домашняя работа №3}
	\begin{enumerate}
		\item \textit{(1 балл)} Точка $M$ расположена на стороне $BC$ параллелограмма $ABCD$. Докажите, что площадь треугольника $AMD$ равна половине площади параллелограмма.
		\item \textit{(1 балл)} Докажите, что диагонали разбивают параллелограмм на четыре равновеликих треугольника.
		\item \textit{(2 балла)} Точки $M$ и $N$ — соответственно середины противоположных сторон $AB$ и $CD$ параллелограмма $ABCD$, площадь которого равна $1$. Найдите площадь четырехугольника, образованного пересечениями прямых $AN$, $BN$, $CM$ и $DM$.
		\item \textit{(2 балла)} Площадь трапеции, основания которой относятся как $3 : 2$, равна $35$. Найдите площади треугольников, на которые трапеция разбивается диагональю.
		\item \textit{(2 балла)} Точки $M$ и $N$ расположены на стороне $BC$ треугольника $ABC$, а точка $K$ — на стороне $AC$, причем $BM : MN : NC = 1 : 1 : 2 и CK : AK = 1 : 4$. Известно, что площадь треугольника $ABC$ равна $1$. Найдите площадь четырехугольника $AMNK$.

		\item \textit{(2 балла)} Найдите площадь ромба со стороной, равной $8$, и острым углом $30\degree$.
	\end{enumerate}
\end{document}