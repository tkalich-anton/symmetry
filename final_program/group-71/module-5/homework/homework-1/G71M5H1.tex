\documentclass[12pt, a4paper]{article}
\usepackage{cmap} % Улучшенный поиск русских слов в полученном pdf-файле
\usepackage[T2A]{fontenc} % Поддержка русских букв
\usepackage[utf8]{inputenc} % Кодировка utf8
\usepackage[english, russian]{babel} % Языки: русский, английский
\usepackage{enumitem}
\usepackage{pscyr} % Нормальные шрифты
\usepackage{amsmath}
\usepackage{amsthm}
\usepackage{amssymb}
\usepackage{scrextend}
\usepackage{titling}
\usepackage{indentfirst}
\usepackage{cancel}
\usepackage{soulutf8}
\usepackage{wrapfig}
\usepackage{gensymb}
\usepackage[dvipsnames,table,xcdraw]{xcolor}
\usepackage{tikz}

%Русские символы в списке
\makeatletter
\AddEnumerateCounter{\asbuk}{\russian@alph}{щ}
\makeatother

%Дублирование знаков при переносе
\newcommand*{\hm}[1]{#1\nobreak\discretionary{}%
	{\hbox{$\mathsurround=0pt #1$}}{}}

\usepackage{graphicx}
\graphicspath{{pic/}}
\DeclareGraphicsExtensions{.pdf,.png,.jpg}

%Изменеие параметров листа
\usepackage[left=15mm,right=15mm,
top=2cm,bottom=2cm,bindingoffset=0cm]{geometry}

\usepackage{fancyhdr}
\pagestyle{fancy}
\usepackage{multicol}

\setlength\parindent{1,5em}
\usepackage{indentfirst}

\begin{document}
	
	\lhead{Группа 71}
	\chead{Модуль 5 Домашняя работа №1}
	\rhead{Школа <<Симметрия>>}
	
	\begin{enumerate}
		\item \textit{(3 балла)} Преобразуйте выражение в многочлен:
		\begin{multicols}{2}
			\begin{enumerate}[label=\asbuk*)]
				\item $(x+7)^2$
				\item $(2x-1)^2$
				\item $(3x-10)^2$
				\item $(x^2+y^2)^2$
				\item $(3a^2-5b^3)^2$
				\item $(x^2+2x^3)^2$
			\end{enumerate}
		\end{multicols}
		\item \textit{(3 балла)} Представьте многочлен в виде квадрата суммы или квадрата разности:
		\begin{multicols}{2}
			\begin{enumerate}[label=\asbuk*)]
				\item $a^2+2ab+b^2$
				\item $x-2x+1$
				\item $9+6a+a^2$
				\item $a^6+2a^3b^3+b^6$
				\item $16p^2+40pq+25q^2$
				\item $4m^2+9n^2+12mn$
			\end{enumerate}
		\end{multicols}
		\item \textit{(2 балла)} Представьте выражение в виде многочлена:
		\begin{multicols}{2}
			\begin{enumerate}[label=\asbuk*)]
				\item $(x-a)(x+a)$
				\item $(2a-1)(2a+1)$
				\item $(11a-3b)(3b+11)$
				\item $(0,1-a)(0,1+a)$
			\end{enumerate}
		\end{multicols}
		\item \textit{(2 балла)} Разложите выражение множители с помощью формулы разности квадратов:
		\begin{multicols}{2}
			\begin{enumerate}[label=\asbuk*)]
				\item $x^2-9$
				\item $16-y^2$
				\item $4x^2-100$
				\item $0,25x^2-0,81y^2$
			\end{enumerate}
		\end{multicols}
	\end{enumerate}
\end{document}