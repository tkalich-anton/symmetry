\documentclass[12pt, a4paper]{article}
\usepackage{cmap} % Улучшенный поиск русских слов в полученном pdf-файле
\usepackage[T2A]{fontenc} % Поддержка русских букв
\usepackage[utf8]{inputenc} % Кодировка utf8
\usepackage[english, russian]{babel} % Языки: русский, английский
\usepackage{enumitem}
\usepackage{pscyr} % Нормальные шрифты
\usepackage{amsmath}
\usepackage{amsthm}
\usepackage{amssymb}
\usepackage{scrextend}
\usepackage{titling}
\usepackage{indentfirst}
\usepackage{cancel}
\usepackage{soulutf8}
\usepackage{wrapfig}
\usepackage{gensymb}
\usepackage[dvipsnames,table,xcdraw]{xcolor}
\usepackage{tikz}

%Русские символы в списке
\makeatletter
\AddEnumerateCounter{\asbuk}{\russian@alph}{щ}
\makeatother

%Дублирование знаков при переносе
\newcommand*{\hm}[1]{#1\nobreak\discretionary{}%
	{\hbox{$\mathsurround=0pt #1$}}{}}

\usepackage{graphicx}
\graphicspath{{pic/}}
\DeclareGraphicsExtensions{.pdf,.png,.jpg}

%Изменеие параметров листа
\usepackage[left=15mm,right=15mm,
top=2cm,bottom=2cm,bindingoffset=0cm]{geometry}

\usepackage{fancyhdr}
\pagestyle{fancy}
\usepackage{multicol}

\setlength\parindent{1,5em}
\usepackage{indentfirst}

\begin{document}
		
\lhead{Группа 71}
\chead{Модуль 2 Домашняя работа №3}
\rhead{Школа <<Симметрия>>}

\begin{enumerate}
	\item \textit{(2 балла)} В равнобедренном треугольнике $ABC$ из вершин $B$ и $C$ проведены биссектрисы $BK$ и $CM$ соответственно. Докажите, что $BK=CM$
	\item \textit{(2 балла)} В прямоугольнике $ABCD$ из вершин $B$ и $D$ проведены биссектрисы углов прямоугольника. Докажите, что эти биссектрисы отсекают от прямоугольника два равных треугольника.
	\item \textit{(2 балла)} Через вершины $A$ и $C$ треугольника $ABC$ проведены прямые, перпендикулярные биссектрисе угла $ABC$, пересекающие прямые $CB$ и $BA$ в точках $K$ и $M$ соответственно. Найдите $AB$, если $BM = 8$, $KC = 1$.
	\item \textit{(2 балла)} Прямая, проведенная через вершину $A$ треугольника $ABC$ перпендикулярно его медиане $BD$, делит эту медиану пополам. Найдите отношение сторон $AB$ и $AC$.
	\item \textit{(2 балла)} У Миши в 4 раза больше марок, чем у Андрея. Если Миша отдаст Андрею 8 марок, то у него станет марок вдвое больше. Сколько марок у каждого мальчика?
\end{enumerate}
\end{document}