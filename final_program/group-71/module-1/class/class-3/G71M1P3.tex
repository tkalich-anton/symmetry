\documentclass[12pt, aspectratio=169]{beamer}

\usetheme[progressbar=frametitle]{metropolis}

\usepackage{pgfplots}
\usepgfplotslibrary{dateplot}

\usepackage[russian]{babel}
\usepackage[utf8]{inputenc}
\usepackage[T2A, T1]{fontenc}
\usepackage{pscyr} % Нормальные шрифты

\usefonttheme[onlymath]{serif}

\usepackage{amsmath}
\usepackage{amsfonts}
\usepackage{multicol}

\title{Уравнения с одной переменной.}
\subtitle{Группа 71 \\[8pt] Модуль 1 Урок 3}
\institute{Школа Симметрия}
\date{13.09.2021}

\begin{document}
\maketitle
\begin{frame}{План урока}
	\tableofcontents[hideallsubsections, pausesections]
\end{frame}

\section{Уравнения и корни уравнений}

\begin{frame}{Уравнения и корни уравнений}
	\onslide<1->
	\textbf{Уравнение} — равенство одного или двух алгебраических выражений.\\
	То есть равенство, содержащее переменную.\\[1em]
	\onslide<2->
	\textbf{Корни уравнения} — значения переменных, при которых равенство обращается в верное.\\[1em]
	\onslide<3->
	\textbf{Решить уравнение} — значит найти \underline{все} его корни или доказать, что корней нет.
\end{frame}

\begin{frame}{Равносильные уравнения}
	Два уравнения называют \textbf{равносильными}, если они имеют одинаковые корни или если оба уравнения не имеют корней.
\end{frame}

\section{Свойства уравнений}

\begin{frame}{Свойства уравнений}
	\begin{itemize}
		\item<1-> {
			Если в уравнении перенести слагаемое из одной части в другую, изменеив его знак, то получится уравнение, равносильное данному.
			
			\textit{Пример:}
			\begin{center}
				$3x+8=10$\\
				$3x=10-8$
			\end{center}
		}
		\item<2-> {
			Если обе части уравнения разделить или множить на одно и то же отличное от нуля число, то получится уравнение, равносильное данному.
			
			\textit{Пример:}
			\begin{center}
				$10x+15=5x\quad|:5 $\\
				$2x+3=1x$
			\end{center}
		}
	\end{itemize}
\end{frame}

\section{Линейное уравнение с одной переменной}

\begin{frame}{Линейное уравнение с одной переменной}
	Уравнение вида \mathversion{bold}$ax=b$\mathversion{normal}, где $x$ — переменная, а $a$ и $b$ — некоторые числа, называется линейным уравнением с одной переменной.
\end{frame}

\begin{frame}{Различные случаи}
		1. Если $a\ne0$, тогда можем разделить обе части уравнения на $a$ и поулчим $x=\dfrac{b}{a}$.
		
		\textit{Пример:}
		\begin{eqnarray*}
			3x &=& 5 \quad |:3 \\
			x &=& \dfrac{5}{3}
		\end{eqnarray*}
	
		\textbf{Ответ:} $x=\dfrac{5}{3}$
\end{frame}

\begin{frame}{Различные случаи}
	2. Если $a=0$, а $b\ne0$, то уравнение не будет иметь корней. Очевидно, что какое бы мы значение вместо $x$ не подставили, умножив его на $0$ можно получить только $0$.
	
	\textit{Пример:}
	\begin{eqnarray*}
		0\cdot x &=& 4
	\end{eqnarray*}
	
	\textbf{Ответ:} нет решений.
\end{frame}

\begin{frame}{Различные случаи}
	3. Если $a=0$ и $b=0$, то уравнение будет иметь бесконечное количество решений. Не сложно заметить, что какое бы число мы не подставляли вместо $x$, умножив его на $0$ мы всегда будем поулчать $0$.
	
	\textit{Пример:}
	\begin{eqnarray*}
		0\cdot x &=& 0
	\end{eqnarray*}
	
	\textbf{Ответ:} любое число.
\end{frame}

\section{Практика}

\begin{frame}{Практика}
	\textbf{Задание №1} Решить уравнения:
	\begin{multicols}{3}
		\begin{enumerate}
			\item $5x=20$
			\item $3x=-150$
			\item $-2x=-36$
			\item $1,2=0,5x$
			\item $42x=13$
			\item $\dfrac{1}{5}x=17$
			\item $\dfrac{3}{4}x=15$
			\item $-\dfrac{3}{7}x=27$
			\item $5x=-\dfrac{15}{7}$
			\item $5x=0$
		\end{enumerate}
	\end{multicols}
\end{frame}

\begin{frame}{Практика}
	\textbf{Задание №2} Решить уравнения:
	\begin{multicols}{2}
		\begin{enumerate}
			\item $4x+140=0$
			\item $54-3x=0$
			\item $-1,8x-9=0$
			\item $3,5x+2,8=0$
			\item $-\dfrac{1}{17}x-\dfrac{3}{34}=0$
			\item $-x+3\dfrac{5}{7}=3\dfrac{1}{3}$
			\item $1,7-0,5k=3+4,5k$
			\item $1\dfrac{1}{3}x+5=\dfrac{1}{3}x+3$
			\item $x=x$
			\item $y-\dfrac{3}{5}y$
			\item $3x=6x$
		\end{enumerate}
	\end{multicols}
\end{frame}

\begin{frame}{Практика}
\textbf{Задание №3} Решить уравнения:
\begin{multicols}{2}
	\begin{enumerate}
		\item $(x+3)-(x-2)=12$
		\item $\dfrac{2}{7}x=\dfrac{1}{2}$
		\item $3k-2-(k+3)=4$
		\item $21x=19-(3+13x)$
		\item $(13x-15)-(9+6x)=-3x$
	\end{enumerate}
\end{multicols}
\end{frame}

\begin{frame}{Практика}
	\textbf{Задание №4} Решить уравнение:
	\begin{enumerate}
		\item $1,2x-(x+3,8)=(\dfrac{1}{5}x+1,5)-\dfrac{14}{20}$
		\item $(\dfrac{1}{2}x+1,3)-(3,6-4,5x)=(5,4-0,3x)+(10\dfrac{2}{3}x+\dfrac{3}{8})$
	\end{enumerate}
	
	\textbf{Задание №5} При каком значении переменной значение выражения $13x-51$ равно $1$?\\
	
	\textbf{Задание №6} При каком значении переменной $x$ выражения $2x+8$ и $-2x-14$ равны?\\
	
	\textbf{Задание №7} При каком значении переменной $x$ выражение $-x+14$ больше выражения $3x-8$ на $2$?
\end{frame}

\end{document}