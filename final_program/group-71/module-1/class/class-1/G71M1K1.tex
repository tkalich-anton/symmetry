\documentclass[12pt, a4paper]{article}
\usepackage{cmap} % Улучшенный поиск русских слов в полученном pdf-файле
\usepackage[T2A]{fontenc} % Поддержка русских букв
\usepackage[utf8]{inputenc} % Кодировка utf8
\usepackage[english, russian]{babel} % Языки: русский, английский
\usepackage{enumitem}
\usepackage{pscyr} % Нормальные шрифты

%Русские символы в списке
\makeatletter
\AddEnumerateCounter{\asbuk}{\russian@alph}{щ}
\makeatother
%Дублирование знаков при переносе
\newcommand*{\hm}[1]{#1\nobreak\discretionary{}%
	{\hbox{$\mathsurround=0pt #1$}}{}}

\usepackage{graphicx}
\graphicspath{{pic/}}
\DeclareGraphicsExtensions{.pdf,.png,.jpg}

%Изменеие параметров листа
\usepackage[left=2cm,right=2cm,
top=2cm,bottom=2cm,bindingoffset=0cm]{geometry}

\usepackage{amsmath,amsthm,amssymb,scrextend}
\usepackage{fancyhdr}
\pagestyle{fancy}
\usepackage{multicol}

\setlength\parindent{1,5em}
\usepackage{indentfirst}
\setlength{\parskip}{0cm}

\begin{document}
		
	\lhead{Группа 71}
	\chead{Модуль 1}
	\rhead{Школа <<Симметрия>>}
	
	\section*{Конспект к уроку №1}
	\subsection*{Типы выражений}
	\textbf{Арифметическое (числовое) выражение} — выражение, составленное из чисел, связанных между собой знаками арифметических операций: сложение, вычитание, умножение, деление и возведение в степень.
	
	\textit{Пример:}
	\begin{center}
		$(14+\dfrac37):0,5$
	\end{center}

	\textbf{Алгебраическое (буквенное) выражение } — выражение, составленное из чисел и переменных (букв), связанных между собой знаками арифметических операций.
	
	\textit{Пример:}
	\begin{center}
		$3a\cdot b^2 + 18$
	\end{center}
	
	\subsection*{Одночлены и многочлены}
	
	\textbf{Одночлен} — выражение, в котором последнее по порядку действие умножение, деление или возведение степень. Число или буква (переменая) также является одночленом.
	
	\textit{Примеры:}
	\begin{center}
		$3$ ; $x$ ; $5y$ ; $-4x^2$ ; $(7x)^2$ ; $4(x-5)^2$
	\end{center}

	\textbf{Подобные слагаемые} — слагаемые (одночлены), имеющие одинаковую буквенную частью.

	Чтобы привести подобные слагаемые, нужно сложить их коэффициенты и результат умножить на общую буквенную часть.
	
	\textit{Подобные:}
	\begin{center}
		 $7x$ и $12x$ ; $7a^2b$ и $-3a^2b$.
	\end{center}
	
	\textit{Не подобные:}
	\begin{center}
		$5x$ и $6y$ ; $7a^2b$ и $2ab^2$ ; $-4abc$ и $12ab^2c$.
	\end{center}

	\textbf{Многочлен} — выражение, в котором последнее по порядку действие сложение или вычитание.
	
	\textit{Примеры:}
	\begin{center}
		 $7+x$ ; $3x^2-2x+4$ ; $(5+x)^2-12$
	\end{center}

	Многочлен состоит из алгебраической суммы одночленов, которые называются членами многочлена.
	
	\textit{Пример:}
	\begin{center}
		$3x^3-5x^2+6x-12 = 3x^3+(-5x^2)+6x+(-12)$
	\end{center}

	\newpage
	
	\subsection*{Стандартные виды одночленов и многочленов}
	
	\textbf{Стандартный вид одночлена} — такой его вид, в котором он представляет собой произведение числового множителя и натуральных степеней разных переменных (букв).
	
	\textit{Одночлены, приведенные к стандартному виду:}
	\begin{center}
		 $3x^2$ ; $-15a^3b^2$ ; $7$ ; $a$
	\end{center}

	\textit{Одночлены, \underline{не} приведенные к стандартному виду:}
	\begin{center}
		$5x^3x^2$ ; $aab^2a$ ; $13a5b$
	\end{center}
	
	\textbf{Стандартный вид многочлена} — это многочлен, все члены которого являются одночленами стандартного вида, среди которых нет подобных членов.
	
	Также правилом хорошего тона является расположить члены многочлена по возрастанию (или убыванию) степеней какой-нибудь переменной.
	
	\textit{Многочлены, приведенные к стандартному виду:}
	\begin{center}
		 $3x^2+4x-7$ ; $15x-4x^2y+x^3y^5+5x^4z$
	\end{center}

	\textit{Многочлены, \underline{не} приведенные к стандартному виду:}
	\begin{center}
		$3x+4x+5x^2$ ; $15xxx^2-4x^3+15$
	\end{center}
		
	\subsection*{Виды алгебраических выражений}
	
	\textbf{Целое выражение} — это числа, переменные, а также всевозможные выражения, составленные из них при помощи действий сложения, вычитания, умножения и возведения в степень, которые также могут содержать скобки и деление на отличное от нуля число. Выражения такого типа не содержат деления на переменную.
	
	\textit{Примеры целых выражений:}
	\begin{center}
		$\dfrac{3}{7}$ ; $40x^2y$ ; $\dfrac{7y^2x^3}{5}$
	\end{center}

	\textbf{Дробное выражение} — отношение двух, как правило, целых алгебраических выражений.
	
	\textit{Примеры дробных выражений:}
	\begin{center}
		$\dfrac{45}{x+4}+4$ ; $\dfrac{12x^2-4}{(x+y)^3}$ ; $\dfrac{5}{x}$
	\end{center}
	
	\subsection*{Когда выражение имеет смысл?}
	
	\textbf{Выражение имеет смысл} (при определенных значниях переменных) — значит, что при подстановке определенных значений вместо переменных, значение выражения можно вычислить.
	
	Если при подстановки значений вместо переменных происходит деление на ноль или другие невозможные для подсчета арифметические операции, которые мы пройдем позже, то говорят, что выражение не имеет смысла.
	
	Множество всех допустимых значений переменных, входящих в это выражение, при которых данное выражение имеет смысл, называют \textbf{областью допустимых значений} выражения.
	
	Целые выражения имеют смысл при любых значениях переменных.
	
	\subsection*{Равенство и тождество}
	
	\textbf{Равенство} — это два числа или выражения, соединённых между собой знаком «$=$». Эти числа или выражения называются частями равенства: слева от знака «$=$» — левая часть равенства, справа — правая часть.
	
	Когда равенство состоит только из числовых (арифметических) выражений, значение левой и правой части равенства можно вычислить и если эти значения окажутся равными, то говорят, что равенство верное. Если же полученные значения не равны, то равенство считают неверным.
	
	Однако, если одна или обе части равенства являются алгебраическими выражениями, то верность или неверность равенства определить становится сложно.
	
	Существуют разные ситуации. В одном случае, какие бы значения вместо переменных мы не подставляли, обратить равенство в верное не получится. Во втором случае, при некоторых значениях переменных равенство становится верным, а при других оказывается ложным. А в третьем случае, какие бы значения вместо переменных мы не подставляли, равенство двух выражений всегда оказывается верным.
	
	Рассмотрим подробнее последний случай.\\
	
	\textbf{Тождество} \textit{(от греч. «тот же, такой же»)} — это равенство двух выражений, верное при любых значениях переменных.\\
	
	\textbf{Тождественно равные выражения} — это выражения, значения которых равны при любых значениях переменных, которые в них входят.\\

	\textbf{Тождественное преобразование} — это переход от исходного выражения к тождественно равному.
	
	\subsection*{Какие преобразования являются тождественными?}
	
	\begin{enumerate}
		\item 
			\textbf{Выполнение действий над числами}\\[0.5em]
			\textit{Пример:}
			$$4x^2+5\cdot6-20=4x^2+10$$
		\item
			\textbf{Перестановка местами слагаемых или множителей}\\[0.5em]
			\textit{Примеры:}
			$$2x+7=7+2x;$$
			$$3x\cdot 2y=2y\cdot 3x.$$
		\item
			\textbf{Приведение подобных слагаемых}\\[0.5em]
			\textit{Пример:}
			$$7ab+ab-5ab=3ab$$
		\item
			\textbf{Группировка слагаемых или множителей}\\[0.5em]
			\textit{Пример 1:}
			$$7x+5x^2-8=7x+(5x^2-8)$$
			\textit{Пример 2:}
			$$6y^3-5y^2+4y+9=6x^3-(5y^2-4y-9)$$\\
			\textit{Пример 3:}
			$$5\cdot4\cdot x^2\cdot x^3\cdot x=(5\cdot4)\cdot(x^2\cdot x^3\cdot x)=20x^6$$
		\item
			\textbf{Вынесение общего множителя за скобки}\\[0.5em]
			\textit{Пример:}
			$$3x^2y+6xy^2=3xy\cdot(x+2y)$$
		\item
			\textbf{Метод группировки}\\[0.5em]
			\textit{Пример:}
			$$ax+bx+ay+by=x(a+b)+y(a+b)=(a+b)(x+y)$$
		\item
			\textbf{Раскрытие скобок}\\[0.5em]
			\textit{Пример 1:}
			$$4a+(7b-13x)=4a+7b-13x$$
			\textit{Пример 2:}
			$$(3x+7)(-x^2+2)=-3x^3+6x-7x^2+14$$
		\item
			\textbf{Применение формул сокращенного умножения}
			$$\begin{array}{cccc}
				\text{Разность квадратов}&(a+b)(a-b)& =&a^2-b^2,\\
				\text{Квадрат суммы}&(a+b)^2& =&a^2+2ab+b^2,\\
				\text{Квадрат разности}&(a-b)^2& =&a^2-2ab+b^2,\\
				\text{Сумма кубов}&(a+b)(a^2-ab+b^2)& =&a^3+b^3,\\
				\text{Разность кубов}&(a-b)(a^2+ab+b^2)& =&a^3-b^3,\\
				\text{Куб суммы}&(a+b)^3& =&a^3+3a^2b+3ab^2+b^3,\\
				\text{Куб разности}&(a-b)^3& =&a^3-3a^2b+3ab^2-b^3.\\
			\end{array}$$
		\item
			\textbf{Применение свойств степеней}
			$$\begin{array}{c}
				a^0  =  1,\\
				a^1  =  a,\\
				a^m \cdot a^n  =  a^{m+n},\\
				a^m : a^n  =  a^{m-n},\\
				(a \cdot b)^n  =  a^n \cdot b^n,\\
				(a^n)^m  =  a^{n \cdot m},\\
				(\dfrac{a}{b})^n  =  \dfrac{a^n}{b^n},\\
			\end{array}$$
		\item
			\textbf{Приведение дробей к общему знаменателю}\\
			\textit{Пример:}
			$$\dfrac{a}{7}+\dfrac{a}{14}=\dfrac{2a}{14}+\dfrac{a}{14}=\dfrac{3a}{14}$$
		\item
			\textbf{Прибавление и последующее вычитание одного и того же числа}\\
			\textit{Пример:}
			$$x^2+2x=x^2+2x+1-1=(x+1)^2-1$$
	\end{enumerate}

	\subsection*{Упрощение выражения}
	
	\textbf{Упростить выражение} — с помощью одного или нескольких (тождественных) преобразований превратить исходное выражение в выражение с меньшим количеством действий.
	
	\textbf{Доказать тождество} — с помощью (тождественных) преобразований превратить левую часть равенства в правую или наоборот правую часть равенства в левую или же преобразовать обе части равенста так, чтобы они стали равны какому-нибудь третьему выражению или одному и тому же числу.

\end{document}