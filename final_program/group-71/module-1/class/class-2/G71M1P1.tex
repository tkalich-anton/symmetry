\documentclass[12pt, aspectratio=169]{beamer}

\usetheme[progressbar=frametitle]{metropolis}

\usepackage{pgfplots}
\usepgfplotslibrary{dateplot}

\usepackage[russian]{babel}
\usepackage[utf8]{inputenc}
\usepackage[T2A, T1]{fontenc}
\usepackage{multicol}

\usefonttheme[onlymath]{serif}

\usepackage{amsmath,amsthm,amssymb,scrextend, amsfonts}

\title{Преобразования выражений.}
\subtitle{Группа 81 \\[8pt] Модуль 1 Урок 2}
\institute{Школа Симметрия}
\date{09.09.2021}

\begin{document}
\maketitle
\begin{frame}{План урока}
	\tableofcontents[hideallsubsections, pausesections]
\end{frame}

\section{Повторение}

\begin{frame}{Повторение}
	\onslide<1->
	\textbf{Арифметическое выражение}\\
	\onslide<2->
	\textbf{Алгебраическое выражение}\\
	\onslide<3->
	\textbf{Алгебраическая сумма}\\
	\onslide<3->
	\textbf{Раскрытие скобок и заключение части выражения в скобки}\\
\end{frame}

\begin{frame}{Значение выражения}
	\onslide<1->
	\textbf{Найти значение выражения} — подставить вместо переменных заданные значения и решить числовой пример.\\[12pt]
	\textbf{Задание №1} Найдите значение выражения:
	\begin{enumerate}
		\item $1,2\cdot(4-3a)+0,4a-5,8$ при $a=-\dfrac{3}{5}$
		\item $-\dfrac{7}{9}\cdot(-0,2x-2,7)+0,6\cdot(2-3x)$ при $x=3\dfrac{2}{3}$
		\item $9x^2 - 3y^3$ при $x=-\dfrac{1}{3}, y=\dfrac{1}{2}$
		\item $1\dfrac{2}{5}\cdot(x^2-6)-\dfrac{5}{8}\cdot(0,6x-2,7)$ при $x=0,5$
	\end{enumerate}
\end{frame}

\section{Равенство и тождество}

\begin{frame}{Значение выражения}
	\onslide<1->
	\textbf{Равенство} — это два числа или выражения, соединённых между собой знаком "$=$". Эти числа или выражения называются частями равенства: слева от знака "$=$" — левая часть равенства, справа — правая часть.\\[12pt]
	\onslide<2->
	\textbf{Тождество} — это равенство, верное при любых значениях переменных.\\[12pt]
	\textbf{Тождественно равные выражения} — это выражения, значения которых равны при любых значениях переменных, которые в них входят.\\[12pt]
	\onslide<3->
	Рассмотрим выражения $2x+y$ и $2xy$.\\
	\onslide<4->
	А теперь рассмотрим выражения $3(x+y)$ и $3x+3y$.
\end{frame}

\section{Тождественные преобразования}

\begin{frame}{Тождественные преобразования.}
	\textbf{Тождественное преобразование} — это переход от исходного выражения к тождественно равному.
\end{frame}

\begin{frame}{Какие преобразования являются тождественными?}
	\textbf{Тождественное преобразование} — это переход от исходного выражения к тождественно равному.
\end{frame}

\section{Какие преобразования являются тождественными?}

\begin{frame}{Какие преобразования являются тождественными?}
	\begin{enumerate}
		\item<1-> {\textbf{Выполнение действий над числами}\\
			\textit{Пример:} $4x^2+5\cdot6-20=4x^2+10$}
		\item<2-> {\textbf{Перестановка местами слагаемых или множителей}\\
			\textit{Пример 1:} $2x+7=7+2x$;\\
			\textit{Пример 2:} $3x\cdot 2y=2y\cdot 3x$.
		}
		\item<3-> {
			\textbf{Группировка слагаемых или множителей}\\
			\textit{Пример 1:} $7x+5x^2-8=7x+(5x^2-8)$\\
			\textit{Пример 2:} $6y^3-5y^2+4y+9=6x^3-(5y^2-4y-9)$\\
			\textit{Пример 3:} $5\cdot4\cdot x^2\cdot x^3\cdot x=(5\cdot4)\cdot(x^2\cdot x^3\cdot x)=20x^6$
		}
	\end{enumerate}
\end{frame}


\begin{frame}{Какие преобразования являются тождественными?}
	\begin{enumerate}
		\setcounter{enumi}{3}
		\item<1-> {\textbf{Вынесение общего множителя за скобки}\\ \textit{Пример:} $3x^2y+6xy^2=3xy\cdot(x+2y)$}
		\item<2-> {
			\textbf{Раскрытие скобок}\\
			\textit{Пример 1:} $4a+(7b-13x)=4a+7b-13x$\\
			\textit{Пример 2:} $(3x+7)(-x^2+2)=-3x^3+6x-7x^2+14$
		}
		\item<3-> {
			\textbf{Прибавление и последующее вычитание одного и того же числа}\\
			\textit{Пример:} $x^2+2x=x^2+2x+1-1=(x+1)^2-1$
		}
		\item<4-> {
			\textbf{Пприведение подобных слагаемых}\\
			\textit{Пример:} $3x+7x-2x=8x$
		}
	\end{enumerate}
\end{frame}

\section{Практика}

\begin{frame}{Практика}
	\onslide<1->
	\textbf{Задание №2} Какое преобразование позволяет утверждать, что тождественно равны:
	\begin{multicols}{2}
		\begin{enumerate}
			\item $ab+2$ и $2+ab$
			\item $x^2+8-3$ и $x^2+5$
			\item $a\cdot b\cdot d\cdot c$ и $a\cdot c\cdot b\cdot d$
			\item $5(x+y)$ и $5x+5y$
		\end{enumerate}
	\end{multicols}
	\onslide<1->
	\textbf{Задание №3} Являются ли тождественно равными выражения?
	\begin{multicols}{2}
		\begin{enumerate}
			\item $(3a)(7b)$ и $21ab$
			\item $-2a$ и $0$
			\item $-3x+7x$ и $4x$
			\item $a+b$ и $(a+b)\cdot 0$
			\item $2a+2b$ и $(a+b)\cdot 2$
			\item $3(3x)-8x\cdot2$ и $-6x$
		\end{enumerate}
	\end{multicols}
\end{frame}

\begin{frame}{Практика}
	\onslide<1->
	\textbf{Задание №2} Найдите значение выражения:\\
	$(x-1)+(12+7x)$ при $x=0,75$
	\onslide<2->
	\textbf{Найти значение выражения} — упростить выражение, подставить вместо переменных заданные значения и решить числовой пример.
	\onslide<3->
	Найдите значение выражения:
	\begin{multicols}{2}
		\begin{enumerate}
			\item $12+7x-(1-3x)$ при $x=-1,7$
			\item $1,2x+4(0,2x+11)-(x+24))$ при $x=-5$
			\item $37-(x-16)+(11x-53)$ при $x=-1,7$
			\item $6(\dfrac{1}{2}x-\dfrac{1}{3}y)+4(3x-y)$ при $x=-0,25, y=-0,5$
		\end{enumerate}
	\end{multicols}
\end{frame}

\end{document}