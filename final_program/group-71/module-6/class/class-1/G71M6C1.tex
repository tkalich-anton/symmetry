\documentclass[12pt, a4paper]{article}
\usepackage{cmap} % Улучшенный поиск русских слов в полученном pdf-файле
\usepackage[T2A]{fontenc} % Поддержка русских букв
\usepackage[utf8]{inputenc} % Кодировка utf8
\usepackage[english, russian]{babel} % Языки: русский, английский
\usepackage{enumitem}
\usepackage{pscyr} % Нормальные шрифты
\usepackage{amsmath}
\usepackage{amsthm}
\usepackage{amssymb}
\usepackage{scrextend}
\usepackage{titling}
\usepackage{indentfirst}
\usepackage{cancel}
\usepackage{soulutf8}
\usepackage{wrapfig}
\usepackage{gensymb}
\usepackage[dvipsnames,table,xcdraw]{xcolor}
\usepackage{tikz}

%Русские символы в списке
\makeatletter
\AddEnumerateCounter{\asbuk}{\russian@alph}{щ}
\makeatother

%Дублирование знаков при переносе
\newcommand*{\hm}[1]{#1\nobreak\discretionary{}%
	{\hbox{$\mathsurround=0pt #1$}}{}}

\usepackage{graphicx}
\graphicspath{{pic/}}
\DeclareGraphicsExtensions{.pdf,.png,.jpg}

%Изменеие параметров листа
\usepackage[left=15mm,right=15mm,
top=2cm,bottom=2cm,bindingoffset=0cm]{geometry}

\usepackage{fancyhdr}
\pagestyle{fancy}
\usepackage{multicol}

\setlength\parindent{1,5em}
\usepackage{indentfirst}
\begin{document}
	
	\lhead{Группа 71}
	\chead{Модуль 6 Урок №1}
	\rhead{Школа <<Симметрия>>}
	\begin{enumerate}
		\item Докажите следующие свойства окружности:
			\begin{enumerate}[label=\asbuk*)]
			\item  диаметр, перпендикулярный хорде, делит ее пополам;
			\item  диаметр, проходящий через середину хорды, не являющейся диаметром, перпендикулярен этой хорде;
			\item окружность симметрична относительно каждого своего
			диаметра;
			\item  дуги окружности, заключенные между параллельными
			хордами, равны;
			\item  хорды, удаленные от центра окружности на равные расстояния, равны.
			\end{enumerate}
		\item Через точку окружности проведены диаметр и хорда, равная радиусу. Найдите угол между ними.
		\item Через точку $A$ окружности с центром $O$ проведены диаметр $AB$ и хорда $AC$. Докажите, что угол $BAC$ вдвое меньше угла $BOC$.
		\item Найдите угол между радиусами $OA$ и $OB$, если расстояние от центра $O$ окружности до хорды $AB$ вдвое меньше $AB$.
		\item Даны две концентрические окружности и пересекающая их прямая. Докажите, что отрезки этой прямой, заключенные между окружностями, равны.
		\item Две хорды окружности взаимно перпендикулярны. Докажите, что расстояние от точки их пересечения до центра окружности равно расстоянию между их серединами.
	\end{enumerate}
\end{document}