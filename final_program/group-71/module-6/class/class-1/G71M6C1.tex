\documentclass[12pt, a4paper]{article}
\usepackage{cmap} % Улучшенный поиск русских слов в полученном pdf-файле
\usepackage[T2A]{fontenc} % Поддержка русских букв
\usepackage[utf8]{inputenc} % Кодировка utf8
\usepackage[english, russian]{babel} % Языки: русский, английский
\usepackage{enumitem}
\usepackage{pscyr} % Нормальные шрифты
\usepackage{amsmath}
\usepackage{amsthm}
\usepackage{amssymb}
\usepackage{scrextend}
\usepackage{titling}
\usepackage{indentfirst}
\usepackage{cancel}
\usepackage{soulutf8}
\usepackage{wrapfig}
\usepackage{gensymb}
\usepackage[dvipsnames,table,xcdraw]{xcolor}
\usepackage{tikz}

%Русские символы в списке
\makeatletter
\AddEnumerateCounter{\asbuk}{\russian@alph}{щ}
\makeatother

%Дублирование знаков при переносе
\newcommand*{\hm}[1]{#1\nobreak\discretionary{}%
	{\hbox{$\mathsurround=0pt #1$}}{}}

\usepackage{graphicx}
\graphicspath{{pic/}}
\DeclareGraphicsExtensions{.pdf,.png,.jpg}

%Изменеие параметров листа
\usepackage[left=15mm,right=15mm,
top=2cm,bottom=2cm,bindingoffset=0cm]{geometry}

\usepackage{fancyhdr}
\pagestyle{fancy}
\usepackage{multicol}

\setlength\parindent{1,5em}
\usepackage{indentfirst}
\begin{document}
	
	\lhead{Группа 71}
	\chead{Модуль 6 Урок №1}
	\rhead{Школа <<Симметрия>>}
	\begin{enumerate}
		\item Через точку $A$, лежащую на окружности, проведены диаметр $AB$ и хорда $AC$, причем $AC = 8$ и $\angle AC = 30 \degree$ Найдите хорду $CM$, перпендикулярную $AB$.
		\item Через концы диаметра окружности проведены две хорды, пересекающиеся на окружности и равные $12$ и $16$. Найдите расстояния от центра окружности до этих хорд.
		\item Известно, что $AB$ — диаметр окружности, а хорды $AC$ и $BD$ параллельны. Докажите, что $AC = BD$, а $CD$ — также диаметр.
		\item Биссектрисы внутреннего и внешнего угла при вершине $A$ треугольника $ABC$ пересекают прямую $BC$ в точках $P$ и $Q$. Докажите, что окружность, построенная на отрезке $PQ$ как на диаметре, проходит через точку $A$.
		\item \textit{} На катете $AC$ прямоугольного треугольника $ABC$ как на диаметре построена окружность, пересекающая гипотенузу $AB$ в точке $K$. Найдите $CK$, если $AC = 2$ и $\angle A = 30\degree$.
	\end{enumerate}
\end{document}