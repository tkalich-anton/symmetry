\documentclass[12pt, a4paper]{article}
\usepackage{cmap} % Улучшенный поиск русских слов в полученном pdf-файле
\usepackage[T2A]{fontenc} % Поддержка русских букв
\usepackage[utf8]{inputenc} % Кодировка utf8
\usepackage[english, russian]{babel} % Языки: русский, английский
\usepackage{enumitem}
\usepackage{pscyr} % Нормальные шрифты
\usepackage{amsmath}
\usepackage{amsthm}
\usepackage{amssymb}
\usepackage{scrextend}
\usepackage{titling}
\usepackage{indentfirst}
\usepackage{cancel}
\usepackage{soulutf8}
\usepackage{wrapfig}
\usepackage{gensymb}
\usepackage[dvipsnames,table,xcdraw]{xcolor}
\usepackage{tikz}

%Русские символы в списке
\makeatletter
\AddEnumerateCounter{\asbuk}{\russian@alph}{щ}
\makeatother

%Дублирование знаков при переносе
\newcommand*{\hm}[1]{#1\nobreak\discretionary{}%
	{\hbox{$\mathsurround=0pt #1$}}{}}

\usepackage{graphicx}
\graphicspath{{pic/}}
\DeclareGraphicsExtensions{.pdf,.png,.jpg}

%Изменеие параметров листа
\usepackage[left=15mm,right=15mm,
top=2cm,bottom=2cm,bindingoffset=0cm]{geometry}

\usepackage{fancyhdr}
\pagestyle{fancy}
\usepackage{multicol}

\setlength\parindent{1,5em}
\usepackage{indentfirst}
\begin{document}
	
	\lhead{Группа 71}
	\chead{Модуль 6 Урок №4}
	\rhead{Школа <<Симметрия>>}
	\begin{enumerate}
		\item Докажите, что окружность, построенная на стороне равностороннего треугольника как на диаметре, проходит через середины двух других сторон треугольника.
		\item Докажите, что окружность, построенная на боковой стороне равнобедренного треугольника как на диаметре, проходит через середину основания.

		\item Окружность, построенная на биссектрисе $AD$ треугольника $ABC$ как на диаметре, пересекает стороны $AB$ и $AC$ соответственно в точках $M$ и $N$, отличных от $A$. Докажите, что $AM = AN$.
		\item Две окружности пересекаются в точках $A$ и $B$, $AM и AN$ — диаметры окружностей. Докажите, что точки $M$, $N$ и $B$ лежат на одной прямой.
		\item Окружность, построенная на катете прямоугольного треугольника как на диаметре, делит гипотенузу пополам. Найдите углы треугольника.
		\item Докажите, что отличная от A точка пересечения окружностей, построенных на сторонах $AB$ и $AC$ треугольника $ABC$ как на диаметрах, лежит на прямой $BC$.
		\item Окружность, построенная на катете прямоугольного треугольника как на диаметре, делит гипотенузу в отношении $1 : 3$. Найдите острые углы треугольника.
	\end{enumerate}
\end{document}