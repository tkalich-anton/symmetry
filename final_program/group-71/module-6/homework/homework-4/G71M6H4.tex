\documentclass[12pt, a4paper]{article}
\usepackage{cmap} % Улучшенный поиск русских слов в полученном pdf-файле
\usepackage[T2A]{fontenc} % Поддержка русских букв
\usepackage[utf8]{inputenc} % Кодировка utf8
\usepackage[english, russian]{babel} % Языки: русский, английский
\usepackage{enumitem}
\usepackage{pscyr} % Нормальные шрифты
\usepackage{amsmath}
\usepackage{amsthm}
\usepackage{amssymb}
\usepackage{scrextend}
\usepackage{titling}
\usepackage{indentfirst}
\usepackage{cancel}
\usepackage{soulutf8}
\usepackage{wrapfig}
\usepackage{gensymb}
\usepackage[dvipsnames,table,xcdraw]{xcolor}
\usepackage{tikz}

%Русские символы в списке
\makeatletter
\AddEnumerateCounter{\asbuk}{\russian@alph}{щ}
\makeatother

%Дублирование знаков при переносе
\newcommand*{\hm}[1]{#1\nobreak\discretionary{}%
	{\hbox{$\mathsurround=0pt #1$}}{}}

\usepackage{graphicx}
\graphicspath{{pic/}}
\DeclareGraphicsExtensions{.pdf,.png,.jpg}

%Изменеие параметров листа
\usepackage[left=15mm,right=15mm,
top=2cm,bottom=2cm,bindingoffset=0cm]{geometry}

\usepackage{fancyhdr}
\pagestyle{fancy}
\usepackage{multicol}

\setlength\parindent{1,5em}
\usepackage{indentfirst}
\begin{document}
	
	\lhead{Группа 71}
	\chead{Модуль 6 Домашняя работа №4}
	\rhead{Школа <<Симметрия>>}
	\section*{Домашняя работа №4}
	\begin{enumerate}
		\item \textit{(2 балла)} К окружности, вписанной в квадрат со стороной, равной a, проведена касательная, пересекающая две его стороны.
		Найдите периметр отсеченного треугольника.

		\item \textit{(2 балла)} Из точки $M$, лежащей вне двух концентрических окружностей, проведены четыре прямые, касающиеся окружностей в точках $A$, $B$, $C$ и $D$. Докажите, что точки $M$, $A$, $B$, $C$, $D$ расположены на одной окружности.
		\item \textit{(3 балла)} Окружность, вписанная в треугольник $ABC$, касается его сторон $AB$, $BC$ и $AC$ соответственно в точках $K$, $M$ и $N$.
		Найдите угол $KMN$, если $\angle A = 70\degree$.
		\item \textit{(3 балла)} В равнобедренный треугольник с основанием, равным $a$, вписана окружность и к ней проведены три касательные так, что они отсекают от данного треугольника три маленьких треугольника, сумма периметров которых равна $b$. Найдите боковую сторону данного треугольника.
	\end{enumerate}
\end{document}