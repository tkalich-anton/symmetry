\documentclass[12pt, a4paper]{article}
\usepackage{cmap} % Улучшенный поиск русских слов в полученном pdf-файле
\usepackage[T2A]{fontenc} % Поддержка русских букв
\usepackage[utf8]{inputenc} % Кодировка utf8
\usepackage[english, russian]{babel} % Языки: русский, английский
\usepackage{enumitem}
\usepackage{pscyr} % Нормальные шрифты
\usepackage{amsmath}
\usepackage{amsthm}
\usepackage{amssymb}
\usepackage{scrextend}
\usepackage{titling}
\usepackage{indentfirst}
\usepackage{cancel}
\usepackage{soulutf8}
\usepackage{wrapfig}
\usepackage{gensymb}
\usepackage[dvipsnames,table,xcdraw]{xcolor}
\usepackage{tikz}

%Русские символы в списке
\makeatletter
\AddEnumerateCounter{\asbuk}{\russian@alph}{щ}
\makeatother

%Дублирование знаков при переносе
\newcommand*{\hm}[1]{#1\nobreak\discretionary{}%
	{\hbox{$\mathsurround=0pt #1$}}{}}

\usepackage{graphicx}
\graphicspath{{pic/}}
\DeclareGraphicsExtensions{.pdf,.png,.jpg}

%Изменеие параметров листа
\usepackage[left=15mm,right=15mm,
top=2cm,bottom=2cm,bindingoffset=0cm]{geometry}

\usepackage{fancyhdr}
\pagestyle{fancy}
\usepackage{multicol}

\setlength\parindent{1,5em}
\usepackage{indentfirst}
\begin{document}
	
	\lhead{Группа 71}
	\chead{Модуль 6 Домашняя работа №1}
	\rhead{Школа <<Симметрия>>}
	\begin{enumerate}
		\item \textit{(2 балла)} Угол между радиусами $OA$ и $OB$ окружности равен $60\degree$. Найдите хорду $AB$, если радиус окружности равен $R$.
		\item \textit{(2 балла)} Найдите угол между радиусами $OA$ и $OB$, если расстояние от центра $O$ окружности до хорды $AB$ вдвое меньше $OA$.
		\item \textit{(3 балла)} Равные хорды окружности с центром $O$ пересекаются в точке $M$. Докажите, что $MO$ — биссектриса угла между ними.
		\item \textit{(3 балла)} Прямая, проходящая через общую точку $A$ двух окружностей, пересекает вторично эти окружности в точках $B$ и $C$ соответственно. Расстояние между проекциями центров окружностей на эту прямую равно $12$. Найдите $BC$, если известно, что точка $A$ лежит на отрезке $BC$.

	\end{enumerate}
\end{document}