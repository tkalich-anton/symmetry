\documentclass[12pt, a4paper]{article}
\usepackage{../../../../../style}
\begin{document}
\lhead{Группа 71} \chead{Модуль 6 Домашняя работа №1} \rhead{Школа <<Симметрия>>} \cfoot{}
\begin{enumerate}
	\item \textit{(2 балла)} Найдите значение выражения: 
	\begin{enumerate}[label=\asbuk*)]
		\item $(x+y)^3-x^3-3xy(x+y)$, при $x=-\dfrac{1}{2}; y=2$
		\item $(a-b)^3+(2a-b)^3-3a^3$, при $a=-0,2; b=-0,1$
	\end{enumerate}	
	\item \textit{(2 балла)} Угол между радиусами $OA$ и $OB$ окружности равен $60\degree$. Найдите хорду $AB$, если радиус окружности равен $7$.
	\item \textit{(2 балла)} Найдите угол между радиусами $OA$ и $OB$, если расстояние от центра $O$ окружности до хорды $AB$ вдвое меньше $OA$.
	\item \textit{(2 балла)} Равные хорды окружности с центром $O$ пересекаются в точке $M$. Докажите, что $MO$ — биссектриса угла между ними.
	\item \textit{(2 балла)} Прямая, проходящая через общую точку $A$ двух окружностей, пересекает вторично эти окружности в точках $B$ и $C$ соответственно. Расстояние между проекциями центров окружностей на эту прямую равно $12$. Найдите $BC$, если известно, что точка $A$ лежит на отрезке $BC$.

\end{enumerate}
\end{document}