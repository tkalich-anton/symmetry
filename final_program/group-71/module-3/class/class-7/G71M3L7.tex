\documentclass[12pt, a4paper]{article}
\usepackage{cmap} % Улучшенный поиск русских слов в полученном pdf-файле
\usepackage[T2A]{fontenc} % Поддержка русских букв
\usepackage[utf8]{inputenc} % Кодировка utf8
\usepackage[english, russian]{babel} % Языки: русский, английский
\usepackage{enumitem}
\usepackage{pscyr} % Нормальные шрифты
\usepackage{soulutf8}
\usepackage{amsmath}
\usepackage{amsthm}
\usepackage{amssymb}
\usepackage{scrextend}
\usepackage{titling}
\usepackage{indentfirst}
\usepackage{cancel}
\usepackage{soulutf8}
\usepackage{wrapfig}
\usepackage{gensymb}
\usepackage[dvipsnames,table,xcdraw]{xcolor}
\usepackage{tikz}

%Русские символы в списке
\makeatletter
\AddEnumerateCounter{\asbuk}{\russian@alph}{щ}
\makeatother

%Дублирование знаков при переносе
\newcommand*{\hm}[1]{#1\nobreak\discretionary{}%
	{\hbox{$\mathsurround=0pt #1$}}{}}

\usepackage{graphicx}
\graphicspath{{pic/}}
\DeclareGraphicsExtensions{.pdf,.png,.jpg}

%Изменеие параметров листа
\usepackage[left=15mm,right=15mm,
top=2cm,bottom=2cm,bindingoffset=0cm]{geometry}


\usepackage{fancyhdr}
\pagestyle{fancy}
\usepackage{multicol}

\setlength\parindent{1,5em}
\usepackage{indentfirst}

\begin{document}
		
\lhead{Группа 71}
\chead{Модуль 3 Урок 1}
\rhead{Школа <<Симметрия>>}

\begin{enumerate}
	\item Вычислите:
	$$4,51\cdot3\dfrac{1}{2}-7\dfrac{2}{3}-\left(-5,49\cdot3\dfrac{1}{2}+10\dfrac{1}{3}\right)$$
	\item Упростите выражение и найдите значение выражения:
	$$9a^2b^2+4aab-5ba^2+7a^2b-aba-(3ab)^2$$ при $a=1; b=\dfrac{1}{2}$
	\item Упростите выражение:
	\begin{multicols}{2}
		\begin{enumerate}[label=\asbuk*)]
			\item $(x-3)(x+3)$
			\item $(7x^2+a^2)(x^2-3a^2)$
			\item $(x^2+xy-y^2)(x+y)$
			\item $(b-c)(b^2-bc-c^2)$
			\item $-3b^3(b+2)(1-b)$
			\item $(x+1)(x+2)(x+3)$
		\end{enumerate}
	\end{multicols}
	\item Докажите тождество:
	\begin{multicols}{2}
		\begin{enumerate}[label=\asbuk*)]
			\item $(c-8)(c+3)=c^2-5c-24$
			\item $16-(a+3)(a+2)=4-(6+a)(a-1)$
		\end{enumerate}
	\end{multicols}
	\item Докажите, что при всех целых $n$ значение выражения
	$n(n-1)-(n+3)(n+2)$
	делится на 6.
	\item Решить уравнение:
	\begin{multicols}{2}
		\begin{enumerate}[label=\asbuk*)]
			\item $5(x-1)-4(x-2)=10$
			\item $2(x+2)-3(x-2)=5-4(3x-1)$
		\end{enumerate}
	\end{multicols}
	\item Два поезда вышли одновременно навстречу друг другу с двух станций, удаленных друг от друга на 520 км. Через какое время расстояние между поездами будет равно 65 км, если скорость одного поезда 60 км/ч, а второго 70 км/ч?
\end{enumerate}

\end{document}