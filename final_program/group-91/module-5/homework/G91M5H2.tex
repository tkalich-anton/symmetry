\documentclass[12pt, a4paper]{article}
\usepackage{cmap} % Улучшенный поиск русских слов в полученном pdf-файле
\usepackage[T2A]{fontenc} % Поддержка русских букв
\usepackage[utf8]{inputenc} % Кодировка utf8
\usepackage[english, russian]{babel} % Языки: русский, английский
\usepackage{enumitem}
\usepackage{pscyr} % Нормальные шрифты
\usepackage{amsmath}
\usepackage{amsthm}
\usepackage{amssymb}
\usepackage{scrextend}
\usepackage{titling}
\usepackage{indentfirst}
\usepackage{cancel}
\usepackage{soulutf8}
\usepackage{wrapfig}
\usepackage{gensymb}
\usepackage[dvipsnames,table,xcdraw]{xcolor}
\usepackage{tikz}

%Русские символы в списке
\makeatletter
\AddEnumerateCounter{\asbuk}{\russian@alph}{щ}
\makeatother

%Дублирование знаков при переносе
\newcommand*{\hm}[1]{#1\nobreak\discretionary{}%
	{\hbox{$\mathsurround=0pt #1$}}{}}

\usepackage{graphicx}
\graphicspath{{pic/}}
\DeclareGraphicsExtensions{.pdf,.png,.jpg}

%Изменеие параметров листа
\usepackage[left=15mm,right=15mm,
top=2cm,bottom=2cm,bindingoffset=0cm]{geometry}

\usepackage{fancyhdr}
\pagestyle{fancy}
\usepackage{multicol}

\setlength\parindent{1,5em}
\usepackage{indentfirst}
\begin{document}
	
	\lhead{Группа 91}
	\chead{Модуль 5 Домашняя работа №2}
	\rhead{Школа <<Симметрия>>}
	\section*{Домашняя работа №2}
	\begin{enumerate}
		\item \textit{(1 балл)} Вычислите:
		\begin{multicols}{2}
			\begin{enumerate}[label=\asbuk*)]
				\item $42 \cdot 53 - 32 \cdot 53 - 42 \cdot 63 + 32 \cdot 63$
				\item $(3,2+(-1,7))+1,7$
				\item $(937-811):63+\dfrac{3-21}{9}-2\cdot(7-2^4:2)$
				\item $\left( -12\dfrac{2}{3}\right):3\dfrac{1}{6} +13,5:4,5$
			\end{enumerate}
		\end{multicols}
		\item \textit{(2 балла)} Решите уравнения:
		\begin{multicols}{2}
			\begin{enumerate}[label=\asbuk*)]
				\item $x+3=2x-4$
				\item $2x-6=3x$
				\item $3x-5=-2x+7+5x-12$
				\item $7-0,2x-(21,28-1,6)=0$
				\item $5-3(x+5)=7-(2+3x)$
				\item $1\dfrac{1}{5}-0,5x-0,4+\dfrac{2}{5}x=0$
			\end{enumerate}
		\end{multicols}
		\item \textit{(2 балла)} Решите квадратные уравнения:
		\begin{multicols}{2}
			\begin{enumerate}[label=\asbuk*)]
				\item $5x^2-6x+1=0$
				\item $2x^2+5x+3=0$
				\item $x^2-6x+8=0$
				\item $x^2-5\dfrac{1}{5}x+1=0$
			\end{enumerate}
		\end{multicols}
		\item \textit{(1 балл)} Решите неравенства:
		\begin{multicols}{2}
			\begin{enumerate}[label=\asbuk*)]
				\item $3x+8>2x+1$
				\item $-2y+3<y$
				\item $0,1x-0,2>\dfrac{1}{100}$
				\item $3>5x+19$
			\end{enumerate}
		\end{multicols}
		\item \textit{(1 балл)} Саша с папой решили покататься на колесе обозрения. Всего на колесе десять кабинок, из них $5$ – синие, $2$ – зелёные, остальные – красные. Кабинки по очереди подходят к платформе для посадки. Найдите вероятность того, что Саша прокатится в красной кабинке.
		\item \textit{(1 балл)} Определите вероятность того, что при бросании игрального кубика (правильной кости) выпадет нечетное число очков.
		\item \textit{(1 балл)} Грузовик перевозит партию щебня массой $90$ тонн, ежедневно увеличивая норму перевозки на одно и то же число тонн. Известно, что за первый день было перевезено $2$ тонны щебня. Определите, сколько тонн щебня было перевезено за десятый день, если вся работа была выполнена за $12$ дней.
		\item \textit{(1 балл)} Васе надо решить $245$ задач. Ежедневно он решает на одно и то же количество задач больше по сравнению с предыдущим днём. Известно, что за первый день Вася решил $11$ задач. Определите, сколько задач решил Вася в последний день, если со всеми задачами он справился за $7$ дней.
	\end{enumerate}
\end{document}