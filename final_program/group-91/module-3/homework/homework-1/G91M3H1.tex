	\documentclass[12pt, a4paper]{article}
\usepackage{cmap} % Улучшенный поиск русских слов в полученном pdf-файле
\usepackage[T2A]{fontenc} % Поддержка русских букв
\usepackage[utf8]{inputenc} % Кодировка utf8
\usepackage[english, russian]{babel} % Языки: русский, английский
\usepackage{enumitem}
\usepackage{pscyr} % Нормальные шрифты
\usepackage{soulutf8}
\usepackage{amsmath}
\usepackage{amsthm}
\usepackage{amssymb}
\usepackage{scrextend}
\usepackage{titling}
\usepackage{indentfirst}
\usepackage{cancel}
\usepackage{soulutf8}
\usepackage{wrapfig}
\usepackage{gensymb}
\usepackage[dvipsnames,table,xcdraw]{xcolor}
\usepackage{tikz}

%Русские символы в списке
\makeatletter
\AddEnumerateCounter{\asbuk}{\russian@alph}{щ}
\makeatother

%Дублирование знаков при переносе
\newcommand*{\hm}[1]{#1\nobreak\discretionary{}%
	{\hbox{$\mathsurround=0pt #1$}}{}}

\usepackage{graphicx}
\graphicspath{{pic/}}
\DeclareGraphicsExtensions{.pdf,.png,.jpg}

%Изменеие параметров листа
\usepackage[left=15mm,right=15mm,
top=2cm,bottom=2cm,bindingoffset=0cm]{geometry}


\usepackage{fancyhdr}
\pagestyle{fancy}
\usepackage{multicol}

\setlength\parindent{1,5em}
\usepackage{indentfirst}

\begin{document}
		
\lhead{Группа 91}
\chead{Модуль 3 ДЗ№1}
\rhead{Школа <<Симметрия>>}

\begin{enumerate}
	\item \textit(2 балла) Вычислить:
	$$\dfrac{\dfrac{5}{6}-\dfrac{21}{45}}{1\dfrac{5}{6}}\cdot\dfrac{1,125+1\dfrac{3}{4}-\dfrac{5}{12}}{0,59}$$
	\item \textit(2 балла) Упростить выражение $$\left(\dfrac{a+b}{b}-\dfrac{a}{a+b}\right):\left(\dfrac{a+b}{a}-\dfrac{b}{a+b}\right)$$
	\item \textit(1 балл) Вычислить значение корня:
	\begin{multicols}{2}
		\begin{enumerate}[label=\asbuk*)]
			\item $\sqrt{36\cdot2500}$
			\item $\sqrt{5\dfrac{4}{9}}$
			\item $\sqrt{\dfrac{36}{100}\cdot\dfrac{4}{25}\cdot\dfrac{64}{144}}$
			\item $\sqrt{360\cdot90}$
			\item $\sqrt{4,9\cdot12,1}$
		\end{enumerate}
	\end{multicols}
	\item \textit(1 балл) Вычислите значение выражения:
	\begin{multicols}{2}
		\begin{enumerate}[label=\asbuk*)]
			\item $\sqrt{11^4}$
			\item $\sqrt{3^4\cdot3^2}$
			\item $\sqrt{0,16}+(2\sqrt{0,1})^2$
			\item $(5\sqrt{2})^2-(2\sqrt{5})^2$
			\item $\sqrt{18 \cdot 50}$
			\item $\sqrt{21 \cdot 35 \cdot 15}$
		\end{enumerate}
	\end{multicols}
	\item \textit(1 балл) Сравните значения выражений: $3\sqrt{3}$ и $\sqrt{12}$
	\item \textit(1 балл) Расположите в порядке возрастания числа:
	$3\sqrt{3}$, $2\sqrt{6}$, $\sqrt{29}$, $4\sqrt{2}$, $2\sqrt{11}$
	\item \textit(2 балла) Решить уравнение:
	\begin{multicols}{2}
		\begin{enumerate}[label=\asbuk*)]
			\item $(x+4)-(x-1)=6x$
			\item $1,6x-(x-2,8)=(0,2x+1,5)-0,7$
			\item $15(x+2)-30=12x$
			\item $2x^2=5+3x$
			\item $2x^2-3x=0$
			\item $(x+4)(x-6)=0$
			\item $(2x-3)(x^2+3x+2)=0$
			\item $(x^2+1)(x^2+5x+6)=0$
			\item $x^3-4x^2=x$
		\end{enumerate}
	\end{multicols}
\end{enumerate}

\end{document}