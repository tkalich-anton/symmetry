\documentclass[12pt, a4paper]{article}
\usepackage{cmap} % Улучшенный поиск русских слов в полученном pdf-файле
\usepackage[T2A]{fontenc} % Поддержка русских букв
\usepackage[utf8]{inputenc} % Кодировка utf8
\usepackage[english, russian]{babel} % Языки: русский, английский
\usepackage{enumitem}
\usepackage{pscyr} % Нормальные шрифты
\usepackage{soulutf8}
\usepackage{amsmath}
\usepackage{amsthm}
\usepackage{amssymb}
\usepackage{scrextend}
\usepackage{titling}
\usepackage{indentfirst}
\usepackage{cancel}
\usepackage{soulutf8}
\usepackage{wrapfig}
\usepackage{gensymb}
\usepackage[dvipsnames,table,xcdraw]{xcolor}
\usepackage{tikz}

%Русские символы в списке
\makeatletter
\AddEnumerateCounter{\asbuk}{\russian@alph}{щ}
\makeatother

%Дублирование знаков при переносе
\newcommand*{\hm}[1]{#1\nobreak\discretionary{}%
	{\hbox{$\mathsurround=0pt #1$}}{}}

\usepackage{graphicx}
\graphicspath{{pic/}}
\DeclareGraphicsExtensions{.pdf,.png,.jpg}

%Изменеие параметров листа
\usepackage[left=15mm,right=15mm,
top=2cm,bottom=2cm,bindingoffset=0cm]{geometry}


\usepackage{fancyhdr}
\pagestyle{fancy}
\usepackage{multicol}

\setlength\parindent{1,5em}
\usepackage{indentfirst}

\begin{document}
		
\lhead{Группа 91}
\chead{Модуль 3 Урок 1}
\rhead{Школа <<Симметрия>>}

\begin{enumerate}
	\item Вычислить:
	$$\dfrac{\dfrac{5}{6}-\dfrac{21}{45}}{1\dfrac{5}{6}}\cdot\dfrac{1,125+1\dfrac{3}{4}-\dfrac{5}{12}}{0,59}$$
	\item Упростить выражение
	\begin{multicols}{2}
		\begin{enumerate}[label=\asbuk*)]
			\item $\dfrac{x^2-2x}{x-3}-\dfrac{4x-9}{x-3}$
			\item $\dfrac{a^2-4a+4}{a^2+ab-2a-2b}$
			\item $\left(\dfrac{y^2-xy}{x^2+xy}-xy+y^2\right)\cdot\dfrac{x}{x-y}+\dfrac{y}{x+y}$
			\item $\left(\dfrac{a+b}{b}-\dfrac{a}{a+b}\right):\left(\dfrac{a+b}{a}-\dfrac{b}{a+b}\right)$
		\end{enumerate}
	\end{multicols}
	\item Вычислить значение корня:
	\begin{multicols}{2}
		\begin{enumerate}[label=\asbuk*)]
			\item $\sqrt{81\cdot900}$
			\item $\sqrt{12\dfrac{1}{4}}$
			\item $\sqrt{\dfrac{25}{81}\cdot\dfrac{16}{49}\cdot\dfrac{196}{9}}$
			\item $\sqrt{810\cdot40}$
			\item $\sqrt{8\cdot98}$
			\item $\sqrt{2,5\cdot14,4}$
		\end{enumerate}
	\end{multicols}
	\item Вычислите значение выражения:
	\begin{multicols}{2}
		\begin{enumerate}[label=\asbuk*)]
			\item $\sqrt{28}\cdot\sqrt{7}$
			\item $\sqrt{50}\cdot\sqrt{4,5}$
			\item $\sqrt{1,2}\cdot\sqrt{3\dfrac{1}{3}}$
			\item $\sqrt{5^2}$
			\item $\sqrt{11^4}$
			\item $\sqrt{3^4\cdot3^2}$
			\item $\sqrt{0,16}+(2\sqrt{0,1})^2$
			\item $(5\sqrt{2})^2-(2\sqrt{5})^2$
		\end{enumerate}
	\end{multicols}
	\item Сравните значения выражений:
	\begin{multicols}{2}
		\begin{enumerate}[label=\asbuk*)]
			\item $3\sqrt{3}$ и $\sqrt{12}$
			\item $2\sqrt{5}$ и $3\sqrt{2}$
		\end{enumerate}
	\end{multicols}
	\item Расположите в порядке возрастания числа:
	$3\sqrt{3}$, $2\sqrt{6}$, $\sqrt{29}$, $4\sqrt{2}$, $2\sqrt{11}$
	\item Решить уравнение:
	\begin{multicols}{2}
		\begin{enumerate}[label=\asbuk*)]
			\item $(x+4)-(x-1)=6x$
			\item $1,6x-(x-2,8)=(0,2x+1,5)-0,7$
			\item $15(x+2)-30=12x$
			\item $3x+(x-2)=2(2x-1)$
			\item $\dfrac{4x-1}{12}+\dfrac{7}{4}=\dfrac{5-x}{9}$
			\item $3x^2-8x+5=0$
			\item $5x^2=9x+2$
			\item $6x+9=x^2$
		\end{enumerate}
	\end{multicols}
	\item Упростить выражение:
	\begin{multicols}{2}
		\begin{enumerate}[label=\asbuk*)]
			\item $(1-\sqrt{x})(1+\sqrt{x}+x)$
			\item $(\sqrt{a}+2)(a-2\sqrt{a}+4)$
		\end{enumerate}
	\end{multicols}
\end{enumerate}

\end{document}