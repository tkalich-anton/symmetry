\documentclass[12pt, a4paper]{article}
\usepackage{cmap} % Улучшенный поиск русских слов в полученном pdf-файле
\usepackage[T2A]{fontenc} % Поддержка русских букв
\usepackage[utf8]{inputenc} % Кодировка utf8
\usepackage[english, russian]{babel} % Языки: русский, английский
\usepackage{enumitem}
\usepackage{pscyr} % Нормальные шрифты
\usepackage{soulutf8}
\usepackage{amsmath}
\usepackage{amsthm}
\usepackage{amssymb}
\usepackage{scrextend}
\usepackage{titling}
\usepackage{indentfirst}
\usepackage{cancel}
\usepackage{soulutf8}
\usepackage{wrapfig}
\usepackage{gensymb}
\usepackage[dvipsnames,table,xcdraw]{xcolor}
\usepackage{tikz}

%Русские символы в списке
\makeatletter
\AddEnumerateCounter{\asbuk}{\russian@alph}{щ}
\makeatother

%Дублирование знаков при переносе
\newcommand*{\hm}[1]{#1\nobreak\discretionary{}%
	{\hbox{$\mathsurround=0pt #1$}}{}}

\usepackage{graphicx}
\graphicspath{{pic/}}
\DeclareGraphicsExtensions{.pdf,.png,.jpg}

%Изменеие параметров листа
\usepackage[left=15mm,right=15mm,
top=2cm,bottom=2cm,bindingoffset=0cm]{geometry}


\usepackage{fancyhdr}
\pagestyle{fancy}
\usepackage{multicol}

\setlength\parindent{1,5em}
\usepackage{indentfirst}

\begin{document}
		
\lhead{Группа 91}
\chead{Модуль 4 Урок 3}
\rhead{Школа <<Симметрия>>}

\begin{enumerate}
	\item Вычислить: $$5\dfrac{4}{7}:1\dfrac{5}{21}-\left(5\dfrac{2}{15}\cdot\dfrac{3}{22}+1\dfrac{14}{15}\right)$$
	\item Вычислить: $$(\sqrt{6}+1)(\sqrt{6}-1)$$
	\item Решить уравнение:$$x^2+x-2=0$$
	\item Решить уравнение:$$(5-x)(3x+2)=0$$
	\item Решить уравнение:$$(x^2+2x+1)(x^2-5x+7)=0$$
	\item Решить уравнение:$$\dfrac{x^2+2x}{x-2}=0$$
	\item В прямоугольном треугольнике $ABC$ ($\angle C = 90\degree$) известно, что $AB = 4$, $\angle A = 60\degree$. Найдите $BC$ и $AC$.
	\item Основания прямоугольной трапеции равны $6$ и $8$. Один из углов при меньшем основании равен $120\degree$. Найдите диагонали трапеции.
	\item Найдите сторону квадрата, вписанного в окружность радиуса $8$.
	
\end{enumerate}

\end{document}