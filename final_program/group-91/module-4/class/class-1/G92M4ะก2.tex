\documentclass[12pt, a4paper]{article}
\usepackage{cmap} % Улучшенный поиск русских слов в полученном pdf-файле
\usepackage[T2A]{fontenc} % Поддержка русских букв
\usepackage[utf8]{inputenc} % Кодировка utf8
\usepackage[english, russian]{babel} % Языки: русский, английский
\usepackage{enumitem}
\usepackage{pscyr} % Нормальные шрифты
\usepackage{soulutf8}
\usepackage{amsmath}
\usepackage{amsthm}
\usepackage{amssymb}
\usepackage{scrextend}
\usepackage{titling}
\usepackage{indentfirst}
\usepackage{cancel}
\usepackage{soulutf8}
\usepackage{wrapfig}
\usepackage{gensymb}
\usepackage[dvipsnames,table,xcdraw]{xcolor}
\usepackage{tikz}

%Русские символы в списке
\makeatletter
\AddEnumerateCounter{\asbuk}{\russian@alph}{щ}
\makeatother

%Дублирование знаков при переносе
\newcommand*{\hm}[1]{#1\nobreak\discretionary{}%
	{\hbox{$\mathsurround=0pt #1$}}{}}

\usepackage{graphicx}
\graphicspath{{pic/}}
\DeclareGraphicsExtensions{.pdf,.png,.jpg}

%Изменеие параметров листа
\usepackage[left=15mm,right=15mm,
top=2cm,bottom=2cm,bindingoffset=0cm]{geometry}


\usepackage{fancyhdr}
\pagestyle{fancy}
\usepackage{multicol}

\setlength\parindent{1,5em}
\usepackage{indentfirst}

\begin{document}
	
	\lhead{Группа 91}
	\chead{Модуль 4 Урок №2}
	\rhead{Школа <<Симметрия>>}
	\begin{enumerate}
		\item Вычислить $$\left(2\dfrac{1}{2}:10+10:2\dfrac{1}{2}-2\dfrac{1}{6}\right)\cdot\dfrac{36}{125}$$
		\item Упростить выражение $$2(2x-4)(3x+1)-\dfrac{1}{2}(6x-4)(x+8)$$ и найти значение выражения при $x=-4$
		\item Сократить дробь $$\dfrac{x^2-xy}{2xy+2x^2}$$
		\item Решить систему неравенств 
		$$
		\left\{
		\begin{array}{l}
			\dfrac{3x+2}{2}>\dfrac{2x+3}{3}\\\\
			\dfrac{x+2}{3}<\dfrac{x+3}{2}\\
		\end{array}
		\right.
		$$
		\item Решить систему неравенств 
		$$
		\left\{
		\begin{array}{l}
			x^2-14x+45 < 0\\
			x^2-11x+30 > 0\\
		\end{array}
		\right.
		$$
		\item Диагонали прямоугольника равны 8 и пересекаются под углом в $60\degree$. Найдите меньшую сторону прямоугольника.
		\item Высота параллелограмма, проведенная из вершины тупого угла, равна $2$ и делит сторону параллелограмма пополам. Острый угол параллелограмма равен $30\degree$. Найдите диагональ, проведенную из вершины тупого угла, и углы, которые она образует со сторонами.
	\end{enumerate}
\end{document}