\documentclass[12pt, a4paper]{article}
\usepackage{cmap} % Улучшенный поиск русских слов в полученном pdf-файле
\usepackage[T2A]{fontenc} % Поддержка русских букв
\usepackage[utf8]{inputenc} % Кодировка utf8
\usepackage[english, russian]{babel} % Языки: русский, английский
\usepackage{enumitem}
\usepackage{pscyr} % Нормальные шрифты
\usepackage{amsmath}
\usepackage{amsthm}
\usepackage{amssymb}
\usepackage{scrextend}
\usepackage{titling}
\usepackage{indentfirst}
\usepackage{cancel}
\usepackage{soulutf8}
\usepackage{wrapfig}
\usepackage{gensymb}
\usepackage[dvipsnames,table,xcdraw]{xcolor}
\usepackage{tikz}

%Русские символы в списке
\makeatletter
\AddEnumerateCounter{\asbuk}{\russian@alph}{щ}
\makeatother

%Дублирование знаков при переносе
\newcommand*{\hm}[1]{#1\nobreak\discretionary{}%
	{\hbox{$\mathsurround=0pt #1$}}{}}

\usepackage{graphicx}
\graphicspath{{pic/}}
\DeclareGraphicsExtensions{.pdf,.png,.jpg}

%Изменеие параметров листа
\usepackage[left=15mm,right=15mm,
top=2cm,bottom=2cm,bindingoffset=0cm]{geometry}

\usepackage{fancyhdr}
\pagestyle{fancy}
\usepackage{multicol}
\setlength\parindent{1,5em}
\usepackage{indentfirst}
\begin{document}
	
	\lhead{Группа 91}
	\chead{Модуль 6 Урок №4}
	\rhead{Школа <<Симметрия>>}
	\begin{enumerate}
		\item Решите квадратные уравнения:
		\begin{multicols}{2}
			\begin{enumerate}[label=\asbuk*)]
			\item $0,5x(2+x)=0$
			\item $3x(x-0,5)=0$
			\item $(x-7)(7+x)=0$
			\item $3(x-5)(5+x)=0$
			\item $0,8(x+1)(1-x)$
		\end{enumerate}
		\end{multicols}
	\item Решите линейные неравенства:
	\begin{multicols}{2}
		\begin{enumerate}[label=\asbuk*)]
		\item $-12y+25\leqslant12y-25$
		\item $100x+223>\dfrac{1}{100}x-223,01$
		\item $50x+12<(x-3)\cdot3$
		\item $6y+3>2y-2-5y$
		\item $3x+3\geqslant\dfrac{3}{5}$
		\item $120x>840$
		\item $3\dfrac{1}{9}x+14<5\dfrac{2}{9}x-10$
		\item $8\dfrac{1}{4}+x\geqslant5\dfrac{1}{2}$
		\end{enumerate}
	\end{multicols}
	\end{enumerate}
\end{document}