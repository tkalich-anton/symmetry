\documentclass[12pt, a4paper]{article}
\usepackage{cmap} % Улучшенный поиск русских слов в полученном pdf-файле
\usepackage[T2A]{fontenc} % Поддержка русских букв
\usepackage[utf8]{inputenc} % Кодировка utf8
\usepackage[english, russian]{babel} % Языки: русский, английский
\usepackage{enumitem}
\usepackage{pscyr} % Нормальные шрифты
\usepackage{amsmath}
\usepackage{amsthm}
\usepackage{amssymb}
\usepackage{scrextend}
\usepackage{titling}
\usepackage{indentfirst}
\usepackage{cancel}
\usepackage{soulutf8}
\usepackage{wrapfig}
\usepackage{gensymb}
\usepackage[dvipsnames,table,xcdraw]{xcolor}
\usepackage{tikz}

%Русские символы в списке
\makeatletter
\AddEnumerateCounter{\asbuk}{\russian@alph}{щ}
\makeatother

%Дублирование знаков при переносе
\newcommand*{\hm}[1]{#1\nobreak\discretionary{}%
	{\hbox{$\mathsurround=0pt #1$}}{}}

\usepackage{graphicx}
\graphicspath{{pic/}}
\DeclareGraphicsExtensions{.pdf,.png,.jpg}

%Изменеие параметров листа
\usepackage[left=15mm,right=15mm,
top=2cm,bottom=2cm,bindingoffset=0cm]{geometry}

\usepackage{fancyhdr}
\pagestyle{fancy}
\usepackage{multicol}
\setlength\parindent{1,5em}
\usepackage{indentfirst}
\begin{document}
	
	\lhead{Группа 91}
	\chead{Модуль 6 Урок №5}
	\rhead{Школа <<Симметрия>>}
	\begin{enumerate}
		\item Стас выбирает трёхзначное число. Найдите вероятность того, что оно делится на $48$.
		\item Телевизор у Саши сломался и показывает только один случайный канал. Саша включает телевизор. В это время по трем каналам из тридцати показывают кинокомедии. Найдите вероятность того, что Саша попадет на канал, где комедия не идет.
		\item В фирме такси в данный момент свободно $30$ машин: $1$ чёрная, $9$ жёлтых и $20$ зелёных. По вызову выехала одна из машин, случайно оказавшаяся ближе всего к заказчику. Найдите вероятность того, что к нему приедет жёлтое такси.
		\item В денежно-вещевой лотерее на $100000$ билетов разыгрывается $1250$ вещевых и $810$ денежных выигрышей. Какова вероятность денежного выигрыша?
		\item В коробке $14$ пакетиков с чёрным чаем и $6$ пакетиков с зелёным чаем. Павел наугад вынимает один пакетик. Какова вероятность того, что это пакетик с зелёным чаем?
		\item В лыжных гонках участвуют $13$ спортсменов из России, $2$ спортсмена из Норвегии и $5$ спортсменов из Швеции. Порядок, в котором спортсмены стартуют, определяется жребием. Найдите вероятность того, что первым будет стартовать спортсмен не из России.
		\item В среднем из $50$ карманных фонариков, поступивших в продажу, семь неисправных. Найдите вероятность того, что выбранный наудачу в магазине фонарик окажется исправен.
		\item Арифметическая прогрессия $a_n$ задана формулой $n$-го члена $a_{n+1}=a_n+2$  и известно, что $a_1=3$. Найдите пятый член этой прогрессии.
		\item Грузовик перевозит партию щебня массой $360$ тонн, ежедневно увеличивая норму перевозки на одно и то же число тонн. Известно, что за первый день было перевезено $3$ тонны щебня. Определите, сколько тонн щебня было перевезено за девятый день, если вся работа была выполнена за $18$ дней.
		\item Бригада рабочих могла выполнить всю работу за $24$ ч, если бы работали одновременно все рабочие. Однако по плану в первый час работал один рабочий, во второй час – $2$ рабочих, в третий – $3$ и т.д. до тез пор, пока в работу не включились все рабочие. И только несколько часов перед завершением работала вся бригада. Время работы, предусмотренное планом, было бы сокращено на 6 часов, если бы с самого начала работы работала бы вся бригада, за исключением пяти рабочих. Найдите количество рабочих.
		\item Футбольный мяч катится так, что за первую секунду он проходит путь $0,6$ м, а в каждую следующую секунду путь увеличивается на $0,6$ м по сравнению с предыдущей. Сколько секунд будет катиться мячик по горке длиной $6$ метров?
	\end{enumerate}
\end{document}