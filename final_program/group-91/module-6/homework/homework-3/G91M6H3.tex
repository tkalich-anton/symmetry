\documentclass[12pt, a4paper]{article}
\usepackage{cmap} % Улучшенный поиск русских слов в полученном pdf-файле
\usepackage[T2A]{fontenc} % Поддержка русских букв
\usepackage[utf8]{inputenc} % Кодировка utf8
\usepackage[english, russian]{babel} % Языки: русский, английский
\usepackage{enumitem}
\usepackage{pscyr} % Нормальные шрифты
\usepackage{soulutf8}
\usepackage{amsmath}
\usepackage{amsthm}
\usepackage{amssymb}
\usepackage{scrextend}
\usepackage{titling}
\usepackage{indentfirst}
\usepackage{cancel}
\usepackage{soulutf8}
\usepackage{wrapfig}
\usepackage{gensymb}
\usepackage[dvipsnames,table,xcdraw]{xcolor}
\usepackage{tikz}

%Русские символы в списке
\makeatletter
\AddEnumerateCounter{\asbuk}{\russian@alph}{щ}
\makeatother

%Дублирование знаков при переносе
\newcommand*{\hm}[1]{#1\nobreak\discretionary{}%
	{\hbox{$\mathsurround=0pt #1$}}{}}

\usepackage{graphicx}
\graphicspath{{pic/}}
\DeclareGraphicsExtensions{.pdf,.png,.jpg}

%Изменеие параметров листа
\usepackage[left=15mm,right=15mm,
top=2cm,bottom=2cm,bindingoffset=0cm]{geometry}


\usepackage{fancyhdr}
\pagestyle{fancy}
\usepackage{multicol}

\setlength\parindent{1,5em}
\usepackage{indentfirst}

\begin{document}
	
	\lhead{Группа 91}
	\chead{Модуль 6 Домашняя работа 3}
	\rhead{Школа <<Симметрия>>}
	
	\begin{enumerate}
		\item \textit{(1 балл)} В группе из $20$ российских туристов несколько человек владеют иностранными языками. Из них пятеро говорят только по-английски, трое только по-французски, двое по-французски и по-английски. Какова вероятность того, что случайно выбранный турист говорит по-французски?
		\item \textit{(1 балл)} В лыжных гонках участвуют $13$ спортсменов из России, $2$ спортсмена из Норвегии и $5$ спортсменов из Швеции. Порядок, в котором спортсмены стартуют, определяется жребием. Найдите вероятность того, что первым будет стартовать спортсмен из Норвегии или Швеции.
		\item \textit{(1 балл)} В среднем из каждых $200$ поступивших в продажу аккумуляторов $196$ аккумуляторов заряжены. Найдите вероятность того, что купленный аккумулятор не заряжен.
		\item \textit{(1 балл)} Из $1600$ пакетов молока в среднем $80$ протекают. Какова вероятность того, что случайно выбранный пакет молока не течёт?
		\item \textit{(2 балла)} Бригада маляров красит забор длиной $270$ метров, ежедневно увеличивая норму покраски на одно и то же число метров. Известно, что за первый и последний день в сумме бригада покрасила $90$ метров заборов. Определите, сколько дней бригада маляров красила весь забор.
		\item \textit{(2 балла)} Турист идет из одного города в другой, каждый день проходя больше, чем в предыдущий день, на одно и то же расстояние. Известно, что за первый день турист прошел $9$ километров. Определите, сколько километров прошел турист за шестой день, если весь путь он прошел за $7$ дней, а расстояние между городами составляет $105$ километров.
		\item \textit{(2 балла)} Альпинисты в первый день восхождения поднялись на высоту $1400$ м, а затем каждый следующий день поднимались на высоту на $100$ м меньше, чем в предыдущий. За сколько дней они покорили высоту $5000$ м?
	\end{enumerate}
\end{document}