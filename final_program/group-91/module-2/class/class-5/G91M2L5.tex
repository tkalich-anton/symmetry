\documentclass[12pt, a4paper]{article}
\usepackage{cmap} % Улучшенный поиск русских слов в полученном pdf-файле
\usepackage[T2A]{fontenc} % Поддержка русских букв
\usepackage[utf8]{inputenc} % Кодировка utf8
\usepackage[english, russian]{babel} % Языки: русский, английский
\usepackage{enumitem}
\usepackage{pscyr} % Нормальные шрифты
\usepackage{soulutf8}
\usepackage{amsmath}
\usepackage{amsthm}
\usepackage{amssymb}
\usepackage{scrextend}
\usepackage{titling}
\usepackage{indentfirst}
\usepackage{cancel}
\usepackage{soulutf8}
\usepackage{wrapfig}
\usepackage{gensymb}
\usepackage[dvipsnames,table,xcdraw]{xcolor}
\usepackage{tikz}

%Русские символы в списке
\makeatletter
\AddEnumerateCounter{\asbuk}{\russian@alph}{щ}
\makeatother

%Дублирование знаков при переносе
\newcommand*{\hm}[1]{#1\nobreak\discretionary{}%
	{\hbox{$\mathsurround=0pt #1$}}{}}

\usepackage{graphicx}
\graphicspath{{pic/}}
\DeclareGraphicsExtensions{.pdf,.png,.jpg}

%Изменеие параметров листа
\usepackage[left=15mm,right=15mm,
top=2cm,bottom=2cm,bindingoffset=0cm]{geometry}


\usepackage{fancyhdr}
\pagestyle{fancy}
\usepackage{multicol}

\setlength\parindent{1,5em}
\usepackage{indentfirst}

\begin{document}
		
\lhead{Группа 91}
\chead{Модуль 2 Урок 5}
\rhead{Школа <<Симметрия>>}

\begin{enumerate}
	\item Угол при основании $BC$ равнобедренного треугольника $ABC$ вдвое больше угла при вершине $A$, $BD$ — биссектриса треугольника. Докажите, что $AD = BC$.
	\item Прямая, проходящая через вершину $A$ треугольника $ABC$, пересекает сторону $BC$ в точке $M$. При этом $BM = AB$, $\angle BAM = 35\degree$, $\angle CAM = 15\degree$. Найдите углы треугольника $ABC$.
	\item На сторонах $AC$ и $BC$ треугольника $ABC$ взяты соответственно точки $M$ и $N$, причем $MN||AB$ и $MN = AM$.
	Найдите угол $BAN$, если $\angle B = 45\degree$ и $\angle60$.
	\item Два угла треугольника равны 10◦ и 70◦. Найдите угол между высотой и биссектрисой, проведенными из вершины третьего угла треугольника.
	\item На стороне $AB$ квадрата $ABCD$ построен равносторонний треугольник $ABM$. Найдите угол $DMC$.
	\item Острый угол прямоугольного треугольника равен $30\degree$. Докажите, что высота и медиана, проведенные из вершины прямого угла, делят его на три равные части.
	\item Через точку $A$, лежащую на окружности, проведены диаметр $AB$ и хорда $AC$, причем $AC = 8$ и $\angle BAC = 30\degree$. Найдите хорду $CM$, перпендикулярную $AB$.
	\item Известно, что $AB$ — диаметр окружности, а хорды $AC$ и $BD$ параллельны. Докажите, что $AC = BD$, а $CD$ также диаметр.
\end{enumerate}

\end{document}