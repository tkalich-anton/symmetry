\documentclass[12pt, a4paper]{article}
\usepackage{cmap} % Улучшенный поиск русских слов в полученном pdf-файле
\usepackage[T2A]{fontenc} % Поддержка русских букв
\usepackage[utf8]{inputenc} % Кодировка utf8
\usepackage[english, russian]{babel} % Языки: русский, английский
\usepackage{enumitem}
\usepackage{pscyr} % Нормальные шрифты
\usepackage{soulutf8}
\usepackage{amsmath}
\usepackage{amsthm}
\usepackage{amssymb}
\usepackage{scrextend}
\usepackage{titling}
\usepackage{indentfirst}
\usepackage{cancel}
\usepackage{soulutf8}
\usepackage{wrapfig}
\usepackage{gensymb}
\usepackage[dvipsnames,table,xcdraw]{xcolor}
\usepackage{tikz}

%Русские символы в списке
\makeatletter
\AddEnumerateCounter{\asbuk}{\russian@alph}{щ}
\makeatother

%Дублирование знаков при переносе
\newcommand*{\hm}[1]{#1\nobreak\discretionary{}%
	{\hbox{$\mathsurround=0pt #1$}}{}}

\usepackage{graphicx}
\graphicspath{{pic/}}
\DeclareGraphicsExtensions{.pdf,.png,.jpg}

%Изменеие параметров листа
\usepackage[left=15mm,right=15mm,
top=2cm,bottom=2cm,bindingoffset=0cm]{geometry}


\usepackage{fancyhdr}
\pagestyle{fancy}
\usepackage{multicol}

\setlength\parindent{1,5em}
\usepackage{indentfirst}

\begin{document}
		
\lhead{Группа 91}
\chead{Модуль 2 ДЗ№1}
\rhead{Школа <<Симметрия>>}

\section*{Домашняя работа №1}
\begin{enumerate}
	\item \textit{(1 балл)} Один из двух смежных углов на $20\degree$ меньше другого. Найдите эти углы.
	\item \textit{(1 балл)} Углы $\alpha$ (альфа), $\beta$ (бетта), $\gamma$ (гамма) — смежные. Известно, что угол $\alpha$ в два раза больше угла $\beta$, а угол $\beta$ в три раза больше угла $\gamma$. Чему равны эти углы?
	\item \textit{(1 балл)} Прямой угол разделен двумя лучами на три угла. Один из них на 10\degree больше другого и на 10\degree меньше третьего. Найдите эти углы.
	\item \textit{(1 балл)} Из точки $O$ на плоскости выходят три луча $OA$, $OB$, $OC$. Известно, что $\angle AOB=91\degree$, $\angle BOC=90\degree$. Найдите $\angle AOC$
	\item \textit{(1 балл)} Из точки, данной на окружности, проведены диаметр и хорда, равная радиусу. Найдите угол между ними.
	\item \textit{(1 балл)} Радиус окружности равен 13, хорда равна 10. Найдите её расстояние от центра.
	\item \textit{(2 балла)} Решите уравнение:
	\begin{multicols}{2}
		\begin{enumerate}[label=\asbuk*)]
			\item $\dfrac{1}{4} - \dfrac{1}{3}x = 4\dfrac{1}{4} - 3x$
			\item $\dfrac{6x -1}{5} - \dfrac{2-x}{4} = \dfrac{3x+2}{2}$
			\item $4x^2+7x+3=0$
			\item $3x^2+11x-34=0$
			\item $(5x+3)^2=5(x+3)$
			\item $\dfrac{x^2}{x+3}=\dfrac{5x-6}{x+3}$
		\end{enumerate}
	\end{multicols}
	\item \textit{(2 балла)} Упростите выражение и найдите значение выражения:
	\begin{multicols}{2}
		\begin{enumerate}[label=\asbuk*)]
			\item $\dfrac{x^2-10x+25}{3x+12}\cdot\dfrac{x^2-16}{2x-10}$ при $x=-1$
			\item $\dfrac{a^3-42}{a-7}-\dfrac{-28-a^3}{a-7}$ при $a=2$
		\end{enumerate}
	\end{multicols}
	\end{enumerate}
\end{document}