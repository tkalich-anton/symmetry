\documentclass[12pt, a4paper]{article}
\usepackage{cmap} % Улучшенный поиск русских слов в полученном pdf-файле
\usepackage[T2A]{fontenc} % Поддержка русских букв
\usepackage[utf8]{inputenc} % Кодировка utf8
\usepackage[english, russian]{babel} % Языки: русский, английский
\usepackage{enumitem}
\usepackage{pscyr} % Нормальные шрифты
\usepackage{amsmath}
\usepackage{amsthm}
\usepackage{amssymb}
\usepackage{scrextend}
\usepackage{titling}
\usepackage{indentfirst}
\usepackage{cancel}
\usepackage{soulutf8}
\usepackage{wrapfig}
\usepackage{gensymb}
\usepackage[dvipsnames,table,xcdraw]{xcolor}
\usepackage{tikz}
\usepackage{pgfplots}
\usepackage{mathrsfs}
\usepackage{longtable}

\usepackage{pgfplots}

%Русские символы в списке
\makeatletter
\AddEnumerateCounter{\asbuk}{\russian@alph}{щ}
\makeatother

%Дублирование знаков при переносе
\newcommand*{\hm}[1]{#1\nobreak\discretionary{}%
	{\hbox{$\mathsurround=0pt #1$}}{}}

\usepackage{graphicx}
\graphicspath{{pic/}}
\DeclareGraphicsExtensions{.pdf,.png,.jpg}

%Изменеие параметров листа
\usepackage[left=15mm,right=15mm,
top=2cm,bottom=2cm,bindingoffset=0cm]{geometry}

\usepackage{fancyhdr}
\pagestyle{fancy}
\usepackage{multicol}

\setlength\parindent{1,5em}
\usepackage{indentfirst}

\pgfplotsset{}
\pgfplotsset{
	width=10cm,
	compat=newest,
	every axis/.append style={
		axis x line=middle,
		axis y line=middle,
		unit vector ratio = 1 1,
		xlabel={$x$},
		xlabel style={below left},
		ylabel={$y$},
		ylabel style={below left},
		ymajorgrids=true,
		xmajorgrids=true,
		xticklabels={},
		yticklabels={},
		samples=1000,
		grid=both,
		grid style={thick},
		very thick,
		ticks=both,
		xtick={-100,-99,...,100},
		ytick={-100,-99,...,100},
		restrict y to domain=--100:100
}}

\begin{document}
	
\lhead{Группа 102}
\chead{Модуль 5 Домашняя работа №1}
\rhead{Школа <<Симметрия>>}

\begin{enumerate}
	\item \textit{(2 балла)} Вычислить:
	\begin{enumerate}[label=\asbuk*)]
		\item $2\sqrt{245}+\dfrac{1}{6}\sqrt{58^2-22^2}-30\sqrt{1,8}$
		\item $\dfrac{(7\sqrt{27}-7\sqrt{8})(\sqrt{27}+\sqrt{8})}{27^2-64}$
		\begin{flushright}
			\rotatebox{180}{\fbox{15}}
		\end{flushright}
	\end{enumerate}
	\item \textit{(2 балла)} Упростить выражение:
	$$\left(\dfrac{2x^2y+2xy^2}{7x^3+x^2y+7xy^2+y^3}\cdot\dfrac{7x+y}{x^2-y^2}+\dfrac{x-y}{x^2+y^2}\right)\cdot(x^2-y^2)$$
	\item \textit{(4 балла)} Построить график функции:
	\begin{multicols}{2}
		\begin{enumerate}[label=\asbuk*)]
			\item $y=-4(x+1)^2+1$
			\item $y=\dfrac{1}{2}\sqrt{x}-5$
			\item $y=2\sin2x$
			\item $y=\sin(x+\pi)+0,5$
		\end{enumerate}
	\end{multicols}
	\item \textit{(1 балл)} На рисунке изображен график функции вида $y=\dfrac{a}{x+b}+c$, где числа $a$, $b$ и $c$ — целые. Найдите $f\left(-\dfrac{8}{5}\right)$.
	\begin{figure}[h]
		\centering
		\begin{tikzpicture}[scale=0.6, >=stealth]
			\begin{axis}[
				xmin=-10,xmax=2,
				ymin=-5,ymax=5,
				ylabel style={below right},]
				\addplot [red,line width=1.4pt,domain=-10:10] {2/(x+1)+2};
				\addplot[red,line width=1pt,dash pattern=on 7pt off 9pt,domain=-10:10] {2};
				\draw[red,line width=1pt,dash pattern=on 7pt off 9pt,domain=-10:10] (-1,-10)--(-1,10);
				\fill[fill=red] (0,4) circle (2pt);
				\fill[fill=red] (-3,1) circle (2pt);
			\end{axis}
		\end{tikzpicture}
	\end{figure}
	\item \textit{(1 балл)} На рисунке изображен график функции вида $y=ax^2+bx+c$, где числа $a$, $b$ и $c$ — целые. Вычислите $f\left(\dfrac{1}{4}\right)-f\left(\dfrac{1}{2}\right)$.
	\begin{figure}[h]
		\centering
		\begin{tikzpicture}[scale=0.6, >=stealth]
			\begin{axis}[
				xmin=-2,xmax=5,
				ymin=-2,ymax=5]
				\addplot [red,line width=1.4pt,domain=-5:5] {3*x^2-5*x+1};
				\fill[fill=red] (1,-1) circle (2pt);
				\fill[fill=red] (0,1) circle (2pt);
			\end{axis}
		\end{tikzpicture}
	\end{figure}
\end{enumerate}
\end{document}