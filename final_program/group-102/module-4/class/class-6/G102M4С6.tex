\documentclass[12pt, a4paper]{article}
\usepackage{cmap} % Улучшенный поиск русских слов в полученном pdf-файле
\usepackage[T2A]{fontenc} % Поддержка русских букв
\usepackage[utf8]{inputenc} % Кодировка utf8
\usepackage[english, russian]{babel} % Языки: русский, английский
\usepackage{enumitem}
\usepackage{pscyr} % Нормальные шрифты
\usepackage{soulutf8}
\usepackage{amsmath}
\usepackage{amsthm}
\usepackage{amssymb}
\usepackage{scrextend}
\usepackage{titling}
\usepackage{indentfirst}
\usepackage{cancel}
\usepackage{soulutf8}
\usepackage{wrapfig}
\usepackage{gensymb}
\usepackage[dvipsnames,table,xcdraw]{xcolor}
\usepackage{tikz}

%Русские символы в списке
\makeatletter
\AddEnumerateCounter{\asbuk}{\russian@alph}{щ}
\makeatother

%Дублирование знаков при переносе
\newcommand*{\hm}[1]{#1\nobreak\discretionary{}%
	{\hbox{$\mathsurround=0pt #1$}}{}}

\usepackage{graphicx}
\graphicspath{{pic/}}
\DeclareGraphicsExtensions{.pdf,.png,.jpg}

%Изменеие параметров листа
\usepackage[left=15mm,right=15mm,
top=2cm,bottom=2cm,bindingoffset=0cm]{geometry}


\usepackage{fancyhdr}
\pagestyle{fancy}
\usepackage{multicol}

\setlength\parindent{1,5em}
\usepackage{indentfirst}

\begin{document}
	
	\lhead{Группа 102}
	\chead{Модуль 4 Урок №6}
	\rhead{Школа <<Симметрия>>}
	\begin{enumerate}
		\item $\left|\dfrac{x^2-10x+9}{x^2-9}\right|\geq1$
		\item $\dfrac{x^2-8x+24}{|x^2-16|}\leq1$
		\item $\sqrt{x^2-15x-7}>3$
		\item $\sqrt{x^4-2x+6}\geq x^2$
		\item $\sqrt{4x+3}\leq\sqrt{5x+1}$
		\item $\sqrt{24x-5x^2}>\sqrt{2x^2-23x+66}$
		\item $\sqrt{-x^2-5x-4}\geq x+4$
		\item $4\sqrt{x^2+x-2}\geq 5x-2$
		\item $(16-x^2)\sqrt{49-x^2}\geq0$
		\item $\dfrac{x+8}{x+1}\sqrt{\dfrac{x-3}{x-8}}\leq0$
		\item $\dfrac{(x^2-4)\sqrt{7x-x^2}}{2x^2-19x+35}\leq0$
		\item $15\sqrt{x}-7x\geq2$
	\end{enumerate}
\end{document}