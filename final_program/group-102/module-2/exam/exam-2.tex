\documentclass[12pt, a4paper]{article}
\usepackage{cmap} % Улучшенный поиск русских слов в полученном pdf-файле
\usepackage[T2A]{fontenc} % Поддержка русских букв
\usepackage[utf8]{inputenc} % Кодировка utf8
\usepackage[english, russian]{babel} % Языки: русский, английский
\usepackage{enumitem}
\usepackage{pscyr} % Нормальные шрифты
\usepackage{soulutf8}
\usepackage{amsmath}
\usepackage{amsthm}
\usepackage{amssymb}
\usepackage{scrextend}
\usepackage{titling}
\usepackage{indentfirst}
\usepackage{cancel}
\usepackage{soulutf8}
\usepackage{wrapfig}
\usepackage{gensymb}
\usepackage[dvipsnames,table,xcdraw]{xcolor}
\usepackage{tikz}
\usepackage{multicol}

%Русские символы в списке
\makeatletter
\AddEnumerateCounter{\asbuk}{\russian@alph}{щ}
\makeatother

%Дублирование знаков при переносе
\newcommand*{\hm}[1]{#1\nobreak\discretionary{}%
	{\hbox{$\mathsurround=0pt #1$}}{}}

\usepackage{graphicx}
\graphicspath{{pic/}}
\DeclareGraphicsExtensions{.pdf,.png,.jpg}

%Изменеие параметров листа
\usepackage[left=15mm,right=15mm,
top=1cm,bottom=2cm,bindingoffset=0cm]{geometry}

%\usepackage{fancyhdr}
%\pagestyle{fancy}

\setlength\parindent{1,5em}
\usepackage{indentfirst}

\begin{document}

\section*{Билет 1}
\begin{enumerate}
	\item \textit{(1 балл)} В каких четвертях знаки синуса и косинуса совпадают?
	\item \textit{(1 балл)} Какой четверти может принадлежать угол $x$, если $\sin x$ положительный?
	\item \textit{(1 балл)} Переведите 30 градусов в радианы.
	\item \textit{(1 балл)} Назовите хотя бы один угол в радианной мере, косинус которого равен 1.
	\item \textit{(1 балл)} Сформулируйте основное тригонометрическое тождество.
	\item \textit{(1 балл)} Вычислите $\sin (-45)$.
	\item \textit{(1 балл)} Вычислите $\sin 405$.
	\item \textit{(2 балла)} Выведите формулу $\sin x \cdot \cos y$.
	\item \textit{(2 балла)} Выведите формулу $\sin x + \sin y$.
	\item \textit{(3 балла)} Вычислите $\sin 225 \cdot \cos 120  \cdot \tg 330 \cdot \ctg 240$
	\item \textit{(3 балла)} Упростите выражение
	\item \textit{(3 балла)} Известно, что ...
\end{enumerate}
\section*{Билет 2}
\begin{enumerate}
	\item \textit{(1 балл)} В каких четвертях знаки синуса и косинуса совпадают?
	\item \textit{(1 балл)} Какой четверти может принадлежать угол $x$, если $\sin x$ положительный?
	\item \textit{(1 балл)} Переведите 30 градусов в радианы.
	\item \textit{(1 балл)} Назовите хотя бы один угол в радианной мере, косинус которого равен 1.
	\item \textit{(1 балл)} Сформулируйте основное тригонометрическое тождество.
	\item \textit{(1 балл)} Вычислите $\sin (-45)$.
	\item \textit{(1 балл)} Вычислите $\sin 405$.
	\item \textit{(2 балла)} Выведите формулу $\sin x \cdot \cos y$.
	\item \textit{(2 балла)} Выведите формулу $\sin x + \sin y$.
	\item \textit{(3 балла)} Вычислите $\sin 225 \cdot \cos 120  \cdot \tg 330 \cdot \ctg 240$
	\item \textit{(3 балла)} Упростите выражение
	\item \textit{(3 балла)} Известно, что ...
\end{enumerate}
\end{document}