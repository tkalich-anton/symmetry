\documentclass[12pt, a4paper]{article}
\usepackage{cmap} % Улучшенный поиск русских слов в полученном pdf-файле
\usepackage[T2A]{fontenc} % Поддержка русских букв
\usepackage[utf8]{inputenc} % Кодировка utf8
\usepackage[english, russian]{babel} % Языки: русский, английский
\usepackage{enumitem}
\usepackage{pscyr} % Нормальные шрифты


\usepackage{amsmath,amsthm,amssymb,scrextend, cancel}
\usepackage[dvipsnames,table,xcdraw]{xcolor}
\usepackage{fancyhdr}
\usepackage{multicol}
\usepackage{indentfirst}
\usepackage{graphicx}

%Изменеие параметров листа
\usepackage[left=10mm,right=10mm,
top=2cm,bottom=2cm,bindingoffset=0cm]{geometry}

%Русские символы в списке
\makeatletter
\AddEnumerateCounter{\asbuk}{\russian@alph}{щ}
\makeatother
%Дублирование знаков при переносе
\newcommand*{\hm}[1]{#1\nobreak\discretionary{}%
	{\hbox{$\mathsurround=0pt #1$}}{}}

\setlength\parindent{1,5em}
\setlength{\parskip}{0cm}
\pagestyle{fancy}

\begin{document}
		
\lhead{Группа 101}
\chead{Модуль 1 Урок 8}
\rhead{<<Симметрия>>}

\section*{Решение рациональных и иррациональных уравнений}
Решить уравнения:
\begin{multicols}{2}
	\begin{enumerate}
		\item $(x+3)^4-13(x+3)+36=0$
		\item $x^2=\dfrac{49x+343}{x+7}$
		\item $\dfrac{x^2}{x+8}=\dfrac{64}{x+8}$
		\item $\sqrt{8-x}=2-x$
		\item $x^2+3\sqrt{x^2-3x+11}=3x+4$
		\item $(2x+3)\sqrt{23x-14-3x^2}=0$
		\item $(2-x)\sqrt{x^2-x-20}=12-6x$
		\item $\dfrac{x-2}{x^3}=2x-x^2$
		\item $\sqrt{(x+5)(x-7)}=\sqrt{x+5}$
		\item $(x+4)\sqrt{x-6}=\sqrt{x-6}$
		\item $\dfrac{x+0,5}{9x+3}+\dfrac{8x^2+3}{9x^2-1}=\dfrac{x+2}{3x-1}$
		\item $\dfrac{x^4-6x^3+9x^2-36}{2x-3+\sqrt{33}}=0$
		\item $\left\{
		\begin{aligned}
			x+y^2&=2,\\
			2y^2+x^2&=3
		\end{aligned}
		\right.$
		\item $\left\{
		\begin{aligned}
			x^2&=5y-6,\\
			y^2&=5x-6
		\end{aligned}
		\right.$
		\item $\left\{
		\begin{aligned}
			x+7y&=1,\\
			x^2-49y^2&=5
		\end{aligned}
		\right.$
		\item $\sqrt{x^2-2x-1}=\dfrac{14}{\sqrt{x^2-2x-1}}-5$
		\item $x-\sqrt{x}=30$
	\end{enumerate}
\end{multicols}

\end{document}