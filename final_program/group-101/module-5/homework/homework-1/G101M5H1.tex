\documentclass[12pt, a4paper]{article}
\usepackage{cmap} % Улучшенный поиск русских слов в полученном pdf-файле
\usepackage[T2A]{fontenc} % Поддержка русских букв
\usepackage[utf8]{inputenc} % Кодировка utf8
\usepackage[english, russian]{babel} % Языки: русский, английский
\usepackage{enumitem}
\usepackage{pscyr} % Нормальные шрифты
\usepackage{soulutf8}
\usepackage{amsmath}
\usepackage{amsthm}
\usepackage{amssymb}
\usepackage{scrextend}
\usepackage{titling}
\usepackage{indentfirst}
\usepackage{cancel}
\usepackage{soulutf8}
\usepackage{wrapfig}
\usepackage{gensymb}
\usepackage[dvipsnames,table,xcdraw]{xcolor}
\usepackage{tikz}

%Русские символы в списке
\makeatletter
\AddEnumerateCounter{\asbuk}{\russian@alph}{щ}
\makeatother

%Дублирование знаков при переносе
\newcommand*{\hm}[1]{#1\nobreak\discretionary{}%
	{\hbox{$\mathsurround=0pt #1$}}{}}

\usepackage{graphicx}
\graphicspath{{pic/}}
\DeclareGraphicsExtensions{.pdf,.png,.jpg}

%Изменеие параметров листа
\usepackage[left=15mm,right=15mm,
top=2cm,bottom=2cm,bindingoffset=0cm]{geometry}


\usepackage{fancyhdr}
\pagestyle{fancy}
\usepackage{multicol}

\setlength\parindent{1,5em}
\usepackage{indentfirst}

\begin{document}
	
	\lhead{Группа 101}
	\chead{Модуль 5 Урок 2}
	\rhead{Школа <<Симметрия>>}
	
	\begin{enumerate}
		\item Решить уравнение:
		$\sqrt{5x+4}=-x-2$
		\item Решить неравенство:
		$|x^2-34x|\leq30$
		\item Переведите градусную меру угла в радианную:
		$$360\degree;\quad 120\degree;\quad -135\degree;\quad 210\degree;\quad -945\degree$$
		
		\item Переведите радианную меру в градусную:
		$$\dfrac{5\pi}{3};\quad \dfrac{7\pi}{6};\quad -\dfrac{7\pi}{4};\quad -\dfrac{13\pi}{4};\quad 7,5\pi$$
		
		\item Какой четверти принадлежит угол?
		$$\dfrac{11\pi}{4};\quad 1\dfrac{5}{3}\pi;\quad -375\degree;\quad \dfrac{26\pi}{5};\quad 820\degree;\quad -\dfrac{29\pi}{6}$$
		
		\item Вычислить:
		\begin{multicols}{4}
			\begin{enumerate}[label=\asbuk*)]
				\item $\sin 45\degree$
				\item $\cos 30\degree$
				\item $\cos 270\degree$
				\item $\sin 180\degree$
				\item $\sin (-60\degree)$
				\item $\tg 270\degree$
				\item $\cos 150\degree$
				\item $\sin (-120\degree)$
				\item $\sin \dfrac{5\pi}{3}$
				\item $\cos \dfrac{5\pi}{2}$
				\item $\cos \dfrac{2\pi}{4}$
				\item $\ctg \dfrac{3\pi}{4}$
			\end{enumerate}
		\end{multicols}
		\item Вычислить:
		\begin{enumerate}[label=\asbuk*)]
			\item $3\cos\dfrac{\pi}{3}-2\sin\dfrac{2\pi}{3}+7\cos\left(-\dfrac{2\pi}{3}\right)-\sin\left(-\dfrac{5\pi}{4}\right)$
			\item $2\sin\left(-\dfrac{5\pi}{6}\right)+11\cos\left(-\dfrac{7\pi}{3}\right)+\sin\dfrac{7\pi}{6}-8\cos\dfrac{2\pi}{3}$
			\item $-6\cos\left(-\dfrac{\pi}{6}\right)-2\sin\left(-\dfrac{\pi}{2}\right)-5\sin\left(-\dfrac{5\pi}{6}\right)+\cos\dfrac{7\pi}{6}$
		\end{enumerate}
		\item Вычислить
		\begin{multicols}{2}
			\begin{enumerate}[label=\asbuk*)]
				\item $\sin\alpha$, если: $\cos\alpha=-\dfrac{1}{3},\quad \pi<\alpha<\dfrac{3\pi}{2}$
				\item $\cos\alpha$, если: $\sin\alpha=-0,6,\dfrac{3\pi}{2}<\alpha<2\pi$
			\end{enumerate}
		\end{multicols}
		\item Упростить выражение:
		\begin{multicols}{2}
			\begin{enumerate}[label=\asbuk*)]
				\item $1-\cos^2x$
				\item $\sin^2x-1$
				\item $(\cos x - 1)(1 + \cos x)$
				\item $\cos^2 x - \sin^2 x + 1$
			\end{enumerate}
		\end{multicols}
		\item Сравнить
		\begin{multicols}{2}
			\begin{enumerate}[label=\asbuk*)]
				\item $\cos 91\degree$ и $\cos 92\degree$
				\item $\cos (-260\degree)$ и $\cos 210\degree$
			\end{enumerate}
		\end{multicols} 
	\end{enumerate}
	
\end{document}