\documentclass[12pt, a4paper]{article}
\usepackage{cmap} % Улучшенный поиск русских слов в полученном pdf-файле
\usepackage[T2A]{fontenc} % Поддержка русских букв
\usepackage[utf8]{inputenc} % Кодировка utf8
\usepackage[english, russian]{babel} % Языки: русский, английский
\usepackage{enumitem}
\usepackage{pscyr} % Нормальные шрифты
\usepackage{soulutf8}
\usepackage{amsmath}
\usepackage{amsthm}
\usepackage{amssymb}
\usepackage{scrextend}
\usepackage{titling}
\usepackage{indentfirst}
\usepackage{cancel}
\usepackage{soulutf8}
\usepackage{wrapfig}
\usepackage{gensymb}
\usepackage[dvipsnames,table,xcdraw]{xcolor}
\usepackage{tikz}

%Русские символы в списке
\makeatletter
\AddEnumerateCounter{\asbuk}{\russian@alph}{щ}
\makeatother

%Дублирование знаков при переносе
\newcommand*{\hm}[1]{#1\nobreak\discretionary{}%
	{\hbox{$\mathsurround=0pt #1$}}{}}

\usepackage{graphicx}
\graphicspath{{pic/}}
\DeclareGraphicsExtensions{.pdf,.png,.jpg}

%Изменеие параметров листа
\usepackage[left=15mm,right=15mm,
top=2cm,bottom=2cm,bindingoffset=0cm]{geometry}


\usepackage{fancyhdr}
\pagestyle{fancy}
\usepackage{multicol}

\setlength\parindent{1,5em}
\usepackage{indentfirst}

\begin{document}
	
	\lhead{Группа 101}
	\chead{Модуль 3 ДЗ№1}
	\rhead{Школа <<Симметрия>>}
	
	\section*{Тригонометрия.}
	\begin{enumerate}
		\item \textit{(1 балл)} Вычислите
		$\dfrac{1-\sin^2x}{1-\cos^2x}+\tg x \ctg x$
		\item \textit{(4 балла)} Упростите
		\begin{enumerate}[label=\asbuk*)]
			\item $\left( \dfrac{\cos (2,5 \pi + x)}{\ctg (3\pi + x)}-\sin(-x)\tg\left( \dfrac{5\pi}{2}+x\right) \right)+\dfrac{\tg x}{\tg \left( \dfrac{3\pi}{2}+x\right)}$
			\item $\dfrac{\sin \left( x-\dfrac{\pi}{4}\right)}{\sin\left( \dfrac{\pi}{4}+x\right) }\ctg \left( x-\dfrac{5\pi}{4}\right) - \cos \left( \dfrac{\pi}{2}+x\right)\sin(x-\pi) $
		\end{enumerate}
		\item \textit{(4 балла)} Решите уравнения
		\begin{enumerate}[label=\asbuk*)]
			\item $2x^3+8x=x^2+4$
			\item $2(5x-1)^2+35x-11=0$
			\item $\dfrac{x}{x-3}-\dfrac{5}{x+3}=\dfrac{18}{x^2-9}$
			\item $3(6x^2-13x+6)^2-10(6x^2-13x)=53$
		\end{enumerate}
		\item \textit{(1 балл)} Решить систему уравнений:
		$
		\left\{
		\begin{aligned}
			\dfrac{x}{y}-\dfrac{y}{x}=\dfrac{5}{6},\\
			x^2-y^2=5
		\end{aligned}
		\right.
		$
	\end{enumerate}
\end{document}